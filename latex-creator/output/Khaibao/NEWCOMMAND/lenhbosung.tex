%%%===========Lệnh bổ sung ===============%%%
\usepackage{adjustbox}
\usepackage{paracol}
\usepackage[inline]{asymptote}
\columnratio{0.65}
\globalcounter{kp}
\globalcounter{cauhoicount}
\setlength{\columnseprule}{0pt}
\def\gachngang{\resizebox{0.3cm}{!}{\_}}
\tikzfading[name=fade in one, inner color=transparent!0,outer color=transparent!100]
\tikzfading[name=fade in two, inner color=transparent!20,outer color=transparent!80]
\tikzfading[name=fade in three, inner color=transparent!50,outer color=transparent!70]
\tikzfading[name=fade in four, inner color=transparent!100,outer color=transparent!30]
\setlist[itemize,1]{itemsep=0pt,topsep=0pt,label=\protect{\small\color{\maunhan}\rhombusdot},wide=0cm,leftmargin=0.5cm}
\titlespacing*{\subsubsection}{0cm}{0pt}{0pt}
\titlespacing*{\paragraph}{0cm}{0pt}{0pt}
\NewDocumentCommand{\Rightpicture}{O{0.45}O{4pt}mm}{%
	\par\noindent
	\adjustbox{valign=t}{
		\begin{minipage}[htp!]{#1\linewidth}
			\vspace*{#2}
			#3
		\end{minipage}
	}
	\hfill
	\adjustbox{valign=t}{
		\begin{center}
			\begin{minipage}[htp!]{0.98\linewidth-#1\linewidth}
				#4
			\end{minipage}
		\end{center}
	}
}

\newenvironment{vidu}[1][\mycolor]{
	\begin{hopvidu}[#1]
		\par\noindent\indam[#1]{Ví dụ:}
	}{\end{hopvidu}}
\AtBeginEnvironment{ex}{\renewcommand{\dotEX}{}}
\titlespacing*{\subsubsection}{0cm}{0pt}{0pt}
\titlespacing*{\paragraph}{0cm}{0pt}{0pt}


\newenvironment{myitemize}
{\begin{itemize}[leftmargin=*,nosep,topsep=0pt,partopsep=0pt,before=\vspace{-\baselineskip},after=\vspace{-\baselineskip}]}
	{\end{itemize}}
%\renewcommand{\thefootnote}{(*)}
\renewcommand{\thefootnote}{\color{\maunhan}[\arabic{footnote}]}
\usepackage{ifpdf}
%%%====Định dạng ô nguyên tố===========%%%

\tikzset{
	pics/nguyen to/.style n args={7}{
		code={
			\def\width{3cm}  % Chiều rộng mặc định
			\def\height{3.8cm}  % Chiều cao mặc định
			\def\mau{dnvang}  % Màu mặc định
			% Xử lý các tùy chọn
			\pgfkeys{/nguyen to/.cd,#1}
			% Vẽ hình chữ nhật
			\path[fill=\mau!50] (0,0) rectangle (\width,\height);
			\path[fill=\mau!50!black] (0,0) rectangle (\width,0.12*\height);
			% Số hiệu nguyên tử
			\node[anchor=north west, font=\color{\mau!40!black}\footnotesize\sffamily] (SHNT) at (0.04*\width,0.98*\height) {#2};
			% Ghi chú Số hiệu nguyên tử
			\if\ifshownotes1
			\node [anchor=east, font=\color{\mycolor!30!black}\footnotesize] at (-0.3*\width,1.1*\height) (shnt) {Số hiệu nguyên tử};
			\draw[\maunhan,line join=round,line join=round] (SHNT.north)|- (shnt);
			\path[fill=\maunhan](SHNT.north)circle (1pt)(shnt.east)circle (1pt);
			\fi
			% Khối lượng nguyên tử
			\node[anchor=north east, font=\color{\mau!40!black}\footnotesize] (KLNT) at (0.96*\width,0.98*\height) {#3};
			% Ghi chú Khối lượng nguyên tử
			\if\ifshownotes1
			\node [anchor=west,text width=2.5cm,align=left, font=\color{\mycolor!30!black}\footnotesize] at (1.3*\width,1.1*\height) (klnt) {Nguyên tử khối trung bình};
			\path[draw=\maunhan,line join=round,line join=round] (KLNT.north)|- (klnt.west);
			\path[fill=\maunhan] (KLNT.north) circle (1pt) (klnt.west) circle (1pt);
			\fi
			% Ký hiệu nguyên tố
			\node[anchor=center, font=\color{\mau!40!black}\Huge\bfseries\fontfamily{phv}\selectfont] at (0.5*\width,0.65*\height) (KHNT) {#4};
			% Ghi chú Ký hiệu nguyên tố
			\if\ifshownotes1
			\node [anchor=west,text width=2.5cm,align=left, font=\color{\mycolor!30!black}\footnotesize] at (1.3*\width,0.65*\height) (khnt) {Kí hiệu nguyên tố hóa học};
			\path[draw=\maunhan,line join=round,line join=round] (KHNT.east) -- (khnt.west);
			\path[fill=\maunhan] (KHNT.east) circle (1pt) (khnt.west) circle (1pt);
			\fi
			% Tên nguyên tố
			\node[anchor=north, font=\color{\mau!40!black}\bfseries\scriptsize\fontfamily{ugq}\selectfont] at (0.5*\width,0.52*\height)(NAME) {#5};
			% Ghi chú Tên nguyên tố
			\if\ifshownotes1
			\node [anchor=east,left=1cm of NAME, font=\color{\mycolor!30!black}\footnotesize]  (name) {Tên nguyên tố hóa học};
			\path[draw=\maunhan,line join=round,line join=round] (name.east) -- (NAME.west);
			\path[fill=\maunhan] (NAME.west) circle (1pt) (name.east) circle (1pt);
			\fi
			% Cấu hình e
			\node[anchor=south, font=\color{\mau!40!black}\scriptsize] at (0.5*\width,0.15*\height)(config) {#6};
			% Ghi chú Cấu hình e
			\if\ifshownotes1
			\node [anchor=west,right = 1.5cm of config, font=\color{\mycolor!30!black}\footnotesize]  (CHE) {Cấu hình electron};
			\path[draw=\maunhan,line join=round,line join=round] (config.east) -- (CHE.west);
			\path[fill=\maunhan] (CHE.west) circle (1pt) (config.east) circle (1pt);
			\fi
			% Số oxi hóa
			\node[anchor=center, font=\color{\mau!20}\footnotesize] at (0.5*\width,0.06*\height) (oxihoa) {#7};
			% Ghi chú Số oxi hóa
			\if\ifshownotes1
			\node [anchor=west,below left=3mm of oxihoa, font=\color{\mycolor!30!black}\footnotesize]  (OXH) {Số oxi hóa};
			\path[draw=\maunhan,line join=round,line join=round] (oxihoa.south) |- (OXH.east);
			\path[fill=\maunhan] (OXH.east) circle (1pt) (oxihoa.south) circle (1pt);
			\fi
		}
	},
	/nguyen to/.cd,
	width/.store in=\width,
	height/.store in=\height,
	color/.store in=\mau,
	show notes/.is choice,
	show notes/true/.code={\def\ifshownotes{1}},
	show notes/false/.code={\def\ifshownotes{0}},
	show notes=false % Mặc định hiển thị ghi chú
}

\newcommand{\nguyento}[7][]{%
	\begin{tikzpicture}
		\pic {nguyen to={#1}{#2}{#3}{#4}{#5}{#6}{#7}};
	\end{tikzpicture}%
}





%%%=============================================%%%
%\renewcommand{\chaptername}{Chủ đề}
\renewcommand{\thesection}{Bài \arabic{section}.}
\renewcommand{\thesubsection}{\Roman{subsection}.}
\counterwithout*{section}{chapter}
%\everymath{\displaystyle\color{dnvang!70!black}}
%\everymath{\displaystyle}
\everymath{\rm}