\documentclass[Main.tex]{subfiles}
\begin{document}
\chapter{Hình học phẳng}
\section{Đường tròn ngoại tiếp}
\begin{Muctieu}
    \begin{itemize}
        \item Rèn luyện kỹ năng xác định tâm và bán kính đường tròn ngoại tiếp các đa giác thường gặp.
        \item Vận dụng định lý Pytago và các hệ thức lượng trong tam giác để tính bán kính.
        \item Kỹ năng vẽ hình minh họa chính xác bằng TikZ.
    \end{itemize}
\end{Muctieu}

\subsection{Bài tập}
\begin{dang}{Tính bán kính đường tròn ngoại tiếp các hình cơ bản}
    \begin{phuongphap}
        \begin{itemize}
            \item \textbf{Tam giác vuông:} Tâm là trung điểm cạnh huyền, bán kính bằng nửa cạnh huyền. $R = \frac{BC}{2}$.
            \item \textbf{Hình chữ nhật, Hình vuông:} Tâm là giao điểm hai đường chéo, bán kính bằng nửa đường chéo.
            \item \textbf{Tam giác đều:} Tâm trùng với trọng tâm, $R = \frac{2}{3}h$.
            \item \textbf{Tam giác cân:} Xác định tâm trên đường cao, sử dụng tính chất trung trực hoặc công thức $R = \frac{abc}{4S}$.
        \end{itemize}
    \end{phuongphap}

    % ==============================================================================
    % BÀI 1: TAM GIÁC VUÔNG
    % ==============================================================================
    \begin{bt}
        Cho tam giác $ABC$ vuông tại $A$ có cạnh góc vuông $AB = 6$ cm, $AC = 8$ cm. 
        Tính bán kính đường tròn ngoại tiếp tam giác $ABC$ và minh họa bằng hình vẽ.
        \loigiai{
            \begin{center}
            \begin{tikzpicture}[scale=0.2]
                \coordinate (A) at (0,0);
                \coordinate (B) at (6,0);
                \coordinate (C) at (0,8);
                \coordinate (M) at ($(B)!0.5!(C)$); % Trung điểm cạnh huyền
                
                \draw[thick] (A) -- (B) -- (C) -- cycle;
                \draw (M) circle (5); % Bán kính = 5
                
                % Góc vuông
                \draw (A) rectangle ++(0.5,0.5);
                
                % Điểm
                \foreach \p/\pos in {A/below left, B/below right, C/above left, M/above right} {
                    \fill (\p) circle (3pt);
                    \node[\pos] at (\p) {$\p$};
                }
                
                % Kích thước
                \node[below] at ($(A)!0.5!(B)$) {$6$};
                \node[left] at ($(A)!0.5!(C)$) {$8$};
                
                \draw[dashed] (M) -- (A);
            \end{tikzpicture}
            \end{center}
            
            Ta có tam giác $ABC$ vuông tại $A$. Áp dụng định lý Pytago:
            \[ BC = \sqrt{AB^2 + AC^2} = \sqrt{6^2 + 8^2} = \sqrt{36 + 64} = \sqrt{100} = 10 \text{ (cm)}. \]
            Tâm đường tròn ngoại tiếp tam giác vuông là trung điểm cạnh huyền.
            Bán kính đường tròn ngoại tiếp là:
            \[ R = \frac{BC}{2} = \frac{10}{2} = 5 \text{ (cm)}. \]
        }
    \end{bt}

    % ==============================================================================
    % BÀI 2: HÌNH CHỮ NHẬT
    % ==============================================================================
    \begin{bt}
        Cho hình chữ nhật $ABCD$ có chiều dài $AB = 12$ cm, chiều rộng $BC = 5$ cm.
        Tính bán kính đường tròn đi qua 4 đỉnh của hình chữ nhật đó.
        \loigiai{
            \begin{center}
            \begin{tikzpicture}[scale=0.3]
                \coordinate (A) at (0,5);
                \coordinate (B) at (12,5);
                \coordinate (C) at (12,0);
                \coordinate (D) at (0,0);
                \coordinate (O) at (6,2.5); % Tâm
                
                \draw[thick] (A) -- (B) -- (C) -- (D) -- cycle;
                \draw (O) circle (6.5); % R = 6.5
                
                \draw[dashed] (A) -- (C);
                \draw[dashed] (B) -- (D);
                
                \foreach \p/\pos in {A/above left, B/above right, C/below right, D/below left, O/below} {
                    \fill (\p) circle (3pt);
                    \node[\pos] at (\p) {$\p$};
                }
                
                \node[above] at ($(A)!0.5!(B)$) {$12$};
                \node[right] at ($(B)!0.5!(C)$) {$5$};
            \end{tikzpicture}
            \end{center}
            
            Xét tam giác vuông $ABC$ (vuông tại $B$):
            \[ AC = \sqrt{AB^2 + BC^2} = \sqrt{12^2 + 5^2} = \sqrt{144 + 25} = \sqrt{169} = 13 \text{ (cm)}. \]
            Tâm đường tròn ngoại tiếp hình chữ nhật là giao điểm hai đường chéo, bán kính bằng nửa độ dài đường chéo.
            \[ R = \frac{AC}{2} = \frac{13}{2} = 6,5 \text{ (cm)}. \]
        }
    \end{bt}

    % ==============================================================================
    % BÀI 3: HÌNH VUÔNG
    % ==============================================================================
    \begin{bt}
        Cho hình vuông $ABCD$ có cạnh bằng $4$ cm. Tính bán kính đường tròn ngoại tiếp hình vuông.
        \loigiai{
            \begin{center}
            \begin{tikzpicture}[scale=0.6]
                \coordinate (A) at (0,4);
                \coordinate (B) at (4,4);
                \coordinate (C) at (4,0);
                \coordinate (D) at (0,0);
                \coordinate (O) at (2,2);
                
                \draw[thick] (A) -- (B) -- (C) -- (D) -- cycle;
                \draw (O) circle (2.828); % R = 2*sqrt(2) approx 2.828
                
                \draw[dashed] (A) -- (C);
                \draw[dashed] (B) -- (D);
                
                \fill (O) circle (2pt) node[below]{$O$};
                \node[above left] at (A) {$A$};
                \node[above right] at (B) {$B$};
                \node[below right] at (C) {$C$};
                \node[below left] at (D) {$D$};
                
                \node[above] at ($(A)!0.5!(B)$) {$4$};
            \end{tikzpicture}
            \end{center}
            
            Đường chéo của hình vuông $AC = AB\sqrt{2} = 4\sqrt{2}$ cm.
            Bán kính đường tròn ngoại tiếp:
            \[ R = \frac{AC}{2} = \frac{4\sqrt{2}}{2} = 2\sqrt{2} \approx 2,83 \text{ (cm)}. \]
        }
    \end{bt}

    % ==============================================================================
    % BÀI 4: TAM GIÁC ĐỀU
    % ==============================================================================
    \begin{bt}
        Cho tam giác đều $ABC$ cạnh $a = 6$ cm. Tính bán kính đường tròn ngoại tiếp tam giác.
        \loigiai{
            \begin{center}
            \begin{tikzpicture}[scale=0.4]
                \coordinate (B) at (0,0);
                \coordinate (C) at (6,0);
                \coordinate (A) at (3, 5.196); % 3*sqrt(3) approx 5.196
                \coordinate (O) at (3, 1.732); % y = 1/3 * height = 1.732
                \coordinate (H) at (3,0);
                
                \draw[thick] (A) -- (B) -- (C) -- cycle;
                \draw[dashed] (A) -- (H);
                \draw (O) circle (3.464); % R = 3.464
                
                \fill (A) circle (3pt) node[above]{$A$};
                \fill (B) circle (3pt) node[below left]{$B$};
                \fill (C) circle (3pt) node[below right]{$C$};
                \fill (O) circle (3pt) node[right]{$O$};
                \fill (H) circle (3pt) node[below,shift={(115:15pt)}]{$H$};
                
                \node[below] at ($(B)!0.5!(C)$) {$6$};
            \end{tikzpicture}
            \end{center}
            
            Đường cao tam giác đều: $AH = \frac{a\sqrt{3}}{2} = \frac{6\sqrt{3}}{2} = 3\sqrt{3}$ cm.
            Tâm $O$ đường tròn ngoại tiếp trùng với trọng tâm tam giác đều.
            Bán kính:
            \[ R = \frac{2}{3}AH = \frac{2}{3} \cdot 3\sqrt{3} = 2\sqrt{3} \approx 3,46 \text{ (cm)}. \]
        }
    \end{bt}

    % ==============================================================================
    % BÀI 5: TAM GIÁC CÂN
    % ==============================================================================
    \begin{bt}
        Cho tam giác $ABC$ cân tại $A$ có $AB = AC = 5$ cm, $BC = 6$ cm. Tính bán kính đường tròn ngoại tiếp.
        \loigiai{
            \begin{center}
            \begin{tikzpicture}[scale=0.6]
                \coordinate (B) at (-3,0);
                \coordinate (C) at (3,0);
                % H là gốc (0,0). BH = 3. 
                % AH = sqrt(5^2 - 3^2) = 4.
                \coordinate (A) at (0,4);
                \coordinate (H) at (0,0);
                
                % R = 5^2 / (2*4) = 25/8 = 3.125
                % O có toạ độ (0, 4 - 3.125) = (0, 0.875)
                \coordinate (O) at (0, 0.875);
                
                \draw[thick] (A) -- (B) -- (C) -- cycle;
                \draw[dashed] (A) -- (H);
                \draw (O) circle (3.125);
                
                \foreach \p/\pos in {A/90, B/-135, C/-45, O/0, H/135}{
                    \fill (\p) circle (3pt);
                    \node[shift={(\pos:9pt)}] at (\p) {$\p$};
                }
                
                \node[left] at ($(A)!0.5!(B)$) {$5$};
                \path (B) -- (C) node[midway, below=0.1cm] {$6$};
                \draw[dashed] (B) .. controls ++(-45:2) and ++(-135:2) .. (C);
            \end{tikzpicture}
            \end{center}
            
            Gọi $H$ là trung điểm $BC$, suy ra $AH \perp BC$. Ta có $BH = \frac{BC}{2} = 3$ cm.
            Áp dụng Pytago trong $\Delta ABH$:
            \[ AH = \sqrt{AB^2 - BH^2} = \sqrt{5^2 - 3^2} = \sqrt{16} = 4 \text{ (cm)}. \]
            Bán kính đường tròn ngoại tiếp tam giác cân:
            \[ R = \frac{AB^2}{2AH} = \frac{5^2}{2 \cdot 4} = \frac{25}{8} = 3,125 \text{ (cm)}. \]
        }
    \end{bt}
\end{dang}
\end{document}
