\def\myytb{}
\def\myqrcodeytb{}
\def\myqrcodezalo{}
\def\quetmaqr{}
\def\thamgianhomhoctap{}
\newpage
\setcounter{ex}{0}
%%%======================%%%
\Tieudegiua[\maunhan]{Đề cương ôn tập môn khoa học lớp 5}
\Opensolutionfile{ans}[Ans/Dapan-KH-5-CKII-02]
\hienthiloigiaiex
\tieumuc{Phần Trắc nghiệm (3 điểm):Khoanh vào chữ cái đặt trước câu trả lời đúng nhất}
%%%============EX_1==============%%%
\begin{ex}[0,5 điểm]
	Để sản xuất ra muối từ nước biển, người ta đã sử dụng phương pháp nào?
	\choice
	{Lọc}
	{Lắng}
	{\True Phơi nắng}
	{Chưng cất}
	\loigiai{}
\end{ex}
%%%============EX_2==============%%%
\begin{ex}[0,5 điểm]Hỗn hợp nào dưới đây không phải là dung dịch?
\choice
{\True Nước và dầu}
{Nước và giấm}
{Nước muối}
{Nước đường}
\loigiai{}
\end{ex}
%%%============EX_3==============%%%
\begin{ex}[0,5 điểm]Chim và thú đều có bản năng gì trong quá trình nuôi con? 
\choice
{Nuôi con cho đến khi con của chúng đủ lông, đủ cánh và biết bay}
{\True Nuôi con cho đến khi con của chúng có thể tự đi kiếm ăn}
{Nuôi con bằng sữa cho đến khi con của chúng biết bay}
{Sinh con và nuôi con bằng sữa cho đến khi con của chúng có thể tự đi kiếm ăn}
\loigiai{}
\end{ex}
%%%============EX_4==============%%%
\begin{ex}[0,5 điểm]Theo em, đặc điểm nào là quan trọng nhất của nước sạch? 
\choice
{Dễ uống}
{Giúp nấu ăn ngon}
{Không mùi và không vị}
{\True Giúp phòng tránh được các bệnh về đường tiêu hóa, bệnh ngoài da, đau mắt,\ldots}
\loigiai{}
\end{ex}
%%%============EX_5==============%%%
\begin{ex}[1 điểm]Để đề phòng dòng điện quá mạnh có thể gây cháy đường dây và cháy nhà, người ta lắp thêm vào mạch điện cái gì?
\choice
{Một cái quạt}
{Một bóng đèn}
{\True Một cầu chì}
{Một chuông điện}
\loigiai{}
\end{ex}

\tieumuc{Phần Tự luận (7 điểm)}
\hienthiloigiaiex
%%%============EX_1==============%%%
\begin{ex}[1 điểm]Chọn từ trong ngoặc đơn để điền vào mỗi chỗ chấm sau đây cho phù hợp:(Tinh trùng, đực và cái, trứng, thụ tinh, cơ thể mới)
	\begin{itemize}
		\item Đa số loài vật chia thành hai giống $\ldots\ldots$. Giống đực có cơ quan sinh dục đực tạo ra $\ldots\ldots$. Con cái có cơ quan sinh dục cái tạo ra $\ldots\ldots$.
		\item Hiện tượng tinh trùng kết hợp với trứng gọi là sự $\ldots\ldots$. Hợp tử phân chia nhiều lần và phát triển thành $\ldots\ldots$ mang những đặc tính của bố mẹ.
	\end{itemize}

	\loigiai{\begin{itemize}
			\item Đa số loài vật chia thành hai giống \indam[black]{đực và cái}. Giống đực có cơ quan sinh dục đực tạo ra \indam[black]{tinh trùng}. Con cái có cơ quan sinh dục cái tạo ra \indam[black]{trứng}.
			\item Hiện tượng tinh trùng kết hợp với trứng gọi là sự \indam[black]{thụ tinh}. Hợp tử phân chia nhiều lần và phát triển thành \indam[black]{cơ thể mới} mang những đặc tính của bố mẹ.
	\end{itemize}}
\end{ex}
%%%=========ex_7=========%%%
\taoNdongke[4]{ex}
\begin{ex}(1 điểm)Nối nội dung ở cột A với nội dung ở cột B cho phù hợp
	\loigiai{}
\end{ex}
\begin{ex}[1 điểm]
	\loigiai{}
\end{ex}
%%%==============BT_1==============%%%
\hienthiloigiaiex
\begin{ex}[1 điểm-Điền đúng mỗi chỗ trồng được 0,25 điểm]Việc phá rứng gây ra những hậu quả gì?
	\loigiai{
	Việc phá rừng ồ ạt đã làm cho:
	\begin{itemize}
		\item Khí hậu bị thay đổi; lũ lụt, hạn hán xảy ra thường xuyên;
		\item Đất bị xói mòn trở nên bạc màu;
		\item Động vật và thực vật quý hiếm giảm dần, một số loài đã bị tuyệt chủng và một số loài có nguy cơ bị tuyệt chủng.
	\end{itemize}
	}
\end{ex}

%%%==============BT_2==============%%%
\begin{ex}[1 điểm]Năng lượng gió có thể dùng để làm gì?
	\loigiai{Năng lượng gió có thể dùng để chạy thuyền buồm, làm quay tua-bin của máy phát điện,\ldots }
\end{ex}

%%%==============BT_3==============%%%
\begin{ex}[1 điểm]Cho một vài ví dụ về sự biến đổi hóa học 
\loigiai{
\begin{itemize}
	\item Chưng đường trên ngọn lửa, đường cháy khét.
	\item Cho vôi sống vào nước tạo thành vôi tôi.
\end{itemize}
}
\end{ex}

%%%==============BT_4==============%%%
\begin{ex}[1 điểm]Bảo vệ môi trường là nhiệm vụ của ai?. \\
	Để góp phần bảo vệ môi trường xung quanh, em cần phải làm gì?
\loigiai{
Bảo vệ môi trường là nhiệm vụ chung của tất cả mọi người trên thế giới. \\
Để góp phần bảo vệ môi trường xung quanh, em cần phải làm các việc sau:
\begin{itemize}
	\item Tích cực dọn dẹp vệ sinh trường, lớp, nhà ở,\ldots
	\item Đổ rác đúng nơi quy định, hạn chế sử dụng túi nilon,\ldots
	\item Tiết kiệm điện, nước trong sinh hoạt.
	\item Tích cực trồng cây xanh,\ldots
\end{itemize}
}
\end{ex}
\Closesolutionfile{ans}