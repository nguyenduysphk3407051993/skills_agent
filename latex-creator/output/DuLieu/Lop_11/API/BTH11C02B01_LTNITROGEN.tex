
%%%%=================EX_01====================%%%
\begin{ex}
    Thành phần nào chiếm tỉ lệ lớn nhất trong không khí?
    \choice
    {Oxygen ($O_2$)}
    {\True Nitrogen ($N_2$)}
    {Carbon dioxide ($CO_2$)}
    {Argon (Ar)}
    \loigiai{Nitrogen chiếm khoảng $78\%$ thể tích không khí, là thành phần phổ biến nhất.}
\end{ex}

%%%%=================EX_02====================%%%
\begin{ex}
    Số oxi hóa của nitrogen trong $NaNO_3$ là bao nhiêu?
    \choice
    {-3}
    {+1}
    {+3}
    {\True +5}
    \loigiai{Số oxi hóa của Na là +1, của O là -2. Gọi số oxi hóa của N là x. Ta có: +1 + x + 3(-2) = 0 => x = +5.}
\end{ex}

%%%%=================EX_03====================%%%
\begin{ex}
    Công thức hóa học của diêm tiêu Chile là gì?
    \choice
    {$KNO_3$}
    {\True $NaNO_3$}
    {$NH_4NO_3$}
    {$Ca(NO_3)_2$}
    \loigiai{Diêm tiêu Chile có công thức hóa học là $NaNO_3$ (sodium nitrate).}
\end{ex}

%%%%=================EX_04====================%%%
\begin{ex}
    Liên kết trong phân tử $N_2$ là loại liên kết gì?
    \choice
    {Liên kết đơn}
    {Liên kết đôi}
    {\True Liên kết ba}
    {Liên kết ion}
    \loigiai{Phân tử $N_2$ có liên kết ba giữa hai nguyên tử nitrogen.}
\end{ex}

%%%%=================EX_05====================%%%
\begin{ex}
    Quá trình Haber-Bosch dùng để sản xuất hợp chất nào?
    \choice
    {Axit nitric ($HNO_3$)}
    {\True Ammonia ($NH_3$)}
    {Nitrogen monoxide (NO)}
    {Nitrogen dioxide ($NO_2$)}
    \loigiai{Quá trình Haber-Bosch là quá trình tổng hợp ammonia ($NH_3$) từ nitrogen và hydrogen.}
\end{ex}

%%%%=================EX_06====================%%%
\begin{ex}
    Tại sao nitrogen ở nhiệt độ thường khá trơ về mặt hóa học?
    \choice
    {Do nitrogen có độ âm điện nhỏ}
    {\True Do liên kết ba trong phân tử $N_2$ rất bền vững}
    {Do nitrogen là khí hiếm}
    {Do nitrogen tan tốt trong nước}
    \loigiai{Liên kết ba trong phân tử $N_2$ rất bền vững, cần năng lượng lớn để phá vỡ, nên nitrogen ở nhiệt độ thường khá trơ.}
\end{ex}

%%%%=================EX_07====================%%%
\begin{ex}
    Nitrogen lỏng được sử dụng để làm gì trong y học?
    \choice
    {Điều trị bỏng}
    {\True Bảo quản mẫu sinh học}
    {Khử trùng dụng cụ}
    {Sản xuất thuốc}
    \loigiai{Nitrogen lỏng được sử dụng để bảo quản mẫu sinh học, tế bào, mô trong y học nhờ nhiệt độ rất thấp.}
\end{ex}

%%%%=================EX_08====================%%%
\begin{ex}
    Số electron chưa liên kết trong phân tử $N_2$ là bao nhiêu?
    \choice
    {0}
    {\True 4}
    {6}
    {8}
    \loigiai{Mỗi nguyên tử N có 5 electron lớp ngoài cùng, trong phân tử $N_2$ có 6 electron tham gia liên kết ba, còn lại 4 electron chưa liên kết.}
\end{ex}

%%%%=================EX_09====================%%%
\begin{ex}
    Khí nào không duy trì sự cháy và hô hấp?
    \choice
    {$O_2$}
    {\True $N_2$}
    {$CO_2$}
    {$H_2$}
    \loigiai{Khí nitrogen ($N_2$) không duy trì sự cháy và hô hấp.}
\end{ex}

%%%%=================EX_10====================%%%
\begin{ex}
    Trong phản ứng tổng hợp ammonia, nitrogen đóng vai trò là chất gì?
    \choice
    {Chất khử}
    {\True Chất oxi hóa}
    {Xúc tác}
    {Môi trường}
    \loigiai{Trong phản ứng tổng hợp ammonia, nitrogen nhận electron, số oxi hóa giảm từ 0 xuống -3, nên nitrogen đóng vai trò là chất oxi hóa.}
\end{ex}


%%%%=================TF_01====================%%%
\begin{ex}
	Nitrogen chiếm khoảng $21\%$ thể tích không khí.
	\choiceTF[t]
	{\True Nitrogen chiếm khoảng $78\%$ thể tích không khí.}
	{Nitrogen chiếm khoảng $21\%$ thể tích không khí.}
	{\True Oxygen chiếm khoảng $21\%$ thể tích không khí.}
	{Oxygen chiếm khoảng $78\%$ thể tích không khí.}
	\loigiai{Nitrogen chiếm khoảng $78\%$ thể tích không khí, còn oxygen chiếm khoảng $21\%$.}
\end{ex}

%%%%=================TF_02====================%%%
\begin{ex}
	Diêm tiêu Chile được sử dụng để sản xuất phân bón, thuốc nổ, dược phẩm.
	\choiceTF[t]
	{Diêm tiêu Chile chỉ được sử dụng để sản xuất phân bón.}
	{\True Diêm tiêu Chile được sử dụng để sản xuất phân bón, thuốc nổ, dược phẩm.}
	{Diêm tiêu Chile không được sử dụng để sản xuất thuốc nổ.}
	{Diêm tiêu Chile chỉ được sử dụng trong y học.}
	\loigiai{Diêm tiêu Chile ($NaNO_3$) có nhiều ứng dụng, bao gồm sản xuất phân bón, thuốc nổ, dược phẩm, kính.}
\end{ex}

%%%%=================TF_03====================%%%
\begin{ex}
	Phân tử $N_2$ có liên kết ba, bao gồm 2 liên kết sigma và 1 liên kết pi.
	\choiceTF[t]
	{\True Phân tử $N_2$ có liên kết ba, bao gồm 1 liên kết sigma và 2 liên kết pi.}
	{Phân tử $N_2$ có liên kết ba, bao gồm 2 liên kết sigma và 1 liên kết pi.}
	{Phân tử $N_2$ có liên kết đôi.}
	{Phân tử $N_2$ có liên kết đơn.}
	\loigiai{Phân tử $N_2$ có liên kết ba, bao gồm 1 liên kết sigma và 2 liên kết pi.}
\end{ex}

%%%%=================TF_04====================%%%
\begin{ex}
	Quá trình Haber-Bosch cần thực hiện ở nhiệt độ và áp suất cao.
	\choiceTF[t]
	{Quá trình Haber-Bosch cần thực hiện ở nhiệt độ và áp suất thấp.}
	{\True Quá trình Haber-Bosch cần thực hiện ở nhiệt độ và áp suất cao.}
	{Quá trình Haber-Bosch không cần xúc tác.}
	{Quá trình Haber-Bosch dùng để sản xuất axit nitric.}

	\loigiai{Quá trình Haber-Bosch cần thực hiện ở nhiệt độ và áp suất cao, có xúc tác để tổng hợp ammonia từ nitrogen và hydrogen.}
\end{ex}

%%%%=================TF_05====================%%%
\begin{ex}
	Nitrogen ở nhiệt độ thường dễ dàng phản ứng với hydrogen.
	\choiceTF[t]
	{\True Nitrogen ở nhiệt độ thường khá trơ, khó phản ứng với hydrogen.}
	{Nitrogen ở nhiệt độ thường dễ dàng phản ứng với hydrogen.}
	{\True  Nitrogen cần nhiệt độ và áp suất cao, có xúc tác mới phản ứng được với hydrogen.}
	{Nitrogen không phản ứng được với hydrogen.}
	\loigiai{Nitrogen ở nhiệt độ thường khá trơ, cần nhiệt độ và áp suất cao, có xúc tác mới phản ứng được với hydrogen để tạo thành ammonia.}
\end{ex}

%%%%=================TF_06====================%%%
\begin{ex}
	Nitrogen lỏng có nhiệt độ sôi là $0^\circ C$.
	\choiceTF[t]
	{\True Nitrogen lỏng có nhiệt độ sôi rất thấp, khoảng $-196^\circ C$.}
	{Nitrogen lỏng có nhiệt độ sôi là $0^\circ C$.}
	{Nitrogen lỏng có nhiệt độ sôi là $100^\circ C$.}
	{Nitrogen lỏng không có nhiệt độ sôi.}

	\loigiai{Nitrogen lỏng có nhiệt độ sôi rất thấp, khoảng $-196^\circ C$.}
\end{ex}

%%%%=================TF_07====================%%%
\begin{ex}
	Nitrogen được dùng để bơm vào bao bì thực phẩm để bảo quản thực phẩm lâu hơn.
	\choiceTF[t]
	{Nitrogen không được sử dụng trong bảo quản thực phẩm.}
	{\True Nitrogen được dùng để bơm vào bao bì thực phẩm để bảo quản thực phẩm lâu hơn.}
	{Oxygen được dùng để bơm vào bao bì thực phẩm để bảo quản thực phẩm lâu hơn.}
	{Carbon dioxide được dùng để bơm vào bao bì thực phẩm để bảo quản thực phẩm lâu hơn.}
	\loigiai{Nitrogen là khí trơ, được dùng để bơm vào bao bì thực phẩm để tránh oxi hóa, bảo quản thực phẩm lâu hơn.}
\end{ex}

%%%%=================TF_08====================%%%
\begin{ex}
	Bệnh giảm áp ở thợ lặn là do nitrogen tạo bọt khí trong mạch máu khi áp suất giảm đột ngột.
	\choiceTF[t]
	{Bệnh giảm áp ở thợ lặn không liên quan đến nitrogen.}
	{\True Bệnh giảm áp ở thợ lặn là do nitrogen tạo bọt khí trong mạch máu khi áp suất giảm đột ngột.}
	{Bệnh giảm áp ở thợ lặn là do thiếu oxygen.}
	{Bệnh giảm áp ở thợ lặn là do ngộ độc carbon dioxide.}
	\loigiai{Khi áp suất giảm đột ngột, nitrogen hòa tan trong máu tạo bọt khí trong mạch máu, gây ra bệnh giảm áp ở thợ lặn.}
\end{ex}

%%%%=================TF_09====================%%%
\begin{ex}
	Nitrogen trong không khí có thể kết hợp trực tiếp với oxygen ở nhiệt độ thường.
	\choiceTF[t]
	{\True Nitrogen chỉ kết hợp với oxygen ở nhiệt độ rất cao, ví dụ như trong sấm sét.}
	{Nitrogen trong không khí có thể kết hợp trực tiếp với oxygen ở nhiệt độ thường.}
	{\True Phản ứng giữa nitrogen và oxygen tạo ra nitrogen monoxide (NO).}
	{Phản ứng giữa nitrogen và oxygen tạo ra ammonia ($NH_3$).}
	\loigiai{Nitrogen chỉ kết hợp với oxygen ở nhiệt độ rất cao (trên $3000^\circ C$ hoặc tia lửa điện), tạo ra NO.}
\end{ex}

%%%%=================TF_10====================%%%
\begin{ex}
	Công thức cấu tạo của phân tử $N_2$ là $N-N$.
	\choiceTF[t]
	{\True Công thức cấu tạo của phân tử $N_2$ là $N \equiv N$.}
	{Công thức cấu tạo của phân tử $N_2$ là $N-N$.}
	{Công thức cấu tạo của phân tử $N_2$ là $N=N$.}
	{Công thức cấu tạo của phân tử $N_2$ là $N^++N^-$.}
	\loigiai{Công thức cấu tạo của phân tử $N_2$ là $N \equiv N$, thể hiện liên kết ba giữa hai nguyên tử nitrogen.}
\end{ex}


%%%%=================BT_01====================%%%
\begin{bt}
	Giải thích tại sao sau cơn mưa dông kèm sấm sét, cây cối thường xanh tốt hơn.
	\loigiai{Sấm sét tạo ra nhiệt độ cao, nitrogen trong không khí phản ứng với oxygen tạo thành NO. NO tiếp tục phản ứng với oxygen và nước tạo thành $HNO_3$. $HNO_3$ tan trong nước mưa, phân ly thành ion nitrate ($NO_3^-$), là nguồn phân đạm cho cây trồng, giúp cây cối xanh tốt hơn.}
\end{bt}

%%%%=================BT_02====================%%%
\begin{bt}
	Viết phương trình phản ứng của nitrogen với hydrogen và oxygen (trong điều kiện sấm sét).
	\loigiai{
	Phản ứng với hydrogen:
	$N_2 + 3H_2 \xrightarrow[áp suất]{nhiệt độ, xúc tác} 2NH_3$

	Phản ứng với oxygen (sấm sét):
	$N_2 + O_2 \xrightarrow[]{>3000^\circ C} 2NO$
	}
\end{bt}

%%%%=================BT_03====================%%%
\begin{bt}
	Mô tả cấu tạo phân tử $N_2$ và giải thích tính trơ của nitrogen ở nhiệt độ thường.
	\loigiai{Phân tử $N_2$ gồm hai nguyên tử nitrogen liên kết với nhau bằng liên kết ba (1 liên kết sigma và 2 liên kết pi). Liên kết ba này rất bền vững, cần năng lượng lớn để phá vỡ, do đó nitrogen ở nhiệt độ thường khá trơ về mặt hóa học.}
\end{bt}

%%%%=================BT_04====================%%%
\begin{bt}
	Nêu các ứng dụng của nitrogen trong đời sống và sản xuất.
	\loigiai{
	Ứng dụng của nitrogen:
	\begin{itemize}
		\item Làm lạnh (nitrogen lỏng): bảo quản mẫu sinh học, thực phẩm.
		\item Tổng hợp ammonia ($NH_3$): sản xuất phân bón, hóa chất.
		\item Tạo môi trường trơ: bảo quản thực phẩm, trong các phản ứng hóa học.
		\item Sản xuất thuốc nổ (từ diêm tiêu Chile).
	\end{itemize}
	}
\end{bt}


%%%%=================BT_05====================%%%
\begin{bt}
	Giải thích tại sao không sử dụng phản ứng trực tiếp giữa nitrogen và oxygen để sản xuất NO trong công nghiệp.
	\loigiai{Phản ứng trực tiếp giữa nitrogen và oxygen đòi hỏi nhiệt độ rất cao (trên $3000^\circ C$), tốn nhiều năng lượng và hiệu suất thấp, không kinh tế. Trong công nghiệp, người ta sử dụng quá trình oxi hóa ammonia để sản xuất NO hiệu quả hơn.}
\end{bt}

%%%%=================BT_06====================%%%
\begin{bt}
	Xác định số oxi hóa của nitrogen trong các hợp chất sau: $N_2$, $NH_3$, $NO$, $NO_2$, $HNO_3$.
	\loigiai{
	Số oxi hóa của nitrogen:
	\begin{itemize}
		\item $N_2$: 0
		\item $NH_3$: -3
		\item $NO$: +2
		\item $NO_2$: +4
		\item $HNO_3$: +5
	\end{itemize}
	}
\end{bt}
