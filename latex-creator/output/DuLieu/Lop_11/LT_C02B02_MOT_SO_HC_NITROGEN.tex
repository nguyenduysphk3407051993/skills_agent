%%%Phần lý thuyết
\subsubsection{Ammonia ($NH_3$)}
\Noibat[\maunhan][][][]{Cấu tạo phân tử và tính chất vật lí}

\begin{figure}[!htp]
	\begin{center}
		%% Hình phụ 1
		\subcaptionbox{\label{ctlewis}}[6cm]{\chemfig[atom sep =5em]{H-\charge{90:2.0pt =\:[{.style={draw=none,fill=\mycolor}}]}{N}(-[:-90]H)-H}}
		%% Hình phụ 2
		\subcaptionbox{\label{ctphantu}}[6cm]{\chemfig[atom sep =3em]{H?[a]-[:40,2]N(<[:-70,2]H?[a,,dashed]?[b])<:[:-30,2]H?[a,,dashed]?[b,,dashed]}}
		\caption{Công thức Lewis \ref{ctlewis} và dạng hình học \ref{ctphantu}  của $NH_3$ \label{NH3}}
	\end{center}
\end{figure}
Ammonia ($NH_3$) là một chất khí không màu, có mùi khai đặc trưng, nhẹ hơn không khí ($M_{NH_3} = 17$ g/mol; $M_{\text{không khí}} = 29$ g/mol). Ammonia tan rất nhiều trong nước, tạo thành dung dịch amoniac. Khí ammonia dễ hóa lỏng ở nhiệt độ thấp ($-33,4^\circ C$) và áp suất cao.

\begin{hoivadap}
	Dựa vào cấu tạo phân tử của ammonia, giải thích tại sao khí $NH_3$ tan rất nhiều trong nước.
\end{hoivadap}

\Noibat[\maunhan][][][]{Tính chất hóa học của Ammonia}

\Noibat[\maunhan][][\faBook][]{Tính base}

Ammonia thể hiện tính bazơ yếu do cặp electron chưa liên kết trên nguyên tử nitrogen. Nó có khả năng nhận proton ($H^+$).
\begin{itemize}
	\item \textbf{Tác dụng với nước:} Một phần nhỏ ammonia phản ứng với nước tạo ion amoni ($NH_4^+$) và ion hiđroxit ($OH^-$), làm dung dịch có môi trường bazơ (làm quỳ tím hóa xanh, phenolphtalein hóa hồng).
	\[
	\text{NH}_3\text{ (g)} + \text{H}_2\text{O}\text{ (l)} \xharpoonarrow[][][1] \text{NH}_4^+\text{ (aq)} + \text{OH}^-\text{ (aq)}
	\]
	\item \textbf{Tác dụng với acid:} Ammonia phản ứng với acid tạo thành muối amoni.
	\[
	\text{NH}_3\text{ (g)} + \text{HCl}\text{ (g)} \xrightarrow \text{NH}_4\text{Cl}\text{ (s)}
	\]
	(Phản ứng tạo khói trắng amoni clorua khi hai khí tiếp xúc).
	\[
	3\text{NH}_3\text{ (aq)} + \text{H}_3\text{PO}_4\text{ (aq)} \xrightarrow (\text{NH}_4)_3\text{PO}_4\text{ (aq)}
	\]
	\item \textbf{Tác dụng với dung dịch muối của kim loại yếu (kết tủa hiđroxit):} Dung dịch amoniac tác dụng với dung dịch muối của một số kim loại tạo kết tủa hiđroxit.
	\[
	3\text{NH}_3\text{ (aq)} + \text{AlCl}_3\text{ (aq)} + 3\text{H}_2\text{O}\text{ (l)} \xrightarrow \text{Al(OH)}_3 \downarrow + 3\text{NH}_4\text{Cl}\text{ (aq)}
	\]
\end{itemize}
	\begin{hoivadap}
		Khi cho quỳ tím ẩm tiếp xúc với khí ammonia thì có hiện tượng gì?
	\end{hoivadap}
\Noibat[\maunhan][][\faBook][]{Tính khử}
\[
4\overset{-3}{\text{N}}\text{H}_3(\text{g}) + 3\text{O}_2(\text{g}) \xrightarrow[$t^\circ$] 2\overset{0}{\text{N}_2}(\text{g}) + 6\text{H}_2\text{O}(\text{g})
\]
\[
4\overset{-3}{\text{N}}\text{H}_3(\text{g}) + 5\text{O}_2(\text{g}) \xrightarrow[$800-900^\circ\text{C}, \text{Pt}$][][2] 4\overset{+2}{\text{N}}\text{O}(\text{g}) + 6\text{H}_2\text{O}(\text{g})
\]

\Noibat[\maunhan][][][]{Tổng hợp Ammonia ($NH_3$) theo quá trình haber}
	\begin{center}
		\includegraphics[width=9cm]{Images/anhhoa11/C02_B03_AMMONIA/chu_trinh_haber.png}
		\captionof{figure}{Sơ đồ nguyên tắc quá trình haber tổng hợp ammonia }
	\end{center}
\Noibat[\maunhan][][\faApple][]{Phương trình hóa học và đặc điểm chung}
Quá trình tổng hợp ammonia từ nitrogen ($N_2$) và hydrogen ($H_2$) là một phản ứng thuận nghịch.
\[
\text{N}_2\text{(g)} + 3\text{H}_2\text{(g)} \xharpoonarrow[$400-600^\circ\text{C}$][$200\, \text{bar}, \text{Fe}$][2] 2\text{NH}_3\text{(g)} \quad \Delta_rH^\circ_{298} = -92 \text{ kJ}
\]
Đây là một phản ứng tỏa nhiệt, có nghĩa là quá trình thuận lợi về mặt năng lượng khi giải phóng nhiệt ra môi trường. Quá trình tổng hợp ammonia là một trong những quy trình hóa học quy mô lớn nhất và dẫn đầu về mức tiêu thụ năng lượng trên thế giới.

\Noibat[\maunhan][][\faApple][]{Các giai đoạn và điều kiện thực hiện}
Quá trình Haber-Bosch để tổng hợp ammonia thường được thực hiện theo các bước chính sau trong công nghiệp:
\begin{itemize}
	\item \textbf{Hỗn hợp khí đầu vào }
	Hỗn hợp khí nitrogen và hydrogen được chuẩn bị và đưa vào với tỉ lệ mol $1:3$. Nitrogen thường được lấy từ không khí, còn hydrogen thường được sản xuất từ khí thiên nhiên hoặc nước.
	\item \textbf{Tháp tổng hợp ammonia (Lò phản ứng)}
	Hỗn hợp khí đầu vào được nén và đưa vào tháp tổng hợp ammonia dưới các điều kiện tối ưu nhằm đạt được hiệu suất và tốc độ phản ứng cao nhất có thể:
	\begin{itemize}
		\item \textbf{Nhiệt độ:} Khoảng $380 - 450^\circ\text{C}$ (hoặc có thể $400 - 600 ^\circ\text{C}$ tùy công nghệ). Nhiệt độ này được lựa chọn để cân bằng giữa việc tăng tốc độ phản ứng (nhiệt độ cao) và việc dịch chuyển cân bằng theo chiều thuận (do phản ứng tỏa nhiệt, nhiệt độ thấp sẽ làm tăng hiệu suất cân bằng).
		\item \textbf{Áp suất:} Khoảng $200$ bar (hoặc $150 - 200$ bar). Áp suất cao được sử dụng vì theo nguyên lí Le Chatelier, phản ứng tổng hợp ammonia có số mol khí giảm (từ $4$ mol khí chất phản ứng tạo thành $2$ mol khí sản phẩm). Tăng áp suất sẽ làm cân bằng dịch chuyển theo chiều giảm số mol khí, tức là chiều thuận, do đó làm tăng hiệu suất tạo ra $NH_3$.
		\item \textbf{Chất xúc tác:} Sử dụng bột sắt ($Fe$) được hoạt hóa (thường có pha thêm các oxit như $Al_2O_3$, $K_2O$). Chất xúc tác giúp tăng tốc độ cả phản ứng thuận và nghịch, giúp hệ đạt trạng thái cân bằng nhanh hơn, từ đó tăng năng suất sản xuất $NH_3$ mà không làm thay đổi vị trí cân bằng.
	\end{itemize}
	
	\item \textbf{Tháp làm lạnh và tách sản phẩm }
	Hỗn hợp khí sau phản ứng (gồm $N_2$, $H_2$ chưa phản ứng và $NH_3$ đã tạo thành) được dẫn đến tháp làm lạnh. Tại đây, do nhiệt độ sôi của ammonia ($-33$ $^\circ\text{C} $) cao hơn nhiều so với $N_2$ ($-196$ $^\circ\text{C} $) và $H_2$ ($-253$ $^\circ\text{C} $), ammonia sẽ hóa lỏng và được tách riêng ra dưới dạng lỏng.
	
	\item \textbf{Tái chế khí chưa phản ứng}
	Hỗn hợp khí $N_2$ và $H_2$ chưa phản ứng sau khi tách $NH_3$ sẽ được đưa trở lại tháp tổng hợp để tiếp tục phản ứng. Việc tái chế này giúp tối ưu hóa hiệu suất sử dụng nguyên liệu và tăng hiệu quả kinh tế của toàn bộ quy trình.
\end{itemize}
\begin{hoivadap}
	Dựa vào nhiệt độ sôi của $N_2$, $H_2$ và $NH_3$ (lần lượt là $-196$ $^\circ\text{C} $, $-253$ $^\circ\text{C} $ và $-33$ $^\circ\text{C} $), hãy giải thích chi tiết phương pháp tách $NH_3$ ra khỏi hỗn hợp sau phản ứng trong quy trình Haber-Bosch.
\end{hoivadap}
\subsubsection{Muối ammonium}


\Noibat[\maunhan][][][]{Sự điện li và tính tan ion Ammonium ($NH_4^+$)}

Khi tan trong nước, muối amoni phân li hoàn toàn thành ion. Điều này giải thích tại sao dung dịch các muối amoni có tính dẫn điện tốt.
Ion amoni ($NH_4^+$) được xem là một acid yếu theo thuyết Brønsted-Lowry. Trong dung dịch, ion $NH_4^+$ có khả năng nhường proton ($H^+$) cho nước, tạo thành ammonia và ion hiđronium ($H_3O^+$). Đây là một phản ứng thuận nghịch:
\[
\text{NH}_4^+\text{ (aq)} + \text{H}_2\text{O}\text{ (l)} \xharpoonarrow \text{NH}_3\text{ (aq)} + \text{H}_3\text{O}^+\text{ (aq)}
\]
Tính acid yếu này của ion amoni ($NH_4^+$) là lý do vì sao phân bón chứa gốc amoni có thể làm tăng độ chua của đất khi được bón lâu ngày, do giải phóng $H_3O^+$.

\begin{Bancobiet}
	Phân bón chứa ion amoni ($NH_4^+$) như đạm amoni (ví dụ $NH_4Cl$, $(NH_4)_2SO_4$) có tính chất làm tăng độ chua của đất sau quá trình cây hấp thụ $NH_4^+$ và nitrat hóa. Do đó, loại phân này thường thích hợp cho đất chua nhưng cần lưu ý điều chỉnh pH đất nếu bón liên tục.
\end{Bancobiet}

\Noibat[\maunhan][][][]{Tác dụng của Muối amoni với Base}

Một trong những tính chất hóa học quan trọng của muối amoni là khả năng tác dụng với các dung dịch kiềm (base). Phản ứng này đặc biệt xảy ra dễ dàng khi đun nóng, giải phóng khí ammonia ($NH_3$) có mùi khai đặc trưng.
\[
\text{NH}_4^+\text{ (aq)} + \text{OH}^-\text{ (aq)} \xrightarrow[$t^\circ$] \text{NH}_3\uparrow + \text{H}_2\text{O}\text{ (l)}
\]
Phản ứng này được sử dụng để nhận biết ion amoni ($NH_4^+$) trong dung dịch. Khi thêm dung dịch kiềm vào dung dịch chứa ion $NH_4^+$ và đun nóng nhẹ, nếu có khí mùi khai thoát ra làm xanh giấy quỳ ẩm thì có sự hiện diện của $NH_4^+$.

\Noibat[\maunhan][][][]{Ví dụ phản ứng:}
\begin{itemize}
	\item \textbf{Amoni clorua tác dụng với natri hiđroxit:}
	\[
	\text{NH}_4\text{Cl}\text{ (aq)} + \text{NaOH}\text{ (aq)} \xrightarrow[$t^\circ$] \text{NaCl}\text{ (aq)} + \text{NH}_3\uparrow + \text{H}_2\text{O}\text{ (l)}
	\]
	\item \textbf{Amoni sunfat tác dụng với natri hiđroxit:}
	\[
	(\text{NH}_4)_2\text{SO}_4\text{ (aq)} + 2\text{NaOH}\text{ (aq)} \xrightarrow[$t^\circ$] \text{Na}_2\text{SO}_4\text{ (aq)} + 2\text{NH}_3\uparrow + 2\text{H}_2\text{O}\text{ (l)}
	\]
\end{itemize}

\begin{hoivadap}
	Một học sinh muốn kiểm tra xem một mẫu phân bón X có chứa ion amoni ($NH_4^+$) hay không. Em hãy đề xuất một thí nghiệm đơn giản để nhận biết sự có mặt của ion này.
\end{hoivadap}
\Noibat[\maunhan][][][]{Phản ứng nhiệt phân của muối ammonium}

Khi đun nóng, các muối amoni bị phân hủy. Sản phẩm nhiệt phân phụ thuộc vào bản chất của gốc axit.
\begin{itemize}
	\item \textbf{Muối của axit không có tính oxi hóa (ví dụ: $NH_4Cl, (NH_4)_2CO_3$):} Bị phân hủy tạo ra khí $NH_3$ và axit tương ứng.
	\[
	\text{NH}_4\text{Cl}\text{ (s)} \xrightarrow[$t^\circ$] \text{NH}_3\text{ (g)} + \text{HCl}\text{ (g)}
	\]
	\[
	(\text{NH}_4)_2\text{CO}_3\text{ (s)} \xrightarrow[$t^\circ$] 2\text{NH}_3\text{ (g)} + \text{CO}_2\text{ (g)} + \text{H}_2\text{O}\text{ (g)}
	\]
	\item \textbf{Muối của axit có tính oxi hóa (ví dụ: $NH_4NO_2, NH_4NO_3$):} Phản ứng oxi hóa-khử nội phân tử. Nitrogen trong ion $NH_4^+$ (số oxi hóa $-3$) bị oxi hóa bởi nitrogen trong gốc axit (số oxi hóa dương) để tạo ra các sản phẩm khí ($N_2, N_2O$).
	\[
	\text{NH}_4\text{NO}_2\text{ (s)} \xrightarrow[$t^\circ$] \text{N}_2\text{ (g)} + 2\text{H}_2\text{O}\text{ (g)}
	\]
	(Phản ứng này thường dùng để điều chế khí $N_2$ trong phòng thí nghiệm)
	\[
	\text{NH}_4\text{NO}_3\text{ (s)} \xrightarrow[$t^\circ$] \text{N}_2\text{O}\text{ (g)} + 2\text{H}_2\text{O}\text{ (g)}
	\]
	(Ở nhiệt độ cao hơn, $NH_4NO_3$ có thể phân hủy tạo $N_2$).
\end{itemize}

\Noibat[\maunhan][][][]{Ứng dụng của Muối ammonium}

Muối amoni được sử dụng chủ yếu làm phân đạm trong nông nghiệp, cung cấp nguồn nitrogen cần thiết cho cây trồng. Các loại phân đạm phổ biến bao gồm:
\begin{itemize}
	\item Amoni clorua ($NH_4Cl$)
	\item Amoni nitrat ($NH_4NO_3$)
	\item Amoni sunfat ($(NH_4)_2SO_4$)
	\item Diamoni hiđrophotphat $(NH_4)_2HPO_4$)
\end{itemize}
Ngoài ra, một số muối amoni còn có ứng dụng trong sản xuất pháo hoa (ví dụ $NH_4NO_3$), trong ngành y tế và công nghiệp khác.

\begin{tongket}{Kiến thức cần nhớ}
	\begin{itemize}
		\item \textbf{Ammonia ($NH_3$):} Tan rất tốt trong nước (do phân tử phân cực và tạo liên kết hiđro với các phân tử nước).
		\item \textbf{Muối ammonium:}
		\begin{itemize}
			\item Tính tan: Hầu hết tan tốt trong nước.
			\item Tính acid yếu của ion $NH_4^+$: $NH_4^+ + H_2O \rightleftharpoons NH_3 + H_3O^+$.
			\item Tác dụng với base: $NH_4^+ + OH^- \xrightarrow[$t^\circ$] NH_3\uparrow + H_2O$ (dùng để nhận biết $NH_4^+$).
			\item Nhiệt phân:
			\begin{itemize}
				\item Muối của axit không oxi hóa: Tạo $NH_3$ và axit.
				\item Muối của axit có tính oxi hóa: Phản ứng oxi hóa-khử nội phân tử, tạo $N_2$ hoặc $N_2O$.
			\end{itemize}
			\item Ứng dụng: Chủ yếu làm phân đạm.
		\end{itemize}
	\end{itemize}
\end{tongket}






