\setlength{\baselineskip}{1.12\baselineskip}
\begin{enumerate}[a)]
	\item \noindent  Ta có các phương trình phản ứng sau:\\
	$\begin{array}{lcccccccr}
		&2Al& +& 3H_2SO_4 &\muiten{->}& Al_2({SO_4})_3& + & 3H_2 &(1)\\
		& x &\muiten{->}& \dfrac{3x}{2} &\muiten{->}&\dfrac{x}{2}&\muiten{->}& \dfrac{3x}{2} & \small\text{(mol)}\\
		& & &  & & & &  &\\
		& Mg & +& H_2SO_4 &\muiten{->}& MgSO4& + & H_2 & (2)\\
		&y &\muiten{->}& y &\muiten{->}& y &\muiten{->}& y & \small\text{(mol)}\\
		& & &  & & & &  &\\
	\end{array}$\\
	\item Theo phương trình (1) và (2), ta có $n_{H_2} = \dfrac{3x}{2} + y = \dfrac{9.744}{22\cdot4} = 0{,}435\mathrm{~(I)} $\\
	Mặt khác theo giả thiết ta có khối lượng hỗn hợp:
	$27x + 24y = 9{,}09\mathrm{~(II)}$\\
	Kết hợp (I) với (II) ta có hệ phương trình: \\
	$\left\{
	\begin{array}{*{6}c}
		\dfrac{3}{2}x& + & \phantom{24}y & = & 0{,}435\\
		 27x & + & 24y& = & 9{,}09 \\
	\end{array}
	\right.$ 
	$\Leftrightarrow$
	$\left\{\begin{array}{ccc}
		x& = & 0{,}15\\
		y& = & 0{,}21 \\
	\end{array}\right.$ \\
	$\Rightarrow$ $\% m_{Al} =\dfrac{0{,}15\cdot 27}{9{,}09}\cdot 1000\% \approx 44{,}55\%$ ; $\% m_{Mg} = 100\% - 44{,}55\% \approx 55{,}45\%$
	\item Từ phương trình (1) và (2) ta có: $n_{{H_2SO_4}_{\text{phản ứng}}} = n_{H_2} = 0{,}435 \mathrm{~mol}$\\
	$\Rightarrow$ $n_{{H_2SO_4}_{\text{ban đầu}}} = n_{{H_2SO_4}_{\text{phản ứng}}} + n_{{H_2SO_4}_{\text{dư}}} = 0{,}435 + 0{,}2\cdot 0{,}435 = 0{,}522\mathrm{~(mol)}   $\\
	$\Rightarrow$ ${C\%}_{H_2SO_4}= \dfrac{0{,}522\cdot98}{400}\cdot 100\% \approx 12{,}789 \%$
\end{enumerate}


%%%=================== Bắt đầu BT_50 ============================%%%
\begin{bt}Cho $10.26\mathrm{~gam}$ hỗn hợp kim loại Al và Mg  tác dụng với $300\mathrm{~gam}$ dd HCl (dùng dư $20 \%$ so với lượng phản ứng) phản ứng kết thức thu được $12.096\mathrm{~\text{lít}}$ khí (đktc)
	\begin{enumerate}[a)]
		\item Viết phương trình phản ứng
		\item Tính phần trăm khối lượng mối kim loại trong hỗn hợp ban đầu.
		\item Tính nồng độ phần trăm của dung dịch axit đã dùng.
	\end{enumerate}
	\huongdan{
		\begin{enumerate}[a)]
			\item \noindent  Ta có các phương trình phản ứng sau:\\
			$\begin{array}{lcccccccr}
				&2Al&+& 6HCl &\muiten{->}&2AlCl_3& + & 3H_2 &(1)\\
				& x &\muiten{->}& 3x &\muiten{->}&x&\muiten{->}& \dfrac{3x}{2} & \small\text{(mol)}\\
				& & &  & & & &  &\\
				&Mg&+& HCl &\muiten{->}& MgCl_2& + & H_2 &(2)\\
				& x &\muiten{->}& 2x &\muiten{->}&x&\muiten{->}& x & \small\text{(mol)}\\
				& & &  & & & &  &\\
			\end{array}$\\
			\item Theo phương trình (1) và (2), ta có
			$n_{H_2} = \dfrac{3}{2}\cdot x + y = \dfrac{12.096}{22\cdot4}= 0.54\mathrm{~(I)} $\\
			Mặt khác theo giả thiết ta có khối lượng hỗn hợp:$27x + 24y = 10.26\mathrm{~(II)}$\\
			Kết hợp (I) với (II) ta có hệ phương trình: \\
			$\left\{
			\begin{array}{*{6}c}
				\dfrac{3}{2}\cdot x & + & y & = & 0.54\\
				27x & + &24y& = & 10.26\\
			\end{array}
			\right.$
			$\Leftrightarrow$
			$\left\{\begin{array}{ccc}
				x& = & 0.3\\
				y& = & 0.09 \\
			\end{array}\right.$
			$\Rightarrow$
			$\left\{\begin{array}{l}
				m_{Al}= 0.3\cdot\mathtt{\text{27}}=8.1\mathrm{~gam}\\
				m_{Mg}= 0.09\cdot\mathtt{\text{24}}=2.16\mathrm{~gam}\\
			\end{array}\right.$ \\
			$\Rightarrow$ $\% m_{Al} =\dfrac{8.1}{10.26}\cdot 100\% \approx 78.95\%$ ;
			$\% m_{Mg} = 100\% - 78.95\% \approx 21.05\%$
			\item Từ phương trình (1) và (2) ta có: $n_{{HCl}_{\text{phản ứng}}} = 2n_{H_2} = 1.08\mathrm{~mol}$\\
			$\Rightarrow$ $n_{{HCl}_{\text{ban đầu}}} = n_{{HCl}_{\text{phản ứng}}} + n_{{HCl}_{\text{dư}}} = 1.08 + 0{,}2\cdot 1.08 = 
			1.296\mathrm{~(mol)}$\\
			$\Rightarrow$ ${C\%}_{HCl}= \dfrac{1.296\cdot36.5}{\mathtt{\text{300}}}\cdot 100\% \approx 15.768 \%$
		\end{enumerate}
	}
\end{bt}
%%%=================== Kết thúc BT_50 ============================%%%

