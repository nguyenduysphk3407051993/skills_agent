
\documentclass[border=4pt,tikz]{standalone}
\usepackage[utf8]{vietnam}
\usetikzlibrary{calc,angles,intersections,shapes.geometric,arrows,decorations.markings,arrows.meta,patterns.meta,patterns}
\usetikzlibrary {matrix}
\usepackage{tikz-3dplot,pgfplots}
\pgfplotsset{compat=1.15}
\usepgfplotslibrary{polar}
\usepackage{amsmath}
\usepackage{xcolor} 
\definecolor{dnvang}{HTML}{994D1C}
\definecolor{dnxanh}{HTML}{0766AD}%0B60B0 %3A98B9 %3081D0 %0766AD
\definecolor{dnxanhdam}{HTML}{19376D}
\definecolor{dndo}{HTML}{BB2649}
\def\mycolor{dnvang}
\def\mauphu{dnxanh}
\def\maudam{dnxanhdam}
\def\maunhan{dndo}
\begin{document}
%%%===Quy tắc Klechkowski=====%%%%
\begin{center}
	\begin{tikzpicture}[line join=round,line cap=round,line width=1pt]
		\tikzstyle{mynode} =[
		font=\color{white}\bfseries\fontfamily{qag}\selectfont,
		inner sep =2pt,
		outer sep =2pt,
		align =center,
		circle,
		text width =0.5cm,
		minimum width = 0.8cm,
		minimum height =0.8cm
		]
		\tikzstyle{mymatrix} = [
		matrix of nodes,
		nodes={mynode},
		column sep=2.5mm-\pgflinewidth,
		row sep = 2.5mm-\pgflinewidth,
		ampersand replacement=\&,
		fill=\mycolor!30,rounded corners
		]
		\matrix(mt) [mymatrix]
		{
			 1s \&    \&    \&    \\
			 2s \& 2p \&    \&    \\
			 3s \& 3p \& 3d \&    \\
			 4s \& 4p \& 4d \& 4f \\
			 5s \& 5p \& 5d \&    \\
			 6s \& 6p \&    \&    \\
			 7s \& 	  \&    \&    \\
		};
		\foreach \x/\y/\t/\u in {1/1/1/1,2/1/2/1,2/2/3/1,3/2/4/1,3/3/5/1,4/3/6/1,4/4/7/1}{
			\draw[-stealth,\maunhan!90!black] ($(mt-\x-\y.north east)+(45:1.0mm)$)--($(mt-\t-\u.south west)+(-135:2mm)$);
		}
		\foreach \x/\y/\c/\n in {
		1/1/\mycolor/1s,2/1/\mycolor/2s,3/1/\mycolor/3s,4/1/\mycolor/4s,5/1/\mycolor/5s,6/1/\mycolor/6s,7/1/\mycolor/7s,2/2/\mauphu/2p,3/2/\mauphu/3p,4/2/\mauphu/4p,5/2/\mauphu/5p,6/2/\mauphu/6p,3/3/\maunhan/3d,4/3/\maunhan/4d,5/3/\maunhan/5d,4/4/violet/4f
		}{
			\path (mt-\x-\y) node [mynode,fill=\c!80!white] {\n};
		}
		\path (mt.north) node [font=\color{\maunhan}\bfseries\fontfamily{qag}\selectfont,align=center,anchor=south]{Quy tắc Klechkowski};
	\end{tikzpicture}
\end{center}
\end{document}
