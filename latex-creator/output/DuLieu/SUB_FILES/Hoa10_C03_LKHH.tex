\documentclass[Main.tex]{subfiles}
	\NewDocumentCommand{\khungion}{O{+}O{-2}m}{
		\raisebox{#2pt}[0pt][0pt]{%% Thêm [0pt][0pt] để giữ nguyên baseline
			\tikz[baseline]{
				\path (0,0) node[inner sep=5pt,outer sep=0pt] (char){#3};
				\draw[rounded corners=0.5pt,line cap=round,line join=round,line width=1pt] 
				($(char.north west)+(3pt,0)$)--($(char.north west)+(-4pt,0)$)--
				($(char.south west)+(-4pt,0)$)--($(char.south west)+(3pt,0)$);
				\draw[rounded corners=0.5pt,line cap=round,line join=round,line width=1pt] 
				($(char.north east)+(-3pt,0)$)--($(char.north east)+(4pt,0)$)coordinate (D)--
				($(char.south east)+(4pt,0)$)--($(char.south east)+(-3pt,0)$);
				\path (char.north east) node[shift={(35:5pt)}]{\textbf{#1}};
			}%
		}%
	}
	%%%Lệnh vé công thức electron
	\makeatletter
	\define@key{electron}{angle}{\def\electron@angle{#1}}
	\define@key{electron}{color}{\def\electron@color{#1}}
	\define@key{electron}{type}{\def\electron@type{#1}}
	\define@key{electron}{radius}{\def\electron@radius{#1}}
	
	\NewDocumentCommand{\setElectronKeys}{m}{%
		\setkeys{electron}{angle=0,color=\maunhan,type=0,radius=9pt,#1}%
	}
	
	% Định nghĩa các electron
	\newcommand{\electronM}[1][\maunhan]{\tikz{\fill[#1](0,0) circle (1.2pt);}}
	\newcommand{\electronH}[1][\maunhan]{\tikz{\fill[#1](0,0) circle (1.2pt) (0,4pt) circle (1.2pt);}}
	\newcommand{\electronB}[1][\maunhan]{\tikz{\fill[#1](0,0) circle (1.2pt) (0,3pt) circle (1.2pt)(0,6pt) circle (1.2pt);}}
	
	\NewDocumentCommand{\congthuce}{%
		O{}  % Tùy chọn 1
		O{}  % Tùy chọn 2
		O{}  % Tùy chọn 3
		O{}  % Tùy chọn 4
		O{}  % Tùy chọn 5
		O{}  % Tùy chọn 6
		m    % Ký hiệu nguyên tố 1
		m    % Ký hiệu nguyên tố 2
	}{\begin{tikzpicture}[baseline={(#7.base)}]
			% Vẽ nguyên tố trung tâm
			\node[font=\large\bfseries\sffamily, inner sep=3pt, anchor=base] (#7) at (0,0) {#8};
			
			% Chỉ xử lý các tùy chọn không rỗng
			\if\relax\detokenize{#1}\relax\else
			\setElectronKeys{#1}%
			\ifnum\electron@type>0
			\ifnum\electron@type=1
			\path (#7) -- ++(\electron@angle:\electron@radius) node[pos=1,sloped] {\electronM[\electron@color]};
			\fi
			\ifnum\electron@type=2
			\path (#7) -- ++(\electron@angle:\electron@radius) node[pos=1,sloped] {\electronH[\electron@color]};
			\fi
			\ifnum\electron@type=3
			\path (#7) -- ++(\electron@angle:\electron@radius) node[pos=1,sloped] {\electronB[\electron@color]};
			\fi
			\fi
			\fi
			
			\if\relax\detokenize{#2}\relax\else
			\setElectronKeys{#2}%
			\ifnum\electron@type>0
			\ifnum\electron@type=1
			\path (#7) -- ++(\electron@angle:\electron@radius) node[pos=1,sloped] {\electronM[\electron@color]};
			\fi
			\ifnum\electron@type=2
			\path (#7) -- ++(\electron@angle:\electron@radius) node[pos=1,sloped] {\electronH[\electron@color]};
			\fi
			\ifnum\electron@type=3
			\path (#7) -- ++(\electron@angle:\electron@radius) node[pos=1,sloped] {\electronB[\electron@color]};
			\fi
			\fi
			\fi
			
			\if\relax\detokenize{#3}\relax\else
			\setElectronKeys{#3}%
			\ifnum\electron@type>0
			\ifnum\electron@type=1
			\path (#7) -- ++(\electron@angle:\electron@radius) node[pos=1,sloped] {\electronM[\electron@color]};
			\fi
			\ifnum\electron@type=2
			\path (#7) -- ++(\electron@angle:\electron@radius) node[pos=1,sloped] {\electronH[\electron@color]};
			\fi
			\ifnum\electron@type=3
			\path (#7) -- ++(\electron@angle:\electron@radius) node[pos=1,sloped] {\electronB[\electron@color]};
			\fi
			\fi
			\fi
			
			\if\relax\detokenize{#4}\relax\else
			\setElectronKeys{#4}%
			\ifnum\electron@type>0
			\ifnum\electron@type=1
			\path (#7) -- ++(\electron@angle:\electron@radius) node[pos=1,sloped] {\electronM[\electron@color]};
			\fi
			\ifnum\electron@type=2
			\path (#7) -- ++(\electron@angle:\electron@radius) node[pos=1,sloped] {\electronH[\electron@color]};
			\fi
			\ifnum\electron@type=3
			\path (#7) -- ++(\electron@angle:\electron@radius) node[pos=1,sloped] {\electronB[\electron@color]};
			\fi
			\fi
			\fi
			
			\if\relax\detokenize{#5}\relax\else
			\setElectronKeys{#5}%
			\ifnum\electron@type>0
			\ifnum\electron@type=1
			\path (#7) -- ++(\electron@angle:\electron@radius) node[pos=1,sloped] {\electronM[\electron@color]};
			\fi
			\ifnum\electron@type=2
			\path (#7) -- ++(\electron@angle:\electron@radius) node[pos=1,sloped] {\electronH[\electron@color]};
			\fi
			\ifnum\electron@type=3
			\path (#7) -- ++(\electron@angle:\electron@radius) node[pos=1,sloped] {\electronB[\electron@color]};
			\fi
			\fi
			\fi
			
			\if\relax\detokenize{#6}\relax\else
			\setElectronKeys{#6}%
			\ifnum\electron@type>0
			\ifnum\electron@type=1
			\path (#7) -- ++(\electron@angle:\electron@radius) node[pos=1,sloped] {\electronM[\electron@color]};
			\fi
			\ifnum\electron@type=2
			\path (#7) -- ++(\electron@angle:\electron@radius) node[pos=1,sloped] {\electronH[\electron@color]};
			\fi
			\ifnum\electron@type=3
			\path (#7) -- ++(\electron@angle:\electron@radius) node[pos=1,sloped] {\electronB[\electron@color]};
			\fi
			\fi
			\fi
		\end{tikzpicture}
	}
	\makeatother
\begin{document}
	\setcounter{chapter}{2}
	\chapter{Liên kết hóa học}
	\sodongkeH[6]{hoivadap}
%	\tatloigiai
	\hienthiloigiai
	\section{Quy tắc octet}
\begin{Muctieu}
	\begin{itemize}
		\item Trình bày được quy tắc octet với các nguyên tố nhóm $A$.
		\item Vận dụng được quy tắc octet trong quá trình hình thành liên kết hoá học ở các nguyên tố nhóm $A$.
	\end{itemize}
\end{Muctieu}
\begin{kd}
	\hinhphai{\lq\lq Hãy quan sát hai hình ảnh bên:
		\begin{enumerate}
			\item  Một quả bóng đang lăn từ đỉnh đồi xuống chân đồi (hình a)
			\item  Electron ở lớp vỏ ngoài cùng của nguyên tử Natri (Na) (hình b)
		\end{enumerate}
		Theo em, hai hiện tượng này có điểm gì giống nhau về mặt xu hướng năng lượng (muốn trở về trạng thái năng lượng thấp hơn hay cao hơn)?\rq\rq}{\tikz{
			\node[name=ball] at (0,0) {\includegraphics[width=4cm]{Images/anhhoahoc10/anhminhoa/ball_on_hill.png}};
			\node[right=0.5cm of ball, name = na] {\includegraphics[width=4cm]{Images/Tikz/Na.pdf}};
			\node [below= 2mm of ball, font=\scriptsize] {\textbf{Hình a}};
			\node [below= 2mm of na,font=\scriptsize] {\textbf{Hình b}};
		}
	}
\end{kd}
\subsection{Nội dung bài học}
	\subsubsection{Liên kết hóa học}
\Noibat[\maunhan][][\faArrowCircleORight][]{Tìm hiểu sự hình thành liên kết hóa học}
	\begin{center}
		\includegraphics[width=12cm]{Images/Tikz/suhinhthanhlienketion.pdf}
		\includegraphics[width=12cm]{Images/Tikz/suhinhthanhlienketF2.pdf}
		\captionof{figure}{Sự hình thành liên kết trong phân tử Sodium chloride và Flourine \label{fig:Suhinhthanhphantu}}
	\end{center}
\begin{hoivadap}
	\begin{cauhoi}
		Dựa vào hình \ref{fig:Suhinhthanhphantu}các em có nhận xét gì về cấu hình e lớp ngoài cùng ( số e xung quanh mỗi nguyên tử) của các nguyên tử sau khi tham gia tạo thành liên kết?
	\end{cauhoi}
	\loigiai{%
		Sau khi tham gia liên kết hóa học các nguyên tử đều có 8 electron ở lớp ngoài cùng
	}
\end{hoivadap}

\begin{hopdongian}
	Theo thuyết cấu tạo hoá học, sự liên kết giữa các nguyên tử tạo thành phân tử hay tinh thể được giải thích bằng sự giảm năng lượng khi các nguyên tử kết hợp lại với nhau. Khi tạo liên kết hoá học thì nguyên tử có xu hướng đạt tới cấu hình electron bền vững của khí hiếm (2 e hoặc 8 e ở lớp ngoài cùng).
\end{hopdongian}
\begin{ghinho}
	\textit{\textbf{Liên kết hoá học} là sự kết hợp giữa các nguyên tử tạo thành phân tử hay tinh thể bền vững hơn.}
\end{ghinho}
\subsubsection{Quy tắc Octet}
\Noibat[\maunhan][][\faArrowCircleORight][]{Tìm hiểu quy tắc octet (bát tử)}
	\begin{ghinho}
	\indam[\maunhan]{Quy tắc octet (bát tử):}
	Trong quá trình hình thành liên kết hoá học, nguyên tử của các nguyên tố nhóm A có xu hướng tạo thành lớp vỏ ngoài cùng có 8 electron tương ứng với khí hiếm gẩn nhất (hoặc 2 electron với khí hiếm helium). 
	\end{ghinho}
\Noibat[\maunhan][][\faArrowCircleORight][]{Tìm hiểu cách vận dụng quy tắc Octet  trong hình thành phân tử Nitrogen ($\textbf{N}_\text{2}$)}
	\begin{center}
		\includegraphics[width=9cm]{Images/Tikz/suhinhthanhlienketN2.pdf}
		\captionof{figure}{Sự hình thành liên kết trong phân tử Nitrogen\label{fig:phantuN2}}
	\end{center}
	\begin{hoivadap}
		Từ hình \ref{fig:phantuN2}, cho biết mối nguyên tử nitrogen đã đạt được cấu hình electron bến vững của nguyên tử khí hiếm nào.
		\loigiai{Mỗi nguyên tử nitrogen đã đạt đuọc cấu hình elctron bền vững của nguyên tố khí hiếm Ne}
	\end{hoivadap}
	\begin{hoivadap}
		Nguyên tử của các nguyên tố hydrogen và fluorine có xu huớng cho đi, nhận thêm hay góp chung các electron hoá trị khi tham gia liên kết hình thành phân tử hydrogen fluoride (HF)?
		\loigiai{Nguyên tử H và F đều cần thêm 1 electron nữa để đạt cấu hình bền của khí hiếm nên mỗi nguyên tử H và F sẽ chia sẻ 1 electron để góp chung}
	\end{hoivadap}
\Noibat[\maunhan][][\faArrowCircleORight][]{Tìm hiểu cách vận dụng quy tắc octet trong sự hình thành ion dương, ion âm}
	\begin{center}
		\includegraphics[width=9cm]{Images/Tikz/suhinhthanhionNa.pdf}
		\captionof{figure}{Sự tạo thành ion $Na^+$}
		\includegraphics[width=9cm]{Images/Tikz/suhinhthanhionF.pdf}
		\captionof{figure}{Sự tạo thành ion $F^-$}
	\end{center}
	\begin{hoivadap}
		Biết phân tử magnesium oxide hình thành bởi các ion $\mathrm{Mg}^{2+}$ và $\mathrm{O}^{2-}$. Vận dụng quy tắc octet, trình bày sự hình thành các ion trên từ những nguyên tử tương ứng.
		\loigiai{%
			\begin{center}
				\begin{tabular}{cccccc}
				&$Mg$&$\rightarrow$&$Mg^{2+}$&$+$&$2e$\\
				Cấu hình electron& $[Ne]3s^2$&$\rightarrow$&$[Ne]:$&&\\
				&$O$&$+$&$2e$&$\rightarrow$&$O^{2-}$\\
				Cấu hình electron& $[He]2s^22p^4$&$\rightarrow$&$[Ne]:$&&\\
		    	\end{tabular}
			\end{center}
		    \\
			Phương trình phân tử: $2Mg + O_2 \rightarrow 2MgO$}
	\end{hoivadap}

\subsection{Các dạng bài tập}
	\phan{Trắc nghiệm nhiều lựa chọn}
%%%=============SOẠN EX===============%%%
\Opensolutionfile{ansex}[Ans/LGEX-C03B01_QTOCTET_01.tex]
\Opensolutionfile{ans}[Ans/Ans-C03B01_QTOCTET_01.tex]
%%%=============EX_1=============%%%
\begin{ex}
	Công thức cấu tạo nào sau đây không đủ electron theo quy tắc octet?
	\choice
	{\setcharge{{.style={fill=\mycolor!50!black,draw=none}}}
		\chemfig{
			H
			-[,0.48,,,draw=none]
			\charge{[.radius=0.2ex]
				90:2pt=\:,
				180:2pt=\:,
				-90:2pt=\:,
				0:2pt=\:
			}{N}
			(-[-90,0.48,,,draw=none]H)-[,0.48,,,draw=none]H
		}
		\resetcharge}
	%%%
	{\True \setcharge{{.style={fill=\mycolor!50!black,draw=none}}}
		\chemfig{
			H
			-[,0.48,,,draw=none]
			\charge{[.radius=0.2ex]
				180:2pt=\:,
				-90:2pt=\:,
				0:2pt=\:
			}{B}
			(-[-90,0.48,,,draw=none]H)-[,0.48,,,draw=none]H
		}
		\resetcharge}
	%%%
	{\setcharge{{.style={fill=\mycolor!50!black,draw=none}}}
		\chemfig{
			\charge{[.radius=0.2ex]
				0:0.4pt=\:,
				120:0.5pt=\:,
				-120:0.5pt=\:
			}{O}
			-[,0.51,,,draw=none]
			\charge{[.radius=0.2ex]
				180:2pt=\:,
				0:2pt=\:
			}{C}
			-[,0.51,,,draw=none]
			\charge{[.radius=0.2ex]
				180:0.4pt=\:,
				60:0.5pt=\:,
				-60:0.5pt=\:
			}{O}
		}
		\resetcharge}
	%%%
	{\setcharge{{.style={fill=\mycolor!50!black,draw=none}}}
		\chemfig{
			\charge{[.radius=0.2ex]
				0:0.5pt=\:,
				180:0.5pt=\:,
				90:0.5pt=\:,
				-90:0.5pt=\:
			}{Cl}
			-[,0.45,,,draw=none]
			\charge{[.radius=0.2ex]
				0:0.5pt=\:,
				90:0.5pt=\:,
				-90:0.5pt=\:
			}{Cl}
		}
		\resetcharge}
	%%%
	\loigiai{Trong BH$_3$, nguyên tử B chỉ có 6 electron lớp ngoài cùng, chưa đạt octet.}
\end{ex}
%%%=============EX_2=============%%%
\begin{ex}
	Liên kết hoá học là
	\choice
	{sự kết hợp của các hạt cơ bản hình thành nguyên tử bền vững}
	{\True sự kết hợp giữa các nguyên tử tạo thành phân tử hay tinh thể bền vững hơn}
	{sự kết hợp của các phân tử hình thành các chất bền vững}
	{sự kết hợp của chất tạo thành vật thể bền vững}
	\loigiai{Liên kết hóa học là sự kết hợp giữa các nguyên tử tạo thành phân tử hay tinh thể bền vững hơn. Sự kết hợp này làm giảm năng lượng của hệ, tạo ra hệ bền vững hơn.}
\end{ex}
%%%=============EX_3=============%%%
\begin{ex}
	Theo quy tắc octet, khi hình thành liên kết hoá học, các nguyên tử có xu hướng nhường, nhận hoặc góp chung electron để đạt tới cấu hình electron bền vững giống như
	\choice
	{kim loại kiềm gần kề}
	{kim loại kiềm thổ gần kề}
	{nguyên tử halogen gần kề}
	{\True nguyên tử khí hiếm gần kề}
	\loigiai{Theo quy tắc octet, khi hình thành liên kết hoá học, các nguyên tử có xu hướng nhường, nhận hoặc góp chung electron để đạt tới cấu hình electron bền vững của khí hiếm gần nhất.}
\end{ex}
%%%=============EX_4=============%%%
\begin{ex}
	Khi hình thành liên kết hoá học, nguyên tử có số hiệu nào sau đây có xu hướng nhường 2 electron để đạt cấu hình electron bền vững theo quy tắc octet?
	\choice
	{\True $(Z=12)$}
	{$(Z=9)$}
	{$(Z=11)$}
	{$(Z=10)$}
	\loigiai{Nguyên tử có Z = 12 (Mg) có cấu hình electron là [Ne]3s$^2$. Khi hình thành liên kết hóa học, Mg có xu hướng nhường 2 electron để đạt cấu hình electron bền vững của khí hiếm Ne.}
\end{ex}
%%%=============EX_5=============%%%
\begin{ex}
	Trong công thức $CS_2$, tổng số cặp electron lớp ngoài cùng của C và S chưa tham gia liên kết là
	\choice
	{$2$}
	{$3$}
	{\True $4$}
	{$5$}
	\loigiai{Cấu hình electron của C là [He]2s$^2$2p$^2$ (2 cặp electron chưa liên kết). Cấu hình electron của S là [Ne]3s$^2$3p$^4$ (2 cặp electron chưa liên kết). Vậy tổng số cặp electron lớp ngoài cùng của C và 2 nguyên tử S chưa tham gia liên kết là 2 + 2$\times$2 = 4.}
\end{ex}
%%%=============EX_6=============%%%
\begin{ex}
	Phân tử nào sau đây có các nguyên tử đều đã đạt cấu hình electron bão hoà theo quy tắc octet?
	\choice
	{$\mathrm{BeH}_2$}
	{$\mathrm{AlCl}_3$}
	{$\mathrm{PCl}_5$}
	{\True $\mathrm{SiF}_4$}
	\loigiai{Trong phân tử SiF$_4$, nguyên tử Si có 8 electron lớp ngoài cùng và 4 nguyên tử F đều có 8 electron lớp ngoài cùng.}
\end{ex}
%%%=============EX_7=============%%%
\begin{ex}
	Quy tắc octet không đúng với trường hợp phân tử chất nào sau đây?
	\choice
	{$H_2O$}
	{\True $NO_2$}
	{$CO_2$}
	{$\mathrm{Cl}_2$}
	\loigiai{Phân tử NO$_2$ có nguyên tử N có 7 electron lớp ngoài cùng, không tuân theo quy tắc octet.}
\end{ex}
%%%=============EX_8=============%%%
\begin{ex}
	Vì sao các nguyên tử lại liên kết với nhau thành phân tử?
	\choice
	{\True Để mỗi nguyên tử trong phân tử đạt được cơ cấu electron ổn định, bền vững}
	{Đề mỗi nguyên tử trong phân tử đều đạt 8 electron ở lớp ngoài cùng}
	{Để tổng số electron ngoài cùng của các nguyên tử trong phân tử là 8}
	{Để lớp ngoài củng của mỗi nguyên tử trong phân tử có nhiều electron độc thân nhất}
	\loigiai{Các nguyên tử liên kết với nhau thành phân tử để đạt được cấu hình electron bền vững hơn, làm giảm năng lượng của hệ.}
\end{ex}
%%%=============EX_9=============%%%
\begin{ex}
	Nguyên tử nào sau đây có khuynh hướng đạt cấu hình electron bền của khí hiếm neon khi tham gia hình thành liên kết hoá học?
	\choice
	{\True Chlorine}
	{Sulfur}
	{Oxygen}
	{Hydrogen}
	\loigiai{Nguyên tử chlorine (Cl) có 7 electron lớp ngoài cùng. Khi tham gia liên kết hóa học, Cl có xu hướng nhận thêm 1 electron để đạt cấu hình electron bền vững của khí hiếm neon.}
\end{ex}
%%%=============EX_10=============%%%
\begin{ex}
	Sodium hydride $(\mathrm{NaH})$ là một hợp chất được sử dụng như một chất lưu trữ hydrogen trong các phương tiện chạy bằng pin nhiên liệu do khả năng giải phóng hydrogen của nó. Trong sodium hydride, nguyên tử sodium có cấu hình electron bền của khí hiếm
	\choice
	{helium}
	{argon}
	{krypton}
	{\True neon}
	\loigiai{Sodium (Na) có cấu hình electron [Ne]3s$^1$. Khi tham gia liên kết hóa học, Na có xu hướng nhường 1 electron để đạt cấu hình electron bền vững của khí hiếm neon.}
\end{ex}
%%%=============EX_11=============%%%
\begin{ex}
	Khi tham gia hình thành liên kết hoá học, các nguyên tử lithium và chlorine có khuynh hướng đạt cấu hình electron bền của lần lượt các khí hiếm nào dưới đây?
	\choice
	{Helium và argon}
	{\True Helium và neon}
	{Neon và argon}
	{Argon và helium}
	\loigiai{Li có cấu hình electron [He]2s$^1$ có xu hướng nhường 1e để đạt cấu hình của He. Cl có cấu hình electron [Ne]3s$^2$3p$^5$ có xu hướng nhận 1e để đạt cấu hình của Ne.}
\end{ex}
%%%=============EX_12=============%%%
\begin{ex}
	Trong phân tử HBr, nguyên tử hydrogen và bromine đã lần lượt đạt cấu hình electron bền của các khi hiếm nào dưới đây?
	\choice
	{Neon và argon}
	{\True Helium và krypton}
	{Helium và radon}
	{Helium và argon}
	\loigiai{Trong phân tử HBr, H đạt cấu hình bền của He, Br đạt cấu hình bền của Kr.}
\end{ex}
%%%=============EX_13=============%%%
\begin{ex}
	Trong các hợp chất, nguyên tử magnesium đã đạt được cấu hình bền của khí hiếm gần nhất bằng cách
	\choice
	{\True cho đi 2 electron}
	{nhận vào 1 electron}
	{cho đi 3 electron}
	{nhận vào 2 electron}
	\loigiai{Magnesium (Mg) có cấu hình electron [Ne]3s$^2$. Trong các hợp chất, Mg đạt cấu hình bền vững của khí hiếm neon bằng cách nhường đi 2 electron.}
\end{ex}
%%%=============EX_14=============%%%
\begin{ex}
	Cho các phân tử sau: $\mathrm{Cl}_2, H_2O, \mathrm{NaF}$ và $CH_4$. Có bao nhiêu nguyên tử trong các phân tử trên đạt cấu hình electron bền của khi hiếm neon?
	\choice
	{\True $3$}
	{$2$}
	{$5$}
	{$4$}
	\loigiai{
		\begin{itemize}
			\item Trong $\mathrm{Cl}_2$:
			\begin{itemize}
				\item Mỗi nguyên tử Cl $(Z=17)$ có cấu hình $[Ne]3s^23p^5$
				\item Khi liên kết, mỗi Cl dùng chung 1 electron $\rightarrow$ đạt cấu hình $[Ar]$ chứ không phải $[Ne]$
			\end{itemize}
			\item Trong $\mathrm{H_2O}$:
			\begin{itemize}
				\item O $(Z=8)$ có cấu hình $1s^22s^22p^4$
				\item Khi liên kết với 2H, O nhận thêm 2 electron $\rightarrow$ đạt cấu hình $[Ne]$
			\end{itemize}
			\item Trong $\mathrm{NaF}$:
			\begin{itemize}
				\item F $(Z=9)$ nhận 1 electron từ Na $\rightarrow$ đạt cấu hình $[Ne]$
				\item Na $(Z=11)$ cho đi 1 electron $\rightarrow$ đạt cấu hình $[Ne]$
			\end{itemize}
			\item Trong $\mathrm{CH_4}$:
			\begin{itemize}
				\item C $(Z=6)$ có cấu hình $1s^22s^22p^2$
				\item Khi liên kết với 4H, C dùng chung 4 electron $\rightarrow$ đạt cấu hình $[Ne]$
			\end{itemize}
			
		\end{itemize}
		Vậy có 3 nguyên tử đạt cấu hình electron của Ne là: O (trong $\mathrm{H_2O}$), F và Na (trong $\mathrm{NaF}$).}
\end{ex}
%%%=============EX_15=============%%%
\begin{ex}
	Nguyên tử trong phân tử nào dưới đây ngoại lệ với quy tắc octet?
	\choice
	{$H_2O$}
	{$NH_3$}
	{HCl}
	{\True $BF_3$}
	\loigiai{Trong $BF_3$, nguyên tử B chỉ có 6 electron ở lớp ngoài cùng.}
\end{ex}
%%%=============EX_16=============%%%
\begin{ex}
	Nguyên tử oxygen $(Z=8)$ có xu hướng nhường hay nhận bao nhiêu electron để đạt lớp vỏ thoả mãn quy tắc octet? Chọn phương án đúng.
	\choice
	{Nhường $6$ electron}
	{\True Nhận $2$ electron}
	{Nhường $8$ electron}
	{Nhận $6$ electron}
	\loigiai{%
	Oxygen có cấu hình e lớp ngoài cùng là $2s^22p^4$ $\Rightarrow$ có 6 e lớp ngoài cùng , theo quy tắc octet Oxygen có xu hướng nhận thêm 2 e để trở thành cấu hình bền của khí hiếm
	}
\end{ex}
%%%=============EX_17=============%%%
\begin{ex}
	Nguyên tử lithium $(Z=3)$ có xu hướng nhường hay nhận bao nhiêu electron để lớp vỏ thoả mãn quy tắc octet? Chọn phương án đúng.
	\choice
	{\True Nhường $1$ electron}
	{Nhận $7$ electron}
	{Nhường $11$ electron}
	{Nhận $1$ electron}
	\loigiai{
	Lithium có cấu hình e là $1s^22s^1$ $\Rightarrow$ có 1 e lớp ngoài cùng , theo quy tắc octet Oxygen có xu hướng nhường đi 1 e để đạt cấu hình e bền của khí hiếm He $1s^2$.
	}
\end{ex}
%%%=============EX_18=============%%%
\begin{ex}
	Nguyên tử nào sau đầy có thể nhường hoặc nhận $4$ electron để đạt cấu hình electron bền vững?
	\choice
	{\True Silicon}
	{Beryllium}
	{Nitrogen}
	{Selenium}
	\loigiai{%
		Silicon có cấu hình e là $1s^22s^22p^63s^23p^2$ $\Rightarrow$ có 4 e lớp ngoài cùng , theo quy tắc octet Oxygen có xu hướng nhường đi 4 e để đạt cấu hình e bền của khí hiếm Ne $1s^22s^22p^6$ hoặc cũng có thể nhận thêm 4 e để đạt cấu hình e bền của khí hiếm Ar $1s^22s^22p^63s^23p^6$.
	}
\end{ex}
%%%=============EX_19=============%%%
\begin{ex}
	Nguyên tử nào sau đây không có xu hướng nhường hoặc nhận electron để đạt được lớp vỏ thoả mãn quy tắc octet?
	\choice
	{Nitrogen}
	{Oxygen}
	{Sodium}
	{\True Hydrogen}
	\loigiai{%
	Hydrogen không thể đạt được lớp vỏ thoả mãn quy tắc octet mà chỉ có thể đạt được lớp vỏ của khí hiếm gần nó nhất là helium (2 electron).
	}
\end{ex}
%%%=============EX_20=============%%%
\begin{ex}
	Nguyên tử nào trong các nguyên tử sau đây không có xu hướng nhường electron để đạt lớp vỏ thoả mãn quy tắc octet?
	\choice
	{Calcium}
	{Magnesium}
	{Potassium}
	{\True Chlorine}
	\loigiai{%
	Chlorine có 17 e và có 7 e lớp ngoài cùng nên có xu hướng nhận thêm 1 e để đạt cấu hình bền của khí hiếm Ar.
	}
\end{ex}
%%%=============EX_21=============%%%
\begin{ex}
	Mô hình mô tả quá trình tạo liên kết hoá học sau đây phù hợp với xu hướng tạo liên kết hoá học của nguyên tử nào?
	\begin{center}
		\includegraphics[height=3cm]{Images/Tikz/xuhuongnhane_ofphotphorus.pdf}
	\end{center}
	\choice
	{Aluminium}
	{Nitrogen}
	{\True Phosphorus}
	{Oxygen}
	\loigiai{%
	Dựa vào hình vẽ ta thấy nguyên tử có 15 e  và có 5 lớp ngoài cùng và nhận thêm 3 electron để đạt cấu hình e bền của khí hiếm Ar $\Rightarrow$ đây là nguyên tử Phosphorus.
	}
\end{ex}
%%%=============EX_22=============%%%
\begin{ex}
	Nguyên tử có mô hình cấu tạo sau đây có xu hướng nhường hoặc nhận electron như thế nào khi hình thành liên kết hoá học?
	\begin{center}
		\includegraphics[height=4cm]{Images/Tikz/xuhuongnhane_ofpotasium.pdf}
	\end{center}
	\choice[2]
	{Nhận 1 electron}
	{\True Nhường 1 electron}
	{Nhận 7 electron}
	{Không có xu hướng nhường hoặc nhận electron}
	\loigiai{%
	Nguyen tử có 19 electron và 1 electron ở lớp ngoài cùng nên có xu hướng nhường đi 1 electron để đạt cấu hình bền của khí hiếm Ar
	}
\end{ex}
%%%=============EX_23=============%%%
\begin{ex}
	Nguyên tử có mô hình cấu tạo sau sẽ có xu hướng tạo thành ion mang điện tích nào khi nó thoả mãn quy tắc octet?
	\begin{center}
		\includegraphics[height=3cm]{Images/Tikz/xuhuongnhane_ofAluminium.pdf}
	\end{center}
	\choice
	{\True $3+$}
	{$5+$}
	{3-}
	{$5-$}
	\loigiai{%
	Dựa vào hình vẽ ta thấy nguyên tử có 13 electron và có 3 electron ở lớp ngoài cùng nên có xu hướng nhường đi 3 e để đạt cấu hình bền của khí hiếm Ne. Khi nguyên tử mất đi 3 e sẽ trở thành cation mang điện tích $+3$.
	}
\end{ex}
%%%=============EX_24=============%%%
\begin{ex}
	Phân tử nào sau đây KHÔNG tuân theo quy tắc octet?
	\choice
	{$H_2O$}
	{$CO_2$}
	{\True $BH_3$}
	{$CH_4$}
	\loigiai{Trong $BH_3$, nguyên tử B chỉ có 6 electron ở lớp ngoài cùng, không đạt được octet.}
\end{ex}
%%%=============EX_25=============%%%
\begin{ex}
	Nguyên tử F (Z=9) khi nhận thêm 1 electron sẽ có cấu hình electron giống với:
	\choice
	{$O^{2-}$}
	{$Na^+$}
	{\True Ne}
	{He}
	\loigiai{Khi F nhận thêm 1 electron, nó sẽ có 10 electron, giống với cấu hình electron của khí hiếm Ne.}
\end{ex}
%%%=============EX_26=============%%%
\begin{ex}
	Để đạt được cấu hình bền vững của khí hiếm, nguyên tử Al (Z=13) có xu hướng:
	\choice
	{Nhận 3 electron}
	{\True Nhường 3 electron}
	{Góp chung 3 electron}
	{Không tham gia liên kết}
	\loigiai{Al thuộc nhóm IIIA, có 3 electron ở lớp ngoài cùng. Để đạt cấu hình bền vững, Al có xu hướng nhường 3 electron, tạo ra ion $Al^{3+}$.}
\end{ex}
%%%=============EX_27=============%%%
\begin{ex}
	Trong ion $F^-$, số electron ở lớp ngoài cùng là:
	\choice
	{7 electron}
	{\True 8 electron}
	{9 electron}
	{10 electron}
	\loigiai{Ion $F^-$ được tạo thành khi nguyên tử F nhận thêm 1 electron. Lúc này, $F^-$ có 8 electron ở lớp ngoài cùng.}
\end{ex}
%%%=============EX_28=============%%%
\begin{ex}
	Những nguyên tử nào sau đây có xu hướng nhường electron?
	\choice
	{\True Na, K, Ca}
	{F, Cl, Br}
	{N, P, As}
	{O, S, Se}
	\loigiai{%
		$Na, K, Ca $là các kim loại kiềm và kiềm thổ, có xu hướng nhường electron để đạt cấu hình khí hiếm.}
\end{ex}
%%%=============EX_29=============%%%
\begin{ex}
	Trong phân tử $O_2$, số cặp electron dùng chung là:
	\choice
	{1 cặp}
	{\True 2 cặp}
	{3 cặp}
	{4 cặp}
	\loigiai{Trong phân tử $O_2$, hai nguyên tử O góp chung 2 cặp electron để đạt cấu hình bền vững.}
\end{ex}
%%%=============EX_30=============%%%
\begin{ex}
	Khi tham gia liên kết hóa học, nguyên tử H có xu hướng:
	\choice
	{Nhường electron}
	{Nhận electron}
	{\True Góp chung electron}
	{Không tham gia liên kết}
	\loigiai{Nguyên tử H có 1 electron. Khi tham gia liên kết hóa học, nó thường góp chung electron này để đạt cấu hình bền vững của He.}
\end{ex}
%%%=============EX_31=============%%%
\begin{ex}
	Trong các phân tử sau, phân tử nào tuân theo quy tắc octet?
	\choice
	{$PCl_5$}
	{$SF_6$}
	{\True $H_2O$}
	{NO}
	\loigiai{Trong $H_2O$, cả O và H đều đạt được cấu hình bền vững theo quy tắc octet.}
\end{ex}
%%%=============EX_32=============%%%
\begin{ex}
	Ion $Na^+$ có cấu hình electron giống với khí hiếm nào?
	\choice
	{He}
	{\True Ne}
	{Ar}
	{Kr}
	\loigiai{Khi mất 1 electron, $Na^+$ có 10 electron, giống với cấu hình electron của Ne.}
\end{ex}
%%%=============EX_33=============%%%
\begin{ex}
	Nguyên tử của nguyên tố nhóm VIIA (17) có xu hướng:
	\choice
	{Nhường 7 electron}
	{\True Nhận 1 electron}
	{Góp chung 7 electron}
	{Không tham gia liên kết}
	\loigiai{Các nguyên tố nhóm VIIA có 7 electron ở lớp ngoài cùng, chúng có xu hướng nhận 1 electron để đạt cấu hình khí hiếm.}
\end{ex}
%%%=============EX_34=============%%%
\begin{ex}
	Trong phân tử $CH_4$, nguyên tử C:
	\choice
	{Nhường 4 electron}
	{Nhận 4 electron}
	{\True Góp chung 4 electron}
	{Không tham gia liên kết}
	\loigiai{Trong phân tử $CH_4$, nguyên tử C góp chung 4 electron với 4 nguyên tử H để tạo thành 4 liên kết C-H.}
\end{ex}
%%%=============EX_35=============%%%
\begin{ex}
	Quy tắc octet không áp dụng cho:
	\choice
	{Các nguyên tố nhóm A}
	{Các nguyên tố khí hiếm}
	{Các nguyên tố phi kim}
	{\True Các nguyên tố nhóm B}
	\loigiai{Quy tắc octet thường không áp dụng cho các nguyên tố nhóm B (nguyên tố chuyển tiếp) vì chúng có phân lớp d và f tham gia liên kết, có thể tạo ra nhiều loại liên kết phức tạp hơn.}
\end{ex}
%%%=============EX_36=============%%%
\begin{ex}
	Vì sao các nguyên tử lại liên kết với nhau thành phân tử?
	\choice
	{\True Để mối nguyên tử trong phân tử đạt được cơ cấu electron ổn định, bền vững}
	{Để mỗi nguyên tử trong phân tử đều đạt 8 electron ở lớp ngoài cùng}
	{Để tổng số electron ngoài cùng của các nguyên tử trong phân tử là 8 }
	{Để lớp ngoài cúng của mỗi nguyên tử trong phân tử có nhiều electron độc thân nhất}
	\loigiai{}
\end{ex}

%%%=============EX_37=============%%%
\begin{ex}
	Nguyên tử nào sau đây có khuynh hướng đạt cấu hình electron bền của khí hiếm neon khi tham gia hình thành liên kêt hoá học?
	\choice
	{Chlorine}
	{Sulfur}
	{\True Oxygen}
	{Hydrogen}
	\loigiai{}
\end{ex}

%%%=============EX_38=============%%%
\begin{ex}
	Sodium hydride $(\mathrm{NaH})$ là một hợp chất được sử dụng như một chất lưu trữ hydrogen trong các phương tiện chạy bằng pin nhiên liệu do khả năng giải phóng hydrogen của nó. Trong sodium hydride, nguyên tử sodium có cấu hình electron bền của khí hiếm
	\choice
	{helium}
	{argon}
	{krypton}
	{\True neon}
	\loigiai{}
\end{ex}

%%%=============EX_39=============%%%
\begin{ex}
	Khi tham gia hình thành liên kết hoá học, các nguyên tử lithium và chlorine có khuynh hướng đạt cấu hình electron bền của lần lượt các khí hiếm nào dưới đây?
	\choice
	{Helium và argon}
	{Helium và neon}
	{Neon và argon}
	{\True Argon và helium}
	\loigiai{}
\end{ex}

%%%=============EX_40=============%%%
\begin{ex}
	Trong phân tử HBr , nguyên tử hydrogen và bromine đã lần lượt đạt cấu hình electron bền của các khí hiếm nào dưới đây?
	\choice
	{Neon và argon}
	{Helium và xenon}
	{Helium và radon}
	{\True Helium và krypton}
	\loigiai{}
\end{ex}

%%%=============EX_41=============%%%
\begin{ex}
	Trong các hợp chất, nguyên tử magnesium đã đạt được cấu hình bền của khí hiếm gần nhất bằng cách
	\choice
	{\True cho đi 2 electron}
	{nhận vào 1 electron}
	{cho đi 3 electron}
	{nhận vào 2 electron}
	\loigiai{Trong quá trình hình thành phân tử magnesium oxide MgO , nguyên tử magnesium đã đạt được cấu hình bền của khí hiếm gần nhất bằng cách cho đi 2 electron.}
\end{ex}

%%%=============EX_42=============%%%
\begin{ex}
	Cho các phân tử sau: $\mathrm{Cl}_2, \mathrm{H}_2 \mathrm{O}, \mathrm{NaF}$ và $\mathrm{CH}_4$. Có bao nhiêu nguyên tử trong các phân tử trên đạt cấu hình electron bền của khí hiếm neon?
	\choice
	{3 }
	{2}
	{5 }
	{\True 4 }
	\loigiai{Có 4 nguyên tử trong các phân tử đã cho đạt cấu hình electron bền của khí hiếm neon là $\mathrm{O}, \mathrm{Na}, \mathrm{F}$ và C .}
\end{ex}

%%%=============EX_43=============%%%
\begin{ex}
	Nguyên tử trong phân tử nào dưới đây ngoại lệ với quy tắc octet?
	\choice
	{$\mathrm{H}_2 \mathrm{O}$}
	{$\mathrm{NH}_3$}
	{HCl }
	{\True $\mathrm{BF}_3$}
	\loigiai{Trong phân tử $\mathrm{BF}_3$, nguyên tử B mới chỉ có 6 electron ở lớp ngoài cùng, chưa đạt được cơ cấu bền của khí hiếm gần nhất.}
\end{ex}
%%%=============EX_44=============%%%
\begin{ex}
	Theo quy tắc octet, xu hướng chung của các nguyên tử nguyên tố nhóm IA là nhường
	\choice
	{2 electron}
	{3 electron}
	{\True 1 electron}
	{4 electron}
	\loigiai{}
\end{ex}

%%%=============EX_45=============%%%
\begin{ex}
	Để thỏa mãn quy tắc octet, nguyên tử chlorine ($Z=17$) có xu hướng
	\choice
	{nhường 1 electron}
	{\True nhận 1 electron}
	{nhường 3 electron}
	{nhận 3 electron}
	\loigiai{}
\end{ex}

%%%=============EX_46=============%%%
\begin{ex}
	Khi hình thành liên kết hóa học, nguyên tử có số hiệu nào sau đây có xu hướng nhường 2 electron để đạt
	cấu hình electron bền vững theo quy tắc octet?
	\choice
	{$Z=11$}
	{$Z=9$}
	{\True $Z=12$}
	{$Z=10$}
	\loigiai{}
\end{ex}

%%%=============EX_47=============%%%
\begin{ex}
	Theo quy tắc octet nguyên tử nào sau đây nhận 1 electron để đạt cấu trúc ion bền?
	\choice
	{X ($Z=8$)}
	{\True Y ($Z=9$)}
	{T ($Z=11$)}
	{Q ($Z=12$)}
	\loigiai{}
\end{ex}

%%%=============EX_48=============%%%
\begin{ex}
	Quy tắc octet \textbf{không đúng} với trường hợp phân tử chất nào sau đây?
	\choice
	{$H_2O$}
	{\True$NO_2$}
	{$CO_2$}
	{$Cl_2$}
	\loigiai{}
\end{ex}

%%%=============EX_49=============%%%
\begin{ex}
	Phân tử nào dưới đây các nguyên tử liên kết \textbf{không} tuân theo quy tắc octet?
	\choice
	{$H_2O$}
	{$NH_3$}
	{$CH_4$}
	{\True $NO$}
	\loigiai{}
\end{ex}

%%%=============EX_50=============%%%
\begin{ex}
	Quy tắc octet không đúng với phân tử nào sau đây?
	\choice
	{$H_2O$}
	{$NH_3$}
	{$CO_2$}
	{\True $PCl_5$}
	\loigiai{}
\end{ex}
\Closesolutionfile{ans}
\Closesolutionfile{ansex}
%\bangdapan{Ans-C03B01_QTOCTET_01.tex}
\phan{Trắc nghiệm đúng sai}
%%%=============SOẠN EXTF===============%%%
\Opensolutionfile{ansex}[Ans/LGTF-C03B01QTOT.tex]
\Opensolutionfile{ansbook}[Ansbook/AnsTF-C03B01QTOT.tex]
\Opensolutionfile{ans}[Ans/Tempt-C03B01QTOT.tex]
%%%=============EX_1=============%%%
\begin{ex}
	Xét các phát biểu về quy tắc octet:
	\choiceTF
	{\True Nguyên tử các nguyên tố có xu hướng nhận hoặc nhường electron để đạt cấu hình electron của khí hiếm gần nhất}
	{Quy tắc octet chỉ áp dụng cho các nguyên tố phi kim}
	{\True Các nguyên tử có thể đạt được cấu hình octet bằng cách chia sẻ các electron hóa trị}
	{\True Trong phân tử, các nguyên tử thường có xu hướng đạt được 8 electron ở lớp ngoài cùng}
	\loigiai{
		\begin{itemchoice}[T1,F2,T3,T4]
			\itemch Đây là nguyên lý cơ bản của quy tắc octet
			\itemch Quy tắc octet áp dụng cho cả kim loại và phi kim
			\itemch Việc chia sẻ electron hóa trị tạo thành liên kết cộng hóa trị
			\itemch Cấu hình electron bền vững thường có 8 electron lớp ngoài cùng
		\end{itemchoice}
	}
\end{ex}
%%%=============EX_2=============%%%
\begin{ex}
	Về quy tắc bát tử :
	\choiceTF
	{\True Các nguyên tử trong phân tử thường đạt 8 electron lớp ngoài cùng}
	{Tất cả các nguyên tố chu kỳ 2 đều tuân theo quy tắc bát tử}
	{\True Cấu hình electron của khí hiếm là cấu hình bền vững}
	{Nguyên tử Li có 8 electron ở lớp vỏ ngoài cùng}
	\loigiai{
		\begin{itemchoice}[T1,F2,T3,F4]
			\itemch Đây là quy luật phổ biến trong tự nhiên khi hình thành liên kết
			\itemch Be và B trong chu kỳ 2 là những ngoại lệ của quy tắc bát tử
			\itemch Khí hiếm có cấu hình electron đặc biệt bền vững
			\itemch Li có 1 electron lớp ngoài cùng, không phải 8 electron
		\end{itemchoice}
	}
\end{ex}
%%%=============EX_3=============%%%
\begin{ex}
	Về các trường hợp đặc biệt:
	\choiceTF
	{\True Hiđro chỉ cần 1 electron để đạt cấu hình bền của Heli}
	{\True Các nguyên tử có thể đạt được octet bằng cách nhận, nhường hoặc dùng chung electron}
	{Mọi nguyên tử đều phải đạt đủ 8 electron để tạo thành phân tử bền}
	{\True Một số nguyên tử có thể bền với ít hơn 8 electron ở lớp ngoài cùng}
	\loigiai{
		\begin{itemchoice}[T1,T2,F3,T4]
			\itemch H là trường hợp đặc biệt vì nó thuộc chu kỳ 1
			\itemch Có nhiều cách để nguyên tử đạt được cấu hình bền
			\itemch Có những trường hợp ngoại lệ như H (2e), B (6e), Be (4e)
			\itemch Be trong hợp chất của nó chỉ có 4 electron vẫn bền
		\end{itemchoice}
	}
\end{ex}
%%%=============EX_4=============%%%
\begin{ex}
	Về mối liên hệ với bảng tuần hoàn:
	\choiceTF
	{\True Số electron tối đa ở lớp ngoài cùng của các nguyên tử trong một chu kỳ luôn bằng số thứ tự của nhóm A}
	{Tất cả các nguyên tố họ p đều tuân theo quy tắc bát tử}
	{\True Các electron lớp ngoài cùng quyết định khả năng tham gia phản ứng của nguyên tử}
	{\True Các nguyên tố nhóm A có số electron hóa trị bằng số thứ tự nhóm}
	\loigiai{
		\begin{itemchoice}[T1,F2,T3,T4]
			\itemch Đây là quy luật quan trọng trong bảng tuần hoàn
			\itemch B thuộc họ p nhưng là ngoại lệ của quy tắc bát tử
			\itemch Electron lớp ngoài quyết định tính chất hóa học
			\itemch Số electron hóa trị tương ứng với số thứ tự nhóm A
		\end{itemchoice}
	}
\end{ex}
%%%=============EX_5=============%%%
\begin{ex}
	Về năng lượng và cấu hình electron
	\choiceTF
	{\True Cấu hình electron của khí hiếm có năng lượng thấp nhất trong cùng chu kỳ}
	{\True Sự bền vững của cấu hình bát tử liên quan đến năng lượng ion hóa cao}
	{Các nguyên tử luôn đạt được cấu hình bát tử bằng cách nhận thêm electron}
	{\True Độ bền của cấu hình bát tử liên quan đến sự đối xứng của các orbital}
	\loigiai{
		\begin{itemchoice}[T1,T2,F3,T4]
			\itemch Năng lượng thấp thể hiện tính bền vững cao
			\itemch Năng lượng ion hóa cao của khí hiếm chứng tỏ độ bền vững
			\itemch Nguyên tử có thể đạt bát tử bằng nhiều cách khác nhau
			\itemch Orbital đầy và đối xứng tạo nên độ bền cao của cấu hình
		\end{itemchoice}
	}
\end{ex}
%%%=============EX_6=============%%%
\begin{ex}
	Về quan hệ giữa cấu trúc electron và quy tắc bát tử
	\choiceTF
	{\True Orbital p chỉ chứa tối đa 6 electron nên cần thêm 2 electron từ orbital s để đạt cấu hình bát tử}
	{Mọi nguyên tử đều cần đủ 8 electron để tạo thành phân tử bền}
	{\True Cấu hình bát tử liên quan đến sự lấp đầy hoàn toàn các orbital s và p}
	{\True Độ bền của cấu hình bát tử liên quan đến sự đối xứng của các orbital s và p}
	\loigiai{
		\begin{itemchoice}[T1,F2,T3,T4]
			\itemch Cấu hình bát tử gồm 2 electron s và 6 electron p
			\itemch H, He và một số nguyên tố khác là ngoại lệ
			\itemch Orbital s và p đầy tạo nên cấu hình electron bền
			\itemch Sự đối xứng của orbital làm tăng độ bền của nguyên tử
		\end{itemchoice}
	}
\end{ex}
%%%=============EX_7=============%%%
\begin{ex}
	Cho các phát biểu sau về quy tắc octet:
	\choiceTF
	{\True Nguyên tử của hầu hết các nguyên tố nhóm A có xu hướng đạt cấu hình 8 electron lớp ngoài cùng khi hình thành liên kết hóa học.}
	{Nguyên tử của nguyên tố nhóm B luôn có xu hướng đạt cấu hình 8 electron lớp ngoài cùng.}
	{Quy tắc octet áp dụng cho tất cả các nguyên tố.}
	{Nguyên tử H có xu hướng đạt 8 electron lớp ngoài cùng khi tham gia liên kết hóa học.}
	\loigiai{
		\begin{itemchoice}[T1,F2,F3,F4]
			\itemch Nguyên tử của hầu hết các nguyên tố nhóm A có xu hướng đạt cấu hình 8 electron lớp ngoài cùng khi hình thành liên kết hóa học. Đây là nội dung của quy tắc octet.
			\itemch Nguyên tử của nguyên tố nhóm B không nhất định đạt cấu hình 8 electron lớp ngoài cùng.
			\itemch Quy tắc octet không áp dụng cho tất cả các nguyên tố, ví dụ như $H$, $Li$, $Be$, $B$,\ldots
			\itemch Nguyên tử H có xu hướng đạt 2 electron lớp ngoài cùng khi tham gia liên kết hóa học.
		\end{itemchoice}
	}
\end{ex}
%%%=============EX_8=============%%%
\begin{ex}
	Cho các nhận định sau về việc áp dụng quy tắc octet:
	\choiceTF
	{Quy tắc octet không phải lúc nào cũng đúng với mọi hợp chất.}
	{\True Có thể dựa vào quy tắc octet để dự đoán công thức của một số hợp chất.}
	{Quy tắc octet chỉ áp dụng cho hợp chất ion.}
	{\True Quy tắc octet có thể áp dụng cho cả hợp chất cộng hóa trị.}
	\loigiai{
		\begin{itemchoice}[T1,T2,F3,T4]
			\itemch Quy tắc octet không phải lúc nào cũng đúng với mọi hợp chất. Có những trường hợp ngoại lệ.
			\itemch Có thể dựa vào quy tắc octet để dự đoán công thức của một số hợp chất.
			\itemch Quy tắc octet không chỉ áp dụng cho hợp chất ion mà còn áp dụng cho hợp chất cộng hóa trị.
			\itemch Quy tắc octet có thể áp dụng cho cả hợp chất cộng hóa trị.
		\end{itemchoice}
	}
\end{ex}
%%%=============EX_9=============%%%
\begin{ex}
	Cho các phát biểu sau về phân tử $BF_3$:
	\choiceTF
	{\True $BF_3$ không tuân theo quy tắc octet.}
	{$B$ trong $BF_3$ đạt cấu hình bền của khí hiếm $He$.}
	{\True Mỗi nguyên tử $F$ trong $BF_3$ đạt cấu hình bền của $Ne$.}
	{\True $BF_3$ là phân tử bền vững.}
	\loigiai{
		\begin{itemchoice}[T1,F2,T3,T4]
			\itemch $BF_3$ không tuân theo quy tắc octet vì $B$ chỉ có 6 electron lớp ngoài cùng.
			\itemch $B$ trong $BF_3$ có 6 electron lớp ngoài cùng, không đạt cấu hình bền của khí hiếm $He$ (2 electron).
			\itemch Mỗi nguyên tử $F$ trong $BF_3$ đạt cấu hình bền của $Ne$ (8 electron lớp ngoài cùng).
			\itemch $BF_3$ là phân tử bền vững mặc dù không tuân theo quy tắc octet.
		\end{itemchoice}
	}
\end{ex}
%%%=============EX_10=============%%%
\begin{ex}
	Cho các phát biểu sau về phân tử $N_2$:
	\choiceTF
	{\True Mỗi nguyên tử $N$ có 8 electron lớp ngoài cùng.}
	{ $N_2$ có liên kết đôi.}
	{ $N_2$ không tuân theo quy tắc octet.}
	{\True Mỗi nguyên tử $N$ góp 3 electron tạo liên kết.}
	\loigiai{
		\begin{itemchoice}[T1,F2,F3,T4]
			\itemch Mỗi nguyên tử $N$ có 8 electron lớp ngoài cùng (đạt cấu hình bền vững).
			\itemch $N_2$ có liên kết ba.
			\itemch $N_2$ tuân theo quy tắc octet.
			\itemch Mỗi nguyên tử $N$ góp 3 electron tạo liên kết ba.
		\end{itemchoice}
	}
\end{ex}
%%%=============EX_11=============%%%
\begin{ex}
	Cho các phát biểu sau về ion $Mg^{2+}$:
	\choiceTF
	{\True Có cấu hình electron giống khí hiếm $Ne$.}
	{Có 12 electron.}
	{ $Mg$ có xu hướng nhận 2 electron để tạo thành $Mg^{2+}$.}
	{\True $Mg$ thuộc nhóm IIA.}
	\loigiai{
		\begin{itemchoice}[T1,F2,F3,T4]
			\itemch $Mg^{2+}$ có cấu hình electron giống khí hiếm $Ne$.
			\itemch $Mg^{2+}$ có 10 electron.
			\itemch $Mg$ có xu hướng nhường 2 electron để tạo thành $Mg^{2+}$.
			\itemch $Mg$ thuộc nhóm IIA.
		\end{itemchoice}
	}
\end{ex}
%%%=============EX_12=============%%%
\begin{ex}
	Cho các phát biểu sau về $CCl_4$:
	\choiceTF
	{\True $CCl_4$ tuân theo quy tắc octet.}
	{\True Mỗi nguyên tử $Cl$ đạt cấu hình 8e lớp ngoài cùng.}
	{\True $C$ trung tâm đạt cấu hình 8e lớp ngoài cùng.}
	{$CCl_4$ có 3 liên kết cộng hóa trị.}
	\loigiai{
		\begin{itemchoice}[T1,T2,T3,F4]
			\itemch $CCl_4$ tuân theo quy tắc octet.
			\itemch Mỗi nguyên tử $Cl$ đạt cấu hình 8e lớp ngoài cùng.
			\itemch $C$ trung tâm đạt cấu hình 8e lớp ngoài cùng.
			\itemch $CCl_4$ có 4 liên kết cộng hóa trị.
		\end{itemchoice}
	}
\end{ex}
%%%=============EX_13=============%%%
\begin{ex}
	Cho các phát biểu sau về $SF_6$:
	\choiceTF
	{$SF_6$ tuân theo quy tắc octet.}
	{$S$ có 8 electron lớp ngoài cùng.}
	{\True Mỗi $F$ có 8 electron lớp ngoài cùng.}
	{\True $SF_6$ có 6 liên kết cộng hóa trị.}
	\loigiai{
		\begin{itemchoice}[F1,F2,T3,T4]
			\itemch $SF_6$ không tuân theo quy tắc octet, $S$ có 12 electron lớp ngoài cùng.
			\itemch $S$ có 12 electron lớp ngoài cùng.
			\itemch Mỗi $F$ có 8 electron lớp ngoài cùng.
			\itemch $SF_6$ có 6 liên kết cộng hóa trị.
		\end{itemchoice}
	}
\end{ex}
%%%=============EX_14=============%%%
\begin{ex}
	Cho các nhận định liên quan đến quy tắc octet
	\choiceTF
	{\True Quy tắc Octet nói rằng các nguyên tử có xu hướng nhận, nhường hoặc góp chung electron để đạt được 8 electron ở lớp ngoài cùng. (trừ Helium)}
	{Quy tắc Octet áp dụng cho tất cả các nguyên tố trong bảng tuần hoàn}
	{Các nguyên tố nhóm A luôn tuân theo quy tắc octet khi tạo liên kết hóa học}
	{\True Quy tắc Octet không thể giải thích được cấu hình của các phân tử thuộc nhóm B}
	{}
	\loigiai{}
\end{ex}

%%%=============EX_15=============%%%
\begin{ex}
	Về ion
	\choiceTF
	{\True Khi sodium (Na) mất 1 electron, nó trở thành ion $Na^+$}
	{\True Khi fluorine nhận thêm 1 electron, nó trở thành ion $F^-$}
	{Nguyên tử helium (He) tuân theo quy tắc Octet}
	{\True Nguyên tử nitrogen trong phân tử $N_2$ tuân theo quy tắc Octet}
	\loigiai{}
\end{ex}

%%%=============EX_16=============%%%
\begin{ex}
	Phân tích đặc điểm các nguyên tử, phân tử \textbf{không} theo quy tắc octet
	\choiceTF
	{Nguyên tử fluorine có cấu hình electron bền vững sau khi nhận thêm 1 electron}
	{\True Phân tử $SF_6$ tuân theo quy tắc Octet}
	{Phân tử nitrogen ($N_2$) được tạo thành bởi 3 cặp electron chung}
	{Nguyên tử oxygen trong phân tử $O_2$ có 2 cặp electron chưa liên kết}
	\loigiai{}
\end{ex}

%%%=============EX_17=============%%%
\begin{ex}
	Khi hình thành liên kết hoá học trong phân tử $CCl_4$:
	\choiceTF
	{\True Mỗi nguyên tử chlorine đều có 7 electron ở lớp ngoài cùng}
	{\True Mỗi nguyên tử chlorine cần góp chung thêm 1 electron để đạt cấu hình bền vững}
	{\True Nguyên tử carbon có 4 electron hóa trị, nên nguyên tử carbon sẽ góp chung với mỗi nguyên tử chlorine 1 electron}
	{Nguyên tử carbon và chlorine sau khi góp chung electron đều sẽ đạt cấu hình bền vững của khí hiếm neon}
	\loigiai{}
\end{ex}
\Closesolutionfile{ans}
\Closesolutionfile{ansbook}
\Closesolutionfile{ansex}
%\bangdapanTF{AnsTF-C03B01QTOT.tex}
\phan{Bài tập tự luận}
%%%=============SOẠN BT===============%%%
\Opensolutionfile{ansbth}[Ans/LGBT-C03B01_QTOCTET_01.tex]
\Opensolutionfile{ansbt}[Ans/AnsBT-C03B01_QTOCTET_01.tex]
	%%%=============BT_1=============%%%
	\begin{bt}
		Hãy ghép mỗi nguyên tử ở cột A với nội dung được mô tả ở cột B cho phù hợp.
		\\
		\begin{tabular}{L{0.35\linewidth}L{0.65\linewidth}}
			\textbf{Cột A}
			\begin{enumerate}[a)]
				\item $\mathrm{Ne}(\mathrm{Z}=10)$
				\item $\mathrm{F}(\mathrm{Z}=9)$
				\item $\mathrm{Mg}(\mathrm{Z}=12)$
				\item $\mathrm{He}(\mathrm{Z}=2)$
			\end{enumerate}
			&
			\textbf{Cột B}
			\begin{enumerate}[1.]
				\item có xu hướng nhận thêm 1 electron.
				\item có cấu hình lớp vỏ ngoài cùng 8 electron bền vững.
				\item có $x u$ hướng nhường đi 2 electron.
				\item có cấu hình lớp vỏ ngoài cùng 2 electron bền vững.
			\end{enumerate}\\
		\end{tabular}
		\loigiai{\begin{tabular}{cccc}
				a) -- 2.& b) -- 1.&c) -- 3.&d) -- 4.
		\end{tabular}}
	\end{bt}
	%%%=============BT_2=============%%%
	\begin{bt}
		Em hãy vẽ mô hình mô tả quá trình tạo lớp vỏ thoả mãn quy tắc octet trong các trường hợp sau đây:
		\begin{enumerate}[a)]
			\item Nguyên tử $\mathrm{O}(\mathrm{Z}=8)$ nhận 2 electron để tạo anion $\mathrm{O}^{2-}$.
			\item Nguyên tử $\mathrm{Ca}(\mathrm{Z}=20)$ nhường 2 electron để tạo cation $\mathrm{Ca}^{2+}$.
			\item Hai nguyên tử fluorine "góp chung electron" để đạt được lớp vỏ thoả mãn quy tắc octet.
		\end{enumerate}
		\loigiai{
		\begin{enumerate}[a)]
			\item Nguyên tử $\mathrm{O}(\mathrm{Z}=8)$ nhận 2 electron để tạo anion $\mathrm{O}^{2-}$.
			\begin{center}
				\includegraphics[height=3.5cm]{Images/Tikz/xuhuongnhanelectron-Oxigen.pdf}
			\end{center}
			\item Nguyên tử $\mathrm{Ca}(\mathrm{Z}=20)$ nhường 2 electron để tạo cation $\mathrm{Ca}^{2+}$.
			\begin{center}
				\includegraphics[height=4.5cm]{Images/Tikz/xuhuongnhuongelectron-Calcium.pdf}
			\end{center}
			\item Hai nguyên tử fluorine "góp chung electron" để đạt được lớp vỏ thoả mãn quy tắc octet.
			\begin{center}
				\includegraphics[height=3.5cm]{Images/Tikz/xuhuonggopchungelectron-Flourine.pdf}
			\end{center}
		\end{enumerate}
		}
	\end{bt}
	%%%=============BT_3=============%%%
	\begin{bt}
		Trong tự nhiên, các khí hiếm tồn tại dưới dạng nguyên tử tự do. Các nguyên tử của khí hiếm không liên kết với nhau tạo thành phân tử và rất khó liên kết với các nguyên tử của các nguyên tố khác. Ngược lại nguyên tử các nguyên tố khác lại liên kết với nhau tạo thành phân tử hay tinh thể. Giải thích
		\loigiai{\begin{itemize}
				\item Nguyên tử khí hiếm đều có cấu hình electron bão hoà là $n s^2 n p^6$ (trừ helium có cấu hình $1 \mathrm{~s}^2$ ) làm cho nguyên tử khí hiếm rất bền vững nên các nguyên tử khí hiếm rất khó tham gia phản ứng hoá học. Trong tự nhiên, các khí hiếm đều tồn tại ở trạng thái nguyên tử (hay còn gọi là phân tử một nguyên tử) tự do, bền vững (nên còn gọi là các khí trơ).
				\item Nguyên tử của các nguyên tố khác có xu hướng liên kết với nhau để đạt được cấu hình electron bền vững của khí hiếm, ví dự: $\mathrm{H}_2, \mathrm{Cl}_2, \mathrm{HCl}, \mathrm{CO}_2, \ldots$ hay tự tập hợp lại thành các khối tinh thể, ví dụ: tỉnh thể $\mathrm{NaCl}_2, \ldots$
		\end{itemize}}
	\end{bt}
	%%%=============BT_4=============%%%
	\begin{bt}
		Cấu hình electron lớp ngoài cùng của nguyên tử potassium (kali) là $4\mathrm{s}^1$, cấu hình electron lớp ngoài cùng của nguyên tử bromine là $4\mathrm{s}^24p^5$. Làm thế nào các nguyên tử potassium và bromine có được cấu hình electron của nguyên tử khí hiếm theo quy tắc octet
		\loigiai{\begin{itemize}
				\item Nguyên tử potassium chỉ có 1 electron ở lớp ngoài cùng nên dễ dàng nhường đi electron này để tạo thành ion dương. Ion dương $\left(\mathrm{K}^{+}\right)$có cấu hình electron lớp ngoài cùng giống với khi hiếm argon $\left(3 \mathrm{~s}^2 3 \mathrm{p}^6\right)$ đứng trước potassium trong bảng tuần hoàn.
				\item Nguyên tử bromine có 7 electron ở lớp electron ngoài cùng nên dễ dàng nhận thêm 1 electron tạo ra anion bromide $\left(\mathrm{Br}^{-}\right)$có cấu hình electron lớp ngoài cùng giống với khí hiếm krypton $\left(4 \mathrm{~s}^2 4 \mathrm{p}^6\right)$, đứng sau bromine trong bảng tuần hoàn.
		\end{itemize}}
	\end{bt}
	%%%=============BT_5=============%%%
	\begin{bt}
		Khi hình thành liên kết $H+\mathrm{Cl} \rightarrow \mathrm{HCl}$ và khi phá vỡ liên kết $\mathrm{HCl} \rightarrow H+\mathrm{Cl}$ thì hệ thu năng lượng hay toả năng lượng. Năng lượng phân tử HCl lớn hơn hay nhỏ hơn năng lượng hệ hai nguyên tử H và Cl riêng rẽ? Trong hai hệ đó thì hệ nào bền hơn?
		\loigiai{%
			\begin{itemize}
				\item  Khi hình thành liên kết $\mathrm{H}+\mathrm{Cl} \rightarrow \mathrm{H}-\mathrm{Cl}$ thì hệ toả ra năng lượng và ngược lại khi phá vỡ liên kết $\mathrm{H}-\mathrm{Cl} \rightarrow \mathrm{H}+\mathrm{Cl}$ thì hệ thu thêm năng lượng.
				\item  Xét về mặt năng lượng thì phân tử $\mathrm{H}-\mathrm{Cl}$ có năng lượng nhỏ hơn hệ hai nguyên tử H và Cl riêng rẽ. Trong hai hệ đó thì hệ $\mathrm{H}-\mathrm{Cl}$ bền hơn hệ H và Cl .
			\end{itemize}
		}
	\end{bt}
	%%%=============BT_6=============%%%
	\begin{bt}
		Trong phân tử $\mathrm{Na}_2\mathrm{S}$, cấu hình electron của các nguyên tử có tuân theo quy tắc octet không?
		\loigiai{%
			Cấu hình electron của Na:
			$\underset{1s^2}{\squarerow[2ud][0.5][\maunhan]{1}}$ $\underset{2s^2}{\squarerow[2ud][0.5][\maunhan]{1}}$
			$\underset{2p^6}{\squarerow[2ud,2ud,2ud][0.5][\maunhan]{3}}$
			$\underset{3s^1}{\squarerow[1u][0.5][\maunhan]{1}}$
			\\
			Cấu hình electron của S:
			$\underset{1s^2}{\squarerow[2ud][0.5][\maunhan]{1}}$ $\underset{2s^2}{\squarerow[2ud][0.5][\maunhan]{1}}$
			$\underset{2p^6}{\squarerow[2ud,2ud,2ud][0.5][\maunhan]{3}}$
			$\underset{3s^2}{\squarerow[2ud][0.5][\maunhan]{1}}$
			$\underset{3p^6}{\squarerow[2ud,2ud,2ud][0.5][\maunhan]{3}}$
			$\underset{4s^2}{\squarerow[2ud][0.5][\maunhan]{1}}$
			\begin{itemize}
				\item Khi Na kết hợp với S , mỗi nguyên tử Na nhường đi 1 electron hoá trị duy nhất để tạo thành cation $\mathrm{Na}^{+}$có 8 electron ở vỏ nguyên tử giống với khí hiếm neon. Nguyên tử S có 6 electron hoá trị nhận thêm 2 electron từ hai nguyên tử Na tạo thành ion sulfide $\mathrm{S}^{2-}$ có 8 electron ở vỏ nguyên tử giống với khí hiếm argon.
				\item Hai nguyên tử Na và S đều đạt cấu hình electron bão hoà theo quy tắc octet trong phân tử sodium sulfide $\mathrm{Na}_2 \mathrm{~S}$.
			\end{itemize}
		}
	\end{bt}
	%%%=============BT_7=============%%%
	\begin{bt}
		Vận dụng quy tắc octet để giải thích sự hình thành liên kết trong các phân tử: $O_2, CO_2, \mathrm{CaCl}_2, \mathrm{KBr}$
		\loigiai{%
			\begin{enumerate}
				\item Phân tử $O_2$:\\
				\schemestart
					\chemfig{\charge{[.radius=0.2ex]0:2pt=\:,90:2pt=\:,180:2pt=\:}{O}} 
					\+ 
					\chemfig{\charge{[.radius=0.2ex]0:2pt=\:,90:2pt=\:,180:2pt=\:}{O}}%
					\arrow{->}[,,,-stealth]
					\chemfig{%
						\charge{[.radius=0.2ex]0:2pt=\:,90:2pt=\:,180:2pt=\:}{O}
						-[,0.6,,,draw=none]
						\charge{[.radius=0.2ex]0:2pt=\:,90:2pt=\:,180:2pt=\:}{O}
					}%
				\schemestop \quad hay \quad \chemfig{%
					\charge{[.radius=0.2ex]90:2pt=\:,180:2pt=\:}{O}
					=[,0.6]
					\charge{[.radius=0.2ex]0:2pt=\:,90:2pt=\:}{O}
				}%
				\item Phân tử $CO_2$:\\
				\schemestart
					\chemfig{\charge{[.radius=0.2ex]0:2pt=\:,90:2pt=\:,180:2pt=\:}{O}} 
					\+ 
					\chemfig{\charge{[.radius=0.2ex]180:2pt=\:,0:2pt=\:}{C}} 
					\+
					\chemfig{\charge{[.radius=0.2ex]0:2pt=\:,90:2pt=\:,180:2pt=\:}{O}}%
					\arrow{->}[,,,-stealth]
					\chemfig{%
						\charge{[.radius=0.2ex]0:2pt=\:,90:2pt=\:,180:2pt=\:}{O}
						-[,0.6,,,draw=none]
						\charge{[.radius=0.2ex]0:2pt=\:,180:2pt=\:}{C}
						-[,0.6,,,draw=none]
						\charge{[.radius=0.2ex]0:2pt=\:,90:2pt=\:,180:2pt=\:}{O}
					}%
				\schemestop\quad hay \quad \chemfig{%
					\charge{[.radius=0.2ex]90:2pt=\:,180:2pt=\:}{O}
					=[,0.6]C=[,0.6]\charge{[.radius=0.2ex]0:2pt=\:,90:2pt=\:}{O}
				}%
				\item Phân tử $\mathrm{CaCl}_2$:\\
				\schemestart
				\chemfig{\charge{[.radius=0.2ex]0:2pt=\.,90:2pt=\:,180:2pt=\:,-90:2pt=\:}{Cl}} 
				\+ 
				\chemfig{\charge{[.radius=0.2ex]0:2pt=\.,180:2pt=\.}{Ca}}
				\+
				\chemfig{\charge{[.radius=0.2ex]0:2pt=\:,90:2pt=\:,180:2pt=\.,-90:2pt=\:}{Cl}}
				\arrow{->}[,,,-stealth]
				\khungion[-]{\chemfig{\charge{[.radius=0.2ex]0:2pt=\:,90:2pt=\:,180:2pt=\:,-90:2pt=\:}{Cl}}} 
				\+ 
				\khungion[2+]{\chemfig{Ca}}
				\+
				\khungion[-]{\chemfig{\charge{[.radius=0.2ex]0:2pt=\:,90:2pt=\:,180:2pt=\:,-90:2pt=\:}{Cl}}}
				\schemestop
				\item Phân tử $\mathrm{KBr}$:\\
				\schemestart
				\chemfig{\charge{[.radius=0.2ex]0:2pt=\.}{K}} 
				\+ 
				\chemfig{\charge{[.radius=0.2ex]180:2pt=\.,0:2pt=\:,90:2pt=\:,-90:2pt=\:}{Br}}
				\arrow{->}[,,,-stealth]
				\khungion{\chemfig{K}}\+\khungion[-]{\chemfig{\charge{[.radius=0.2ex]180:2pt=\:,0:2pt=\:,90:2pt=\:,-90:2pt=\:}{Br}}}
				\schemestop
			\end{enumerate}
		}
	\end{bt}

	%%%=============BT_8=============%%%
	\begin{bt}
		Đá vôi (thành phần chính là $\mathrm{CaCO}_3$) được dùng để sản xuất vôi, trong lĩnh vực xây dựng, $\ldots$ Barium nitrate $\mathrm{Ba}\left(NO_3\right)_2$ có trong thành phần của kính quang học, gốm, men,\ldots Phèn đơn aluminium sulfate (thành phần chính là $\mathrm{Al}_2\left(SO_4\right)_3$) được sử dụng rộng rãi trong xử lí nước thải, trong công nghệ sản xuất giấy, công nghệ nhuộm vải và công nghệ lọc nước và nuôi trồng thuỷ sản,\ldots Dựa vào quy tắc octet, đề xuất công thức cấu tạo của các chất trên
		\loigiai{%
			\begin{enumerate}
				\item $CaCO_3$:
				\chemfig{Ca?[a]-[:45]O-[:-45]C(-[:-135]O?[a])=O}
				\item $Ba(NO_3)_2$
				\chemfig{O=[:-45]N(-[:-135,,,,-stealth]O)-O-Ba-O-N(-[:-45,,,,-stealth]O)=[:45]O}
				\item $Al_2(SO_4)_3$
				\chemfig{S(-[:-135,,,,-stealth]O)(-[:135,,,,-stealth]O)(-[:-45]O?[a])-[:45]O-[:-45]Al?[a]-O-[:-45]S(-[:-135,,,,-stealth]O)(-[:-45,,,,-stealth]O)-[:45]O-Al?[b]-[:45]O-[:-45]S(-[:-135]O?[b])(-[:-45,,,,-stealth]O)-[:45,,,,-stealth]O}
			\end{enumerate}
		}
	\end{bt}
	%%%=============BT_9=============%%%
	\begin{bt}
		Hợp chất X tạo bởi hai nguyên tố $A$, $D$ có khối lượng phân tử là 76. X là dung môi không phân cực, thường được sử dụng làm nguyên liệu trong tồng hợp chất hữu cơ chứa lưu huỳnh và được sử dụng rộng rãi trong sản xuất vải viscoza mềm. A có công thức hydride dạng $AH_4$ và D có công thức oxide ứng với hoá trị cao nhất dạng $DO_3$.
		\begin{enumerate}
			\item  Hãy thiết lập công thức phân tử của X. Biết rằng A có số oxi hoá cao nhất trong X.
			\item  Đề xuất công thức cấu tạo của X và cho biết các nguyên tử thành phần của X khi liên kết có đủ electron theo quy tắc octet không?
		\end{enumerate}
		\loigiai{
			\begin{enumerate}
				\item A thuộc nhóm IVA và D thuộc nhóm VIA $\Rightarrow$ số oxi hoá cao nhất của A trong X là +4 còn số oxi hoá của D trong X là -2 .
				Công thức phân tử X có dạng $\mathrm{AD}_2$. Ta có: $\mathrm{A}+2 \mathrm{D}=76$.
				\\
				$\Rightarrow$ Nguyên tử khối trung bình của $\mathrm{A}, \mathrm{D}$ là: $\dfrac{76}{3}=25,33$.
				\\
				$\Rightarrow A$ và $D$ thuộc chu kì $2,3 \Rightarrow$ Có các cặp nguyên tố sau:
				$\mathrm{C}=12$ và $\mathrm{O}=16 ; \mathrm{C}=12$ và $\mathrm{S}=32 ; \mathrm{Si}=28$ và $\mathrm{O}=16 ; \mathrm{Si}=28$ và $\mathrm{S}=32$.
				$\mathrm{C}=12$ và $\mathrm{S}=32$ thoả mãn $\mathrm{A}+2 \mathrm{D}=76$
				\\
				$\Rightarrow$ Công thức $\mathrm{X}: \mathrm{CS}_2$.
				\item Đề xuất công thức cấu tạo:\,\, \chemfig{\charge{120:1pt=\:,-120:1pt=\:}{S}=C=\charge{60:1pt=\:,-60:1pt=\:}{S}}\,\, $\mathrm{CS}_2$ có cấu trúc thẳng giống $\mathrm{CO}_2$. Các nguyên tử C và S đều có 8 electron lớp ngoài cùng theo quy tắc octet.
			\end{enumerate}
		}
	\end{bt}
	%%%=============BT_10=============%%%
	\begin{bt}
		Em hãy nêu tên và công thức hoá học của 1 chất ở thể rắn, 1 chất ở thể lỏng và 1 chất ở thể khí (trong điều kiện thường), trong đó nguyên tử oxygen đạt được cấu hình bền của khí hiếm neon.
		\loigiai{
			Để đạt cấu hình electron bền của Ne (1s$^2$2s$^2$2p$^6$), nguyên tử O cần nhận thêm 2 electron để có 8 electron lớp ngoài cùng.
			\begin{enumerate}
				\item Chất rắn: Natri oxit ($\mathrm{Na}_2\mathrm{O}$)
				\begin{itemize}
					\item O nhận 2 electron từ 2 nguyên tử Na để tạo thành ion O$^{2-}$
					\item Mỗi nguyên tử Na cho đi 1 electron để tạo thành ion Na$^+$
					\item Liên kết ion hình thành giữa các ion Na$^+$ và O$^{2-}$
				\end{itemize}
				\item Chất lỏng: Nước ($\mathrm{H}_2\mathrm{O}$)
				\begin{itemize}
					\item O chia sẻ electron với 2 nguyên tử H
					\item Mỗi liên kết O-H là liên kết đơn (1 cặp electron được chia sẻ)
					\item O đạt cấu hình octet nhờ 2 cặp electron liên kết và 2 cặp electron độc thân
				\end{itemize}
				\item Chất khí: Carbon dioxide ($\mathrm{CO}_2$)
				\begin{itemize}
					\item Mỗi nguyên tử O tạo liên kết đôi với nguyên tử C trung tâm
					\item Mỗi O chia sẻ 2 cặp electron với C để đạt cấu hình octet
					\item Phân tử có cấu trúc thẳng O=C=O
				\end{itemize}
			\end{enumerate}
		}
	\end{bt}
	%%%=============BT_11=============%%%
	\begin{bt}
		Potassium iodide $(KI)$ được  sử dụng như một loại thuốc long đờm, giúp làm lỏng và phá vỡ chất nhầy trong đường thở, thường dùng cho các bệnh nhân hen suyễn, viêm phế quản mãn tính. Trong trường hợp bị nhiễm phóng xạ, KI còn giúp ngăn tuyến giáp hấp thụ iodine phóng xạ, bảo vệ và giảm nguy cơ ung thư tuyến giáp. Trong phân tử KI, các nguyên tử K và I đều đã đạt được cơ cấu bền của khí hiếm gần nhất. Đó lần lượt là những khí hiếm nào?
		\loigiai{%
			\begin{enumerate}
				\item Xác định cấu hình electron của K và I:
				\begin{itemize}
					\item K (Z = 19): 1s$^2$2s$^2$2p$^6$3s$^2$3p$^6$4s$^1$
					\item I (Z = 53): 1s$^2$2s$^2$2p$^6$3s$^2$3p$^6$3d$^{10}$4s$^2$4p$^6$4d$^{10}$5s$^2$5p$^5$
				\end{itemize}
				\item Trong KI:
				\begin{itemize}
					\item K cho đi 1 electron ($4s^1$) để tạo thành ion K$^+$
					\item I nhận thêm 1 electron để tạo thành ion I$^-$
				\end{itemize}
				\item Cấu hình electron của các ion:
				\begin{itemize}
					\item K$^+$: 1s$^2$2s$^2$2p$^6$3s$^2$3p$^6$ (giống Ar)
					\item I$^-$: 1s$^2$2s$^2$2p$^6$3s$^2$3p$^6$3d$^{10}$4s$^2$4p$^6$4d$^{10}$5s$^2$5p$^6$ (giống Xe)
				\end{itemize}
				Vậy:
				\begin{itemize}
					\item K đạt cấu hình electron của khí hiếm Argon (Ar)
					\item I đạt cấu hình electron của khí hiếm Xenon (Xe)
				\end{itemize}
			\end{enumerate}
		}
	\end{bt}
	%%%=============BT_12=============%%%
	\begin{bt}
		Em hãy nêu tên và công thức hoá học của 1 chất ở thể rắn, 1 chất ở thể lỏng và 1 chất ở thể khí (trong điều kiện thường), trong đó nguyên tử oxygen đạt được cấu hình bền của khí hiếm neon.
		\loigiai{Nguyên tử oxygen đạt được cấu hình bền của khí hiếm neon trong MgO (chất rắn), $\mathrm{H}_2\mathrm{O}$ (chất lỏng) và $\mathrm{O}_2$ (chất khí).}
	\end{bt}
	
	%%%=============BT_13=============%%%
	\begin{bt}
		Potassium iodide $(KI)$ được sử dụng như một loại thuốc long đờm, giúp làm Iỏng và phá vỡ chất nhầy trong đường thở, thường dùng cho các bệnh nhân hen suyễn, viêm phế quản mãn tính. Trong trường hợp bị nhiễm phóng xạ, KI còn giúp ngăn tuyến giáp hấp thụ iodine phóng xạ, bảo vệ và giảm nguy cơ ung thư tuyến giáp. Trong phân tử KI, các nguyên tử K và I đều đã đạt được cơ cấu bền của khí hiếm gần nhất. Đó lần lượt là những khí hiếm nào?
		\loigiai{Trong phân tử potassium iodide (KI), nguyên tử K và I Iần lượt đạt được cơ cấu bền của khí hiếm gần nhất là argon (Ar) và xenon (Xe).}
	\end{bt}
	%%%=============BT_14=============%%%
	\begin{bt}
		Biểu diễn công thức electron, công thức Lewis và CTCT của các phân tử sau:
		$\text{H}_2\text{O}$ ; $\text{NH}_3$ ; $\text{CH}_4$ ; $\text{CO}_2$ ; $\text{CCl}_4$ ; $\text{H}_2\text{S}$ ; $\text{CS}_2$ ; $\text{N}_2$ ; $\text{O}_2$ ; $\text{HCl}$ ; $\text{BF}_3$ ; $\text{PCl}_5$ ; $\text{SF}_6$ ; $\text{BCl}_3$ ; $\text{AlCl}_3$ ; $\text{PF}_5$ ; $\text{HF}$ ; $\text{H}_2\text{CO}$ ; $\text{HNO}_3$; $\text{SO}_2$; $\text{CO}$; $\text{NO}_2$; $\text{NO}$;$\text{CH}_4$; $\text{C}_2\text{H}_4$; $\text{C}_2\text{H}_2$,
		\loigiai{}
	\end{bt}
\Closesolutionfile{ansbt}
\Closesolutionfile{ansbth}
%\bangdapanSA{AnsBT-C03B01_QTOCTET_01.tex}

	\section{Liên kết ion}
\begin{Muctieu}
	\begin{itemize}
		\item Trình bày được khái niệm và sự hình thành liên kết ion (nêu một số ví dụ điển hình tuân theo quy tắc octet).
		\item Nêu được cấu tạo tinh thể NaCl . Giải thích được vì sao các hợp chất ion thường ở trạng thái rắn trong điều kiện thường (dạng tinh thể ion).
		\item Lắp được mô hình tinh thể NaCl (theo mô hình có sẵn).
	\end{itemize}
\end{Muctieu}
\begin{kd}
	Hợp chất NaCl nóng chảy ở nhiệt độ cao và có khả năng dẫn điện khi nóng chảy hoặc khi hoà tan trong dung dịch. Yếu tố nào trong phân tử NaCl gây ra các tính chất trên?
\end{kd}
\subsection{Nội dung bài học}
\subsubsection{Sự tạo thành ion}
	\begin{figure}[thb]
		\begin{center}
			\subcaptionbox[0.4\linewidth]{Sự tạo thành ion $Na^+$\label{subfig:ionNa}}{\includegraphics[height=3cm]{Images/Tikz/suhinhthanhionNa.pdf}}
			\hspace{2cm}
			\subcaptionbox[0.4\linewidth]{Sự tạo thành ion $O^{2-}$\label{subfig:ionO}}{\includegraphics[height=3cm]{Images/Tikz/suhinhthanhionO.pdf}}
		\end{center}
		\caption{Minh họa quá trình hình thành ion\label{fig:taothanhion}}
	\end{figure}
	\begin{tomtat}
		\begin{itemize}
			\item Khi \indam[\maunhan]{cho electron}, nguyên tử trở thành \indam[\maunhan]{ion dương} (cation).
			\item Khi \indam[\maunhan]{nhận electron}, nguyên tử trở thành \indam[\maunhan]{ion âm} (anion).
			\item Giá trị điện tích trên cation hoặc anion bằng số electron mà nguyên tử đã nhường hoặc nhận.
		\end{itemize}
	\end{tomtat}
	\begin{hoivadap}
		\begin{cauhoi}
			Quan sát Hình \ref{fig:taothanhion}, nhận xét số electron trên lớp vỏ với số proton trong hạt nhân của mỗi ion tạo thành.
		\end{cauhoi}
		\begin{cauhoi}
			Trình bày cách tính điện tích của các ion thu được khi nguyên tử nhường hoặc nhận thêm electron trong Hình \ref{fig:taothanhion}.
		\end{cauhoi}
		\begin{cauhoi}
			Ion $\mathrm{Na}^{+}$và ion $\mathrm{O}^{2-}$ thu được có bền vững vế mặt hoá học không? Chúng có cấu hình electron nguyên tử của nguyên tố nào?
		\end{cauhoi}
		\loigiai{}
	\end{hoivadap}
\subsubsection{Sự tạo thành liên kết ion}
	\begin{figure}[thb]
		\begin{center}
			\includegraphics[height=6cm]{Images/Tikz/suhinhthanhlienketion.pdf}
		\end{center}
		\caption{Minh họa quá trình hình thành liên kết ion trong phân tử NaCl\label{fig:lkionNaCl}}
	\end{figure}
\begin{tomtat}
	\begin{itemize}
		\item Liên kết ion là liên kết được hình thành bởi lực hút tĩnh điện giửa các ion mang điện tích trái dấu.
		\item Liên kết ion thường được hình thành khi kim loại điển hình tác dụng với phi kim điển hình.
	\end{itemize}
\end{tomtat}
	\begin{hoivadap}
		\begin{cauhoi}
			Cho các ion: $\mathrm{Na}^{+}, \mathrm{Mg}^{2+}, \mathrm{O}^{2-}, \mathrm{Cl}^{-}$. Những ion nào có thể kết hợp với nhau tạo thành liên kết ion?
		\end{cauhoi}
		\begin{cauhoi}
			Mô tả sự tạo thành liên kết ion trong:
				\begin{enumerate}[a)]
				\item Calcium oxide.
				\item Magnesium chloride.
				\end{enumerate}
		\end{cauhoi}
		\loigiai{}
	\end{hoivadap}
\subsubsection{Tinh thể ion}
	\Noibat[\maunhan][][\faStar][]{Cấu trúc tinh thể ion}
		\begin{hopdongian}
			Các ion được sắp xếp theo một trật tự nhất định trong không gian theo kiểu mạng lưới, trong đó ở các nút của mạng lưới là những ion dương và ion âm được sắp xếp luân phiên, liên kết chặt chẽ với nhau do sự cân bằng giữa lực hút (các ion trái dấu hút nhau) và lực đẩy (các ion cùng dấu đẩy nhau), tạo thành mạng tinh thể ion.
		\end{hopdongian}
		\begin{figure}[thb]
			\begin{center}
				\subcaptionbox[0.4\linewidth]{Tinh thể NaCl thực tế\label{subfig:img_crytalNaCl}}{\includegraphics[height=5cm]{Images/anhhoahoc10/anhminhoa/Sodium_chloride_crystals.jpg}}
				\hspace{2cm}
				\subcaptionbox[0.4\linewidth]{Ô mạng tinh thể NaCl \label{subfig:crytalNaCl}}{\includegraphics[height=5cm]{Images/Tikz/crytalNaCl.pdf}}
			\end{center}
			\caption{Tinh thể NaCl thực tế và mô hình ô mạng tinh thể NaCl}
		\end{figure}
	\Noibat[\maunhan][][\faStar][]{Độ bền và tính chất hợp chất ion}
	\vspace{0.25cm}
	\begin{tomtat}
		\begin{itemize}
			\item Trong tinh thể ion, giữa các ion có lực hút tĩnh điện rất mạnh nên các hợp chất ion thường là chất rắn, khó nóng chảy, khó bay hơi ở điều kiện thường.
			\item Do lực hút tĩnh điện rất mạnh giữa các ion nên các tinh thể ion khá rắn chắc, nhưng khá giòn.
		\end{itemize}
	\end{tomtat}
	\begin{hoivadap}
		\begin{cauhoi}
			Hãy trả lời các câu hỏi sau:
			\begin{enumerate}[a)]
				\item Vì sao muối ăn có nhiệt độ nóng chảy cao $\left(801^{\circ} \mathrm{C}\right)$ ?
				\item Hợp chất ion dẫn điện trong trường hợp nào? Vì sao?
			\end{enumerate}
		\end{cauhoi}
		\loigiai{}
	\end{hoivadap}
	
\subsection{Bài tập}
\phan{Trắc nghiệm nhiều lựa chọn}
%%%=============SOẠN EX===============%%%
\Opensolutionfile{ansex}[Ans/LGEX-C03_B02_LKION.tex]
\Opensolutionfile{ans}[Ans/Ans-C03_B02_LKION.tex]
%%%==============Cau_1==============%%%
\begin{ex}
	Liên kết ion được tạo thành giữa hai nguyên tử bằng
	\choice
	{một hay nhiều cặp electron dùng chung}
	{một hay nhiều cặp electron dùng chung chỉ do một nguyên tử đóng góp}
	{\True lực hút tĩnh điện giữa các ion mang điện tích trái dấu}
	{một hay nhiều cặp electron dùng chung và các cặp electron này lệch về nguyên tử có độ âm điện lớn hơn}
	\loigiai{}
\end{ex}
%%%==============HetCau_1==============%%%

%%%==============Cau_2==============%%%
\begin{ex}
	Liên kết ion là loại liên kết hoá học được hình thành nhờ lực hút tĩnh điện giữa các phần tử nào sau đây?
	\choice
	{\True cation và anion}
	{các anion}
	{cation và electron tự do}
	{electron và hạt nhân nguyên tử}
	\loigiai{}
\end{ex}
%%%==============HetCau_2==============%%%

%%%==============Cau_3==============%%%
\begin{ex}
	Biểu diễn sự tạo thành ion nào sau đây đúng?
	\choice
	{$\mathrm{Na}+\mathrm{le} \to \mathrm{Na}^{+}$}
	{$\mathrm{Cl}_2\to 2\mathrm{Cl}^{-}+2e$}
	{$O_2+2e \to 2O^{2-}$}
	{\True $\mathrm{Al} \to \mathrm{Al}^{3+}+3e$}
	\loigiai{}
\end{ex}
%%%==============HetCau_3==============%%%

%%%==============Cau_4==============%%%
\begin{ex}
	Số electron và số proton trong ion $NH_4^{+}$ là
	\choice
	{11 electron và 11 proton}
	{\True 10 electron và 11 proton}
	{11 electron và 10 proton}
	{11 electron và 12 proton}
	\loigiai{}
\end{ex}
%%%==============HetCau_4==============%%%

%%%==============Cau_5==============%%%
\begin{ex}
	Cặp nguyên tử nào sau đây không tạo hợp chất dạng $X_2^{+} Y^{2-}$ hoặc $X^{2+} Y_2^{-}$?
	\choice
	{Na và O}
	{K và S}
	{\True Ca và O}
	{Ca và Cl}
	\loigiai{}
\end{ex}
%%%==============HetCau_5==============%%%

%%%==============Cau_6==============%%%
\begin{ex}
	Tính chất nào sau đây là tính chất của hợp chất ion?
	\choice
	{Hợp chất ion có nhiệt độ nóng chảy thấp}
	{\True Hợp chất ion có nhiệt độ nóng chảy cao}
	{Hợp chất ion dễ hoá lỏng}
	{Hợp chất ion có nhiệt độ sôi không xác định.}
	\loigiai{}
	\end{ex}
	
	%%%==============Cau_7==============%%%
	\begin{ex}
		Cho các phân tử sau: $\mathrm{HCl}, \mathrm{NaCl}, \mathrm{CaCl}_2, \mathrm{AlCl}_3$. Phân tử có liên kết mang nhiều tính chất ion nhất là
		\choice
		{HCl}
		{\True NaCl}
		{$\mathrm{CaCl}_2$}
		{$\mathrm{AlCl}_3$}
		\loigiai{}
	\end{ex}
	
	%%%==============Cau_8==============%%%
	\begin{ex}
		Dãy gồm các phân tử đều có liên kết ion là
		\choice
		{$\mathrm{Cl}_2, \mathrm{Br}_2, I_2, \mathrm{HCl}$}
		{$\mathrm{HCl}, H_2\mathrm{~S}, \mathrm{NaCl}, N_2O$}
		{\True $\mathrm{Na}_2O, \mathrm{KCl}, \mathrm{BaCl}_2, \mathrm{Al}_2O_3$}
		{$\mathrm{MgO}, H_2SO_4, H_3PO_4, \mathrm{HCl}$}
		\loigiai{}
	\end{ex}

	%%%=============EX_9=============%%%
	\begin{ex}
		Liên kết ion là liên kết được tạo thành bằng
		\choice
		{lực hút tĩnh điện giữa các electron tự do với ion dương kim loại}
		{cặp electron chung giữa hai nguyên tử}
		{\True lực hút tĩnh điện giữa các ion mang điện trái dấu}
		{cặp electron chung chỉ do một nguyên tử đóng góp}
		\loigiai{}
	\end{ex}
	
	%%%=============EX_10=============%%%
	\begin{ex}
		Ion dương được hình thành khi nguyên tử
		\choice
		{\True nhường electron}
		{nhận electron}
		{nhường proton}
		{nhận proton}
		\loigiai{}
	\end{ex}
	
	%%%=============EX_11=============%%%
	\begin{ex}
		Khi hình thành liên kết hóa học, nguyên tử $\mathrm{Na}(Z=11)$ có xu hướng nhường electron tạo thành ion
		\choice
		{\True $\mathrm{Na}^{+}$}
		{$\mathrm{Na}^{2+}$}
		{$\mathrm{Na}^{-}$}
		{$\mathrm{Na}^{2-}$}
		\loigiai{}
	\end{ex}
	
	%%%=============EX_12=============%%%
	\begin{ex}
		Phân tử nào sau đây có liên kết ion?
		\choice
		{$NH_3$}
		{$H_2\mathrm{~S}$}
		{HCl}
		{\True NaBr}
		\loigiai{}
	\end{ex}
	
	%%%=============EX_13=============%%%
	\begin{ex}
		Liên kết trong phân tử nào sau đây là liên kết ion?
		\choice
		{\True CaO}
		{$NH_3$}
		{$\mathrm{Cl}_2O_5$}
		{$\mathrm{Br}_2O_7$}
		\loigiai{}
	\end{ex}
	
	%%%=============EX_14=============%%%
	\begin{ex}
		Biết potasium có $Z=19$, trong phân tử $K_2O$, mỗi ion $K^{+}$ có số electron là
		\choice
		{\True $18$}
		{$19$}
		{$20$}
		{$10$}
		\loigiai{}
	\end{ex}
	
	%%%=============EX_15=============%%%
	\begin{ex}
		Ion nào sau đây thuộc loại ion đa nguyên tử?
		\choice
		{\True $NH_4^{+}$}
		{$\mathrm{Na}^{+}$}
		{$\mathrm{Ca}^{2+}$}
		{$\mathrm{Cl}^{-}$}
		\loigiai{}
	\end{ex}
	
	%%%=============EX_16=============%%%
	\begin{ex}
		Phân tử nào sau đây có chứa ion đa nguyên tử?
		\choice
		{$K_2\mathrm{~S}$}
		{\True $NH_4\mathrm{Cl}$}
		{$\mathrm{AlBr}_3$}
		{ZnO}
		\loigiai{}
	\end{ex}
	
	%%%=============EX_17=============%%%
	\begin{ex}
		Oxide của nguyên tố nào sau đây có liên kết ion?
		\choice
		{Nitrogen}
		{Carbon}
		{Sulfur}
		{\True Calcium}
		\loigiai{}
	\end{ex}
	
	%%%=============EX_18=============%%%
	\begin{ex}
		Phân tử nào dưới đây chỉ chứa ion đơn nguyên tử?
		\choice
		{$K_2SO_4$}
		{$NH_4\mathrm{Cl}$}
		{\True NaCl}
		{$\mathrm{Zn}\left(NO_3\right)_2$}
		\loigiai{}
	\end{ex}
	
	%%%=============EX_19=============%%%
	\begin{ex}
		Hợp chất nào sau đây có liên kết ion?
		\choice
		{$H_2\mathrm{~S}$}
		{$H_2O$}
		{\True $\mathrm{MgCl}_2$}
		{$CO_2$}
		\loigiai{}
	\end{ex}
	
	%%%=============EX_20=============%%%
	\begin{ex}
		Quá trình tạo thành ion nào sau đây được viết đúng?
		\choice
		{$K+1e \to K^{+}$}
		{$\mathrm{Cl}_2\to 2\mathrm{Cl}^{-}+2e$}
		{$O_2+2e \to 2O^{2-}$}
		{\True $\mathrm{Al} \to \mathrm{Al}^{3+}+3e$}
		\loigiai{}
	\end{ex}
	
	%%%=============EX_21=============%%%
	\begin{ex}
		Ion nào sau đây là ion đa nguyên tử?
		\choice
		{$\mathrm{Na}^{+}$}
		{\True $NO_3^{-}$}
		{$\mathrm{Cl}^{-}$}
		{$O^{2-}$}
		\loigiai{}
	\end{ex}
	
	%%%=============EX_22=============%%%
	\begin{ex}
		Sodium chloride là một hợp chất có thể tan trong nước lạnh và có nhiệt độ nóng chảy cao. Liên kết trong phân tử sodium chloride là liên kết
		\choice
		{cộng hóa trị không phân cực}
		{\True liên kết ion}
		{hydrogen}
		{cộng hóa trị phân cực}
		\loigiai{}
	\end{ex}
	
	%%%=============EX_23=============%%%
	\begin{ex}
		Tính chất nào sau đây không đúng với hợp chất ion?
		\choice
		{Nhiệt độ nóng chảy cao}
		{Tan tốt trong nước}
		{\True Không dẫn điện}
		{Rắn chắc nhưng khá giòn}
		\loigiai{}
	\end{ex}
	
	%%%=============EX_24=============%%%
	\begin{ex}
		Tính chất nào sau đây là tính chất của hợp chất ion?
		\choice
		{Nhiệt độ nóng chảy thấp}
		{\True Tan nhiều trong nước}
		{Dễ bay hơi}
		{Nhiệt độ sôi thấp}
		\loigiai{}
	\end{ex}
	
	%%%=============EX_25=============%%%
	\begin{ex}
		Trong các hợp chất sau: $\mathrm{CaO}, \mathrm{Ba}\left(NO_3\right)_2, \mathrm{Na}_2O, KF, K_2SO_4, NH_4\mathrm{Cl}$, số hợp chất chứa ion đa nguyên tử là
		\choice
		{$2$}
		{\True $3$}
		{$4$}
		{$5$}
		\loigiai{}
	\end{ex}
	
	%%%=============EX_26=============%%%
	\begin{ex}
		Liên kết trong phân tử muối clorua của kim loại kiềm nào sau đây mang nhiều đặc tính ion nhất?
		\choice
		{CsCl}
		{LiCl}
		{\True KCl}
		{RbCl}
		\loigiai{}
	\end{ex}
	
	%%%=============EX_27=============%%%
	\begin{ex}
		Sự kết hợp của các nguyên tử nào sau đây không tạo hợp chất dạng $X_2^{+} Y^{2-}$ hoặc $X^{2+} Y_2^{-2}$?
		\choice
		{$_{11} \mathrm{Na}$ và $_8O$}
		{$_{12} \mathrm{Mg}$ và $_{17} \mathrm{Cl}$}
		{\True $_{20} \mathrm{Ca}$ và $_8O$}
		{$_{20} \mathrm{Ca}$ và $_{17} \mathrm{Cl}$}
		\loigiai{}
	\end{ex}
	
	%%%=============EX_28=============%%%
	\begin{ex}
		Cặp nguyên tử có cấu hình electron nào sau đây có thể tạo liên kết ion?
		\choice
		{$1s^22s^22p^3$ và $1s^22s^22p^5$}
		{\True $1s^22s^22p^63s^1$ và $1s^22s^22p^5$}
		{$1s^22s^22p^3$ và $1s^22s^22p^4$}
		{$1s^2$ và $1s^22s^22p^4$}
		\loigiai{}
	\end{ex}
	
	%%%=============EX_29=============%%%
	\begin{ex}
		Nguyên tử $X$ có 12 electron, nguyên tử $Y$ có 17 electron. Công thức hợp chất và loại liên kết hình thành giữa hai nguyên tử này là
		\choice
		{\True $XY_2$, liên kết ion}
		{$X_3Y_2$, liên kết cộng hóa trị}
		{$X_2Y$, liên kết cộng hóa trị}
		{XY, liên kết ion}
		\loigiai{}
	\end{ex}
	
	%%%=============EX_30=============%%%
	\begin{ex}
		Cho cấu hình electron nguyên tử của các nguyên tố sau: $X(1s^22s^22p^63s^23p^64s^1), Y(1s^22s^22p^4)$. Hợp chất ion được tạo thành từ X và Y có công thức là
		\choice
		{XY}
		{$X_2Y_3$}
		{$XY_2$}
		{\True $X_2Y$}
		\loigiai{}
	\end{ex}
\Closesolutionfile{ans}
\Closesolutionfile{ansex}
%\bangdapan{Ans-C03_B02_LKION.tex}
\phan{Trắc nghiệm đúng sai}
%%%=============SOẠN EXTF===============%%%
\Opensolutionfile{ansex}[Ans/LGTF-C03_B02_LKION.tex]
\Opensolutionfile{ansbook}[Ansbook/AnsTF-C03_B02_LKION.tex]
\Opensolutionfile{ans}[Ans/Tempt-C03_B02_LKION.tex]
%%%=============EX_1=============%%%
\begin{ex}
	Liên kết ion được hình thành do lực hút tĩnh điện giữa các ion mang điện tích trái dấu.
	\choiceTF
	{\True Liên kết ion được hình thành do lực hút tĩnh điện giữa các ion mang điện tích trái dấu.}
	{Liên kết ion được hình thành do lực hút tĩnh điện giữa các ion mang điện tích cùng dấu.}
	{\True Liên kết ion chỉ được hình thành giữa kim loại điển hình và phi kim điển hình.}
	{Liên kết ion có tính định hướng.}
	\loigiai{
		\begin{itemchoice}[T1,F2,T3,F4]
			\itemch Lực hút tĩnh điện giữa các ion mang điện tích trái dấu là bản chất của liên kết ion.
			\itemch Các ion mang điện tích cùng dấu sẽ đẩy nhau.
			\itemch Kim loại điển hình dễ nhường electron tạo cation, phi kim điển hình dễ nhận electron tạo anion, tạo điều kiện hình thành liên kết ion.
			\itemch Liên kết ion không có tính định hướng, lực hút tĩnh điện giữa các ion trái dấu không theo một hướng nhất định.
		\end{itemchoice}
	}
\end{ex}
%%%=============EX_2=============%%%
\begin{ex}
	Về cấu trúc và tính chất của hợp chất ion, cho biết tính đúng/ sai các phát biểu sau?
	\choiceTF
	{Các ion trong hợp chất ion sắp xếp tự do, không theo trật tự nào.}
	{\True Các ion trong hợp chất ion sắp xếp theo một trật tự nhất định, tạo thành mạng tinh thể ion.}
	{Mạng tinh thể ion không bền vững.}
	{\True Hợp chất ion có nhiệt độ nóng chảy và nhiệt độ sôi cao.}
	\loigiai{
		\begin{itemchoice}[F1,T2,F3,T4]
			\itemch Các ion trong hợp chất ion được sắp xếp theo trật tự nhất định trong mạng tinh thể.
			\itemch  Các ion trong hợp chất ion sắp xếp có trật tự, lực hút tĩnh điện mạnh, tạo thành mạng tinh thể ion.
			\itemch Mạng tinh thể ion rất bền vững do lực hút tĩnh điện mạnh giữa các ion.
			\itemch Do mạng tinh thể ion bền vững nên hợp chất ion có nhiệt độ nóng chảy và nhiệt độ sôi cao.
		\end{itemchoice}
	}
\end{ex}
%%%=============EX_3=============%%%
\begin{ex}
	Phân tích các phát biểu sau về hợp chất natri clorua ($NaCl$)
	\choiceTF
	{\True $NaCl$ được tạo thành từ kim loại điển hình $Na$ và phi kim điển hình $Cl$.}
	{Liên kết trong $NaCl$ là liên kết cộng hóa trị.}
	{\True $Na$ dễ nhường 1 electron tạo $Na^+$, $Cl$ dễ nhận 1 electron tạo $Cl^-$.}
	{Phân tử $NaCl$ ở trạng thái rắn, lỏng, khí đều dẫn điện.}
	\loigiai{
		\begin{itemchoice}[T1,F2,T3,F4]
			\itemch $NaCl$ được tạo từ kim loại điển hình $Na$ và phi kim điển hình $Cl$, nên có liên kết ion.
			\itemch Liên kết trong $NaCl$ là liên kết ion.
			\itemch $Na$ dễ nhường 1e tạo $Na^+$, $Cl$ dễ nhận 1e tạo $Cl^-$, tạo ra lực hút tĩnh điện hình thành liên kết ion.
			\itemch $NaCl$ chỉ dẫn điện ở trạng thái nóng chảy và dung dịch.
		\end{itemchoice}
	}
\end{ex}
%%%=============EX_4=============%%%
\begin{ex}
	Về tính chất của liên kết ion và hợp chất ion
	\choiceTF
	{\True Độ bền của liên kết ion tỉ lệ thuận với hiệu độ âm điện giữa các nguyên tố tham gia liên kết.}
	{Liên kết ion thể hiện tính định hướng trong không gian}
	{Tất cả các hợp chất ion đều tan tốt trong nước.}
	{\True Hợp chất ion không dẫn điện ở trạng thái rắn.}
	\loigiai{
		\begin{itemchoice}[T1,F2,T3,T4]
			\itemch Sự chênh lệch độ âm điện lớn dẫn đến sự hình thành ion dễ dàng hơn, liên kết ion bền vững hơn.
			\itemch Liên kết ion là lực hút tĩnh điện giữa các ion trái dấu.Lực này tác dụng theo mọi hướng trong không gian (đẳng hướng)
			\itemch Có một số hợp chất ion khó tan trong nước như: AgCl, BaSO$_4$, PbCl$_2$.
			\itemch  Ở trạng thái rắn, các ion bị cố định trong mạng tinh thể, không di chuyển được nên hợp chất ion không dẫn điện.
		\end{itemchoice}
	}
\end{ex}
%%%=============EX_5=============%%%
\begin{ex}
	Về hợp chất ion.
	\choiceTF
	{\True  $KCl$ tan tốt trong nước do nước là dung môi phân cực mạnh.}
	{Các hợp chất ion đều tan tốt trong nước.}
	{\True Liên kết ion trong $KCl$ bị phá vỡ khi hòa tan vào nước.}
	{\True Dung dịch $KCl$ dẫn điện.}
	\loigiai{
		\begin{itemchoice}[T1,F2,T3,T4]
			\itemch Nước là dung môi phân cực mạnh, có khả năng hydrat hóa các ion.
			\itemch Không phải tất cả các hợp chất ion đều tan tốt trong nước.
			\itemch Khi hòa tan vào nước, nước sẽ hydrat hóa các ion, phá vỡ liên kết ion.
			\itemch Dung dịch $KCl$ dẫn điện do có các ion tự do di chuyển.
		\end{itemchoice}
	}
\end{ex}
%%%=============EX_6=============%%%
\begin{ex}
	Nghiên cứu tính chất vật lí của canxi clorua ($CaCl_2$)
	\choiceTF
	{$CaCl_2$ là hợp chất phân tử nên dễ bay hơi ở nhiệt độ thường.}
	{\True Để nóng chảy $CaCl_2$ cần nhiệt độ rất cao do lực hút tĩnh điện mạnh giữa các ion trong mạng tinh thể.}
	{Tinh thể $CaCl_2$ có khả năng dẫn điện do chứa các ion $Ca^{2+}$ và $Cl^-$.}
	{\True $CaCl_2$ nóng chảy dẫn điện tốt vì các ion được giải phóng khỏi mạng tinh thể và chuyển động tự do.}
	\loigiai{
		\begin{itemchoice}[F1,T2,F3,T4]
			\itemch Về bản chất của $CaCl_2$ là hợp chất ion, không phải hợp chất phân tử
			\itemch Về nhiệt độ nóng chảy của $CaCl_2$ cao do lực hút tĩnh điện mạnh giữa ion $Ca^{2+}$ và $Cl^-$ và cấu trúc mạng tinh thể ion bền vững
			\itemch Mặc dù chứa các ion nhưng tinh thể $CaCl_2$ không dẫn điện. Nguyên nhân: các ion được sắp xếp có trật tự và cố định trong mạng tinh thể do đó các ion không thể di chuyển tự do nên không thể dẫn điện
			\itemch  Về khả năng dẫn điện của $CaCl_2$ nóng chảy: Khi nóng chảy, mạng tinh thể ion bị phá vỡ.Các ion $Ca^{2+}$ và $Cl^-$ được giải phóng.Các ion có thể di chuyển tự do do đó, $CaCl_2$ nóng chảy dẫn điện tốt
		\end{itemchoice}
	}
\end{ex}
%%%=============EX_7=============%%%
\begin{ex}
	Một nguyên tố X có cấu hình electron lớp ngoài cùng là $3s^23p^5$.
	\choiceTF
	{\True X là phi kim và có xu hướng nhận 1 electron}
	{Hợp chất của X với Na có công thức $Na_2X$}
	{Ion $X^-$ có cấu hình electron của khí hiếm Kr}
	{\True X có thể tạo liên kết cộng hóa trị với H theo tỉ lệ $1:1$}
	\loigiai{%
		\begin{itemchoice}[T1,F2,F3,T4]
			\itemch X có 7e lớp ngoài cùng (2e ở 3s và 5e ở 3p) nên là phi kim mạnh, có xu hướng nhận 1e để đạt cấu hình electron khí hiếm.
			\itemch Na cho 1e, X nhận 1e nên tỉ lệ $Na:X = 1:1$, công thức là NaX.
			\itemch $X^-$: $[Ne]3s^23p^6 = [Ar]$, không phải Kr.
			\itemch X có thể dùng chung 1e với H tạo liên kết cộng hóa trị, tạo thành phân tử HX.
		\end{itemchoice}
	}
\end{ex}
%%%=============EX_8=============%%%
\begin{ex}
	Cho các phát biểu sau về liên kết ion:
	\choiceTF
	{\True Hình thành do lực hút tĩnh điện giữa các ion trái dấu.}
	{Chỉ tồn tại giữa kim loại và phi kim.}
	{\True Hợp chất ion thường là chất rắn ở điều kiện thường.}
	{Hợp chất ion luôn dẫn điện tốt ở trạng thái rắn.}
	\loigiai{%
		\begin{itemchoice}[T1,F2,T3,F4]
			\itemch Liên kết ion hình thành do lực hút tĩnh điện giữa các ion trái dấu.
			\itemch Liên kết ion có thể tồn tại giữa kim loại và phi kim, hoặc giữa ion kim loại và nhóm phi kim (ví dụ $NH_4^+$, $SO_4^{2-}$).
			\itemch Hợp chất ion thường là chất rắn ở điều kiện thường.
			\itemch Hợp chất ion dẫn điện tốt ở trạng thái nóng chảy và dung dịch.
		\end{itemchoice}
	}
\end{ex}
%%%=============EX_9=============%%%
\begin{ex}
	Cho phản ứng: $2\text{ Na }+\text{ Cl }_2\to 2\text{ NaCl }$. Trong hợp chất $\text{ NaCl }$:
	\choiceTF
	{\True Ion $\text{ Na }^+$ có cấu hình electron của $\text{ Ne }$}
	{$\text{ Na }$ nhận thêm 1 electron từ $\text{ Cl }$}
	{Ion $\text{ Cl }^-$ có cấu hình electron của $\text{ Kr }$}
	{\True Liên kết được hình thành do lực hút tĩnh điện}
	\loigiai{
		\begin{itemchoice}[T1,F2,F3,T4]
			\itemch $\text{ Na }$ ($Z=11$): $[\text{ Ne }]3s^1\to \text{ Na }^+: [\text{ Ne }]$, có cấu hình electron của $\text{ Ne }$.
			\itemch $\text{ Na }$ cho 1e (không phải nhận) để tạo thành ion $\text{ Na }^+$.
			\itemch Ion $\text{ Cl }^-$ có cấu hình của $\text{ Ar }$ (không phải $\text{ Kr }$).
			\itemch Liên kết ion trong $\text{ NaCl }$ là lực hút tĩnh điện giữa $\text{ Na }^+$ và $\text{ Cl }^-$.
		\end{itemchoice}
	}
\end{ex}
%%%=============EX_10=============%%%
\begin{ex}
	Cho nguyên tử M có cấu hình electron $[\text{ Ne }]3s^2$. Hợp chất của M với ion $\text{ SO }_4^{2-}$:
	\choiceTF
	{Có công thức $\text{ MSO }_4$}
	{\True Có công thức $\text{ M }_2\text{ SO }_4$}
	{\True M là kim loại kiềm thổ}
	{\True M cho 2 electron tạo ion $\text{ M }^{2+}$}
	\loigiai{
		\begin{itemchoice}[F1,T2,T3,T4]
			\itemch $\text{ MSO }_4$ sai vì không cân bằng điện tích ($\text{ M }^{2+}$ và $\text{ SO }_4^{2-}$).
			\itemch $\text{ M }_2\text{ SO }_4$ đúng vì $2(\text{ M }^{2+})$ cân bằng điện tích với $\text{ SO }_4^{2-}$.
			\itemch $[\text{ Ne }]3s^2$ là cấu hình của kim loại kiềm thổ (nhóm IIA).
			\itemch M có 2e hóa trị nên cho 2e để tạo ion $\text{ M }^{2+}$.
		\end{itemchoice}
	}
\end{ex}
%%%=============EX_11=============%%%
\begin{ex}
	Cho phản ứng: $2\text{K} + \text{Br}_2 \rightarrow 2\text{KBr}$. Xét quá trình tạo liên kết:
	\choiceTF
	{\True $\text{K}$ có năng lượng ion hóa nhỏ hơn $\text{Na}$}
	{Ion $\text{K}^+$ và $\text{Br}^-$ có cùng kích thước}
	{\True $\text{Br}$ có ái lực electron lớn hơn $\text{Cl}$}
	{\True Liên kết trong $\text{KBr}$ yếu hơn trong $\text{NaCl}$}
	\loigiai{
		\begin{itemchoice}[T1,F2,T3,T4]
			\itemch K ở chu kỳ lớn hơn nên năng lượng ion hóa nhỏ hơn Na.
			\itemch $\text{K}^+$ có kích thước nhỏ hơn $\text{Br}^-$ do mất 1e.
			\itemch Br ở chu kỳ lớn hơn nên có ái lực electron lớn hơn Cl.
			\itemch Do khoảng cách giữa các ion lớn hơn nên lực hút tĩnh điện yếu hơn.
		\end{itemchoice}
	}
\end{ex}
%%%=============EX_12=============%%%
\begin{ex}
	Cho hai hợp chất $\text{MgO}$ và $\text{CaO}$. Xét các nhận định:
	\choiceTF
	{\True Cả hai đều là hợp chất ion}
	{$\text{MgO}$ bền hơn $\text{CaO}$}
	{\True $\text{Mg}^{2+}$ nhỏ hơn $\text{Ca}^{2+}$}
	{\True Cả hai đều có nhiệt độ nóng chảy cao}
	\loigiai{
		\begin{itemchoice}[T1,F2,T3,T4]
			\itemch $\text{Mg}^{2+}$, $\text{Ca}^{2+}$ và $\text{O}^{2-}$ tạo liên kết ion.
			\itemch $\text{CaO}$ bền hơn do $\text{Ca}^{2+}$ có năng lượng ion hóa nhỏ hơn.
			\itemch $\text{Mg}^{2+}$ ở chu kỳ 3, $\text{Ca}^{2+}$ ở chu kỳ 4.
			\itemch Liên kết ion mạnh nên nhiệt độ nóng chảy cao.
		\end{itemchoice}
	}
\end{ex}
%%%=============EX_13=============%%%
\begin{ex}
	Ion $\text{Al}^{3+}$ trong hợp chất với ion $\text{PO}_4^{3-}$:
	\choiceTF
	{Tạo thành hợp chất $\text{AlPO}_4$}
	{\True Có cấu hình electron của $\text{Ne}$}
	{\True Tạo thành hợp chất $\text{AlPO}_4$ không tan trong nước}
	{\True Tạo liên kết ion với $\text{PO}_4^{3-}$}
	\loigiai{
		\begin{itemchoice}[F1,T2,T3,T4]
			\itemch $\text{AlPO}_4$ đúng vì điện tích cân bằng, $\text{Al}^{3+}:\text{PO}_4^{3-} = 1:1$.
			\itemch $\text{Al}^{3+}$ có cấu hình electron $1s^22s^22p^6$ giống $\text{Ne}$.
			\itemch $\text{AlPO}_4$ là hợp chất ion có độ tan thấp trong nước.
			\itemch Liên kết giữa $\text{Al}^{3+}$ và $\text{PO}_4^{3-}$ là liên kết ion.
		\end{itemchoice}
	}
\end{ex}
%%%=============EX_14=============%%%
\begin{ex}
	Cho hai hợp chất $\text{LiF}$ và $\text{CsF}$. So sánh:
	\choiceTF
	{\True $\text{LiF}$ có tính ion mạnh hơn}
	{Ion $\text{Li}^+$ lớn hơn ion $\text{Cs}^+$}
	{\True $\text{LiF}$ có nhiệt độ nóng chảy cao hơn}
	{\True $\text{F}^-$ đều có cấu hình của $\text{Ne}$}
	\loigiai{
		\begin{itemchoice}[T1,F2,T3,T4]
			\itemch LiF có độ chênh lệch điện tích lớn hơn nên tính ion mạnh hơn.
			\itemch $\text{Li}^+$ (chu kỳ 2) nhỏ hơn $\text{Cs}^+$ (chu kỳ 6).
			\itemch Lực hút tĩnh điện mạnh hơn nên nhiệt độ nóng chảy cao hơn.
			\itemch $\text{F}^-$ có cấu hình electron $1s^22s^22p^6$ giống $\text{Ne}$.
		\end{itemchoice}
	}
\end{ex}
%%%=============EX_15=============%%%
\begin{ex}
	Cho dung dịch chứa $\text{Ba}^{2+}$ tác dụng với dung dịch chứa $\text{SO}_4^{2-}$:
	\choiceTF
	{\True Tạo thành kết tủa $\text{BaSO}_4$}
	{Kết tủa có màu}
	{\True $\text{BaSO}_4$ không tan trong nước}
	{\True Liên kết trong $\text{BaSO}_4$ là liên kết ion}
	\loigiai{
		\begin{itemchoice}[T1,F2,T3,T4]
			\itemch $\text{Ba}^{2+}$ kết hợp với $\text{SO}_4^{2-}$ tạo kết tủa $\text{BaSO}_4$.
			\itemch $\text{BaSO}_4$ là kết tủa màu trắng.
			\itemch $\text{BaSO}_4$ có độ tan rất nhỏ trong nước.
			\itemch Liên kết giữa $\text{Ba}^{2+}$ và $\text{SO}_4^{2-}$ là liên kết ion.
		\end{itemchoice}
	}
\end{ex}
%%%=============EX_16=============%%%
\begin{ex}
	Các ion $\text{Na}^+$, $\text{K}^+$, $\text{Mg}^{2+}$, $\text{Ca}^{2+}$ trong nước:
	\choiceTF
	{\True Đều được hình thành từ kim loại}
	{Có cùng kích thước với nhau}
	{\True Đều tạo được hợp chất với $\text{Cl}^-$}
	{\True Đều có cấu hình electron của khí hiếm}
	\loigiai{
		\begin{itemchoice}[T1,F2,T3,T4]
			\itemch Đều là ion của các kim loại kiềm và kiềm thổ.
			\itemch Kích thước khác nhau do khác chu kỳ và điện tích.
			\itemch Tạo thành $\text{NaCl}$, $\text{KCl}$, $\text{MgCl}_2$, $\text{CaCl}_2$.
			\itemch $\text{Na}^+$, $\text{K}^+$ có cấu hình $\text{Ne}$, $\text{Ar}$; $\text{Mg}^{2+}$, $\text{Ca}^{2+}$ có cấu hình $\text{Ne}$, $\text{Ar}$.
		\end{itemchoice}
	}
\end{ex}
%%%=============EX_17=============%%%
\begin{ex}
	Xét các tính chất của hợp chất $\text{CaCl}_2$:
	\choiceTF
	{\True Hút ẩm mạnh}
	{Không tan trong nước}
	{\True Tinh thể có màu trắng}
	{\True Dẫn điện khi nóng chảy}
	\loigiai{
		\begin{itemchoice}[T1,F2,T3,T4]
			\itemch $\text{CaCl}_2$ là chất hút ẩm mạnh, được dùng làm chất hút ẩm.
			\itemch $\text{CaCl}_2$ tan tốt trong nước.
			\itemch $\text{CaCl}_2$ là tinh thể ion màu trắng.
			\itemch Khi nóng chảy, các ion tự do chuyển động nên dẫn điện.
		\end{itemchoice}
	}
\end{ex}
%%%=============EX_18=============%%%
\begin{ex}
	Trong phản ứng tạo $\text{MgCl}_2$:
	\choiceTF
	{\True $\text{Mg}$ mất 2 electron thành $\text{Mg}^{2+}$}
	{$\text{Cl}_2$ nhận 1 electron thành $2\text{Cl}^-$}
	{\True Liên kết ion được tạo thành sau khi trao đổi electron}
	{\True Ion $\text{Mg}^{2+}$ có kích thước nhỏ hơn nguyên tử $\text{Mg}$}
	\loigiai{
		\begin{itemchoice}[T1,F2,T3,T4]
			\itemch $\text{Mg}$ mất 2e để đạt cấu hình electron của $\text{Ne}$.
			\itemch Mỗi nguyên tử $\text{Cl}$ nhận 1e (không phải $\text{Cl}_2$ nhận 1e).
			\itemch Sau khi trao đổi electron, các ion tạo liên kết ion.
			\itemch Do mất 2e nên $\text{Mg}^{2+}$ có kích thước nhỏ hơn nguyên tử $\text{Mg}$.
		\end{itemchoice}
	}
\end{ex}
%%%=============EX_19=============%%%
\begin{ex}
	So sánh $\text{NaCl}$ và $\text{AgCl}$:
	\choiceTF
	{\True Đều là hợp chất ion}
	{Có cùng độ tan trong nước}
	{\True $\text{AgCl}$ có màu trắng}
	{\True $\text{Ag}^+$ và $\text{Na}^+$ có cùng điện tích}
	\loigiai{
		\begin{itemchoice}[T1,F2,T3,T4]
			\itemch Cả hai đều được tạo thành từ kim loại và phi kim.
			\itemch $\text{NaCl}$ tan tốt trong nước, $\text{AgCl}$ không tan.
			\itemch $\text{AgCl}$ là kết tủa màu trắng.
			\itemch Đều là ion +1 do mất 1 electron.
		\end{itemchoice}
	}
\end{ex}
%%%=============EX_20=============%%%
\begin{ex}
	Xét sự phân ly của $\text{Na}_2\text{CO}_3$ trong nước:
	\choiceTF
	{\True Tạo thành ion $\text{Na}^+$ và $\text{CO}_3^{2-}$}
	{Tỉ lệ số ion $\text{Na}^+$ và $\text{CO}_3^{2-}$ là 1:1}
	{\True Dung dịch dẫn điện}
	{\True Phản ứng được với dung dịch $\text{CaCl}_2$}
	
	\loigiai{
		\begin{itemchoice}[T1,F2,T3,T4]
			\itemch $\text{Na}_2\text{CO}_3 \rightarrow 2\text{Na}^+ + \text{CO}_3^{2-}$
			\itemch Tỉ lệ số ion $\text{Na}^+:\text{CO}_3^{2-}$ = 2:1.
			\itemch Các ion tự do trong dung dịch dẫn điện tốt.
			\itemch Tạo kết tủa $\text{CaCO}_3$ khi tác dụng với $\text{CaCl}_2$.
		\end{itemchoice}
	}
\end{ex}
%%%=============EX_21=============%%%
\begin{ex}
	Các phát biểu sau là đúng hay sai khi nói về tính chất của hợp chất ion?
	\choiceTF
	{\True Có nhiệt độ nóng chảy cao}
	{\True Tan nhiều trong nước và tạo ra dung dịch dẫn điện tốt}
	{\True Không dẫn điện ở trạng thái rắn}
	{\True Ở trạng thái nóng chảy, dẫn điện rất tốt}
	\loigiai{}
\end{ex}

%%%=============EX_22=============%%%
\begin{ex}
	Các phát biểu sau là đúng hay sau khi nói về hợp chất sodium oxide $\left(\mathrm{Na}_2O\right)$?
	\choiceTF
	{\True Trong phân tử $\mathrm{Na}_2O$, các ion $\mathrm{Na}^{+}$ và ion $O^{2-}$ đều đạt cấu hình electron bền vững của khí hiếm neon}
	{\True Mỗi nguyên tử Na nhường 1 electron cho nguyên tử O để tạo thành ion dương}
	{\True Ở điều kiện thường, là chất rắn, khó nóng chảy, khó bay hơi}
	{Không tan trong nước, chỉ tan trong dung môi không phân cực như benzene, carbon tetrachloride,\ldots}
	\loigiai{}
\end{ex}

%%%=============EX_23=============%%%
\begin{ex}
	Các phát biểu sau là đúng hay sai khi nói về hợp chất magnesium oxide $(\mathrm{MgO})$?
	\choiceTF
	{\True Là hợp chất ion}
	{\True Là chất rắn ở điều kiện thường}
	{\True Có nhiệt độ nóng chảy rất cao (khoảng $2852^{\circ} C$)}
	{\True Phân tử tạo bởi lực hút tĩnh điện giữa ion $\mathrm{Mg}^{2+}$ và $O^{2-}$}
	\loigiai{}
\end{ex}
\Closesolutionfile{ans}
\Closesolutionfile{ansbook}
\Closesolutionfile{ansex}
%\bangdapanTF{AnsTF-C03_B02_LKION.tex}
\phan{Bài tập tự luận}
%%%=============SOẠN BT===============%%%
\Opensolutionfile{ansbth}[Ans/LGBT-C03_B02_LKION.tex]
\Opensolutionfile{ansbt}[Ans/AnsBT-C03_B02_LKION.tex]
%%%==============Bai_BT1==============%%%
\begin{bt}
	Cho các ion sau: $K^{+}; \mathrm{Be}^{2+}; \mathrm{Cr}^{3+}; F^{-}; \mathrm{Se}^{2-}; N^{3-}$.
	Viết phương trình biểu diễn sự hình thành mỗi ion trên.
	\loigiai{%
	Phương trình biểu diễn sự hình thành các ion:\\
	\begin{tabular}{*{3}{L{0.2\linewidth}}}
		$\mathrm{K} \rightarrow \mathrm{~K}^{+}+\mathrm{le}$ & $\mathrm{Be} \rightarrow \mathrm{Be}^{2+}+2 \mathrm{e}$ & $\mathrm{Cr} \rightarrow \mathrm{Cr}^{3+}+3 \mathrm{e}$ \\
		$\mathrm{~F}+\mathrm{e} \rightarrow \mathrm{~F}^{-}$ & $\mathrm{Se}+2 \mathrm{e} \rightarrow \mathrm{Se}^{2-}$ & $\mathrm{N}+3 \mathrm{e} \rightarrow \mathrm{~N}^{3-}$
	\end{tabular}
	}
\end{bt}
%%==============HetBai_BT1==============%%%

%%%==============Bai_BT2==============%%%
\begin{bt}
	Cho các ion sau: $_{20}\mathrm{Ca}^{2+}; _{13}\mathrm{Al}^{3+}; _{\phantom{9}9}\mathrm{F}^{-}; _{16}\mathrm{S}^{2-}; _{\phantom{7}7}\mathrm{N}^{3-}$.
	\begin{enumerate}
		\item Viết cấu hình electron của mỗi ion.
		\item Mỗi cấu hình đã viết giống với cấu hình electron của nguyên tử nào?
	\end{enumerate}
	\loigiai{%
		\begin{enumerate}
			\item Cấu hình electron:\\
			\begin{tabular}{L{0.35\linewidth}L{0.1\linewidth}L{0.35\linewidth}L{0.1\linewidth}}
				$_{20}\mathrm{Ca}^{2+}:$ $1 s^2 2 s^2 2 p^6 3 s^2 3 \mathrm{p}^6$ & \text {(I)} & $_{13}\mathrm{Al}^{3+}:$  $1\mathrm{~s}^22\mathrm{~s}^22\mathrm{p}^6$ &\text {(II)}\\
				 $_{\phantom{9}9}\mathrm{~F}^{-}: 1 \mathrm{~s}^2 2 \mathrm{~s}^2 2 \mathrm{p}^6$ & \text {(III)} & ${ }_{16} \mathrm{~S}^{2-}: 1 s^2 2 s^2 2 p^6 3 s^2 3 p^6$&\text {(IV)}\\
				 ${ }_7 \mathrm{~N}^{3-}: 1 \mathrm{~s}^2 2 \mathrm{~s}^2 2 \mathrm{p}^6$&\text {(V)}&&
			\end{tabular}
			\item Các cấu hình (II), (III), (V) giống cấu hình electron của khí hiếm 10 Ne . Các cấu hình (I), (IV) giống cấu hình electron của khí hiếm 18 Ar .
		\end{enumerate}
	}
\end{bt}
%%%==============HetBai_BT2==============%%%

%%%==============Bai_BT3==============%%%
\begin{bt}
	Vì sao các hợp chất ion thường là chất rắn ở nhiệt độ phòng?
	\loigiai{Các hợp chất ion thường là chất rắn ở nhiệt độ phòng vì hợp chất ion có cấu trúc mạng tinh thể ion. Lực tĩnh điện mạnh giữa các phần tử mạng với nhau làm cho khoảng cách giữa các phần tử ngắn lại.}
\end{bt}
%%%==============HetBai_BT3==============%%%

%%%==============Bai_BT4==============%%%
\begin{bt}
	Cho các chất sau: $K_2O, H_2O, H_2\mathrm{~S}, SO_2, \mathrm{NaCl}, K_2\mathrm{~S}, \mathrm{CaF}_2, \mathrm{HCl}$.
	Trong phân tử chất nào có liên kết ion?
	\loigiai{Những phân tử có liên kết ion là: $\mathrm{K}_2 \mathrm{O}, \mathrm{K}_2 \mathrm{~S}, \mathrm{NaCl}_{,} \mathrm{CaF}_2$.}
\end{bt}
%%%==============HetBai_BT4==============%%%

%%%==============Bai_BT5==============%%%
\begin{bt}
	Kể ra những hợp chất ion tạo thành từ các ion sau: $F^{-}, K^{+}, O^{2-}, \mathrm{Ca}^{2+}$.
	%%%VD
	\loigiai{Các hợp chất ion là: $\mathrm{KF}, \mathrm{K}_2 \mathrm{O}, \mathrm{CaF}_2, \mathrm{CaO}$.}
\end{bt}
%%%==============HetBai_BT5==============%%%

%%%==============Bai_BT6==============%%%
\begin{bt}
	Dùng sơ đồ để biểu diễn sự hình thành liên kết trong mỗi hợp chất ion sau đây:
	\begin{enumerate}
		\item magnesium fluoride $\left(\mathrm{MgF}_2\right)$;
		\item potassium fluoride (KF);
		\item sodium oxide $\left(\mathrm{Na}_2O\right)$;
		\item calcium oxide $(\mathrm{CaO})$.
	\end{enumerate}
	\loigiai{%
	\begin{enumerate}
		\item  Magnesium fluoride:\\
		\schemestart
			\chemfig{\charge{[.radius=0.2ex]0:2pt=\.,90:2pt=\:,-90:2pt=\:,180:2pt=\:}{F}} 
			\+ 
			\chemfig{\charge{[.radius=0.2ex]0:2pt=\.,180:2pt=\.}{Mg}} 
			\+
			\chemfig{\charge{[.radius=0.2ex]0:2pt=\:,90:2pt=\:,-90:2pt=\:,180:2pt=\.}{F}} 
			\arrow{->}[,,,-stealth]
			\khungion[-]{\chemfig{\charge{[.radius=0.2ex]0:2pt=\:,90:2pt=\:,-90:2pt=\:,180:2pt=\:}{F}}}
			\+ 
			\khungion[2+][-1.5]{\chemfig{Mg}} 
			\+
			\khungion[-]{\chemfig{\charge{[.radius=0.2ex]0:2pt=\:,90:2pt=\:,-90:2pt=\:,180:2pt=\:}{F}}}
			\arrow{->}[,,,-stealth]
			\chemfig{MgF_2}
		\schemestop
		\item  Potassium fluoride:\\
		\schemestart 
			\chemfig{\charge{[.radius=0.2ex]0:2pt=\.}{K}} 
			\+
			\chemfig{\charge{[.radius=0.2ex]0:2pt=\:,90:2pt=\:,-90:2pt=\:,180:2pt=\.}{F}} 
			\arrow{->}[,,,-stealth]
			\khungion[+][-1.5]{\chemfig{K}} 
			\+
			\khungion[-]{\chemfig{\charge{[.radius=0.2ex]0:2pt=\:,90:2pt=\:,-90:2pt=\:,180:2pt=\:}{F}}}
			\arrow{->}[,,,-stealth]
			\chemfig{KF}
		\schemestop
		\item  Sodium oxide:\\
		\schemestart
			\chemfig{\charge{[.radius=0.2ex]0:2pt=\.}{Na}} 
			\+ 
			\chemfig{\charge{[.radius=0.2ex]0:2pt=\.,180:2pt=\.,90:2pt=\:,-90:2pt=\:}{O}} 
			\+
			\chemfig{\charge{[.radius=0.2ex]180:2pt=\.}{Na}} 
			\arrow{->}[,,,-stealth]
			\khungion[+]{\chemfig{Na}}
			\+ 
			\khungion[2-][-1.5]{			\chemfig{\charge{[.radius=0.2ex]0:2pt=\:,180:2pt=\:,90:2pt=\:,-90:2pt=\:}{O}} } 
			\+
			\khungion[+]{\chemfig{Na}}
			\arrow{->}[,,,-stealth]
			\chemfig{Na_2O}
		\schemestop
		\item  Calcium oxide:\\
		\schemestart 
			\chemfig{\charge{[.radius=0.2ex]0:2pt=\:}{Ca}} 
			\+
			\chemfig{\charge{[.radius=0.2ex]0:2pt=\:,90:2pt=\:,-90:2pt=\:}{O}} 
			\arrow{->}[,,,-stealth]
			\khungion[2+][-1.5]{\chemfig{Ca}} 
			\+
			\khungion[2-]{\chemfig{\charge{[.radius=0.2ex]0:2pt=\:,180:2pt=\:,90:2pt=\:,-90:2pt=\:}{O}}}
			\arrow{->}[,,,-stealth]
			\chemfig{CaO}
		\schemestop
	\end{enumerate}
	}
\end{bt}
%%%==============HetBai_BT6==============%%%

%%%==============Bai_BT7==============%%%
\begin{bt}
	Anion $X^{-}$ có cấu hình electron nguyên tử ở phân lớp ngoài cùng là $3p^6$.
	\begin{enumerate}
		\item Viết cấu hình electron của nguyên tử $X$. Cho biết $X$ là nguyên tố kim loại hay phi kim.
		\item Giải thích bản chất liên kết giữa X với barium.
	\end{enumerate}
	\loigiai{\begin{enumerate}
			\item  Khi nhận electron, nguyên tử $X$ biến thành anion $X^{-}$. Cấu hình electron của X là $1 \mathrm{~s}^2 2 \mathrm{~s}^2 2 \mathrm{p}^6 3 \mathrm{~s}^2 3 \mathrm{p}^5$, X là chlorine. X là phi kim điển hình.
			\item  Barium là nguyên tố kim loại điển hình ở chu kì 6 , nhóm IIA. Barium dễ nhường 2 electron hoá trị và tạo ra cation có điện tích $2+$. Khi chlorine kết hợp với barium, nguyên tử barium nhường 2 electron cho hai nguyên tử chlorine (mỗi nguyên tử chlorine nhận 1 electron), tạo thành các ion $\mathrm{Ba}^{2+}$ và $\mathrm{Cl}^{-}$. Các ion này mang điện trái dấu sẽ hút nhau bằng lực hút tĩnh điện
	\end{enumerate}}
\end{bt}
%%%==============HetBai_BT7==============%%%

%%%==============Bai_BT8==============%%%
\begin{bt}
	Nguyên tố X tích luỹ trong các tế bào thực vật nên rau và trái cây tươi là nguồn cung cấp tốt nguyên tố $X$ cho cơ thể. Các nghiên cứu chỉ ra khẩu phần ăn chứa nhiều X có thể giảm nguy cơ cao huyết áp và đột quy. Nguyên tố Z được dùng chế tạo dược phẩm, phẩm nhuộm và chất nhạy với ánh sáng. Nguyên tử X chỉ có 7 electron trên phân lớp s; còn nguyên tử Z chỉ có 17 electron trên phân lớp p.
	\begin{enumerate}
		\item Viết công thức hoá học của hợp chất tạo bởi $X$ và $Z$.
		\item Hợp chất tạo bởi X và Z có tính dẫn điện không? Vì sao?
		\item Trong thực tế cuộc sống, hợp chất tạo bởi X và Z được dùng để làm gi?
	\end{enumerate}
	\loigiai{\begin{enumerate}
			\item  Nguyên tử X chỉ có 7 electron trên phân lớp s nên cấu hình electron của $X$ là: $1s^22s^22p^63s^23p^64s^1$.
			
			Nguyên tử $Z$ chỉ có 17 e trên phân lớp p nên cấu hình electron của $Z$ là:
			\[1\mathrm{s}^22\mathrm{s}^22\mathrm{p}^63\mathrm{s}^23\mathrm{p}^64\mathrm{s}^23\mathrm{d}^{10}4\mathrm{p}^5\]
			$\Rightarrow \mathrm{X}$ là ${ }_{19} \mathrm{~K}$ và Z là ${ }_{35} \mathrm{Br}$.
			$\Rightarrow$ Công thức hoá học của hợp chất tạo bởi X và Z là KBr .
			\item  Hợp chất KBr có tính dẫn điện khi nóng chảy hoặc tan trong dung dịch vì nó là hợp chất ion.
			\item  Trong thực tế, KBr được sử dụng rộng rãi như thuốc chống co giật và an thần, nó là muối ion điển hình, hoàn toàn phân cực và đạt độ $\mathrm{pH}=7$ trong dung dịch nước.
	\end{enumerate}}
\end{bt}

%%%=============BT_9=============%%%
\begin{bt}
	Cho dãy các ion: $K^{+}$, $S^{2-}$, $\mathrm{Al}^{3+}$, $SO_4^{2-}$, $NH_4^{+}$, $CO_3^{2-}$, $\mathrm{Na}^{+}$, $NO_3^{-}$, $\mathrm{Cl}^{-}$, $\mathrm{Mg}^{2+}$. Có bao nhiêu ion thuộc loại cation?
	\shortans{}
	\loigiai{}
\end{bt}

%%%=============BT_10=============%%%
\begin{bt}
	Cho dãy các ion: $K^{+}, S^{2-}, \mathrm{Al}^{3+}, SO_4^{2-}, NH_4^{+}, CO_3^{2-}, \mathrm{Na}^{+}, NO_3^{-}, \mathrm{Cl}^{-}, \mathrm{Mg}^{2+}$. Có bao nhiêu ion thuộc loại anion?
	\shortans{}
	\loigiai{}
\end{bt}

%%%=============BT_11=============%%%
\begin{bt}
	Cho các ion sau: $\mathrm{Na}^{+}, \mathrm{Mg}^{2+}, \mathrm{Ca}^{2+}, F^{-}, \mathrm{Al}^{3+}, O^{2-}, N^{3-}$. Có bao nhiêu ion có cấu hình electron của khí hiếm neon?
	\shortans{}
	\loigiai{}
\end{bt}

%%%=============BT_12=============%%%
\begin{bt}
	Cho các hợp chất sau: $\mathrm{NaCl}, \mathrm{MgO}, \mathrm{KBr}, \mathrm{CaF}_2, CH_4, HNO_3$. Có bao nhiêu hợp chất chứa liên kết ion trong phân tử?
	\shortans{}
	\loigiai{}
\end{bt}

%%%=============BT_13=============%%%
\begin{bt}
	Cho các hợp chất sau: $\mathrm{MgCl}_2, \mathrm{CaO}, \mathrm{HBr}, NH_4NO_3, \mathrm{CCl}_4, \mathrm{PCl}_5$. Có bao nhiêu hợp chất chứa liên kết ion trong phân tử?
	\shortans{}
	\loigiai{}
\end{bt}
\Closesolutionfile{ansbt}
\Closesolutionfile{ansbth}
%\bangdapanSA{AnsBT-C03_B02_LKION.tex}

	\section{Liên kết cộng hóa trị}
\begin{Muctieu}
	\begin{itemize}
		\item Trình bày được khái niệm và lấy được ví dụ về liên kết cộng hoá trị (liên kết đơn, đôi, ba) khi áp dụng quy tắc octet.
		\item Viết được công thức Lewis của một số chất đơn giản.
		\item Trình bày được khái niệm về liên kết cho - nhận.
		\item Phân biệt được các loại liên kết (liên kết cộng hoá trị không phân cực, phân cực, liên kết ion) dựa theo độ âm điện.
		\item Giải thích được sự hình thành liên kết $\sigma$ và liên kết $\pi$ qua sự xen phủ $AO$.
		\item Trình bày được khái niệm năng lượng liên kết (cộng hoá trị).
		\item Lắp ráp được mô hình một số phân tử có liên kết cộng hoá trị.
	\end{itemize}
\end{Muctieu}
\begin{kd}
	Nguyên tử hydrogen và chlorine dễ dàng kết hợp để tạo thành phân tử hydrogen chloride $(HCl)$, liên kết trong trường hợp này có gì khác so với liên kết ion trong phân tử sodium chloride ($NaCl$) ?
\end{kd}
\subsection{Nội dung bài học}
\subsubsection{Sự tạo thành liên kết cộng hóa trị}
	\Noibat{Tìm hiểu sự hình thành liên kết đơn, đôi, ba trong một số phân tử}
	\Noibat[\maunhan][][\faStar][]{Sự hình thành liên kết đơn, liên kết cho - nhận}
	\begin{center}
		\schemestart
			\chemfig{\charge{[.radius=0.2ex]0:2pt=\.}{H}}
			\+
			\chemfig{\charge{[.radius=0.2ex]0:2pt=\:,-90:2pt=\:,90:2pt=\:,180:2pt=\.}{Cl}}
			\arrow(c1.east--c2.mid west){->}[,0.9,,-stealth]
			\chemname{			\chemfig{H-[,0.52,,,draw=none]\charge{[.radius=0.2ex]0:2pt=\:,-90:2pt=\:,90:2pt=\:,180:2pt=\:}{Cl}}}{Công thức electron}
		\schemestop 
		\hspace{1.5cm};\hspace{1.5cm}
		\chemname{\chemfig{H-[,0.7]\charge{[.radius=0.2ex]0:2pt=\:,-90:2pt=\:,90:2pt=\:}{Cl}}}{Công thức Lewis} \hspace{1.5cm};\hspace{1.5cm} \chemname{\chemfig{H-[,0.7]Cl}}{Công thức cấu tạo}
		\captionof{figure}{Sự tạo thành liên kết đơn trong phân tử HCl}
	\end{center}
	%%%Liên kết cho - nhận
	\vspace*{0.25cm}
	\begin{center}
		\schemestart
		\chemfig{H-[,0.52,,,draw=none]\charge{[.radius=0.2ex]0:2pt=\:,90:2pt=\:,180:2pt=\:,270:2pt=\:}{N}(-[:-90,0.52,,,draw=none]H)-[:90,0.52,,,draw=none]H}
		\+
		\chemfig{H^+}
		\arrow{->}[,0.9,,-stealth]
		\chembelow[1.2cm]{\khungion{\chemfig{H-[,0.52,,,draw=none]\charge{[.radius=0.2ex]0:2pt=\:,90:2pt=\:,180:2pt=\:,270:2pt=\:}{N}(-[:-90,0.52,,,draw=none]H)(-[:90,0.52,,,draw=none]H)-[,0.52,,,draw=none]H}}}{\text{Công thức electron}}
		\hspace{1.5cm};\hspace{1.5cm}
		\chembelow[1.2cm]{\khungion{\chemfig{H-[,0.55]N(-[:-90,0.55]H)(-[:90,0.55]H)-[,0.55,,,-stealth]H}}}{\text{Công thức cấu tạo}}
		\schemestop
		\vspace*{1.2cm}
		\captionof{figure}{Sự tạo thành liên kết cho - nhận trong ion ${NH_4}^+$}
	\end{center}
	\begin{luuy}
		\indam[\maunhan]{Liên kết cho - nhận} là một trường hợp đặc biệt của liên kết cộng hoá trị, trong đó cặp electron chung chỉ do một nguyên tử đóng góp.
	\end{luuy}
	\Noibat[\maunhan][][\faStar][]{Sự hình thành liên kết đôi}
		%%%Phân tử O2
		\begin{center}
			\schemestart
			\chemfig{\charge{[.radius=0.2ex]0:2pt=\:,180:2pt=\:,90:2pt=\:}{O}}
			\+
			\chemfig{\charge{[.radius=0.2ex]0:2pt=\:,180:2pt=\:,90:2pt=\:}{O}}
			\arrow(.east--.mid west){->}[,0.9,,-stealth]
			\chemname{\chemfig{\charge{[.radius=0.2ex]0:2pt=\:,180:2pt=\:,90:2pt=\:}{O}-[,0.58,,,draw=none]\charge{[.radius=0.2ex]0:2pt=\:,180:2pt=\:,90:2pt=\:}{O}}}{Công thức electron}
			\schemestop 
			\hspace{1.5cm};\hspace{1.5cm}
			\chemname{\chemfig{\charge{[.radius=0.2ex]90:2pt=\:,180:2pt=\:}{O}=[,0.58]\charge{[.radius=0.2ex]0:2pt=\:,90:2pt=\:}{O}}}{Công thức Lewis} \hspace{1.5cm};\hspace{1.5cm}
			 \chemname{\chemfig{O=[,0.58]O}}{Công thức cấu tạo}
			\captionof{figure}{Sự tạo thành liên kết đôi trong phân tử $O_2$}
		\end{center}
		%%%Phân tử CO2
		\vspace{0.25cm}
		\begin{center}
			\schemestart
			\chemfig{\charge{[.radius=0.2ex]0:2pt=\.,180:2pt=\.,90:2pt=\:,-90:2pt=\:}{O}}
			\+
			\chemfig{\charge{[.radius=0.2ex]0:2pt=\.,180:2pt=\.,90:2pt=\:}{C}}
			\+
			\chemfig{\charge{[.radius=0.2ex]0:2pt=\.,180:2pt=\.,90:2pt=\:,-90:2pt=\:}{O}}
			\arrow(.east--.mid west){->}[,0.9,,-stealth]
			\chemname{\chemfig{\charge{[.radius=0.2ex]0:2pt=\:,180:2pt=\:,90:2pt=\:}{O}=[,0.58,,,draw=none]\charge{[.radius=0.2ex]0:3pt=\:,180:3pt=\:}{C}=[,0.58,,,draw=none]\charge{[.radius=0.2ex]0:2pt=\:,180:2pt=\:,90:2pt=\:}{O}}}{Công thức electron}
			\schemestop 
			\hspace{1.5cm};\hspace{1.5cm}
			\chemname{\chemfig{\charge{[.radius=0.2ex]180:2pt=\:,90:2pt=\:}{O}=[,0.58]C=[,0.58]\charge{[.radius=0.2ex]0:2pt=\:,90:2pt=\:}{O}}}{Công thức Lewis} \hspace{1.5cm};\hspace{1.5cm}
			\chemname{\chemfig{O=[,0.58]C=[,0.58]O}}{Công thức cấu tạo}
			\captionof{figure}{Sự tạo thành liên kết đôi trong phân tử $CO_2$}
		\end{center}
	\Noibat[\maunhan][][\faStar][]{Sự hình thành liên kết ba}
	%%%Phân tử N2
	\begin{center}
		\schemestart
		\chemfig{\charge{[.radius=0.2ex]0:2pt=\.,180:2pt=\:,90:2pt=\.,-90:2pt=\.}{N}}
		\+
		\chemfig{\charge{[.radius=0.2ex]0:2pt=\:,180:2pt=\.,90:2pt=\.,-90:2pt=\.}{N}}
		\arrow(.east--.mid west){->[][][1.2pt]}[,0.9,,-stealth]
		\chemname{\chemfig{\charge{[.radius=0.2ex]180:2pt=\:,0:2pt=\.,23:2.20pt=\.,-23:2.20pt=\.}{N}-[,0.58,,,draw=none]\charge{[.radius=0.2ex]0:2pt=\:,180:2pt=\.,157:2.20pt=\.,-157:2.20pt=\.}{N}}}{Công thức electron}
		\schemestop 
		\hspace{1.5cm};\hspace{1.5cm}
		\chemname{\chemfig{\charge{[.radius=0.2ex]180:2pt=\:}{N}~[,0.58]\charge{[.radius=0.2ex]0:2pt=\:}{N}}}{Công thức Lewis} \hspace{1.5cm};\hspace{1.5cm}
		\chemname{\chemfig{N~[,0.58]N}}{Công thức cấu tạo}
		\captionof{figure}{Sự tạo thành liên kết đôi trong phân tử $N_2$}
	\end{center}
	\Noibat{Tìm hiểu cách viết công thức Lewis}
	\begin{hopdongian}
		\indam[\maunhan]{Công thức Lewis} của một phân tử được xây dựng từ công thức electron của phân tử, trong đó mỗi cặp electron chung giữa hai nguyên tử tham gia liên kết được thay bằng một gạch nối "-".
	\end{hopdongian}
	\begin{phuongphap}
		\begin{itemize}
			\item  \indam{Bước 1.} Xác định tổng số electron hóa trị bằng cách cộng số nhóm của tất cả các nguyên tử trong phân tử.
			\item  \indam{Bước 2.} Xác định nguyên tử trung tâm. Đây thường là nguyên tử có độ âm điện nhỏ nhất.
			\item  \indam{Bước 3.} Nối các nguyên tử bằng liên kết đơn.
			\item  \indam{Bước 4.} Hoàn thành octet cho các nguyên tử liên kết với nguyên tử trung tâm. Lưu ý rằng hydro chỉ cần 2 electron để hoàn thành octet.
			\item  \indam{Bước 5.} Đặt bất kỳ electron còn lại nào lên nguyên tử trung tâm.
			\item  \indam{Bước 6.} Nếu nguyên tử trung tâm không có octet, hãy thử tạo liên kết đôi hoặc liên kết ba.
			\item  \indam{Bước 7.} Tính điện tích hình thức trên mỗi nguyên tử.Điện tích hình thức là hiệu số giữa số electron hóa trị mà một nguyên tử có ở trạng thái tự do và số electron mà nó "sở hữu" trong cấu trúc Lewis. Mục tiêu là có điện tích hình thức bằng 0 trên mỗi nguyên tử, hoặc càng gần 0 càng tốt.
		\end{itemize}
	\end{phuongphap}
	%%%==============Vidu1==============%%%
	\begin{vd}[Công thức lewis]
		Viết công thức Lewis của phân tử $CO_2$.
		\begin{itemize}
			\item \indam{Bước 1:} Tổng số electron hoá trị của phân tử $CO_2$ là:
			\[4+6\times 2=16\]
			\item  \indam{Bước 2:} Sơ đồ khung biểu diễn liên kết của phân tử $CO_2$:
			\[\mathrm{O}-\mathrm{C}-\mathrm{O}\]
			\item \indam{Bước 3:} Số electron hoá trị chưa tham gia liên kết trong sơ đồ là:
			\[16-2\times 2=12\]
			\item \indam{Bước 4,5,6:}Hoàn thiện octet cho các nguyên tử có độ âm điện lớn hơn trong sơ đồ:
			Số electron hoá trị còn lại: $12-6\times 2=0$.
			Nguyên tử trung tâm C có 4 electron hoá trị, chưa đạt octet.
			\item \indam{Bước 7:} Vì C chưa đạt octet, cẩn chuyển một cặp electron của mỗi nguyên tử oxygen thành cặp electron chung giữa C và O để C đạt octet. Công thức Lewis của phân tử $CO_2$ thu được là:
		\end{itemize}
		\[\chemfig{\charge{[.radius=0.2ex]180:2pt=\:,90:2pt=\:}{O}=[,0.58]C=[,0.58]\charge{[.radius=0.2ex]0:2pt=\:,90:2pt=\:}{O}}\]
		\end{vd}
	%%%==============HetVidu1==============%%%
\subsubsection{Độ âm điện và liên kết hóa học}
\Noibat[\maunhan][][\faStar][]{Phân biệt liên kết cộng hoá trị phân cực và không phân cực}

\begin{ghinho}
	\indam[\maunhan]{Liên kết cộng hoá trị không phân cực} là liên kết cộng hoá trị trong đó \indam[\maunhan]{cặp electron chung không lệch} về phía nguyên tử nào.
	
	\indam[\maunhan]{Liên kết cộng hoá trị phân cực} là liên kết cộng hoá trị trong đó \indam[\maunhan]{cặp electron chung lệch} về phía nguyên tử có \indam[\maunhan]{độ âm điện lớn hơn}.
\end{ghinho}

\Noibat[\maunhan][][\faStar][]{Phân biệt loại liên kết trong phân tử dựa trên giá trị hiệu độ âm điện}

\begin{center}
	\begin{tabular}{|l|l|}
	\hline \multicolumn{1}{|c|}{\textbf{Hiệu độ âm điện $(\Delta \chi)$}}  &  \multicolumn{1}{c|}{\textbf{Loại liên kết}}  \\
	\hline $0 \leq \Delta \chi<0,4$ & Cộng hoá trị không phân cực \\
	\hline $0,4 \leq \Delta \chi<1,7$ & Cộng hoá trị phân cực \\
	\hline$\Delta \chi \geq 1,7$ & lon \\
	\hline
	\end{tabular}
	\captionof{table}{Hiệu độ âm điện $(\Delta \chi)$ và loại liên kết tương ứng}
\end{center}
\subsubsection{Mô tả liên kết cộng hóa trị bằng sự xen phủ orbital nguyên tử}
	\Noibat{Sự xen phủ trục các orbital nguyên tử tạo liên kết $\sigma$ (sigma)}
		\Noibat[\maunhan][][\faStar][0.5]{Xen phủ s - s}
			%%%============= Sự xen phủ AO S-S =====================%%%
			\begin{center}
				\begin{tikzpicture}[declare function={d=0.9cm;h=.05*d;},node font=\bfseries\sffamily]
						\tikzstyle{linestyle} = [line width=2pt,\maunhan!80,->,>=stealth]
						\tikzstyle{myshapestyle} = [line width=1pt,ball color = \maunhan!90,opacity=.65]
						\draw[\maunhan,dashed,line width=1pt] ([xshift=-d]0,0)--([xshift=d]0,0);
						\fill[myshapestyle] (0,0) circle (.75*d);
   						\node at (1.5*d,0) (plus) {\color{\maunhan}\LARGE\bfseries\sffamily+};
						\draw[\maunhan,dashed,line width=1pt] ([xshift=-d]3*d,0)--([xshift=d]3*d,0);
						\fill[myshapestyle] (3*d,0) circle (.75*d);
						\draw[linestyle] (4.5*d,0)--(6*d,0);
						\draw[\maunhan,dashed,line width=1pt] ([xshift=-1.5*d]8*d,0)--([xshift=1.5*d]8*d,0);
						\fill[myshapestyle] (7.5*d,0) circle (.75*d);
						\fill[myshapestyle] (8.5*d,0) circle (.75*d);
						\node at ([shift={(-90:1.2*d)}]0,0) {AO s};
						\node at ([shift={(-90:1.2*d)}]3*d,0) {AO s};
						\node at ([shift={(-90:1.2*d)}]8.0*d,0) {Xen phủ trục s-s};
					\end{tikzpicture}
			\end{center}
		\Noibat[\maunhan][][\faStar][0.5]{Xen phủ s - p}
		
		\begin{tikzpicture}[declare function={d=1.8cm;r=.55*d;h=.125*d;R=.36*d;k=0.65*d;}]
			\tikzstyle{linestyle} = [line width=1pt,\maunhan!80,dashed]
			\tikzstyle{myshapestyle} = [line width=1pt,opacity=.90,ball color =\mauphu!90]
			\tikzset{
				pics/.cd,
				AOs/.style args={#1/#2}{code={%
					\if\relax\detokenize{#1}\relax
					\def\ballcolor{red}
					\else
					\def\ballcolor{#1}
					\fi,
					\if\relax\detokenize{#2}\relax
					\def\opacity{0.8}
					\else
					\def\opacity{#2}
					\fi
					\draw[linestyle] ([xshift=-1.2*R]0*d,0)--([xshift=1.2*R]0*d,0);
					\fill[myshapestyle,ball color = \ballcolor,opacity=\opacity] (0*d,0) circle (R);
					}
				},
				AOp/.style={code={%
					\draw[linestyle] (0,{-d - h})--(0,{d + h});
					\path[myshapestyle,pic actions] (0,0)..controls +(0:{.25*r}) and +(0:r)..(0,d)--
					(0,d)..controls +(180:r) and +(180:{.25*r})..(0,0);
					\path[myshapestyle,pic actions] (0,-d)..controls +(180:r) and +(180:{.25*r})..(0,0)--
					(0,0)..controls +(0:{.25*r}) and +(0:r)..(0,-d);
					} 
				}
			}
			\path (0,0) coordinate (A)
			(1.5*k,0) coordinate (B)
			(4*k,0) coordinate (C)
			(10*k,0) coordinate (D)
			;
			\pic [local bounding box=AOsa] at (A) {AOs={\maunhan}/{}};
			\node(plus) at (B) {\color{\maunhan}\LARGE\bfseries\sffamily +};
			\pic [local bounding box=AOPy,rotate= 90] at (C) {AOp};
			\pic [local bounding box=AOPz,rotate= 90] at (D) {AOp};
			\pic [local bounding box=AOsb,opacity=0.3] at ([xshift=-1.1*d]D) {AOs={\maunhan}/{}};
			\path[draw,->,-latex,line width=.035*d,\maunhan!80] ([xshift=0.25cm]AOPy.east)--([xshift=-0.25cm]AOsb.west);
			\node [font=\bfseries\sffamily,shift={(-90:0.75*d)} ]at (A){AO s};
			\node [font=\bfseries\sffamily,shift={(-90:0.75*d)} ]at (C){AO p};
			\node [font=\bfseries\sffamily,shift={(-90:0.75*d)} ]at ($(AOsb.west)!0.5!(AOPz.east)$){Xen phủ trục s-p};
		\end{tikzpicture}
		\Noibat[\maunhan][][\faStar][0.5]{Xen phủ p - p}
		%%%============= Sự xen phủ AO p-p =====================%%%
		\begin{center}
			\begin{tikzpicture}[declare function={d=1.5cm;r=.55*d;h=0.125*d;},node font=\bfseries\sffamily]
				\tikzstyle{linestyle} = [->,>=stealth,line width=1pt,\maunhan!80]
				\tikzstyle{myshapestyle} = [line width=1pt,ball color = \maudam!90,opacity=.90]
				\draw[line width=1pt,\maunhan!80] ([xshift=-1.1*d]0*d,0)--([xshift=1.1*d]0*d,0);
				\path[myshapestyle] (-d,0)..controls +(90:r) and +(90:{.25*r})..(0,0)--
				(0,0)..controls +(-90:{.25*r}) and +(-90:r)..(-d,0);
				\path[myshapestyle] (d,0)..controls +(90:r) and +(90:{.25*r})..(0,0)--
				(0,0)..controls +(-90:{.25*r}) and +(-90:r)..(d,0);
				%%%==================================%%%
				\draw[line width=1pt,\maunhan!80] ([xshift=-1.1*d]3*d,0)--([xshift=1.1*d]3*d,0);
				\path[myshapestyle] (2*d,0)..controls +(90:r) and +(90:{.25*r})..(3*d,0)--
				(3*d,0)..controls +(-90:{.25*r}) and +(-90:r)..(2*d,0);
				\path[myshapestyle] (4*d,0)..controls +(90:r) and +(90:{.25*r})..(3*d,0)--
				(3*d,0)..controls +(-90:{.25*r}) and +(-90:r)..(4*d,0);
				%%%==================================%%%
				\draw[line width=1pt,\maunhan!80] ([xshift=-2.05*d]7.875*d,0)--([xshift=2.05*d]7.875*d,0);
				\path[myshapestyle] (6*d,0)..controls +(90:r) and +(90:{.25*r})..(7*d,0)--
				(7*d,0)..controls +(-90:{.25*r}) and +(-90:r)..(6*d,0);
				\path[myshapestyle] (8*d,0)..controls +(90:r) and +(90:{.25*r})..(7*d,0)--
				(7*d,0)..controls +(-90:{.25*r}) and +(-90:r)..(8*d,0);
				%%%==================================%%%
				\path[myshapestyle] (7.75*d,0)..controls +(90:r) and +(90:{.25*r})..(8.75*d,0)--
				(8.75*d,0)..controls +(-90:{.25*r}) and +(-90:r)..(7.75*d,0);
				\path[myshapestyle] (9.75*d,0)..controls +(90:r) and +(90:{.25*r})..(8.75*d,0)--
				(8.75*d,0)..controls +(-90:{.25*r}) and +(-90:r)..(9.75*d,0);
				\draw[linestyle,line width=0.05*d] (4.5*d,0)--(5.5*d,0);
				\node at (1.5*d,0) {\color{\maunhan}\LARGE\bfseries\sffamily +};
				\node at ([shift=(-90:.55*d)]0*d,0) {AO p};
				\node at ([shift=(-90:.55*d)]3*d,0) {AO p};
				\node at ([shift=(-90:.55*d)]7.875*d,0) {Xen phủ trục p-p};
			\end{tikzpicture}
		\end{center}
	\Noibat{Sự xen phủ bên các orbital nguyên tử tạo liên kết $\pi$ (pi)}
	%
	%%%%=============Sự xen phủ bên=====================%%%
	\begin{center}
			\begin{tikzpicture}[declare function={d=2cm;r=.55*d;h=.125*d;R=.36*d;k=0.65*d;}]
			\tikzstyle{linestyle} = [line width=1pt,\maunhan]
			\tikzstyle{myshapestyle} = [line width=1pt,ball color = \maudam]
			\tikzset{
				pics/.cd,
					AOs/.style={code={
						\draw[linestyle] ([xshift=-1.2*R]0*d,0)--([xshift=1.2*R]0*d,0);
						\path[myshapestyle,pic actions] (0*d,0) circle (R);
					}
				},
					AOP/.style={code={
						\draw[linestyle] (0,{-d - h})--(0,{d + h});
						\path[myshapestyle,pic actions] (0,0)..controls +(0:{.25*r}) and +(0:r)..(0,d)--
						(0,d)..controls +(180:r) and +(180:{.25*r})..(0,0);
						\path[myshapestyle,pic actions] (0,-d)..controls +(180:r) and +(180:{.25*r})..(0,0)--
						(0,0)..controls +(0:{.25*r}) and +(0:r)..(0,-d);
					}
				}
			}
			\path (0,0) coordinate (A)
			(1.5*k,0) coordinate (B)
			(3*k,0) coordinate (C)
			(7*k,0) coordinate (D)
			;
			\pic [local bounding box=AOPx] at (A) {AOP};
			\node(plus) at (B) {\color{\maunhan}\LARGE\bfseries\sffamily +};
			\pic [local bounding box=AOPy] at (C) {AOP};
			\pic [local bounding box=AOPxH,opacity=0.9] at (D) {AOP};
			\pic [local bounding box=AOPyH,opacity=0.9] at ([xshift=0.355*d]D) {AOP};
			\path[draw,->,-latex,line width=.035*d,\maunhan!80] (AOPy)--(AOPxH);
			\node [font=\bfseries\sffamily,shift={(-90:1.3*d)} ]at (A){AO p};
			\node [font=\bfseries\sffamily,shift={(-90:1.3*d)} ]at (C){AO p};
			\node [font=\bfseries\sffamily,shift={(-90:1.3*d)} ]at ($(AOPxH)!0.5!(AOPyH)$){Xen phủ bên p-p};
		\end{tikzpicture}
	\end{center}
	\begin{ghinho}
		\begin{itemize}
			\item  \indam[\maunhan]{Liên kết $\sigma$} là loại liên kết cộng hoá trị được hình thành do sự \indam[\maunhan]{xen phủ trục} của hai orbital. Vùng xen phủ nẳm trên đường nối tâm hai nguyên tử.
			\item  \indam[\maunhan]{Liên kết $\pi$} là loại liên kết cộng hoá trị được hình thành do sự \indam[\maunhan]{xen phủ bên} của hai orbital. Vùng xen phủ nằm hai bên đường nối tâm hai nguyên tử.
			\item Do mức độ xen phủ trục lớn hơn mức độ xen phủ bên nên \indam[\maunhan]{liên kết $\sigma$ bền hơn liên kết $\pi$}
		\end{itemize}
	\end{ghinho}
	\begin{vidu}Giải thích sự hình thành phân tử $O_2$ bằng sự xen phủ
		%%%===============GIẢI THÍCH SỤ HÌNH THÀNH PHÂN TỬ OXI================%%%
		\par\begin{tikzpicture}[declare function={d=2.5cm;r=.55*d;h=.125*d;R=.85*d;k=.035*d;},node distance= k and k]
			\node (AOPX){%
				\begin{tikzpicture}[scale=.5]
					\tikzstyle{linestyle} = [->,>=stealth,line width=.02*d,\maunhan!80]
					\tikzstyle{myshapestyle} = [ball color = \maudam!90,opacity=.90]
					\draw[line width=.02*d,\maunhan!80] ({-1*d - h},0)--({1*d + h},0);
					\draw[line width=.02*d,\maunhan!80] (0*R,{-d - h})--(0*R,{d + h});
					\path[myshapestyle] (-d,0)..controls +(90:r) and +(90:{.25*r})..(0,0)--
					(0,0)..controls +(-90:{.25*r}) and +(-90:r)..(-d,0);
					\path[myshapestyle] (d,0)..controls +(90:r) and +(90:{.25*r})..(0,0)--
					(0,0)..controls +(-90:{.25*r}) and +(-90:r)..(d,0);
					%%============================================================%%
					\path[myshapestyle] (0*R,0*d)..controls +(0:{.25*r}) and +(0:r)..(0*R,d)--
					(0*R,d)..controls +(180:r) and +(180:{.25*r})..(0*R,0);
					\path[myshapestyle] (0*R,-d)..controls +(180:r) and +(180:{.25*r})..(0*R,0*d)--
					(0*R,0*d)..controls +(0:{.25*r}) and +(0:r)..(0*R,-d);
					\end{tikzpicture}
				} ;
			\node [right=of AOPX] (plus) {\color{\maunhan}\LARGE\bfseries\sffamily +};
			\node [right=of plus] (AOPY){%
				\begin{tikzpicture}[scale=.5]
						\tikzstyle{linestyle} = [->,>=stealth,line width=.02*d,\maunhan!80]
						\tikzstyle{myshapestyle} = [ball color = \maudam!90,opacity=.90]
						\draw[line width=.02*d,\maunhan!80] ({-1*d - h},0)--({1*d + h},0);
						\draw[line width=.02*d,\maunhan!80] (0*R,{-d - h})--(0*R,{d + h});
						\path[myshapestyle] (-d,0)..controls +(90:r) and +(90:{.25*r})..(0,0)--
						(0,0)..controls +(-90:{.25*r}) and +(-90:r)..(-d,0);
						\path[myshapestyle] (d,0)..controls +(90:r) and +(90:{.25*r})..(0,0)--
						(0,0)..controls +(-90:{.25*r}) and +(-90:r)..(d,0);
						%%============================================================%%
						\path[myshapestyle] (0*R,0*d)..controls +(0:{.25*r}) and +(0:r)..(0*R,d)--
						(0*R,d)..controls +(180:r) and +(180:{.25*r})..(0*R,0);
						\path[myshapestyle] (0*R,-d)..controls +(180:r) and +(180:{.25*r})..(0*R,0*d)--
						(0*R,0*d)..controls +(0:{.25*r}) and +(0:r)..(0*R,-d);
				\end{tikzpicture}
			};
			\node[right=of AOPY](muiten){\tikz{\path[draw,->,-latex,line width=.075*d,\maunhan!80] (0,0)--(3,0);}};
			\node [right=of muiten] (AOXENPHU){%
				\begin{tikzpicture}[scale=.5]
					\tikzstyle{linestyle} = [->,>=stealth,line width=.02*d,\maunhan!80]
					\tikzstyle{myshapestyle} = [ball color = \maudam!90,opacity=.90]
					\draw[line width=.02*d,\maunhan!80] ({-d - h},0)--({2.8*d + h},0);
					\draw[line width=.02*d,\maunhan!80] (0*R,{-d - h})--(0*R,{d + h});
					\path[myshapestyle] ([shift={(.49cm,-.085cm)}]0,1*d)..controls +(-30:.49*d) and +(-150:.49*d)..([shift={(-.49cm,-.085cm)}]1.8*d,1*d)--
					(1.8*d,d)..controls +(180:r) and +(180:{.25*r})..(1.8*d,0)--
					(1.8*d,0*d)..controls +(135:.82*d) and +(45:.82*d)..(0,0*d)--
					(0*R,0*d)..controls +(0:{.25*r}) and +(0:r)..(0*R,d)--cycle;
					\begin{scope}[transform canvas={yscale=-1}]
						\path[myshapestyle] ([shift={(.49cm,-.085cm)}]0,1*d)..controls +(-30:.49*d) and +(-150:.49*d)..([shift={(-.49cm,-.085cm)}]1.8*d,1*d)--
						(1.8*d,d)..controls +(180:r) and +(180:{.25*r})..(1.8*d,0)--
						(1.8*d,0*d)..controls +(135:.82*d) and +(45:.82*d)..(0,0*d)--
						(0*R,0*d)..controls +(0:{.25*r}) and +(0:r)..(0*R,d)--cycle;
					\end{scope}
					\path[myshapestyle] (-d,0)..controls +(90:r) and +(90:{.25*r})..(0,0)--
					(0,0)..controls +(-90:{.25*r}) and +(-90:r)..(-d,0);
					\path[myshapestyle] (d,0)..controls +(90:r) and +(90:{.25*r})..(0,0)--
					(0,0)..controls +(-90:{.25*r}) and +(-90:r)..(d,0);
					%%============================================================%%
					\path[myshapestyle] (0*R,0*d)..controls +(0:{.25*r}) and +(0:r)..(0*R,d)--
					(0*R,d)..controls +(180:r) and +(180:{.25*r})..(0*R,0);
					\path[myshapestyle] (0*R,-d)..controls +(180:r) and +(180:{.25*r})..(0*R,0*d)--
					(0*R,0*d)..controls +(0:{.25*r}) and +(0:r)..(0*R,-d);
					%%%=============================================================%%%
					\path[myshapestyle] (.8*d,0)..controls +(90:r) and +(90:{.25*r})..(1.8*d,0)--
					(1.8*d,0)..controls +(-90:{.25*r}) and +(-90:r)..(.8*d,0);
					\path[myshapestyle] (2.8*d,0)..controls +(90:r) and +(90:{.25*r})..(1.8*d,0)--
					(1.8*d,0)..controls +(-90:{.25*r}) and +(-90:r)..(2.8*d,0);
					%%==============================================%%
					\draw[line width=.02*d,\maunhan!80] (1.8*d,{-d - h})--(1.8*d,{d + h});
					\path[myshapestyle] (1.8*d,0*d)..controls +(0:{.25*r}) and +(0:r)..(1.8*d,d)--
					(1.8*d,d)..controls +(180:r) and +(180:{.25*r})..(1.8*d,0);
					\path[myshapestyle] (1.8*d,-d)..controls +(180:r) and +(180:{.25*r})..(1.8*d,0*d)--
					(1.8*d,0*d)..controls +(0:{.25*r}) and +(0:r)..(1.8*d,-d);
					%%================================================%%
				\end{tikzpicture}
				} ;
			\node[below=of AOPX,yshift=.15cm]{\color{\maunhan}\bfseries\sffamily O};
			\node[below=of AOPY,yshift=.15cm]{\color{\maunhan}\bfseries\sffamily O};
			\node[below=of AOXENPHU,yshift=.15cm]{\color{\maunhan}\bfseries\sffamily phân tử O$\mathbf{_2}$};
			\node[above=of AOXENPHU,yshift=-1.2cm]{\color{\maunhan}\LARGE $\mathbf{\pi}$};
			\node[below=of AOXENPHU,yshift=2.0cm]{\color{\maunhan}\LARGE $\mathbf{\sigma}$};
		\end{tikzpicture}
	\end{vidu}
	%%%==============VD_1==============%%%
	\begin{vidu}
		Tổng năng lượng liên kết trong phân tử $\mathrm{CH}_4$ là $1660 \mathrm{kJ} / \mathrm{mol}$.
		$$ \mathrm{CH}_4(\mathrm{g}) \rightarrow \mathrm{C}(\mathrm{g})+4 \mathrm{H}(\mathrm{g}) \quad \mathrm{E}_{\mathrm{b}}=1660\mathrm{kJ}/\mathrm{mol}$$. Do đó, năng lượng liên kết trung bình của một liên kết $\mathrm{C}-\mathrm{H}$ là $\dfrac{1660}{4}=415 \mathrm{kJ} / \mathrm{mol}$.
	\end{vidu}
	%%%==============HetVD_1==============%%%
\subsubsection{Năng lượng liên kết cộng hóa trị}
\begin{ghinho}
	\indam[\maunhan]{Năng lượng của một liên kết hoá học} là năng lượng cần thiết để  \indam[\maunhan]{phá vỡ 1 mol liên kết đó ở thể khí}, tạo thành các \indam[\maunhan]{nguyên tử ở thể khí}.
	
	Giá trị năng lượng của một liên kết hoá học là thước đo \indam[\maunhan]{độ bền liên kết}.
\end{ghinho}
%%%
\begin{vidu}
	Cho các phương trình phản ứng sau:
	$$
	\begin{array}{ll}
		\mathrm{H}_2(\mathrm{g}) \rightarrow 2\mathrm{H}(\mathrm{g}) & \mathrm{E}_{\mathrm{b}}=432 \mathrm{~kJ}/\mathrm{mol} \quad (1) \\
		\mathrm{N}_2(\mathrm{g}) \rightarrow 2\mathrm{N}(\mathrm{g}) & \mathrm{E}_{\mathrm{b}}=945 \mathrm{~kJ}/\mathrm{mol} \quad (2)
	\end{array}
	$$
	Ta nói năng lượng liên kết ${E}_{b}$ trong phân tử $\mathrm{H}_2$ và $N_2$ lần lượt là $432$ $kJ/mol$ và $945$ $kJ/mol$. Điều này có nghĩa cần cung cấp $432$ kJ và $945$ kJ để lần lượt phá vỡ $1$ mol khí $H_2$ và $1$ mol khí ${N}_2$ thành các nguyên tử ở thể khí.
\end{vidu}
%%%
\begin{ghinho}
	\begin{itemize}
		\item \indam[\maunhan]{Liên kết đơn} luôn luôn là \indam[\maunhan]{liên kết $\sigma$}, được tạo thành từ sự xen phủ trục và \indam[\maunhan]{thường bền vững}
		\item \indam[\maunhan]{Liên kết đôi} gồm  \indam[\maunhan]{1 liên kết $\sigma$} và  \indam[\maunhan]{1 liên kết $\pi$}.
		\item \indam[\maunhan]{Liên kết ba} gồm \indam[\maunhan]{1 liên kết $\sigma$} và \indam[\maunhan]{2 liên kết $\pi$}
	\end{itemize}
\end{ghinho}
\begin{luuy}
	Liên kết giữa hai nguyên tử được thực hiện bởi \indam[\maunhan]{một liên kết $\sigma$} và \indam[\maunhan]{một hay nhiều  liên kết $\pi$} được gọi là \indam[\maunhan]{liên kết bội}
\end{luuy}
\subsection{Bài tập}
\setchemfig{atom sep=2em}
\phan{Bài tập tự luận}
%%==============Bai_BT1==============%%%
\begin{bt}
	Cho giá trị độ âm điện của các nguyên tố: Cl ($3{,}16$), O ($3{,}44$), N ($3{,}04$), H ($2{,}20$), Al ($1{,}61$), Na ($0{,}93$). Xác định kiểu liên kết (liên kết ion? cộng hóa trị không phân cực? cộng hóa trị phân cực?) trong các phân tử sau: $HCl$, $H_2$, $NH_3$, $Na_2O$, $O_2$, $NaCl$, $AlCl_3$.
	\loigiai{
		\begin{tabular}{|c|c|c|}
			\hline
			Phân tử & Hiệu độ âm điện & Kiểu liên kết \\
			\hline
			$HCl$ & $3{,}16-2{,}2=0{,}96< 1{,}7$ & Cộng hóa trị phân cực \\
			\hline
			$H_2$ & $2{,}2-2{,}2=0$ & Cộng hóa trị không phân cực \\
			\hline
			$NH_3$ & $3{,}04-2{,}2=0{,}84< 1{,}7$ & Cộng hóa trị phân cực \\
			\hline
			$Na_2O$ & $3{,}44-0{,}93=2{,}51> 1{,}7$ & Ion \\
			\hline
			$O_2$ & $3{,}44-3{,}44=0$ & Cộng hóa trị không phân cực \\
			\hline
			$NaCl$ & $3{,}16-0{,}93=2{,}23> 1{,}7$ & Ion \\
			\hline
			$AlCl_3$ & $3{,}16-1{,}61=1{,}55< 1{,}7$ & Cộng hóa trị phân cực \\
			\hline
		\end{tabular}
	}
\end{bt}
%%==============HetBai_BT1==============%%%

%%%==============Bai_BT2==============%%%
\begin{bt}
	Dự đoán kiểu liên kết hóa học trong các phân tử sau đây: $Cl_2$, $NH_3$, $KCl$, $O_2$, $NaF$, $CaCl_2$, $HCl$, $MgO$.
	\loigiai{
		\begin{itemize}
			\item Liên kết cộng hóa trị không phân cực: $Cl_2$, $O_2$.
			\item Liên kết cộng hóa trị phân cực: $NH_3$, $HCl$.
			\item Liên kết ion: $KCl$, $NaF$, $CaCl_2$, $MgO$.
		\end{itemize}
	}
\end{bt}
%%%==============HetBai_BT2==============%%%

%%%==============Bai_BT3==============%%%
\begin{bt} Ammonia ($NH_3$) khan (nguyên chất) được bơm vào đất ở dạng khí, là nguồn phân đạm phổ biến ở Bắc Mỹ do giá thành và tuổi thọ tương đối lâu trong đất so với các dạng phân đạm khác. Do tính ổn định của ammonia khan trên đất lạnh, nông dân trồng ngô thường bón ammonia khan vào mùa thu để bắt đầu hoạt động gieo trồng vào mùa xuân. Viết công thức elctron, công thức Lewis và công thức cấu tạo của ammonia.
	\loigiai{%
	\begin{center}
		\begin{tikzpicture}[ampersand replacement=\&]
			\matrix (m) [matrix of nodes,
			nodes={anchor=center,minimum width=4.6cm,align=center,inner sep =5pt,font=\sffamily\bfseries},
			column sep=0pt-\pgflinewidth,
			row sep =0pt-\pgflinewidth,
			nodes in empty cells,
			row 1/.style={minimum height = 0.65cm},
			row 2/.style={minimum height = 2cm},
			]
			{Công thức electron \& Công thức Lewis \& Công thức cấu tạo \\
			\chemfig{H-[,0.85,,,draw=none]\charge{[.radius=0.2ex]0:2pt=\:,180:2pt=\:,90:2pt=\:,-90:2pt=\:}{N}(-[:-90,0.85,,,draw=none]H)-[,0.85,,,draw=none]H}
			\& \chemfig{H-\charge{[.radius=0.2ex]90:2pt=\:}{N}(-[:-90]H)-H}
			\& \chemfig{H-N(-[:-90]H)-H} \\
			};
			% Vẽ đường viền
			\draw[thick,\mycolor] (m-1-1.north west) rectangle (m-2-3.south east);
			\draw[thick,\mycolor] (m-1-1.south west) -- (m-1-3.south east);
			\draw[thick,\mycolor] (m-1-1.north east) -- (m-2-1.south east);
			\draw[thick,\mycolor] (m-1-2.north east) -- (m-2-2.south east);
		\end{tikzpicture}
	\end{center}}
\end{bt}
%%%==============HetBai_BT3==============%%%
%%%==============Bai_BT4==============%%%
\begin{bt}[CTST-SBT] Ozone ($O_3$) là một loại khí có tính oxi hoá mạnh, phân tử gồm ba nguyên tử oxygen. Ozone xuất hiện ở tầng đối lưu và tầng bình lưu của khí quyển. Tuỳ thuộc vào vị trí của ozone trong các tầng trên mà nó ảnh hưởng đến sự sống trên Trái Đất theo các cách tốt, xấu khác nhau. Phân tử ozone có sự hiện diện liên kết cho – nhận. Viết công thức Lewis và công thức cấu tạo của ozone.
	\loigiai{
	\begin{center}
	\begin{tikzpicture}[ampersand replacement=\&]
		\matrix (m) [matrix of nodes,
		nodes={anchor=center,minimum width=4.6cm,align=center,inner sep =5pt,font=\sffamily\bfseries},
		column sep=0pt-\pgflinewidth,
		row sep =0pt-\pgflinewidth,
		nodes in empty cells,
		row 1/.style={minimum height = 0.65cm},
		row 2/.style={minimum height = 2cm},
		]
		{Công thức electron \& Công thức Lewis \& Công thức cấu tạo \\
			\chemfig{\charge{[.radius=0.2ex]0:2pt=\:,90:2pt=\:,180:2pt=\:}{O}-[,0.85,,,draw=none]\charge{[.radius=0.2ex]0:2pt=\:[.style={fill=\maunhan,draw=none}],90:2pt=\:[.style={fill=\maunhan,draw=\maunhan}],180:2pt=\:[.style={fill=\maunhan,draw=\maunhan}]}{O}-[,0.65,,,draw=none]\charge{[.radius=0.2ex]0:2pt=\:,-90:2pt=\:,90:2pt=\:}{O}}
			\& 				\chemfig{\charge{[.radius=0.2ex]90:2pt=\:,180:2pt=\:}{O}=\charge{[.radius=0.2ex]90:2pt=\:[.style={fill=\maunhan,draw=\maunhan}]}{O}-[,1,,,-stealth]\charge{[.radius=0.2ex]0:2pt=\:,-90:2pt=\:,90:2pt=\:}{O}}
			\&
			\chemfig{O=O-[,1,,,-stealth]O} \\
		};
		% Vẽ đường viền
		\draw[thick,\mycolor] (m-1-1.north west) rectangle (m-2-3.south east);
		\draw[thick,\mycolor] (m-1-1.south west) -- (m-1-3.south east);
		\draw[thick,\mycolor] (m-1-1.north east) -- (m-2-1.south east);
		\draw[thick,\mycolor] (m-1-2.north east) -- (m-2-2.south east);
	\end{tikzpicture}
	\end{center}}
\end{bt}
%%%==============HetBai_BT4==============%%%
%%%==============Bai_BT5==============%%%
\begin{bt}
	Viết công thức electron, công thức Lewis và công thức cấu tạo của các phân tử sau:
	\begin{enumerate}
		\item $Cl_2$, $O_2$, $N_2$. 
		\item  $HCl$, $H_2S$, $CH_4$, $C_2H_4$, $C_2H_2$.
		\item $SO_2$, $SO_3$, $HNO_3$, $H_2SO_4$, $H_2CO_3$, $H_3PO_4$. 
		\item $HClO$, $HClO_2$, $HClO_3$, $HClO_4$.
	\end{enumerate}
	\loigiai{
		% Tạo command để vẽ một hàng của bảng
		\NewDocumentCommand{\matrixrow}{O{-14pt}O{1cm}mmmm}{%
			\noindent\begin{tikzpicture}[ampersand replacement=\&]
				\matrix (m) [matrix of nodes,
				nodes={anchor=center,align=center,inner sep=5pt,font=\sffamily},
				column sep=0pt-\pgflinewidth,
				row sep=0pt-\pgflinewidth,
				nodes in empty cells,
				row 1/.style={minimum height = #2},
				column 1/.style={minimum width = 3cm},
				column 2/.style={minimum width = 4.5cm},
				column 3/.style={minimum width = 4.5cm},
				column 4/.style={minimum width = 4.5cm}
				]
				{#3 \& #4 \& #5 \& #6\\};
				% Vẽ đường viền
				\draw[thick,\mycolor] (m-1-1.north west) rectangle (m-1-4.south east);
				\draw[thick,\mycolor] (m-1-1.north east) -- (m-1-1.south east);
				\draw[thick,\mycolor] (m-1-2.north east) -- (m-1-2.south east);
				\draw[thick,\mycolor] (m-1-3.north east) -- (m-1-3.south east);
			\end{tikzpicture}\\[#1]}
		\begin{longtable}{@{}c@{}}
			% Header cho trang đầu
			\begin{tikzpicture}[ampersand replacement=\&]
				\matrix (n) [matrix of nodes,
				nodes={anchor=center,align=center,
				inner sep=5pt,fill=\mycolor!20,
				font=\sffamily\bfseries},
				column sep=0pt-\pgflinewidth,
				row sep=0pt-\pgflinewidth,
				nodes in empty cells,
				row 1/.style={minimum height = 0.85cm},
				column 1/.style={minimum width = 3cm},
				column 2/.style={minimum width = 4.5cm},
				column 3/.style={minimum width = 4.5cm},
				column 4/.style={minimum width = 4.5cm}
				]
				{Phân tử \& Công thức electron \& Công thức Lewis \& Công thức cấu tạo \\};
				\draw[thick,\mycolor] (n-1-1.north west) rectangle (n-1-4.south east);
				\draw[thick,\mycolor] (n-1-1.north east) -- (n-1-1.south east);
				\draw[thick,\mycolor] (n-1-2.north east) -- (n-1-2.south east);
				\draw[thick,\mycolor] (n-1-3.north east) -- (n-1-3.south east);
			\end{tikzpicture}\\[-14pt]
			\endfirsthead
			% Header cho các trang tiếp theo
			\begin{tikzpicture}[ampersand replacement=\&]
				\matrix (n) [matrix of nodes,
				nodes={anchor=center,align=center,
				inner sep=5pt,fill=\mycolor!20,
				font=\sffamily\bfseries},
				column sep=0pt-\pgflinewidth,
				row sep=0pt-\pgflinewidth,
				nodes in empty cells,
				row 1/.style={minimum height = 0.85cm},
				column 1/.style={minimum width = 3cm},
				column 2/.style={minimum width = 4.5cm},
				column 3/.style={minimum width = 4.5cm},
				column 4/.style={minimum width = 4.5cm}
				]
				{Phân tử \& Công thức electron \& Công thức Lewis \& Công thức cấu tạo \\};
				\draw[thick,\mycolor] (n-1-1.north west) rectangle (n-1-4.south east);
				\draw[thick,\mycolor] (n-1-1.north east) -- (n-1-1.south east);
				\draw[thick,\mycolor] (n-1-2.north east) -- (n-1-2.south east);
				\draw[thick,\mycolor] (n-1-3.north east) -- (n-1-3.south east);
			\end{tikzpicture}\\[0pt]
			\endhead
			% Dong 1
			\matrixrow{\chemfig{Cl_2}}{
				\chemfig{\charge{[.radius=0.2ex]0:2pt=\:,90:2pt=\:,180:2pt=\:,-90:2pt=\:}{Cl}-[0,0.85,,,draw=none]\charge{[.radius=0.2ex]0:2pt=\:,90:2pt=\:,-90:2pt=\:}{Cl}}
			}{
				\chemfig{\charge{[.radius=0.2ex]90:2pt=\:,180:2pt=\:,-90:2pt=\:}{Cl}-\charge{[.radius=0.2ex]0:2pt=\:,90:2pt=\:,-90:2pt=\:}{Cl}}
			}{
				\chemfig{Cl-Cl}
			}
			%%Dòng 2
			\matrixrow{\chemfig{O_2}}{
				\chemfig{\charge{[.radius=0.2ex]0:2pt=\:,90:2pt=\:,180:2pt=\:}{O}-[:0,0.85,,,draw=none]\charge{[.radius=0.2ex]0:2pt=\:,90:2pt=\:,180:2pt=\:}{O}}
			}{
				\chemfig{\charge{[.radius=0.2ex]90:2pt=\:,180:2pt=\:}{O}=\charge{[.radius=0.2ex]90:2pt=\:,0:2pt=\:}{O}}
			}{
				\chemfig{O=O}
			}
			%%%Dong 3
			\matrixrow{\chemfig{N_2}}{
				\chemfig{\charge{[.radius=0.2ex]90:2pt=\:,0:2pt=\.,0:2pt=\.[.style={yshift=9pt,fill=black}],0:2pt=\.[.style={yshift=-9pt,fill=black}]}{N}-[:0,0.85,,,draw=none]\charge{[.radius=0.2ex]90:2pt=\:,180:2pt=\.,180:2pt=\.[.style={yshift=9pt,fill=black}],180:2pt=\.[.style={yshift=-9pt,fill=black}]}{N}}
			}{
				\chemfig{\charge{[.radius=0.2ex]90:2pt=\:}{N}~\charge{[.radius=0.2ex]90:2pt=\:}{N}}
			}{
				\chemfig{N~N}
			}
			%%%Dong4
			\matrixrow{\chemfig{HCl}}{
				\chemfig{H-[:0,0.85,,,draw=none]\charge{[.radius=0.2ex]180:2pt=\:,0:2pt=\:,90:2pt=\:,-90:2pt=\:}{Cl}}
			}{
				\chemfig{H-\charge{[.radius=0.2ex]0:2pt=\:,90:2pt=\:,-90:2pt=\:}{Cl}}
			}{
				\chemfig{H-Cl}
			}
			%%%Dong 5
			\matrixrow{\chemfig{H_2S}}{
				\chemfig{H-[:0,0.85,,,draw=none]\charge{[.radius=0.2ex]180:2pt=\:,0:2pt=\:,90:2pt=\:,-90:2pt=\:}{S}-[:0,0.85,,,draw=none]H}
			}{
				\chemfig{H-\charge{[.radius=0.2ex]90:2pt=\:,-90:2pt=\:}{S}-H}
			}{
				\chemfig{H-S-H}
			}
			%%%Dong 6
			\matrixrow[-14pt][2.5cm]{\chemfig{CH_4}}{
				\chemfig{H-[,,,,draw=none]\charge{[.radius=0.2ex]0:2pt=\:,180:2pt=\:,90:2pt=\:,-90:2pt=\:}{C}(-[:-90,,,,draw=none]H)(-[:90,,,,draw=none]H)-[,,,,draw=none]H}
			}{
				\chemfig{H-C(-[:-90]H)(-[:90]H)-H}
			}{
				\chemfig{H-C(-[:-90]H)(-[:90]H)-H}
			}
			%%%Dong 8
			\matrixrow[-14pt][2.5cm]{\chemfig{C_2H_4}}{
				\chemfig{H-[:-60,,,,draw=none]\charge{[.radius=0.2ex]0:2pt=\:,120:2pt=\:,-120:2pt=\:}{C}(-[:-120,,,,draw=none]H)-[,0.85,,,draw=none]\charge{[.radius=0.2ex]180:2pt=\:,60:2pt=\:,-60:2pt=\:}{C}(-[:60,,,,draw=none]H)-[:-60,,,,draw=none]H}
			}{
				\chemfig{H-[:-60]C(-[:-120]H)=C(-[:60]H)-[:-60]H}
			}{
				\chemfig{H-[:-60]C(-[:-120]H)=C(-[:60]H)-[:-60]H}
			}
			%%%Dong 9
			\matrixrow[-14pt][1.2cm]{\chemfig{C_2H_2}}{
				\chemfig{H-[,0.85,,,draw=none]\charge{[.radius=0.2ex]180:2pt=\:,0:2pt=\.[.style={yshift=9pt,fill=black}],0:2pt=\.[.style={yshift=0pt,fill=black}],0:2pt=\.[.style={yshift=-9pt,fill=black}]}{C}-[,0.85,,,draw=none]\charge{[.radius=0.2ex]0:2pt=\:,180:2pt=\.[.style={yshift=9pt,fill=black}],180:2pt=\.[.style={yshift=0pt,fill=black}],180:2pt=\.[.style={yshift=-9pt,fill=black}]}{C}-[,0.85,,,draw=none]H}
			}{
				\chemfig{H-C~C-H}
			}{
				\chemfig{H-C~C-H}
			}
			%%%Dong 10
			\matrixrow[-14pt][2cm]{%
				\chemfig{SO_2}
			}{%
				\chemfig{\charge{[.radius=0.2ex]45:2pt=\:,180:2pt=\:,-90:2pt=\:}{O}-[:45,1,,,draw=none]\charge{[.radius=0.2ex]-45:2pt=\:[.style={fill=\maunhan,draw=\maunhan}],-135:2pt=\:[.style={fill=\maunhan,draw=\maunhan}],90:2pt=\:[.style={fill=\maunhan,draw=\maunhan}]}{S}-[:-45,1,,,draw=none]\charge{[.radius=0.2ex]0:2pt=\:,-90:2pt=\:,180:2pt=\:}{O}}
			}{%
				\chemfig{\charge{[.radius=0.2ex]180:2pt=\:,-90:2pt=\:}{O}=[:45]\charge{[.radius=0.2ex]90:2pt=\:}{S}-[:-45,,,,-stealth]\charge{[.radius=0.2ex]180:2pt=\:,-90:2pt=\:,0:2pt=\:}{O}}
			}{%
				\chemfig{O=[:45]S-[:-45,,,,-stealth]O}
			}	
			%%%Dong 11
			\matrixrow[-14pt][2.2cm]{%
				\chemfig{SO_3}
			}{%
				\chemfig{\charge{[.radius=0.2ex]180:2pt=\:,90:2pt=\:,-90:2pt=\:}{O}-[:30,0.85,,,draw=none]\charge{[.radius=0.2ex]90:2pt=\:[.style={fill=\maunhan,draw=\maunhan}],-30:2pt=\:[.style={fill=\maunhan,draw=\maunhan}],-150:2pt=\:[.style={fill=\maunhan,draw=\maunhan}]}{S}(=[:90,0.85,,,draw=none]\charge{[.radius=0.2ex]0:2pt=\:,180:2pt=\:,-90:2pt=\:}{O})-[:-30,0.85,,,draw=none]\charge{[.radius=0.2ex]0:2pt=\:,90:2pt=\:,-90:2pt=\:}{O}}
			}{%
				\chemfig{\charge{[.radius=0.2ex]180:2pt=\:,90:2pt=\:,-90:2pt=\:}{O}-[:30,,,,<-,>=stealth]S(=[:90]\charge{[.radius=0.2ex]180:2pt=\:,0:2pt=\:}{O})-[:-30,,,,-stealth]\charge{[.radius=0.2ex]0:2pt=\:,90:2pt=\:,-90:2pt=\:}{O}}
			}{%
				\chemfig{O-[:30,,,,<-,>=stealth]S(=[:90]O)-[:-30,,,,-stealth]O}
			}					
			%%%Dong12
			\matrixrow[-14pt][2.2cm]{%
				\chemfig{HNO_3}
			}{%
				\chemfig{\charge{[.radius=0.2ex]0:2pt=\.}{H}-[:0,,,,draw=none]\charge{[.radius=0.2ex]90:2pt=\:,-90:2pt=\:,0:2pt=\.,180:2pt=\.}{O}-[0,,,,draw=none]\charge{[.radius=0.2ex]45:2pt=\:[.style={fill=\maunhan,draw=\maunhan}],-45:1.5pt=\:[.style={fill=\maunhan,draw=\maunhan}],180:2pt=\.[.style={fill=\maunhan,draw=\maunhan}]}{N}(-[:-45,,,,draw=none]\charge{[.radius=0.2ex]90:2pt=\:,0:2pt=\:,-90:2pt=\:}{O})=[:45,1,,,draw=none]\charge{[.radius=0.2ex]90:2pt=\:,0:2pt=\:,-135:1pt=\:}{O}}				
			}{%
				\chemfig{H-\charge{[.radius=0.2ex]90:2pt=\:,-90:2pt=\:}{O}-N(-[:-45,,,,->,>=stealth]\charge{[.radius=0.2ex]90:2pt=\:,0:2pt=\:,-90:2pt=\:}{O})=[:45]\charge{[.radius=0.2ex]90:2pt=\:,0:2pt=\:}{O}}
			}{%
				\chemfig{H-O-N(-[:-45,,,,->,>=stealth]O)=[:45]O}
			}
			%%%Dong13
			\matrixrow[-14pt][2.3cm]{%
				\chemfig{H_2SO_4}
			}{%
				\chemfig{\charge{0:3pt=\.}{H}-[:0,,,,draw=none]\charge{[.radius=0.2ex]90:2pt=\:,-90:2pt=\:,180:2pt=\.,-45:2.5pt=\.}{O}-[:-45,1.2,,,,draw=none]\charge{[.radius=0.2ex]45:4pt=\:[.style={fill=\maunhan,draw=\maunhan}],135:4pt=\.[.style={fill=\maunhan,draw=\maunhan}],-45:4pt=\:[.style={fill=\maunhan,draw=\maunhan}],-135:4pt=\.[.style={fill=\maunhan,draw=\maunhan}]}{S}(-[:45,1.2,,,draw=none]\charge{[.radius=0.2ex]90:2pt=\:,-90:2pt=\:,0:2pt=\:}{O})(-[:-45,1.2,,,draw=none]\charge{[.radius=0.2ex]90:2pt=\:,-90:2pt=\:,0:2pt=\:}{O})-[:-135,1.2,,,draw=none]\charge{[.radius=0.2ex]90:2pt=\:,-90:2pt=\:,180:2pt=\.,45:2pt=\.}{O}-[:180,,,,draw=none]\charge{0:3pt=\.}{H}}			
			}{%
				\chemfig{H-\charge{[.radius=0.2ex]90:2pt=\:,-90:2pt=\:}{O}-[:-45,1.2]S(-[:45,1.2,,,-stealth]\charge{[.radius=0.2ex]90:2pt=\:,-90:2pt=\:,0:2pt=\:}{O})(-[:-45,1.2,,,-stealth]\charge{[.radius=0.2ex]90:2pt=\:,-90:2pt=\:,0:2pt=\:}{O})-[:-135,1.2]\charge{[.radius=0.2ex]90:2pt=\:,-90:2pt=\:}{O}-[:180]H}
			}{%
				\chemfig{H-O-[:-45,1.2]S(-[:45,1.2,,,-stealth]O)(-[:-45,1.2,,,-stealth]O)-[:-135,1.2]O-[:180]H}
			}
			%%%Dong14
			\matrixrow[-14pt][2.3cm]{%
				\chemfig{H_2CO_3}
			}{%
				\chemfig{\charge{[.radius=0.2ex]0:3.5pt=\.}{H}-[:0,,,,draw=none]\charge{[.radius=0.2ex]90:2pt=\:,-90:2pt=\:,180:2pt=\.,-45:2pt=\.}{O}-[:-45,1.2,,,draw=none]\charge{[.radius=0.2ex]0:3.5pt=\:[.style={fill=\maunhan,draw=\maunhan}],135:3.5pt=\.[.style={fill=\maunhan,draw=\maunhan}],-135:3.5pt=\.[.style={fill=\maunhan,draw=\maunhan}]}{C}(=[:0,,,,draw=none]\charge{[.radius=0.2ex]90:2pt=\:,-90:2pt=\:,180:2pt=\:}{O})-[:-135,1.2,,,draw=none]\charge{[.radius=0.2ex]90:2pt=\:,-90:2pt=\:,180:2pt=\.,45:2pt=\.	}{O}-[:180,,,,draw=none]\charge{[.radius=0.2ex]0:3.5pt=\.}{H}}		
			}{%
				\chemfig{H-\charge{[.radius=0.2ex]90:2pt=\:,-90:2pt=\:}{O}-[:-45,1.2]C(=[:0]\charge{[.radius=0.2ex]90:2pt=\:,-90:2pt=\:}{O})-[:-135,1.2]\charge{[.radius=0.2ex]90:2pt=\:,-90:2pt=\:}{O}-[:180]H}
			}{%
				\chemfig{H-O-[:-45,1.2]C(=[:0]O)-[:-135,1.2]O-[:180]H}
			}						
		\end{longtable}	
	 }
\end{bt}
%%%==============HetBai_BT5==============%%%
%%%==============Bai_BT6==============%%%
\begin{bt}[CTST-SBT] Hydrogen sulfide $\left(H_2\mathrm{~S}\right)$ là một chất khí không màu, mùi trứng thối, độc. Theo tài liệu của Cơ quan Quản lí an toàn và sức khoẻ nghề nghiệp Hoa Kì, nồng độ $H_2\mathrm{~S}$ khoảng 100 ppm gây kích thích màng phổi. Nồng độ khoảng $400-700\mathrm{ppm}, H_2\mathrm{~S}$ gây nguy hiểm đến tính mạng chỉ trong 30 phút. Nồng độ trên 800 ppm gây mất ý thức và làm tử vong ngay lập tức.
	\begin{enumerate}
		\item Viết công thức Lewis và công thức cấu tạo của $H_2\mathrm{~S}$.
		\item Em hiểu thế nào về nồng độ ppm của $H_2\mathrm{~S}$ trong không khí?
		\item Một gian phòng trống $\left(25^{\circ} C\right.$; 1 bar) có kích thước 3 mx 4 mx 6 m bị nhiễm 10 gam khí $H_2\mathrm{~S}$. Tính nồng độ ppm của $H_2\mathrm{~S}$ trong gian phòng trên. Đánh giá mức độ độc hại của $H_2\mathrm{~S}$ trong trường hợp này. Cho biết 1 mol khí ở $25^{\circ} C$ và 1 bar có thể tích $24,79\mathrm{~L}$.
	\end{enumerate}
	\loigiai{
		\begin{enumerate}
			\item Công thức Lewis: \chemfig{H-\charge{[.radius=0.2ex]90:2pt=\:}{S}-H}; công thức cấu tạo: $H-S-H$.
			\item Nồng độ ppm của $H_2\mathrm{S}$ trong không khí là số lít khí $H_2\mathrm{S}$ có trong $1000000$ L không khí.
			\\
			Ví dụ nếu trong 1000 L không khí có sẵn $0,1$ L $H_2S$
			thì trong $1000000$ L không khí có $\dfrac{1000000 \times 0{,}1}{1000}=100$ L $H_2\mathrm{S}$.
			\\
			Ta nói nồng độ ppm của $H_2\mathrm{~S}$ trong không khí là 100 ppm.
			\item Thể tích không khí $=$ thể tích gian phòng $=3\times 4\times6=72\mathrm{~m}^3=72000\mathrm{~L}$.
			\\
			Thể tích của 10 gam $H_2\mathrm{~S}=\dfrac{24{,}79\times10}{34}=7,3\mathrm{~L}$.
			\\
			Trong $72000$ L không khí có $7,3\mathrm{~L} H_2\mathrm{~S} \Rightarrow$
			trong $1000000$ L không khí có $\dfrac{1000000\times7{,}3}{72000}=101{,}38LH_2\mathrm{~S}$.
			\\
			Vậy nồng độ $H_2\mathrm{~S}$ trong gian phòng là $101{,}38$ ppm nên gây kích thích màng phổi.
		\end{enumerate}
	}
\end{bt}
%%%==============HetBai_BT6==============%%%
%%%==============Bai_BT7==============%%%
\begin{bt}[CD-SBT] Cho biết năng lượng liên kết $H-H$ là $436\mathrm{~kJ} \mathrm{~mol}^{-1}$. Hãy tính năng lượng cần thiết (theo eV) để phá vỡ liên kết trong phân tử $H_2$, cho biết $1\mathrm{eV}=1,602\times 10^{-19} \mathrm{~J}$.
	\loigiai{
		Năng lượng cần thiết để phá vỡ liên kết trong phân tử $H_2$ là
		$$
		\dfrac{436 \cdot 1000}{\mathrm{~N}_A \cdot 1,602 \cdot 10^{-19}}=\dfrac{436 \cdot 1000}{6,02 \cdot 10^{23} \cdot 1,602 \cdot 10^{-19}}=4,52 \mathrm{eV}
		$$
	}
\end{bt}
%%%==============HetBai_BT7==============%%%
\phan{Trắc nghiệm nhiều lựa chọn}
\setchemfig{atom sep=2em}
%%%=============SOẠN EX===============%%%
\Opensolutionfile{ansex}[Ans/LGEX-C03_B03_Lien_Ket_Cong_Hoa_Tri.tex]
\Opensolutionfile{ans}[Ans/Ans-C03_B03_Lien_Ket_Cong_Hoa_Tri.tex]
%%%=============EX_1=============%%%
\begin{ex}
	Liên kết cộng hóa trị là liên kết được hình thành giữa hai nguyên tử bằng cách
	\choice
	{chuyển electron từ nguyên tử này sang nguyên tử khác.}
	{\True dùng chung electron.}
	{hút tĩnh điện.}
	{cho nhận proton.}
	\loigiai{Liên kết cộng hóa trị được hình thành bằng cách dùng chung một hay nhiều cặp electron giữa hai nguyên tử.}
\end{ex}
%%%=============EX_2=============%%%
\begin{ex}
	Nguyên tử Cl có 7 electron lớp ngoài cùng, khi hình thành liên kết với một nguyên tử Cl khác, mỗi nguyên tử Cl có xu hướng
	\choice
	{nhận thêm 2 electron.}
	{nhường đi 1 electron.}
	{\True góp chung 1 electron.}
	{nhường đi 7 electron.}
	\loigiai{Nguyên tử Cl có 7 electron lớp ngoài cùng, để đạt cấu hình electron bền vững của khí hiếm, mỗi nguyên tử Cl sẽ góp chung 1 electron để tạo thành 1 cặp electron chung.}
\end{ex}
%%%=============EX_3=============%%%
\begin{ex}
	Liên kết trong phân tử nào sau đây là liên kết cộng hóa trị không cực?
	\choice
	{HCl}
	{HBr}
	{\True Cl$_2$}
	{HF}
	\loigiai{Liên kết cộng hóa trị không cực được hình thành giữa hai nguyên tử giống nhau. Vậy Cl$_2$ có liên kết cộng hóa trị không cực.}
\end{ex}
%%%=============EX_4=============%%%
\begin{ex}
	Phân tử nào sau đây có liên kết cộng hóa trị phân cực?
	\choice
	{N$_2$}
	{H$_2$}
	{\True NH$_3$}
	{O$_2$}
	\loigiai{Liên kết cộng hóa trị phân cực được hình thành giữa hai nguyên tử khác nhau. Vậy NH$_3$ có liên kết cộng hóa trị phân cực.}
\end{ex}
%%%=============EX_5=============%%%
\begin{ex}
	Trong phân tử HCl, cặp electron liên kết bị lệch về phía nguyên tử nào?
	\choice
	{H}
	{\True Cl}
	{Lệch về cả hai phía}
	{Không bị lệch}
	\loigiai{Trong phân tử HCl, do Cl có độ âm điện lớn hơn H nên cặp electron liên kết bị lệch về phía nguyên tử Cl.}
\end{ex}
%%%=============EX_6=============%%%
\begin{ex}
	Dãy nào sau đây gồm các chất chỉ có liên kết cộng hóa trị?
	\choice
	{NaCl, MgO, CaF$_2$}
	{\True CO$_2$, H$_2$O, NH$_3$}
	{$NaOH$, $KOH$, $Ba(OH)_2$}
	{KCl, AlCl$_3$, FeCl$_3$}
	\loigiai{CO$_2$, H$_2$O, NH$_3$ là các hợp chất được tạo thành từ các nguyên tử phi kim nên chỉ chứa liên kết cộng hóa trị.}
\end{ex}
%%%=============EX_7=============%%%
\begin{ex}
	Số cặp electron dùng chung trong phân tử CO$_2$ là
	\choice
	{1}
	{2}
	{\True 4}
	{3}
	\loigiai{Trong phân tử CO$_2$, nguyên tử C góp chung 4 electron, mỗi nguyên tử O góp chung 2 electron, hình thành 2 liên kết đôi, tương ứng với 4 cặp electron dùng chung.}
\end{ex}
%%%=============EX_8=============%%%
\begin{ex}
	Cho độ âm điện của H là $2{,}2$ và của O là $3{,}44$. Vậy liên kết O-H trong phân tử H$_2$O là
	\choice
	{liên kết ion.}
	{liên kết cộng hóa trị không phân cực.}
	{\True liên kết cộng hóa trị có cực.}
	{liên kết kim loại.}
	\loigiai{Do H và O là hai phi kim, có độ âm điện chênh lệch nhưng không quá lớn ($3{,}44 - 2{,}2 = 1{,}24$) nên liên kết O-H là liên kết cộng hóa trị có cực.}
\end{ex}
%%%=============EX_9=============%%%
\begin{ex}
	Liên kết cộng hóa trị được tạo thành do
	\choice
	{lực hút tĩnh điện giữa các ion.}
	{\True sự dùng chung cặp electron giữa hai nguyên tử.}
	{sự cho nhận electron giữa hai nguyên tử.}
	{lực hút giữa hạt nhân và các electron.}
	\loigiai{Liên kết cộng hóa trị được hình thành do sự dùng chung một hay nhiều cặp electron giữa hai nguyên tử.}
\end{ex}
%%%=============EX_10=============%%%
\begin{ex}
	Chất nào sau đây có liên kết cộng hóa trị không cực?
	\choice
	{$H_2O$}
	{\True $Br_2$}
	{$NH_3$}
	{$HCl$}
	\loigiai{$Br_2$ được tạo thành từ hai nguyên tử Br giống nhau nên liên kết trong phân tử $Br_2$ là liên kết cộng hóa trị không cực.}
\end{ex}
%%%=============EX_11=============%%%
\begin{ex}
	Cặp chất nào sau đây đều chỉ chứa liên kết cộng hóa trị?
	\choice
	{NaCl và MgO}
	{NaOH và KOH}
	{\True $CH_4$ và $NH_3$}
	{KCl và CaO}
	\loigiai{$CH_4$ và $NH_3$ là các hợp chất được tạo thành từ các nguyên tử phi kim nên chỉ chứa liên kết cộng hóa trị.}
\end{ex}
%%%=============EX_12=============%%%
\begin{ex}
	Trong phân tử $N_2$, hai nguyên tử nitơ liên kết với nhau bằng cách
	\choice
	{mỗi nguyên tử nitơ góp 1 electron.}
	{mỗi nguyên tử nitơ góp 2 electron.}
	{\True mỗi nguyên tử nitơ góp 3 electron.}
	{một nguyên tử nitơ góp 2 electron, nguyên tử còn lại góp 4 electron.}
	\loigiai{Trong phân tử $N_2$, mỗi nguyên tử nitơ góp 3 electron để tạo thành 3 cặp electron chung (liên kết ba).}
\end{ex}
%%%=============EX_13=============%%%
\begin{ex}
	Phân tử nào sau đây có liên kết cho - nhận?
	\choice
	{$H_2O$}
	{\True $CO$}
	{$NH_3$}
	{$Cl_2$}
	\loigiai{Trong phân tử $CO$, cặp electron liên kết thứ ba là do nguyên tử O cho nguyên tử C. }
\end{ex}
%%%=============EX_14=============%%%
\begin{ex}
	Độ âm điện của một nguyên tố đặc trưng cho
	\choice
	{khả năng nhường electron của nguyên tử đó khi hình thành liên kết hóa học.}
	{\True khả năng hút electron của nguyên tử đó khi hình thành liên kết hóa học.}
	{khả năng tham gia phản ứng hóa học của nguyên tử đó.}
	{khả năng tạo thành liên kết ion của nguyên tử đó.}
	\loigiai{Độ âm điện của một nguyên tố đặc trưng cho khả năng hút electron của nguyên tử nguyên tố đó khi hình thành liên kết hóa học.}
\end{ex}
%%%=============EX_15=============%%%
\begin{ex}
	Liên kết trong phân tử nào sau đây là liên kết cộng hóa trị có cực?
	\choice
	{O$_2$}
	{N$_2$}
	{\True HF}
	{Cl$_2$}
	\loigiai{Liên kết cộng hóa trị có cực được hình thành giữa hai nguyên tử phi kim khác nhau. Vậy HF có liên kết cộng hóa trị có cực.}
\end{ex}
%%%=============EX_16=============%%%
\begin{ex}
	Cho các phân tử: H$_2$O, NH$_3$, CO$_2$, CH$_4$. Phân tử có độ phân cực lớn nhất là
	\choice
	{CO$_2$}
	{CH$_4$}
	{\True H$_2$O}
	{NH$_3$}
	\loigiai{H$_2$O có độ phân cực lớn nhất do nguyên tử O có độ âm điện lớn và cấu trúc phân tử dạng góc làm cho mômen lưỡng cực lớn.}
\end{ex}
%%%=============EX_17=============%%%
\begin{ex}
	Liên kết cộng hóa trị trong phân tử nào sau đây có cực nhất?
	\choice
	{H-Cl}
	{H-Br}
	{\True H-F}
	{H-I}
	\loigiai{Trong các halogen, F có độ âm điện lớn nhất nên liên kết H-F có cực nhất.}
\end{ex}
%%%=============EX_18=============%%%
\begin{ex}
	Nguyên tử X có 4 electron lớp ngoài cùng. X có thể hình thành với H
	\choice
	{1 liên kết cộng hóa trị}
	{2 liên kết cộng hóa trị}
	{3 liên kết cộng hóa trị}
	{\True 4 liên kết cộng hóa trị}
	\loigiai{Nguyên tử X có 4 electron lớp ngoài cùng, mỗi electron sẽ góp chung với 1 electron của nguyên tử H để tạo thành liên kết cộng hóa trị. Vậy X có thể hình thành với H 4 liên kết cộng hóa trị (ví dụ như CH$_4$).}
\end{ex}
%%%=============EX_19=============%%%
\begin{ex}
	Trong phân tử nước (H$_2$O), góc liên kết  $\widehat{HOH}$ xấp xỉ là:
	\choice
	{180$^\circ$}
	{120$^\circ$}
	{90$^\circ$}
	{\True $104{,}5^\circ$}
	\loigiai{Trong phân tử nước, nguyên tử O có 2 cặp electron chưa liên kết, đẩy 2 liên kết O-H lại gần nhau, làm cho góc liên kết $\widehat{HOH}$ xấp xỉ $104{,}5^\circ$.}
\end{ex}
%%%=============EX_20=============%%%
\begin{ex}
	Số cặp electron chưa liên kết trên nguyên tử trung tâm của phân tử NH$_3$ là
	\choice
	{0}
	{\True 1}
	{2}
	{3}
	\loigiai{Trong phân tử NH$_3$, nguyên tử N có 5 electron lớp ngoài cùng, trong đó có 3 electron tham gia liên kết với 3 nguyên tử H, còn lại 1 cặp electron chưa liên kết.}
\end{ex}
%%%=============EX_21=============%%%
\begin{ex}
	Cho biết độ âm điện của các nguyên tố: $H (2{,}20)$; $O (3{,}44)$; $Cl (3{,}16)$; $S (2{,}58)$. Liên kết trong phân tử nào sau đây có độ phân cực lớn nhất?
	\choice
	{H$_2$O}
	{\True HCl}
	{H$_2$S}
	{SO$_2$}
	\loigiai{Độ phân cực của liên kết phụ thuộc vào hiệu độ âm điện giữa hai nguyên tử. Hiệu độ âm điện càng lớn thì liên kết càng phân cực.
		\begin{itemize}
			\item H$_2$O: $3{,}44 - 2{,}20 = 1{,}24$
			\item HCl: $3{,}16 - 2{,}20 = 0{,}96$
			\item H$_2$S: $2{,}58 - 2{,}20 = 0{,}38$
			\item SO$_2$: $3{,}44 - 2{,}58 = 0{,}86$
		\end{itemize}
		Vậy liên kết trong phân tử HCl có độ phân cực lớn nhất.}
\end{ex}
%%%=============EX_22=============%%%
\begin{ex}
	Dãy gồm các chất trong phân tử chỉ chứa liên kết đơn là
	\choice
	{N$_2$, O$_2$, F$_2$}
	{CO$_2$, SO$_2$, H$_2$O}
	{\True CH$_4$, NH$_3$, H$_2$O}
	{C$_2$H$_4$, C$_2$H$_2$, CO$_2$}
	\loigiai{Liên kết đơn là liên kết được tạo thành bởi 1 cặp electron chung. Trong các chất trên, chỉ có CH$_4$, NH$_3$ và H$_2$O có liên kết đơn.}
\end{ex}
%%%=============EX_23=============%%%
\begin{ex}
	Nguyên tử A có 3 electron ở lớp ngoài cùng, nguyên tử B có 7 electron ở lớp ngoài cùng. Công thức phân tử của hợp chất tạo thành giữa A và B là
	\choice
	{AB$_2$}
	{A$_2$B}
	{AB$_3$}
	{\True A$_2$B$_3$}
	\loigiai{Để đạt cấu hình bền vững, A có xu hướng cho 3 electron, B có xu hướng nhận 1 electron. Vậy 2 nguyên tử A sẽ liên kết với 3 nguyên tử B, tạo thành hợp chất A$_2$B$_3$.}
\end{ex}
%%%=============EX_24=============%%%
\begin{ex}
	Trong phân tử nào sau đây, nguyên tử trung tâm không tuân theo quy tắc bát tử?
	\choice
	{CO$_2$}
	{NH$_3$}
	{H$_2$O}
	{\True BF$_3$}
	\loigiai{Trong phân tử BF$_3$, nguyên tử B chỉ có 6 electron lớp ngoài cùng (tạo 3 liên kết với 3 nguyên tử F).}
\end{ex}
%%%=============EX_25=============%%%
\begin{ex}
	Liên kết đôi gồm
	\choice
	{hai liên kết $\sigma$}
	{hai liên kết $\pi$}
	{\True một liên kết $\sigma$ và một liên kết $\pi$}
	{hai liên kết ion}
	\loigiai{Liên kết đôi gồm một liên kết $\sigma$ (sigma) bền vững và một liên kết $\pi$ (pi) kém bền vững hơn.}
\end{ex}
%%%=============EX_26=============%%%
\begin{ex}
	Cho các phân tử sau: H$_2$, HCl, HF, HBr, HI. Phân tử có năng lượng liên kết lớn nhất là
	\choice
	{H$_2$}
	{HCl}
	{\True HF}
	{HI}
	\loigiai{Năng lượng liên kết phụ thuộc vào độ bền của liên kết. Trong các phân tử trên, liên kết H-F có độ bền lớn nhất do độ âm điện của F lớn nhất, dẫn đến năng lượng liên kết lớn nhất.}
\end{ex}
%%%=============EX_27=============%%%
\begin{ex}
	Ý nào sau đây \textbf{không đúng} khi nói về liên kết cộng hóa trị?
	\choice
	{Liên kết cộng hóa trị được hình thành do sự dùng chung electron giữa hai nguyên tử.}
	{Liên kết cộng hóa trị có thể là liên kết đơn, liên kết đôi hoặc liên kết ba.}
	{Liên kết cộng hóa trị được hình thành giữa hai nguyên tử phi kim.}
	{\True Liên kết cộng hóa trị luôn là liên kết có cực.}
	\loigiai{Liên kết cộng hóa trị có thể là liên kết có cực hoặc không cực. Liên kết cộng hóa trị không cực được hình thành giữa hai nguyên tử giống nhau.}
\end{ex}
%%%=============EX_28=============%%%
\begin{ex}
	Phân tử nào sau đây có chứa cả liên kết cộng hóa trị và liên kết cho - nhận?
	\choice
	{$HCl$}
	{$CO_2$}
	{\True $HNO_3$}
	{$H_2O$}
	\loigiai{Trong phân tử $HNO_3$, có 2 liên kết cộng hóa trị (N-O) và 1 liên kết cho - nhận ($N-O$).}
\end{ex}
%%%=============EX_29=============%%%
\begin{ex}
	Để đạt được cấu hình electron bền vững của khí hiếm gần nhất, nguyên tử clo có xu hướng
	\choice
	{nhường đi 1 electron}
	{\True nhận thêm 1 electron}
	{góp chung 1 electron}
	{nhận thêm 7 electron}
	\loigiai{Nguyên tử clo có 7 electron lớp ngoài cùng, để đạt được cấu hình electron bền vững của khí hiếm gần nhất (Argon), clo có xu hướng nhận thêm 1 electron.}
\end{ex}
%%%=============EX_30=============%%%
\begin{ex}
	Cho các chất sau: $Cl_2$, $HCl$, $NaCl$, $NaF$.  Số chất chứa liên kết cộng hóa trị là
	\choice
	{1}
	{\True 2}
	{3}
	{4}
	\loigiai{$Cl_2$ và $HCl$ là các hợp chất được tạo thành từ các phi kim nên chứa liên kết cộng hóa trị.}
\end{ex}
%%%=============EX_31=============%%%
\begin{ex}
	Cho các chất sau: $Cl_2$, $O_2$, $H_2S$, $NaCl$, $NaF$, $NH_3$, $CCl_4$, $SO_2$. Số chất chứa liên kết cộng hóa trị phân cực là
	\choice
	{1}
	{2}
	{3}
	{\True 4}
	\loigiai{$H_2S$, $NH_3$, $CCl_4$, $SO_2$ là các hợp chất chứa liên kết cộng hóa trị phân cực.}
\end{ex}
%%%=============EX_32=============%%%
\begin{ex}
	Cho các chất sau: $Cl_2$, $O_2$, $H_2S$, $NaCl$, $NaF$, $NH_3$, $CCl_4$, $SO_2$. Số chất chứa liên kết cộng hóa trị phân cực là
	\choice
	{1}
	{2}
	{3}
	{\True 4}
	\loigiai{$H_2S$, $NH_3$, $CCl_4$, $SO_2$ là các hợp chất chứa liên kết cộng hóa trị phân cực.}
\end{ex}
%%%=============EX_33=============%%%
\begin{ex}
	Trong số các chất sau, chất nào chỉ chứa liên kết $\sigma$
	\choice
	{\chemfig{CH~CH}}
	{\chemfig{CH_2=CH_2}}
	{\chemfig{O=O}}
	{\True \chemfig{H-C(-[:90]H)(-[:-90]H)-H}}
	\loigiai{Liên kết $\sigma$ luôn luôn là liên kết đơn. Trong công thức \chemfig{H-C(-[:90]H)(-[:-90]H)-H} chỉ chứa các liên kết đơn.}
\end{ex}
%%%=============EX_34=============%%%
\begin{ex}
	Trong phân tử ammonia $\mathrm{N}_2$, số cặp electron chung giữa  hai nguyên tử nitrogen là
	\choice
	{1}
	{\True 3}
	{2}
	{4}
	\loigiai{Cấu hình electron của nitrogen là $1s^22s^22p^3$ $\Rightarrow$ có $5$ electron ở lớp ngoài cùng. Theo quy tắc bát tử , khi hình thành liên kết mỗi nguyên tử N "đưa ra" 3 electron để dùng chung do đó số cặp electron chung giữa  hai nguyên tử nitrogen là 3.
	}
\end{ex}
%%%=============EX_35=============%%%
\begin{ex}
	Chất vừa có liên kết cộng hoá trị phân cực, vừa có liên kết cộng hoá trị không phân cực là
	\choice
	{$\mathrm{NH}_3$}
	{\True $\mathrm{C}_2 \mathrm{~F}_6$}
	{$\mathrm{CO}_2$}
	{$\mathrm{H}_2 \mathrm{O}$}
	\loigiai{
		\begin{itemize}
			\item \textbf{$CO_2$:}
			\begin{itemize}
				\item Liên kết C=O là liên kết cộng hóa trị có cực (do độ âm điện của C và O khác nhau).
				\item Tuy nhiên, do $CO_2$ có cấu trúc thẳng, hai liên kết C=O có cực hướng về hai phía ngược nhau nên triệt tiêu lẫn nhau, làm cho phân tử $CO_2$ không phân cực.
			\end{itemize}
			\item \textbf{$H_2O$:} Liên kết O-H là liên kết cộng hóa trị có cực.
			\item \textbf{$NH_3$:} Liên kết N-H là liên kết cộng hóa trị có cực.
			\item \textbf{$C_2F_6$:}
			\begin{itemize}
				\item Liên kết C-F là liên kết cộng hóa trị có cực (do độ âm điện của C và F khác nhau).
				\item Liên kết C-C là liên kết cộng hóa trị không phân cực (do hai nguyên tử C có độ âm điện bằng nhau).
			\end{itemize}
		\end{itemize}
	}
\end{ex}
{\par\noindent\indam[black]{Sử dụng giá trị độ âm điện các nguyên tố được cho trong bảng sau để trả lời các câu 36, 37 , 38.}}
\begin{center}
	\begin{tabular}{|c|c|c|c|}
		\hline Nguyên tố & Độ âm điện & Nguyên tố & Độ âm điện \\
		\hline Na & $0{,}93$ & $0$ & $3{,}44$ \\
		\hline $H$ & $2{,}20$ & Br & $2{,}96$ \\
		\hline $C$ & $2{,}55$ & Cl & $3{,}16$ \\
		\hline $N$ & $3{,}04$ & $F$ & $3{,}98$ \\
		\hline
	\end{tabular}
\end{center}
Dưới đây là phần bổ sung lời giải chi tiết cho các câu hỏi của bạn:

%%%=============EX_36=============%%%
\begin{ex}
	Liên kết nào dưới đây là liên kết cộng hoá trị không phân cực?
	\choice
	{$\mathrm{Na}-O$}
	{$O-H$}
	{$\mathrm{Na}-C$}
	{\True $C-H$}
	\loigiai{Liên kết cộng hóa trị không phân cực được hình thành giữa hai nguyên tử có độ âm điện bằng nhau hoặc chênh lệch độ âm điện rất nhỏ. Trong các liên kết trên, liên kết C-H có chênh lệch độ âm điện nhỏ nhất nên là liên kết cộng hóa trị không phân cực.}
\end{ex}
%%%=============EX_37=============%%%
\begin{ex}
	Lực kéo electron về phía nguyên tử nitrogen mạnh nhất ở liên kết nào dưới đây?
	\choice
	{$N-H$}
	{\True $N-F$}
	{$N-\mathrm{Cl}$}
	{$N-\mathrm{Br}$}
	\loigiai{Nguyên tử có độ âm điện càng lớn thì lực hút electron càng mạnh. Flo (F) là nguyên tố có độ âm điện lớn nhất trong bảng tuần hoàn, do đó liên kết N-F sẽ có lực kéo electron về phía nguyên tử nitrogen mạnh nhất.}
\end{ex}
%%%=============EX_38=============%%%
\begin{ex}
	Liên kết nào trong các liên kết sau là phân cực nhất?
	\choice
	{$C-H$}
	{\True $C-F$}
	{$C-\mathrm{Cl}$}
	{$C-\mathrm{Br}$}
	\loigiai{Liên kết càng phân cực khi chênh lệch độ âm điện giữa hai nguyên tử càng lớn. Trong các liên kết trên, liên kết C-F có chênh lệch độ âm điện lớn nhất (do F có độ âm điện lớn nhất) nên là liên kết phân cực nhất.}
\end{ex}
%%%=============EX_39=============%%%
\begin{ex}
	Hợp chất nào sau đây chứa cả liên kết cộng hoá trị và liên kết ion?
	\choice
	{$CH_2O$}
	{$CH_4$}
	{$Na_2O$}
	{\True $KOH$}
	\loigiai{
		KOH chứa cả liên kết cộng hóa trị (giữa O và H) và liên kết ion (giữa K và OH).
		Các hợp chất còn lại chỉ chứa liên kết cộng hóa trị.}
\end{ex}
%%%=============EX_40=============%%%
\begin{ex}
	Các liên kết trong phân tử nitrogen được tạo thành do sự xen phủ của
	\choice
	{các orbital s với nhau.}
	{2 orbital s và 1 orbital p với nhau.}
	{1 orbital s và 2 orbital p với nhau.}
	{\True 3 orbital p giống nhau về hình dạng và kích thước, chỉ khác nhau về sự định hướng trong không gian.}
	\loigiai{
		Nitrogen có cấu hình electron lớp ngoài cùng là $2s^22p^3$.
		Phân tử $N_2$ có liên kết ba, được hình thành do sự xen phủ của 3 orbital p của mỗi nguyên tử nitrogen.}
\end{ex}
%%%=============EX_41=============%%%
\begin{ex}
	Điều nào sau đây \textbf{sai} khi nói về tính chất của hợp chất cộng hoá trị?
	\choice
	{Các hợp chất cộng hoá trị có nhiệt độ nóng chảy và nhiệt độ sôi thấp hơn các hợp chất ion.}
	{Các hợp chất cộng hoá trị có thể ở thể rắn, lỏng hoặc khí trong điều kiện thường.}
	{\True Các hợp chất cộng hoá trị đều dẫn điện tốt.}
	{Các hợp chất cộng hoá trị không phân cực tan được trong dung môi không phân cực.}
	\loigiai{Đa số các hợp chất cộng hóa trị không dẫn điện (trừ than chì).
	}
\end{ex}
%%%=============EX_42=============%%%
\begin{ex}
	Đặt độ dài các liên kết $N-N, N=N$ và $N\equiv N$ lần lượt là $I_1; I_2$ và $I_3$. Thứ tự tăng dần độ dài các liên kết là
	\choice
	{\True$I_3; I_2; I_1$}
	{$I_1; I_3; I_2$}
	{$I_2; I_1; I_3$}
	{$I_1; I_2; I_3$}
	\loigiai{Số cặp electron dùng chung càng nhiều thì lực hút giữa các nguyên tử càng mạnh, làm cho độ dài liên kết càng ngắn. Vậy nên $I_3 < I_2 < I_1$}
\end{ex}
%%%=============EX_43=============%%%
\begin{ex}
	Phát biểu nào sau đây đúng với độ bền của một liên kết?
	\choice
	{Khi nhiều liên kết được hình thành giữa hai nguyên tử, độ bền của liên kết sẽ giảm}
	{Độ bền của liên kết tăng khi độ dài của liên kết tăng}
	{\True Độ bền của liên kết tăng khi độ dài của liên kết giảm}
	{Độ bền của liên kết không phụ thuộc vào độ dài liên kết}
	\loigiai{Độ bền liên kết phụ thuộc vào độ dài liên kết. Nói chung, độ bền của liên kết tăng khi độ dài của liên kết giảm và ngược lại.}
\end{ex}
%%%=============EX_44=============%%%
\begin{ex}
	Liên kết cộng hoá trị là liên kết hoá học được hình thành giữa hai nguyên tử bằng
	\choice
	{một electron chung}
	{sự cho - nhận electron}
	{một cặp electron góp chung}
	{\True một hay nhiều cặp electron dùng chung}
	\loigiai{Liên kết cộng hóa trị là liên kết được hình thành bằng một hay nhiều cặp electron dùng chung giữa hai nguyên tử. Các cặp electron này được gọi là cặp electron liên kết.
	}
\end{ex}

%%%=============EX_45=============%%%
\begin{ex}
	Hợp chất nào sau đây có liên kết cộng hoá trị không phân cực?
	\choice
	{LiCl }
	{$\mathrm{CF}_2 \mathrm{Cl}_2$}
	{$\mathrm{CHCl}_3$}
	{\True $\mathrm{N}_2$}
	\loigiai{Liên kết cộng hóa trị không phân cực hình thành giữa hai nguyên tử giống nhau (cùng độ âm điện). Trong các hợp chất trên, chỉ có N$_2$ là hợp chất tạo thành từ hai nguyên tử giống nhau (N và N).}
\end{ex}

%%%=============EX_46=============%%%
\begin{ex}
	Hợp chất nào sau đây có liên kết cộng hoá trị phân cực?
	\choice
	{$\mathrm{H}_2$}
	{\True $\mathrm{CHCl}_3$}
	{$\mathrm{CH}_4$}
	{$\mathrm{N}_2$}
	\loigiai{Liên kết cộng hóa trị phân cực hình thành giữa hai nguyên tử khác nhau (chênh lệch độ âm điện). Trong các hợp chất trên, $\mathrm{CHCl}_3$ có liên kết cộng hóa trị phân cực do Cl có độ âm điện lớn hơn C và H.}
\end{ex}

%%%=============EX_47=============%%%
\begin{ex}
	Liên kết $\sigma$ là liên kết hình thành do
	\choice
	{sự xen phủ bên của hai orbital}
	{cặp electron dùng chung}
	{lực hút tũnh điện giữa hai ion}
	{\True sự xen phủ trục của hai orbital}
	\loigiai{Liên kết $\sigma$ được hình thành do sự xen phủ trục của hai orbital. Trục của hai orbital là đường thẳng nối tâm hai nguyên tử.
	}
\end{ex}

%%%=============EX_48=============%%%
\begin{ex}
	Liên kết $\pi$ là liên kết hình thành do
	\choice
	{\True sự xen phủ bên của hai orbital}
	{cặp electron dùng chung}
	{lực hút tũnh điện giữa hai ion}
	{sự xen phủ trục của hai orbital}
	\loigiai{Liên kết $\pi$ được hình thành do sự xen phủ bên của hai orbital. Sự xen phủ bên là sự xen phủ của hai orbital song song với nhau.}
\end{ex}

%%%=============EX_49=============%%%
\begin{ex}
	Liên kết trong phân tử nào sau đầy được hình thành nhờ sự xen phủ orbital $\mathrm{p}-\mathrm{p}$ ?
	\choice
	{$\mathrm{H}_2$}
	{\True $\mathrm{Cl}_2$}
	{$\mathrm{NH}_3$}
	{HCl }
	\loigiai{Liên kết trong phân tử Cl$_2$ được hình thành do sự xen phủ trục của hai orbital 3p của hai nguyên tử Cl.}
\end{ex}

%%%=============EX_50=============%%%
\begin{ex}
	Liên kết trong phân tử nào sau đây được hình thành nhờ sự xen phủ orbital s-s?
	\choice
	{\True $\mathrm{H}_2$}
	{$\mathrm{Cl}_2$}
	{$\mathrm{NH}_3$}
	{HCl }
	\loigiai{Liên kết trong phân tử H$_2$ được hình thành do sự xen phủ trục của hai orbital 1s của hai nguyên tử H.}
\end{ex}

%%%=============EX_51=============%%%
\begin{ex}
	Liên kết trong phân tử nào sau đây được hình thành nhờ sự xen phủ orbital s-p?
	\choice
	{$\mathrm{H}_2$}
	{$\mathrm{Cl}_2$}
	{ \True HCl }
	{$\mathrm{O}_2$}
	\loigiai{Liên kết trong phân tử HCl được hình thành do sự xen phủ trục của orbital 1s của nguyên tử H và orbital 3p của nguyên tử Cl.}
\end{ex}

%%%=============EX_52=============%%%
\begin{ex}
	Các liên kết trong phân tử oxygen gồm
	\choice
	{2 liên kết $\pi$}
	{2 liên kết $\sigma$}
	{\True 1 liên kết $\sigma, 1$ liên kết $\pi$}
	{1 liên kết $\sigma$}
	\loigiai{Phân tử oxygen (O$_2$) có một liên kết đôi, bao gồm một liên kết $\sigma$ (hình thành do sự xen phủ trục của hai orbital 2p) và một liên kết $\pi$ (hình thành do sự xen phủ bên của hai orbital 2p).}
\end{ex}

%%%=============EX_53=============%%%
\begin{ex}
	Số liên kết $\sigma$ và $\pi$ có trong phân tử $\mathrm{C}_2 \mathrm{H}_2$ lần lượt là
	\choice
	{2 và 3 }
	{\True 3 và 2 }
	{2 và 2 }
	{3 và 1 }
	\loigiai{Phân tử C$_2$H$_2$ có công thức cấu tạo là H-C$\equiv$C-H.
		
	\noindent Vậy có 3 liên kết $\sigma$ (1 liên kết C-C và 2 liên kết C-H) và 2 liên kết $\pi$ trong liên kết ba C$\equiv$C.}
\end{ex}

%%%=============EX_54=============%%%
\begin{ex}
	Dãy nào sau đây gồm các chất chỉ có liên kết cộng hoá trị?
	\choice
	{$\mathrm{BaCl}_2, \mathrm{NaCl}, \mathrm{NO}_2$}
	{$\mathrm{SO}_2, \mathrm{CO}_2, \mathrm{Na}_2 \mathrm{O}_2$}
	{\True $\mathrm{SO}_3, \mathrm{H}_2 \mathrm{~S}, \mathrm{H}_2 \mathrm{O}$}
	{$\mathrm{CaCl}_2, \mathrm{~F}_2 \mathrm{O}, \mathrm{HCl}$}
	\loigiai{Liên kết cộng hóa trị thường được hình thành giữa các nguyên tử phi kim. Trong các dãy trên, chỉ có dãy $\mathrm{SO}_3, \mathrm{H}_2 \mathrm{~S}, \mathrm{H}_2 \mathrm{O}$ gồm các hợp chất tạo thành từ các nguyên tử phi kim.}
\end{ex}

%%%=============EX_55=============%%%
\begin{ex}
	Cho hai nguyên tố $\mathrm{X}(\mathrm{Z}=20)$ và $\mathrm{Y}(\mathrm{Z}=17)$. Công thức hợp chất tạo thành từ nguyên tố $\mathrm{X}, \mathrm{Y}$ và liên kết trong phân tử là
	\choice
	{XY: liên kết cộng hoá trị}
	{$\mathrm{X}_2 \mathrm{Y}_3$ : liên kết cộng hoá trị}
	{$\mathrm{X}_2 \mathrm{Y}$ : liên kết ion}
	{\True $\mathrm{XY}_2$ : liên kết ion}
	\loigiai{
		X (Z = 20): $1s^22s^22p^63s^23p^64s^2$ $\Rightarrow$ X là kim loại, có xu hướng nhường 2 electron để đạt cấu hình bền vững của khí hiếm.
		\\
		Y (Z = 17): $1s^22s^22p^63s^23p^5$ $\Rightarrow$ Y là phi kim, có xu hướng nhận 1 electron để đạt cấu hình bền vững của khí hiếm.
		\\
		$\Rightarrow$  Công thức hợp chất là XY$_2$, liên kết trong phân tử là liên kết ion.}
\end{ex}
%%%=============EX_56=============%%%
\begin{ex}
	Trong nguyên tử $C$, những electron có khả năng tham gia hình thành liên kết cộng hoá trị thuộc phân lớp nào sau đây?
	\choice
	{1s}
	{$2$s}
	{\True $2s,2p$}
	{$1s, 2s, 2p$}
	\loigiai{Electron tham gia hình thành liên kết là các electron lớp ngoài cùng. Cấu hình electron của $C$ là $1s^22s^22p^2$. Vậy các electron có khả năng tham gia hình thành liên kết cộng hóa trị thuộc phân lớp 2s và 2p.}
\end{ex}

%%%=============EX_57=============%%%
\begin{ex}
	Những phát biểu nào sau đây là không đúng?
	\choice
	{Các nguyên tử liên kết với nhau theo xu hướng tạo hệ bền vững hơn}
	{Các nguyên tử liên kết với nhau theo xu hướng tạo hệ có năng lượng thấp hơn}
	{Các nguyên tử liên kết với nhau theo xu hướng tạo lớp vỏ electron được octet}
	{\True Các nguyên tử liên kết với nhau theo xu hướng tạo hệ có năng lượng cao hơn}
		\loigiai{Các nguyên tử liên kết với nhau theo xu hướng tạo hệ bền vững hơn, có năng lượng thấp hơn và đạt được cấu hình electron bền vững của khí hiếm (thường là octet). Nguyên tử phi kim có thể liên kết với nguyên tử kim loại (liên kết ion) hoặc với nguyên tử phi kim khác (liên kết cộng hóa trị).}
	\end{ex}
	
	%%%=============EX_58=============%%%
	\begin{ex}
		Liên kết cộng hoá trị thường được hình thành giữa
		\choice
		{các nguyên tử nguyên tố kim loại với nhau}
		{\True các nguyên tử nguyên tố phi kim với nhau}
		{các nguyên tử nguyên tố kim loại với các nguyên tử nguyên tố phi kim}
		{các nguyên tử khi hiếm với nhau}
		\loigiai{
			\begin{itemize}
				\item Liên kết cộng hóa trị thường được hình thành giữa các nguyên tử phi kim.
				\item Liên kết giữa các nguyên tử kim loại là liên kết kim loại.
				\item Liên kết giữa nguyên tử kim loại và phi kim là liên kết ion.
			\end{itemize}    
		}
	\end{ex}
	
	%%%=============EX_59=============%%%
	\begin{ex}
		Số lượng cặp electron dùng chung trong các phân tử $H_2$, $O_2$, $N_2$, $F_2$ lần lượt là:
		\choice
		{$1,2,3,4$}
		{\True $1,2,3,1$}
		{2,2, 2,2}
		{$1,2,2,1$}
		\loigiai{
			\begin{itemize}
				\item H$_2$: 1 cặp electron dùng chung (liên kết đơn).
				\item O$_2$: 2 cặp electron dùng chung (liên kết đôi).
				\item N$_2$: 3 cặp electron dùng chung (liên kết ba).
				\item F$_2$: 1 cặp electron dùng chung (liên kết đơn).
			\end{itemize}    
		}
	\end{ex}
	
	%%%=============EX_60=============%%%
	\begin{ex}
		Trong phân tử HF, số cặp electron dùng chung và cặp electron hoá trị riêng của nguyên tử F lần lượt là:
		\choice
		{\True 1 và 3}
		{2 và 2}
		{3 và 1}
		{1 và 4}
		\loigiai{
			\begin{itemize}
				\item Phân tử HF có 1 cặp electron dùng chung để tạo thành liên kết cộng hóa trị.
				\item F có 7 electron lớp ngoài cùng, trong đó có 1 electron dùng chung với H, còn lại 6 electron tạo thành 3 cặp electron hóa trị riêng.
			\end{itemize}    
		}
	\end{ex}
	%%%==============Cau_EX61==============%%%
	\begin{ex}
		Cho công thức Lewis của các phân tử sau:
		\begin{center}
			\chemfig{H-[,1]\charge{[.radius=0.2ex]90:2pt=\:}{N}(-[:-90,1]H)-[,1]H}
				\quad;\quad
				\chemfig{\charge{[.radius=0.2ex]90:2pt=\:,-90:2pt=\:,180:2pt=\:}{Cl}-[:30,1]B(-[:90,1]\charge{[.radius=0.2ex]90:2pt=\:,0:2pt=\:,180:2pt=\:}{Cl})-[:-30,1]\charge{[.radius=0.2ex]90:2pt=\:,-90:2pt=\:,0:2pt=\:}{Cl}}
				\quad;\quad
				\chemfig{H-Be-H}
				\quad;\quad
			\chemfig{H-[,1]C(-[:-90,1]H)(-[:90,1]H)-[,1]H}
		\end{center}
		Số phân tử mà nguyên tử trung tâm không thoả mãn quy tắc octet là
		\choice
		{$1$}
		{$2$}
		{\True $3$}
		{$4$}
		\loigiai{}
	\end{ex}
	%%%==============HetCau_EX61==============%%%
	%%%==============Begin Câu 62===============%%%
	\begin{ex}
		Công thức nào sau đây ứng với công thức Lewis của phân tử $\mathrm{PCl}_3$ ?
		\begin{center}
		\tikz[baseline,declare function={d=2.75;}]{
		\path (1*d,0) node (a) {\chemfig{\charge{[.radius=0.2ex]90:2pt=\:,-90:2pt=\:}{Cl}=\charge{[.radius=0.2ex]0:2pt=\:}{P}(-[:-90]\charge{[.radius=0.2ex]-90:2pt=\:,0:2pt=\:,180:2pt=\:}{Cl})-[:90]\charge{[.radius=0.2ex]90:2pt=\:,0:2pt=\:,180:2pt=\:}{Cl}}};
		\path ($(a.south)+(0.4cm,-0.2cm)$) node [anchor=north]{(1)};
		%%%
		\path (2*d,0) node (b) {\chemfig{\charge{[.radius=0.2ex]180:2pt=\:,90:2pt=\:,-90:2pt=\:}{Cl}-\charge{[.radius=0.2ex]0:2pt=\:}{P}(-[:-90]\charge{[.radius=0.2ex]-90:2pt=\:,0:2pt=\:,180:2pt=\:}{Cl})-[:90]\charge{[.radius=0.2ex]90:2pt=\:,0:2pt=\:,180:2pt=\:}{Cl}}};
		\path ($(b.south)+(0.4cm,-0.2cm)$) node [anchor=north]{(2)};
		%%
		\path (3*d,0) node (c) {\chemfig{\charge{[.radius=0.2ex]90:2pt=\:,-90:2pt=\:,180:2pt=\:}{Cl}~\charge{[.radius=0.2ex]0:2pt=\:}{P}(~[:-90]\charge{[.radius=0.2ex]-90:2pt=\:,0:2pt=\:,180:2pt=\:}{Cl})~[:90]\charge{[.radius=0.2ex]90:2pt=\:,0:2pt=\:,180:2pt=\:}{Cl}}};
		\path ($(c.south)+(0.4cm,-0.2cm)$) node [anchor=north]{(3)};
		%%%
		\path (4*d,0) node (d) {\chemfig{\charge{[.radius=0.2ex]180:2pt=\:,90:2pt=\:,-90:2pt=\:}{Cl}-P(-[:-90]\charge{[.radius=0.2ex]-90:2pt=\:,0:2pt=\:,180:2pt=\:}{Cl})-[:90]\charge{[.radius=0.2ex]90:2pt=\:,0:2pt=\:,180:2pt=\:}{Cl}}};
		\path ($(d.south)+(0.4cm,-0.2cm)$) node [anchor=north]{(4)};
		}
		\end{center}
		\choice
		{Công thức (4)}
		{Công thức (1)}
		{Công thức (2)}
		{Công thức (3)}
		\loigiai{P có 5 electron hóa trị theo quy tắc octet p sẽ đưa ra 3 electron để dùng chung với 3 nguyên tửCl và còn 1 đôi e chưa tham gia liên kết do đó theo công thức lewis có 3 liên kết đơn và 1 đôi e chưa liên kết}
	\end{ex}
	%%%=============End Câu 62===============%%%
	%%%==============Cau_EX63==============%%%
	\begin{ex}
		Dựa vào hiệu độ âm điện giữa hai nguyên tố, cho biết liên kết trong phân tử nào sau đây là phân cực nhất.
		\choice
		{\True HF}
		{HCl}
		{HBr}
		{HI}
		\loigiai{Hiệu độ âm điện càng lớn độ phân cực càng lớn. Ta thấy độ âm điện của F là lớn nhất do đó trong phân tử HF có độ phân cực lớn nhất.}
	\end{ex}
	%%%=============EX_64=============%%%
	\begin{ex}
		Khi tham gia hình thành liênn kết trong các phân tử $HF, F_2$; orbital tham gia xen phủ tạo liên kết của nguyên tử F thuộc về phân lớp nào, có hình dạng gì?
		\choice
		{Phân lớp 2 s, hình cầu}
		{Phân lớp 2 s, hình số tám nổi}
		{\True Phân lớp 2 p, hình số tám nổi}
		{Phân lớp 2 p, hình cánh hoa}
		\loigiai{}
	\end{ex}
	
	%%%=============EX_65=============%%%
	\begin{ex}
		Số orbital của cả hai nguyên tử N tham gia xen phủ tạo liên kết trong phân tử $N_2$ là
		\choice
		{\True $3$}
		{$4$}
		{$5$}
		{$6$}
		\loigiai{}
	\end{ex}
	
	%%%=============EX_66=============%%%
	\begin{ex}
		Liên kết trong phân tử nào dưới đây không được hình thành do sự xen phủ giữa các orbital cùng loại (ví dụ cùng là orbital s, hoặc cùng là orbital p)?
		\choice
		{$\mathrm{Cl}_2$}
		{$H_2$}
		{\True $NH_3$}
		{$\mathrm{Br}_2$}
		\loigiai{}
	\end{ex}
	
	%%%=============EX_67=============%%%
	\begin{ex}
		Phát biểu nào sau đây không đúng?
		\choice
		{\True Chỉ có các AO có hình dạng giống nhau mới xen phủ với nhau để tạo liên kết}
		{Khi hình thành liên kết cộng hoá trị giữa hai nguyên tử, luôn có một liên kết $\sigma$}
		{Liên kết $\sigma$ bền vững hơn liên kết $\pi$}
		{Có hai kiểu xen phủ hình thành liên kết là xen phủ trục và xen phủ bên}
		\loigiai{}
	\end{ex}
	
	%%%=============EX_68=============%%%
	\begin{ex}
		Số lượng electron tham gia hình thành liên kết đơn, đôi và ba lần lượt là:
		\choice
		{1,2 và 3}
		{\True 2,4 và 6}
		{1,3 và 5}
		{2,3 và 4}
		\loigiai{}
	\end{ex}
	%%%=============EX_69=============%%%
	\begin{ex}
		Phân tử nào sau đây không phân cực:
		\choice
		{$SO_2$}
		{\True $CO_2$}
		{$NH_3$}
		{$H2_O$}
		\loigiai{Mặc dù liên kết giữa C và O trong  $CO_2$ là liên kết cộng hóa trị phân cực tuy nhiên Do cấu trúc đối xứng, các moment lưỡng cực của 2 liên kết $C=O$ bằng nhau về độ lớn nhưng ngược chiều.Kết quả là các moment lưỡng cực triệt tiêu lẫn nhau}
	\end{ex}
	%%%==============Cau_EX70==============%%%
	\begin{ex}
		Cho độ âm điện của các nguyên tố: H (2,20); C (2,55); N (3,04); O (3,44); F (3,98). Hãy cho biết trong các hợp chất sau: $NH_3$, $CO_2$, $HF$, $H_2O$, $CH_4$, chất nào có chứa liên kết cộng hóa trị phân cực?
		\choice
		{$CH_4$, $CO_2$}
		{\True $NH_3$, $HF$, $H_2O$}
		{$HF$, $H_2O$}
		{$NH_3$, $CO_2$, $HF$, $H_2O$}
		\loigiai{}
	\end{ex}
	%%%==============HetCau_EX70==============%%%
	%%%==============Begin Câu 71===============%%%
	\begin{ex}
		Cho biết năng lượng liên kết $H-I$ và $H-Br$ lần lượt là $297$ $kJmol^{-1}$ và $364$ $kJmol^{-1}$.Phát biểu sau đây là không đúng?
		\choice
		{Liên kết $H-I$ là bền vững hơn so với liên kết $H-Br$}
		{Khi đun nóng, HI bị phân huỷ (thành $H_2$ và $I_2$) ở nhiệt độ cao hơn so với HBr (thành $H_2$ và $Br_2$)}
		{Cần cung cấp $297$ $kJ$ và $364$ $kJ$ để lần lượt phá vỡ 1 mol khí $HI$ và 1 mol khí $HBr$ thành các nguyên tử ở thể khí.}
		{Khi đun nóng, HI bị phân huỷ (thành $H_2$ và $I_2$) ở nhiệt độ thấp hơn so với HBr (thành $H_2$ và $Br_2$)}
		\loigiai{%
			\begin{itemize}
				\item Năng lượng liên kết càng lớn thì liên kết càng bền vững. Vì năng lượng liên kết H-Br ($364$  $kJ mol^{-1}$) lớn hơn năng lượng liên kết $H-I$ ($297$  $kJ mol^{-1}$) nên liên kết $H-Br$ bền hơn liên kết $H-I$
				\item Nhiệt độ phân huỷ: Liên kết càng kém bền vững thì càng dễ bị phá vỡ bởi nhiệt, do đó cần ít năng lượng hơn (nhiệt độ thấp hơn) để phân huỷ. Vì liên kết H-I kém bền vững hơn nên HI sẽ bị phân huỷ ở nhiệt độ thấp hơn so với HBr.
			\end{itemize}
		}
	\end{ex}
	%%%=============End Câu 71===============%%%
\Closesolutionfile{ans}
\Closesolutionfile{ansex}
%\bangdapan{Ans-C03_B03_Lien_Ket_Cong_Hoa_Tri.tex}
\phan{Trắc nghiệm đúng sai}
%%%=============SOẠN EXTF===============%%%
\Opensolutionfile{ansex}[Ans/LGTF-C03_B03_LIEN_KET_CONG_HOA_TRI.TEX]
\Opensolutionfile{ansbook}[Ansbook/AnsTF-C03_B03_LIEN_KET_CONG_HOA_TRI.TEX]
\Opensolutionfile{ans}[Ans/Tempt-C03_B03_LIEN_KET_CONG_HOA_TRI.TEX]
%%%=============EX_1=============%%%
\begin{ex}
	Cho các phân tử: $H_2O$, $NH_3$, $CH_4$, $CO_2$.
	\choiceTF
	{Tất cả các phân tử trên đều có liên kết cộng hóa trị không cực.}
	{\True $CO_2$ là phân tử có liên kết cộng hóa trị có cực nhưng phân tử không phân cực.}
	{Các phân tử $H_2O$ và $NH_3$ có hình dạng giống nhau.}
	{\True Góc liên kết H-O-H trong $H_2O$ nhỏ hơn góc liên kết H-C-H trong $CH_4$.}
	\loigiai{
		\begin{itemchoice}[F1,T2,F3,T4]
			\itemch $H_2O$, $NH_3$, $CH_4$ có liên kết cộng hóa trị có cực.
			\itemch $CO_2$ có cấu tạo thẳng nên mặc dù có liên kết C=O phân cực nhưng lại là phân tử không phân cực.
			\itemch $H_2O$ có hình dạng gấp khúc, $NH_3$ có hình dạng hình chóp tam giác.
			\itemch Do ảnh hưởng của cặp electron hóa trị riêng trên nguyên tử O, góc liên kết H-O-H bị giảm xuống còn khoảng $104,5^o$, nhỏ hơn góc liên kết $109,5^o$ trong $CH_4$.
		\end{itemchoice}
	}
\end{ex}
%%%=============EX_2=============%%%
\begin{ex}
	Xét về sự hình thành phân tử $N_2$
	\choiceTF
	{\True Phân tử $N_2$ có 3 cặp electron chung.}
	{\True Liên kết trong phân tử $N_2$ là liên kết cộng hóa trị phân cực.}
	{Trong phân tử $N_2$ liên kết ba gồm 1 liên $\sigma$ và 2 liên kết $\pi$.}
	{Phân tử $N_2$ phân cực.}
	\loigiai{
		\begin{itemchoice}[T1,T2,T3,F4]
			\itemch $N_2$ có cấu hình electron lớp ngoài cùng là $2s^22p^3$, mỗi nguyên tử N góp chung 3 electron tạo thành 3 cặp electron chung.
			\itemch Liên kết ba trong $N_2$ là liên kết cộng hóa trị không phân cực.
			\itemch Liên kết ba trong phân tử $N_2$ gồm 1 liên kết $\sigma$ và 2 liên kết $\pi$.
			\itemch $N_2$ là phân tử không phân cực.
		\end{itemchoice}
	}
\end{ex}
%%%=============EX_3=============%%%
\begin{ex}
	Cho các phân tử: $H_2O$, $NH_3$, $CH_4$, $CO_2$.
	\choiceTF
	{Tất cả các phân tử trên đều có liên kết cộng hóa trị không cực.}
	{\True $CO_2$ là phân tử có liên kết cộng hóa trị có cực nhưng phân tử không phân cực.}
	{Các phân tử $H_2O$ và $NH_3$ có hình dạng giống nhau.}
	{\True Góc liên kết H-O-H trong $H_2O$ nhỏ hơn góc liên kết H-C-H trong $CH_4$.}
	\loigiai{
		\begin{itemchoice}[F1,T2,F3,T4]
			\itemch $H_2O$, $NH_3$, $CH_4$ có liên kết cộng hóa trị có cực.
			\itemch $CO_2$ có cấu tạo thẳng nên mặc dù có liên kết C=O phân cực nhưng lại là phân tử không phân cực.
			\itemch $H_2O$ có hình dạng gấp khúc, $NH_3$ có hình dạng hình chóp tam giác.
			\itemch Do ảnh hưởng của cặp electron hóa trị riêng trên nguyên tử O, góc liên kết H-O-H bị giảm xuống còn khoảng $104,5^o$, nhỏ hơn góc liên kết $109,5^o$ trong $CH_4$.
		\end{itemchoice}
	}
\end{ex}
%%%=============EX_4=============%%%
\begin{ex}
	Xét phân tử $NH_3$
	\choiceTF
	{\True Liên kết N-H trong $NH_3$ là liên kết cộng hóa trị phân cực.}
	{Khi tham gia liên kết hóa học N dùng 5 elctron hóa trị tạo liên kết với 3 nguyên tử H}
	{\True Trong phân tử $NH_3$ liên kết $\sigma$ $N-H$ hình thành do sự xen phủ trục của $1$ AO 2p trong N với 1 AO s của H}
	{\True Phân tử $NH_3$ có cấu trúc dạng chóp tam giác}
	\loigiai{
		\begin{itemchoice}[F1,T2,F3,T4]
			\itemch $|3,04 - 2,20| = 0,84 > 0,4.$ nên liên kết N-H là liên kết cộng hóa trị phân cực.
			\itemch Khi tham gia liên kết hóa học N dùng 3 elctron hóa trị tạo liên kết với 3 nguyên tử H
			\itemch Trong phân tử $NH_3$ liên kết $\sigma$ $N-H$ hình thành do sự xen phủ trục của $1$ AO 2p trong N với 1 AO s của H
			\itemch Phân tử $NH_3$ có cấu trúc dạng chóp tam giác.
		\end{itemchoice}
	}
\end{ex}
%%%=============EX_5=============%%%
\begin{ex}
	Liên kết cộng hóa trị là liên kết được hình thành giữa hai nguyên tử bằng một hay nhiều cặp electron dùng chung.
	\choiceTF
	{\True Trong phân tử $HCl$, giữa nguyên tử $H$ và nguyên tử $Cl$ có 1 cặp electron dùng chung}
	{\True Trong phân tử $O_2$, giữa hai nguyên tử $O$ có 2 cặp electron dùng chung}
	{\True Trong phân tử $N_2$, giữa hai nguyên tử $N$ có 3 cặp electron dùng chung}
	{\True Trong phân tử $CO_2$, giữa một nguyên tử $C$ và hai nguyên tử $O$ có 4 cặp electron dùng chung}
	\loigiai{}
\end{ex}
%%%=============EX_6=============%%%
\begin{ex}
	Cho các công thức: (1) \chemfig{H-[,0.65,,,draw=none]\charge{[.radius=0.2ex]0:1pt=\:,180:1pt=\:,90:2pt=\:,-90:2pt=\:}{O}-[,0.65,,,draw=none]H} ,\quad
	(2) \chemfig{O=C=O},\quad
	(3) \chemfig{\charge{[.radius=0.2ex]90:2pt=\:}{N}~\charge{[.radius=0.2ex]90:2pt=\:}{N}} ,\quad (4) \chemfig{H-\charge{[.radius=0.2ex]90:2pt=\:,-90:2pt=\:}{O}-H} ,\quad (5) \chemfig{\charge{[.radius=0.2ex]90:2pt=\:,-90:2pt=\:,0:2pt=\:}{O}-[,0.85,,,draw=none]\charge{[.radius=0.2ex]90:2pt=\:,-90:2pt=\:,180:2pt=\:}{O}}.
	\choiceTF
	{Công thức (1) và (3) là công thức electron}
	{\True Công thức (2) là công thức cấu tạo}
	{\True Công thức (3), (4) là công thức Lewis}
	{Công thức (1), (3), (4), (5) là công thức Lewis}
	\loigiai{}
\end{ex}
%%%=============EX_7=============%%%
\begin{ex}[CD - SGK]
	Cho các phát biểu:
	\choiceTF
	{Nếu cặp electron chung bị lệch về phía một nguyên tử thì đó là liên kết cộng hóa trị không cực}
	{\True Nếu cặp electron chung bị lệch về phía một nguyên tử thì đó là liên kết cộng hóa trị có cực}
	{Cặp electron chung luôn được tạo nên từ 2 electron của cùng một nguyên tử}
	{\True Cặp electron chung được tạo nên từ 2 electron hóa trị. Có bao nhiêu phát biểu đúng trong các phát biểu trên?}
	\loigiai{}
\end{ex}
%%%=============EX_8=============%%%
\begin{ex}
	Cho độ dài liên kết và năng lượng liên kết của một số liên kết trong bảng sau:
	\begin{table}[h]
		\centering
		\begin{tabular}{|c|c|c|c|}
			\hline
			& $C-C$ & $C=C$ & C$\equiv$C \\
			\hline
			Độ dài liên kết ($A^{o}$) & $1{,}54$ & $1{,}34$ & $1{,}20$ \\
			\hline
			Năng lượng liên kết ($kJ/mol$) & $347$ & $614$ & $839$ \\
			\hline
		\end{tabular}
		\caption{Thông tin về liên kết $C-C$, $C=C$ và C$\equiv$C}
		\label{bang_lienket_cacbon}
	\end{table}
	\choiceTF
	{\True Liên kết C$-$C có độ dài lớn nhất}
	{Liên kết C$=$C có năng lượng nhỏ nhất}
	{Liên kết C$\equiv$C có độ dài nhỏ nhất và năng lượng lớn nhất}
	{\True Liên kết có độ dài càng lớn thì năng lượng liên kết càng nhỏ và ngược lại}
	\loigiai{
		\begin{itemchoice}[T1,F2,F3,T4]
			\itemch
			\itemch vì $\mathrm{C} \equiv \mathrm{C}$ có năng lượng liên kết lớn nhất.
			\itemch vì liên kết có độ dài nhỏ nhất và năng lượng lớn nhất là liên kết $\mathrm{C} \equiv \mathrm{C}$.
			\itemch
		\end{itemchoice}
	}
\end{ex}
%%%=============EX_9=============%%%
\begin{ex}[CTST-SBT]
	Xét các phát biểu về độ bền của một liên kết.
	\choiceTF
	{Khi nhiều liên kết được hình thành giữa hai nguyên tử, độ bền của liên kết sẽ giảm}
	{Độ bền của liên kết tăng khi độ dài của liên kết tăng}
	{\True Độ bền của liên kết tăng khi độ dài của liên kết giảm}
	{Độ bền của liên kết không phụ thuộc vào độ dài liên kết}
	\loigiai{
		\begin{itemchoice}[F1,F2,T3,F4]
			\itemch Vì càng nhiều liên kết độ bền càng tăng. VD: Độ bền giảm: C $\equiv$ C $>$ C $=$ C $>$ C $-$ C
			\itemch Vì độ bền liên kết tăng khi độ dài liên kết giảm.
			\itemch
			\itemch Vì độ bền liên kết tỉ lệ nghịch với độ dài liên kết.
		\end{itemchoice}
	}
\end{ex}
%%%=============EX_10=============%%%
\begin{ex}
	Dựa vào độ âm điện người ta có thể phân loại liên kết thành liên kết ion, liên kết cộng hóa trị không phân cực, liên kết cộng hóa trị phân cực.
	\choiceTF
	{\True Liên kết cộng hóa trị không phân cực là liên kết cộng hóa trị trong đó cặp electron dùng chung không lệch về phía nguyên tử nào}
	{Liên kết cộng hóa trị phân cực là liên kết cộng hóa trị trong đó cặp electron dùng chung lệch về phía nguyên tử có độ âm điện nhỏ hơn}
	{Hiệu độ âm điện giữa hai nguyên tử từ $0$ đến $0,4$ thì liên kết thuộc loại cộng hóa trị phân cực}
	{\True Hiệu độ âm điện giữa hai nguyên tử lớn hơn hoặc bằng $1,7$ thì liên kết thuộc loại ion}
	\loigiai{
		\begin{itemchoice}[T1,F2,F3,T4]
			\itemch
			\itemch Vì trong liên kết cộng hóa trị phân cực thì cặp electron dùng chung lệch về phía nguyên tử có độ âm điện lớn hơn.
			\itemch Vì hiệu độ âm điện từ $0$ đến $0{,}4$ thì liên kết thuộc loại cộng hóa trị không phân cực.
			\itemch
		\end{itemchoice}
	}
\end{ex}
%%%=============EX_11=============%%%
\begin{ex}
	Nguyên tử của nguyên tố X có cấu hình electron $1s^2,2s^2,2p^6,3s^2,3p^6,4s^1$, nguyên tử của nguyên tố Y có cấu hình electron $1s^2,2s^2,2p^5$.
	\choiceTF
	{\True X thuộc chu kì 4, nhóm IA, là một kim loại}
	{Y thuộc chu kì 2, nhóm VA, là một phi kim}
	{\True Liên kết giữa X và Y thuộc loại liên kết ion}
	{Ở điều kiện thường, hợp chất tạo thành bởi X và Y ở trạng thái lỏng}
	\loigiai{
		\begin{itemchoice}[T1,F2,T3,F4]
			\itemch 
			\itemch vì Y thuộc nhóm VII
			\itemch vì X là một kim loại mạnh và Y là một phi kim mạnh.
			\itemch vì hợp chất tạo bởi X và Y thuộc loại hợp chất ion, là chất rắn ở điều kiện thường.
		\end{itemchoice}
	}
\end{ex}
%%%=============EX_12=============%%%
\begin{ex}[CTST-SBT]
	Xét tính chất của hợp chất cộng hóa trị.
	\choiceTF
	{\True Các hợp chất cộng hoá trị có nhiệt độ nóng chảy và nhiệt độ sôi thấp hơn các hợp chất ion}
	{\True Các hợp chất cộng hoá trị có thể ở thể rắn, lỏng hoặc khí trong điều kiện thường}
	{Các hợp chất cộng hoá trị đều dẫn điện tốt}
	{\True Các hợp chất cộng hoá trị không phân cực tan được trong dung môi không phân cực}
	\loigiai{
		\begin{itemchoice}[T1,T2,F3,T4]
			\itemch 
			\itemch
			\itemch vì hợp chất cộng hóa trị không phân cực thì không dẫn điện ở mọi nơi.
			\itemch
		\end{itemchoice}
	}
\end{ex}
%%%=============EX_13=============%%%
\begin{ex}[KNTT-SBT]
	Cho các chất: Nước, muối ăn, băng phiến ($C_{10}H_8$), butane ($C_4H_{10}$) và các giá trị nhiệt độ sôi của các chất trên không theo thứ tự $-138^oC$, $80^oC$, $0^oC$, $801^oC$.
	\choiceTF
	{\True Nhiệt độ nóng chảy của nước là $0^oC$}
	{Nhiệt độ nóng chảy của băng phiến là $-138^oC$}
	{Nhiệt độ nóng chảy của butane là $80^oC$}
	{\True Muối ăn có nhiệt độ nóng chảy cao nhất vì muối ăn (NaCl) là hợp chất ion}
	\loigiai{
		Muối ăn (NaCl) là hợp chất ion nên nhiệt độ nóng chảy cao nhất ($801^oC$), $H_2O$ nóng chảy ở $0^oC$, $C_{10}H_8$ phân tử khối lớn hơn $C_4H_{10}$ nên nhiệt độ nóng chảy cao hơn: $C_{10}H_8$ ($80^oC$), $C_4H_{10}$ ($-138^oC$).
		\begin{itemchoice}[T1,F2,F3,T4]
			\itemch
			\itemch vì băng phiến có nhiệt độ nóng chảy $80^oC$.
			\itemch vì butane có nhiệt độ nóng chảy $-138^oC$.
			\itemch 
		\end{itemchoice}
	}
\end{ex}
%%%=============EX_14=============%%%
\begin{ex}[CD-SBT]
	Xét các phát biểu về liên kết sigma ($\sigma$) và liên kết pi ($\pi$).
	\choiceTF
	{Chỉ có các AO có hình dạng giống nhau mới xen phủ với nhau để tạo liên kết}
	{\True Khi hình thành liên kết cộng hóa trị giữa hai nguyên tử, luôn có một liên kết $\sigma$}
	{\True Liên kết $\sigma$ bền vững hơn liên kết $\pi$}
	{\True Có hai kiểu xen phủ hình thành liên kết là xen phủ trục và xen phủ bên}
	\loigiai{
		\begin{itemchoice}[F1,T2,T3,T4]
			\itemch vì sự xen phủ có thể tạo bởi các AO có hình dạng khác nhau.
			\itemch
			\itemch
			\itemch
		\end{itemchoice}
	}
\end{ex}
%%%=============EX_15=============%%%
\begin{ex}[CTST-SBT]Xét các phát biểu về độ bền của một liên kết.
	\choiceTF
	{Khi nhiều liên kết được hình thành giữa hai nguyên tử, độ bền của liên kết sẽ giảm}
	{Độ bền của liên kết tăng khi độ dài của liên kết tăng}
	{\True Độ bền của liên kết tăng khi độ dài của liên kết giảm}
	{Độ bền của liên kết không phụ thuộc vào độ dài liên kết}
	\loigiai{
		\begin{itemchoice}[F1,F2,T3,F4]
			\itemch vì càng nhiều liên kết độ bền càng tăng. VD: Độ bền giảm: C $\equiv$ C $>$ C $=$ C $>$ C $-$ C
			\itemch vì độ bền liên kết tăng khi độ dài liên kết giảm.
			\itemch 
			\itemch vì độ bền liên kết tỉ lệ nghịch với độ dài liên kết.
		\end{itemchoice}
	}
\end{ex}
%%%=============EX_16=============%%%
\begin{ex}[CD-SBT]Cho biết năng lượng liên kết H$-$I và H$-$Br lần lượt là $297$ $kJ/mol$ và $364$ $kJ/mol$.
	\choiceTF
	{\True Khi đun nóng, HI bị phân hủy (thành $H_2$ và $I_2$) ở nhiệt độ thấp hơn so với HBr (thành $H_2$ và $Br_2$)}
	{\True Liên kết H$-$Br là bền vững hơn so với liên kết H$-$I}
	{Khi đun nóng, HI bị phân hủy (thành $H_2$ và $I_2$) ở nhiệt độ cao hơn so với HBr (thành $H_2$ và $Br_2$)}
	{\True Liên kết H$-$I dài hơn liên kết H$-$Br}
	\loigiai{
		\begin{itemchoice}[T1,T2,F3,T4]
			\itemch 
			\itemch 
			\itemch vì liên kết H$-$I năng lượng thấp hơn liên kết H$-$Br nên nhiệt độ phân hủy thấp hơn.
			\itemch vì năng lượng liên kết tỉ lệ nghịch với độ dài liên kết.
		\end{itemchoice}
	}
\end{ex}
\Closesolutionfile{ans}
\Closesolutionfile{ansbook}
\Closesolutionfile{ansex}
%\bangdapanTF{AnsTF-C03_B03_LIEN_KET_CONG_HOA_TRI.TEX}
\phan{Bài tập trả lời ngắn}
%%%=============SOẠN BT===============%%%
\Opensolutionfile{ansbth}[Ans/LGBT-C03_B03_LIEN_KET_CONG_HOA_TRI.tex]
\Opensolutionfile{ansbt}[Ans/AnsBT-C03_B03_LIEN_KET_CONG_HOA_TRI.tex]
%%%==============Bai_BT1==============%%%
\begin{bt}[CTST-SBT] Trong phân tử ammonia ($NH_3$), số cặp electron chung giữa nguyên tử nitrogen và các nguyên tử hydrogen là bao nhiêu?
	\shortans{3}
	\loigiai{}
\end{bt}
%%%==============HetBai_BT1==============%%%

%%%==============Bai_BT2==============%%%
\begin{bt}
	Trong phân tử methane ($CH_4$), số cặp electron chung giữa nguyên tử carbon và các nguyên tử hydrogen là bao nhiêu?
	\shortans{4}
	\loigiai{}
\end{bt}
%%%==============HetBai_BT2==============%%%

%%%==============Bai_BT3==============%%%
\begin{bt}
	Cho các hợp chất sau: $Na_2O$, $H_2$, $H_2O$, $HCl$, $Cl_2$, $O_3$. Có bao nhiêu chất mà trong phân tử chứa liên kết cộng hóa trị không phân cực?
	\shortans{3} 
	\loigiai{Bao gồm: $H_2$, $Cl_2$, $O_3$.}
\end{bt}
%%%==============HetBai_BT3==============%%%

%%%==============Bai_BT4==============%%%
\begin{bt}
	Cho dãy các chất: $N_2$, $H_2$, $NH_3$, $NaCl$, $HCl$, $H_2O$. Có bao nhiêu chất trong dãy mà phân tử chỉ chứa liên kết cộng hóa trị phân cực?
	\shortans{3} 
	\loigiai{Bao gồm: $NH_3$, $HCl$, $H_2O$.}
\end{bt}
%%%==============HetBai_BT4==============%%%

%%%==============Bai_BT5==============%%%
\begin{bt}
	Cho các phân tử: $H_2$; $CO_2$; $Cl_2$; $N_2$; $I_2$; $C_2H_4$; $C_2H_2$. Có bao nhiêu phân tử có liên kết ba trong phân tử?
	\shortans{2} 
	\loigiai{Bao gồm: $N_2$ ($N\equiv N$) và $C_2H_2$ ($H-C\equiv C-H$).}
\end{bt}
%%%==============HetBai_BT5==============%%%

%%%==============Bai_BT6==============%%%
\begin{bt}[KNTT-SGK] Tổng số liên kết $\sigma$ và $\pi$ có trong phân tử $C_2H_4$ là bao nhiêu?
	\shortans{6} 
	\loigiai{Gồm 5 liên kết $\sigma$ và 1 liên kết $\pi$}
\end{bt}
%%%==============HetBai_BT6==============%%%

%%%==============Bai_BT7==============%%%
\begin{bt}
	Trong các phân tử: $CO_2$, $NH_3$, $C_2H_2$, $SO_2$, $H_2O$ có bao nhiêu phân tử phân cực?
	\shortans{3} 
	\loigiai{Bao gồm: $NH_3$, $SO_2$, $H_2O$.}
\end{bt}
%%%==============HetBai_BT7==============%%%

%%%==============Bai_BT8==============%%%
\begin{bt}[CD-SBT] Số obital của cả hai nguyên tử N tham gia xen phủ tạo liên kết trong phân tử $N_2$ là bao nhiêu?
	\shortans{6} 
	\loigiai{$N$ ($Z=7$): $1s^22s^22p^3$: $\squarerow[2ud][0.5][\maunhan]{1}$ $\squarerow[2ud][0.5][\maunhan]{1}$ $\squarerow[1u,1u,1u][0.5][\maunhan]{3}$ $\Rightarrow$ Mỗi nguyên tử N mang 3 AO p ra xen phủ $\Rightarrow$ tổng 2 nguyên tử N là 6 AO.}
\end{bt}
%%%==============HetBai_BT8==============%%%
%%%==============Bai_BT9==============%%%
\begin{bt}[CD-SGK] Cho các phát biểu:
	\begin{enumerate}[a)]
		\item Nếu cặp electron chung bị lệch về phía một nguyên tử thì đó là liên kết cộng hóa trị không cực.
		\item Nếu cặp electron chung bị lệch về phía một nguyên tử thì đó là liên kết cộng hóa trị có cực.
		\item Cặp electron chung luôn được tạo nên từ 2 electron của cùng một nguyên tử.
		\item Cặp electron chung được tạo nên từ 2 electron hóa trị. Có bao nhiêu phát biểu đúng trong các phát biểu trên?
	\end{enumerate}
	\shortans{2} 
	\loigiai{Bao gồm: b, d.
		\begin{enumerate}
			\item Sai vì cặp electron dùng chung bị lệch về một phía nguyên tử thì đó là liên kết cộng hóa trị có cực.
			\item Sai vì chỉ trong liên kết cho – nhận thì cặp e dùng chung mới của cùng một nguyên tử.
	\end{enumerate}}
\end{bt}
%%%==============HetBai_BT9==============%%%

%%%==============Bai_BT10==============%%%
\begin{bt}[CD-SBT] Cho các phát biểu sau về phân tử $CO_2$:
	\begin{enumerate}[a)]
		\item Liên kết giữa hai nguyên tử C và O là liên kết cộng hoá trị không phân cực
		\item Liên kết giữa hai nguyên tử C và O là liên kết cộng hoá trị phân cực
		\item Phân tử $CO_2$ có 4 electron hoá trị riêng.
		\item Phân tử $CO_2$ có 4 cặp electron hoá trị riêng.
		\item Trong phân tử $CO_2$ có 3 liên kết $\sigma$ và 1 liên kết $\pi$
		\item Trong phân tử $CO_2$ có 2 liên kết $\sigma$ và 2 liên kết $\pi$
		\item Trong phân tử $CO_2$ có 1 liên kết $\sigma$ và 3 liên kết $\pi$
	\end{enumerate}
	Có bao nhiêu phát biểu không đúng trong các phát biểu trên?
	\shortans{4} 
	\loigiai{Bao gồm: a, c, e, h.
		$CO_2: \ddot{O}=C=\ddot{O}$: Trong $CO_2$: Liên kết C – O là liên kết cộng hóa trị phân cực; có 4 cặp electron hóa trị đã ghép đôi nhưng chưa tham gia liên kết (cặp electron hóa trị riêng), có 2 liên kết $\sigma$ và 2 liên kết $\pi$.}
\end{bt}
%%%==============HetBai_BT10==============%%%
\Closesolutionfile{ansbt}
\Closesolutionfile{ansbth}
%\bangdapanSA{AnsBT-C03_B03_LIEN_KET_CONG_HOA_TRI.tex}
	\section{Liên kết hydrogen và tương~tác~van~der~waals}
\begin{Muctieu}
	\begin{itemize}
		\item Trình bày được khái niệm liên kết hydrogen. Vận dụng để giải thích được sự xuất hiện liên kết hydrogen.
		\item Nêu được vai trò, ảnh hưởng của liên kết hydrogen tới tính chắt vật lí của nước.
		\item Nêu được khái niệm vế tương tác van der Waals và ảnh hưởng của tương tác này tới nhiệt độ nóng chảy, nhiệt độ sôi của các chất.
	\end{itemize}
\end{Muctieu}
\begin{kd}
	\immini{Hãy tưởng tượng một con tắc kè đang leo trèo thoăn thoắt trên bức tường nhẵn bóng.  Bàn chân của chúng có hàng triệu sợi lông siêu nhỏ,  tạo ra một lực hút với bề mặt tường đủ mạnh để chống lại trọng lực.  Đó chính là sức mạnh của tương tác Van der Waals.
		
		\lq\lq Tương tác Van der Waals là một trong những lực liên kết yếu tồn tại giữa các phân tử. Ngoài ra, còn một loại lực liên kết yếu khác nữa là liên kết Hydrogen.\rq\rq
		
		\lq\lq Trong bài này, chúng ta sẽ cùng nhau tìm hiểu về bản chất và vai trò của hai loại lực này trong việc quyết định tính chất của các chất và các hiện tượng trong tự nhiên.\rq\rq}{\includegraphics[width=6cm]{Images/anhhoahoc10/tackehoa.png}}
\end{kd}
\subsection{Nội dung bài học}
\subsubsection{Liên kết hydrogen}
	\Noibat[\maunhan][][\faStar][]{Tìm hiểu về liên kết hydrogen}
	%%%Liên kết H giữa các phân tử H20
	\begin{center}
		\resizebox{!}{3cm}{
			\begin{tikzpicture}[%
			line cap=round,line join=round,declare function={r=1.5cm;}
			]
			\tikzstyle{element_style} = [inner sep=2pt,font=\large\bfseries\fontfamily{qag}\selectfont]
			\tikzset{
				water/.pic={
					\path (0,0) coordinate (A) 
					($(A)+(-135:r)$) coordinate (B)
					($(A)+(-45:r)$) coordinate (C)
					;
					\path (A) node[element_style] (O) {O}
					(B) node[element_style] (Ha) {H}
					(C) node[element_style] (Hb) {H}
					;
					\draw (Ha)--(O)--(Hb);
					\path ($(A)+(3pt,0)$) node [above=3pt,text=\maunhan,font=\scriptsize] {$\sigma^-$};
					\path ($(B)+(135:11pt)$) node [text=\maunhan,font=\scriptsize] {$\sigma^+$};
					\path ($(C)+(45:15pt)$) node [text=\maunhan,font=\scriptsize] {$\sigma^+$};
				}
			}
			\path (0,0) pic[local bounding box=a] {water};
			\path (5,0) pic[local bounding box=b] {water};
			\path (2.5,-2.5) pic[local bounding box=c] {water};
			\path ($(a.south east)+(-0.4cm,-1.2pt)$)--($(c.north)+(-135:14pt)$) node[pos=0.5,sloped,midway] (lkHa) {
				\tikz{\fill[\maunhan] (0,0)circle(2pt)(8pt,0)circle(2pt)(16pt,0)circle(2pt);}
			};
			\path ($(b.south west)+(0.3cm,-1.3pt)$)--($(c.north)+(-45:12pt)$) node[pos=0.5,sloped,midway] (lkHb) {
				\tikz{\fill[\maunhan] (0,0)circle(2pt)(8pt,0)circle(2pt)(16pt,0)circle(2pt);}
			};
			\path ($(lkHb)+(1,-0.3)$) node[right] (n) {liên kết hydrogen};
			\draw[-latex] (n.west)--(lkHb);
		\end{tikzpicture}
		}
		\captionof{figure}{Liên kết hydrogen giữa các phân tử nước}
	\end{center}
	%%%Liên kết H giữa các phân tử NH3
	\begin{center}
		\resizebox{!}{3cm}{
			\begin{tikzpicture}[%
			line cap=round,line join=round,declare function={r=1.7cm;d=3.25cm;}
			]
			\tikzstyle{element_style} = [inner sep=2pt,font=\large\bfseries\fontfamily{qag}\selectfont]
			\tikzset{
				ammonia/.pic={
					\path [pic actions](0,0) coordinate (A) node[element_style](Nitrogen) {N};
					\path [pic actions]($(Nitrogen.north)+(4pt,7pt)$) node[text=\maunhan] {$\sigma^-$};
					\foreach \g/\n/\j/\gh in{120/a/a/120,180/b/b/90,-120/c/c/-120}{
						\path ($(A)+(\g:r)$) coordinate (\n) node[element_style](H\j){H};
						\draw (Nitrogen)--(H\j) node [font=\scriptsize,text=\maunhan,shift={(\gh:12pt)}] {$\sigma^+$};
					}
					
				}
			}
			\path (-0.20*d,0) node[text width=2cm,inner sep =6pt] (BD) {\phantom{A}};
			\path (1*d,0) pic[local bounding box=AmoniacM] {ammonia};
			\path (2*d,0) pic[local bounding box=AmoniacH] {ammonia};
			\path (3*d,0) pic[local bounding box=AmoniacB] {ammonia};
			\path (3.7*d,0) node[text width=2cm,inner sep =6pt](KT) {\phantom{A}};
			\foreach \x/\y in{BD/AmoniacM,AmoniacM/AmoniacH,AmoniacH/AmoniacB,AmoniacB/KT}{
				\path (\x.east)--(\y.west) node[pos=0.5,sloped,midway,xshift=-3pt] {
					\tikz{\fill[\maunhan] (0,0)circle(2pt)  (8pt,0)circle(2pt) (16pt,0)circle(2pt);}
				};
			}
		\end{tikzpicture}
		}
		\captionof{figure}{Liên kết hydrogen giữa các phân tử ammonia}
	\end{center}
	\begin{tomtat}
		\indam[\maunhan]{Liên kết hydrogen} là một loại liên kết yếu được hình thành giữa nguyên tử H (đã liên kết với một nguyên tử có độ âm điện lớn) với một nguyên tử khác (có độ âm điện lớn) còn cặp electron riêng. Các nguyên tư có độ âm điện lớn thường găp trong liên kết hydrogen là $N, O, F$.
	\end{tomtat}
	\begin{ghinho}
		Điều kiện cần và đủ để tạo thành liên kết hydrogen:
		\begin{itemize}
			\item Nguyên tử hydrogen liên kết với các nguyên tử có độ âm điện lớn như F, $\mathrm{O}, \mathrm{N}, \ldots$
			\item Nguyên tử $F, O, N, \ldots$ liên kết với hydrogen phải có ít nhất một cặp electron hoá trị chưa liên kết.
		\end{itemize}
	\end{ghinho}
	\Noibat[\maunhan][][\faStar][]{Tìm hiểu vai trò, ảnh hưởng của liên kết hydrogen tới tính chất vật lí của nước}
	\begin{tomtat}
		\begin{itemize}
			\item Nhờ có liên kết hydrogen mà ở điểu kiện thường nước ở thể lỏng, có nhiệt độ sôi cao $\left(100^{\circ} \mathrm{C}\right)$.
			\item Nước còn là một dung môi tốt, không chỉ hòa tan được nhiều hợp chất ion, mà còn hòa tan được nhiều hợp chất có liên kết cộng hóa trị phân cực. Đặc biệt, các hợp chất có thể tạo liên kết hydrogen với nước thường tan tốt trong nước.
		\end{itemize}
	\end{tomtat}
	\vspace{0.5cm}
	\begin{center}
		\resizebox{!}{3.0cm}{
			\begin{tikzpicture}[line cap=round,line join=round,declare function={r=1.7cm;d=3.5cm;}
			]
			\def\hydrogenbond{\tikz{\fill[\maunhan] (0,0)circle(2pt)  (8pt,0)circle(2pt) (16pt,0)circle(2pt);}}
			%%%
			\tikzset{
				element_style/.style={inner sep=2pt,font=\large\bfseries\fontfamily{qag}\selectfont},
				pics/Compound/.style args={#1/#2/#3}{
					code={
						%\begin{scope}[transform canvas={rotate around x=90}]
						\path [pic actions] (0,0) coordinate (A) node[element_style] (Oxigen) {O};
						\path [pic actions] ($(Oxigen)+(-4pt,12pt)$) node[text=\maunhan] {$\sigma^-$};
						\path [pic actions] ($(A)+(180:r)$) coordinate (B) node[element_style] (atomM) {H};
						\path [pic actions] ($(atomM)+(0pt,12pt)$) node[text=\maunhan] {$\sigma^+$};
						\path [pic actions] ($(A)+(#3:r)$) coordinate (C) node[element_style] (atomH) {#1};
						\path [pic actions] ($(atomH)+(5pt,12pt)$) node[text=\maunhan] {#2};
						\draw (atomM)--(Oxigen)--(atomH);
						%\end{scope}
					}
				}
			}
			% Draw the compound
			\path (-0.25*d,0) node[text width=2cm,inner sep =6pt] (BD) {\phantom{A}};
			\path (d,0) pic[local bounding box=compoundM] {Compound=R/\phantom{X}/-60}
			(2*d,0) pic[local bounding box=compoundT]{Compound=H/$\sigma^+$/-60}
			(3*d-0.85cm,-1.45cm) pic[local bounding box=compoundB,rotate=180]{Compound=R/\phantom{X}/60}
			(4*d-0.75cm,-1.45cm) pic[local bounding box=compoundF,rotate=180]{Compound=H/$\sigma^+$/60}
			;
			\path (5.2*d,0) node[text width=2cm,inner sep =6pt] (KT) {\phantom{A}};
			%%%Vẽ Lien Ket H
				\path (BD.east)--($(compoundM.west)+(0,0.5cm)$) node[pos=0.5,sloped,midway,xshift=1pt] {\hydrogenbond};
				\path ($(compoundM.east)+(0,0.5cm)$)--($(compoundT.west)+(0,0.5cm)$) node[pos=0.5,sloped,midway,xshift=-12pt] {\hydrogenbond};
				\path ($(compoundT.south east)+(0,8pt)$)--($(compoundB.north west)+(8pt,-7.65pt)$) node[pos=0.5,sloped,midway,xshift=5pt] {\hydrogenbond};
				\path ($(compoundB.north east)+(8pt,-7.65pt)$)--($(compoundF.north west)+(8pt,-7.65pt)$) node[pos=0.5,sloped,midway,xshift=5pt] {\hydrogenbond};
				\path ($(compoundF.north east)+(-1cm,-7.65pt)$)--($(KT.west)+(0pt,-1.46cm)$) node[pos=0.5,sloped,midway,xshift=5pt] {\hydrogenbond};
		\end{tikzpicture}
	}
	\captionof{figure}{Liên kết hydrogen giữa  nước và rượu}
	\end{center}
	%%%
	\begin{center}
		\includegraphics[width=12cm]{Images/anhhoahoc10/LienketHydrogen/NH3_Hydrogen_bond.png}
		\captionof{figure}{Liên kết hydrogen giữa  nước và Ammonia}
	\end{center}
	\begin{Bancobiet}
		Nước ở trạng thái rắn có thể tích lớn hơn khi ở trạng thái lỏng. Đó là do nước đá có cấu trúc tinh thể phân tử với bốn phân tử $\mathrm{H}_2 \mathrm{O}$ phân bố ở bốn đỉnh của một tứ diện đều, bên trong là cấu trúc rỗng (Hình \ref{fig:nuocda} ). Điều này lí giải tại sao nước đá nổi được trên mặt nước lỏng.
		\begin{center}
			\includegraphics[width=8cm]{Images/anhhoahoc10/LienketHydrogen/nuocda.png}
			\captionof{figure}{Cấu trúc tinh thể phân tử nước đá\label{fig:nuocda}}
		\end{center}
	\end{Bancobiet}
\subsubsection{Tương tác van der waals}
	\Noibat[\maunhan][][\faStar][]{Giới thiệu về tương tác van der Waals (van đơ Van)}
	\begin{center}
		\includegraphics[width=6cm]{Images/anhhoahoc10/LienketHydrogen/luongcuctamthoi.png}
		\captionof{figure}{Lưỡng cực tạm thời được hình thành do sự phân bố không đống đếu của các electron trong phân tử}
	\end{center}
	\begin{center}
		\includegraphics[width=6cm]{Images/anhhoahoc10/LienketHydrogen/luong_cuc_cam_ung.png}
		\captionof{figure}{Mạng lưới tương tác lưỡng cực cảm ứng được tạo thành bởi lưởng cực tạm thời}
	\end{center}
	\vspace{0.25cm}
	\begin{tomtat}
		\indam[\maunhan]{Tương tác van der Waals} là lực tương tác yếu giửa các phân tử, được hình thành do sự xuất hiện của các lưỡng cực tạm thời và lưỡng cực cảm ứng.
	\end{tomtat}
	\Noibat[\maunhan][][\faStar][]{Tìm hiểu ảnh hưởng của tương tác van der Waals đến nhiệt độ nóng chảy và nhiệt độ sôi các chất}
	\begin{tomtat}
		\indam[\maunhan]{Tương tác van der Waals} làm tăng nhiệt độ nóng chảy và nhiệt độ sôi của các chất. Khi khối lượng phân tử tăng, kích thước phân tử tăng thì tương tác van der Waals tăng.
	\end{tomtat}
	
\subsection{Bài tập}
\begin{dang}{Lý thuyết về liên kết hydrogen và tương tác Vanderwalls}
	\begin{pp}
		Liên kết hydrogen hình thành khi nguyên tử H liên kết cộng hóa trị với một nguyên tử có độ âm điện lớn nên nguyên tử hydro mang $\delta^{+}$ (X-H với X là F, O, N) tương tác tĩnh điện yếu với nguyên tử Y có độ âm điện lớn và còn cặp electron tự do (Y là F, O, N).
		\begin{center}
			\begin{tikzpicture}[node distance=1.5cm]
				\node (X) {$\textbf{X}^{\sigma^{-}}$};
				\node[right=of X] (H) {$\textbf{H}^{\sigma^{+}}$};
				\node[right= 0.85cm of H] (Y) {$\textbf{Y}^{\sigma^{-}}$};
				\path [draw=\maunhan,thick](X)--(H) node [pos=0.25] {\tikz\fill[\maunhan] (0,0)--++(-150:5pt) coordinate (Goc)--++(-30:5pt)--([xshift=2.5pt]Goc)--cycle;};
				\path (H)--(Y) node[pos=0.5] (lkH){
					\tikz\fill[\maunhan] (0,0) circle(1.5pt) (6pt,0) circle(1.5pt) (12pt,0) circle(1.5pt);
					};
				\node[below=1cm of lkH,font=\small,xshift=0.2cm](text){Liên kết hydrogen};
				\draw [-stealth] (text)--(lkH);
			\end{tikzpicture}
		\end{center}
		\begin{itemize}
			\item  Liên kết hydrogen là một liên kết yếu, biểu diễn bằng dấu 3 chấm \lq\lq $\ldots$  \rq\rq.
			\item  Độ mạnh của liên kết hydrogen phụ thuộc vào độ phân cực của liên kết H-X và mật độ electron (hoặc độ âm điện) của nguyên tử Y.
			\begin{itemize}
				\item  Liên kết X-H càng phân cực thì liên kết hydrogen càng bền vững.
				\item  Nguyên tử Y có mật độ electron (hoặc độ âm điện) càng lớn thì liên kết hydrogen càng bền vững.
			\end{itemize}
			\item  Liên kết hydrogen làm tăng nhiệt độ nóng chảy, nhiệt độ sôi, tăng sức căng bề mặt, độ tan.
			\item  Tương tác van der Waals là tương tác tĩnh điện lưỡng cực - lưỡng cực được hình thành giữa các phân tử hay nguyên tử.
			\item  Tương tác van der Waals phụ thuộc vào hai yếu tố chính:
			\begin{itemize}
				\item  Số lượng electron (số proton) trong nguyên tử.
				\item  Điểm tiếp xúc giữa các phân tử.
			\end{itemize}
			\item  Tương tác van der Waals làm tăng nhiệt độ nóng chảy, nhiệt độ sôi của các chất và giải thích trạng thái tồn tại của các chất.
			\item  Độ mạnh (độ bền) theo thứ tự: liên kết ion $>$ liên kết cộng hóa trị $>$ liên kết hydrogen $>$ tương tác van der Waals.
		\end{itemize}
	\end{pp}
\end{dang}
\phan{Bài tập tự luận}
%=============SOẠN BT===============%%%
\Opensolutionfile{ansbth}[Ans/LGBT-C03B04_LKH_TTVANDERWALLS]
\Opensolutionfile{ansbt}[Ans/AnsBT-C03B04_LKH_TTVANDERWALLS]
%%%=========Bài tập tự luận sách thm khảo=====================%%%
%=============BT_1=============%%%
\begin{bt}
	Vẽ sơ đồ biểu diễn liên kết hydrogen giữa
	\begin{enumerate}
		\item Hai phân tử hydrogen fluoride ($HF$).
		\item Hai phân tử ammonia ($NH_3$).
	\end{enumerate}
	\loigiai{%
			\begin{enumerate}
			\item Nguyên tử H trong phân tử $HF$ rất linh động, có điện tích dương.($\sigma^+$) đủ lớn để hút cặp electron hóa trị chưa liên kết trên nguyên tử F có độ âm điện điện lớn tạo liên kếthydrogen
			\begin{center}
				\tikz[line cap=round,line join=round,inner sep=3pt]{
					\tikzstyle{style_text} = [font=\Large]
					\def\lienketH{\tikz{\fill[\maunhan] (0,0) circle (1.5pt) (6pt,0) circle (1.5pt) (12pt,0 )circle (1.5pt);}}
					\node[style_text] (Fa) {F};
					\node [text=\maunhan] at ($(Fa.north)+(2pt,3pt)$) {$\sigma^-$};
					\node [left=1cm of Fa,style_text] (Ha){H};
					\node [right=1.2cm of Fa,style_text] (Hb){H}; 
					\node [text=\maunhan] at ($(Hb.north)+(2pt,3pt)$) {$\sigma^+$};
					\node [right=1.0cm of Hb,style_text] (Fb){F}; 
					\draw(Ha)--(Fa)(Hb)--(Fb);
					\node [left=0.2pt of Ha] {\lienketH};
					\node [right=0.2pt of Fb] {\lienketH};
					\node  at ($(Fa)!0.5!(Hb)$)  {\lienketH};			
				}
			\end{center}
			\item Liên kết hydrogen giữa hai phân tử $NH_3$.
				\begin{center}
					\tikz[line cap=round,line join=round,inner sep=3pt]{
						\tikzstyle{style_text} = [font=\Large]
						\def\lienketH{\tikz{\fill[\maunhan] (0,0) circle (1.5pt) (6pt,0) circle (1.5pt) (12pt,0 )circle (1.5pt);}}
						\node[style_text] (Na) {N};
						\node [text=\maunhan] at ($(Na.north)+(12pt,3pt)$) {$\sigma^-$};
						\node [left=1cm of Na,style_text] (Ha){H};
						\node [right=1.2cm of Na,style_text] (Hb){H}; 
						\node [text=\maunhan] at ($(Hb.north)+(2pt,3pt)$) {$\sigma^+$};
						\node [right=1.0cm of Hb,style_text] (Nb){N}; 
						\draw(Ha)--(Na)(Hb)--(Nb);
						\node [left=0.2pt of Ha] {\lienketH};
						\node [right=0.2pt of Nb] {\lienketH};
						\node  at ($(Na)!0.5!(Hb)$)  {\lienketH};
						\draw (Na)--++(90:1cm)node[anchor=south] {H}
						(Na)--++(-90:1cm)node[anchor=north] {H}
						(Nb)--++(-90:1cm)node[anchor=north] {H}
						(Nb)--++(90:1cm)node[anchor=south] {H}
						;		
					}
				\end{center}
		\end{enumerate}
	}
\end{bt}
%=============BT_2=============%%%
\begin{bt}
	Vẽ sơ đồ biểu diễn liên kết hydrogen giữa
	\begin{enumerate}
		\item Hai phân tử $H_2O$.
		\item Phân tử hydrogen fluoride ($HF$) và phân tử nước.
	\end{enumerate}
	\loigiai{%
		\begin{enumerate}
		\item Liên kết hydrogen giữa hai phân tử $H_2O$.
			\begin{center}
				\begin{tikzpicture}[declare function={r=1;}, line cap=round,line join=round,inner sep=2pt,font=\Large]
				\def\lienketH{\tikz{\fill[\maunhan] (0,0) circle (1.5pt) (6pt,0) circle (1.5pt) (12pt,0 )circle (1.5pt);}}
				\path node (O) {O};
				\draw (O)--++(0:r) node [anchor=west] (Ha){H}
				(O)--++(-135:r) node [anchor=north east] (Hb){H}
				;
				\path node [right=2.25cm of O] (Oh) {O};
				\draw (Oh)--++(0:r) node [anchor=west] (Hc){H}
				(Oh)--++(-135:r) node [anchor=north east] (Hd){H}
				;
				\path ($(Ha)!0.5!(Oh)$)  node {\lienketH};
				\path (Oh.north)  node[anchor=south,font=\small,text=\maunhan] {$\sigma^{-}$};
				\path (Ha.north)  node[anchor=south,font=\small,text=\maunhan] {$\sigma^{+}$};
				\end{tikzpicture}
			\end{center}
		\item Liên kết hydrogen giữa hai phân tử hydrogen fluoride ($HF$) và phân tử nước.
			\begin{center}
				\begin{tikzpicture}[declare function={r=1;}, line cap=round,line join=round,inner sep=2pt,font=\Large]
					\def\lienketH{\tikz{\fill[\maunhan] (0,0) circle (1.5pt) (6pt,0) circle (1.5pt) (12pt,0 )circle (1.5pt);}}
					\path node (O) {O};
					\draw (O)--++(180:r) node [anchor=east] (Ha){H}
					(O)--++(-45:r) node [anchor=north west] (Hb){H}
					;
					\path node [right=1.25cm of O] (Hc) {H};
					\draw (Hc)--++(0:r) node [anchor=west] (F){F}
					;
					\path node [right=2.5cm of F] (Oh) {O};
					\draw (Oh)--++(180:r) node [anchor=east] (Hd){H}
					(Oh)--++(-45:r) node [anchor=north west] (He){H}
					;
					\foreach \x/\t in {Oh/-,Ha/+,O/-,Hc/+,F/-,Hd/+}{
						\path (\x.north)  node[anchor=south,font=\small,text=\maunhan] {$\sigma^{\t}$};
					}
					\path ($(O)!0.5!(Hc)$)  node {\lienketH}
					($(F)!0.5!(Hd)$)  node {\lienketH}
					node[left=0.1cm of Ha] {\lienketH}
					node[right=0.1cm of Oh] {\lienketH}
					;
					
				\end{tikzpicture}
			\end{center}
		\end{enumerate}
	}
\end{bt}
%=============BT_3=============%%%
\begin{bt}
	Giải thích vì sao $H_2O$ có phân tử khối $(18)$ nhỏ hơn $H_2S$ $(34)$ nhưng nhiệt độ nóng chảy và nhiệt độ sôi của $H_2O$ lại cao hơn phân tử $H_2S$?
	Bảng nhiệt độ nóng chảy và nhiệt độ sôi của $H_2O$ và $H_2S$ tại áp suất $1$ bar
	\begin{center}
		\begin{tabular}{|c|c|c|c|}
		\hline
		\text{Chất} & \text{Khối lượng phân tử} & \text{Nhiệt độ nóng chảy ($^\circ C$)} & \text{Nhiệt độ sôi ($^\circ C$)} \\
		\hline
		$H_2O$& 18& 0& 100\\
		\hline
		$H_2S$& 34&$-82{,}3$&$-60{,}3$\\
		\hline
		\end{tabular}
	\end{center}
	\loigiai{%
		Nhiệt độ nóng chảy, nhiệt độ sôi của một chất phụ thuộc chính vào hai yếu tố:
		\begin{itemize}
			\item  Khối lượng phân tử và liên kết giữa các phân tử.
			\item  Khối lượng phân tử càng lớn, nhiệt độ nóng chảy và nhiệt độ sôi càng cao.
			\item  Liên kết giữa các phân tử càng mạnh thì nhiệt độ nóng chảy và nhiệt độ sôi càng cao.
		\end{itemize}
		
		Mặc dù phân tử \( \mathrm{H}_2 \mathrm{O} \) có khối lượng phân tử nhỏ hơn phân tử \( \mathrm{H}_2 \mathrm{S} \), nhưng liên kết giữa các phân tử \( \mathrm{H}_2 \mathrm{O} \) lại mạnh hơn nhờ có liên kết hydrogen. Vì vậy, nhiệt độ nóng chảy và nhiệt độ sôi của \( \mathrm{H}_2 \mathrm{O} \) cao hơn \( \mathrm{H}_2 \mathrm{S} \).
		
		\begin{center}
			\begin{tikzpicture}[declare function={r=1;},line join=round,line cap=round,font=\Large]
				\def\lienketH{\tikz{\fill[\maunhan] (0,0) circle (1.5pt) (6pt,0) circle (1.5pt) (12pt,0 )circle (1.5pt);}}
				\path node (O) {O};
				\tikzset{
					nuoc/.pic={
						\path node[inner sep=2pt] (O){O};
						\draw (O)--++(-45:r) node[anchor=north west](Ha) {H};
						\draw (O)--++(-135:r) node[anchor=north east](Hb) {H};
						
						\foreach \x/\t in {O/-,Ha/+,Hb/+}{
							\path (\x.north)  node[anchor=south,font=\small,text=\maunhan,shift={(-90:4pt)}] {$\sigma^{\t}$};
						}
					}
				}
				\path (0,0) coordinate (A) pic[local bounding box=a,name prefix=a-]{nuoc};
				\path (A)--++(135:3cm) coordinate (B) pic[local bounding box=b,name prefix=b-,anchor=south east]{nuoc};
				\path (A)--++(45:3cm) coordinate (C) pic[local bounding box=c,name prefix=c-,anchor=south west]{nuoc};
				%
				\path (a-O)--(b-Ha) node [sloped,midway]{\lienketH};
				\path (a-O)--(c-Hb) node [sloped,midway]{\lienketH};
			\end{tikzpicture}
			\captionof{figure}{Liên kết hydrogen giữa các phân tử \(\mathrm{H}_2 \mathrm{O} \)}
		\end{center}
	}
\end{bt}
%=============BT_4=============%%%
\begin{bt}
	Hãy giải thích sự tăng dần nhiệt độ nóng chảy, nhiệt độ sôi của các khí hiếm?
	Cho bảng sau về nhiệt độ nóng chảy, nhiệt độ sôi của các khí hiếm
	\\
	\begin{tabular}{|c|c|c|c|c|c|c|}
		\hline
		\text{Halogen} & \text{He} & \text{Ne} & \text{Ar} & \text{Xe} & \text{Kr} & \text{Rn} \\
		\hline
		\text{Nhiệt độ sôi ($^\circ C$)} &-269&-246&-186&-152&-108&-62\\
		\hline
		\text{Nhiệt độ nóng chảy ($^\circ C$)} &-272&-247&-189&-157&-119&-71\\
		\hline
	\end{tabular}
	\loigiai{Đi từ He đến Rn, số lượng electron trong nguyên tử tăng dần làm cho tương tác van der Waals tăng dần, do đó nhiệt độ nóng chảy, nhiệt độ sôi tăng dần}
\end{bt}
%=============BT_5=============%%%
\begin{bt}
	Hãy giải thích vì sao butane có nhiệt độ sôi $(-0,5^\circ C)$ cao hơn so với isobutan $(-11,7^\circ C)$?
	\loigiai{Do diện tích tiếp xúc giữa các phân tử butan lớn hơn nhiều so với isobutan làm cho tương tác van der Waals tăng nên nhiệt độ sôi của butan cao hơn so với isobutan.
	
	\begin{center}
		\resizebox{!}{3.5cm}{
				\begin{tikzpicture}[declare function={r=2.5;},line join=round,line cap=round,font=\Large,inner sep=1pt]
			\def\vanderwallsforce{\tikz{\draw[decorate, decoration={snake, amplitude=1mm, segment length=1mm},\maunhan] (0,0)--(1,0);}}
			\tikzset{
				n-butan/.pic={
					\path node (metyl-1) {$CH_3$}
					++(30:r ) node (etyl-1) {$CH_2$}
					++(-30:r) node (etyl-2) {$CH_2$}
					++(30:r) node (metyl-2) {$CH_3$}
					;
					\draw[\maunhan,thick](metyl-1)--(etyl-1)--(etyl-2)--(metyl-2);
				}
			}
			%
			\path (0,0) pic[local bounding box=a,anchor=center,name prefix=a-] {n-butan};
			\path (0,2) pic[local bounding box=b,name prefix=b-,anchor=south] {n-butan};
			%Vẽ Liên kết
			\foreach [count =\j from 1]\x/\y in {metyl-1/metyl-1,etyl-1/etyl-1,etyl-2/etyl-2,metyl-2/metyl-2}{
				\path  (a-\x)--(b-\y) node [sloped,midway](bond-\j){\vanderwallsforce};
			}
			%
			\foreach[count =\t from 5] \x/\y in {bond-1/bond-2,bond-2/bond-3,bond-3/bond-4}{
				\path  (\x)--(\y) node [sloped,midway](bond-\t){\vanderwallsforce};
			}
			\path ($(bond-6) +(-90:2.5cm)$) node {butan};
		\end{tikzpicture}
		}
	\hspace{1cm}
	\resizebox{!}{3.7cm}{
			\begin{tikzpicture}[declare function={r=2.0;},line join=round,line cap=round,font=\large,inner sep=1pt]
		\def\vanderwallsforce{\tikz{\draw[decorate, decoration={snake, amplitude=1mm, segment length=1mm},\maunhan] (0,0)--(1,0);}}
		\tikzset{
			isobutan/.pic={
				\path node (metyl-1) {$CH$}
				(metyl-1)--++(90:r ) node (metyl-2) {$CH_3$}
				(metyl-1)--++(210:r) node (metyl-3) {$CH_3$}
				(metyl-1)--++(330:r) node (metyl-4) {$CH_3$}
				;
				\draw[\maunhan,thick](metyl-1) -- (metyl-2)
				(metyl-1)--(metyl-3) (metyl-1)--(metyl-4)
				;
			}
		}
		%
		\path (0,0) pic[local bounding box=a,anchor=center,name prefix=a-] {isobutan};
		\path (7,0) pic[local bounding box=b,name prefix=b-,anchor=center] {isobutan};
		%Vẽ Liên kết
		\path  (a-metyl-4)--(b-metyl-3) node [sloped,midway](bond){\vanderwallsforce};
		\path ($(bond)+(-90:1.5cm)$) node {isobutan};
	\end{tikzpicture}
	}
	\end{center}
	}
\end{bt}
%=============BT_6=============%%%
\begin{bt}
	Hãy giải thích vì sao ở điều kiện thường $Br_2$ ở trạng thái lỏng, còn $Cl_2$ ở trạng thái khí?
	\loigiai{Do số lượng electron trong phân tử $\mathrm{Br}_2$ nhiều hơn phân tử $\mathrm{Cl}_2$ nên tương tác van der Waals trong phân tử $\mathrm{Br}_2$ mạnh hơn trong phân tử $\mathrm{Cl}_2$, vì vậy $\mathrm{Br}_2$ tồn tại ở trạng thái lỏng ở nhiệt độ thường, còn $\mathrm{Cl}_2$ ở trạng thái khí}
\end{bt}
%%============Bài tập tư luyện Sách tham khảo============%%%
%%%=============BT_1=============%%%
\begin{bt}
	Hoàn thành các sơ đồ tạo thành ion sau
	\begin{enumerate}
		\item $K \xrightarrow K^+ \; +\; ?$
		\item $Ca \xrightarrow Ca^{2+} \;+ \;\; ?$
		\item $Br \; + \;?\xrightarrow Br^-$
		\item $S \;+\; ?\xrightarrow S^{2-}$
	\end{enumerate}
	\loigiai{
		\begin{enumerate}
			\item $K \xrightarrow K^+\; + \;e^-$
			\item $Ca \xrightarrow Ca^{2+}\; + \;2e^-$
			\item $Br\; + \;e^-\xrightarrow Br^-$
			\item $S\; + \;2e^-\xrightarrow S^{2-}$
		\end{enumerate}
	}
\end{bt}
%%%=============BT_2=============%%%
\begin{bt}
	Viết cấu hình electron của các ion: $Na^+$, $Mg^{2+}$, $Al^{3+}$, $F^-$, $O^{2-}$. Các ion trên có cấu hình electron giống khí hiếm nào?
	\loigiai{\begin{itemize}
		\item $Na^+$: $1s^22s^22p^6$
		\item $Mg^{2+}$: $1s^22s^22p^6$
		\item $Al^{3+}$: $1s^22s^22p^6$
		\item $F^-$: $1s^22s^22p^6$
		\item $O^{2-}$: $1s^22s^22p^6$
		\end{itemize}
		Các ion trên có cấu hình electron giống khí hiếm Ne:$1s^22s^22p^6$}
\end{bt}
%%%=============BT_3=============%%%
\begin{bt}
	Phân đạm cung cấp nitrogen cho cây dưới dạng nitrate ion ($NO_3^-$) và ammonium ion ($NH_4^+$). Có bao nhiêu phân tử hợp chất ion được tạo ra từ các ion: $NH_4^+$, $NO_3^-$, $Cl^-$, $SO_4^{2-}$?
	\loigiai{Có thể tạo ra 3 hợp chất ion: $NH_4NO_3$, $NH_4Cl$, $(NH_4)_2SO_4$.}
\end{bt}
%%%=============BT_4=============%%%
\begin{bt}
	Cho các ion sau: $K^+$, $Ca^{2+}$, $Al^{3+}$, $Cl^-$, $O^{2-}$. Hãy viết công thức phân tử các hợp chất được tạo nên từ các ion trên.
	\loigiai{$KCl$, $K_2O$, $CaCl_2$, $CaO$, $AlCl_3$, $Al_2O_3$.}
\end{bt}
%%%=============BT_5=============%%%
\begin{bt}
	Cho các phát biểu sau về tính chất của hợp chất ion:
	\begin{enumerate}[(a)]
		\item Trong hợp chất ion liên kết được tạo thành do lực hút tĩnh điện giữa các ion mang điện tích trái dấu.
		\item Hợp chất ion được hình thành giữa kim loại điển hình và phi kim điển hình,
		\item Có nhiệt độ nóng chảy và nhiệt độ sôi cao.
		\item Thường tồn tại ở trang thái khí ở điều kiện thường.
		\item Có nhiệt độ nóng chảy và nhiệt độ sôi thấp.
		\item Thường tồn tại ở trang thái rắn ở điều kiện thường.
	\end{enumerate}
	Có bao nhiêu tính chất là đúng trong hợp chất ion?
	\loigiai{Có 3 tính chất đúng là (a), (b), (c), và (f). 2 tính chất sai là (d) và (e).}
\end{bt}
\begin{bt}
	Giải thích sự hình thành liên kết ion trong các phân tử:$KCl$, $Na_2O$, $MgCl_2$.
	\loigiai{\begin{enumerate}[a)]
			\item $KCl$: $K$ mất 1 electron tạo thành $K^+$, $Cl$ nhận 1 electron tạo thành $Cl^-$. $K^+$ và $Cl^-$ hút nhau bằng lực hút tĩnh điện tạo thành $KCl$.\\
			$\begin{array}{lccccc}
				\textbf{Giai đoạn 1:} & K &\xrightarrow & K^+& + &1e\\
				\textbf{Cấu hình electron} & [Ar]4s^1 &  & [Ar]& &\\
				& Cl &+& 1e& \xrightarrow &Cl^-\\
				\textbf{Cấu hình electron} & [Ne]3s^23p^5 &  & [Ar]& &\\
				\textbf{Giai đoạn 2:} &K^+ &+ & Cl^-& \xrightarrow&KCl
			\end{array}$
			\item $Na_2O$: Mỗi $Na$ mất 1 electron tạo thành $Na^+$, $O$ nhận 2 electron tạo thành $O^{2-}$. Hai ion $Na^+$ và một ion $O^{2-}$ hút nhau bằng lực hút tĩnh điện tạo thành $Na_2O$.\\
			$\begin{array}{lccccc}
				\textbf{Giai đoạn 1:} & 2Na &\xrightarrow & 2Na^+& + &2e\\
				\textbf{Cấu hình electron} & [Ne]3s^1 &  & [Ne]& &\\
				& O &+& 2e& \xrightarrow &O^{2-}\\
				\textbf{Cấu hình electron} & [He]2s^22p^4 &  & [Ne]& &\\
				\textbf{Giai đoạn 2:} &2Na^+ &+ & O^{2-}& \xrightarrow&Na_2O
			\end{array}$
			\item $MgCl_2$: $Mg$ mất 2 electron tạo thành $Mg^{2+}$, mỗi $Cl$ nhận 1 electron tạo thành $Cl^-$. Một ion $Mg^{2+}$ và hai ion $Cl^-$ hút nhau bằng lực hút tĩnh điện tạo thành $MgCl_2$.\\
			$\begin{array}{lccccc}
				\textbf{Giai đoạn 1:} & Mg &\xrightarrow & Mg^{2+}& + &2e\\
				\textbf{Cấu hình electron} & [Ne]3s^2 &  & [Ne]& &\\
				& 2Cl &+& 2e& \xrightarrow &2Cl^-\\
				\textbf{Cấu hình electron} & [Ne]3s^23p^5 &  & [Ar]& &\\
				\textbf{Giai đoạn 2:} &Mg^{2+} &+ & 2Cl^-& \xrightarrow&MgCl_2
			\end{array}$
	\end{enumerate}}
\end{bt}
%%%=============BT_7=============%%%
\begin{bt}
	Nêu cấu trúc tinh thể sodium chloride ($NaCl$), vì sao muối ăn có nhiệt độ nóng chảy cao?
	\loigiai{
		Tinh thể $NaCl$ có cấu trúc lập phương, mỗi ion $Na^+$ được bao quanh bởi 6 ion $Cl^-$ và ngược lại. Muối ăn có nhiệt độ nóng chảy cao vì liên kết ion rất bền vững, cần nhiều năng lượng để phá vỡ.
		\begin{center}
			\includegraphics[height=5cm]{Images/Tikz/crytalNaCl.pdf}
		\end{center}
	}
\end{bt}
%%%=============BT_8=============%%%
\begin{bt}
	Hoàn thành những thông tin còn thiếu trong bảng sau.
	\\
	\begin{tabular}{|c|c|c|}
		\hline
		Công thức hợp chất ion & Cation & Anion \\
		\hline
		$\mathrm{CaCl}_2$ & $?$ & $?$ \\
		\hline
		$?$ & $\mathrm{Na}^{+}$ & $O^{2-}$ \\
		\hline
		KF & $?$ & $?$ \\
		\hline
	\end{tabular}
	\loigiai{\begin{tabular}{|c|c|c|}
			\hline
			Công thức hợp chất ion & Cation & Anion \\
			\hline
			$\mathrm{CaCl}_2$ & $\mathrm{Ca}^{2+}$ & $\mathrm{Cl}^{-}$ \\
			\hline
			$\mathrm{Na}_2\mathrm{O}$ & $\mathrm{Na}^{+}$ & $O^{2-}$ \\
			\hline
			KF & $\mathrm{K}^{+}$ & $\mathrm{F}^{-}$ \\
			\hline
	\end{tabular}}
\end{bt}
%%%=============BT_9=============%%%
\begin{bt}
	Quặng boxide là một loại quặng có nguồn gốc từ đá núi lửa có màu hồng, nâu được hình thành từ quá trình phong hóa các đá giàu nhôm hoặc tích tụtù các quặng có trước bởi quá trình xói mòn. Quặng boxide phân bồ chủ yếu trong vành đai xung quanh xích đạo đặc biệt trong môi trường nhiệt đới. Công thíc của quặng boxide là $\mathrm{Al}_2O_3\cdot 2H_2O$. Hãy trình bày sự hình thành liên kết ion trong phân từ aluminium oxide $\left(\mathrm{Al}_2O_3\right)$.
	\loigiai{Nguyên tử $Al$ cho 3 electron lớp ngoài cùng trở thành ion $Al^{3+}$, nguyên tử $O$ nhận 2 electron trở thành ion $O^{2-}$.\\
		Hai ion $Al^{3+}$ và ba ion $O^{2-}$ hút nhau bằng lực hút tĩnh điện tạo thành phân tử $Al_2O_3$.\\
		$\begin{array}{lccccc}
			\textbf{Giai đoạn 1:} & 2Al &\xrightarrow & 2Al^{3+}& + &6e\\
			\textbf{Cấu hình electron} & [Ne]3s^23p^1 &  & [Ne]& &\\
			& 3O &+& 6e& \xrightarrow &3O^{2-}\\
			\textbf{Cấu hình electron} & [He]2s^22p^4 &  & [Ne]& &\\
			\textbf{Giai đoạn 2:} &2Al^{3+} &+ & 3O^{2-}& \xrightarrow&Al_2O_3
		\end{array}$}
\end{bt}
%%%=============BT_10=============%%%
\begin{bt}
	Zinc oxide ($ZnO$). Hãy trình bày sự hình thành liên kết ion trong phân tử zinc oxide.
	\loigiai{$Zn$ mất 2 electron tạo thành $Zn^{2+}$, $O$ nhận 2 electron tạo thành $O^{2-}$. Hai ion $Zn^{2+}$ và $O^{2-}$ hút nhau bằng lực hút tĩnh điện tạo thành $ZnO$.\\
		$\begin{array}{lccccc}
			\textbf{Giai đoạn 1:} & Zn &\xrightarrow & Zn^{2+}& + &2e\\
			\textbf{Cấu hình electron} & [Ar]3d^{10}4s^2 &  & [Ar]3d^{10}& &\\
			& O &+& 2e& \xrightarrow &O^{2-}\\
			\textbf{Cấu hình electron} & [He]2s^22p^4 &  & [Ne]& &\\
			\textbf{Giai đoạn 2:} &Zn^{2+} &+ & O^{2-}& \xrightarrow&ZnO
		\end{array}$}
\end{bt}
%%%=============BT_11=============%%%
\begin{bt}
	Hãy biểu diễn sự hình thành các cặp electron chung cho phân tử $O_2$. Từ đó, viết công thức Lewis của phân tử này.
	\loigiai{
		Hai nguyên tử $O$ mỗi nguyên tử góp 2 electron tạo thành 2 cặp electron chung.
		\\
		\begin{tikzpicture}[declare function={r=1.2;R=10pt;},
			line cap=round, line join=round]
			\tikzstyle{element_style} = [font=\Large\bfseries, inner sep=3pt]
			% Macro định nghĩa electron
			% Node trung tâm
			\path(0*r,0) coordinate (A) node[element_style](Oa){O};
			\path(1*r,0) coordinate (B) node[element_style](plus){+};
			\path(2*r,0) coordinate (C) node[element_style](Ob){O};
			\path(3.2*r,0) coordinate (D) node (arrow){$\xrightarrow$};
			\path(4.3*r,0) coordinate (E) node[element_style](Oc){O};
			\path(5*r,0) coordinate (F) node[element_style](Od){O};
			% Vẽ các electron ở các góc ở Oa
			\foreach [count=\i from=1] \g/\c in {90/\mauphu, 180/\mauphu,0/\maunhan} {
			\path (Oa)--++(\g:R) coordinate (O-\i) 
				node[pos=1,sloped] {\electronH[\c]};
			}
			% Vẽ các electron ở các góc ở Ob
			\foreach [count=\i from=1] \g/\c in {90/\mauphu, 180/\maunhan,0/\mauphu} {
				\path (Ob)--++(\g:R) coordinate (O-\i) 
				node[pos=1,sloped] {\electronH[\c]};
			}
			% Vẽ các electron ở các góc ở Oc
			\foreach [count=\i from=1] \g/\c in {90/\mauphu, 180/\mauphu,0/\maunhan} {
				\path (Oc)--++(\g:R) coordinate (O-\i) 
				node[pos=1,sloped] {\electronH[\c]};
			}
			% Vẽ các electron ở các góc ở Od
			\foreach [count=\i from=1] \g/\c in {90/\mauphu, 180/\maunhan,0/\mauphu} {
				\path (Od)--++(\g:R) coordinate (O-\i) 
				node[pos=1,sloped] {\electronH[\c]};
			}
		\end{tikzpicture}
	\\
	Công thức lewis của phân tử Oxigen :
		\begin{tikzpicture}[declare function={r=1.0;R=10pt;},baseline={(0pt,-2.5pt)},inner sep=0pt,outer sep=auto]
			\tikzstyle{element_style} = [font=\Large\bfseries, inner sep=3pt]
			% Node trung tâm
			\path(0*r,0) coordinate (A) node[element_style](Oa){O};
			\path(1.25*r,0) coordinate (B) node[element_style](Ob){O};;
			% Vẽ các electron ở các góc ở Oa
			\foreach [count=\i from=1] \g/\c in {90/\mauphu, 180/\mauphu} {
				\path (Oa)--++(\g:R) coordinate (O-\i) 
				node[pos=1,sloped] {\electronH[\c]};
			}
			% Vẽ các electron ở các góc ở Ob
			\foreach [count=\i from=1] \g/\c in {90/\mauphu, 0/\mauphu} {
				\path (Ob)--++(\g:R) coordinate (O-\i) 
				node[pos=1,sloped] {\electronH[\c]};
			}
			%%Vẽ nối đôi
			\draw[double,thick,double distance=2pt](Oa)--(Ob);
		\end{tikzpicture}
	}
\end{bt}
%%%=============BT_12=============%%%
\begin{bt}
	Hãy biểu diễn sự hình thành các cặp electron chung cho phân tử $Cl_2$, $CH_4$.
	Từ đó, viết công thức Lewis của các phân tử này.
	\loigiai{%
	Nguyên tử chlorine (Cl): $[Ne]3s^23p^5$. Mỗi nguyên tử Cl cùng góp 1 electron để tạo một cặp electron chung đạt được cấu hình của khí hiếm.\\
	\begin{tikzpicture}[declare function={r=1;},line cap=round,line join=round,inner sep=0pt]
		\tikzstyle{atom_style} = [font=\Large\bfseries, inner sep=3pt]
		\path (r,0) coordinate (A) node {\congthuce[angle=90,color=\mauphu,type=2][angle=-90,color=\mauphu,type=2][angle=0,color=\maunhan,type=1][angle=180,color=\mauphu,type=2]{Center}{Cl}};
		\path (3*r,0) coordinate (B) node {\congthuce[angle=90,color=\mauphu,type=2][angle=-90,color=\mauphu,type=2][angle=0,color=\mauphu,type=2][angle=180,color=\maunhan,type=1]{Center}{Cl}};
		\path ($(A)!0.5!(B)$) coordinate (C) node[atom_style] {+};
		\path (6*r,0) coordinate (D) node {\congthuce[angle=90,color=\mauphu,type=2][angle=-90,color=\mauphu,type=2][angle=0,color=\maunhan,type=2,radius=12pt][angle=180,color=\mauphu,type=2]{Center}{Cl}};
		\path (6.8*r,0)coordinate (E) node {\congthuce[angle=90,color=\mauphu,type=2][angle=-90,color=\mauphu,type=2][angle=0,color=\mauphu,type=2]{Center}{Cl}};
		\path($(B)!0.5!(D)$) node {$\xrightarrow$};
	\end{tikzpicture}
	\\
	Công thức lewis của $Cl_2$.
	\begin{tikzpicture}[declare function={r=1;},baseline={([yshift=-3pt]current bounding box.base)},outer sep=0pt,inner sep=0pt]
		\path (0,0) node [name = Cla] {\congthuce[angle=90,color=cyan,type=2][angle=180,color=cyan,type=2][angle=-90,color=cyan,type=2]{Center}{Cl}};
		%%%
		\path (1.4,0) node [name = Clb] {\congthuce[angle=90,color=cyan,type=2][angle=-90,color=cyan,type=2][angle=0,color=cyan,type=2]{Center}{Cl}};
		%%%
		\draw[thick] (Cla.east)--(Clb.west);
	\end{tikzpicture}
	\\
	Cấu hình electron:(C): $1s^22s^22p^2$; (H): $1s^1$\\
	4 nguyên tử H và 1 nguyên tử C cùng góp 4 electron để tạo thành 4 cặp electron chung.\\
	\begin{tabular}{|c|c|}
		\hline
		Công thức electron&Công thức Lewis\\
		\hline
			\begin{tikzpicture}[declare function={r=1;},baseline={([yshift=-3pt]current bounding box.base)},outer sep=0pt,inner sep=0pt]
			\tikzstyle{atom_style} = [font=\large\bfseries\sffamily, inner sep=3pt]
			\path (0,0)coordinate (A) node (center) {\congthuce[angle=90,color=cyan,type=2][angle=-90,color=cyan,type=2][angle=0,color=cyan,type=2][angle=180,color=cyan,type=2]{Ca}{C}};
			%%%
			\foreach \g in{90,-90,0,180}{
				\path (center)--++(\g:0.7cm) node[pos=1,atom_style]{H};
			}
		\end{tikzpicture}
		&\begin{tikzpicture}[declare function={r=1;},baseline={([yshift=-3pt]current bounding box.base)},outer sep=0pt,inner sep=0pt]
			\tikzstyle{atom_style} = [font=\large\bfseries\sffamily, inner sep=3pt]
			\path (0,0)coordinate (A) node[atom_style] (center) {C};
			%%%
			\foreach[count=\i from=1] \g in{90,-90,0,180}{
				\path (center)--++(\g:0.9cm)  node [pos=1,inner sep=1pt,atom_style](H-\i){H};
				\draw (center)--(H-\i);
			}
		\end{tikzpicture}\\
		\hline
	\end{tabular}
	}
\end{bt}
%%%=============BT_13=============%%%
\begin{bt}
	Hãy biểu diễn sự hình thành các cặp electron chung cho phân tử $C_2H_4$, $C_2H_2$. Từ đó, viết công thức Lewis của các phân tử này.
	\loigiai{
		
	}
\end{bt}
%%%=============BT_14=============%%%
\begin{bt}
	Hãy biểu diễn sự hình thành các cặp electron chung cho phân tử $H_2S$. Từ đó, viết công thức Lewis của phân tử này.
	\loigiai{}
\end{bt}
%%%=============BT_15=============%%%
\begin{bt}
	Dựa theo độ âm điện, hãy cho biết loại liên kết trong các phân tử: $Na_2O$, $H_2O$, $CH_4$, $NaCl$. Cho bảng giá trị độ âm điện
	\begin{center}
		\begin{tabular}{|c|c|c|c|c|c|c|}
			\hline
			Nguyên tố & Na & S & H & C & Cl & O \\
			\hline
			Độ âm điện & $0{,}93$ & $2{,}58$ & $2{,}20$ & $2{,}55$ & $3{,}16$ & $3{,}44$ \\
			\hline
		\end{tabular}
	\end{center}
	\loigiai{}
\end{bt}
%%%=============BT_16=============%%%
\begin{bt}
	Sắp xếp các chất sau theo chiều tăng dần độ phân cực của liên kết: $HCl$, $HF$, $HBr$, $HI$. Biết độ âm điện các nguyên tố được cho ở bảng sau:
	\begin{center}
		\begin{tabular}{|c|c|c|c|c|c|}
			\hline
			Nguyên tố & H & F & Cl & Br & I \\
			\hline
			Độ âm điện & $2{,}20$ & $3{,}98$ & $3{,}16$ & $2{,}96$ & $2{,}66$ \\
			\hline
		\end{tabular}
	\end{center}
	\loigiai{}
\end{bt}
%%%=============BT_17=============%%%
\begin{bt}
	Vẽ sơ đồ biểu diễn liên kết hydrogen giữa hai phân tử $H_2O$.
	\loigiai{}
\end{bt}
%%%=============BT_18=============%%%
\begin{bt}
	Giải thích vì sao nước đá nhẹ và nổi trên mặt nước?
	\loigiai{}
\end{bt}
%%%=============BT_19=============%%%
\begin{bt}
	Cho bảng sau về nhiệt độ sôi và độ tan trong nước của $NH_3$ và $PH_3$
	\begin{center}
		\begin{tabular}{|c|c|c|}
			\hline
			Chất & $NH_3$ & $PH_3$ \\
			\hline
			Nhiệt độ sôi & $-33{,}34^\circ C$ & $-87{,}7^\circ C$ \\
			\hline
			Độ tan & $89{,}9$ g/ $100$ ml ở $0^\circ C$ & $31{,}2$ mg/ $100$ ml ($17^\circ C$) \\
			\hline
		\end{tabular}
	\end{center}
	Giải thích vì sao nhiệt độ sôi và độ tan của $NH_3$ lớn hơn $PH_3$
	\loigiai{}
\end{bt}
%%%=============BT_20=============%%%
\begin{bt}
	Cho bảng nhiệt độ sôi của ethanol và dimethyl ether.
	\begin{center}
		\begin{tabular}{|c|c|c|}
			\hline
			Chất & Khối lượng phân tử & Nhiệt độ sôi \\
			\hline
			ethanol & $46$ & $78{,}3^\circ C$ \\
			\hline
			dimethyl ether & $46$ & $-23^\circ C$ \\
			\hline
		\end{tabular}
	\end{center}
	Hãy giải thích vì sao hai chất có khối lượng phân tử bằng nhau nhưng nhiệt độ sôi lại khác xa nhau.
	\loigiai{}
\end{bt}
%%%=============BT_21=============%%%
\begin{bt}
	Giải thích vì sao nhện nước có thể di chuyển trên mặt nước?
	\loigiai{}
\end{bt}
%%%=============BT_22=============%%%
\begin{bt}
	Cho bảng sau về nhiệt độ nóng chảy, nhiệt độ sôi của các halogen
	\begin{center}
		\begin{tabular}{|c|c|c|c|c|}
			\hline
			Halogen & $F_2$ & $Cl_2$ & $Br_2$ & $I_2$ \\
			\hline
			Trạng thái ở $25^\circ C$ & Khí & Khí & Lỏng & Rắn \\
			\hline
			Nhiệt độ sôi ($^\circ C$) & $-188{,}1$ & $-34{,}1$ & $59{,}2$ & $185{,}5$ \\
			\hline
			Nhiệt độ nóng chảy ($^\circ C$) & $-219{,}6$ & $-101{,}0$ & $-7{,}3$ & $113{,}6$ \\
			\hline
		\end{tabular}
	\end{center}
	Hãy giải thích sự tăng dần nhiệt độ nóng chảy, nhiệt độ sôi của các halogen.
	\loigiai{}
\end{bt}
%%%=============BT_23=============%%%
\begin{bt}
	Giải thích vì sao có thể thu được ethanol ($C_2H_5OH$) bằng phương pháp chưng cất?
	\loigiai{}
\end{bt}
%%%=============BT_24=============%%%
\begin{bt}
	Cho đồ thị biểu diễn nhiệt độ sôi của halides Hãy giải thích xu hướng nhiệt dộ sôi của các halides.
	\loigiai{}
\end{bt}
%%%=============BT_25=============%%%
\begin{bt}
	Giải thích vì sao propanol $\left(CH_3CH_2CH_2OH\right)$ tan trong nước nhutres propan $CH_3CH_2CH_3$ thi không?
	\loigiai{}
\end{bt}
%%%=============BT_26=============%%%
\begin{bt}
	Giải thích vì sao nhiệt độ sôi của butan $\left(36^{\circ} C\right)$ cao hơn so với neopentive $\left(9,5^{\circ} C\right)$?
	\loigiai{}
\end{bt}
%%%=============BT_27=============%%%
\begin{bt}
	Giải thích vì sao $H_2O$ có nhiệt độ sôi $\left(100^{\circ} C\right)$ cao hơn phân tử $NH_3$. $33,34^{\circ} C$?
	\loigiai{}
\end{bt}
%%%=============BT_28=============%%%
\begin{bt}
	Giải thích vì sao thằn lằn có thể bỏ trên trần nhà?
	\loigiai{}
\end{bt}
%%==============Bài tập tự luận Sách BT Cánh diều===========%%%%
%%==============Bai_BT1==============%%%
\begin{bt}
	Hãy giải thích lí do khác nhau về nhiệt độ sôi của các cặp chất có cùng số
	electron sau đây: $CH_3CH_3$ ($184{,}5$ K) và $CH_3-F$ ($194{,}7$ K).
	\loigiai{Phân tử $CH_3-F$ có tương tác giữa các phân tử mạnh hơn do có liên kết C-F phân cực hơn hơn liên kết C-C trong phân tử $CH_3-CH_3$}
\end{bt}
%%==============HetBai_BT1==============%%%

%%==============Bai_BT2==============%%%
\begin{bt}
	Ở điều kiện thường, các khí hiếm tồn tại ở dạng khí đơn nguyên tử.
	Hãy giải thích sự biến đổi nhiệt độ sôi của các khí hiếm từ He tới Rn theo số
	liệu cho trong bảng sau:
	\begin{center}
		\begin{tabular}{|c|c|c|c|c|c|c|}
			\hline
			Khí hiếm & He & Ne & Ar & Kr & Xn & Rn \\
			\hline
			Số hiệu nguyên tử & $2$ & $10$ & $18$ & $36$ & $54$ & $86$ \\
			\hline
			Nhiệt độ sôi ($^\circ$ C) & $-269$ & $-246$ & $-186$ & $-152$ & $-108$ & $-62$ \\
			\hline
		\end{tabular}
	\end{center}
	\loigiai{Do khối lượng nguyên tử tăng dần và theo chiều tăng của Z, số electron và kích thước nguyên tử tăng dần gây nên sự phân cực tạm thời của nguyên tử mạnh hơn nên tương tác van der Waals mạnh dần lên.}
\end{bt}
%%==============HetBai_BT2==============%%%

%%==============Bai_BT3==============%%%
\begin{bt}
	Trong dung dịch, acetic acid có thể tồn tại dạng dimer (hai phân tử kết hợp) do sự hình thành liên kết hydrogen giữa hai phân tử. Hãy vẽ sơ đồ biểu diễn liên kết hydrogen giữa hai phân tử acetic acid hình thành dimer.
	\loigiai{}
\end{bt}
%%==============HetBai_BT3==============%%%

%%==============Bai_BT4==============%%%
\begin{bt}
	Hãy giải thích sự biến đổi về nhiệt độ nóng chảy của dãy hydrogen halide sau.
	\begin{center}
		\begin{tabular}{|c|c|c|c|c|}
			\hline
			Halogen halide & HF & HCl & HBr & HI \\
			\hline
			Nhiệt độ nóng chảy ($^\circ$ C) & $-83{,}1$ & $-114{,}8$ & $-88{,}5$ & $-50{,}8$ \\
			\hline
		\end{tabular}
	\end{center}
	\loigiai{Giữa các phân tử HF có liên kết hydrogen nên nhiệt độ nóng chảy cao hơn so với HCl. Từ HCl tới HI do kích thước nguyên tử halogen tăng, tương tác van der Waals giữa các phân tử tăng nên nhiệt độ nóng chảy tăng.}
\end{bt}
%%==============HetBai_BT4==============%%%

%%==============Bai_BT5==============%%%
\begin{bt}
	Nhiệt độ sôi của ba hợp chất được cho trong bảng sau:
	\begin{center}
		\begin{tabular}{|c|c|c|}
			\hline
			Hợp chất & Khối lượng phân tử (g mol$^{-1}$) & Nhiệt độ sôi ($^\circ$C) \\
			\hline
			2-hexanone & $100{,}16$ & $128{,}0$ \\
			\hline
			heptane & $100{,}20$ & $98{,}0$ \\
			\hline
			1-hexanol & $102{,}17$ & $156{,}0$ \\
			\hline
		\end{tabular}
	\end{center}
	Không cần tra cứu cấu trúc, em hãy trả lời các câu hỏi sau về ba hợp chất này:
	\begin{enumerate}
		\item Hợp chất nào có thể hình thành liên kết hydrogen?
		\item Hợp chất nào phân cực nhưng không hình thành liên kết hydrogen?
		\item Hợp chất nào ít phân cực, không tạo liên kết hydrogen?
	\end{enumerate}
	\loigiai{Ba chất có khối lượng phân tử tương đương nhau nên chất có nhiệt độ sôi cao nhất là chất có thể hình thành liên kết hydrogen, đó là 1-hexanol.
		
		Chất có phân tử phân cực sẽ có liên kết van der Waals giữa các phân tử mạnh hơn, có nhiệt độ sôi xếp thứ hai (ảnh hưởng của liên kết hydrogen tới nhiệt độ sôi là mạnh hơn tương tác van der Waals), do đó chất phân cực là 2-hexanone.
		Còn lại là heptane.}
\end{bt}
%%==============HetBai_BT5==============%%%


%%=======Bài tập tự luận sách chân trời sáng tao==============%%%%
%%==============Bai_BT1==============%%%
\begin{bt}
	Biểu diễn liên kết hydrogen giữa các phân tử sau:
	\begin{enumerate}
		\item methanol ($CH_3OH$) và nước.
		\item ethylene glycol ($HOCH_2CH_2OH$) và nước.
	\end{enumerate}
	Từ đó nhận xét tính tan của methanol và ethylene glycol trong nước.
	\loigiai{}
\end{bt}
%%==============HetBai_BT1==============%%%

%%==============Bai_BT2==============%%%
\begin{bt}
	Ethylene glycol ($HOCH_2CH_2OH$) là một chất chống đông trong công nghiệp ô tô, hàng không do có khả năng can thiệp vào liên kết hydrogen của nước, làm các phân tử nước khó liên kết hơn, khiến nước khó đóng băng hơn.
	Biểu diễn liên kết hydrogen liên phân tử và nội phân tử trong ethylene glycol.
	\loigiai{}
\end{bt}
%%==============HetBai_BT2==============%%%

%%==============Bai_BT3==============%%%
\begin{bt}
	Hãy so sánh tương tác van der Waals với liên kết ion.
	\loigiai{Tương tác van der Waals và liên kết ion đều là các lực hút tĩnh điện. Tuy nhiên, tương tác van der Waals là lực hút tĩnh điện giữa các phân tử trung hoà nên yếu hơn nhiều so với liên kêt ion là lực hút tĩnh điện giữa các ion trái dấu.}
\end{bt}
%%==============HetBai_BT3==============%%%

%%==============Bai_BT4==============%%%
\begin{bt}
	Thiết bị chụp cộng hưởng từ hạt nhân (NMR) sử dụng nitrogen lỏng để làm mát nam châm siêu dẫn. Nitrogen lỏng sôi ở $–195{,}8$ °C. Dự đoán nhiệt độ sôi của oxygen lỏng sẽ cao hay thấp hơn so với nitrogen lỏng? Giải thích.
	\loigiai{Oxygen có khối lượng phân tử cao hơn nitrogen, do đó tương tác van der Waals giữa các phân tử oxygen mạnh hơn so với nitrogen. Kết quả oxygen lỏng có nhiệt độ sôi cao hơn nitrogen lỏng. Thật vậy, oxygen lỏng sôi ở $-183^{\circ} \mathrm{C}$, trong khi nitrogen lỏng sôi ở $-195,8^{\circ} \mathrm{C}$.}
\end{bt}
%%==============HetBai_BT4==============%%%

%%==============Bai_BT5==============%%%
\begin{bt}
	Giải thích vì sao các tương tác van der Waals giữa các phân tử có kích thước lớn lại mạnh hơn so với các phân tử có kích thước nhỏ.
	\loigiai{Phân tử có kích thước lớn thường đi đôi với nhiều electron. Chính vì vậy khả năng tạo các lưỡng cực tức thời và lưỡng cực cảm ứng của các phân tử có kích thước lớn cũng nhiều hơn, từ đó tương tác van der Waals giữa các phân tử lớn cũng mạnh hơn, nên các phân tử có kích thước lớn "dính" với nhau hơn so với các phân tử có kích thước nhỏ.}
\end{bt}
%%==============HetBai_BT5==============%%%

%%==============Bai_BT6==============%%%
\begin{bt}
	Giải thích tại sao ở điều kiện thường, các nguyên tố trong nhóm halogen như fluorine và chlorine ở trạng thái khí, còn bromine ở trạng thái lỏng và iodine ở trạng thái rắn.
	\loigiai{Khi đi từ $\mathrm{F}_2$ đến $\mathrm{I}_2$, do khối lượng phân tử các halogen tăng dần làm tương tác van der Waals giữa các phân tử halogen cũng tăng dần, kết quả các phân tử halogen "dính" với nhau chặt hơn, nên fluorine và chlorine ở trạng thái khi, còn bromine ở trạng thái lỏng và iodine ở trạng thái rắn.}
\end{bt}
%%==============HetBai_BT6==============%%%

%%==============Bai_BT7==============%%%
\begin{bt}
	Nhiệt độ sôi của các hợp chất với hydrogen của các nguyên tố nhóm VA, VIA và VIIA được biểu diễn qua đồ thị sau:
	\begin{enumerate}
		\item Giải thích nhiệt độ sôi cao bất thường của các hợp chất với hydrogen của các nguyên tố đầu tiên trong mỗi nhóm.
		\item Nhận xét nhiệt độ sôi của các hợp chất với hydrogen của các nguyên tố còn lại ở mỗi nhóm và giải thích nguyên nhân sự biến đổi nhiệt độ sôi của chúng.
	\end{enumerate}
	\loigiai{%
	\begin{enumerate}[a)]
		\item  Các nguyên tố đầu tiên trong mỗi nhóm VA, VIA, VIIA $(\mathrm{N}, \mathrm{O}, \mathrm{F})$ có kích thước nhỏ và có độ âm điện lớn, kết quả trong các hợp chất $\mathrm{NH}_3 ; \mathrm{H}_2 \mathrm{O} ; \mathrm{HF}$ xuất hiện liên kết hydrogen liên phân tử làm các hợp chất này có nhiệt độ sôi cao bất thường so với các hợp chất còn lại trong mỗi nhóm.
		\item  Hợp chất với hydrogen của các nguyên tố còn lại trong mỗi nhóm có nhiệt độ sôi tăng dần khi khối lượng phân tử của chúng tăng. Vì khi khối lượng phân tử tăng, tương tác van der Waals giữa các phân tử trong hợp chất cũng tăng làm các phân tử "dính" với nhau chặt hơn, dẫn đến nhiệt độ sôi của chúng dần cao hơn.
	\end{enumerate}
	}
\end{bt}
%%==============HetBai_BT7==============%%%

%%==============Bai_BT8==============%%%
\begin{bt}
	So sánh nhiệt độ nóng chảy và nhiệt độ sôi của pentane ($CH_3CH_2CH_2CH_2CH_3$) và neopentane ($(CH_3)_4C$). Giải thích nguyên nhân sự khác biệt trên.
	\loigiai{Hai hợp chất đã cho có cùng công thức phân tử, tức cùng khối lượng phân tử. Tuy nhiên phân tử neopentane có dạng hình cầu nên diện tích bề mặt tiếp xúc giữa các phân tử neopentane nhỏ hơn so với các phân tử pentane. Kết quả các phân tử pentane \lq\lq dính\rq\rq\ với nhau hơn so với các phân tử neopentane nên nhiệt độ nóng chảy và nhiệt độ sôi của pentane $\left(-130^{\circ} \mathrm{C}\right.$ và $\left.36,0^{\circ} \mathrm{C}\right)$, cao hơn so với neopentane $\left(-16,6^{\circ} \mathrm{C}\right.$ và $\left.9,5^{\circ} \mathrm{C}\right)$.}
\end{bt}
%%==============HetBai_BT8==============%%%

%%==============Bai_BT9==============%%%
\begin{bt}
	Giải thích vì sao tetrachloromethane ($CCl_4$) tuy là phân tử không cực nhưng có nhiệt độ sôi cao hơn trichloromethane ($CHCl_3$) là phân tử có cực.
	\loigiai{$\mathrm{CHCl}_3$ là một phân tử phân cực, trong khi $\mathrm{CCl}_4$ là một phân tử không phân cực. Như vậy, $\mathrm{CHCl}_3$ đáng lý phải có nhiệt độ sôi cao hơn $\mathrm{CCl}_4$. Tuy nhiên thực tế $\mathrm{CCl}_4$ lại có nhiệt độ sôi cao là $76,8^{\circ} \mathrm{C}$, cao hơn so với $\mathrm{CHCl}_3$ là $61,2^{\circ} \mathrm{C}$. Điều này là do phân tử $\mathrm{CCl}_4$ có kích thước lớn hơn $\mathrm{CHCl}_3$ nên có số electron cũng nhiều hơn $\mathrm{CHCl}_3$, do đó tương tác van der Waals giữa các phân tử $\mathrm{CCl}_4$ mạnh hơn so với $\mathrm{CHCl}_3$ làm cho $\mathrm{CCl}_4$ có nhiệt độ sôi cao hơn $\mathrm{CHCl}_3$.}
\end{bt}
%%==============HetBai_BT9==============%%%


%%Bài tập tụ luận sách bài tập KNTTVCS=================%%%
%%==============Bai_BT1==============%%%
\begin{bt}
	Cho các chất sau: $C_2H_6$; $CH_3OH$; $CH_3COOH$.
	Chất nào có thể tạo được liên kết hydrogen? Vì sao?
	\loigiai{$\mathrm{CH}_3 \mathrm{OH}$ và $\mathrm{CH}_3 \mathrm{COOH}$ chứa nguyên tử O có độ âm điện lớn $(3,44)$ và nguyên tử H liên kết với nguyên tử O trong nhóm -OH là nguyên tử hydrogen linh động tạo ra liên kết hydrogen:
		\[\cdots \mathrm{O}-\mathrm{H} \cdots \mathrm{O}-\mathrm{H} \cdots\]}
\end{bt}
%%==============HetBai_BT1==============%%%

%%==============Bai_BT2==============%%%
\begin{bt}
	Khối lượng mol (g/mol) của nước, ammonia và methane lần lượt bằng 18,17 và 16. Nước sôi ở $100^\circ C$, còn ammonia sôi ở $-33{,}35^\circ C$ và methane sôi ở $-161{,}58^\circ C$. Giải thích vì sao các chất trên có khối lượng mol xấp xỉ nhau nhưng nhiệt độ sôi của chúng lại chênh lệch nhau.
	\loigiai{Nhiệt độ sôi của $H_2O$ lớn hơn rất nhiều so với $NH_3$ và $CH_4$ vì phân tử $H_2O$ và $NH_3$ có liên kết hydrogen liên phân tử (còn $CH_4$ không có); do độ âm điện O > N nên liên kết hydrogen trong $H_2O$ bền hơn trong $NH_3$.
	}
\end{bt}
%%==============HetBai_BT2==============%%%

%%==============Bai_BT3==============%%%
\begin{bt}
	Trong dung dịch ethanol ($C_2H_5OH$) có những kiểu liên kết hydrogen nào? Kiểu nào bền nhất và kém bền nhất? Mô tả bằng hình vẽ.
	\loigiai{Dung dịch ethanol có $C_2H_5OH$ và $H_2O$, cả hai phân tử này đều chứa nguyên tử O có độ âm điện lớn ($3{,}44$) và nguyên tử H liên kết với nguyên tử O trong nhóm $-OH$ là nguyên tử hydrogen linh động tạo ra liên kết hydrogen.
	Có bốn kiểu liên kết hydrogen trong dung dịch ethanol: alcohol - alcohol; nước – nước; alcohol – nước và nước – alcohol.
		
	Liên kết hydrogen càng bền khi nguyên tử có độ âm điện lớn hơn và nguyên tử H linh động hơn. Trong bốn kiểu trên: kiểu bền nhất là liên kết giữa H của nước với O của alcohol (nước – alcohol). Kiểu kém bền nhất là liên kết giữa H của alcohol với O của alcohol (alcohol - alcohol).}
\end{bt}
%%==============HetBai_BT3==============%%%

%%==============Bai_BT4==============%%%
\begin{bt}
	Trong phân tử nước và ammonia, phân tử nào có thể tạo nhiều liên kết hydrogen hơn? Vì sao?
	\loigiai{\begin{itemize}
			\item  Số liên kết hydrogen trung bình được tạo thành trên mỗi phân tử phụ thuộc vào:
			\begin{itemize}
				\item  Số nguyên tử hydrogen liên kết với F, O hoặc N trong phân tử.
				\item  Số lượng các cặp electron chưa liên kết có mặt trên F, O, N.
			\end{itemize}
			\item  Mỗi phân tử nước có hai nguyên tử hydrogen và hai cặp electron chưa liên kết nên phân tử nước có nhiều liên kết hydrogen với các phân tử nước khác. Nó có mức trung bình là hai liên kết hydrogen trên mỗi phân tử.
			\item  Ammonia có ít liên kết hydrogen hơn nước. Trung bình nó có thể hình thành chỉ một liên kết hydrogen trên mỗi phân tử. Mặc dù mỗi phân tử ammonia có ba nguyên tử hydrogen gắn với nguyên tử nitrogen, nhưng nó chỉ có một cặp electron duy nhất có thể tham gia vào quá trình hình thành liên kết hydrogen.
	\end{itemize}}
\end{bt}
%%==============HetBai_BT4==============%%%

%%==============Bai_BT5==============%%%
\begin{bt}
	Dầu mỏ chứa hỗn hợp nhiều hydrocarbon như: octane ($C_8H_{18}$) có trong xăng; butane ($C_4H_{10}$) có trong gas. Khi chưng cất dầu mỏ, octane hay butane sẽ bay hơi trước? Giải thích.
	\loigiai{Khi chưng cất dầu mỏ, butane sẽ bay hơi trước octane. Vì octane ($M = 114$) có phân tử khối lớn hơn butane ($M = 58$) nên có nhiệt độ sôi cao hơn.}
\end{bt}
%%==============HetBai_BT5==============%%%

%%==============Bai_BT6==============%%%
\begin{bt}
	Cho các chất và các trị số nhiệt độ sôi ($^\circ C$) sau: $H_2O$, $H_2S$, $H_2Se$, $H_2Te$ và $-42$; $-2$; $100$; $-61$.
	Ghép các trị số nhiệt độ sôi vào mỗi chất sao cho phù hợp và giải thích.
	\loigiai{\begin{itemize}
			\item  Giá trị nhiệt độ sôi của từng chất:
			$H_2O$ (100 °C); $H_2S$ (-61 °C); $H_2Se$ (-42 °C) và $H_2Te$ (-2 °C).
			\item  \indam[\maunhan]{Giải thích:} sự tăng nhiệt độ sôi từ $H_2S$ đến $H_2Te$ là do khối lượng phân tử tăng lên. Nếu $H_2O$ chỉ có lực van der Waals giữa các phân tử thì nhiệt độ sôi của nó dự đoán vào khoảng $-80^\circ C$. Tuy nhiên, nhiệt độ sôi của $H_2O$ là $100^\circ C$, cao hơn nhiều, đó là vì phân tử $H_2O$ còn có liên kết hydrogen liên phân tử, làm cho liên kết giữa các phân tử $H_2O$ bền vững hơn.
	\end{itemize}}
\end{bt}
%%==============HetBai_BT6==============%%%
\Closesolutionfile{ansbt}
\Closesolutionfile{ansbth}
%\bangdapanSA{AnsBT-C03B04_LKH_TTVANDERWALLS}

%%=======================PHẦN TRẮC NGHIỆM======================%%%%
\phan{Trắc nghiệm nhiều lựa chọn}
%=============SOẠN EX===============%%%
\Opensolutionfile{ansex}[Ans/LGEX-C03B04_LKH_TTVANDERWALLS]
\Opensolutionfile{ans}[Ans/Ans-C03B04_LKH_TTVANDERWALLS]

%Câu hỏi trắc nghiệm sách bài tập Cánh diều
%==============Cau_1==============%%%
\begin{ex}
	Phát biểu nào sau đây là \textbf{đúng}?
	\choice
	{Bất kì phân tử nào có chứa nguyên tử hydrogen cũng có thể tạo liên kết hydrogen với phân tử cùng loại}
	{Liên kết hydrogen là liên kết hình thành do sự góp chung cặp electron hoá trị giữa nguyên tử hydrogen và nguyên tử có độ âm điện lớn}
	{Liên kết hydrogen là loại liên kết yếu nhất giữa các phân tử}
	{\True Ảnh hưởng của liên kết hydrogen tới nhiệt độ sôi và nhiệt độ nóng chảy của chất là mạnh hơn ảnh hưởng của tương tác van der Waals}
	\loigiai{}
\end{ex}
%==============HetCau_1==============%%%

%==============Cau_2==============%%%
\begin{ex}
	Cho các phân tử: $H_2O$, $NH_3$, $HF$, $H_2S$, $CO_2$, $HCl$. Số phân tử có thể tạo liên kết hydrogen với phân tử cùng loại là
	\choice
	{\True $3$}
	{$4$}
	{$5$}
	{$6$}
	\loigiai{Chỉ có $\mathrm{H}_2 \mathrm{O}, \mathrm{NH}_3, \mathrm{HF}$ mới tạo được liên kết hydro với các phân tử cùng loại; còn $\mathrm{H}_2 \mathrm{S}, \mathrm{CO}_2, \mathrm{HCl}$ thì không.}
\end{ex}
%==============HetCau_2==============%%%

%==============Cau_3==============%%%
\begin{ex}
	Thứ tự nào sau đây thể hiện độ mạnh giảm dần của các loại liên kết?
	\choice
	{\True Liên kết ion > liên kết cộng hoá trị > liên kết hydrogen > tương tác van der Waals}
	{Liên kết ion > liên kết cộng hoá trị > tương tác van der Waals > liên kết hydrogen}
	{Liên kết cộng hoá trị > liên kết ion > liên kết hydrogen > tương tác van der Waals}
	{Tương tác van der Waals > liên kết hydrogen > liên kết cộng hoá trị > liên kết ion}
	\loigiai{}
\end{ex}
%==============HetCau_3==============%%%

%==============Cau_4==============%%%
\begin{ex}
	Giữa các nguyên tử He có thể có loại liên kết nào?
	\choice
	{Liên kết cộng hoá trị}
	{Liên kết hydrogen}
	{\True Tương tác van der Waals}
	{Không có bất kì liên kết nào}
	\loigiai{Giữa các phân tử không phân cực hoặc giữa các nguyên tử khí hiếm vẫn có thời điểm xuất hiện sự phân cực tạm thời (do nguyên tử chứa các hạt mang điện là proton và electron), do đó luôn có tương tác van der Waals.
	}
\end{ex}
%==============HetCau_4==============%%%

%==============Cau_5==============%%%
\begin{ex}
	Quy tắc octet không được sử dụng khi xem xét sự hình thành của hai loại liên kết hoặc tương tác nào sau đây?
	\begin{enumerate}[(1)]
		\item  Liên kết cộng hoá trị.
		\item  Liên kết ion.
		\item  Liên kết hydrogen.
		\item  Tương tác van der Waals.
	\end{enumerate}
	\choice
	{(1) và (2)}
	{(2) và (3)}
	{(1) và (3)}
	{\True (3) và (4)}
	\loigiai{}
\end{ex}
%==============HetCau_5==============%%%

%==============Cau_6==============%%%
\begin{ex}
	Nếu giữa phân tử chất tan và dung môi có thể tạo thành liên kết hydrogen
	hoặc có tương tác van der Waals càng mạnh với nhau thì càng tan tốt vào nhau.
	Lí do nào sau đây là phù hợp để giải thích dầu hỏa (thành phần chính là
	hydrocarbon) không tan trong nước?
	\choice
	{Cả nước và dầu đều là các phân tử có cực}
	{\True Nước là phân tử phân cực và dầu là không/ít phân cực}
	{Nước là phân tử không phân cực và dầu là phân cực}
	{Cả nước và dầu đều không phân cực}
	\loigiai{}
\end{ex}
%==============HetCau_6==============%%%

%==============Cau_7==============%%%
\begin{ex}
	Ethanol tan vô hạn trong nước do
	\choice
	{cả nước và ethanol đều là phân tử phân cực}
	{\True nước và ethanol có thể tạo liên kết hydrogen với nhau}
	{ethanol có thể tạo liên kết hydrogen với các phân tử ethanol khác}
	{ethanol và nước có tương tác van der Waals mạnh}
	\loigiai{}
\end{ex}
%==============HetCau_7==============%%%

%==============Cau_8==============%%%
\begin{ex}
	Chất nào trong số các chất sau tồn tại ở thể lỏng trong điều kiện thường?
	\choice
	{\True $CH_3OH$}
	{$CF_4$}
	{$SiH_4$}
	{$CO_2$}
	\loigiai{Giữa các phân tử $\mathrm{CH}_3 \mathrm{OH}$ có thể hinh thành liên kết hydrogen.}
\end{ex}
%==============HetCau_8==============%%%

%==============Cau_9==============%%%
\begin{ex}
	Dựa vào liên kết giữa các phân tử, hãy cho biết halogen nào sau đây có nhiệt
	độ sôi cao nhất.
	\choice
	{$F_2$}
	{$Cl_2$}
	{$Br_2$}
	{\True $I_2$}
	\loigiai{Do $I_2$ có khối lượng phân tử lớn nhất và đồng thời có kích thước lớn nhất nên tương tác van der Waals giữa các phân tử mạnh hơn. 
	}
\end{ex}
%==============HetCau_9==============%%%


%Câu hỏi trắc nghiệm sách bài tập chân trời sáng tạo 
%==============Cau_1==============%%%
\begin{ex}
	Hợp chất nào sau đây tạo được liên kết hydrogen liên phân tử?
	\choice
	{$H_2S$}
	{$PH_3$}
	{$HI$}
	{\True $CH_3OH$}
	\loigiai{}
\end{ex}
%==============HetCau_1==============%%%

%==============Cau_2==============%%%
\begin{ex}
	Mặc dù chlorine có độ âm điện là $3{,}16$ xấp xỉ với nitrogen là $3{,}04$ nhưng giữa
	các phân tử HCl không tạo được liên kết hydrogen với nhau, trong khi giữa
	các phân tử $NH_3$ tạo được liên kết hydrogen với nhau, nguyên nhân là do
	\choice
	{độ âm điện của chlorine nhỏ hơn của nitrogen}
	{phân tử $NH_3$ chứa nhiều nguyên tử hydrogen hơn phân tử HCl}
	{tổng số nguyên tử trong phân tử $NH_3$ nhiều hơn so với phân tử HCl}
	{\True kích thước nguyên tử chlorine lớn hơn nguyên tử nitrogen nên mật độ điện
		tích âm trên chlorine không đủ lớn để hình thành liên kết hydrogen}
	\loigiai{}
\end{ex}
%==============HetCau_2==============%%%

%==============Cau_3==============%%%
\begin{ex}
	Sơ đồ nào sau đây thể hiện đúng liên kết hydrogen giữa 2 phân tử hydrogen
	fluoride (HF)?
	\choice
	{\True $H^{\delta+}-F^{\delta-}\ldots H^{\delta+}-F^{\delta-}$}
	{$H^{\delta+}-F^{\delta+}\ldots H^{\delta-}-F^{\delta-}$}
	{$H^{\delta-}-F^{\delta+}\ldots H^{\delta-}-F^{\delta+}$}
	{$H^{\delta+}-F^{\delta-}\ldots H^{\delta+}-F^{\delta+}$}
	\loigiai{}
\end{ex}
%==============HetCau_3==============%%%

%==============Cau_4==============%%%
\begin{ex}
	Điều nào sau đây đúng khi nói về liên kết hydrogen liên phân tử?
	\choice
	{\True Là lực hút tĩnh điện giữa nguyên tử H (thường trong các liên kết H-F, H-N, H-O ở phân tử này) với một trong các nguyên tử có độ âm điện mạnh
		(thường là N, O, F) ở một phân tử khác}
	{Là lực hút giữa các phân tử khác nhau}
	{Là lực hút tĩnh điện giữa các ion trái dấu}
	{Là lực hút giữa các nguyên tử trong một hợp chất cộng hoá trị}
	\loigiai{Liên kết hydrogen liên phân tử là lực hút tĩnh điện giữa nguyên tử H (thường trong các liên kết $\mathrm{H}-\mathrm{F} ; \mathrm{H}-\mathrm{N} ; \mathrm{H}-\mathrm{O}$ ở phân tử này) với một trong các nguyên tử có độ âm điện mạnh (thường là $\mathrm{N} ; \mathrm{O} ; \mathrm{F}$ ) ở một phân tử khác.}
\end{ex}
%==============HetCau_4==============%%%

%==============Cau_5==============%%%
\begin{ex}
	Điều nào sau đây đúng khi nói về liên kết hydrogen nội phân tử?
	\choice
	{Là lực hút giữa các proton của nguyên tử này với các electron ở nguyên tử khác}
	{\True Là lực hút tĩnh điện giữa nguyên tử H (thường trong các liên kết H-F, H-N, H-O) ở một phân tử với một trong các nguyên tử có độ âm điện mạnh (thường là N, O, F) ở ngay chính phân tử đó}
	{Là lực hút giữa các ion trái dấu}
	{Là lực hút giữa các phân tử có chứa nguyên tử hydrogen}
	\loigiai{Liên kết hydrogen nội phân tử là lực hút tĩnh điện giửa nguyên tử $H$ ( thưởng trong các liên kết $\mathrm{H}-\mathrm{F} ; \mathrm{H}-\mathrm{N} ; \mathrm{H}-\mathrm{O}$ ) ở một phân tử với một trong các nguyên tử có độ âm điện manh (thường là $\mathrm{N} ; \mathrm{O} ; \mathrm{F}$ ) ở ngay chính phân tử đó.}
\end{ex}
%==============HetCau_5==============%%%

%==============Cau_6==============%%%
\begin{ex}
	Tương tác van der Waals xuất hiện là do sự hình thành các lưỡng cực tạm thời cũng như các lưỡng cực cảm ứng. Các lưỡng cực tạm thời xuất hiện là do sự chuyển động của
	\choice
	{các nguyên tử trong phân tử}
	{\True các electron trong phân tử}
	{các proton trong hạt nhân}
	{các neutron và proton trong hạt nhân}
	\loigiai{Tương tác van der Waals xuất hiện là do sự hình thành các lưỡng cực tạm thời cũng như các lưỡng cực cảm ứng. Các lưỡng cực tạm thởi xuất hiện là đo sự chuyển động của các electron trong phân tử, đỏ là lúc electron tập trung về một phía trong phân tử.}
\end{ex}
%==============HetCau_6==============%%%

%==============Cau_7==============%%%
\begin{ex}
	Trong các khí hiếm sau, khí hiếm có nhiệt độ sôi cao nhất là
	\choice
	{Ne}
	{\True Xe}
	{Ar}
	{Kr}
	\loigiai{Do có khối lượng phân tử lớn nhất nên tương tác van der Waals giữa các phân tử Xe là lớn nhất, dẫn đến khi hiếm Xe có nhiệt độ sồi cao nhất.}
\end{ex}
%==============HetCau_7==============%%%

%Cau hỏi trắc nghiệm sách bài tập kết nối tri thức
%==============Cau_1==============%%%
\begin{ex}
	Liên kết hydrogen là loại liên kết hoá học được hình thành giữa các
	nguyên tử nào sau đây?
	\choice
	{Phi kim và hydrogen trong hai phân tử khác nhau}
	{Phi kim và hydrogen trong cùng một phân tử}
	{Phi kim có độ âm điện lớn và nguyên tử hydrogen}
	{\True $F$, $O$, $N$, $\ldots$ có độ âm điện lớn, đồng thời có cặp electron hoá trị chưa liên kết
		và nguyên tử hydrogen linh động}
	\loigiai{}
\end{ex}
%==============HetCau_1==============%%%

%==============Cau_2==============%%%
\begin{ex}
	Tương tác van der Waals được hình thành do
	\choice
	{tương tác tĩnh điện lưỡng cực – lưỡng cực giữa các nguyên tử}
	{tương tác tĩnh điện lưỡng cực – lưỡng cực giữa các phân tử}
	{\True tương tác tĩnh điện lưỡng cực – lưỡng cực giữa các nguyên tử hay phân tử}
	{lực hút tĩnh điện giữa các phân tử phân cực}
	\loigiai{}
\end{ex}
%==============HetCau_2==============%%%

%==============Cau_3==============%%%
\begin{ex}
	Chất nào sau đây có thể tạo liên kết hydrogen?
	\choice
	{$PF_3$}
	{$CH_4$}
	{\True $CH_3OH$}
	{$H_2S$}
	\loigiai{}
\end{ex}
%==============HetCau_3==============%%%

%==============Cau_4==============%%%
\begin{ex}
	Chất nào sau đây *không* thể tạo được liên kết hydrogen?
	\choice
	{$H_2O$}
	{\True $CH_4$}
	{$CH_3OH$}
	{$NH_3$}
	\loigiai{}
\end{ex}
%==============HetCau_4==============%%%

%==============Cau_5==============%%%
\begin{ex}
	Tương tác van der Waals tồn tại giữa những
	\choice
	{ion}
	{hạt proton}
	{hạt neutron}
	{\True phân tử}
	\loigiai{}
\end{ex}
%==============HetCau_5==============%%%

%==============Cau_6==============%%%
\begin{ex}
	Cho các chất sau: $F_2$, $Cl_2$, $Br_2$, $I_2$.
	Chất có nhiệt độ nóng chảy thấp nhất là
	\choice
	{\True $F_2$}
	{$Cl_2$}
	{$Br_2$}
	{$I_2$}
	\loigiai{}
\end{ex}
%==============HetCau_6==============%%%

%==============Cau_7==============%%%
\begin{ex}
	Cho các chất sau: $F_2$, $Cl_2$, $Br_2$, $I_2$.
	Chất có nhiệt độ sôi cao nhất là
	\choice
	{$F_2$}
	{$Cl_2$}
	{$Br_2$}
	{\True $I_2$}
	\loigiai{}
\end{ex}
%==============HetCau_7==============%%%

%==============Cau_8==============%%%
\begin{ex}
	Dãy chất nào sau đây xếp theo thứ tự nhiệt độ sôi tăng dần?
	\choice
	{$H_2O$, $H_2S$, $CH_4$}
	{$H_2S$, $CH_4$, $H_2O$}
	{$CH_4$, $H_2O$, $H_2S$}
	{\True $CH_4$, $H_2S$, $H_2O$}
	\loigiai{}
\end{ex}
%==============HetCau_8==============%%%

%==============Cau_9==============%%%
\begin{ex}
	Cho các khí hiếm sau: He, Ne, Ar, Kr, Xe.
	Khí hiếm có nhiệt độ nóng chảy thấp nhất và cao nhất lần lượt là
	\choice
	{Xe và He}
	{Ar và Ne}
	{\True He và Xe}
	{He và Kr}
	\loigiai{}
\end{ex}
%==============HetCau_9==============%%%

%==============Cau_10==============%%%
\begin{ex}
	Cho các chất sau: $C_2H_6$; $H_2O$; $NH_3$; $PF_3$; $C_2H_5OH$.
	Số chất tạo được liên kết hydrogen là
	\choice
	{$2$}
	{\True $3$}
	{$4$}
	{$5$}
	\loigiai{}
\end{ex}
%==============HetCau_10==============%%%

%==============Cau_11==============%%%
\begin{ex}
	Giữa $H_2O$ và $HF$ có thể tạo ra ít nhất bao nhiêu kiểu liên kết hydrogen?
	\choice
	{$2$}
	{$3$}
	{\True $4$}
	{$5$}
	\loigiai{}
\end{ex}
%==============HetCau_11==============%%%

%==============Cau_12==============%%%
\begin{ex}
	Nhiệt độ sôi của từng chất methane, ethane, propane và butane là một trong bốn nhiệt độ sau: $0^\circ C$; $-164^\circ C$; $-42^\circ C$ và $-88^\circ C$.
	Nhiệt độ sôi $-88^\circ C$ là của chất nào sau đây?
	\choice
	{methane}
	{propane}
	{\True ethane}
	{butane}
	\loigiai{}
\end{ex}
%==============HetCau_12==============%%%
\Closesolutionfile{ans}
\Closesolutionfile{ansex}
%\bangdapan{Ans-C03B04_LKH_TTVANDERWALLS}
\phan{Trắc nghiệm đúng /sai}
%%%=============SOẠN EXTF===============%%%
\Opensolutionfile{ansex}[Ans/LGTF-C03B04_LKH_TTVDW]
\Opensolutionfile{ansbook}[Ansbook/AnsTF-C03B04_LKH_TTVDW]
\Opensolutionfile{ans}[Ans/Tempt-C03B04_LKH_TTVDW]
%%%%=================EX_01====================%%%%
\begin{ex}
	Trong các chất sau, chất có khả năng tạo liên kết hydrogen giữa các phân tử là:
	\choiceTF[t]
	{\True $ H_2O, NH_3, HF $}
	{$ CH_4, CCl_4, SiH_4 $}
	{\True $ HF, H_2O, H_2S $}
	{$ HCl, HBr, HI $}
	\loigiai{
		\begin{itemchoice}[T1,F2,T3,F4]
			\itemch $ H_2O, NH_3, HF $ đều có nguyên tử H liên kết với các nguyên tử có độ âm điện lớn (O, N, F), cho phép tạo liên kết hydrogen mạnh.
			\itemch Các chất như $ CH_4, CCl_4, SiH_4 $ không có liên kết H với nguyên tố có độ âm điện lớn, nên không hình thành liên kết hydrogen.
			\itemch $ HF, H_2O $ có khả năng tạo liên kết hydrogen mạnh; $ H_2S $ tạo liên kết yếu hơn do độ âm điện của S thấp.
			\itemch $HCl, HBr, HI $ không tạo liên kết hydrogen đáng kể vì độ âm điện của Cl, Br, I thấp hơn nhiều so với O, N, F.
		\end{itemchoice}
	}
\end{ex}

%%%%=================EX_02====================%%%%
\begin{ex}
	Liên kết hydrogen đóng vai trò quan trọng trong tính chất của các hợp chất.
	\choiceTF[t]
	{\True Nhiệt độ sôi của nước cao do liên kết hydrogen}
	{\True ADN duy trì cấu trúc nhờ liên kết hydrogen giữa các base}
	{$H_2S$ có nhiệt độ sôi cao hơn $H_2O$ nhờ liên kết hydrogen}
	{\True Rượu ethanol có khả năng tạo liên kết hydrogen với nước}
	\loigiai{
		\begin{itemchoice}[T1,T2,F3,T4]
			\itemch Liên kết hydrogen bền vững làm tăng nhiệt độ sôi của nước.
			\itemch Trong ADN, liên kết hydrogen tạo sự gắn kết giữa các base nitơ.
			\itemch $H_2S$ không tạo liên kết hydrogen mạnh như $H_2O$, nhiệt độ sôi thấp hơn.
			\itemch Rượu ethanol tạo liên kết hydrogen giữa nhóm OH và nước.
		\end{itemchoice}
	}
\end{ex}

%%%%=================EX_03====================%%%%
\begin{ex}
	Tương tác van der Waals có vai trò quan trọng trong tính chất của chất rắn và lỏng.
	\choiceTF[t]
	{\True Là loại lực yếu, ảnh hưởng chủ yếu đến các phân tử không cực}
	{\True Là nguyên nhân chính làm chất khí thực khác chất khí lý tưởng}
	{Các phân tử không cực không chịu tác dụng của lực này}
	{\True Tăng cường độ với sự tăng khối lượng phân tử}
	\loigiai{
		\begin{itemchoice}[T1,T2,F3,T4]
			\itemch Tương tác van der Waals phổ biến trong các phân tử không cực.
			\itemch Chất khí thực chịu tác động của lực van der Waals làm sai lệch so với chất khí lý tưởng.
			\itemch Các phân tử không cực vẫn chịu lực van der Waals, không như phát biểu sai.
			\itemch Lực van der Waals tăng theo khối lượng phân tử và diện tích bề mặt tiếp xúc.
		\end{itemchoice}
	}
\end{ex}

%%%%=================EX_04====================%%%%
\begin{ex}
	Trong liên kết hydrogen, các yếu tố ảnh hưởng đến độ bền liên kết bao gồm:
	\choiceTF[t]
	{\True Độ âm điện của nguyên tử tham gia liên kết}
	{\True Độ lớn của liên kết đôi trong phân tử}
	{Số lượng proton trong hạt nhân của nguyên tử}
	{\True Góc tạo bởi các nguyên tử trong liên kết hydrogen}
	\loigiai{
		\begin{itemchoice}[T1,T2,F3,T4]
			\itemch Độ âm điện càng lớn, liên kết hydrogen càng bền.
			\itemch Sự cộng hưởng trong liên kết đôi ảnh hưởng đến lực hút.
			\itemch Số proton không ảnh hưởng trực tiếp đến độ bền liên kết hydrogen.
			\itemch Góc phù hợp tăng cường tương tác hydrogen.
		\end{itemchoice}
	}
\end{ex}

%%%%=================EX_05====================%%%%
\begin{ex}
	Chất có khả năng tạo nhiều liên kết hydrogen trong nước:
	\choiceTF[t]
	{\True Glucose nhờ nhiều nhóm OH}
	{\True DNA nhờ các base nitơ có liên kết hydrogen}
	{Metan vì tính đối xứng cao}
	{\True Amino acid nhờ nhóm $NH_2$ và COOH}
	\loigiai{
		\begin{itemchoice}[T1,T2,F3,T4]
			\itemch Glucose có nhiều nhóm OH, tạo nhiều liên kết hydrogen với nước.
			\itemch DNA sử dụng liên kết hydrogen để tạo liên kết giữa các base.
			\itemch Metan không tạo liên kết hydrogen do không có nhóm H liên kết trực tiếp với nguyên tử âm điện cao.
			\itemch Amino acid có nhóm chức tạo được liên kết hydrogen với nước.
		\end{itemchoice}
	}
\end{ex}

%%%%=================EX_06====================%%%%
\begin{ex}
	Trong các hợp chất, tương tác van der Waals và liên kết hydrogen có vai trò khác nhau:
	\choiceTF[t]
	{\True Tương tác van der Waals phổ biến trong phân tử không cực}
	{Liên kết hydrogen tồn tại trong mọi phân tử chứa hydrogen}
	{\True Liên kết hydrogen mạnh hơn van der Waals}
	{\True Van der Waals đóng vai trò chính trong sự hóa lỏng của khí hiếm}
	\loigiai{
		\begin{itemchoice}[T1,F2,T3,T4]
			\itemch Tương tác van der Waals đặc trưng ở các phân tử không cực.
			\itemch Không phải mọi phân tử chứa H đều tạo liên kết hydrogen.
			\itemch Liên kết hydrogen mạnh hơn nhiều so với tương tác van der Waals.
			\itemch Khí hiếm hóa lỏng chủ yếu nhờ lực van der Waals.
		\end{itemchoice}
	}
\end{ex}

%%%%=================EX_07====================%%%%
\begin{ex}
	Nhiệt độ sôi của các chất phụ thuộc vào lực tương tác giữa các phân tử:
	\choiceTF[t]
	{\True Nước có nhiệt độ sôi cao nhờ liên kết hydrogen}
	{Khí hiếm như Neon có nhiệt độ sôi cao nhờ liên kết hydrogen}
	{\True HF có nhiệt độ sôi cao do liên kết hydrogen mạnh}
	{\True Tương tác van der Waals yếu dẫn đến nhiệt độ sôi thấp của $CH_4$}
	\loigiai{
		\begin{itemchoice}[T1,F2,T3,T4]
			\itemch Nước sôi ở nhiệt độ cao do các liên kết hydrogen giữa các phân tử.
			\itemch Neon không tạo liên kết hydrogen, sôi ở nhiệt độ rất thấp.
			\itemch HF có liên kết hydrogen mạnh làm tăng nhiệt độ sôi.
			\itemch $CH_4$ chỉ có lực van der Waals yếu, dẫn đến nhiệt độ sôi thấp.
		\end{itemchoice}
	}
\end{ex}

%%%%=================EX_08====================%%%%
\begin{ex}
	Sự khác biệt giữa liên kết hydrogen và tương tác van der Waals:
	\choiceTF[t]
	{\True Liên kết hydrogen yêu cầu có nguyên tử có độ âm điện cao}
	{\True Tương tác van der Waals xuất hiện ở mọi phân tử}
	{Liên kết hydrogen yếu hơn tương tác van der Waals}
	{\True Van der Waals không yêu cầu nguyên tử có độ âm điện cao}
	\loigiai{
		\begin{itemchoice}[T1,T2,F3,T4]
			\itemch Độ âm điện cao là yếu tố cần thiết cho liên kết hydrogen.
			\itemch Tương tác van der Waals phổ biến ở cả phân tử cực và không cực.
			\itemch Liên kết hydrogen mạnh hơn tương tác van der Waals.
			\itemch Tương tác van der Waals không phụ thuộc vào độ âm điện.
		\end{itemchoice}
	}
\end{ex}

%%%%=================EX_09====================%%%%
\begin{ex}
	Trong các yếu tố ảnh hưởng đến tính chất của nước:
	\choiceTF[t]
	{\True Liên kết hydrogen làm nước có nhiệt dung riêng cao}
	{\True Liên kết hydrogen giúp nước tồn tại ở thể lỏng ở nhiệt độ phòng}
	{Van der Waals là lực chủ yếu làm nước có độ nhớt cao}
	{\True Liên kết hydrogen làm nước có sức căng bề mặt lớn}
	\loigiai{
		\begin{itemchoice}[T1,T2,F3,T4]
			\itemch Nhiệt dung riêng cao của nước là nhờ liên kết hydrogen.
			\itemch Nước tồn tại ở thể lỏng do liên kết hydrogen ổn định ở nhiệt độ phòng.
			\itemch Độ nhớt của nước chủ yếu do liên kết hydrogen, không phải van der Waals.
			\itemch Liên kết hydrogen làm nước có sức căng bề mặt lớn.
		\end{itemchoice}
	}
\end{ex}

%%%%=================EX_10====================%%%%
\begin{ex}
	Ứng dụng thực tế của liên kết hydrogen và van der Waals:
	\choiceTF[t]
	{\True DNA duy trì cấu trúc nhờ liên kết hydrogen}
	{\True Sự ngưng tụ của hơi nước là nhờ liên kết hydrogen}
	{Chất không cực tan trong nước nhờ liên kết hydrogen}
	{\True Các khí hiếm hóa lỏng nhờ lực van der Waals}
	\loigiai{
		\begin{itemchoice}[T1,T2,F3,T4]
			\itemch Liên kết hydrogen trong DNA giúp duy trì cấu trúc xoắn kép.
			\itemch Ngưng tụ hơi nước xảy ra do liên kết hydrogen.
			\itemch Chất không cực không tan trong nước vì không tạo liên kết hydrogen.
			\itemch Lực van der Waals là yếu tố chính trong quá trình hóa lỏng khí hiếm.
		\end{itemchoice}
	}
\end{ex}

%%%%=================EX_11====================%%%%
\begin{ex}
	Các yếu tố quyết định khả năng tạo liên kết hydrogen của một hợp chất:
	\choiceTF[t]
	{\True Sự có mặt của nguyên tử H liên kết với nguyên tố có độ âm điện lớn như O, N, F}
	{\True Sự có mặt của cặp electron chưa liên kết trên nguyên tử âm điện cao}
	{Số lượng nguyên tử hydrogen trong phân tử càng nhiều thì liên kết hydrogen càng bền}
	{\True Cấu trúc hình học của phân tử cho phép tương tác không gian thuận lợi}
	\loigiai{
		\begin{itemchoice}[T1,T2,F3,T4]
			\itemch Độ âm điện lớn của O, N, F làm tăng khả năng tạo liên kết hydrogen.
			\itemch Cặp electron chưa liên kết là yếu tố cần để hình thành liên kết hydrogen.
			\itemch Số lượng hydrogen không quyết định độ bền, mà phụ thuộc vào vị trí và tính chất.
			\itemch Cấu trúc hình học ảnh hưởng đến khả năng hình thành liên kết hydrogen.
		\end{itemchoice}
	}
\end{ex}

%%%%=================EX_12====================%%%%
\begin{ex}
	Sự ảnh hưởng của tương tác van der Waals đến tính chất của chất rắn:
	\choiceTF[t]
	{\True Làm tăng độ bền cơ học của các phân tử không cực trong chất rắn}
	{\True Quyết định nhiệt độ nóng chảy của các phân tử không cực}
	{Chỉ xuất hiện trong các phân tử cực}
	{\True Làm tăng độ bám dính giữa các bề mặt phân tử}
	\loigiai{
		\begin{itemchoice}[T1,T2,F3,T4]
			\itemch Van der Waals tăng cường độ bền cơ học trong chất rắn không cực.
			\itemch Nhiệt độ nóng chảy của chất không cực phụ thuộc vào lực van der Waals.
			\itemch Van der Waals không chỉ xuất hiện ở các phân tử cực mà còn ở không cực.
			\itemch Tương tác này tăng độ bám dính giữa các bề mặt.
		\end{itemchoice}
	}
\end{ex}

%%%%=================EX_13====================%%%%
\begin{ex}
	Tương tác van der Waals và liên kết hydrogen trong các chất sinh học:
	\choiceTF[t]
	{\True Giữ cho protein có cấu trúc bậc ba ổn định}
	{\True Gắn kết các base nitơ trong ADN nhờ liên kết hydrogen}
	{Tạo liên kết bền vững trong cấu trúc tinh thể của kim cương}
	{\True Tham gia vào sự gắn kết của lipid trong màng tế bào}
	\loigiai{
		\begin{itemchoice}[T1,T2,F3,T4]
			\itemch Protein duy trì cấu trúc bậc ba nhờ van der Waals và liên kết hydrogen.
			\itemch Base nitơ trong ADN gắn kết bằng liên kết hydrogen.
			\itemch Kim cương không liên quan đến van der Waals mà nhờ liên kết cộng hóa trị.
			\itemch Tương tác van der Waals góp phần vào sự gắn kết của lipid.
		\end{itemchoice}
	}
\end{ex}

%%%%=================EX_14====================%%%%
\begin{ex}
	Liên kết hydrogen và van der Waals ảnh hưởng đến trạng thái vật chất:
	\choiceTF[t]
	{\True Liên kết hydrogen giúp nước có tính chất đặc biệt ở thể lỏng}
	{\True Tương tác van der Waals giúp các chất khí hiếm tồn tại ở trạng thái lỏng}
	{Liên kết hydrogen yếu hơn tương tác van der Waals nên ít ảnh hưởng đến trạng thái của chất lỏng}
	{\True Van der Waals đóng vai trò chính trong sự đông đặc của khí hiếm}
	\loigiai{
		\begin{itemchoice}[T1,T2,F3,T4]
			\itemch Liên kết hydrogen tạo các tính chất đặc biệt như nhiệt độ sôi cao của nước.
			\itemch Tương tác van der Waals cho phép khí hiếm hóa lỏng ở nhiệt độ thấp.
			\itemch Liên kết hydrogen mạnh hơn van der Waals, không như phát biểu sai.
			\itemch Sự đông đặc của khí hiếm liên quan đến lực van der Waals.
		\end{itemchoice}
	}
\end{ex}

%%%%=================EX_15====================%%%%
\begin{ex}
	Đặc điểm khác biệt giữa liên kết hydrogen và tương tác van der Waals:
	\choiceTF[t]
	{\True Liên kết hydrogen xảy ra khi có nguyên tử H liên kết với nguyên tố có độ âm điện lớn}
	{\True Van der Waals phổ biến ở các phân tử không cực hoặc ít cực}
	{Liên kết hydrogen không phụ thuộc vào sự có mặt của cặp electron tự do}
	{\True Van der Waals yếu hơn nhiều so với liên kết hydrogen}
	\loigiai{
		\begin{itemchoice}[T1,T2,F3,T4]
			\itemch Nguyên tử H liên kết với O, N, hoặc F là điều kiện cho liên kết hydrogen.
			\itemch Van der Waals phổ biến ở các phân tử không cực.
			\itemch Liên kết hydrogen phụ thuộc vào cặp electron tự do, không như phát biểu sai.
			\itemch Lực van der Waals yếu hơn liên kết hydrogen.
		\end{itemchoice}
	}
\end{ex}
\Closesolutionfile{ans}
\Closesolutionfile{ansbook}
\Closesolutionfile{ansex}
%\bangdapanTF{AnsTF-C03B04_LKH_TTVDW}

\end{document}