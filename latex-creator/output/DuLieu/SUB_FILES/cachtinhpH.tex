\documentclass[Main.tex]{subfiles}
\renewcommand*\printatom[1]{\ensuremath{\mathsf{#1}}}
\setchemfig{bond style={\mycolor,line width=0.8pt},atom sep=2.2em,atom style={\mycolor},bond join=true}
\setlist[enumerate,1]{itemsep=-1pt,topsep=3pt,label*=\circlenum{\arabic*}}%

\newcommand{\ngoacvuongtron}[2][]{
	\begin{tikzpicture}[declare function={d=-4pt;},node distance=-d]
		\node (name) {#2};
		\node[anchor = west, above right =of name,shift={(2pt,-3pt)}](plus) {\large{#1}};
		\draw[rounded corners=-d-1pt,\mycolor,ultra thick] (name.north west)--([xshift=d]name.north west)|-($(name.south west) +(0,0)$);
		\draw[rounded corners=-d-1pt,\mycolor,ultra thick] (name.north east)--([xshift=-d]name.north east)|-($(name.south east) +(0,0)$);
	\end{tikzpicture}
}

\begin{document}
\NewDocumentCommand{\SanserifFont}{O{qag}O{12}O{\maunhan}O{\faCoffee}}{
\color{#3}\bfseries\fontsize{#2pt}{6pt}\fontfamily{#1}\selectfont#4
}
%Nhớ tắt 3 lệnh này khi chạy filemain
%\setcounter{tocdepth}{1}
\setcounter{secnumdepth}{4}
%\tableofcontents
\titlespacing*{\subsubsection}{0cm}{0pt}{0pt}
\begin{center}
	{\SanserifFont[qag][16]cách tính pH của dung dịch }
\end{center}
\subsubsection{Lý thuyết cần nhớ}
\Noibat[][][\faAndroid]{Khái niệm pH}
\begin{hopvidu}
	\taodongke[0.9][\dotfill][gray]{15}
\end{hopvidu}
%%%%%===================%%%%%%
\Noibat[\maudam][][\faAndroid]{pH của dung dịch acid/bazo mạnh}
\begin{hopvidu}[\maudam]
	\taodongke[0.9][\dotfill][gray]{10}
\end{hopvidu}
%%%%==========================%%%%
\begin{hopvidu}[\mauphu]
	\taodongke[0.9][\dotfill][gray]{10}
\end{hopvidu}
\Noibat[][][\faBank]{Một số ví dụ minh họa}
\dongkevd
%%%=============VD_1=============%%%
\begin{vd}
	Tính pH của các dung dịch sau:
	\begin{enumerate}
		\item Dung dịch $\mathrm{KOH}$ $0,01\;M$;
		\item Dung dịch $\mathrm{HNO_3}$ $0,1\;M$;
		\item Dung dịch $\mathrm{Ba{(OH)}_2}$ $0,005\;M$;
		\item Dung dịch $\mathrm{H_2SO_4}$ $0,05\;M$.
	\end{enumerate}
	\loigiai{
		\begin{enumerate}
			\item Dung dịch $\mathrm{KOH}$ $0,01\;M$:
			\begin{align*}[\mathrm{OH^-}]
				&= 0,01 \mathrm{M} \\
				\mathrm{pOH} &= -\log[\mathrm{OH^-}] = -\log(0,01) = 2 \\
				\mathrm{pH} &= 14 - \mathrm{pOH} = 14 - 2 = 12
			\end{align*}
			\item Dung dịch $\mathrm{HNO_3}$ $0,1\;M$:
			\begin{align*}[\mathrm{H^+}]
				&= 0,1 \mathrm{M} \\
				\mathrm{pH} &= -\log[\mathrm{H^+}] = -\log(0,1) = 1
			\end{align*}
			\item Dung dịch $\mathrm{Ba{(OH)}_2}$ $0,005\;M$:
			\begin{align*}[\mathrm{OH^-}]
				&= 2 \times 0,005 = 0,01 \mathrm{M} \\
				\mathrm{pOH} &= -\log[\mathrm{OH^-}] = -\log(0,01) = 2 \\
				\mathrm{pH} &= 14 - \mathrm{pOH} = 14 - 2 = 12
			\end{align*}
			\item Dung dịch $\mathrm{H_2SO_4}$ $0,05\;M$:
			\begin{align*}[\mathrm{H^+}]
				&= 2 \times 0,05 = 0,1 \mathrm{M} \\
				\mathrm{pH} &= -\log[\mathrm{H^+}] = -\log(0,1) = 1
			\end{align*}
		\end{enumerate}
	}
\end{vd}
%%%=============VD_2=============%%%
\begin{vd}
	Tính pH của các dung dịch sau:
	\begin{enumEX}{2}
		\item Dung dịch $\mathrm{NaOH}$ $0,001\;M$;
		\item Dung dịch $\mathrm{HCl}$ $0,01\;M$;
		\item Dung dịch $\mathrm{Mg{(OH)}_2}$ $0,002\;M$.
	\end{enumEX}
	\loigiai{
		\begin{enumerate}
			\item Dung dịch $\mathrm{NaOH}$ $0{,}001\;M$:
			\[
			\begin{array}{ccccc}
				NaOH& \xrightarrow& Na^+& +& OH^-\\
				0{,}001&&&\rightarrow&0{,}001
			\end{array}
			\]
			\begin{align*}[\mathrm{OH^-}]
				&= 0,001 \mathrm{M} \\
				\mathrm{pOH} &= -\log[\mathrm{OH^-}] = -\log(0,001) = 3 \\
				\mathrm{pH} &= 14 - \mathrm{pOH} = 14 - 3 = 11
			\end{align*}
			\item Dung dịch $\mathrm{HCl}$ $0,01\;M$:
			\[
			\begin{array}{ccccc}
				HCl& \xrightarrow& H^+&+& Cl^-\\
				0{,}01&\rightarrow &0{,}01&&
			\end{array}
			\]
			\begin{align*}[\mathrm{H^+}]
				&= 0,01 \mathrm{M} \\
				\mathrm{pH} &= -\log[\mathrm{H^+}] = -\log(0,01) = 2
			\end{align*}
			\item Dung dịch $\mathrm{Mg{(OH)}_2}$ $0{,}002\;M$:
			\[
			\begin{array}{ccccc}
				Mg(OH)_2& \xrightarrow& Mg^{2+}&+& 2OH^-\\
				0{,}002&&&\rightarrow &0{,}004
			\end{array}
			\]
			\begin{align*}[\mathrm{OH^-}]
				&= 2 \times 0,002 = 0,004 \mathrm{M} \\
				\mathrm{pOH} &= -\log[\mathrm{OH^-}] = -\log(0,004) = 2,4 \\
				\mathrm{pH} &= 14 - \mathrm{pOH} = 14 - 2,4 = 11,6
			\end{align*}
		\end{enumerate}
	}
\end{vd}
%%%=============VD_3=============%%%
\begin{vd}
	Tính pH của dung dịch sau khi trộn $100$ ml dung dịch $H_2SO_4$ $0{,}1$ M với $200$ ml dung dịch $HCl$ $0{,}2$ M.
	\loigiai{%
		Nồng độ $H^+$ sau khi trộn $[H^+]=\dfrac{2\cdot C_1\cdot V_1 + C_2\cdot V_ 2}{V_1+V_2}= \dfrac{2\cdot 0{,}1\cdot 0{,}1 + 0{,}2\cdot 0{,}2}{0{,}1+0{,}2}=0{,}2$
		\\
		$\Rightarrow pH =-log(0{,}2)=0{,}7$
	}
\end{vd}
%%%=============VD_4=============%%%
\begin{vd}
	Cho $200$ ml $H_2SO_4$ $0{,}05$ M vào $300$ ml dung dịch $\mathrm{NaOH}$ $0{,}06$ M. pH của dung dịch tạo thành là
	\choice
	{$2{,}7$}
	{$1{,}6$}
	{$1{,}9$}
	{\True $2{,}4$}
	\loigiai{%
		$n_{H^+}=2n_{H_2SO_4}=2\cdot0{,}2\cdot0{,}05=0{,}02$ mol;
		$n_{OH^-}=2n_{NaOH} =0{,}3\cdot0{,}06=0{,}018$ mol
		\[
		\begin{matrix}
			H^+&+& OH^-& \xrightarrow & H_2O\\
			0{,}018&\leftarrow&0{,}018&&
		\end{matrix}
		\]
		$\Rightarrow$ $n_{H^+\text{dư}}= 0{,}02-0{,}018=0{,}002$ mol $\Rightarrow [H^+]=\dfrac{0{,}002}{0{,}5}=0{,}004$ M
		\\
		$\Rightarrow pH=-log[H^+]=-log(0{,}004)=2{,}4$
	}
\end{vd}
%%%=============VD_5=============%%%
\begin{vd}
	Dung dịch X là hỗn hợp $Ba{(OH)}_2$ $0{.}1$ M và $NaOH$ $0{.}1$ M. Dung dịch Y là hỗn hợp của $H_2SO_4$ $0{,}0375$ M; $HCl$ $0{,}0125$ M. Trộn $100$ ml dung dịch X với $400$ ml dung dịch Y thu được dung dịch Z. pH của dung dịch Z là
	\choice
	{$1$}
	{$7$}
	{$2$}
	{$6$}
	\loigiai{
		\indam{Phân tích:} Bài toán trộn dung dịch, lưu ý phải tính lại nồng độ các chất vì thể tích dung dịch  thay đổi. Xác định chất dư để tính pH theo chất đó.
		\\[5pt]
		$\left.
		\begin{aligned}
			Ba(OH)_2:[OH^-]=0{,}2 \;M\\
			NaOH :[OH^-]=0{,}1\; M
		\end{aligned}
		\right\}$ $\Rightarrow$ $\sum[OH^-]=0{,}3$ M $\Rightarrow$ $nOH^-=0{,}3\cdot0{,}1=0{,}03$ mol.
		\\
		$\left.
		\begin{aligned}
			H_2SO_4 :[H^+]=0{,}075\;M\\
			HCl:[H^+]=0{,}0125\;M
		\end{aligned}
		\right\}$ $\Rightarrow$ $\sum[H^+]=0{,}0875$ M $\Rightarrow$ $nH^+=0{,}0875\cdot0{,}4=0{,}035$ mol.
		\[
		\begin{matrix}
			& H^+&+& OH^- & \xrightarrow & H_2O\\
			&0{,}03\;\text{mol}&\xleftarrow&0{,}03\;\text{mol}&&
		\end{matrix}
		\]
		$\Rightarrow$ $n_{H^+\text{dư}}=\dfrac{0{,}005}{0{,}5}=0{,}01\;M$ $\Rightarrow pH =2$.
	}
\end{vd}
\Noibat[\maunhan][][\faAndroid]{pH của dung dịch acid/base yếu}
\begin{hopvidu}[\maunhan]
	\taodongke[0.9][\dotfill][gray]{10}
\end{hopvidu}
%%%%==========================%%%%
\begin{hopvidu}[\maunhan]
	\taodongke[0.9][\dotfill][gray]{10}
\end{hopvidu}
\Noibat[][][\faBank]{Một số ví dụ minh họa}
%%%=============VD_1=============%%%
\begin{vd}
	Tính pH của các dung dịch sau:
	\begin{enumerate}
		\item $CH_3COOH$ $0{,}1$M có $K_a=1{,}75\cdot10^{-5}$.
		\item $NH_3$ $0{,}10$M có $K_b=1{,}80\cdot10^{-5}$.
	\end{enumerate}
	\loigiai{
		\begin{enumerate}
			\item \phantom{x}
			
			$
			\begin{matrix}
				&CH_3COOH& +& \mathrm{H}_2\mathrm{O}& \xrightleftharpoons{}& CH_3COO^-&+& H_3O^+&\\
				\text{ban đầu:}	&0{,}1&&&&&&&\\
				\text{phản ứng:}&-x&&&&+x&&+x&\\
				\text{cân bằng:}&0{,}1-x&&&&x&&x&
			\end{matrix}
			$\\
			Ta có $K_a=\dfrac{x \cdot x}{0{,}1-x} = 1{,}75\cdot10^{-5} \Rightarrow [H^+] = x = 1{,}31\cdot10^{-3} $ $\Rightarrow pH =-log(1{,}31\cdot10^{-3}) = 2{,}88$
			\item \phantom{x}
			
			$
			\begin{matrix}
				&\mathrm{NH_3}& +& \mathrm{H}_2\mathrm{O}& \xrightleftharpoons{}& \mathrm{NH_4}^+& +& \mathrm{OH}^-&\\
				\text{ban đầu:}	&0{,}1&&&&&&&\\
				\text{phản ứng:}&-x&&&&+x&&+x&\\
				\text{cân bằng:}&0{,}1-x&&&&x&&x&
			\end{matrix}
			$\\
			$K_b=\dfrac{x \cdot x}{0{,}1-x} = 1{,}80\cdot10^{-5} $
			$\Rightarrow [OH^-] = x = 1{,}33\cdot10^{-3} $ $\Rightarrow [H^+]=\dfrac{10^{-14}}{1{,}33\cdot10^{-3}} = 7{,}5\cdot10^{-12}\\ \Rightarrow pH =-log(7{,}5\cdot10^{-12}) = 11{,}12$
		\end{enumerate}
	}
\end{vd}
%%%=============VD_2=============%%%
\begin{vd}
	Tính pH của dung dịch $HClO$ (axit hypochlorous) $0,05$M biết $K_a = 3,0 \cdot 10^{-8}$.
	\loigiai{
		$
		\begin{matrix}
			&HClO& +& \mathrm{H}_2\mathrm{O}& \xrightleftharpoons{}& ClO^-&+& H_3O^+&\\
			\text{ban đầu:}	&0{,}05&&&&&&&\\
			\text{phản ứng:}&-x&&&&+x&&+x&\\
			\text{cân bằng:}&0{,}05-x&&&&x&&x&
		\end{matrix}
		$
		
		Ta có: $K_a=\dfrac{x \cdot x}{0{,}05-x} = 3{,}0\cdot10^{-8}$
		
		Giả sử $x \ll 0{,}05$, ta có:
		
		$x^2 = 3{,}0\cdot10^{-8} \cdot 0{,}05 = 1{,}5\cdot10^{-9}$
		
		$x = \sqrt{1{,}5\cdot10^{-9}} = 1{,}22\cdot10^{-5}$
		
		Kiểm tra giả thiết: $\dfrac{1{,}22\cdot10^{-5}}{0{,}05} = 2{,}44\cdot10^{-4} \ll 1$ (giả thiết đúng)
		
		Vậy $[H^+] = 1{,}22\cdot10^{-5}$
		
		$pH = -\log[H^+] = -\log(1{,}22\cdot10^{-5}) = 4{,}91$
	}
\end{vd}
%%%=============VD_3=============%%%
\begin{vd}
	Tính pH của dung dịch $CH_3NH_2$ (methylamine) $0,2$M biết $K_b = 4,38 \cdot 10^{-4}$.
	\loigiai{
		$
		\begin{matrix}
			&CH_3NH_2& +& \mathrm{H}_2\mathrm{O}& \xrightleftharpoons{}& CH_3NH_3^+&+& OH^-&\\
			\text{ban đầu:}	&0{,}2&&&&&&&\\
			\text{phản ứng:}&-x&&&&+x&&+x&\\
			\text{cân bằng:}&0{,}2-x&&&&x&&x&
		\end{matrix}
		$
		
		Ta có: $K_b=\dfrac{x \cdot x}{0{,}2-x} = 4{,}38\cdot10^{-4}$
		
		Giải phương trình: $x^2 + 4{,}38\cdot10^{-4}x - 8{,}76\cdot10^{-5} = 0$
		
		$x = \dfrac{-4{,}38\cdot10^{-4} + \sqrt{(4{,}38\cdot10^{-4})^2 + 4\cdot8{,}76\cdot10^{-5}}}{2} = 8{,}85\cdot10^{-3}$
		
		Vậy $[OH^-] = 8{,}85\cdot10^{-3}$
		
		$pOH = -\log[OH^-] = -\log(8{,}85\cdot10^{-3}) = 2{,}05$
		
		$pH = 14 - pOH = 14 - 2{,}05 = 11{,}95$
	}
\end{vd}
%%%=============VD_4=============%%%
\begin{vd}
	Tính pH của dung dịch $HCOOH$ (axit formic) $0,1$M biết $K_a = 1,8 \cdot 10^{-4}$.
	\loigiai{
		$
		\begin{matrix}
			&HCOOH& +& \mathrm{H}_2\mathrm{O}& \xrightleftharpoons{}& HCOO^-&+& H_3O^+&\\
			\text{ban đầu:}	&0{,}1&&&&&&&\\
			\text{phản ứng:}&-x&&&&+x&&+x&\\
			\text{cân bằng:}&0{,}1-x&&&&x&&x&
		\end{matrix}
		$
		
		Ta có: $K_a=\dfrac{x \cdot x}{0{,}1-x} = 1{,}8\cdot10^{-4}$
		
		Giải phương trình: $x^2 + 1{,}8\cdot10^{-4}x - 1{,}8\cdot10^{-5} = 0$
		
		$x = \dfrac{-1{,}8\cdot10^{-4} + \sqrt{(1{,}8\cdot10^{-4})^2 + 4\cdot1{,}8\cdot10^{-5}}}{2} = 4{,}02\cdot10^{-3}$
		
		Vậy $[H^+] = 4{,}02\cdot10^{-3}$
		
		$pH = -\log[H^+] = -\log(4{,}02\cdot10^{-3}) = 2{,}40$
	}
\end{vd}
%%%=============VD_5=============%%%
\begin{vd}
	Tính pH của dung dịch $C_6H_5COOH$ (axit benzoic) $0,02$M biết $K_a = 6,3 \cdot 10^{-5}$.
	\loigiai{
		$
		\begin{matrix}
			&C_6H_5COOH& +& \mathrm{H}_2\mathrm{O}& \xrightleftharpoons{}& C_6H_5COO^-&+& H_3O^+&\\
			\text{ban đầu:}	&0{,}02&&&&&&&\\
			\text{phản ứng:}&-x&&&&+x&&+x&\\
			\text{cân bằng:}&0{,}02-x&&&&x&&x&
		\end{matrix}
		$
		
		Ta có: $K_a=\dfrac{x\cdot x}{0{,}02-x} = 6{,}3\cdot 10^{-5}$
		
		Giả sử $x << 0{,}02$, ta có:
		
		$x^2 = 6{,}3\cdot 10^{-5} \cdot 0{,}02 = 1{,}26\cdot 10^{-6}$
		
		$x = \sqrt{1{,}26 \cdot 10^{-6}} = 1{,}12\cdot 10^{-3}$
		\\
		Kiểm tra giả thiết: $\dfrac{1{,}12\cdot 10^{-3}}{0{,}02} = 0{,}056 < 0{,}05$ (giả thiết đúng)
		\\
		Vậy $[H^+] = 1{,}12\cdot 10^{-3}$ $\Rightarrow$
		$pH = -\log[H^+] = -\log(1{,}12\cdot 10^{-3}) = 2{,}95$
	}
\end{vd}

\end{document}





