\documentclass[Main.tex]{subfiles}
\begin{document}
%%%Nhớ tắt 3 lệnh này khi chạy filemain
%\setcounter{tocdepth}{1}
%\setcounter{secnumdepth}{3}
%\tableofcontents
\titlespacing*{\subsubsection}{0cm}{0pt}{0pt}

{$N_2$(g) + $3H_2$(g) $\xleftrightarrow[$400 -600 ^\circ C$][$200 bar, Fe$][2.2]$ $2NH_3$(g)}

%\subsection{Khái niệm phản ứng thuận nghich và trạng thái cân bầng}
%\section{pH của dung dịch- chuẩn độ acid và base}
%
%\chapter{Nitrogen và sulfur}
%\section{Đơn chất nitrogen}
%\section{Một số hợp chất quan trọng của nitrogen}
%\section{Sulfur và sulfur dioxide}
%\section{Sulfuric acid và muối sulfate}
%
%
%\chapter{Đại cương hoá học hữu cơ}
%\section{Hợp chất hữu cơ và hoá học hữu cơ}
%\section{Phương pháp tách biệt và tinh chế hợp chất hữu cơ}
%\section{Công thức phân tử hợp chất hữu cơ}
%\section{Cấu tạo hợp chất hữu cơ}
%
%\chapter{Hydrocarbon}
%\section{Alkane}
%\section{Phương pháp tách biệt và tinh chế hợp chất hữu cơ}
%\section{Hydrocarbon không no}
%\section{Arene}
%
%\chapter{Dẫn xuất halogen-alcohol-phenol}
%\section{Dẫn xuất halogen}
%\section{Phenol}
%
%
%\chapter[Hợp chất carbonyl - carboxylic acid]{Hợp chất carbonyl (aldehyde - ketone) - carboxylic acid}
%\section{Hợp chất carbonyl}
%\section{Carboxylic acid}
%\begin{dang}{Bài toán tính số hạt}
%	\begin{ppg}
%		\begin{cacbuoc}
%			\item
%			\item
%		\end{cacbuoc}
%	\end{ppg}
%\end{dang}
%	\bangdapan{Ans-Hoa10_C01_CTNT}
%	\bangdapanTF{AnsTF-Hoa10_C01_CTNT}
%	\bangdapanSA{AnsBT-Hoa10_C01_CTNT}
\end{document}

