\documentclass[Main.tex]{subfiles}
\setcounter{section}{1}
\begin{document}
	\begin{center}
		\Noibat[\maunhan][\myfont[20]{qag}][\faApple]{Tìm hiểu về nhiệt động lực học}
	\end{center}
	\subsubsection{Định luật thứ nhất}
	\Noibat{Định luật thứ nhất của nhiệt động lực học}
	\begin{itemize}
		\item \textbf{Định luật thứ nhất} phát biểu rằng nội năng của một hệ thống có thể thay đổi khi có công thực hiện hoặc nhiệt truyền vào. Công thức của định luật thứ nhất là:
		\begin{equation}
			\Delta U = q + w\label{eq:dinhluat2nhietdonghoc}
		\end{equation}
		Trong đó:
		\begin{itemize}
			\item $\Delta U$ là sự thay đổi nội năng của hệ thống.
			\item $q$ là nhiệt lượng truyền vào hệ thống.
			\item $w$ là công thực hiện trên hệ thống.
		\end{itemize}
		\item \textbf{Công thức vi phân} cho một sự thay đổi nhỏ trong trạng thái của hệ:
		\begin{equation}
			dU = \delta q + \delta w
			\label{eq:tichphannoinang}
		\end{equation}
		Biểu thức này mô tả sự thay đổi vi phân của nội năng $U$ khi một lượng nhiệt nhỏ $ \delta q $ và công nhỏ $ \delta w $ được cung cấp cho hệ.
		\item \textbf{Nội năng trong chu trình kín}
		\begin{equation}
			\oint dU = 0
			\label{eq:tichphanchutrinhkin}
		\end{equation}
		Công thức ở phương trình (\ref{eq:tichphanchutrinhkin}) được gọi là tích phân vòng tròn
	\end{itemize}
	\Noibat{Giải thích tổng quát:}
	\begin{itemize}
		\item Khi một hệ thống hấp thụ nhiệt mà không có công thực hiện, toàn bộ nhiệt lượng đó được dùng để tăng nội năng của hệ thống, làm tăng nhiệt độ hoặc thay đổi trạng thái của hệ.
		\item Trong trường hợp hệ thống thực hiện công hoặc nhận công từ môi trường, tổng nội năng thay đổi phụ thuộc vào cả nhiệt lượng truyền vào và công thực hiện.
	\end{itemize}

\end{document}

