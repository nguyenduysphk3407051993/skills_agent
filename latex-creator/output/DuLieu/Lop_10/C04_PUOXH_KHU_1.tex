\newenvironment{cacbuoc}{\begin{enumerate}[label= \color{\maunhan}\bfseries\fontfamily{qag}\selectfont{\faAdjust\;Bước \arabic*:},itemsep=0pt,wide=0cm,leftmargin=0.5cm,topsep=0pt]
	}{\end{enumerate}}
\newtcolorbox{body}{%
	enhanced,
	before skip=1cm,
	breakable,
	colback=\mycolor!20,
	enhanced jigsaw,opacityback=0,opacitybacktitle=0,
	opacityback=0,
	colframe=\mycolor
}
\renewcommand{\thesubsubsection}{\Roman{subsubsection}}
\titlespacing*{\subsection}{3pt}{0pt}{5pt}
\titlespacing*{\subsubsection}{3pt}{5pt}{5pt}
\titlespacing*{\paragraph}{0cm}{0cm}{5pt}
\setcounter{chapter}{3}
\chapter{Phản ứng Oxi hóa - khử}
\section{Số Oxi hóa - Cân bằng phản ứng oxi hóa khử}
%%%====================Bắt đầu khung lý thuyết======================%%%
\subsection{Kiến thức cần nhớ}
\begin{body}
	\subsubsection{Số oxi hóa}
	\begin{dn}
		\indam{Số oxi hóa} của một nguyên tử một nguyên tố trong hơp chất là \indam{điện tích} của nguyên tử nguyên tố đó với giả định đây là hợp chất ion.
	\end{dn}
	\subsubsection{Cách xác định số oxi - hóa}
	\begin{enumerate}
		\item \indam{Quy tắc 1:}
		Trong đơn chất, số oxi hóa của các nguyên tố bằng 0.\\
		VD: $\overset{0}{\mathrm{H}_{2}}$, $\overset{0}{\mathrm{Na}}$, $\overset{0}{\mathrm{O}_{3}}$, $\ldots$
		\item \indam{Quy tắc 2:}
		Trong một phân tử,tổng số oxi hóa  của các nguyên tố nhân với số nguyên tử của từng nguyên tố bằng 0\\
		\item \indam{Quy tắc 3:} 
		\begin{itemize}
			\item Trong các ion \indam{đơn nguyên tử}, số oxi hóa của các nguyên tố bằng điệnu tích của ion đó\\
			\item Trong ion \indam{đa nguyên tử}, tổng số oxi hóa các nguyên tố nhân với số nguyên tử của từng nguyên tố bằng điện tích của ion
		\end{itemize}
		\item \indam{Quy tắc 4:} \\
		\begin{itemize}
			\item Trong đa số các hợp chất, số oxi hóa của hydrogen bằng +1 , trừ hydride kim loại ($NaH$, $CaH_2$)\\
			\item Số oxi hóa của oxygen bằng $-\mathrm{2}$, trừ trường hợp $\overset{+1}{\mathrm{O}_{}}\overset{\vphantom{-1}}{\mathrm{F}_{2}}$ và peoxit $\left(\overset{\vphantom{+1}}{\mathrm{Na}_{2}}\overset{-1}{\mathrm{O}_{2}}\right)$, superoxide $\left(\overset{\vphantom{+1}}{\mathrm{K}_{}}\overset{-\tfrac{1}{2}}{\mathrm{O}_{2}}, \ldots\right)$.\\
			\item Các nguyên tố nhóm IA, IIA luôn có số oxi hóa $+1,+2$, số oxi hóa của $\mathrm{Al}$ là +3 . Số oxi hóa của nguyên tử nguyên tố fluorine trong các hợp chất bằng -1
		\end{itemize}
	\end{enumerate}
	\subsubsection{Phản ứng Oxi hóa - khử}
	\begin{mylt}
		\paragraph{Sự oxi hóa-sự khử}
		\begin{enumerate}[label = \indam{\alph*)}]
			\item \indam{Sự oxi hóa} là sự nhường electron, là sự tăng số oxi hóa.\\
			\indam{Ví dụ:} $\overset{0}{Mg} \rightarrow Mg^{2+} + 2e$
			\item \indam{Sự khử} là sự thu electron, là sự giảm electron.\\
			\indam{Ví dụ:} $Cu^{2+} +2e \rightarrow Cu $
		\end{enumerate}
	\end{mylt}
	\begin{mylt}
		\paragraph{Chất oxi hóa - chất khử}
		\begin{enumerate}[label = \indam{\alph*)}]
			\item \indam{Chất oxi hóa} là chất thu electron.là chất chứa nguyên tố có số oxi hóa giảm sau phản ứng.\\
			\indam{Chất oxihóa} còn gọi là chất \indam{bị khử}.
			\item \indam{Chất khử} là chất nhường electron, là chất có số oxi hóa tăng sau phản ứng\\
			\indam{Chất khử} còn gọi là chất \indam{bị oxi hóa}.
		\end{enumerate}
	\end{mylt}
	\begin{vdnote}
		$ \overset{0}{Mg} + \overset{+2}{Cu}SO_{4} \rightarrow\overset{+2}{Mg}SO_{4}  + \overset{0}{Cu}$.\\
		$Cu^{2+}$ nhận electron, là chất oxi hóa ( số oxi hóa giảm từ +2 về 0).
		$Mg$ nhường electron, là chất khử (số oxi hóa tăng từ 0 lên +2)
	\end{vdnote}
	\begin{vdnote}
		$\overset{+3}{Fe_2}\overset{\vphantom{+3}}{O_3} + 3\overset{+2}{C}O$ $\longrightarrow$ $ 2 \overset{0}{Fe} + 3\overset{+4}{C}O_2$
	\end{vdnote}
	\begin{mylt}
		\paragraph{Phản ứng Oxi hóa - khử }
		\indam{Phản ứng oxi - hóa khử } là phản ứng hóa học trong đó có sự chuyển electron giữa các chất phản ứng.\\
		Nếu dựa vào sự thay đổi số oxi hóa thì phản ứng oxi hóa-khử là phản ứng hóa học trong đó có sự  thay đổi số oxi hóa của một số nguyên tố.\\
		
		\begin{vdnote}
			\[\begin{tikzpicture}
				\tikzstyle{mynode} =[
				font=\normalsize,
				line width =.8pt,
				anchor=center,
				align =center,
				]
				%%%==================%%%
				\tikzstyle{mymatrix} = [
				matrix of nodes,
				nodes in empty cells,
				nodes={mynode},
				column sep=-\pgflinewidth,
				row sep = -\pgflinewidth,
				minimum width = .5cm,
				minimum height = .6cm,
				column 4/.style={
					minimum width = .7cm,
				},
				]
				%%%=======================================================%%%
				\matrix(m) [mymatrix]{
					Mg & + & $CuSO_4$ & [-.7cm]\muiten[][][0.7]{->}& $MgSO_4$ & + & Cu\\
				};
				\node(0)[yshift=4pt,\maunhan] at (m-1-1.north){0};
				\node(2)[xshift =-8pt,yshift=4pt,\maunhan] at (m-1-3.north){+2};
				\node[xshift =-8pt,yshift=4pt,\maunhan] at (m-1-5.north){+2};
				\node[yshift=4pt,\maunhan] at (m-1-7.north){0};
				\draw[->,>=stealth,violet,ultra thick](0.90)--++(90:.5cm)-|(2.90) ;
				\path ($(0.90)+(90:.5cm)$)-- ($(2.90)+(90:.5cm)$) node [pos=.5,above,violet]{- 2 e};
			\end{tikzpicture}\]
			Trong phản ứng trên $Mg$ nhường đi 2 electron cho ion $Cu^{+2}$ trở thành ion $Mg^{+2}$ và ion $Cu^{+2}$ nhận  2 electron từ nguyên tử $Mg$ trở thành nguyên tử Cu.
		\end{vdnote}
		%%%===============Môi Trường Matrix=================%%%
	\end{mylt}
\begin{mylt}
	\paragraph{Ý nghĩa của phản ứng oxi hóa khử }
	Phản ứng oxi hóa - khử là một trong những quá trình quan trọng nhất của thiên nhiên:
	\begin{itemize}
		\item  Sự hô hấp, quá trình thực vật hấp thụ khí cacbonic giải phóng oxi, sự trao đổi chất và hàng loạt quá trình sinh học khác đều có cơ sở là các phản ứng oxi hóa - khử.
		\item  Ngoài ra: Sự đốt cháy nhiên liệu trong các động cơ, các quá trình điện phân, các phản ứng xảy ra trong pin và trong ăcquy đều bao gồm sự oxi hóa và sự khử.
		\item  Hàng loạt quá trình sản xuất như luyện kim, chế tạo hóa chất, chất dẻo, dược phẩm, phân bón hóa học, ... đều không thực hiện được nếu thiếu các phản ứng oxi hóa - khử.
	\end{itemize}
\end{mylt}
	\subsubsection{Cân bằng phản ứng oxi hóa-khử}
	\begin{mylt}
		\GSND[\bfseries\sffamily][\faStar]{Phương pháp thăng bằng electron:}\\
		Phương pháp này dựa trên nguyên tắc:
		\boxct[\mauphu][3pt][\bfseries\sffamily\color{violet}]{Tổng số electron chất khử nhường  = tổng số electron mà chất oxi hóa nhận}
		\indam{\faBook\ Các bước cân bằng:}\vspace{.5cm}
		\begin{vdnote}
			Lập phương trình phản ứng Oxi hóa - khử sau: 
			\[\puhh[$t^{\circ}$]{$Cu$\+ $HNO_3$}{->}{$Cu({NO_3})_2$ \+ $NO\uparrow$ \+ $H_2O$}\]
		\end{vdnote}
		\indam{\itshape Hướng dẫn giải:}
		\begin{cacbuoc}
			\item Xác định số oxi hóa của nguyên tử cóa sự thay đổi số oxi hóa, xác định chất khử, chất oxi hóa
			\[\begin{tikzpicture}
				\tikzstyle{mynode} =[
				font=\normalsize,
				line width =.8pt,
				anchor=center,
				align =center,
				]
				%%%==================%%%
				\tikzstyle{mymatrix} = [
				matrix of nodes,
				nodes in empty cells,
				nodes={mynode},
				column sep=-\pgflinewidth,
				row sep = -\pgflinewidth,
				minimum width = .5cm,
				minimum height = .6cm,
				column 4/.style={
					minimum width = .7cm,
				},
				]
				%%%=======================================================%%%
				\matrix(m) [mymatrix]{
					$Cu$ & + & $HNO_3$ & [-.7cm]\muiten[][][0.7]{->}& $Cu({NO_3})_{2}$ & + & $NO\uparrow$ & + & $H_2O$ \\
				};
				\node(0)[yshift=4pt,\maunhan] at (m-1-1.north){0};
				\node(chatkhu)[yshift=-4pt,\maunhan,font=\footnotesize] at (m-1-1.south){(Chất khử)};
				\node(2)[xshift =-2pt,yshift=4pt,\maunhan] at (m-1-3.north){+5};
				\node(chatoxh)[yshift=-4pt,\maunhan,font=\footnotesize] at (m-1-3.south){(Chất oxi hóa)};
				\node[xshift =-18pt,yshift=4pt,\maunhan] at (m-1-5.north){+3};
				\node[xshift =-8pt,yshift=4pt,\maunhan] at (m-1-7.north){+2};
			\end{tikzpicture}\]
			\item Viết quá trình oxi hóa và quá trình khử
			\[\begin{tikzpicture}
				\tikzstyle{mynode} =[
				font=\normalsize,
				line width =.8pt,
				anchor=center,
				align =center,
				minimum width = 0.4cm,
				minimum height = 1.0cm,
				]
				%%%==================%%%
				\tikzstyle{mymatrix} = [
				matrix of nodes,
				nodes in empty cells,
				nodes={mynode},
				column sep=-\pgflinewidth,
				row sep = 2pt-\pgflinewidth,
				column 1/.style={
					minimum width = 4cm,
					anchor =east,
					align =center,
				},
				]
				%%%=======================================================%%%
				\matrix(m) [mymatrix]{
					\text{Quá trình oxi hóa :} & Cu & [-.5cm]\muiten[][][0.4]{->}& Cu & + & 2e\\
					\text{Quá trình khử :} & N & + & [-.5cm]3e & [-.5cm]\muiten[][][0.4]{->} & NO\\
				};
				\node(0)[yshift=-2pt,\maunhan] at (m-1-2.north){0};
				\node(3)[yshift=-2pt,\maunhan] at (m-1-4.north){+2};
				\node(5)[yshift=-2pt,\maunhan] at (m-2-2.north){+5};
				\node(2)[xshift=-2pt,yshift=-2pt,\maunhan] at (m-2-6.north){+2};
			\end{tikzpicture}\]
			\item Nhân hệ số thích hợp  vào các quá trình  sao cho tổng  số electron chất khử nhường  bằng tổng số electron chất oxi hóa nhận.
			\[\begin{tikzpicture}
				\tikzstyle{mynode} =[
				font=\normalsize,
				line width =.8pt,
				anchor=center,
				align =center,
				minimum width = 0.4cm,
				minimum height = 1.0cm,
				]
				%%%==================%%%
				\tikzstyle{mymatrix} = [
				matrix of nodes,
				nodes in empty cells,
				nodes={mynode},
				column sep=3pt-\pgflinewidth,
				row sep = 2pt-\pgflinewidth,
				column 1/.style={
					minimum width = 4cm,
					anchor =east,
					align =center,
				},
				]
				%%%=======================================================%%%
				\matrix(m) [mymatrix]{
					3 x & Cu & [-.5cm]\muiten[][][0.4]{->}& Cu & + & 2e\\
					2 x & N & + & [-.5cm]3e & [-.5cm]\muiten[][][0.4]{->} & NO\\
				};
				\node(0)[yshift=-2pt,\maunhan] at (m-1-2.north){0};
				\node(3)[yshift=-2pt,\maunhan] at (m-1-4.north){+2};
				\node(5)[yshift=-2pt,\maunhan] at (m-2-2.north){+5};
				\node(2)[xshift=-2pt,yshift=-2pt,\maunhan] at (m-2-6.north){+2};
				\draw[\maunhan,line width=1pt] (m-1-1.north east)--(m-2-1.south east);
			\end{tikzpicture}\]
			\item Đặt các hệ số vào sơ đồ phản ứng. Can bằng số lượng  nguyên tử của các nguyên tố còn lại.
			\[\puhh[$t^{\circ}$]{$3Cu$ \+ ${8HNO_3}_{\text{loãng}}$}{->}{$3Cu({NO_3})_{2}$ \+ $2NO$ \+ $4H_2O$}\]
		\end{cacbuoc}
	\end{mylt}
\end{body}
%%====================Kết thức khung lý thuyết======================%%%
\newpage
\subsection{Các dạng bài tập}
\begin{dangNTD}{Câu hỏi lý thuyết về phản ứng oxi hóa khử}
\end{dangNTD}
\begin{vdm}
\end{vdm}
%\hienthiloigiaivd
\dongkeHaicotvd
%%%=========vd_1=========%%%
\begin{vd}[Nhận biết các quá trình vai trò của các chất][][]Một nguyên tử nhôm (Al) chuyển thành ion $Al^{3+}$ thực hiện :
	\choice
	{%
		nhận thêm 3 electron (quá trình oxi hóa)
	}{%
		nhường đi 3 electron (quá trình khử)
	}{%
		nhận thêm 3 electron (quá trình khử)
	}{%
		\True nhường đi 3 electron (quá trình oxi hóa)
	}
	\loigiai{Quá trình oxi hóa  một chất là làm cho nguyên tử trong chất đó nhường electron hay làm tăng số oxi hóa của nguyên tử trong chất đó.\\
		Quá trình khử  một chất là làm cho nguyên tử trong chất đó nhận electron hay làm giảm số oxi hóa của nguyên tử trong chất đó.\\
		Ta thấy số oxi hóa của nhôm tăng (quá trình oxi hóa) \\
		$\begin{matrix}
			\overset{0}{Al}& \muiten[][][.4]{->} &\overset{+3}{Al} & + & 3e
		\end{matrix}$
	}
\end{vd}
%%%=========vd_2=========%%%
\begin{vd}[Nhận biết các quá trình vai trò của các chất][][]Trong phản ứng : \puhh{$Cl_2$\+ $2KBr$}{->}{$Br_2$\+ 2KCl}, $Cl_2$ đóng vai trò 
	\choice{%
		\True là chất oxi hóa
	}{%
		là chất khử
	}{%
		không bị oxi hóa, không bị khử
	}{%
		vừa bị oxi hóa, vừa bị khử.
	}
	\loigiai{Chất oxi hóa là chất nhận electron hay là chất có số oxi hóa giảm sau phản ứng.\\
		Chất khử là chất nhường electron hay là chất có số oxi hóa tăng sau phản ứng.\\
		Ta thấy số oxi hóa của Cl giảm :\\
		$\begin{matrix}
			Cl_2 & + & 2e & \muiten[][][.4]{->} & 2Cl^{-}
		\end{matrix}$
	}
\end{vd}

%%%=========vd_3=========%%%
\begin{vd}[Nhận biết phản ứng oxi hóa khử][][]
	Phản ứng nào sau đây \indam[black]{không phải} phản ứng oxi hóa khử
	\choice{
		\puhh{$2NaOH$\+ $Cl_2$}{->}{$NaCl$\+$NaClO$\+ $H_2O$}
	}{
		\True \puhh[$t^{\circ}$]{$CaCO_3$}{->}{$CaO$\+$CO_2$}
	}{
		\puhh[$t^{\circ}$]{$2KClO_3$}{->}{$2KCl$\+$3O_2$}
	}{
		\puhh{$Fe$\+ $2HCl$}{->}{$FeCl_2$\+$H_2$}
	}
	\loigiai{%
		Phản ứng oxi hóa - khử là phản ứng  hóa học trong đó có sự thay đổi số oxi hóa.\\
		Phản ứng không phải là phản ứng oxi hóa - khử là phản ứng không có sự thay đổi số oxi hóa của các nguyên tố\\
		Ta có:\\
		$\begin{matrix}
			2NaOH  & + & \overset{0}{Cl}_{2} & \muiten{->} & \overset{\vphantom{+1}}{Na}\overset{-1}{Cl} & + & \overset{\vphantom{+1}}{Na}\overset{+1}{Cl}O  & + & H_2O\\
		\end{matrix}$\\
		$\begin{matrix}
			\overset{\vphantom{+4}}{Ca}\overset{+4}{C}\overset{\vphantom{-2}}{O_3} & \muiten{->}& \overset{+2}{Ca}O & + & \overset{+4}{C}\overset{\vphantom{+4}}{O_2}\\
		\end{matrix}$\\
		$\begin{matrix}
			\overset{\vphantom{-2}}{2K}\overset{+5}{Cl}\overset{\vphantom{-2}}{O_3} & \muiten{->}& \overset{\vphantom{-2}}{2K}\overset{-1}{Cl} & + & 3\overset{-2}{O_2}\\
		\end{matrix}$\\
		$\begin{matrix}
			\overset{0}{Fe}& + & 2\overset{+1}{H}Cl & \muiten{->}& \overset{+2}{Fe}\overset{\vphantom{-2}}{Cl_2} & + & \overset{0}{H_2} \\
		\end{matrix}$\\
		\indam{Nhận xét: }Phản ứng B không có sự thay đổi số oxi hóa của các nguyên tố. $\longrightarrow$ Phản ứng B không phải là phản ứng oxi hóa - khử
	}
\end{vd}


%%%=========vd_4=========%%%
\begin{vd}[Nhận biết phản ứng oxi hóa khử][][]Cho các phản ứng sau đây:
	\begin{enumerate}[a)]
		\item $C + O_{2} \xrightarrow{t^{\circ}} CO_{2}$
		\item $CaO +  H_{2}O \longrightarrow Ca(OH)_{2}$
		\item $CuO + H_{2} \xrightarrow{t^{\circ}} Cu + O_2$
		\item $2KMnO_4 \xrightarrow{t^{\circ}} K_2MnO_4 + MnO_2 + H_2O$
		\item $Cl_2 + 2KOH \longrightarrow KCl + KClO + H_2O$
		\item $Fe_3O_4 + 8HCl \longrightarrow 2FeCl_3 + FeCl_2 + 4H_2O$
	\end{enumerate}
	Số phản ứng thuộc loại phản ứng oxi hóa - khử là:
	\choice{2}{\True 4}{3}{5}
	\loigiai{%
		Phản ứng oxi hóa khử là phản ứng hóa học trong đó có sự thay đổi số oxi hóa của một số nguyên tố.
		\begin{enumerate}[a)]
			\item $\overset{0}{C} + \overset{0}{O}_{2} \xrightarrow{t^{\circ}} \overset{+4}{C}\overset{-2}{O}_{2}$
			\item $\overset{+2}{Ca}\overset{-2}{O} +  \overset{+1}{H_{2}}\overset{-2}{O} \longrightarrow \overset{+2}{Ca}(\overset{-2}{O}\overset{+1}{H})_{2}$
			\item $\overset{+2}{Cu}\overset{-2}{O} + \overset{-2}{H}_{2} \xrightarrow{t^{\circ}} \overset{0}{Cu} + \overset{+1}{H}_2\overset{-2}{O}$
			\item $2K\overset{+7}{Mn}\overset{-2}{O}_4 \xrightarrow{t^{\circ}} K_2\overset{+6}{Mn}O_4 + \overset{0}{O}_2 + \overset{+6}{Mn}O_{2}$
			\item $\overset{0}{Cl}_2 + 2KOH \longrightarrow K\overset{-1}{Cl} + K\overset{+1}{Cl}O + H_2O$
			\item $\overset{+8/3}{Fe_3}\overset{-2}{O_4} + 8\overset{+1}{H}\overset{-1}{Cl} \longrightarrow 2\overset{+3}{Fe}Cl_3 + \overset{+2}{Fe}\overset{-1}{Cl}_2 + 4\overset{+1}{H}_2O$
		\end{enumerate}
		\indam{Nhận xét:} Phản ứng a), c), d), e) có sự thay đổi số oxi hóa của các nguyên tố. $\Rightarrow$ Phản ứng a), c), d), e) là phản ứng oxi hóa - khử 
	}
\end{vd}
%%%=========vd_5=========%%%
\begin{vd}[Cân bằng phản ứng oxi hóa khử][][]Lập phương trình phản ứng oxi hóa - khử sau đây:
	\puhh{ $MnO_2$ \+ $HCl$}{->}{$MnCl_2$ \+ $ Cl_2$ \+ $H_2O$}
	\loigiai{\begin{cacbuoc}
			\item Xác định số oxi hóa của các nguyên tố :\\
			$\overset{+4}{Mn}O_2 + H\overset{-1}{Cl} \longrightarrow \overset{+2}{Mn}Cl_2 + \overset{0}{Cl_2} + H_2O$ 
			\item Quá trình cho - nhận electron\\
			$\begin{matrix}
				2\overset{-1}{Cl} & \longrightarrow& Cl_2 & + & 2e\\
				\overset{+4}{Mn} &  + & 2e & \longrightarrow & \overset{+2}{Mn}
			\end{matrix}$
			\item Đặt hệ số \\
			$\begin{matrix}
				1x|~2\overset{-1}{Cl} & \longrightarrow& Cl_2 & + & 2e\\
				1x|~\overset{+4}{Mn} &  + & 2e & \longrightarrow & \overset{+2}{Mn}
			\end{matrix}$
			\item Phương trình hóa học
			\boxct{$MnO_2 + 4HCl \longrightarrow MnCl_2 + Cl_2 + 2H_2O$}
	\end{cacbuoc}}
\end{vd}
%%%=========vd_6=========%%%
%%%==========================%%%
%%%==========================%%%
\begin{vd}[Cân bằng phản ứng oxi hóa - khử][][Nguồn:CĐA 2010] Cho phản ứng:
	$\mathrm{Na}_2 SO_3+\mathrm{KMnO}_4+\mathrm{NaHSO}_4 \to \mathrm{Na}_2 SO_4+\mathrm{MnSO}_4+K_2 SO_4+H_2 O$.Tổng hệ số của các chất (là những số nguyên, tối giản) trong phương trình phản ứng là
	\choice
	{23}
	{27}
	{47}
	{31}
	\loigiai{
		\begin{cacbuoc}
			\item Xác định số oxi hóa của các nguyên tố :\\
			$Na_2\overset{+4}{S}O_3$ + $K\overset{+7}{Mn}O_4$ + $NaHSO_4$ $\to$ $Na_2\overset{+6}{S}O_4$ + $\overset{+2}{Mn}SO_4$ + $K_2SO_4$ + $H_2 O$
			\item Quá trình cho - nhận electron\\
			$\begin{matrix}
				\overset{+7}{Mn} &  + & 5e & \longrightarrow & \overset{+2}{Mn}\\
				\overset{+4}{S} & \longrightarrow& \overset{+6}{Mn} & + & 2e
			\end{matrix}$
			\item Đặt hệ số \\
			$\begin{array}{lllll}
				2x|~\overset{+7}{Mn} &  + & 5e & \longrightarrow & \overset{+2}{Mn}\\
				5x|~\overset{+4}{S} & \longrightarrow& \overset{+6}{Mn} & + & 2e
			\end{array}$\\
			\item Phương trình hóa học\\
			Hệ số 5 điền cho S ($5Na_2SO_3$ và $?Na_2SO_4$). Hệ số 2 điền cho Mn ($2KMnO4$ và $2MnO4$).
			
			\GSND[][\faComment]{Phân tích:}Hệ số của $\mathrm{Na}_2 \mathrm{SO}_3, \mathrm{KMnO}_4, \mathrm{MnSO}_4$ và $\mathrm{K}_2 \mathrm{SO}_4$ đã được xác định.\\
			Do gốc $\mathrm{SO}_4^{2-}$ là sản phẩm oxi hóa của $\mathrm{SO}_3^{2-}$ đồng thời cũng là sản phẩm của chất làm nhiệm vụ môi trường $\mathrm{NaHSO}_4$. Do vậy cần tìm hệ số của $\mathrm{Na}_2 \mathrm{SO}_4, \mathrm{NaHSO}_4$ và $\mathrm{H}_2 \mathrm{O}$ bằng phương pháp bảo toàn nguyên tố.
			
			\noindent Đặt $x$ là hệ số của $\mathrm{Na}_2 \mathrm{SO}_4$ và $ y$ là hệ số của $\mathrm{H_2O}$ .\\
			- Bảo toàn số nguyên tử  $H\Rightarrow$ hệ số của $NaHSO_4$ là $2y$\\
			$5\mathrm{Na}_2 SO_3+2\mathrm{KMnO}_4+2y\mathrm{NaHSO}_4 \to x\mathrm{Na}_2 SO_4+2\mathrm{MnSO}_4+K_2 SO_4+yH_2 O$\\
			- Bảo toàn số nguyên tử $S \Rightarrow(5+2 y)=(x+2+1) \Rightarrow(x-2 y)=2$ (*)\\
			- Bảo toàn số nguyên tử $O$
			$\Rightarrow(15+8+8 y)=(4 x+12+y) \Rightarrow(4 x-7 y)=11$(**)\\
			Từ (*) và (**) $\Rightarrow x=8 ; y=3$
			\boxct[\mauphu]{$5\mathrm{Na}_2 SO_3+2\mathrm{KMnO}_4+6\mathrm{NaHSO}_4 \to 8\mathrm{Na}_2 SO_4+2\mathrm{MnSO}_4+K_2 SO_4+3H_2 O$}
		\end{cacbuoc}
	}
\end{vd}
%%%=========Vd7=============%%%
\begin{vd}[Cân bằng phản ứng oxi hóa - khử]
	Cân bằng các phản ứng oxi hóa khử sau bằng phương pháp \indam[black]{thăng bằng electron}:
	\begin{enumerate}
		\item $\mathrm{KMnO}_4+\mathrm{HCl} \to \mathrm{MnCl}_2+\mathrm{Cl}_2+\mathrm{KCl}+H_2O$
		\item $K_2\mathrm{Cr}_2O_7+\mathrm{HCl} \to \mathrm{CrCl}_3+\mathrm{Cl}_2+\mathrm{KCl}+H_2O$
	\end{enumerate}
	\loigiai{\begin{enumerate}[(1)]
			\item $\mathrm{KMnO}_4+\mathrm{HCl} \to \mathrm{MnCl}_2+\mathrm{Cl}_2+\mathrm{KCl}+H_2O$
			\begin{cacbuoc}
				\item Xác định số oxi hóa của các nguyên tố :\\
				$K\overset{+7}{Mn}O_4 + H\overset{-1}{Cl} \longrightarrow \overset{+2}{Mn}Cl_2 + \overset{0}{Cl_2} + H_2O$ 
				\item Quá trình cho - nhận electron\\
				$\begin{matrix}
					2\overset{-1}{Cl} & \longrightarrow& Cl_2 & + & 2e\\
					\overset{+7}{Mn} &  + & 5e & \longrightarrow & \overset{+2}{Mn}
				\end{matrix}$
				\item Đặt hệ số \\
				$\begin{array}{lccccc}
					5x\biggl|&2\overset{-1}{Cl} & \longrightarrow& Cl_2 & + & 2e\\
					2x\biggl|&\overset{+7}{Mn} &  + & 5e & \longrightarrow & \overset{+2}{Mn}
				\end{array}$
				\item Phương trình hóa học\\
				Đặt hệ số 2 cho Mn ($2MnCl_2$ và $2KMnO_4$).Đặt hệ số 5 cho Cl ($5Cl_2$ và $?HCl$)\\
				- Bảo toàn K: $\Rightarrow 2KCl $\\
				$\Rightarrow 2\mathrm{KMnO}_4+?\mathrm{HCl} \to 2\mathrm{MnCl}_2+ 5\mathrm{Cl}_2+2\mathrm{KCl}+H_2O$
				\GSND[][\faComment]{Phân tích:}HCl tham gia vào quá trình oxi hóa (tạo ra $5Cl_2$)  và làm môi trường tạo muối ($2MnCl_2$ và $2KCl$ ).Như vậy bảo toàn Cl $\Rightarrow 16HCl $ và bảo toàn H $\Rightarrow 8H_2O $
				\boxct{$2\mathrm{KMnO}_4+16\mathrm{HCl} \to 2\mathrm{MnCl}_2+5\mathrm{Cl}_2+2\mathrm{KCl}+8H_2O$}
			\end{cacbuoc}
			\item $K_2\mathrm{Cr}_2O_7+\mathrm{HCl} \to \mathrm{CrCl}_3+\mathrm{Cl}_2+\mathrm{KCl}+H_2O$
			\begin{cacbuoc}
				\item Xác định số oxi hóa của các nguyên tố :\\
				$K_2\mathrm{\overset{+6}{Cr}}_2O_7+\mathrm{H\overset{-1}{Cl}} \to \mathrm{\overset{+3}{Cr}Cl}_3+\overset{0}{Cl}_2+\mathrm{KCl}+H_2O$ 
				\item Quá trình cho - nhận electron\\
				$\begin{matrix}
					2\overset{+6}{Cr} &  + & 6e & \longrightarrow & 2\overset{+3}{Cr}\\
					2\overset{-1}{Cl} & \longrightarrow& Cl_2 & + & 2e\\
				\end{matrix}$
				\item Đặt hệ số \\
				$\begin{array}{lccccc}
					1x\biggl|&2\overset{+6}{Cr} &  + & 6e & \longrightarrow & 2\overset{+3}{Cr}\\
					3x\biggl|&2\overset{-1}{Cl} & \longrightarrow& Cl_2 & + & 2e\\
				\end{array}$
				\item Phương trình hóa học\\
				Đặt hệ số  cho Cr ( $K_2Cr_2O_7$ và $2CrCl_3$).Đặt hệ số 3 cho Cl ( $3Cl_2$ và $?HCl$).\\
				- Bảo toàn K: $\Rightarrow 2KCl $\\
				$K_2\mathrm{Cr}_2O_7+?\mathrm{HCl} \to 2\mathrm{CrCl}_3+3\mathrm{Cl}_2+2\mathrm{KCl}+H_2O$
				\GSND[][\faComment]{Phân tích:}HCl tham gia vào quá trình oxi hóa (tạo ra $3Cl_2$)  và làm môi trường tạo muối ($2CrCl_3$ và $2KCl$ ).Như vậy bảo toàn Cl $\Rightarrow 14HCl $ và bảo toàn H $\Rightarrow 7H_2O $
				\boxct{$K_2\mathrm{Cr}_2O_7+14\mathrm{HCl} \to 2\mathrm{CrCl}_3+3\mathrm{Cl}_2+2\mathrm{KCl}+7H_2O$}
			\end{cacbuoc}
		\end{enumerate}
	}
\end{vd}
%%%=========Vd8=============%%%
\begin{vd}[Cân bằng phản ứng oxi hóa - khử]
	Cân bằng các phản ứng oxi hóa khử sau bằng phương pháp thăng bằng electron: 
	\begin{enumerate}[(a)]
		\item $\mathrm{FeS}_2+O_2\to \mathrm{Fe}_2O_3+SO_2$
		\item $P+NH_4\mathrm{HClO}_4\to H_3PO_4+\mathrm{Cl}_2+N_2+H_2O$
	\end{enumerate}
	\loigiai{
		\begin{enumerate}[(a)]
			\item $\mathrm{FeS}_2+O_2\to \mathrm{Fe}_2O_3+SO_2$
			\begin{cacbuoc}
				\item Xác định số oxi hóa của các nguyên tố :\\
				$\mathrm{\overset{+2}{Fe}\overset{-1}{S}}_2+\overset{0}{O}_2\to \mathrm{\overset{+3}{Fe}}_2\overset{-2}{O}_3+\overset{+4}{S}\overset{-2}{O}_2$
				\item Quá trình cho - nhận electron\\
				$\begin{array}{ccccccc}			
					\left(\mathrm{FeS}_2\right)^0 & \rightarrow & \mathrm{Fe}^{+3} & + & 2 \mathrm{~S}^{+4}& + &11\mathrm{e} \\
					\mathrm{O}_2 & + & 4\mathrm{e} & \rightarrow & 2\mathrm{O}^{-2}& & \\
				\end{array}$  
				\item Đặt hệ số \\
				$\begin{array}{rccccccc}
					4x\biggl|&\left(\mathrm{FeS}_2\right)^0 & \rightarrow & \mathrm{Fe}^{+3} & + & 2 \mathrm{~S}^{+4}& + &11\mathrm{e} \\
					11x\biggl|& \mathrm{O}_2 & + & 4\mathrm{e} & \rightarrow & 2\mathrm{O}^{-2}& & \\
				\end{array}$
				\item Phương trình hóa học\\
				Đặt hệ số 4 cho Fe ( $4FeS_2$ và $2Fe_2O_3$).Đặt hệ số 11 cho O ($11O_2$).\\
				- Bảo toàn S: $\Rightarrow 8SO_2 $.Kiểm tra O hai vế
				\boxct{$4\mathrm{FeS}_2+11O_2\xrightarrow{\makebox[1.0cm]{$t^{\circ}$}} 2\mathrm{Fe}_2O_3+8SO_2$}
			\end{cacbuoc}
			\item $P+NH_4\mathrm{ClO}_4\to H_3PO_4+\mathrm{Cl}_2+N_2+H_2O$
			\begin{cacbuoc}
				\item Xác định số oxi hóa của các nguyên tố :\\
				$\overset{0}{P}+\overset{-3}{N}H_4\mathrm{\overset{+7}{Cl}O}_4\to H_3\overset{+5}{P}O_4+\mathrm{\overset{0}{Cl}}_2+\overset{0}{N}_2+H_2O$
				\item Quá trình cho - nhận electron\\
				$\begin{array}{ccccccccc}			
					\overset{0}{P}&\rightarrow &\overset{+5}{P}& + & 5e&&&&\\
					2\overset{-3}{N}&+ &2\overset{+7}{Cl}&+ & 8e &\rightarrow &\overset{0}{Cl}_2& + & \overset{0}{N}_2\\
				\end{array}$  
				\item Đặt hệ số \\
				$\begin{array}{rccccccccc}			
					8x\bigg|&\overset{0}{P}&\rightarrow &\overset{+5}{P}& + & 5e&&&&\\
					5x\bigg|&2\overset{-3}{N}&+ &2\overset{+7}{Cl}&+ & 8e &\rightarrow &\overset{0}{Cl}_2& + & \overset{0}{N}_2\\
				\end{array}$  
				\item Phương trình hóa học\\
				Đặt hệ số 8 cho P ( $8P$ và $8H_3PO_4$).Đặt hệ số 5 cho Cl và N ($10NH4ClO_4$, $5Cl_2$ và $5N_2$).\\
				- Bảo toàn H: $\Rightarrow 8H_2O $.Kiểm tra O hai vế
				\boxct{$8P+10NH_4\mathrm{ClO}_4\to 8H_3PO_4+5\mathrm{Cl}_2+5\mathrm{N}_2+8H_2O$}
			\end{cacbuoc}
		\end{enumerate}
	}
\end{vd}
%%%===================Vd9========================%%%
\begin{vd}[Cân bằng phản ứng oxi hóa - khử]Cân bằng phản ứng hóa học sau theo phương pháp thăng bằng electron.
	$\mathrm{FeO}+HNO_3 \to \mathrm{Fe}\left(NO_3\right)_3+ NO_2 + NO + H_2 O \quad\left(n_{NO_2}: n_{NO}=a: b\right)$
	\loigiai{
		\begin{cacbuoc}
			\item Xác định số oxi hóa của các nguyên tố :\\
			$\mathrm{\overset{+2}{Fe}O}+H\overset{+5}{N}O_3 \to \mathrm{\overset{+3}{Fe}}\left(NO_3\right)_3+ \overset{+4}{N}O_2 + \overset{+2}{N}O + H_2 O$
			\item Quá trình cho - nhận electron, đặt chéo hệ số\\
			\begin{tabular}{r|cccccccc}
				&$\overset{+5}{N}$&+&1e &$\xrightarrow{\makebox[1cm]{}}$ & $\overset{+4}{N}$&$\big|x\;a$&&\\
				&$\overset{+5}{N}$&+&3e &$\xrightarrow{\makebox[1cm]{}}$ & $\overset{+2}{N}$&$\big|x\; b$&&\\
				\cline{2-8}
				&$(a+b)\overset{+5}{N}$&+&$(a+3b)e$& $\xrightarrow{\makebox[1cm]{}}$&$aNO_2$&+&$bNO$&\\
				&$\overset{+2}{Fe}$&$\xrightarrow{\makebox[1cm]{}}$&$\overset{+3}{Fe}$ & + & 1e& & &\\
				\hline
				\indam{1X}&$(a+b)\overset{+5}{N}$&+&$(a+3b)e$& $\xrightarrow{\makebox[1cm]{}}$&$aNO_2$&+&$bNO$&\\
				\indam{(a+3b)X}&$\overset{+2}{Fe}$ & $\xrightarrow{\makebox[1cm]{}}$ & $\overset{+3}{Fe}$& + & $1\mathrm{e}$&&&\\
			\end{tabular}
			\item Phương trình hóa học\\
			Đặt hệ số 1 cho N( $aNO_2$ và $bNO$). Đặt hệ số (a+3b) cho Fe $\big((a+3b)FeO \;\text{và} \;(a+3b)Fe{(NO_3)}_{3} \big)$\\
			- Bảo toàn N: $\Rightarrow (4a+10b)HNO_3 $. Bảo toàn H $\Rightarrow (2a+5b)H_2O $. Kiểm tra O hai vế
			\boxct{$(a+3b)\mathrm{FeO}+(4a+10b)HNO_3 \xrightarrow{\makebox[.65cm]{}} (a+3b)\mathrm{Fe}\left(NO_3\right)_3+ aNO_2 + bNO +(2a+5b) H_2 O$}
		\end{cacbuoc}
	}
\end{vd}
%%%===================Vd10========================%%%
\begin{vd}[Cân bằng phản ứng oxi hóa - khử]
	Cân bằng phản ứng oxi khử sau đây bằng phương pháp thăng bằng electron: \\ 
	$\mathrm{Fe}_x O_y+HNO_3 \to \mathrm{Fe}\left(NO_3\right)_3+NO+H_2O$
	\loigiai{
		\begin{cacbuoc}
			\item Xác định số oxi hóa của các nguyên tố :\\
			$\mathrm{\overset{+2y/x}{Fe_x}} O_y+H\overset{+5}{N}O_3 \to \mathrm{\overset{+3}{Fe}}\left(NO_3\right)_3+\overset{+2}{N}O+H_2O$
			\item Quá trình cho - nhận electron, đặt chéo hệ số\\
			\begin{tabular}{r|ccccc}
				3X&$x\overset{+2y/x}{Fe}$&$\xrightarrow{\makebox[1cm]{}}$&$x\overset{+3}{Fe}$&+ &$(3x-2y)e$\\
				$(3x-2y)X$&$\overset{+5}{N}$&+&$3e$&$\xrightarrow{\makebox[1cm]{}}$&$\overset{+2}{N}$\\
			\end{tabular}
			\item Phương trình hóa học\\
			Đặt hệ số 3 cho Fe ($3xFe{(NO_3)}_3$ và $3Fe_xO_y$). Đặt hệ số (3x-2y) cho N $\big((3x-2y)NO \;\text{và} \;?HNO_3\big)$\\
			- Bảo toàn N: $\Rightarrow (12x-2y)HNO_3 $. Bảo toàn H $\Rightarrow (6x-y)H_2O $. Kiểm tra O hai vế
			\boxct{$3\mathrm{Fe}_x O_y+(12x-2y)HNO_3 \xrightarrow{\makebox[1cm]{}} 3x\mathrm{Fe}\left(NO_3\right)_3+(3x-2y)NO+(6x-y)H_2O$}
		\end{cacbuoc}
	}
\end{vd}
%%%%=====================Bài tập tự luyện Dạng 1==========================%%%
\newpage
\begin{bttl}
\end{bttl}
\Opensolutionfile{ans}[Ans/DATNC4]
\luuloigiaiex
\Opensolutionfile{ansex}[LOIGIAITN/LGTNCHUONG4]
\Writetofile{ansex}{\protect\nhanmanh{Lời giải chi tiết phần trắc nghiệm}}
%%%============Phần trắc nghiệm============%%%
%%%%===================EX_1========================%%%
%\begin{ex}[][	][Nhận biết phản ứng oxi hóa khử]
%	Phản ứng nào sau đây là phản ứng Oxi hóa khử
%	\choice{\True $2HgO \xrightarrow{t^{\circ}} 2Hg + O_2$}        
%	{$2Fe(OH)_3 \xrightarrow{t^{\circ}} Fe_2O_3 + 3H_2O$}   
%	{$CaCO_3 \xrightarrow{t^{\circ}} CaO + CO_2$}   
%	{$2NaHCO_3 \xrightarrow{t^{\circ}} Na_2CO_3 + CO_2 + H_2O$}   
%	\loigiai{
%		\begin{enumerate}[(1)]
%			\item $2\overset{+2}{Hg}\overset{-2}{O} \xrightarrow{t^{\circ}} 2\overset{0}{Hg} + \overset{0}{O_2}$
%			\item $\overset{+2}{Ca}\overset{+4}{C}\overset{-2}{O}_3 \xrightarrow{t^{\circ}} \overset{+2}{Ca}\overset{-2}{O} + \overset{+4}{C}\overset{-2}{O}_2$
%			\item $2\overset{+3}{Fe}(\overset{-2}{O}\overset{+1}{H})_3 \xrightarrow{t^{\circ}} \overset{+3}{Fe}_2\overset{-2}{O}_3 + 3\overset{+1}{H}_2\overset{-2}{O}$
%			\item $2\overset{+1}{Na}\overset{+1}{H}\overset{+4}{C}\overset{-2}{O}_3 \xrightarrow{t^{\circ}} \overset{+1}{Na}_2\overset{+4}{C}\overset{-2}{O}_3 + \overset{+4}{C}\overset{-2}{O}_2 + \overset{+1}{H}_2\overset{-2}{O}$
%		\end{enumerate}
%		\indam{Nhận xét:} Phản ứng (2),(3),(4) không có sự thay đổi số oxi hóa. Phản ứng (1) có sự thay đổi số oxi hóa. 
%	}
%\end{ex}
%%%%=============EX_2=============%%%
%\begin{ex}[][][Phân biệt chất oxi hóa, chất khử]
%	$SO_2$  đóng vai trò là chất oxi hóa trong phản ứng nào dưới đây.
%	\choice{\puhh[$t^{\circ}$]{$2SO_2$ \+ $O_2$}{->}{$2SO_3$}}        
%	{\puhh{$SO_2$ \+ $Br_2$ \+ $2H_2O$}{->}{$2HBr$\+ $H_2SO_4$}}   
%	{\True \puhh{$4SO_2$ \+ $2H_2S$}{->}{$3S\uparrow$ \+ $2H_2O$}}   
%	{\puhh{$5SO_2$ \+ $2KMnO_4$\+$2H_2O$ }{->}{$K_2SO_4$ \+ $2MnSO_4$ \+ $2H_2SO_4$}}   
%	\loigiai{
%		Chất oxi hóa là chất có số oxi hóa giảm sau phản ứng.Chất khử là chất có số oxi hóa tăng sau phản ứng.
%		\begin{enumerate}[(1)]
%			\item \puhh[$V_2O_5$][$450^{\circ}C$][1][-4pt]{$2\overset{+4}{S}O_2$ \+ $\overset{0}{O}_2$}{<=>}{$2\overset{+6}{S}\overset{-2}{O_3}$}
%			\item \puhh[][][][-4pt]{$\overset{+4}{S}O_2$ \+ $\overset{0}{Br}_2$ \+ $2H_2O$}{->}{$2H\overset{-1}{Br}$\+ $H_2\overset{+6}{S}O_4$}
%			\item \puhh[][][][-4pt]{$4\overset{+4}{S}O_2$ \+ $2H_2\overset{-2}{S}$}{->}{$3\overset{0}{S}\uparrow$ \+ $2H_2O$}
%			\item \puhh[][][][-4pt]{$5\overset{+4}{S}O_2$ \+ $2K\overset{+7}{Mn}O_4$\+$2H_2O$ }{->}{$K_2\overset{+6}{S}O_4$ \+ $2\overset{+2}{Mn}SO_4$ \+ $2H_2SO_4$}
%		\end{enumerate}
%		\indam{Nhận xét:} Phản ứng (1),(2),(4) S tăng số oxi hóa nên là chất khử. Phản ứng (3) S giảm số oxi hóa từ +4 xuống 0. 
%	}
%\end{ex}
%%%%============EX_3==============%%%
%\begin{ex}
%	$NH_3$ không đóng vai trò là chất khử trong phản ứng
%	\choice
%	{$4NH_3+5O_2\xrightarrow{\mathrm{xt},t^{\circ}} 4NO+6H_2O$}
%	{$2NH_3+3\mathrm{CuO} \xrightarrow{t^{\circ}} 3\mathrm{Cu}+N_2+3H_2O$}
%	{$2NH_3+\mathrm{Cl}_2\to N_2+6\mathrm{HCl}$}
%	{$2NH_3+H_2O_2+\mathrm{MnSO}_4\to \mathrm{MnO}_2+\left(NH_4\right)_2SO_4$}
%	\loigiai{}
%\end{ex}
%%%%============EX_4==============%%%
%\begin{ex}
%	Cho phản ứng hoá học: $\mathrm{Br}_2+5\mathrm{Cl}_2+6H_2O\to 2\mathrm{HBrO}_3+10\mathrm{HCl}$. Phát biểu nào sau đây đúng?
%	\choice
%	{$\mathrm{Br}_2$ là chất oxi hoá, $\mathrm{Cl}_2$ là chất khử}
%	{$\mathrm{Br}_2$ là chất oxi hoá, $H_2O$ là chất khử}
%	{$\mathrm{Br}_2$ là chất khử, $\mathrm{Cl}_2$ là chất oxi hoá}
%	{$\mathrm{Cl}_2$ là chất oxi hoá, $H_2O$ là chất khử}
%	\loigiai{}
%\end{ex}
%%%%============EX_5==============%%%
%\begin{ex}
%	Phản ứng nào sau đây là phản ứng oxi hóa-khử?
%	\choice
%	{$HNO_3+\mathrm{NaOH} \to \mathrm{NaNO}_3+H_2O$}
%	{$N_2 O_5+H_2O \to 2 HNO_3$}
%	{$2 HNO_3+3 H_2 \mathrm{~S} \to 3 \mathrm{~S}+2 NO+4 H_2O$}
%	{$2 \mathrm{Fe}(OH)_3 \xrightarrow{t^{\circ}} \mathrm{Fe}_2 O_3+3 H_2O$}
%	\loigiai{}
%\end{ex}
%%%%============EX_6==============%%%
%\begin{ex}
%	Trong phản ứng: $3 NO_2+H_2O \to 2 HNO_3+NO$. $NO_2$ đóng vai trò
%	\choice
%	{là chất oxi hóa}
%	{là chất oxi hóa, nhưng đồng thời là chất khử}
%	{là chất khử}
%	{không là chất oxi hóa, cũng không là chất khử}
%	\loigiai{}
%\end{ex}
%%%%============EX_7==============%%%
%\begin{ex}
%	Cho phản ứng: $\mathrm{Zn}+\mathrm{CuCl}_2\to \mathrm{ZnCl}_2+\mathrm{Cu}$. Trong phản ứng này, $1\mathrm{~mol} \mathrm{Cu} \mathrm{Cu}^{2+}$ đã
%	\choice
%	{nhận 1 mol electron}
%	{nhận 2 mol electron}
%	{nhường 1 mol electron}
%	{nhường 2 mol electron}
%	\loigiai{}
%\end{ex}
%%%%============EX_8==============%%%
%\begin{ex}
%	Trong phản ứng: $\mathrm{Cl}_2+2\mathrm{KBr} \to \mathrm{Br}_2+2\mathrm{KCl}$. Nguyên tố clo
%	\choice
%	{chỉ bị oxi hoá}
%	{chỉ bị khử}
%	{không bị oxi hoá, cũng không bị khử}
%	{vừa bị oxi hoá, vừa bị khử}
%	\loigiai{}
%\end{ex}
%%%%============EX_9==============%%%
%\begin{ex}
%	Trong phản ứng: $2 \mathrm{Fe}(OH)_3 \to \mathrm{Fe}_2 O_3+3 H_2O$. Nguyên tố sắt
%	\choice
%	{bị oxi hoá}
%	{bị khử}
%	{không bị oxi hoá, cũng không bị khử}
%	{vừa bị oxi hoá, vừa bị khử}
%	\loigiai{}
%\end{ex}
%%%%============EX_10==============%%%
%\begin{ex}
%	Cho phương trình hóa học sau: $3 \mathrm{Cl}_2+6 KOH \to \mathrm{KClO}_3+5 \mathrm{KCl}+3 H_2O \cdot \mathrm{Cl}_2$ đóng vai trò
%	\choice
%	{chỉ là chất oxi hoá}
%	{không phải chất oxi hoá, không phải chất khử}
%	{chỉ là chất khử}
%	{vừa là chất oxi hoá, vừa là chất khử}
%	\loigiai{}
%\end{ex}
%%%%============EX_11==============%%%
%\begin{ex}
%	Cho phản ứng: $3\mathrm{K}_2 \mathrm{MnO}_4+2 H_2O \to 2 \mathrm{KMnO}_4+\mathrm{MnO}_2+4 KOH$. Nguyên tố mangan trong $K_2\mathrm{MnO}_4$ có số oxi hóa
%	\choice
%	{tăng}
%	{giảm}
%	{vừa tăng, vừa giảm}
%	{không thay đổi}
%	\loigiai{}
%\end{ex}
%%%%============EX_12==============%%%
%\begin{ex}
%	Trong các phản ứng dưới đây, phản ứng nào là phản ứng oxi hoá-khử?
%	\choice
%	{$\mathrm{CaCO}_3+H_2O+CO_2 \to \mathrm{Ca}\left(HCO_3\right)_2$}
%	{$P_2 O_5+3 H_2O \to 2 H_3 PO_4$}
%	{$2SO_2+O_2\to 2SO_3$}
%	{$\mathrm{BaO}+H_2O \to \mathrm{Ba}(OH)_2$}
%	\loigiai{}
%\end{ex}
%%%%============EX_13==============%%%
%\begin{ex}
%	Phản ứng phân hủy nào dưới đây không phải phản ứng oxi hoá-khử?
%	\choice
%	{$2 \mathrm{KMnO}_4 \to K_2\mathrm{MnO}_4+\mathrm{MnO}_2+O_2$}
%	{$2 \mathrm{Fe}(OH)_3 \to \mathrm{Fe}_2 O_3+3 H_2O$}
%	{$4\mathrm{KClO}_3\to 3\mathrm{KClO}_4+\mathrm{KCl}$}
%	{$2\mathrm{KClO}_3\to 2\mathrm{KCl}+3O_2$}
%	\loigiai{}
%\end{ex}
%%%%============EX_14==============%%%
%\begin{ex}
%	Cho phản ứng hoá học: $\mathrm{Cr}+O_2\xrightarrow{t^{\circ}} \mathrm{Cr}_2O_3$. Trong phản ứng trên xảy ra
%	\choice
%	{sự oxi hoá $\mathrm{Cr}$ và sự khử $O_2$}
%	{sự khử Cr và sự oxi hoá $O_2$}
%	{sự oxi hoá $\mathrm{Cr}$ và sự oxi hoá $O_2$}
%	{sự khử $\mathrm{Cr}$ và sự khử $O_2$}
%	\loigiai{}
%\end{ex}
%%%%============EX_15==============%%%
%\begin{ex}
%	Lưu huỳnh đóng vai trò là chất oxi hoá trong phản ứng
%	\choice
%	{$S+O_2\xrightarrow{t^{\circ}} SO_2$}
%	{$S+2\mathrm{Na} \xrightarrow{t^{\circ}} \mathrm{Na}_2\mathrm{~S}$}
%	{$S+2 H_2 SO_{4(\text {đặ})} \xrightarrow{t^{\circ}} 3 SO_2+2 H_2O$}
%	{$S+6 HNO_{3(\text {đặc})} \xrightarrow{t^{\circ}} H_2 SO_4+6 NO_2+2 H_2O$}
%	\loigiai{}
%\end{ex}
%%%%============EX_16==============%%%
%\begin{ex}
%	Cho phương trình phản ứng sau: $\mathrm{Zn}+HNO_3 \to \mathrm{Zn}\left(NO_3\right)_2+NO+H_2O$. Nếu hệ số của $HNO_3$ là 8 thì tổng hệ số của $\mathrm{Zn}$ và $NO$ là
%	\choice
%	{$4$}
%	{$3$}
%	{$6$}
%	{$5$}
%	\loigiai{}
%\end{ex}
%%%%============EX_17==============%%%
%\begin{ex}
%	Cho phản ứng: $\mathrm{aFe}+\mathrm{bHNO}_3\to \mathrm{cFe}\left(NO_3\right)_3+\mathrm{dNO}+\mathrm{eH}_2O$. Các hệ số $a, b, c, d$, e là những số nguyên, đơn giản nhât. Tổng $(a+b)$ bằng
%	\choice
%	{$4$}
%	{$3$}
%	{$6$}
%	{$5$}
%	\loigiai{}
%\end{ex}
\nhanmanh{Phần Trắc nghiệm 30 câu}
%%%==============Cau_1==============%%%
\begin{ex}Số oxi hóa là một số đại số đặc trưng cho đại lượng nào sau đây của nguyên tử trong phân tử?
	\choice
	{Hóa trị}
	{\True Điện tích}
	{Khối lượng}
	{Số hiệu nguyên tử}
	\loigiai{}
\end{ex}
%%%==============HetCau_1==============%%%

%%%==============Cau_2==============%%%
\begin{ex}[Đề THPT QG-2018]
	Nguyên tố chromium $(\mathrm{Cr})$ có số oxi hóa+6 trong hợp chất nào sau đây?
	\choice
	{$\mathrm{Cr}(OH)_3$}
	{\True $\mathrm{Na}_2\mathrm{CrO}_4$}
	{$\mathrm{Cr}_2O_3$}
	{$\mathrm{NaCrO}_2$}
	\loigiai{}
\end{ex}
%%%==============HetCau_2==============%%%

%%%==============Cau_3==============%%%
\begin{ex}Chromium(VI) oxide là chất rắn, màu đỏ thẫm, vừa là acidic oxide, vừa là chất oxi hóa mạnh. Số oxi hóa của chromium $(\mathrm{Cr})$ trong oxide trên là
	\choice
	{$0$}
	{\True $+6$}
	{$+2$}
	{$+3$}
	\loigiai{}
\end{ex}
%%%==============HetCau_3==============%%%

%%%==============Cau_4==============%%%
\begin{ex}Cho các hợp chất sau: $NH_3, NH_4\mathrm{Cl}, HNO_3, NO_2$. Số hợp chất chứa nguyên tử nitrogen có số oxi hóa-3 là
	\choice
	{$1$}
	{$3$}
	{\True $2$}
	{$4$}
	\loigiai{}
\end{ex}
%%%==============HetCau_4==============%%%

%%%==============Cau_5==============%%%
\begin{ex}Cho các phân tử sau: $H_2\mathrm{~S}, SO_3, \mathrm{CaSO}_4, \mathrm{Na}_2\mathrm{~S}, H_2SO_4$. Số oxi hóa của nguyên tử $S$ trong các phân tử trên lần lượt là
	\choice
	{$0,+6,+4,+4,+6$}
	{$0,+6,+4,+2,+6$}
	{$+2,+6,+6,-2,+6$}
	{\True $-2,+6,+6,-2,+6$}
	\loigiai{}
\end{ex}
%%%==============HetCau_5==============%%%

%%%==============Cau_6==============%%%
\begin{ex}$\mathrm{Fe}_2O_3$ là thành phần chính của quặng hematite đỏ, dùng để luyện gang. Số oxi hóa của iron ($\left.\mathrm{Fe}\right)$ trong $\mathrm{Fe}_2O_3$ là
	\choice
	{\True $+3$}
	{$+6$}
	{$-3$}
	{$-6$}
	\loigiai{}
\end{ex}
%%%==============HetCau_6==============%%%

%%%==============Cau_7==============%%%
\begin{ex}Ammonia $\left(NH_3\right)$ là nguyên liệu để sản xuất nitric acid và nhiều loại phân bón. Số oxi hóa của nitrogen $(N)$ trong ammonia là
	\choice
	{$+3$}
	{\True $-3$}
	{$+1$}
	{$-1$}
	\loigiai{}
\end{ex}
%%%==============HetCau_7==============%%%

%%%==============Cau_8==============%%%
\begin{ex}Cho các chất sau: $\mathrm{Cl}_2, \mathrm{HCl}, \mathrm{NaCl}, \mathrm{KClO}_3, \mathrm{HClO}_4$. Số oxi hóa của nguyên tử $\mathrm{Cl}$ trong phân tử các chất trên lần lượt là
	\choice
	{$0,+1,+1,+5,+7$}
	{\True $0,-1,-1,+5,+7$}
	{$1,-1,-1,-5,-7$}
	{$0,1,1,5,7$}
	\loigiai{}
\end{ex}
%%%==============HetCau_8==============%%%

%%%==============Cau_9==============%%%
\begin{ex}Iron có số oxi hóa+2 trong hợp chất nào sau đây?
	\choice
	{$\mathrm{Fe}(OH)_3$}
	{$\mathrm{FeCl}_3$}
	{\True $\mathrm{FeSO}_4$}
	{$\mathrm{Fe}_2O_3$}
	\loigiai{}
\end{ex}
%%%==============HetCau_9==============%%%

%%%==============Cau_10==============%%%
\begin{ex}Dấu hiệu để nhận ra phản ứng là phản ứng oxi hóa-khử dựa trên sự thay đổi đại lượng nào sau đây của nguyên tử?
	\choice
	{Số mol}
	{\True Số oxi hóa}
	{Số khối}
	{Số proton}
	\loigiai{}
\end{ex}
%%%==============HetCau_10==============%%%

%%%==============Cau_11==============%%%
\begin{ex}Phản ứng oxi hóa-khử là phản ứng có sự nhường và nhận
	\choice
	{\True electron}
	{neutron}
	{proton}
	{cation}
	\loigiai{}
\end{ex}
%%%==============HetCau_11==============%%%

%%%==============Cau_12==============%%%
\begin{ex}Trong phản ứng oxi hóa-khử, chất oxi hóa là chất
	\choice
	{nhường electron}
	{\True nhận electron}
	{nhận proton}
	{nhường proton}
	\loigiai{}
\end{ex}
%%%==============HetCau_12==============%%%

%%%==============Cau_13==============%%%
\begin{ex}Phản ứng nào dưới đây là phản ứng oxi hoá-khử?
	\choice
	{$HNO_3+\mathrm{NaOH} \to \mathrm{NaNO}_3+H_2O$}
	{$N_2O_5+H_2O\to 2HNO_3$}
	{\True $2HNO_3+3H_2\mathrm{~S} \to 3\mathrm{~S}+2NO+4H_2O$}
	{$2 \mathrm{Fe}(OH)_3 \xrightarrow{t^o} \mathrm{Fe}_2O_3+3 H_2 O$}
	\loigiai{}
\end{ex}
%%%==============HetCau_13==============%%%

%%%==============Cau_14==============%%%
\begin{ex}[Đề THPT QG-2015]
	Phản ứng nào sau đây không phải là phản ứng oxi hóa-khử?
	\choice
	{\True $\mathrm{CaCO}_3\xrightarrow{t^0} \mathrm{CaO}+CO_2$}
	{$2\mathrm{KClO}_3\xrightarrow{t^0} 2\mathrm{KCl}+3O_2$}
	{$2\mathrm{NaOH}+\mathrm{Cl}_2\to \mathrm{NaCl}+\mathrm{NaClO}+H_2O$}
	{$4\mathrm{Fe}(OH)_2+O_2\xrightarrow{t^0} 2\mathrm{Fe}_2O_3+4H_2O$}
	\loigiai{}
\end{ex}
%%%==============HetCau_14==============%%%

%%%==============Cau_15==============%%%
\begin{ex}Trong phản ứng hóa học: $\mathrm{Fe}+H_2SO_4\to \mathrm{FeSO}_4+H_2$; mỗi nguyên tử $\mathrm{Fe}$ đã
	\choice
	{\True nhường 2 electron}
	{nhận 2 electron}
	{nhường 1 electron}
	{nhận 1 electron}
	\loigiai{}
\end{ex}
%%%==============HetCau_15==============%%%

%%%==============Cau_16==============%%%
\begin{ex}Trong các phản ứng hóa học: $2\mathrm{Na}+2H_2O\to 2\mathrm{NaOH}+H_2$, chất oxi hóa là
	\choice
	{\True $H_2O$}
	{$\mathrm{NaOH}$}
	{$\mathrm{Na}$}
	{$H_2$}
	\loigiai{}
\end{ex}
%%%==============HetCau_16==============%%%

%%%==============Cau_17==============%%%
\begin{ex}Cho nước $\mathrm{Cl}_2$ vào dung dịch $\mathrm{NaBr}$ xảy ra phản ứng hóa học: $\mathrm{Cl}_2+2\mathrm{NaBr} \to 2\mathrm{NaCl}+\mathrm{Br}_2$. Trong phản ứng hóa học trên, xảy ra quá trình oxi hóa chất
	\choice
	{$\mathrm{NaCl}$}
	{$\mathrm{Br}_2$}
	{$\mathrm{Cl}_2$}
	{\True $\mathrm{NaBr}$}
	\loigiai{}
\end{ex}
%%%==============HetCau_17==============%%%

%%%==============Cau_18==============%%%
\begin{ex}Phương trình hóa học của phản ứng nào sau đây không thể hiện tính khử của ammonia $\left(NH_3\right)$?
	\choice
	{$4 NH_3+5 O_2 \xrightarrow{\mathrm{xt}, t^0} 4 NO+6 H_2O$}
	{\True $NH_3+\mathrm{HCl} \to NH_4\mathrm{Cl}$}
	{$2NH_3+3\mathrm{Cl}_2\to 6\mathrm{HCl}+N_2$}
	{$4 NH_3+3 O_2 \xrightarrow{t^0} 2 \mathrm{~N}_2+6 H_2O$}
	\loigiai{}
\end{ex}
%%%==============HetCau_18==============%%%

%%%==============Cau_19==============%%%
\begin{ex}Trong phản ứng: $3 \mathrm{Cu}+8 HNO_3 \to 3 \mathrm{Cu}\left(NO_3\right)_2+2 NO+4 H_2O$. Số phân tử nitric acid $\left(HNO_3\right)$ đóng vai trò chất oxi hóa là
	\choice
	{$8$}
	{$6$}
	{4a}
	{\True $2$}
	\loigiai{}
\end{ex}
%%%==============HetCau_19==============%%%

%%%==============Cau_20==============%%%
\begin{ex}[ĐHKA]
	Sục khí $SO_2$ vào dung dịch $\mathrm{KMnO}_4$ (thuốc tím), màu tím nhạt dần rồi mất màu (biết sản phẩm tạo thành là $K_2SO_4, \mathrm{MnSO}_4, H_2SO_4$ và $H_2O$). Nguyên nhân là do
	\choice
	{$SO_2$ đã oxi hóa $\mathrm{KMnO}_4$ thành $\mathrm{MnO}_2$}
	{\True $SO_2$ đã khử $\mathrm{KMnO}_4$ thành $\mathrm{Mn}^{+2}$}
	{$\mathrm{KMnO}_4$ đã khử $SO_2$ thành $S^{+6}$}
	{$H_2O$ đã oxi hóa $\mathrm{KMnO}_4$ thành $\mathrm{Mn}^{+2}$}
	\loigiai{}
\end{ex}
%%%==============HetCau_20==============%%%

%%%==============Cau_21==============%%%
\begin{ex}Sản xuất gang trong công nghiệp bằng các sử dụng khí $CO$ khử $\mathrm{Fe}_2O_3$ ở nhiệt độ cao theo phản ứng sau: $\mathrm{Fe}_2O_3+3 CO \xrightarrow{t^0} 2 \mathrm{Fe}+3 CO_2$. Trong phản ứng trên, chất đóng vai trò chất khử là
	\choice
	{$\mathrm{Fe}_2O_3$}
	{\True $CO$}
	{$\mathrm{Fe}$}
	{$CO_2$}
	\loigiai{}
\end{ex}
%%%==============HetCau_21==============%%%

%%%==============Cau_22==============%%%
\begin{ex}Bromine vừa là chất oxi hóa, vừa là chất khử trong phản ứng nào sau dây?
	\choice
	{\True $3 \mathrm{Br}_2+6 \mathrm{NaOH} \to 5 \mathrm{NaBr}+\mathrm{NaBrO}_3+3 H_2O$}
	{$\mathrm{Br}_2+H_2\to 2\mathrm{HBr}$}
	{$3\mathrm{Br}_2+2\mathrm{Al} \to 2\mathrm{AlBr}_3$}
	{$\mathrm{Br}_2+2KI \to I_2+2\mathrm{KBr}$}
	\loigiai{}
\end{ex}
%%%==============HetCau_22==============%%%

%%%==============Cau_23==============%%%
\begin{ex}[Đề TSCĐ-2014]
	Cho phương trình hóa học:
	$\mathrm{aAl}+\mathrm{bH}_2SO_4\to \mathrm{cAl}_2\left(SO_4\right)_3+\mathrm{dSO}_2+\mathrm{eH}_2O$. Tỉ lệ a: b là
	\choice
	{$1: 1$}
	{$2: 3$}
	{\True $1: 3$}
	{$1: 2$}
	\loigiai{}
\end{ex}
%%%==============HetCau_23==============%%%

%%%==============Cau_24==============%%%
\begin{ex}[Đề TS Đại học A-2009]
	Cho phương trình phản ứng:
	$\mathrm{Fe}_3 O_4+HNO_3 \to \mathrm{Fe}\left(NO_3\right)_3+N_x O_y+H_2O$. Sau khi cân bằng phương trình hoá học trên với hệ số của các chất là những số nguyên, tối giản thì hệ số của $HNO_3$ là
	\choice
	{\True $46x-18y$}
	{$45x-18y$}
	{$13x-9y$}
	{$23x-9y$}
	\loigiai{}
\end{ex}
%%%==============HetCau_24==============%%%

%%%==============Cau_25==============%%%
\begin{ex}[Đề TSĐH A-2010]
	Trong phản ứng: $K_2 \mathrm{Cr}_2 O_7+\mathrm{HCl} \to \mathrm{CrCl}_3+\mathrm{Cl}_2+\mathrm{KCl}+H_2O$. Số phân tử $\mathrm{HCl}$ đóng vai trò chất khử bằng $k$ lần tổng số phân tử $\mathrm{HCl}$ tham gia phản ứng. Giá trị của $k$ là
	\choice
	{$3/ 14$}
	{$4/ 7$}
	{$1/ 7$}
	{\True $3/ 7$}
	\loigiai{}
\end{ex}
%%%==============HetCau_25==============%%%

%%%==============Cau_26==============%%%
\begin{ex}[Đề TSĐH B-2014]
	Cho phản ứng: $SO_2+\mathrm{KMnO}_4+H_2O \to K_2 SO_4+\mathrm{MnSO}_4+H_2 SO_4$. Trong phương trình hóa học của phản ứng trên, khi hệ số của $\mathrm{KMnO}_4$ là 2 thì hệ số của $SO_2$ là
	\choice
	{$4$}
	{\True $5$}
	{$6$}
	{$7$}
	\loigiai{}
\end{ex}
%%%==============HetCau_26==============%%%

%%%==============Cau_27==============%%%
\begin{ex}Cho các phản ứng sau:
	\begin{enumerate}[(a)]
		\item  $SO_3+H_2O\to H_2SO_4$;
		\item  $\mathrm{CaCO}_3+2\mathrm{HCl} \to \mathrm{CaCl}_2+CO_2+H_2O$;
		\item  $C+H_2O\xrightarrow{t^0} CO+H_2$;
		\item  $CO_2+\mathrm{Ca}(OH)_2\to \mathrm{CaCO}_3+H_2O$;
		\item  $\mathrm{Ca}+2H_2O\to \mathrm{Ca}(OH)_2+H_2$;
		\item  $2\mathrm{KMnO}_4\xrightarrow{t^n} K_2\mathrm{MnO}_4+\mathrm{MnO}_2+O_2$.
	\end{enumerate}
	Số phản ứng oxi hóa-khử là
	\choice
	{$2$}
	{\True $3$}
	{$4$}
	{$5$}
	\loigiai{Phản ứng oxi hóa-khử: (c), (e) và (g)}
\end{ex}
%%%==============HetCau_27==============%%%

%%%==============Cau_28==============%%%
\begin{ex}Cho các phản ứng hóa học sau:
	\begin{enumerate}[(a)]
		\item  $\mathrm{CaCO}_3\xrightarrow{t^{\text {"}}} \mathrm{CaO}+CO_2$;
		\item  $CH_4\xrightarrow{t^{\prime \prime}, \mathrm{xt}} C+2H_2$;
		\item  $2\mathrm{Al}(OH)_3\xrightarrow{t^0} \mathrm{Al}_2O_3+3H_2O$;
		\item  $2\mathrm{NaHCO}_3\xrightarrow{t^0} \mathrm{Na}_2CO_3+CO_2+H_2O$.
	\end{enumerate}
	Số phản ứng có kèm theo sự thay đổi số oxi hóa của các nguyên tử là
	\choice
	{\True $1$}
	{$2$}
	{$2$}
	{$4$}
	\loigiai{
		Phản ứng kèm sự thay đổi số oxi hóa: (b)
	}
\end{ex}
%%%==============HetCau_28==============%%%

%%%==============Cau_29==============%%%
\begin{ex}[Đề TSĐH B-2009]
	Cho các phản ứng sau:
	\begin{enumerate}[(a)]
		\item  $4\mathrm{HCl}+\mathrm{PbO}_2\to \mathrm{PbCl}_2+\mathrm{Cl}_2+2H_2O$.
		\item  $2\mathrm{HCl}+2HNO_3\to 2NO_2+\mathrm{Cl}_2+2H_2O$.
		\item  $2\mathrm{HCl}+\mathrm{Zn} \to \mathrm{ZnCl}_2+H_2$.
		\item  $\mathrm{HCl}+NH_4HCO_3\to NH_4\mathrm{Cl}+CO_2+H_2O$.
	\end{enumerate}
	Số phản ứng trong đó $\mathrm{HCl}$ thể hiện tính khử là
	\choice
	{\True $2$}
	{$3$}
	{$1$}
	{$4$}
	\loigiai{
		Phản ứng $\mathrm{HCl}$ thể hiện tính khử: (a) và (b).
	}
\end{ex}
%%%==============HetCau_29==============%%%

%%%==============Cau_30==============%%%
\begin{ex}Calcium chloride dùng trong điện phân để sản xuất calcium kim loại và điều chế các hợp kim của calcium. Với tính chất hút ẩm lớn, calcium chloride được dùng làm tác nhân sấy khí và chất lỏng. Do nhiệt độ đông đặc thấp nên dung dịch calcium(II) chloride được dùng làm chất tải lạnh trong các hệ thống lạnh,... Ngoài ra, calcium chloride còn được làm chất keo tụ trong hóa dược và dược phẩm hay trong công việc khoan dầu khí.
Trong phản ứng tạo thành calcium(II) chloride từ đơn chất: $\mathrm{Ca}+\mathrm{Cl}_2\to \mathrm{CaCl}_2$. Kết luận nào sau đây đúng?
	\choice
	{Mỗi nguyên tử Ca nhận 2e}
	{Mỗi nguyên tử $\mathrm{Cl}$ nhận $2e$}
	{Mỗi phân tử $\mathrm{Cl}_2$ nhường $2e$}
	{\True Mỗi nguyên tử $\mathrm{Ca}$ nhường $2e$.}
\end{ex}
%%==============HetCau_30==============%%%
\begin{ex}Thực hiện các phản ứng sau:
	\begin{enumerate}[a)]
		\item $Ca(OH)_2$\explus$Cl_2$\MuiTen$CaOCl_2$\explus $H_2O$
		\item $3Cl_2$\explus$6KOH$\MuiTen[$t^{\circ}$][][][-1] $5KCl$ \explus $KClO_3$ \explus $3H_2O$
		\item $3Cl_2$ \explus $2FeCl_2$ \MuiTen $2FeCl_3$
		\item $2KClO_3$ \MuiTen[$t^{\circ}$][][][-1] $2KCl$ \explus $3O_2$
	\end{enumerate}
	Trong các phản ứng trên số phản ứng  $Cl_2$ đóng vai trò là chất oxi hóa
	\choice
	{$1$}
	{\True$2$}
	{$3$}
	{$4$}
	\loigiai{ 
		Phản ứng $Cl$ đóng vai trò là chất oxi hóa là (c), (d)
	}
\end{ex}
%%==============HetCau_31==============%%%
\begin{ex}Thực hiện các phản ứng sau:
	\begin{enumerate}[a)]
		\item $Ca(OH)_2$\explus$Cl_2$\MuiTen$CaOCl_2$\explus $H_2O$
		\item $3Cl_2$\explus$6KOH$\MuiTen[$t^{\circ}$][][][-1] $5KCl$ \explus $KClO_3$ \explus $3H_2O$
		\item $3Cl_2$ \explus $2FeCl_2$ \MuiTen $2FeCl_3$
		\item $2KClO_3$ \MuiTen[$t^{\circ}$][][][-1] $2KCl$ \explus $3O_2$
	\end{enumerate}
	Trong các phản ứng trên số phản ứng  $Cl_2$ đóng vai trò là chất oxi hóa
	\choice
	{$1$}
	{\True$2$}
	{$3$}
	{$4$}
	\loigiai{ 
		Phản ứng $Cl$ đóng vai trò là chất oxi hóa là (c), (d)
	}
\end{ex}

%%%=================Tổng ôn lý thuyết 40 câu=========================%%%
\nhanmanh{Phần trắc nghiệm tổng ônlý thuyết 40 câu}
%%%==============Cau_1==============%%%
\begin{ex}[Đề THPT QG-2018]
	Số oxi hóa của chromium $(\mathrm{Cr})$ trong hợp chất $K_2\mathrm{Cr}_2O_7$ là
	\choice
	{$+2$}
	{$+3$}
	{\True $+6$}
	{$+4$}
	\loigiai{}
\end{ex}
%%%==============HetCau_1==============%%%

%%%==============Cau_2==============%%%
\begin{ex}Số oxi hóa của nguyên tử $S$ trong hợp chất $SO_2$ là
	\choice
	{$+2$}
	{\True $+4$}
	{$+6$}
	{$-1$}
	\loigiai{}
\end{ex}
%%%==============HetCau_2==============%%%

%%%==============Cau_3==============%%%
\begin{ex}Cho các chất sau: $C_2H_6, CH_4O$ và $C_2H_4$. Số oxi hóa trung bình của nguyên tử $C$ trong các phân tử trên lần lượt là
	\choice
	{\True $-3,-2,-2$}
	{$-3,-3,-2$}
	{$-2,-2,-2$}
	{$-3,-2,-3$}
	\loigiai{}
\end{ex}
%%%==============HetCau_3==============%%%

%%%==============Cau_4==============%%%
\begin{ex}Hợp chất nào sau đây chứa hai loại nguyên tử iron $(\mathrm{Fe})$ với số oxi hóa+2 và+3?
	\choice
	{$\mathrm{FeO}$}
	{\True $\mathrm{Fe}_3O_4$}
	{$\mathrm{Fe}(OH)_3$}
	{$\mathrm{Fe}_2O_3$}
	\loigiai{}
\end{ex}
%%%==============HetCau_4==============%%%

%%%==============Cau_5==============%%%
\begin{ex}Chromium $(\mathrm{Cr})$ có số oxi hóa+2 trong hợp chất nào sau đây?
	\choice
	{$\mathrm{Cr}(OH)_3$}
	{$\mathrm{Na}_2\mathrm{CrO}_4$}
	{\True $\mathrm{CrCl}_2$}
	{$\mathrm{Cr}_2O_3$}
	\loigiai{}
\end{ex}
%%%==============HetCau_5==============%%%

%%%==============Cau_6==============%%%
\begin{ex}Thuốc tím chứa ion permanganate $\left(\mathrm{MnO}_4^{-}\right)$ có tính oxi hóa mạnh, được sử dụng để sát trùng, diệt khuẩn trong y học, đời sống và nuôi trồng thủy sản. Số oxi hóa của manganse trong ion permanganate là
	\choice
	{$+2$}
	{$+3$}
	{\True $+7$}
	{$+6$}
	\loigiai{}
\end{ex}
%%%==============HetCau_6==============%%%

%%%==============Cau_7==============%%%
\begin{ex}Cho các phân tử sau: $N_2, NH_3, HNO_3$. Số oxi hóa của nguyên tử $N$ trong các phân tử trên lần lượt là
	\choice
	{$0,-3,-4$}
	{$0,+3,+5$}
	{$-3,-3,+4$}
	{\True $0,-3,+5$}
	\loigiai{}
\end{ex}
%%%==============HetCau_7==============%%%

%%%==============Cau_8==============%%%
\begin{ex}Trong hợp chất $SO_3$, số oxi hóa của sulfur (S) là
	\choice
	{$+2$}
	{$+3$}
	{$+5$}
	{\True $+6$}
	\loigiai{}
\end{ex}
%%%==============HetCau_8==============%%%

%%%==============Cau_9==============%%%
\begin{ex}Trong phản ứng oxi hóa-khử, chất nhường electron được gọi là
	\choice
	{\True chất khử}
	{chất oxi hóa}
	{acid}
	{base}
	\loigiai{}
\end{ex}
%%%==============HetCau_9==============%%%

%%%==============Cau_10==============%%%
\begin{ex}Phản ứng kèm theo sự cho và nhận electron được gọi là phản ứng
	\choice
	{đốt cháy}
	{phân hủy}
	{trao đổi}
	{\True oxi hóa-khử}
	\loigiai{}
\end{ex}
%%%==============HetCau_10==============%%%

%%%==============Cau_11==============%%%
\begin{ex}Dấu hiệu để nhận biết một phản ứng oxi hoá-khử là
	\choice
	{tạo ra chất kết tủa}
	{tạo ra chất khí}
	{có sự thay đổi màu sắc của các chất}
	{\True có sự thay đổi số oxi hoá của một số nguyên tố}
	\loigiai{}
\end{ex}
%%%==============HetCau_11==============%%%

%%%==============Cau_12==============%%%
\begin{ex}Phản ứng nào dưới đây không phải là phản ứng oxi hoá-khử?
	\choice
	{\True $\mathrm{Al}_4C_3+12H_2O\to 4\mathrm{Al}(OH)_3+3CH_4$}
	{$2\mathrm{Na}+2H_2O\to 2\mathrm{NaOH}+H_2$}
	{$\mathrm{NaH}+H_2O\to \mathrm{NaOH}+H_2$}
	{$2\mathrm{~F}_2+2H_2O\to 4HF+O_2$}
	\loigiai{}
\end{ex}
%%%==============HetCau_12==============%%%

%%%==============Cau_13==============%%%
\begin{ex}Nguyên tử sulfur chỉ thể hiện tính khử (trong điều kiện phản ứng phủ hợp) trong hợp c sau đây?
	\choice
	{\True $SO_2$}
	{$H_2SO_4$}
	{$H_2\mathrm{~S}$}
	{$\mathrm{Na}_2SO_3$}
	\loigiai{}
\end{ex}
%%%==============HetCau_13==============%%%

%%%==============Cau_14==============%%%
\begin{ex}Xét phản ứng điều chế $H_2$ trong phòng thí nghiệm: $\mathrm{Zn}+2\mathrm{HCl} \to \mathrm{ZnCl}_2+H_2$. Chất đ trò chất khử trong phản ứng là
	\choice
	{$H_2$}
	{$\mathrm{ZnCl}_2$}
	{\True $\mathrm{HCl}$}
	{$\mathrm{Zn}$}
	\loigiai{}
\end{ex}
%%%==============HetCau_14==============%%%

%%%==============Cau_15==============%%%
\begin{ex}Nguyên tử sulfur $(S)$ thể hiện tính khử và tính oxi hóa trong chất nào sau đây?
	\choice
	{$SO_3$}
	{$SO_2$}
	{$H_2SO_4$}
	{\True $H_2\mathrm{~S}$}
	\loigiai{}
\end{ex}
%%%==============HetCau_15==============%%%

%%%==============Cau_16==============%%%
\begin{ex}Nguyên tử carbon $(C)$ có khả năng thể hiện tính oxi hóa, vừa có khả năng thể hiện tính khử trong chất nào sau đây?
	\choice
	{C}
	{\True $CO_2$}
	{$\mathrm{CaCO}_3$}
	{$CH_4$}
	\loigiai{}
\end{ex}
%%%==============HetCau_16==============%%%

%%%==============Cau_17==============%%%
\begin{ex}Dẫn khí $H_2$ đi qua ống sứ đựng bột $\mathrm{CuO}$ nung nóng để thực hiện phản ứng hóa học sau: $\mathrm{CuO}+H_2\xrightarrow{t^0} \mathrm{Cu}+H_2O$. Trong phản ứng trên, chất đóng vai trò chất khử là
	\choice
	{\True $\mathrm{CuO}$}
	{$\mathrm{Cu}$}
	{$H_2$}
	{$H_2O$}
	\loigiai{}
\end{ex}
%%%==============HetCau_17==============%%%

%%%==============Cau_18==============%%%
\begin{ex}Carbon đóng vai trò chất oxi hóa ở phản ứng nào sau đây?
	\choice
	{$C+O_2\xrightarrow{t^0} CO_2$}
	{$C+CO_2\xrightarrow{t^0} 2CO$}
	{\True $C+H_2O \xrightarrow{t^0} CO+H_2$}
	{$C+2H_2\xrightarrow{t^0} CH_4$}
	\loigiai{}
\end{ex}
%%%==============HetCau_18==============%%%

%%%==============Cau_19==============%%%
\begin{ex}Khi tham gia các phản ứng đốt cháy nhiên liệu, oxygen đóng vai trò là
	\choice
	{\True chất khử}
	{acid}
	{base}
	{chất oxi hóa}
	\loigiai{}
\end{ex}
%%%==============HetCau_19==============%%%

%%%==============Cau_20==============%%%
\begin{ex}Chlorine vừa đóng vai trò chất oxi hóa, vừa đóng vai trò chất khử trong phản ứng nào sau đây?
	\choice
	{$2\mathrm{Na}+\mathrm{Cl}_2\xrightarrow{t^0} 2\mathrm{NaCl}$}
	{$H_2+\mathrm{Cl}_2\xrightarrow{\text {as}} 2\mathrm{HCl}$}
	{$2\mathrm{FeCl}_2+\mathrm{Cl}_2\xrightarrow{t^0} 2\mathrm{FeCl}_3$}
	{\True $2 \mathrm{NaOH}+\mathrm{Cl}_2 \to \mathrm{NaCl}+\mathrm{NaClO}+H_2O$}
	\loigiai{}
\end{ex}
%%%==============HetCau_20==============%%%

%%%==============Cau_21==============%%%
\begin{ex}Thực hiện các phản ứng sau:
	(a) $C+O_2\xrightarrow{t^0} CO_2$
	(b) $4\mathrm{Al}+3C\xrightarrow{t^0} \mathrm{Al}_4C_3$
	(c) $C+CO_2\xrightarrow{t^0} 2CO$
	(d) $\mathrm{CaO}+3C\xrightarrow{t^0} \mathrm{CaC}_2+CO$
	Phản ứng trong đó carbon vừa đóng vai trò chất oxi hóa, vừa đóng vai trò chất khử là
	\choice
	{(a)}
	{(b)}
	{(c)}
	{\True (d)}
	\loigiai{}
\end{ex}
%%%==============HetCau_21==============%%%

%%%==============Cau_22==============%%%
\begin{ex}Phản ứng nào dưới đây $NH_3$ không đóng vai trò là chất khử?
	\choice
	{$4 NH_3+5 O_2 \xrightarrow{t^0, \mathrm{xt}} 4 NO+6 H_2O$}
	{$2NH_3+3\mathrm{Cl}_2\to N_2+6\mathrm{HCl}$}
	{$2 NH_3+3 \mathrm{CuO} \xrightarrow{t^0} 3 \mathrm{Cu}+N_2+3 H_2O$}
	{\True $2 NH_3+H_2O_2+\mathrm{MnSO}_4 \to \mathrm{MnO}_2+\left(NH_4\right)_2 SO_4$}
	\loigiai{}
\end{ex}
%%%==============HetCau_22==============%%%

%%%==============Cau_23==============%%%
\begin{ex}Trong phản ứng: $3 NO_2+H_2O \to 2 HNO_3+NO. NO_2$ đóng vai trò
	\choice
	{là chất oxi hoá}
	{là chất oxi hoá, nhưng đồng thời cũng là chất khử}
	{là chất khử}
	{\True không là chất oxi hoá và cũng không là chất khử}
	\loigiai{}
\end{ex}
%%%==============HetCau_23==============%%%

%%%==============Cau_24==============%%%
\begin{ex}Cho phản ứng: $2\mathrm{Na}+\mathrm{Cl}_2\to 2\mathrm{NaCl}$. Trong phản ứng này, nguyên tử sodium $(\mathrm{Na})$
	\choice
	{bị oxi hoá}
	{vừa bị oxi hoá, vừa bị khử}
	{bị khử}
	{\True không bị oxi hoá, không bị khử}
	\loigiai{}
\end{ex}
%%%==============HetCau_24==============%%%

%%%==============Cau_25==============%%%
\begin{ex}Cho phản ứng: $\mathrm{Zn}+\mathrm{CuCl}_2\to \mathrm{ZnCl}_2+\mathrm{Cu}$. Trong phản ứng này, $1\mathrm{~mol} \mathrm{Cu}^{+2}$
	\choice
	{đã nhận 1 mol electron}
	{\True đã nhận 2 mol electron}
	{đã nhường 1 mol electron}
	{đã nhường 2 mol electron}
	\loigiai{}
\end{ex}
%%%==============HetCau_25==============%%%

%%%==============Cau_26==============%%%
\begin{ex}Trong phản ứng dưới đây vai trò của $NO_2$ là gì?
	$2NO_2$ \explus $2NaOH$ \MuiTen $NaNO_3$ \explus $NaNO_2$ \explus $H_2O$
	\choice
	{\True chỉ bị oxi hóa}
	{chỉ bị khử}
	{không bị oxi hóa, không bị khử}
	{vừa bị oxi hóa, vừa bị khử}
	\loigiai{}
\end{ex}
%%%==============HetCau_26==============%%%

%%%==============Cau_27==============%%%
\begin{ex}[Đề TSCĐ-2008]
	Cho phản ứng hóa học: $\mathrm{Fe}+\mathrm{CuSO}_4\to \mathrm{FeSO}_4+\mathrm{Cu}$. Trong phản ứng trên xảy ra
	\choice
	{sự oxi hóa Fe và sự oxi hóa $\mathrm{Cu}$}
	{\True sự khử $\mathrm{Fe}^{2+}$ và sự oxi hóa $\mathrm{Cu}$}
	{sự oxi hóa $\mathrm{Fe}$ và sự khử $\mathrm{Cu}^{2+}$}
	{sự khử $\mathrm{Fe}^{2+}$ và sự khử $\mathrm{Cu}^{2+}$}
	\loigiai{}
\end{ex}
%%%==============HetCau_27==============%%%

%%%==============Cau_28==============%%%
\begin{ex}[Đề TSĐH B-2013]
	Cho phương trình hóa học của phản ứng:
	$2\mathrm{Cr}+3\mathrm{Sn}^{2+} \to 2\mathrm{Cr}^{3+}+3\mathrm{Sn}$. Nhận xét nào sau đây về phản ứng trên là đúng?
	\choice
	{$\mathrm{Sn}^{2+}$ là chất khử, $\mathrm{Cr}^{3+}$ là chất oxi hóa}
	{Cr là chất oxi hóa, $\mathrm{Sn}^{2+}$ là chất khử}
	{Cr là chất khứ, $\mathrm{Sn}^{2+}$ là chất oxi hóa}
	{\True $\mathrm{Cr}^{3+}$ là chất khư, $\mathrm{Sn}^{2+}$ là chất oxi hóa}
	\loigiai{}
\end{ex}
%%%==============HetCau_28==============%%%

%%%==============Cau_29==============%%%
\begin{ex}[Đề TSCĐ-2010]
	Nguyên tử $S$ đóng vai trò vừa là chất khử, vừa là chất oxi hoá trong phản ứng nào sau đây?
	\choice
	{$S+2\mathrm{Na} \xrightarrow{t^0} \mathrm{Na}_2\mathrm{~S}$}
	{$S+6 HNO_3 \xrightarrow{t^{\circ}} H_2SO_4+6 NO_2+2 H_2O$}
	{\True $S+3\mathrm{~F}_2\xrightarrow{t^0} SF_6$}
	{$4 \mathrm{~S}+6 \mathrm{NaOH}_{(\mathrm{dac})} \xrightarrow{t^{\circ}} 2 \mathrm{Na}_2 \mathrm{~S}+\mathrm{Na}_2 \mathrm{~S}_2 O_3+3 H_2O$}
	\loigiai{}
\end{ex}
%%%==============HetCau_29==============%%%

%%%==============Cau_30==============%%%
\begin{ex}[Đề TSĐH A-2013]
	Cho phương trình hóa học:
	$\mathrm{aAl}+\mathrm{bHNO}_3\to \mathrm{cAl}\left(NO_3\right)_3+\mathrm{dNO}+\mathrm{eH}_2O$. Ti lệ $a: b$ là
	\choice
	{$1: 3$}
	{$2: 3$}
	{\True 2: 5}
	{$1: 4$}
	\loigiai{}
\end{ex}
%%%==============HetCau_30==============%%%

%%%==============Cau_31==============%%%
\begin{ex}[Đề TSĐH B-2013]
	Cho phản ứng: $\mathrm{FeO}+HNO_3 \to \mathrm{Fe}\left(NO_3\right)_3+NO+H_2O$. Trong phương trình của phản ứng trên, khi hệ số của $\mathrm{FeO}$ là 3 thì hệ số của $HNO_3$ là
	\choice
	{$6$}
	{$8$}
	{$4$}
	{\True $10$}
	\loigiai{}
\end{ex}
%%%==============HetCau_31==============%%%

%%%==============Cau_32==============%%%
\begin{ex}[Đề TSĐH A-2013]
	Cho phương trình phản ứng:
	$\mathrm{aFeSO}_4+\mathrm{bK}_2\mathrm{Cr}_2O_7+\mathrm{cH}_2SO_4\to \mathrm{dFe}_2\left(SO_4\right)_3+\mathrm{eK}_2SO_4+\mathrm{fCr}_2\left(SO_4\right)_3+\mathrm{gH}_2O$. Tí lệ a: b là
	\choice
	{$6: 1$}
	{$2: 3$}
	{3:2}
	{\True $1: 6$}
	\loigiai{}
\end{ex}
%%%==============HetCau_32==============%%%

%%%==============Cau_33==============%%%
\begin{ex}[Đề TSCD-2012]
	Cho phản ứng hóa học: $\mathrm{Cl}_2+KOH \xrightarrow{t^{\circ}} \mathrm{KCl}+\mathrm{KClO}_3+H_2O$. Tì lệ giựa số nguyên tử chlorine $(\mathrm{Cl})$ đóng vai trò chất oxi hóa và số nguyên tử chlorine đóng vai trò chất khử trong phương trình hóa học của phản ứng đã cho tương úng là
	\choice
	{$1: 5$}
	{$5: 1$}
	{3: 1}
	{\True 1:3}
	\loigiai{}
\end{ex}
%%%==============HetCau_33==============%%%

%%%==============Cau_34==============%%%
\begin{ex}Thực hiện các phản ửng hóa học sau:
	(a) $S+O_2\xrightarrow{\circ} SO_2$;
	(b) $\mathrm{Hg}+S\to \mathrm{HgS}$;
	(c) $H_2+S \xrightarrow{e^{\circ}} H_2\mathrm{~S}$;
	(d) $S+3\mathrm{~F}_2\xrightarrow{t^0} SF_6$.
	Số phản ứng sulfur (S) đóng vai trò chất oxi hóa là
	\choice
	{\True $1$}
	{$2$}
	{$3$}
	{$4$}
	\loigiai{
		Phản ứmg $S$ đóng vai trò chất oxi hóa: (b) và (c).
	}
\end{ex}
%%%==============HetCau_34==============%%%

%%%==============Cau_35==============%%%
\begin{ex}Cho cảc phản ứng sau:
	(a) $\mathrm{Ca}(OH)_2+\mathrm{Cl}_2 \to \mathrm{CaOCl}_2+H_2O$;
	(b) $2 NO_2+2 \mathrm{NaOH} \to \mathrm{NaNO}_3+\mathrm{NaNO}_2+H_2O$;
	(c) $O_3+2\mathrm{Ag} \to \mathrm{Ag}_2O+O_2$;
	(d) $H_2\mathrm{~S}+SO_2 \to 3 \mathrm{~S}+2 H_2O$;
	(e) $4\mathrm{KClO}_3\to \mathrm{KCl}+\mathrm{KClO}_4$.
	
	Số phằn ứng oxi hóa-khứ là
	\choice
	{$2$}
	{\True $3$}
	{$4$}
	{$5$}
	\loigiai{
		Số phản ứng oxi hóa-khư: (a), (b), (c), (d) và (e).
	}
\end{ex}
%%%==============HetCau_35==============%%%

%%%==============Cau_36==============%%%
\begin{ex}Thực hiện các phản ứng sau:
	(a) $\mathrm{Ca}(OH)_2+\mathrm{Cl}_2 \to \mathrm{CaOCl}_2+H_2O$
	(b) $3 \mathrm{Cl}_2+6 KOH \xrightarrow{t} 5 \mathrm{KCl}+\mathrm{KClO}_3+3 H_2O$
	(c) $\mathrm{Cl}_2+2\mathrm{FeCl}_2\to \mathrm{FeCl}_3$
	(d) $\mathrm{KClO}_3\xrightarrow{l^{\circ}} 2\mathrm{KCl}+3O_2$
	Số phản ứng chlorine đóng vai trò chát oxi hóa là
	\choice
	{$1$}
	{\True $2$}
	{$3$}
	{$4$}
	\loigiai{
		Phản ứng $\mathrm{Cl}$ đóng vai trỏ chất oxi hóa: (c) và (d).
	}
\end{ex}
%%%==============HetCau_36==============%%%

%%%==============Cau_37==============%%%
\begin{ex}[Đề TSĐH A-2008]
	Cho các phản ứng sau:
	(a) $4 \mathrm{HCl}+\mathrm{MnO}_2 \to \mathrm{MnCl}_2+\mathrm{Cl}_2+2 H_2O$.
	(b) $2\mathrm{HCl}+\mathrm{Fe} \to \mathrm{FeCl}_2+H_2$.
	(c) $14 \mathrm{HCl}+K_2\mathrm{Cr}_2O_7 \to 2 \mathrm{KCl}+2 \mathrm{CrCl}_3+3 \mathrm{Cl}_2+7 H_2O$.
	(d) $6\mathrm{HCl}+2\mathrm{Al} \to 2\mathrm{AlCl}_3+3H_2$.
	(e) $16 \mathrm{HCl}+2 \mathrm{KMnO}_4 \to 2 \mathrm{KCl}+2 \mathrm{MnCl}_2+5 \mathrm{Cl}_2+8 H_2O$.
	Số phản ứng trong đó $\mathrm{HCl}$ thể hiện tính oxi hóa là
	\choice
	{$2$}
	{$1$}
	{$4$}
	{\True 3}
	\loigiai{Phản ứng $\mathrm{HCl}$ thể hiện tính oxi hóa: (b) và (d)}
\end{ex}
%%%==============HetCau_37==============%%%

%%%==============Cau_38==============%%%
\begin{ex}Trong thiên nhiên manganese (Mn) là nguyên tố tương đối phổ biến, đứng thứ ba trong các kim loại chuyển tiếp, chỉ sau $\mathrm{Fe}$ và Ti. Các khoáng vật chính của manganese là hausmanite $\left(\mathrm{Mn}_3O_4\right)$, pyrolusite $\left(\mathrm{MnO}_2\right)$, braunite $\left(\mathrm{Mn}_2O_3\right)$ và manganite $(\mathrm{MnOOH})$. Manganese tồn tại ở rất nhiều trạng thái số oxi hóa khác nhau từ+2 tới+7.
	1. Cho các chất sau: $\mathrm{Mn}, \mathrm{MnO}_2, \mathrm{MnCl}_2, \mathrm{KMnO}_4$. Số oxi hóa của nguyên tố $\mathrm{Mn}$ trong các chất lần lượt là
	\choice
	{$+2,-2,-4,+8$}
	{\True $0,+4,+2,+7$}
	{$0,+4,-2,+7$}
	{$0,+2,-4,-7$}
	\loigiai{}
\end{ex}
%%%==============HetCau_38==============%%%

%%%==============Cau_39==============%%%
\begin{ex}Trong thiên nhiên manganese (Mn) là nguyên tố tương đối phổ biến, đứng thứ ba trong các kim loại chuyển tiếp, chỉ sau $\mathrm{Fe}$ và Ti. Các khoáng vật chính của manganese là hausmanite $\left(\mathrm{Mn}_3O_4\right)$, pyrolusite $\left(\mathrm{MnO}_2\right)$, braunite $\left(\mathrm{Mn}_2O_3\right)$ và manganite $(\mathrm{MnOOH})$. Manganese tồn tại ở rất nhiều trạng thái số oxi hóa khác nhau từ+2 tới+7.
	Phản ứng nào sau đây không có sự thay đổi số oxi hóa của nguyên tố $\mathrm{Mn}$?
	\choice
	{\True $\mathrm{MnO}_2+4 \mathrm{HCl} \xrightarrow{t^{\circ}} \mathrm{MnCl}_2+\mathrm{Cl}_2+2 H_2O$}
	{$\mathrm{Mn}+O_2\to \mathrm{MnO}_2$}
	{$2 \mathrm{HCl}+\mathrm{MnO} \to \mathrm{MnCl}_2+H_2O$}
	{$6 KI+2 \mathrm{KMnO}_4+4 H_2O \to 3 I_2+2 \mathrm{MnO}_2+8 KOH$}
	\loigiai{}
\end{ex}
%%%==============HetCau_39==============%%%
%%%============Cau_40.1==============%%%
\begin{ex}
	Trong thiên nhiên manganese (Mn) là nguyên tố tương đối phổ biến, đứng thứ ba trong các kim loại chuyển tiếp, chỉ sau $\mathrm{Fe}$ và Ti. Các khoáng vật chính của manganese là hausmanite $\left(\mathrm{Mn}_3O_4\right)$, pyrolusite $\left(\mathrm{MnO}_2\right)$, braunite $\left(\mathrm{Mn}_2O_3\right)$ và manganite $(\mathrm{MnOOH})$. Manganese tồn tại ở rất nhiều trạng thái số oxi hóa khác nhau từ+2 tới+7.
	1. Cho các chất sau: $\mathrm{Mn}, \mathrm{MnO}_2, \mathrm{MnCl}_2, \mathrm{KMnO}_4$. Số oxi hóa của nguyên tố $\mathrm{Mn}$ trong các chất lần lượt là
	\choice
	{$+2,-2,-4,+8$}
	{\True $0,+4,+2,+7$}
	{$0,+4,-2,+7$}
	{$0,+2,-4,-7$}
	\loigiai{}
\end{ex}
%%%============HetCau_40.1==============%%%
%%%============Cau_40.2==============%%%
\begin{ex}
	Trong thiên nhiên manganese (Mn) là nguyên tố tương đối phổ biến, đứng thứ ba trong các kim loại chuyển tiếp, chỉ sau $\mathrm{Fe}$ và Ti. Các khoáng vật chính của manganese là hausmanite $\left(\mathrm{Mn}_3O_4\right)$, pyrolusite $\left(\mathrm{MnO}_2\right)$, braunite $\left(\mathrm{Mn}_2O_3\right)$ và manganite $(\mathrm{MnOOH})$. Manganese tồn tại ở rất nhiều trạng thái số oxi hóa khác nhau từ+2 tới+7.
	Phản ứng nào sau đây không có sự thay đổi số oxi hóa của nguyên tố $\mathrm{Mn}$?
	\choice
	{$\mathrm{MnO}_2+4 \mathrm{HCl} \xrightarrow{t^{\circ}} \mathrm{MnCl}_2+\mathrm{Cl}_2+2 H_2O$}
	{$\mathrm{Mn}+O_2\to \mathrm{MnO}_2$}
	{\True $2 \mathrm{HCl}+\mathrm{MnO} \to \mathrm{MnCl}_2+H_2O$}
	{$6 KI+2 \mathrm{KMnO}_4+4 H_2O \to 3 I_2+2 \mathrm{MnO}_2+8 KOH$}
	\loigiai{}
\end{ex}
%%%============HetCau_40.2==============%%%

\Closesolutionfile{ansex}
\Closesolutionfile{ans}	
%%%%%%%%%%%%%%Trắc nghiệm đúng sai%%%%%%%%%%%%%%%%%%%%%%%%
\nhanmanh{Bài tập trắc nghiệm Đúng Sai}
\Opensolutionfile{ans}[Ans/DATAM]
\luulgEXTF
%%\LGexTF
%%\tatloigiaiex
\Opensolutionfile{ansex}[LOIGIAITN/LGTNTFCHUONG4]
\Opensolutionfile{ansbook}[Ans/DATNTFCHUONG4]
\Writetofile{ansex}{\protect\nhanmanh{Lời giải chi tiết phần trắc nghiệm đúng sai}}
%%%=============EX_1=============%%%
\begin{ex}
	\choiceTF
	{\True Chất khử là chất nhường electron, chất oxi hóa là chất nhận electron}
	{Chất khử là chất có số oxi hóa giảm sau phản ứng, chất oxi hóa là chất có số oxi hóa tăng sau phản ứng}
	{Chất khử tham gia vào quá trình khử, chất oxi hóa tham gia vào quá trình oxi hóa}
	{\True Chất khử và chất oxi hóa có thể là cùng một chất}
	\loigiai{}
\end{ex}
%%%=============EX_2=============%%%
\begin{ex}
	\choiceTF
	{Quá trình khử là quá trình nhường e của chất khử}
	{\True Quá trình  oxi hóa là quá trình làm tăng số oxi hóa của một nguyên tố}
	{Quá trình oxi hóa là quá trình nhận e của chất oxi hóa}
	{\True Quá trình khử xảy ra đối với chất oxi hóa và quá trình oxi hóa xảy ra đối với chất khử}
	\loigiai{}
\end{ex}
%%%=============EX_3=============%%%
\begin{ex}Xét vai trò của Clo trong phản ứng $Cl_2$ \explus $KOH$ \MuiTen $KCl$ \explus $KClO$ \explus $H_2O$
	\choiceTF
	{ $Cl_2 $ là chất oxi hóa $KOH$ là chất khử}
	{\True $Cl_2$ vừa là chất oxi hóa, vừa là chất khử}
	{$Cl_2 $tham gia vào quá trình khử và $KOH$ tham gia vào quá trình oxi hóa}
	{\True $ Cl_2 + 2e$ \MuiTen[][][0.65]$2Cl^{-}$}
	\loigiai{}
\end{ex}
%%%=============EX_4=============%%%
\begin{ex}
	Cho các phát biểu sau:
	\choiceTF
	{Trong tất cả các hợp chất số oxi hóa của H là +1}
	{\True Trong đơn chất, số oxi hoá của nguyên tử bằng 0}
	{\True Số oxi hóa là điện tích hình thức nếu giả định liên kết giứa các nguyên tử là liên kết ion}
	{\True Số oxi hóa của Alumium trong hợp chất $AlCl_3$ là +3}
	\loigiai{}
\end{ex}
%%%=============EX_5=============%%%
\begin{ex}
	Cho các phát biểu sau:
	\choiceTF
	{Phản ứng oxi hóa là phản ứng có sự nhường và nhận proton}
	{\True Cho quá trình $Cu$ \MuiTen $Cu^{+2} +2e$, quá trình này gọi là quá trình oxi hóa}
	{\True Trong phản ứng oxi hóa - khử tổng số electron cho phải bằng tổng số electron nhận}
	{Phản ứng $CaCO_3$ \MuiTen[$t^{\circ}$]$CaO + CO_2$}
	\loigiai{}
\end{ex}
%%%=============EX_6=============%%%
\begin{ex}
	Cho các phát biểu sau:
	\choiceTF
	{Chất oxi hoá là chất cho điện tử, chứa nguyên tố có số oxi hóa tăng sau phản ứng}
	{Chất khử là chất nhận điện tử, chứa nguyên tố có số oxi hóa tăng sau phản ứng}
	{Trong phân tử $NH_4NO_3$ thì số oxi hóa của 2 nguyên tử nitơ là: –3 và+5}
	{\True Cho quá trình: $Fe^{2+} → Fe^{3+}+1e$. Đây là quá trình oxi hóa}
	\loigiai{}
\end{ex}
%%%=============EX_7=============%%%
\begin{ex}[]
	\choiceTF
	{Phản ứng $CaCO_3$ \MuiTen [$t^\circ$][xt][1.2][][][\arrowL] $CaO$  \explus  $CO_2$\MuiTenU là phản ứng oxi hóa - khử}
	{Chất oxi hóa tham gia vào quá trình oxi hóa}
	{Số oxi hóa của Al luôn là $+3$ trong tất cả các hợp chất}
	{\True Tỉ lệ số phân tử $HNO_3$ bị khử và tham gia gia môi trường tạo muối trong phản ứng:
		$Cu + HNO_3 $ \MuiTen $ Cu{(NO_3)}_2 + NO + H_2O$ là 3:1}
	\loigiai{}
\end{ex}
%%%=============EX_8=============%%%
\begin{ex}
	Trong các phát biểu sau phát biểu nào đúng, phát biểu nào sai?
	\choiceTF
	{\True Phản ứng oxi hoá-khử là phản ứng luôn xảy ra đồng thời sự oxi hoá và sự khử}
	{Phản ứng oxi hoá-khử là phản ứng trong đó có sự thay đổi số oxi hoá của tất cả các nguyên tố hóa học}
	{\True Phản ứng oxi hoá-khử là phản ứng trong đó xảy ra sự trao đổi electron giữa các chất}
	{Phản ứng oxi hóa khử là phản ứng phải có ít nhất 2 chất tham gia trong đó có một chất là chất oxi hóa, một chất là chất khử}
	\loigiai{}
\end{ex}
\Closesolutionfile{ansex}
\Closesolutionfile{ansbook}
\Closesolutionfile{ans}
%%%%%%%%=======Phần tự luận==================%%%
\Opensolutionfile{ansbt}[LOIGIAITL/LGTLCHUONG4]
\Writetofile{ansbt}{\protect\thongtin{LỜI GIẢI CHI TIẾT PHẦN TỰ LUẬN}}
\nhanmanh{Bài tập tự luận}
\luuloigiaibt
%%    \dongkebt
%%     \dongkeHaicotbt
%%      \Olybt
%%        \tatloigiaibt
%%          \hienthiloigiaibt
%%            \dienkhuyetLGBT
%%%==============BT_1==============%%%
\begin{bt}[Xác định số oxi hóa]
	Xác định số oxi hóa của mỗi nguyên tử nguyên tố trong các chất hoặc ion sau: $\mathrm{Al}_2O_3; \mathrm{CaF}_2$; $\mathrm{Fe}_2O_3; \mathrm{Na}_2CO_3; \mathrm{KAl}\left(SO_4\right)_2; NO_3^{-}; NH_4^{+}; \mathrm{MnO}_4^{-}$
\end{bt}

%%%==============BT_2==============%%%
\begin{bt}[Xác định số oxi hóa]
	Xác định số oxi hóa của mỗi nguyên tử trong các phân tử và ion sau đây:
	\begin{enumerate}
		\item $H_2SO_3$;
		\item $\mathrm{Al}(OH)_4^{-}$;
		\item $\mathrm{NaAlH}_4$;
		\item $NO_2^{-}$.
	\end{enumerate}
\end{bt}

%%%==============BT_3==============%%%
\begin{bt}[Xác định số oxi hóa]
	Tính số oxi hóa của nguyên tử đánh dấu * trong các chất và ion dưới đây:
	\begin{enumerate}
		\item $K_2\stackrel{*}{\mathrm{Cr}} O_7; \mathrm{KMnO}_4; \stackrel{*}{\mathrm{~K}} \stackrel{*}{\mathrm{ClO}} K_4; \stackrel{*}{\mathrm{~N}} H_4NO_3$
		\item $\stackrel{*}{\mathrm{~A}} O_2^{-}; \stackrel{*}{PO} O_4^{3-}; \stackrel{*}{C} \mathrm{ClO}_3^{-}; \stackrel{*}{SO_4^{2-}}$
	\end{enumerate}
\end{bt}

%%%==============BT_4==============%%%
\begin{bt}[Xác định số oxi hóa]
	Xác định số oxi hóa của nguyên tử $\mathrm{Fe}$ và $S$ trong các chất sau:
	\begin{enumerate}
		\item $\mathrm{Fe}, \mathrm{FeO}, \mathrm{Fe}_2O_3, \mathrm{Fe}(OH)_3, \mathrm{Fe}_3O_4$.
		\item $S, H_2\mathrm{~S}, SO_2, SO_3, H_2SO_4, \mathrm{Na}_2SO_3$.
	\end{enumerate}
\end{bt}

%%%==============BT_5==============%%%
\begin{bt}[Xác định số oxi hóa]
	Xác định số oxi hóa của các nguyên tố trong các chất và ion sau:
	\begin{enumerate}
		\item $\mathrm{Fe}, N_2, SO_3, H_2SO_4, \mathrm{CuS}, \mathrm{Cu}_2\mathrm{~S}, \mathrm{Na}_2O_2, H_3\mathrm{AsO}_4$.
		\item $\mathrm{Br}_2, O_3, \mathrm{HClO}_3, \mathrm{KClO}_4, \mathrm{NaClO}, NH_4NO_3, \mathrm{~N}_2O, \mathrm{NaNO}_2$.
	\end{enumerate}
\end{bt}
%%%=============BT_1=============%%%
\begin{bt}[Xác định chất oxi hóa, chất khử và quá trình]
	Xác định chất oxi hóa, chất khử, quá trình oxi hóa, quá trình khử trong các phản ứng sau:
	\begin{enumerate}
		\item $\mathrm{Ag}^{+}+\mathrm{Fe}^{2+} \to \mathrm{Ag}+\mathrm{Fe}^{3+}$
		\item $3\mathrm{Hg}^{2+}+2\mathrm{Fe} \to 3\mathrm{Hg}+2\mathrm{Fe}^{3+}$
		\item $2\mathrm{As}+3\mathrm{Cl}_2\to 2\mathrm{AsCl}_3$
		\item $\mathrm{Al}+6H^{+}+3NO_3^{-} \to \mathrm{Al}^{3+}+3NO_2+3H_2O$
	\end{enumerate}
	\loigiai{}
\end{bt}

%%%==============BT_1==============%%%
\begin{bt}[Cân bằng phản ứng oxi hóa khử]
	Cân bằng các phản ứng oxi hóa-khử sau (dạng cơ bản)
	\begin{enumerate}[(1)]
		\item $\mathrm{Fe}_2O_3+CO \longrightarrow \mathrm{Fe}+CO_2$
		\item $NH_3+O_2\longrightarrow NO+H_2O$
		\item $\mathrm{NaBr}+\mathrm{Cl}_2\longrightarrow \mathrm{NaCl}+\mathrm{Br}_2$
		\item $\mathrm{Cr}(OH)_3+\mathrm{Br}_2+OH^{-} \longrightarrow \mathrm{CrO}_4^{2-}+\mathrm{Br}^{-}+H_2O$
		\item $H^{+}+\mathrm{MnO}_4^{-}+HCOOH \longrightarrow \mathrm{Mn}^{2+}+H_2O+CO_2$
		\item $\mathrm{Br}_2+KI \longrightarrow I_2+\mathrm{KBr}$
		\item $NO_2+O_2+H_2O\longrightarrow HNO_3$
		\item $C+HNO_3\longrightarrow CO_2+NO+H_2O$
		\item $SO_2+\mathrm{Br}_2+H_2O\longrightarrow H_2SO_4+\mathrm{HBr}$
		\item $H_2\mathrm{~S}+O_2\longrightarrow S+H_2O$
		\item $P+HNO_3\longrightarrow H_3PO_4+NO_2+H_2O$
		\item $H_2\mathrm{~S}+SO_2\longrightarrow S+H_2O$
	\end{enumerate}
\end{bt}
%%%==============BT_1==============%%%
\begin{bt}[Cân bằng phản ứng oxi hóa khử có môi trường]
	Cân bằng các phản ứng oxi hóa-khử sau :
	\begin{enumerate}[(1)]
		\item $\mathrm{HCl}+\mathrm{PbO}_2\longrightarrow \mathrm{PbCl}_2+\mathrm{Cl}_2+H_2O$
		\item $\mathrm{KMnO}_4+\mathrm{HCl} \longrightarrow \mathrm{KCl}+\mathrm{MnCl}_2+\mathrm{Cl}_2+H_2O$
		\item $\mathrm{HCl}+\mathrm{MnO}_2\longrightarrow \mathrm{MnCl}_2+\mathrm{Cl}_2+H_2O$
		\item $\mathrm{KMnO}_4+KNO_2+H_2SO_4\longrightarrow \mathrm{MnSO}_4+KNO_3+K_2SO_4+H_2O$
		\item $\mathrm{Fe}_3O_4+HNO_3\longrightarrow \mathrm{Fe}\left(NO_3\right)_3+NO+H_2O$
		\item $H_2C_2O_4+\mathrm{KMnO}_4+H_2SO_4\longrightarrow CO_2+\mathrm{MnSO}_4+K_2SO_4+H_2O$
		\item $\mathrm{Zn}+HNO_3\longrightarrow \mathrm{Zn}\left(NO_3\right)_2+NO+H_2O$
		\item $K_2\mathrm{Cr}_2O_7+\mathrm{HCl} \longrightarrow \mathrm{KCl}+\mathrm{CrCl}_3+\mathrm{Cl}_2+H_2O$
		\item $\mathrm{Cu}+H_2SO_4$ (đặc) $\longrightarrow \mathrm{CuSO}_4+SO_2+H_2O$
		\item $\mathrm{Al}+H_2SO_4$ (đặc) $\xrightarrow{\makebox[1cm]{$t^{\circ}$}} \mathrm{Al}_2\left(SO_4\right)_3+SO_2+H_2O$
		\item $\mathrm{Mg}+HNO_3\longrightarrow \mathrm{Mg}\left(NO_3\right)_2+NH_4NO_3+H_2O$
		\item $\mathrm{Fe}+HNO_3\longrightarrow \mathrm{Fe}\left(NO_3\right)_3+NO_2+H_2O$
		\item $\mathrm{Zn}+HNO_3\longrightarrow \mathrm{Zn}\left(NO_3\right)_2+N_2O+H_2O$
	\end{enumerate}
\end{bt}
%%%=============BT_2=============%%%
\begin{bt}
	Nước oxi già có tính oxi hóa mạnh, do khả năng oxi hóa của hydrogen peroxide $\left(H_2O_2\right)$.
	\begin{enumerate}
		\item Từ công thức cấu tạo $H-O-O-H$, hãy xác định số oxi hóa của mỗi nguyên tử.
		\item Nguyên tử nguyên tố nào gây nên tính oxi hóa của $H_2O_2$. Viết các quá trình oxi hóa, quá trình khử minh họa.
	\end{enumerate}
	\loigiai{}
\end{bt}
%%%=============BT_3=============%%%
\begin{bt}
	Xăng E5 là một loại xăng sinh học, được tạo thành khi trộn 5 thể tích ethanol $\left(C_2H_5OH\right)$ với 95 thể tích xăng truyền thống, giúp thay thế một phần nhiên liệu hóa thạch, phù hợp với xu thế phát triển chung trên thế giới và góp phần đảm bảo an ninh năng lượng quốc gia. Viết phương trình đốt cháy ethanol tạo thành $CO_2$ và $H_2O$. Phản ứng này có phải là phản ứng oxi hóa-khử hay không? Nó thuộc loại phản ứng cung cấp hay tích trữ năng lượng?
	\loigiai{}
\end{bt}
%%%=============BT_4=============%%%
\begin{bt}
	Trong môi trường acid, anion dichromate $\left(\mathrm{Cr}_2O_7^{2-}\right)$ có màu da cam sẽ bị khử thành cation $\mathrm{Cr}^{3+}$ có màu xanh. Phản ứng này được sử dụng để kiểm tra nồng độ ethanol trong hơi thở của tài xế. Trong máy kiểm tra hơi thở, $K_2\mathrm{Cr}_2O_7$ sẽ oxi hóa ethanol $\left(C_2H_5OH\right)$ thành ethanal $\left(CH_3CHO\right)$, nên có sự đổi màu từ da cam sang xanh theo phương trình hóa học:
	\begin{center}
		$CH_3 CH_2 OH+K_2\mathrm{Cr}_2O_7+4 H_2 SO_4 \xrightarrow[]{\makebox[1cm]{}} 3 CH_3 CHO+\mathrm{Cr}_2\left(SO_4\right)_3+K_2 SO_4+7 H_2 O$
	\end{center}
	Xác định chất oxi hóa và chất khử trong phản ứng trên?
	\loigiai{}
\end{bt}
%%%=============BT_5=============%%%
\begin{bt}
	Trong không khí ẩm, $\mathrm{Fe}(OH)_2$ màu trắng xanh chuyển dần thành $\mathrm{Fe}(OH)_3$ màu nâu đỏ:
	\begin{center}
		$\mathrm{Fe}(OH)_2+O_2+H_2 O \to \mathrm{Fe}(OH)_3$
	\end{center}
	\begin{enumerate}
		\item Hãy xác định các nguyên tử có sự thay đổi số oxi hóa.
		\item Viết quá trình oxi hóa, quá trình khử.
		\item Dùng mũi tên biểu diễn sự chuyển electron từ chất khử sang chất oxi hóa.
	\end{enumerate}
	\loigiai{}
\end{bt}
%%%=============BT_6=============%%%
\begin{bt}
	Xét phản ứng sản xuất $\mathrm{Cl}_2$ trong công nghiệp: $\mathrm{NaCl}+H_2O\xrightarrow{\text {đpdd cmn}} \mathrm{NaOH}+\mathrm{Cl}_2+H_2$
	\begin{enumerate}
		\item Xác định các nguyên tử có sự thay đổi số oxi hóa. Chỉ rõ chất oxi hóa, chất khử.
		\item Lập phương trình hóa học của phản ứng theo phương pháp thăng bằng electron.
	\end{enumerate}
	\loigiai{}
\end{bt}
%%%=============BT_7=============%%%
\begin{bt}
	Viết các quá trình nhường hay nhận electron của các biến đổi trong các dãy sau:
	\begin{enumerate}
		\item $S^{-2} \to S^0\to S^{+4} \to S^{+6} \to S^{+4}$
		\item $N^{-3} \to N^0\to N^{+2} \to N^{+4} \to N^{+5} \to N^{+2}$
	\end{enumerate}
	\loigiai{}
\end{bt}
%%%=============BT_8=============%%%
\begin{bt}
	Một số loại xe ô tô được trang bị một thiết bị an toàn là túi chứa một lượng nhất định hợp chất ion sodium azide bị phân hủy rất nhậnh, giải phóng khí $N_2$ và nguyên tố $\mathrm{Na}$, làm túi phồng lên, bảo vệ được người trong xe tránh khỏi thương tích. Viết $PTHH$ của phản ứng xảy ra và xác định đây có phải là phản ứng oxi hóa-khử không? Vì sao? Xác định số oxi hóa của mỗi nguyên tử trong $\mathrm{NaN}_3$.
	\loigiai{}
\end{bt}
%%%=============BT_9=============%%%
\begin{bt}
	Điền vào chỗ trống trong đoạn thông tin sau:
	Phản ứng $\mathrm{Fe}_2O_3+CO \to \mathrm{Fe}+CO_2$ xảy ra trong quá trình luyện gang từ quặng hematite là phản ứng...(1)$\ldots$vì có sự thay đổi$\ldots$(2)$\ldots$của các nguyên tố $C$ và $\mathrm{Fe}$. $CO$ là$\ldots$(3)$\ldots$, trong đó $C^{+2} \ldots$ (4)$\ldots$electron và $\mathrm{Fe}_2O_3$ là$\ldots$(5)$\ldots$, trong đó mỗi $\mathrm{Fe}^{+3} \ldots(6) \ldots$ electron.
	\loigiai{}
\end{bt}
%%%=============BT_10=============%%%
\begin{bt}[][][]
	Hãy xác định chất khử, chất oxi hóa trong các phản ứng hóa học dưới đây:
	\begin{enumerate}
		\item $2HNO_3+3H_3\mathrm{AsO}_3\to 2NO+3H_3\mathrm{AsO}_4+H_2O$
		\item $\mathrm{NaI}+3\mathrm{HOCl} \to \mathrm{NaIO}_3+3\mathrm{HCl}$
		\item $2\mathrm{KMnO}_4+5H_2C_2O_4+3H_2SO_4\to 10CO_2+K_2SO_4+2\mathrm{MnSO}_4+8H_2O$
		\item $6H_2SO_4+2\mathrm{Al} \to \mathrm{Al}_2\left(SO_4\right)_3+3SO_2+6H_2O$
	\end{enumerate}
	\loigiai{}
\end{bt}
\Closesolutionfile{ansbt}


%%%%============Dạng 2 Bài toán phản ứng oxi hóa khử================%%%
%\begin{dangNTD}{Bài toán về phản ứng Oxi hóa khử}
%\end{dangNTD}
%\begin{vdm}
%\end{vdm}
%%%%=========vd_1=========%%%
%\begin{vd}
%	
%	\loigiai{}
%\end{vd}
%
%%%%=========vd_2=========%%%
%\begin{vd}
%	
%	\loigiai{}
%\end{vd}
%
%%%%=========vd_3=========%%%
%\begin{vd}
%	
%	\loigiai{}
%\end{vd}
%
%%%%=========vd_4=========%%%
%\begin{vd}
%	
%	\loigiai{}
%\end{vd}
%
%%%%=========vd_5=========%%%
%\begin{vd}
%	
%	\loigiai{}
%\end{vd}
%%%%%=====================Bài tập tự luyện Dạng 2==========================%%%
%\newpage
%\begin{bttl}
%\end{bttl}
%%%%==========Phần trắc nghiệm 1 phương án============%%%
%\nhanmanh{Bài Tập Trắc Nghiệm}
%\Opensolutionfile{ans}[Ans/DATNC4-2]
%\luuloigiaiex
%\Opensolutionfile{ansex}[LOIGIAITN/LGTNC4-2]
%%%%==========EX01===============%%%
%\begin{ex}
%	Nội dung câu hỏi trắc nghiệm 1
%	\choice
%	{\True Phương án đúng}
%	{ Phương án sai 1}
%	{Phương án sai 2}
%	{Phương án sai 3}
%	\loigiai{Nội dung lời giải câu TN 1}
%\end{ex}
%%%%==========EX02===============%%%
%\begin{ex}
%	Nội dung câu hỏi trắc nghiệm 2
%	\choice
%	{Phương án sai 1}
%	{\True Phương án đúng}
%	{Phương án sai 2}
%	{Phương án sai 3}
%	\loigiai{Nội dung lời giải câu TN 2}
%\end{ex}
%%%%==========EX03===============%%%
%\begin{ex}
%	Nội dung câu hỏi trắc nghiệm 3
%	\choice
%	{Phương án sai 1}
%	{Phương án sai 2}
%	{Phương án sai 3}
%	{\True Phương án đúng}
%	\loigiai{Nội dung lời giải câu TN 3}
%\end{ex}
%\Closesolutionfile{ansex}
%\Closesolutionfile{ans}	
%
%%%%==========Phần trắc nghiệm đúng sai============%%%
%\nhanmanh{Bài Tập Trắc Nghiệm Đúng Sai}
%\Opensolutionfile{ans}[Ans/DATAM2]
%\Opensolutionfile{ansbook}[Ans/DATNTFC4-2]
%\luulgEXTF
%\Opensolutionfile{ansex}[LOIGIAITN/LGTNTFC4-2]
%%%%==========EX01===============%%%
%\begin{ex}
%	Nội dung câu hỏi trắc nghiệm đúng sai 1
%	\choiceTF
%	{\True Phương án đúng 1}
%	{ Phương án sai 1}
%	{Phương án sai 2}
%	{\True Phương án đúng 2}
%	\loigiai{Nội dung lời giải câu TN đúng sai 1}
%\end{ex}
%%%%==========EX02===============%%%
%\begin{ex}
%	Nội dung câu hỏi trắc nghiệm đúng sai 2
%	\choiceTF
%	{\True Phương án đúng 1}
%	{ Phương án sai 1}
%	{ Phương án sai 2}
%	{Phương án sai 3}
%	\loigiai{Nội dung lời giải câu TN đúng sai 2}
%\end{ex}
%%%%==========EX03===============%%%
%\begin{ex}
%	Nội dung câu hỏi trắc nghiệm đúng sai 3
%	\choiceTF
%	{\True Phương án đúng 1}
%	{ Phương án sai 1}
%	{\True Phương án đúng 2}
%	{\True Phương án đúng 3}
%	\loigiai{Nội dung lời giải câu TN đúng sai 3}
%\end{ex}
%\Closesolutionfile{ansex}
%\Closesolutionfile{ansbook}
%\Closesolutionfile{ans}	
%%%%==============================%%%
%
%%%%==========Phần tự luận============%%%
%\nhanmanh{BÀI TẬP TỰ LUẬN}
%\Opensolutionfile{ansbt}[LOIGIAITL/LGTLC4-2]
%\luuloigiaibt
%%%%==========BT01===============%%%
%\begin{bt}
%	Nội dung bài tập tự luận 1
%	\loigiai{Nội dung lời giải bài tập tự luận 1}
%\end{bt}
%%%%==========BT02===============%%%
%\begin{bt}
%	Nội dung bài tập tụ luận 2
%	\loigiai{Nội dung lời giải bài tập tự luận 2}
%\end{bt}
%%%%==========BT03===============%%%
%\begin{bt}
%	Nội dung bài tập tụ luận 3
%	\loigiai{Nội dung lời giải bài tập tự luận 3}
%\end{bt}
%\Closesolutionfile{ansbt}
%
%\newpage
%\thongtin{ĐÁP ÁN VÀ LỜI GIẢI CHI TIẾT TRẮC NGHIỆM}
%\nhanmanh{Bảng đáp án trắc nghiệm}
%\bangdapan{DATNC4}
%\input{LOIGIAITN/LGTNCHUONG4.tex}
%\thongtin{ĐÁP ÁN VÀ LỜI GIẢI CHI TIẾT TRẮC NGHIỆM ĐÚNG SAI}
%\nhanmanh{Bảng đáp án trắc nghiệm đúng sai}\\
%\bangdapanExTF{DATNTFCHUONG4}
%\input{LOIGIAITN/LGTNTFCHUONG4.tex}
%\input{LOIGIAITL/LGTLCHUONG4.tex}

