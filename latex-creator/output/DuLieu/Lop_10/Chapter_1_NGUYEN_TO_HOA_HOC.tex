\newpage
\section{Nguyên tố hóa học}
\subsection{NỘI DUNG BÀI HỌC}
\subsubsection{Hạt nhân nguyên tử}
\begin{hoplythuyet}
	\begin{enumerate}
		\item Số đơn vị điện tích hạt nhân $(\mathrm{Z})=$ số proton $(\mathrm{P})=$ số electron (E).
		\item Điện tích hạt nhân $=+$ Z .
	\end{enumerate}
\end{hoplythuyet}

\begin{longtable}{|c|c|c|c|c|c|}
	\caption{\indam[\maunhan]{Số lượng các hạt cơ bản và số khối của nguyên tử một số nguyên tố}}
	\label{tab:table2}\\
	\hline 
	\rowcolor{\mauphu!25}$ \textbf { Tên nguyên tố } $ & $ \textbf { Kí hiệu } $ & $ \textbf{P} $ & $ \textbf{N} $ & $ \textbf { Số khối  (A) } $ & $ \textbf{E} $\\
	\hline 
	$ \text { Helium } $ & $ \text { He } $ & 2 & 2 & 4 & 2 \\
	\hline 
	$ \text { Lithium } $ & $ \text { Li } $ & 3 & 4 & 7 & ? \\
	\hline 
	$ \text { Nitrogen } $ & \text { N } & 7 & ? & 14 & 7 \\
	\hline
	$ \text { 0xygen }$ & 0 & 8 & 8 & ? & 8 \\
	\hline
\end{longtable}
\begin{hoivadap}
	Bổ sung các số liệu còn thiếu ở (Bảng \ref{tab:table2})
	\huongdan{%
	\taodongke{5}
}
\end{hoivadap}
\subsubsection{Nguyên tố hóa học}
\begin{kngsnd}
	\indam[\maunhan]{Nguyên tố hóa học} là tập hợp những nguyên tử có cùng số proton trong hạt nhân.
\end{kngsnd}
\GSND[\sffamily\bfseries][\faArrows][\maunhan]{Ví dụ:}Tập hợp những nguyên tử có hạt nhân chứa 6 proton được xếp vào nguyên tố cacbon
\begin{kngsnd}
	\begin{itemize}
	\item Số proton trong một hạt nhân nguyên tử được gọi là \indam[\maunhan]{số hiệu nguyên tử}, kí hiệu là Z.
	\item Tổng số proton (Z) và neutron $(N)$ trong một hạt nhân nguyên tử được gọi là \indam[\maunhan]{số khối}, kí hiệu là $A$.
	\end{itemize}
 \centering\tcbox[width=.25\textwidth,colback=\mycolor!10,colframe=\maunhan]{\NTDtext[\fontsize{25pt}{6pt}\fontfamily{qag}\selectfont][\bfseries][\maunhan]{A = Z + N}}
\end{kngsnd}
\begin{note}
\taodongke{6}
\end{note}
\begin{notegsnd}
Số khối	$  \approx  $ nguyên tử khối (tính gần đúng)
\end{notegsnd}
\begin{kngsnd}
	\indam[\maunhan]{Kí hiệu nguyên tử} ${ }_Z^A X$ cho biết kí hiệu hoá học của nguyên tố $(X)$, số hiệu nguyên tử $(Z)$ và số khối $(A)$.
\end{kngsnd}

\subsubsection{Đồng vị}
\begin{kngsnd}
	Các nguyên tử của cùng một nguyên tố hóa học có hạt nhân khác nhau về số neutron là \indam[\maudam]{đồng vị} của nhau.
\end{kngsnd}
\subsubsection{Nguyên tử khối và nguyên tử khối trung bình}
\begin{kngsnd}
	\begin{itemize}
		\item \indam[\maunhan]{Nguyên tử khối} là khối lượng tương đối của một nguyên tử, cho biết khối lượng của một nguyên tử nặng gấp bao nhiêu lần \indam[\maudam]{đơn vị khối lượng nguyên tử} ($1\mathrm{amu} $).
		\item Mỗi một nguyên tố có nhiều đồng vị do đó người ta sử dụng \indam[\maunhan]{nguyên tử khối trung bình}
	\end{itemize}
\end{kngsnd}
\begin{hoplythuyet}
	Công thức tính nguyên tử khối trung bình của nguyên tố $\mathrm{X}$ :
	$$
	\overline{A x}=\frac{a_1 \times A_1+a_2 \times A_2+\ldots+a_i \times A_i}{100}
	$$
	\begin{itemize}
\item $\overline{\mathrm{A} x}$ là nguyên tử khối trung bình của $\mathrm{X}$.
\item $A_1$ là nguyên tử khối đổng vị thứ $i$.
\item a là tỉ lệ \% số nguyên tử đông vị thứ i.
	\end{itemize}
\end{hoplythuyet}
\begin{hoivadap}
	Trong tự nhiên, nguyên tố copper có hai đồng vị với phần trăm số nguyên tử tương ứng
	là ${ }_{29}^{63} \mathrm{Cu} \quad(69,15 \%)$ và ${ }_{29}^{65} \mathrm{Cu}$
	(30,85\%). Hãy tính nguyên tử
	khối trung bình của nguyên tố copper.
	\huongdan{
	\taodongke{5}
}
\end{hoivadap}

\subsection{Bài tập trắc nghiệm}

\Opensolutionfile{ans}[DAPAN/H10C01B02_BTTN]

%%%=============== CauEX_1 ===============%%%
\begin{ex}[2]
	Phát biểu nào sau đây \indam{không} đúng?
	\choice
	{ Số hiệu nguyên tử bằng số đơn vị điện tích hạt nhân nguyên tử.}
	{ Số khối của hạt nhân bằng tổng số proton và số neutron.}
	{ \True Trong nguyên tử, số đơn vị điện tích hạt nhân bằng số proton và bằng số neutron}
	{ Nguyên tố hoá học là những nguyên tử có cùng số đơn vị điện tích hạt nhân}
	\loigiai{%
	}
\end{ex}
%%%***********EndEX***********%%%
%%%%=============== CauEX_2 ===============%%%
\begin{ex}[2]
	Số hiệu nguyên tử cho biết thông tin nào sau đây?
	\choice
	{\True Số proton}
	{ Số neutron}
	{ Số khối}
	{ Nguyên tử khối}
	\loigiai{%
	}
\end{ex}
%%%%***********EndEX***********%%%
%%%=============== CauEX_3 ===============%%%
\begin{ex}[2]
	Dãy nào sau đây gồm các đồng vị của cùng một nguyên tố hoá học?
	\choice
	{ ${ }_6^{14} \mathrm{X},{ }_7^{14} \mathrm{Y},{ }_8^{14} \mathrm{Z}$}
	{ ${ }_9^{19} \mathrm{X},{ }_{10}^{19} \mathrm{Y},{ }_{10}^{20} \mathrm{Z}$.}
	{\True ${ }_{14}^{28} \mathrm{X},{ }_{14}^{29} \mathrm{Y},{ }_{14}^{30} \mathrm{Z}$.}
	{ ${ }_{18}^{40} \mathrm{X},{ }_{19}^{40} \mathrm{Y},{ }_{20}^{40} \mathrm{Z}$}
	\loigiai{%
	}
\end{ex}
%%%%***********EndEX***********%%%
%%%=============== CauEX_4 ===============%%%
\begin{ex}[2]
	Kí hiệu nguyên tử nào sau đây viết đúng?
	\choice
	{ \True ${ }_7^{15} \mathrm{~N}$.}
	{ ${ }^{16} \mathrm{O}$.}
	{ ${ }_{16} \mathrm{~S}$.}
	{ $\mathrm{Mg}_{12}^{24}$}
	\loigiai{%
	}
\end{ex}
%%%%***********EndEX***********%%%
%%%%=============== CauEX_5 ===============%%%
\begin{ex}[2]
	Thông tin nào sau đây không đúng về ${ }_{82}^{206} \mathrm{~Pb}$?
	\choice
	{ Số đơn vị điện tỉch hạt nhân là 82.}
	{ \True Số proton và neutron là 82.}
	{ Số neutron là 124.}
	{ Số khối là 206}
	\loigiai{%
	}
\end{ex}
%%%%***********EndEX***********%%%
%%%%=============== CauEX_6 ===============%%%
\begin{ex}[2]
	Cho kí hiệu các nguyên tử sau:
	$$
	{ }_6^{14} \mathrm{X},{ }_7^{14} \mathrm{Y},{ }_8^{16} \mathrm{Z},{ }_9^{19} \mathrm{~T},{ }_8^{17} \mathrm{Q},{ }_9^{16} \mathrm{M},{ }_{10}^{19} \mathrm{E},{ }_7^{16} \mathrm{G},{ }_8^{18} \mathrm{~L}.
	$$
	Dãy nào sau đây gồm các nguyên tử thuộc cùng một nguyên tố hoá học?
	\choice
	{ ${ }_6^{14} \mathrm{X},{ }_7^{14} \mathrm{Y},{ }_8^{16} \mathrm{Z}$}
	{ ${ }_8^{16} \mathrm{Z},{ }_9^{16} \mathrm{M},{ }_7^{16} \mathrm{G}$}
	{ ${ }_8^{17} \mathrm{Q},{ }_9^{16} \mathrm{M},{ }_{10}^{19} \mathrm{E}$}
	{ \True ${ }_8^{16} \mathrm{Z},{ }_8^{17} \mathrm{Q},{ }_8^{18} \mathrm{~L}$}
	\loigiai{%
	}
\end{ex}
%%%%***********EndEX***********%%%
%%%%=============== CauEX_7 ===============%%%
\begin{ex}[2]
	Nitrogen có hai đồng vị bền là ${ }_7^{14} \mathrm{~N}$ và ${ }_7^{15} \mathrm{~N}$. Oxygen có ba đồng vị bền là ${ }_8^{16} \mathrm{O},{ }_8^{17} \mathrm{O}$ và ${ }_8^{18} \mathrm{O}$. Số hợp chất $\mathrm{NO}_2$ tạo bởi các đồng vị trên là
	\choice
	{ 3.}
	{ 6.}
	{ 9.}
	{ 12}
	\loigiai{%
	}
\end{ex}
%%%%***********EndEX***********%%%
%%%%=============== CauEX_8 ===============%%%
\begin{ex}[2]
	Trong tự nhiên, bromine có hai đồng vị bền là ${ }_{35}^{79} \mathrm{Br}$ chiếm $50,69 \%$ số nguyên tử và ${ }_{35}^{81} \mathrm{Br}$ chiếm $49,31 \%$ số nguyên tử. Nguyên tử khối trung bình của bromine là
	\choice
	{ 80,00}
	{ 80,112}
	{ 80,986}
	{ 79,986}
	\loigiai{%
	}
\end{ex}
%%%%***********EndEX***********%%%
%%%=============== CauEX_9 ===============%%%
\begin{ex}[2]
	Oxygen có ba đồng vị với tỉ lệ \% số nguyên tử tương ứng là ${ }^{16} \mathrm{O}(99,757 \%)$, ${ }^{17} \mathrm{O}(0,038 \%),{ }^{18} \mathrm{O}(0,205 \%)$. Nguyên tử khối trung bình của oxygen là
	\choice
	{ 16,0.}
	{ 16,2.}
	{ 17,0.}
	{ 18,0}
	\loigiai{%
	}
\end{ex}
%%%%***********EndEX***********%%%
%%%%=============== CauEX_10 ===============%%%
\begin{ex}[2]
	Nguyên tố R có hai đồng vị, nguyên tử khối trung bình là 79,91. Một trong hai đồng vị là ${ }^{79} \mathrm{R}$ (chiếm $54,5 \%$). Nguyên tử khối của đồng vị thứ hai là
	\choice
	{ 80}
	{ 81}
	{ 82}
	{ 80,5}
	\loigiai{%
	}
\end{ex}
%%%%***********EndEX***********%%%
%%%%=============== CauEX_11 ===============%%%
\begin{ex}[2]
	Boron là nguyên tố có nhiều tác dụng đối với cơ thể người như: làm lành vết thương, điều hoà nội tiết sinh dục, chống viêm khớp,... Do ngọn lửa cháy có màu lục đặc biệt nên boron vô định hình được dùng làm pháo hoa. Boron có hai đồng vị là ${ }^{10} \mathrm{~B}$ và ${ }^{11} \mathrm{~B}$, nguyên tử khối trung bình là $10,81$. Tính phần trăm số nguyên tử mỗi đồng vị của boron
\loigiai{%
}
\end{ex}
%%%%***********EndEX***********%%%
%%%%=============== CauEX_12 ===============%%%
\begin{ex}[2]
Đồng vị phóng xạ cobalt (Co-60) phát ra tia $\gamma$ có khả năng đâm xuyên mạnh, dùng điều trị các khối u ở sâu trong cơ thể. Cobalt có ba đồng vị: ${ }_{27}^{59} \mathrm{Co}$ (chiếm $98 \%$), ${ }_{27}^{58} \mathrm{Co}$ và ${ }_{27}^{60} \mathrm{Co}$; nguyên tử khối trung bình là 58,982. Xác định hàm lượng $\%$ của đồng vị phóng xạ Co-60
\loigiai{%
}
\end{ex}
%%%%***********EndEX***********%%%

\Closesolutionfile{ans}







