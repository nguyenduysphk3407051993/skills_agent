%%%%=================EX_01====================%%%
\begin{ex}
    Giải thích tại sao bán kính nguyên tử lại giảm dần từ Li đến F trong cùng chu kỳ 2?
    \choice
    {Số electron lớp ngoài cùng tăng dần}
    {Lực hút của hạt nhân với electron ngoài cùng giảm dần}
    {\True Điện tích hạt nhân tăng dần trong khi số lớp electron không đổi}
    {Khối lượng nguyên tử tăng dần}
    \loigiai{Từ Li đến F, số proton trong hạt nhân tăng dần, trong khi số lớp electron vẫn là 2. Lực hút tĩnh điện giữa hạt nhân và electron ngoài cùng tăng lên, kéo electron lại gần hơn, làm giảm bán kính nguyên tử.}
\end{ex}

%%%%=================EX_02====================%%%
\begin{ex}
    Sắp xếp các nguyên tố sau theo chiều tăng dần bán kính nguyên tử: N, O, P. 
    \choice
    {N < O < P}
    {O < N < P}
    {P < N < O}
    {\True O < N < P}
    \loigiai{N và O thuộc cùng chu kỳ 2, bán kính giảm dần từ trái sang phải nên O < N. P ở chu kỳ 3, có thêm một lớp electron so với N nên bán kính P > N. Vậy: O < N < P}
\end{ex}
%%%%=================EX_03====================%%%
\begin{ex}
    Trong các ion sau, ion nào có bán kính nhỏ nhất? $Na^+$, $Mg^{2+}$,  $Al^{3+}$, $F^-$.
    \choice
    {$Na^+$}
    {$Mg^{2+}$}
    {\True $Al^{3+}$}
    {$F^-$}
    \loigiai{Cả 4 ion đều có cấu hình electron giống khí hiếm Ne. Tuy nhiên,  $Al^{3+}$ có điện tích hạt nhân lớn nhất, lực hút với electron mạnh nhất nên có bán kính nhỏ nhất.}
\end{ex}

%%%%=================EX_04====================%%%
\begin{ex}
    Nguyên nhân chính khiến ion âm ($X^−$) luôn có bán kính lớn hơn nguyên tử trung hoà (X) tương ứng là gì?
    \choice
    {Số proton trong hạt nhân ion ít hơn}
    {Số nơtron trong hạt nhân ion nhiều hơn}
    {Số lớp electron của ion nhiều hơn}
    {\True Lực đẩy giữa các electron trong ion mạnh hơn}
    \loigiai{Khi nguyên tử X nhận thêm electron tạo thành ion $X^-$, số electron tăng lên trong khi số proton không đổi. Lực đẩy giữa các electron mạnh hơn, làm electron lớp ngoài cùng bị đẩy ra xa hạt nhân hơn, dẫn đến bán kính ion lớn hơn.}
\end{ex}

%%%%=================EX_05====================%%%
\begin{ex}
    Hiệu ứng chắn của electron là gì?
    \choice
    {\True Lực đẩy của các electron lớp trong làm giảm lực hút của hạt nhân lên electron lớp ngoài cùng}
    {Lực hút của hạt nhân lên tất cả các electron}
    {Lực đẩy giữa các electron lớp ngoài cùng}
    {Lực hút giữa hạt nhân và electron lớp trong cùng}
    \loigiai{Các electron ở lớp trong sẽ che chắn một phần lực hút của hạt nhân lên các electron lớp ngoài cùng, làm giảm lực hút hiệu dụng. Hiệu ứng này được gọi là hiệu ứng chắn electron.}
\end{ex}

%%%%=================EX_06====================%%%
\begin{ex}
    Giải thích tại sao bán kính nguyên tử của các nguyên tố trong cùng một nhóm A lại tăng dần từ trên xuống dưới?
    \choice
    {Số electron lớp ngoài cùng tăng dần}
    {Điện tích hạt nhân tăng dần}
    {\True Số lớp electron tăng dần}
    {Hiệu ứng chắn electron giảm dần}
    \loigiai{Trong cùng một nhóm A, số lớp electron tăng dần từ trên xuống dưới. Khoảng cách từ hạt nhân đến electron lớp ngoài cùng xa hơn, làm tăng bán kính nguyên tử.}
\end{ex}

%%%%=================EX_07====================%%%
\begin{ex}
    So sánh bán kính của các nguyên tử sau:  K, Ca, Br. 
    \choice
    {K < Ca < Br}
    {\True K > Ca > Br}
    {Br < K < Ca}
    {Br > K > Ca}
    \loigiai{K và Ca thuộc cùng chu kỳ 4, bán kính giảm dần từ trái sang phải nên K > Ca. Br ở cùng chu kỳ với K nhưng nằm bên phải nên bán kính nhỏ hơn K. Vậy K > Ca > Br}
\end{ex}

%%%%=================EX_08====================%%%
\begin{ex}
    Trong các ion sau, ion nào có bán kính lớn nhất? $O^{2-}$, $F^-$, $Na^+$, $Mg^{2+}$.
    \choice
    {$F^-$}
    {$Na^+$}
    {$Mg^{2+}$}
    {\True $O^{2-}$}
    \loigiai{$O^{2-}$ và $F^-$ đều là ion âm, có thêm electron so với nguyên tử trung hoà.  Tuy nhiên, $O^{2-}$ có số proton ít hơn $F^-$, lực hút với electron yếu hơn nên có bán kính lớn hơn.} 
\end{ex}

%%%%=================EX_09====================%%%
\begin{ex}
    Giả sử X và Y là hai nguyên tố thuộc cùng một chu kỳ, X đứng trước Y. Dự đoán về xu hướng biến đổi bán kính nguyên tử và bán kính ion của chúng?
    \choice
    {Bán kính nguyên tử X lớn hơn Y, bán kính ion $X^+$ lớn hơn $Y^{2+}$.}
    {\True Bán kính nguyên tử X lớn hơn Y, bán kính ion $X^+$ nhỏ hơn $Y^{2+}$.}
    {Bán kính nguyên tử X nhỏ hơn Y, bán kính ion $X^+$ lớn hơn $Y^{2+}$.}
    {Bán kính nguyên tử X nhỏ hơn Y, bán kính ion $X^+$ nhỏ hơn $Y^{2+}$.}
    \loigiai{Trong cùng chu kỳ, X đứng trước Y nên bán kính nguyên tử X lớn hơn Y. Tuy nhiên, $Y^{2+}$ mất nhiều electron hơn $X^+$, lực hút của hạt nhân với electron mạnh hơn, bán kính $Y^{2+}$ nhỏ hơn $X^+$.}
\end{ex}

%%%%=================EX_10====================%%%
\begin{ex}
    Vì sao việc so sánh bán kính nguyên tử của nguyên tố khí hiếm lại phức tạp hơn so với các nguyên tố nhóm A?
    \choice
    {Khí hiếm không tạo hợp chất}
    {Khí hiếm có năng lượng ion hóa rất cao}
    {\True Khó xác định bán kính cộng hóa trị cho khí hiếm do chúng ít tham gia liên kết}
    {Khí hiếm có bán kính nguyên tử rất nhỏ}
    \loigiai{Khí hiếm có cấu hình electron lớp ngoài cùng bền vững, ít tham gia liên kết hóa học. Do đó, việc xác định bán kính cộng hóa trị cho khí hiếm trở nên phức tạp hơn so với các nguyên tố nhóm A.}
\end{ex}
``` 
