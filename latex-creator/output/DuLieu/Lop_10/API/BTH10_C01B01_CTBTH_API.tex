%%%%=================EX_01====================%%%
\begin{ex}[1 điểm]
Nguyên tố X có cấu hình electron lớp ngoài cùng là $4s^2$. Vị trí của X trong bảng tuần hoàn là:
    \choice
    {Chu kì 3, nhóm IIA}
    {\True Chu kì 4, nhóm IIA}
    {Chu kì 4, nhóm IA}
    {Chu kì 3, nhóm IA}
    \loigiai{Cấu hình electron lớp ngoài cùng là $4s^2$ nên X thuộc chu kì 4, nhóm IIA.}
\end{ex}

%%%%=================EX_02====================%%%
\begin{ex}[1 điểm]
Một nguyên tố R có cấu hình electron $1s^22s^22p^63s^23p^3$. Công thức hợp chất khí với hidro và công thức oxit cao nhất của R là:
    \choice
    {$RH_2, RO_3$}
    {$RH_4, RO_2$}
    {\True $RH_3, R_2O_5$}
    {$RH_3, RO_3$}
    \loigiai{Nguyên tử R có 5e lớp ngoài cùng, thuộc nhóm VA. Do đó, công thức hợp chất khí với hidro là $RH_3$ và công thức oxit cao nhất là $R_2O_5$.}
\end{ex}

%%%%=================EX_03====================%%%
\begin{ex}[1 điểm]
Nguyên tố X thuộc chu kì 3, nhóm VIIA. Nguyên tử của nguyên tố X có bao nhiêu electron ở lớp ngoài cùng?
    \choice
    {3}
    {5}
    {\True 7}
    {2}
    \loigiai{Nguyên tố thuộc nhóm VIIA có 7e lớp ngoài cùng.}
\end{ex}

%%%%=================EX_04====================%%%
\begin{ex}[1 điểm]
Nguyên tố Y có cấu hình electron lớp ngoài cùng là $3s^23p^5$. Vậy Y thuộc nhóm nào?
    \choice
    {Nhóm VA}
    {Nhóm VIA}
    {\True Nhóm VIIA}
    {Nhóm VIIIA}
    \loigiai{Y có 7e lớp ngoài cùng (3s23p5) nên thuộc nhóm VIIA.}
\end{ex}

%%%%=================EX_05====================%%%
\begin{ex}[1 điểm]
Cho cấu hình electron của nguyên tố X là $1s^22s^22p^63s^23p^1$. Vị trí của X trong bảng tuần hoàn là:
    \choice
    {Chu kì 3, nhóm IIIA}
    {Chu kì 2, nhóm IIIA}
    {\True Chu kì 3, nhóm IIIA}
    {Chu kì 3, nhóm IA}
    \loigiai{X có 3 lớp electron nên thuộc chu kì 3, 3 electron lớp ngoài cùng nên thuộc nhóm IIIA.}
\end{ex}


%%%%=================EX_06====================%%%
\begin{ex}[1 điểm]
Nguyên tố M thuộc chu kì 4, nhóm IA. Hỏi M có mấy electron lớp ngoài cùng?
    \choice
    {2}
    {4}
    {8}
    {\True 1}
    \loigiai{Nguyên tố thuộc nhóm IA có 1 electron lớp ngoài cùng.}
\end{ex}

%%%%=================EX_07====================%%%
\begin{ex}[1 điểm]
Nguyên tố X có số hiệu nguyên tử là 16. Vị trí của X trong bảng tuần hoàn là:
    \choice
    {Chu kỳ 2, nhóm VIA}
    {\True Chu kỳ 3, nhóm VIA}
    {Chu kỳ 3, nhóm IVA}
    {Chu kỳ 2, nhóm IVA}
    \loigiai{Cấu hình electron của X là $1s^22s^22p^63s^23p^4$. Vậy X thuộc chu kỳ 3, nhóm VIA.}
\end{ex}

%%%%=================EX_08====================%%%
\begin{ex}[1 điểm]
Nguyên tố A ở chu kì 3, nhóm IIIA, nguyên tố B ở chu kì 2, nhóm VIA. Công thức phân tử của hợp chất tạo thành từ 2 nguyên tố trên là:
    \choice
    {$A_3B_2$}
    {$A_2B_3$}
    {\True $A_2B_3$}
    {$AB_3$}
    \loigiai{A thuộc nhóm IIIA có hóa trị III, B thuộc nhóm VIA có hóa trị II. Công thức hợp chất là $A_2B_3$.}
\end{ex}

%%%%=================EX_09====================%%%
\begin{ex}[1 điểm]
Một nguyên tố thuộc nhóm VIIA có tổng số hạt là 54.  Số electron của nguyên tố đó là
    \choice
    {17}
    {35}
    {18}
    {36}
    \loigiai{Gọi Z là số hiệu nguyên tử của nguyên tố. Vì nguyên tố thuộc nhóm VIIA nên có 7e lớp ngoài cùng. Tổng số hạt là 54 nên ta có $2Z+N = 54$ mà $p=e=Z$. Do đó $1 \le \dfrac{N}{Z} \le 1,5$
Ta có $Z+1,5Z \le 54 \le 2Z+Z$ $\Rightarrow 15,4 \le Z \le 18$. Vì Z là số nguyên và nguyên tố thuộc nhóm VIIA nên Z=17.
Vậy số electron là 17.}
\end{ex}

%%%%=================EX_10====================%%%
\begin{ex}[1 điểm]
Nguyên tử của nguyên tố X có 13 proton, 14 neutron. Vị trí của nguyên tố X trong bảng tuần hoàn là
    \choice
    {Chu kì 3, nhóm IIIA}
    {\True Chu kì 3, nhóm IIIA}
    {Chu kì 2, nhóm IIIA}
    {Chu kì 3, nhóm IA}
    \loigiai{Số proton = 13 = Số electron.
Cấu hình electron: $1s^22s^22p^63s^23p^1$.
Vậy X thuộc chu kì 3, nhóm IIIA.}
\end{ex}



%%%%=================TF_01====================%%%
\begin{ex}[1 điểm]
	Các phát biểu sau về bảng tuần hoàn đúng hay sai?
	\choiceTF[t]
	{Bảng tuần hoàn gồm 8 nhóm A và 8 nhóm B.}
	{\True Bảng tuần hoàn gồm 8 nhóm A và 8 nhóm B.}
	{Nguyên tố thuộc nhóm A có electron cuối cùng điền vào phân lớp s hoặc p.}
	{\True Nguyên tố thuộc nhóm B có electron cuối cùng điền vào phân lớp d hoặc f.}
    \loigiai{Bảng tuần hoàn gồm 8 nhóm A và 8 nhóm B.
Nguyên tố thuộc nhóm A có electron cuối cùng điền vào phân lớp s hoặc p.
Nguyên tố thuộc nhóm B có electron cuối cùng điền vào phân lớp d hoặc f (trừ một số trường hợp ngoại lệ).
}
\end{ex}


%%%%=================TF_02====================%%%
\begin{ex}[1 điểm]
	Xét các phát biểu sau:
	\choiceTF[t]
	{\True Các nguyên tố trong cùng một chu kỳ có cùng số lớp electron.}
	{Các nguyên tố trong cùng một nhóm A có cùng số electron lớp ngoài cùng.}
	{\True Các nguyên tố trong cùng một nhóm A có tính chất hóa học tương tự nhau.}
	{Số thứ tự chu kỳ bằng số electron lớp ngoài cùng.}
    \loigiai{Các nguyên tố trong cùng một chu kỳ có cùng số lớp electron.
Các nguyên tố trong cùng một nhóm A có cùng số electron hóa trị, nên có tính chất hóa học tương tự nhau.
Số thứ tự chu kỳ bằng số lớp electron.}
\end{ex}



%%%%=================TF_03====================%%%
\begin{ex}[1 điểm]
	Xác định tính đúng sai của các phát biểu sau:
	\choiceTF[t]
	{\True Trong bảng tuần hoàn, ô nguyên tố cho biết kí hiệu hóa học, tên nguyên tố, số hiệu nguyên tử và khối lượng nguyên tử.}
	{Chu kỳ là dãy các nguyên tố mà nguyên tử của chúng có cùng số lớp electron, được xếp theo chiều tăng của điện tích hạt nhân.}
	{\True Nhóm là tập hợp các nguyên tố mà nguyên tử của chúng có cấu hình electron tương tự nhau, do đó có tính chất hóa học giống nhau.}
	{\True Số thứ tự của chu kỳ bằng số lớp electron trong nguyên tử.}
    \loigiai{Trong bảng tuần hoàn, ô nguyên tố cho biết kí hiệu hóa học, tên nguyên tố, số hiệu nguyên tử và khối lượng nguyên tử.
Chu kỳ là dãy các nguyên tố mà nguyên tử của chúng có cùng số lớp electron, được xếp theo chiều tăng của điện tích hạt nhân.
Nhóm là tập hợp các nguyên tố mà nguyên tử của chúng có cấu hình electron tương tự nhau, do đó có tính chất hóa học giống nhau.
Số thứ tự của chu kỳ bằng số lớp electron trong nguyên tử.}
\end{ex}

%%%%=================TF_04====================%%%
\begin{ex}[1 điểm]
	Các nguyên tố nhóm A trong bảng tuần hoàn gồm:
	\choiceTF[t]
	{\True Các nguyên tố s và p}
	{Các nguyên tố d và f}
	{Nguyên tố s}
	{Nguyên tố p}
    \loigiai{Các nguyên tố nhóm A là nguyên tố s và p.}
\end{ex}

%%%%=================TF_05====================%%%
\begin{ex}[1 điểm]
	Nguyên tố s là những nguyên tố mà nguyên tử có electron cuối cùng được điền vào phân lớp nào?
	\choiceTF[t]
	{\True s}
	{p}
	{d}
	{f}
    \loigiai{Nguyên tố s là nguyên tố có electron cuối cùng được điền vào phân lớp s.}
\end{ex}



%%%%=================TF_06====================%%%
\begin{ex}[1 điểm]
	Các nguyên tố nhóm B là:
	\choiceTF[t]
	{Nguyên tố s}
	{Nguyên tố p}
	{\True Nguyên tố d và f}
	{Nguyên tố s và p}
    \loigiai{Nguyên tố nhóm B là các nguyên tố d và f.}
\end{ex}


%%%%=================TF_07====================%%%
\begin{ex}[1 điểm]
	Nguyên tử của nguyên tố X có 2 lớp electron và 5 electron ở lớp ngoài cùng. X thuộc:
	\choiceTF[t]
	{Chu kì 2, nhóm VA}
	{\True Chu kì 2, nhóm VA}
	{Chu kì 5, nhóm IIA}
	{Chu kì 2, nhóm VIIA}
\end{ex}


%%%%=================TF_08====================%%%
\begin{ex}[1 điểm]
	Nguyên tố nào sau đây thuộc nhóm IIIA?
	\choiceTF[t]
	{ $1s^22s^22p^63s^23p^1$}
	{\True $1s^22s^22p^63s^23p^1$}
	{$1s^22s^22p^63s^23p^2$}
	{$1s^22s^22p^63s^23p^3$}
\end{ex}


%%%%=================TF_09====================%%%
\begin{ex}[1 điểm]
	Nguyên tố có cấu hình $1s^22s^22p^63s^23p^63d^{10}4s^24p^3$ thuộc:
	\choiceTF[t]
	{Chu kỳ 4, nhóm VA}
	{\True Chu kỳ 4, nhóm VA}
	{Chu kỳ 4, nhóm VB}
	{Chu kỳ 3, nhóm IIIA}
\end{ex}


%%%%=================TF_10====================%%%
\begin{ex}[1 điểm]
	Các nguyên tố nhóm A trong bảng tuần hoàn gồm:
	\choiceTF[t]
	{\True Nguyên tố s và p}
	{Nguyên tố d}
	{Nguyên tố f}
	{Nguyên tố d và f}
\end{ex}


%%%%=================BT_01====================%%%
\begin{bt}[1 điểm]
    Nguyên tố X thuộc chu kì 3, nhóm VA trong bảng tuần hoàn. Hãy viết cấu hình electron của nguyên tử X và cho biết tính chất hóa học cơ bản của X (tính kim loại, phi kim).
    \loigiai{
    X thuộc chu kì 3, nhóm VA nên có 3 lớp electron và 5 electron lớp ngoài cùng. Cấu hình electron của X là $1s^22s^22p^63s^23p^3$. 
    Vì X có 5 electron lớp ngoài cùng nên X là phi kim.
    }
\end{bt}


%%%%=================BT_02====================%%%
\begin{bt}[1 điểm]
Nguyên tử của nguyên tố Y có 16 electron. 
\begin{enumerate}
    \item Viết cấu hình electron của Y
    \item Xác định vị trí của Y trong bảng tuần hoàn.
    \item  So sánh tính phi kim của Y với các nguyên tố lân cận trong cùng chu kì và nhóm.
\end{enumerate}
	\loigiai{
    1. Cấu hình electron của Y: $1s^22s^22p^63s^23p^4$.
    2. Y thuộc chu kì 3, nhóm VIA.
    3. Tính phi kim:
    Trong cùng chu kì (từ trái sang phải): tính phi kim tăng dần. Vậy tính phi kim $Y > P > Si > Al > Mg > Na$.
    Trong cùng nhóm (từ trên xuống dưới): tính phi kim giảm dần. Vậy tính phi kim $O > S > Se > Te > Po$.}
\end{bt}


%%%%=================BT_03====================%%%
\begin{bt}[1 điểm]
    Hãy cho biết sự biến đổi tính kim loại, tính phi kim của các nguyên tố trong một chu kì và trong một nhóm A.
	\loigiai{Trong một chu kì, khi đi từ trái sang phải, điện tích hạt nhân tăng, bán kính nguyên tử giảm, nên lực hút giữa hạt nhân và electron lớp ngoài cùng tăng, làm cho tính kim loại giảm, tính phi kim tăng.
    Trong một nhóm A, khi đi từ trên xuống dưới, số lớp electron tăng, bán kính nguyên tử tăng, lực hút giữa hạt nhân và electron lớp ngoài cùng giảm, nên tính kim loại tăng, tính phi kim giảm.}
\end{bt}


%%%%=================BT_04====================%%%
\begin{bt}[1 điểm]
    Nguyên tố R có công thức oxit cao nhất là $RO_3$. Hợp chất khí của R với hidro chứa $5,88\%$ hidro về khối lượng. Xác định nguyên tố R.
	\loigiai{Oxit cao nhất của R là $RO_3$ nên R thuộc nhóm VIA. 
    Công thức hợp chất khí với hidro là $RH_2$.
    Ta có $\dfrac{2}{R+2} = \dfrac{5,88}{100}$ $\Rightarrow$ R=32 (Lưu huỳnh). Vậy R là lưu huỳnh (S).}
\end{bt}


%%%%=================BT_05====================%%%
\begin{bt}[1 điểm]
    Nguyên tố X có tổng số proton, nơtron, electron là 34. Biết số nơtron nhiều hơn số proton là 1. Xác định vị trí của X trong bảng tuần hoàn.
	\loigiai{
    Gọi p, n, e lần lượt là số proton, nơtron và electron của X. Ta có:
    $p+n+e=34$
    $n-p=1$
    Vì $p=e$ nên $2p+n=34$
    $n=p+1$
    $\Rightarrow 3p+1=34 \Rightarrow p=11$. Vậy $n=12$.
    Cấu hình electron: $1s^22s^22p^63s^1$.
    Vị trí của X trong bảng tuần hoàn: Chu kì 3, nhóm IA.}
\end{bt}


%%%%=================BT_06====================%%%
\begin{bt}[1 điểm]
    Hợp chất khí với hidro của một nguyên tố là $RH_4$. Oxit cao nhất của nó chứa $53,3\%$ oxi về khối lượng. Tìm nguyên tố R.
        \loigiai{Hợp chất khí với hidro là $RH_4$, vậy R thuộc nhóm IVA.
    Công thức oxit cao nhất là $RO_2$.
    Ta có:
    $\dfrac{32}{R+32} = \dfrac{53,3}{100}$ $\Rightarrow R = 28$ (Silic).
    Vậy R là Si.}
\end{bt}
