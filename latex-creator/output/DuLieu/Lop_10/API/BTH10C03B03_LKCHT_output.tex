```latex
%%%%=================EX_01====================%%%
\begin{ex}
Liên kết cộng hóa trị là liên kết được hình thành bởi:
    \choice
    {Lực hút tĩnh điện giữa các ion mang điện tích trái dấu.}
    {\True Một hay nhiều cặp electron dùng chung giữa hai nguyên tử.}
    {Sự cho và nhận electron giữa hai nguyên tử.}
    {Lực hút tĩnh điện giữa các phân tử.}
    \loigiai{Liên kết cộng hóa trị là liên kết được hình thành bởi một hay nhiều cặp electron dùng chung giữa hai nguyên tử.}
\end{ex}

%%%%=================EX_02====================%%%
\begin{ex}
Hợp chất nào sau đây được tạo nên bởi liên kết cộng hóa trị?
    \choice
    {$NaCl$}
    {$MgO$}
    {\True $H_2O$}
    {$KCl$}
    \loigiai{Hợp chất cộng hóa trị là hợp chất được tạo nên bởi liên kết cộng hóa trị. $H_2O$ được tạo nên bởi liên kết cộng hóa trị giữa nguyên tử O và hai nguyên tử H.}
\end{ex}

%%%%=================EX_03====================%%%
\begin{ex}
Trong công thức Lewis, cặp electron dùng chung giữa hai nguyên tử được biểu diễn bằng:
    \choice
    {Dấu chấm.}
    {\True Đường gạch.}
    {Mũi tên.}
    {Dấu cộng.}
    \loigiai{Trong công thức Lewis, cặp electron dùng chung giữa hai nguyên tử được biểu diễn bằng một đường gạch nối giữa hai nguyên tử.}
\end{ex}

%%%%=================EX_04====================%%%
\begin{ex}
Liên kết giữa hai nguyên tử được gọi là liên kết ba nếu có:
    \choice
    {Một cặp electron dùng chung.}
    {Hai cặp electron dùng chung.}
    {\True Ba cặp electron dùng chung.}
    {Bốn cặp electron dùng chung.}
    \loigiai{Liên kết ba là liên kết được hình thành bởi ba cặp electron dùng chung giữa hai nguyên tử.}
\end{ex}

%%%%=================EX_05====================%%%
\begin{ex}
Liên kết cho nhận là:
    \choice
    {Liên kết được tạo nên bởi một cặp electron dùng chung, mỗi nguyên tử góp một electron.}
    {\True Liên kết được tạo nên bởi một cặp electron, do một nguyên tử cho và nguyên tử kia nhận.}
    {Liên kết được tạo nên bởi lực hút tĩnh điện giữa các ion.}
    {Liên kết được tạo nên bởi sự xen phủ bên của hai orbital.}
    \loigiai{Liên kết cho nhận là liên kết được tạo nên bởi một cặp electron, trong đó cặp electron này do một nguyên tử cho và nguyên tử kia nhận để dùng chung.}
\end{ex}

%%%%=================EX_06====================%%%
\begin{ex}
Trong phân tử $NH_3$, nguyên tử N có bao nhiêu cặp electron hóa trị riêng?
    \choice
    {0}
    {\True 1}
    {2}
    {3}
    \loigiai{Nguyên tử N trong phân tử $NH_3$ có 5 electron hóa trị. Trong đó, 3 electron tạo 3 liên kết đơn với 3 nguyên tử H, còn lại 1 cặp electron hóa trị riêng.}
\end{ex}

%%%%=================EX_07====================%%%
\begin{ex}
Hiệu độ âm điện của liên kết cộng hóa trị không phân cực nằm trong khoảng nào?
    \choice
    {\True $0$ đến $0.4$}
    {$0.4$ đến $1.7$}
    {$1.7$ đến $3.4$}
    {Lớn hơn $3.4$}
    \loigiai{Hiệu độ âm điện của liên kết cộng hóa trị không phân cực nằm trong khoảng từ 0 đến 0.4.}
\end{ex}

%%%%=================EX_08====================%%%
\begin{ex}
Liên kết trong phân tử $NaCl$ là liên kết:
    \choice
    {Cộng hóa trị không phân cực.}
    {Cộng hóa trị có cực.}
    {\True Ion.}
    {Cho nhận.}
    \loigiai{Hiệu độ âm điện giữa Na và Cl lớn hơn 1.7 nên liên kết trong phân tử NaCl là liên kết ion.}
\end{ex}

%%%%=================EX_09====================%%%
\begin{ex}
Liên kết sigma ($\sigma$) được tạo nên từ:
    \choice
    {Xen phủ bên của hai orbital.}
    {\True Xen phủ trục của hai orbital.}
    {Sự cho và nhận electron giữa hai nguyên tử.}
    {Lực hút tĩnh điện giữa các ion.}
    \loigiai{Liên kết sigma ($\sigma$) được tạo nên từ sự xen phủ trục của hai orbital.}
\end{ex}

%%%%=================EX_10====================%%%
\begin{ex}
Liên kết pi ($\pi$) được tạo nên từ:
    \choice
    {Sự xen phủ trục của hai orbital.}
    {\True Sự xen phủ bên của hai orbital.}
    {Sự cho và nhận electron giữa hai nguyên tử.}
    {Lực hút tĩnh điện giữa các ion.}
    \loigiai{Liên kết pi ($\pi$) được tạo nên từ sự xen phủ bên của hai orbital.}
\end{ex}


%%%%=================TF_01====================%%%
\begin{ex}
	Xét các phát biểu sau:
	\begin{enumerate}
        \item Liên kết cộng hóa trị được hình thành do lực hút tĩnh điện giữa các ion trái dấu.
        \item Trong công thức Lewis, cặp electron dùng chung được biểu diễn bằng dấu chấm.
        \item Liên kết đơn là liên kết được tạo thành từ một cặp electron dùng chung.
	\end{enumerate}
	\choiceTF
	{\True Phát biểu 1 sai.}
	{Phát biểu 2 đúng.}
	{\True Phát biểu 3 đúng.}
	{Cả ba phát biểu đều đúng.}
	\loigiai{Phát biểu 1 sai vì liên kết cộng hóa trị được hình thành bởi một hay nhiều cặp electron dùng chung. Phát biểu 2 sai vì trong công thức Lewis, cặp electron dùng chung được biểu diễn bằng một đường gạch. Phát biểu 3 đúng.}
\end{ex}

%%%%=================TF_02====================%%%
\begin{ex}
	Xét các phát biểu sau:
	\begin{enumerate}
        \item Liên kết đôi gồm 1 liên kết $\sigma$ và 1 liên kết $\pi$.
        \item Liên kết ba gồm 1 liên kết $\sigma$ và 2 liên kết $\pi$.
        \item Liên kết cho nhận là một loại liên kết cộng hóa trị.
	\end{enumerate}
	\choiceTF
	{\True Phát biểu 1 đúng.}
	{\True Phát biểu 2 đúng.}
	{\True Phát biểu 3 đúng.}
	{Cả ba phát biểu đều sai.}
	\loigiai{Cả ba phát biểu đều đúng.}
\end{ex}

%%%%=================TF_03====================%%%
\begin{ex}
	Xét các phát biểu sau:
	\begin{enumerate}
        \item Hiệu độ âm điện càng lớn thì liên kết càng phân cực.
        \item Liên kết ion được hình thành do sự dùng chung cặp electron giữa hai nguyên tử.
        \item Năng lượng liên kết là năng lượng cần thiết để phá vỡ một liên kết xác định.
	\end{enumerate}
	\choiceTF
	{\True Phát biểu 1 đúng.}
	{Phát biểu 2 đúng.}
	{\True Phát biểu 3 đúng.}
	{Cả ba phát biểu đều sai.}
	\loigiai{Phát biểu 2 sai vì liên kết ion được hình thành do lực hút tĩnh điện giữa các ion trái dấu.}
\end{ex}

%%%%=================TF_04====================%%%
\begin{ex}
	Xét các phát biểu sau:
	\begin{enumerate}
        \item Năng lượng liên kết càng lớn thì liên kết càng bền.
        \item Liên kết ba bền hơn liên kết đôi.
        \item Liên kết đơn là loại liên kết kém bền vững nhất.
	\end{enumerate}
	\choiceTF
	{\True Phát biểu 1 đúng.}
	{\True Phát biểu 2 đúng.}
	{\True Phát biểu 3 đúng.}
	{Cả ba phát biểu đều sai.}
	\loigiai{Cả ba phát biểu đều đúng.}
\end{ex}

%%%%=================TF_05====================%%%
\begin{ex}
	Xét các phát biểu sau về phân tử $HF$:
	\begin{enumerate}
        \item Nguyên tử H có 2 electron dùng chung và 0 electron riêng.
        \item Nguyên tử F có 2 electron dùng chung và 6 electron riêng.
        \item Liên kết trong phân tử $HF$ là liên kết cộng hóa trị có cực.
	\end{enumerate}
	\choiceTF
	{\True Phát biểu 1 đúng.}
	{\True Phát biểu 2 đúng.}
	{\True Phát biểu 3 đúng.}
	{Cả ba phát biểu đều sai.}
	\loigiai{Cả ba phát biểu đều đúng.}
\end{ex}


%%%%=================TF_06====================%%%
\begin{ex}
	Xét các phát biểu sau về phân tử $HCl$:
	\begin{enumerate}
        \item Nguyên tử H có 2 electron lớp ngoài cùng.
        \item Nguyên tử Cl có 8 electron lớp ngoài cùng.
        \item Liên kết trong phân tử $HCl$ là liên kết cộng hóa trị có cực.
	\end{enumerate}
	\choiceTF
	{\True Phát biểu 1 đúng.}
	{\True Phát biểu 2 đúng.}
	{\True Phát biểu 3 đúng.}
	{Cả ba phát biểu đều sai.}
	\loigiai{Cả ba phát biểu đều đúng.}
\end{ex}

%%%%=================TF_07====================%%%
\begin{ex}
    Ion florua ($F^-$) bền hơn ion $F^{7+}$.
	\choiceTF[t]
	{\True Đúng}
	{Sai}
	{Không xác định được}
	{Cả hai ion đều không bền}
	\loigiai{Để tạo thành ion $F^{7+}$, cần phải tách 7 electron ra khỏi nguyên tử F, đòi hỏi năng lượng rất lớn. Trong khi đó, để tạo thành ion $F^-$, chỉ cần thêm 1 electron vào nguyên tử F, giải phóng năng lượng. Do đó, ion $F^-$ bền hơn ion $F^{7+}$.}
\end{ex}


%%%%=================TF_08====================%%%
\begin{ex}
    Liên kết giữa $NH_3$ và $H^+$ là liên kết ion.
	\choiceTF[t]
	{Đúng}
	{\True Sai}
	{Không xác định được}
	{Vừa là liên kết ion vừa là liên kết cộng hóa trị}
	\loigiai{Liên kết giữa $NH_3$ và $H^+$ là liên kết cộng hóa trị cho nhận, không phải liên kết ion. Liên kết ion được hình thành bởi lực hút tĩnh điện giữa các ion mang điện tích trái dấu. Trong khi đó, liên kết giữa $NH_3$ và $H^+$ được hình thành do nguyên tử N trong $NH_3$ cho cặp electron chưa liên kết của nó cho $H^+$ dùng chung.}
\end{ex}

%%%%=================TF_09====================%%%
\begin{ex}
    Năng lượng liên kết đơn luôn nhỏ hơn năng lượng liên kết đôi.
	\choiceTF[t]
	{\True Đúng}
	{Sai}
	{Không xác định được}
	{Phụ thuộc vào cặp nguyên tử liên kết}
	\loigiai{Liên kết đôi bao gồm một liên kết $\sigma$ và một liên kết $\pi$, trong khi liên kết đơn chỉ có một liên kết $\sigma$. Do đó, năng lượng cần để phá vỡ liên kết đôi lớn hơn năng lượng cần để phá vỡ liên kết đơn, tức là năng lượng liên kết đôi lớn hơn năng lượng liên kết đơn.}
\end{ex}

%%%%=================TF_10====================%%%
\begin{ex}
    Năng lượng liên kết là năng lượng cần cung cấp để phá vỡ một liên kết hóa học.
	\choiceTF[t]
	{\True Đúng}
	{Sai}
	{Không xác định được}
	{Là năng lượng tỏa ra khi hình thành liên kết hóa học.}
	\loigiai{Năng lượng liên kết là năng lượng cần cung cấp để phá vỡ một liên kết hóa học trong 1 mol chất ở thể khí.}
\end{ex}




%%%%=================BT_01====================%%%
\begin{bt}
Hãy biểu diễn sự hình thành liên kết cộng hóa trị trong phân tử $Cl_2$ từ hai nguyên tử Cl. Viết công thức Lewis của phân tử $Cl_2$.
    \loigiai{Nguyên tử Cl có cấu hình electron lớp ngoài cùng là $3s^23p^5$. Để đạt được cấu hình bền vững của khí hiếm, mỗi nguyên tử Cl cần thêm 1 electron. Do đó, hai nguyên tử Cl sẽ góp chung 1 electron để tạo thành một cặp electron dùng chung, hình thành liên kết cộng hóa trị.
Công thức Lewis của phân tử $Cl_2$: Cl-Cl}
\end{bt}


%%%%=================BT_02====================%%%
\begin{bt}
Viết công thức Lewis của phân tử $HCN$. Xác định loại liên kết giữa các nguyên tử trong phân tử.
    \loigiai{Công thức Lewis của $HCN$: H-C≡N
Liên kết giữa H và C là liên kết đơn.
Liên kết giữa C và N là liên kết ba.}
\end{bt}

%%%%=================BT_03====================%%%
\begin{bt}
Mô tả sự hình thành ion $NH_4^+$ từ phân tử $NH_3$ và ion $H^+$.
    \loigiai{Trong phân tử $NH_3$, nguyên tử N có 1 cặp electron chưa liên kết. Ion $H^+$ không có electron. Nguyên tử N sẽ cho cặp electron chưa liên kết của nó cho $H^+$ dùng chung, tạo thành liên kết cho nhận.
$NH_3 + H^+ \rightarrow NH_4^+$}
\end{bt}

%%%%=================BT_04====================%%%
\begin{bt}
Dựa vào độ âm điện, hãy cho biết liên kết trong phân tử $H_2O$ là liên kết cộng hóa trị có cực hay không phân cực. Biết độ âm điện của H là 2.2 và của O là 3.44.
    \loigiai{Hiệu độ âm điện giữa O và H là $|3.44 - 2.2| = 1.24$. Vì $0.4 < 1.24 < 1.7$ nên liên kết trong phân tử $H_2O$ là liên kết cộng hóa trị có cực.}
\end{bt}


%%%%=================BT_05====================%%%
\begin{bt}
Dự đoán loại liên kết trong phân tử $KF$. Biết độ âm điện của K là 0.82 và của F là 3.98.
    \loigiai{Hiệu độ âm điện giữa F và K là $|3.98 - 0.82| = 3.16$. Vì $3.16 > 1.7$ nên liên kết trong phân tử $KF$ là liên kết ion.}
\end{bt}


%%%%=================BT_06====================%%%
\begin{bt}
So sánh năng lượng liên kết trong phân tử $O_2$ (liên kết đôi) và phân tử $N_2$ (liên kết ba). Giải thích.
    \loigiai{Năng lượng liên kết trong phân tử $N_2$ lớn hơn trong phân tử $O_2$. Liên kết ba trong $N_2$ gồm 1 liên kết $\sigma$ và 2 liên kết $\pi$, trong khi liên kết đôi trong $O_2$ gồm 1 liên kết $\sigma$ và 1 liên kết $\pi$. Do đó, năng lượng cần để phá vỡ liên kết ba trong $N_2$ lớn hơn năng lượng cần để phá vỡ liên kết đôi trong $O_2$.}
\end{bt}
```