\subsubsection{Số oxi hoá}
	\Noibat[\maunhan][][][]{Khái niệm}
	\begin{ghinho}
		\indam[\maunhan]{Số oxi hóa} là điện tích quy ước của nguyên tử trong phân tử khi coi tất cả các electron liên kết đều chuyển hoàn toàn về nguyên tử có độ âm điện lớn hơn. Nó được biểu diễn bằng số đại số, dấu viết trước, số viết sau.
	\end{ghinho}
	\Noibat[\maunhan][][][]{Quy tắc xác định số oxi hoá}
	\begin{ghinho}
		\begin{enumerate}
			\item \indam{Quy tắc 1}
			
			Trong đơn chất, số oxi hóa của nguyên tử bằng 0. Ví dụ: Số oxi hóa của oxy trong $O_2$ là 0; số oxi hóa của sắt trong Fe là 0.
			
			\item \indam{Quy tắc 2}
			
			Trong phân tử hợp chất, thông thường số oxi hóa của :
			\begin{itemize}
				\item Hydrogen là +1,
				\item Oxygen là -2.
				\item Kim loại điển hình có số oxi hóa dương bằng số electron hóa trị. Ví dụ: số oxi hóa của Na là +1, của Ca là +2, của Al là +3.
				\item Flo luôn có số oxi hóa là -1 trong mọi hợp chất.
			\end{itemize}
			\indam{Ngoại lệ:}
			\begin{itemize}
				\item Trong hydrua kim loại, H có số oxi hoá -1 (ví dụ: NaH).
				\item Trong peroxit, O có số oxi hoá -1 (ví dụ: $H_2O_2$).
				\item Trong superoxit, O có số oxi hoá $-\frac{1}{2}$ (ví dụ: $KO_2$).
				\item Trong $OF_2$, O có số oxi hoá +2.
			\end{itemize}
			
			\item \indam{Quy tắc 3}
			
			Trong một phân tử, tổng số oxi hóa của tất cả các nguyên tử bằng 0. Ví dụ: trong $H_2O$, tổng số oxi hóa là $2 \times (+1) + (-2) = 0$.
			
			\item \indam{Quy tắc 4}
			\begin{itemize}
				\item Trong \indam[\maunhan]{ion đơn nguyên tử}, số oxi hóa của nguyên tử bằng điện tích ion. Ví dụ: số oxi hóa của $Na^+$ là +1, của $Cl^-$ là -1.
				\item Trong \indam[\maunhan]{ion đa nguyên tử}, tổng số oxi hóa của tất cả các nguyên tử bằng điện tích ion. Ví dụ: trong $SO_4^{2-}$, tổng số oxi hóa là $(+6) + 4 \times (-2) = -2$.
			\end{itemize}
		\end{enumerate}
	\end{ghinho}
\subsubsection{Chất oxi hoá, chất khử, phản ứng oxi hoá – khử.}
	\Noibat[\maunhan][][][]{Sự oxi hóa-sự khử}
		\begin{tomtat}
			\begin{itemize}
				\item \indam{Sự oxi hóa} là sự nhường electron, là sự tăng số oxi hóa.\\
				\indam{Ví dụ:} $\overset{0}{Mg} \rightarrow Mg^{2+} + 2e$
				\item \indam{Sự khử} là sự thu electron, là sự giảm electron.\\
				\indam{Ví dụ:} $Cu^{2+} +2e \rightarrow Cu $
			\end{itemize}
		\end{tomtat}
	\Noibat[\maunhan][][][]{Chất oxi hóa - chất khử}
		\begin{tomtat}
			\begin{itemize}
				\item \indam{Chất oxi hóa} là chất thu electron.là chất chứa nguyên tố có số oxi hóa giảm sau phản ứng.
				\indam{Chất oxi hóa} còn gọi là chất \indam{bị khử}.
				\item \indam{Chất khử} là chất nhường electron, là chất có số oxi hóa tăng sau phản ứng.
				\indam{Chất khử} còn gọi là chất \indam{bị oxi hóa}.
			\end{itemize}
		\end{tomtat}
	\Noibat[\maunhan][][][]{Phản ứng oxi hoá – khử}
		\begin{tomtat}
			\indam{Phản ứng oxi - hóa khử } là phản ứng hoá học xảy ra đồng thời quá trình nhường và quá trình nhận electron.
			
			Dấu hiệu để nhận biết phản ứng oxi hoá - khử là có sự thay đổi số oxi hoá của các nguyên tử
		\end{tomtat}
\subsubsection{Lập phương trình hoá học của phản ứng oxi hoá – khử.}
	\begin{tomtat}
		\Noibat[\maunhan][][\faStar][]{Nguyên tắc}
		\begin{center}
			\boxct[\mauphu][3pt][\bfseries\sffamily\color{violet}]{Tổng số electron chất khử nhường  = tổng số electron mà chất oxi hóa nhận}
		\end{center}
		\Noibat[\maunhan][][\faStar][]{Các bước cân bằng bằng phươg pháp thăng bằng electron}
		\begin{cacbuoc}
			\item Xác định số oxi hóa của nguyên tử có sự thay đổi số oxi hóa, xác định chất khử, chất oxi hóa
			\item Viết quá trình oxi hóa và quá trình khử
			\item Nhân hệ số thích hợp  vào các quá trình  sao cho tổng  số electron chất khử nhường  bằng tổng số electron chất oxi hóa nhận.
			\item Đặt các hệ số vào sơ đồ phản ứng. Cân bằng số lượng  nguyên tử của các nguyên tố còn lại.
		\end{cacbuoc}
	\end{tomtat}
\subsubsection{Phản ứng oxi hoá - khử trong thực tiễn}
	\Noibat[\maunhan][][\faStar][]{Sự cháy}
	\begin{hopdongian}
		Phản ứng cháy là phản ứng oxi hoá - khử xảy ra ở nhiệt độ cao giữa chất cháy và chất oxi hoá.
		\\
		$C + O_2$ $\xrightarrow[$t^\circ$]$ $CO_2$
		\\
		$2C_4H_{10} + 13O_2$ $\xrightarrow[$t^\circ$]$ $8CO_2 + 10H_2O$
	\end{hopdongian}
	\Noibat[\maunhan][][\faStar][]{Sự han gỉ kim loại}
		\begin{hopdongian}[\mauphu]
			Trong không khí ẩm, các vật dụng bằng thép bị oxi hoá tạo gỉ sắt.
			\\
			$4Fe + 3O_2 + xH_2O$ $\xrightarrow$ $2Fe_2O_3.xH_2O$
		\end{hopdongian}
	\Noibat[\maunhan][][\faStar][]{Sản xuất hóa chất}
		\begin{hopdongian}[\maunhan]
			Trong công nghiệp, phần lớn các phản ứng hoá học xảy ra trong các quy trình sản xuất là phản ứng oxi hoá – khử. \\
			Ví dụ: Sulfuric acid là hoá chất quan trọng trong công nghiệp, được sản xuất chủ yếu từ sulfur hoặc quặng pyrite.
		\end{hopdongian}
	\Noibat[\maunhan][][\faStar][]{Chuyển hoá các chất trong tự nhiên}
		\begin{hopdongian}
			Hiện tượng cây lúa phát triển nhanh khi có những cơn mưa rào đầu tiên kèm theo sấm sét vào khoảng cuối mùa xuân.
			
			Tia sét tạo ra tia lửa điện, là điều kiện cho nitrogen phản ứng với oxygen:
			
			$N_2 + O_2$ $\xrightarrow[$\text{tia lửa điện}$]$ $2NO$
			
			Khí NO sinh ra nhanh chóng chuyển hoá thành $NO_2$, sau đó tiếp tục bị oxi hoá thành $HNO_3$:
			
			$2NO + O_2$ $\xrightarrow$ $2NO_2$
			
			$4NO_2 + O_2 + 2H_2O$ $\xrightarrow$ $4HNO_3$
			
			Nitric acid tan vào nước mưa và chuyển hoá thành gốc nitrate ($NO_3^-$), cung cấp chất đạm cho cây lúa. Nhờ quá trình trên, hàng năm một lượng lớn phân đạm tự nhiên được bổ sung cho đất.
		\end{hopdongian}
	\Noibat[\maunhan][][\faStar][]{Xác định nồng độ một chất bằng phản ứng oxi hoá – khử}
	\begin{hopdongian}[\maunhan]
		Ví dụ: Trong quá trình bảo quản, một mẫu iron(II) sulfate bị oxi hoá một phần thành hợp chất iron(III). Hàm lượng iron(II) sulfate còn lại trong mẫu được xác định thông qua phản ứng với dung dịch thuốc tím có nồng độ đã biết:
		\[
		10 \mathrm{FeSO}_4+2 \mathrm{KMnO}_4+8 \mathrm{H}_2 \mathrm{SO}_4 \longrightarrow 5 \mathrm{Fe}_2\left(\mathrm{SO}_4\right)_3+\mathrm{K}_2 \mathrm{SO}_4+2 \mathrm{MnSO}_4+8\mathrm{H}_2 \mathrm{O}
		\]
	\end{hopdongian}