\newcounter{sode}
\refstepcounter{sode}
\begin{tcolorbox}[
	enhanced jigsaw,
	frame empty,
	colback=\mycolor!10,
	arc is angular,
	arc=3mm
	]
	{\renewcommand{\arraystretch}{0.9} \begin{tabular}{C{0.4\textwidth}p{0.2cm}C{0.6\textwidth}}
			{\textbf{PHÒNG GD VÀ ĐT PHÙ MỸ}} && {\textbf{Đề kiểm tra giữa kì 1 Hóa học 10}}\\
			{\textbf{Trường THPT Số 2 Phù Mỹ}} && {\textbf{Năm học 2024 - 2025}}\\
			{\textbf{Đề số \thesode}} && {\itshape Thời gian làm bài: 60 phút}\\
			Mã đề: \textbf{\made} & & (\itshape không kể thời giưn giao đề)
	\end{tabular}}
\end{tcolorbox}
\subsection{Câu trắc nghiệm nhiều phương án lựa chọn.} Thí sinh trả lời từ câu 1 đến câu 18. Mỗi câu hỏi thí sinh chỉ chọn một phương án.

\Opensolutionfile{ans}[Ans/AnsEX]
\setcounter{ex}{0}
	%%%==============Cau_EX1==============%%%
	\begin{ex}
		Hiện tượng nào sau đây là một phản ứng hóa học?
		\choice
		{Nước đá tan trong nước}
		{Đường tan trong nước}
		{Sắt bị gỉ trong không khí ẩm}
		{Muối ăn tan trong nước}
		\loigiai{}
	\end{ex}
	%%%==============HetCau_EX1==============%%%
	
	%%%==============Cau_EX2==============%%%
	\begin{ex}
		Trong nguyên tử, hạt nào không mang điện?
		\choice
		{Proton}
		{Electron}
		{Neutron}
		{Ion}
		\loigiai{}
	\end{ex}
	%%%==============HetCau_EX2==============%%%
	
	%%%==============Cau_EX3==============%%%
	\begin{ex}
		Đặc điểm nào sau đây mô tả đúng về neutron?
		\choice
		{Mang điện âm và có khối lượng nhỏ}
		{Mang điện dương và có khối lượng lớn}
		{Không mang điện và có khối lượng lớn}
		{Mang điện dương và không có khối lượng}
		\loigiai{}
	\end{ex}
	%%%==============HetCau_EX3==============%%%
	
	%%%==============Cau_EX4==============%%%
	\begin{ex}
		Phát biểu nào sau đây về đồng vị là đúng nhất?
		\choice
		{Đồng vị là những nguyên tử có cùng số proton nhưng khác số neutron}
		{Đồng vị là những nguyên tử có cùng số neutron nhưng khác số proton}
		{Đồng vị là những nguyên tử có cùng số khối nhưng khác số proton}
		{Đồng vị là những nguyên tử có cùng số electron nhưng khác số proton}
		\loigiai{}
	\end{ex}
	%%%==============HetCau_EX4==============%%%
	
	%%%==============Cau_EX5==============%%%
	\begin{ex}
		Orbital nào có dạng hình quả tạ?
		\choice
		{Orbital s}
		{Orbital p}
		{Orbital d}
		{Orbital f}
		\loigiai{}
	\end{ex}
	%%%==============HetCau_EX5==============%%%
	
	%%%==============Cau_EX6==============%%%
	\begin{ex}
		Nguyên tử Silicon ($Z=14$) có số electron hóa trị là:
		\choice
		{$2$}
		{$4$}
		{$6$}
		{$14$}
		\loigiai{}
	\end{ex}
	%%%==============HetCau_EX6==============%%%
	
	%%%==============Cau_EX7==============%%%
	\begin{ex}
		Trong bảng tuần hoàn, các nguyên tố được sắp xếp theo quy luật nào?
		\choice
		{Khối lượng nguyên tử tăng dần}
		{Số neutron tăng dần}
		{Số proton tăng dần}
		{Số electron tăng dần}
		\loigiai{}
	\end{ex}
	%%%==============HetCau_EX7==============%%%
	
	%%%==============Cau_EX8==============%%%
	\begin{ex}
		Ion $M^{3+}$ có cấu hình electron $1s^22s^22p^63s^23p^6$. Vị trí của nguyên tố M trong bảng tuần hoàn là:
		\choice
		{Chu kỳ 4, nhóm IIIA}
		{Chu kỳ 3, nhóm IIIA}
		{Chu kỳ 4, nhóm IIIB}
		{Chu kỳ 3, nhóm IIIB}
		\loigiai{}
	\end{ex}
	%%%==============HetCau_EX8==============%%%
	
	%%%==============Cau_EX9==============%%%
	\begin{ex}
		Trong một chu kỳ của bảng tuần hoàn, theo chiều tăng của điện tích hạt nhân:
		\choice
		{Tính kim loại tăng dần, bán kính nguyên tử giảm dần}
		{Tính phi kim giảm dần, độ âm điện tăng dần}
		{Tính kim loại giảm dần, độ âm điện tăng dần}
		{Tính phi kim tăng dần, bán kính nguyên tử tăng dần}
		\loigiai{}
	\end{ex}
	%%%==============HetCau_EX9==============%%%
	
	%%%==============Cau_EX10==============%%%
	\begin{ex}
		Phát biểu nào sau đây là không đúng?
		\choice
		{Proton và neutron tạo nên hạt nhân nguyên tử}
		{Electron chuyển động xung quanh hạt nhân nguyên tử}
		{Khối lượng của electron bằng 1/1836 khối lượng của proton}
		{Số neutron trong hạt nhân xác định số hiệu nguyên tử}
		\loigiai{}
	\end{ex}
	%%%==============HetCau_EX10==============%%%
	
	%%%==============Cau_EX11==============%%%
	\begin{ex}
		Nguyên tử của nguyên tố X có 12 proton, 13 neutron và 12 electron. Số khối của X là:
		\choice
		{$12$}
		{$13$}
		{$24$}
		{$25$}
		\loigiai{}
	\end{ex}
	%%%==============HetCau_EX11==============%%%
	
	%%%==============Cau_EX12==============%%%
	\begin{ex}
		Từ hai đồng vị $^{79}Br$ (50{,}69 $\%$) và $^{81}Br$ (49{,}31 $\%$), số loại phân tử $Br_2$ có thể được tạo thành là:
		\choice
		{$1$}
		{$2$}
		{$3$}
		{$4$}
		\loigiai{}
	\end{ex}
	%%%==============HetCau_EX12==============%%%
	
	%%%==============Cau_EX13==============%%%
	\begin{ex}
		Số electron tối đa trong các phân lớp s, p, d, f lần lượt là:
		\choice
		{$2$, $6$, $10$, $14$}
		{$2$, $8$, $18$, $32$}
		{$1$, $3$, $5$, $7$}
		{$2$, $6$, $10$, $18$}
		\loigiai{}
	\end{ex}
	%%%==============HetCau_EX13==============%%%
	
	%%%==============Cau_EX14==============%%%
	\begin{ex}
		Trong một nhóm A của bảng tuần hoàn, theo chiều tăng của điện tích hạt nhân:
		\choice
		{Tính kim loại tăng, năng lượng ion hóa giảm}
		{Tính phi kim giảm, độ âm điện tăng}
		{Tính kim loại tăng, bán kính nguyên tử tăng}
		{Tính phi kim tăng, độ âm điện giảm}
		\loigiai{}
	\end{ex}
	%%%==============HetCau_EX14==============%%%
	%%%==============Cau_EX15==============%%%
	\begin{ex}
		Nguyên tử Z có tổng số hạt cơ bản là 82. Trong đó, số hạt mang điện nhiều hơn số hạt không mang điện là 18. Số hiệu nguyên tử của Z là:
		\choice
		{$32$}
		{$33$}
		{$34$}
		{$35$}
		\loigiai{}
	\end{ex}
	%%%==============HetCau_EX15==============%%%
	
	%%%==============Cau_EX16==============%%%
	\begin{ex}
		Nguyên tố Boron có hai đồng vị bền: $^{10}B$ chiếm 19{,}9 $\%$ và $^{11}B$ chiếm 80{,}1 $\%$. Nguyên tử khối trung bình của Boron là:
		\choice
		{10{,}20}
		{10{,}55}
		{10{,}80}
		{11{,}20}
		\loigiai{}
	\end{ex}
	%%%==============HetCau_EX16==============%%%
	
	%%%==============Cau_EX17==============%%%
	\begin{ex}
		Tính chất nào sau đây của nguyên tố hóa học không biến đổi tuần hoàn theo chiều tăng của điện tích hạt nhân?
		\choice
		{Tính kim loại và phi kim}
		{Bán kính nguyên tử}
		{Năng lượng ion hóa thứ nhất}
		{Số neutron}
		\loigiai{}
	\end{ex}
	%%%==============HetCau_EX17==============%%%
	
	%%%==============Cau_EX18==============%%%
	\begin{ex}
		Nguyên tử Calcium ($Z=20$) có cấu hình electron là:
		\choice
		{$1s^22s^22p^63s^23p^64s^2$}
		{$1s^22s^22p^63s^23p^63d^2$}
		{$1s^22s^22p^63s^23p^64s^13d^1$}
		{$1s^22s^22p^63s^23p^64s^14p^1$}
		\loigiai{}
	\end{ex}
	%%%==============HetCau_EX18==============%%%
\Closesolutionfile{ans}

\subsection{Câu trắc nghiệm đúng sai.} Thí sinh trả lời từ câu 1 đến câu 4. Trong mỗi ý a), b), c), d) ở mỗi câu, thí sinh chọn đúng hoặc sai (Đ – S).
\Opensolutionfile{ansbook}[Ansbook/AnsTF]
	%%%==============Cau_EX1==============%%%
	\begin{ex}
		Cho các phát biểu sau:
		\choiceTF
		{Cấu hình electron của nguyên tử Cr (24) là $1s^22s^22p^63s^23p^63d^54s^1$}
		{Hai nguyên tử $^{40}K$ và $^{40}Ca$ là đồng vị của nhau}
		{Hạt nhân nguyên tử mang điện tích dương}
		{Lớp electron liên kết với hạt nhân yếu nhất là lớp trong cùng}
		\loigiai{}
	\end{ex}
	%%%==============HetCau_EX1==============%%%
	
	%%%==============Cau_EX2==============%%%
	\begin{ex}
		Bromine được sử dụng trong sản xuất thuốc trừ sâu và chất chống cháy. Nguyên tử bromine có phân lớp electron ngoài cùng là $4p^5$.
		\choiceTF
		{Hạt nhân nguyên tử bromine có 35 proton}
		{Bromine nằm ở chu kỳ 4 và là một phi kim}
		{Hợp chất của bromine với hydrogen có công thức $HBr_2$}
		{Bromine có độ âm điện lớn hơn nguyên tố chlorine}
		\loigiai{}
	\end{ex}
	%%%==============HetCau_EX2==============%%%
	
	%%%%==============Cau_EX3==============%%%
	\begin{ex}
		Các ion $A^+$ và $B^-$ có cấu hình electron phân lớp ngoài cùng là $3p^6$.
		\choiceTF
			{A phản ứng mạnh với nước ở điều kiện thường}
			{B là chất khí ở điều kiện thường}
			{A thuộc chu kỳ 3, nhóm IA; B thuộc chu kỳ 3, nhóm VIIA}
		    {A là nguyên tố s, B là nguyên tố p}
			\loigiai{}
	\end{ex}
	%%%%==============HetCau_EX3==============%%%
	
	%%%==============Cau_EX4==============%%%
	\begin{ex}
		Sodium (Na) là kim loại mềm, có độ dẫn điện và nhiệt cao, được sử dụng rộng rãi trong đèn chiếu sáng đường phố. Sodium (Na) có kí hiệu nguyên tử $^{23}_{11}Na$
		\choiceTF
		{Số đơn vị điện tích hạt nhân là 11}
		{Số proton và neutron là 23}
		{Số khối là 23}
		{Số neutron là 12}
		\loigiai{}
	\end{ex}
	%%%==============HetCau_EX4==============%%%
\Closesolutionfile{ansbook}

\subsection{Câu trắc nghiệm yêu cầu trả lời ngắn.} Thí sinh trả lời từ câu 1 đến câu 6.
\Opensolutionfile{ansbt}[Ans/AnsSA]
\setcounter{bt}{0}
%%%==============Bai_BT1==============%%%
\begin{bt}
	Brass là hợp kim của đồng với kẽm (65 $\%$ đồng; 35 $\%$ kẽm). Brass được sử dụng rộng rãi trong sản xuất nhạc cụ và đồ trang trí do có màu vàng đẹp và khả năng chống ăn mòn tốt. Bán kính nguyên tử của 2 nguyên tố trên lần lượt là 128 pm và 134 pm. Cho biết bán kính nguyên tử của nguyên tố đồng?
	\shortans{}
	\loigiai{}
\end{bt}
%%%==============HetBai_BT1==============%%%

%%%==============Bai_BT2==============%%%
\begin{bt}
	Nguyên tử nguyên tố P có tổng số hạt mang điện và không mang điện là 46. Trong đó số hạt mang điện nhiều hơn số hạt không mang điện là 14. Cho 0{,}3 mol Oxide của P tác dụng với dung dịch $NaOH$ dư thu được m gam muối. Tìm m?
	\shortans{}
	\loigiai{}
\end{bt}
%%%==============HetBai_BT2==============%%%

%%%==============Bai_BT3==============%%%
\begin{bt}
	Nitrogen (N) là một phi kim có nhiều ứng dụng trong công nghiệp và nông nghiệp. Nitrogen được sử dụng trong sản xuất phân bón, thuốc nổ, và làm môi trường bảo quản thực phẩm. Trong bảng tuần hoàn, nguyên tố N nằm ở chu kỳ 2, nhóm VA. Nguyên tử của nguyên tố N có bao nhiêu electron hóa trị?
	\shortans{}
	\loigiai{}
\end{bt}
%%%==============HetBai_BT3==============%%%

%%%==============Bai_BT4==============%%%
\begin{bt}
	Cho 5 nguyên tố có số hiệu nguyên tử lần lượt là: 11,14,17,18,20. Trong các nguyên tố trên, có bao nhiêu nguyên tố là phi kim?
	\shortans{}
	\loigiai{}
\end{bt}
%%%==============HetBai_BT4==============%%%

%%%==============Bai_BT5==============%%%
\begin{bt}
	Một hợp chất có công thức $AB_4$, trong đó A chiếm 26{,}53 $\%$ về khối lượng. Trong hạt nhân của A và B đều có số proton bằng số neutron. Tổng số proton trong phân tử $AB_4$ là 42. Hợp chất này được sử dụng như một nhiên liệu sạch trong tương lai. Nguyên tố B trong hợp chất trên nằm ở chu kỳ mấy của bảng tuần hoàn?
	\shortans{}
	\loigiai{}
\end{bt}
%%%==============HetBai_BT5==============%%%

%%%==============Bai_BT6==============%%%
\begin{bt}
	Trong số những quá trình kể dưới đây:
	\begin{enumerate}
		\item Sự thăng hoa của iốt rắn.
		\item Sự quang hợp của cây xanh.
		\item Sự nóng chảy của sáp nến.
		\item Sự lên men của nho để sản xuất rượu vang.
		\item Sự bay hơi của cồn.
		\item Sự ăn mòn của nhôm trong không khí ẩm.
		\item Sự kết tủa của bạc clorua khi trộn dung dịch bạc nitrat và natri clorua.
	\end{enumerate}
	Có bao nhiêu hiện tượng hóa học?
	\shortans{}
	\loigiai{}
\end{bt}
%%%==============HetBai_BT6==============%%%
%%%%%=============BT_5=============%%%
%%\begin{bt}%[0D6B3-1]
%%	Biểu thức $ A=\cos20^{\circ} +\cos40^{\circ} + \cos60^{\circ}+\cdots+\cos160^{\circ}+\cos180^{\circ}$ có giá trị bằng\\
%%	\shortans{$-1$}
%%	\loigiai
%%	{Ta có
	%%		\begin{eqnarray*}
		%%			A&=&(\cos20^{\circ}+\cos160^{\circ})+(\cos40^{\circ}+\cos140^{\circ})+\cdots+(\cos80^{\circ}+\cos100^{\circ})+\cos180^{\circ}\\
		%%			&=&(\cos20^{\circ}-\cos20^{\circ})+(\cos40^{\circ}-\cos40^{\circ})+\cdots+(\cos80^{\circ}-\cos80^{\circ})+\cos180^{\circ} \\
		%%			&=& 0+ 0 +\cdots+ 0 -1\\
		%%			&=& -1.
		%%		\end{eqnarray*}
	%%	}
%%\end{bt}
\Closesolutionfile{ansbt}

