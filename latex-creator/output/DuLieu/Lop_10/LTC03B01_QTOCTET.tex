\subsubsection{Liên kết hóa học}
\Noibat[\maunhan][][\faArrowCircleORight][]{Tìm hiểu sự hình thành liên kết hóa học}
	\begin{center}
		\includegraphics[width=12cm]{Images/Tikz/suhinhthanhlienketion.pdf}
		\includegraphics[width=12cm]{Images/Tikz/suhinhthanhlienketF2.pdf}
		\captionof{figure}{Sự hình thành liên kết trong phân tử Sodium chloride và Flourine \label{fig:Suhinhthanhphantu}}
	\end{center}
\begin{hoivadap}
	\begin{cauhoi}
		Dựa vào hình \ref{fig:Suhinhthanhphantu}các em có nhận xét gì về cấu hình e lớp ngoài cùng ( số e xung quanh mỗi nguyên tử) của các nguyên tử sau khi tham gia tạo thành liên kết?
	\end{cauhoi}
	\loigiai{%
		Sau khi tham gia liên kết hóa học các nguyên tử đều có 8 electron ở lớp ngoài cùng
	}
\end{hoivadap}

\begin{hopdongian}
	Theo thuyết cấu tạo hoá học, sự liên kết giữa các nguyên tử tạo thành phân tử hay tinh thể được giải thích bằng sự giảm năng lượng khi các nguyên tử kết hợp lại với nhau. Khi tạo liên kết hoá học thì nguyên tử có xu hướng đạt tới cấu hình electron bền vững của khí hiếm (2 e hoặc 8 e ở lớp ngoài cùng).
\end{hopdongian}
\begin{ghinho}
	\textit{\textbf{Liên kết hoá học} là sự kết hợp giữa các nguyên tử tạo thành phân tử hay tinh thể bền vững hơn.}
\end{ghinho}
\subsubsection{Quy tắc Octet}
\Noibat[\maunhan][][\faArrowCircleORight][]{Tìm hiểu quy tắc octet (bát tử)}
	\begin{ghinho}
	\indam[\maunhan]{Quy tắc octet (bát tử):}
	Trong quá trình hình thành liên kết hoá học, nguyên tử của các nguyên tố nhóm A có xu hướng tạo thành lớp vỏ ngoài cùng có 8 electron tương ứng với khí hiếm gẩn nhất (hoặc 2 electron với khí hiếm helium). 
	\end{ghinho}
\Noibat[\maunhan][][\faArrowCircleORight][]{Tìm hiểu cách vận dụng quy tắc Octet  trong hình thành phân tử Nitrogen ($\textbf{N}_\text{2}$)}
	\begin{center}
		\includegraphics[width=9cm]{Images/Tikz/suhinhthanhlienketN2.pdf}
		\captionof{figure}{Sự hình thành liên kết trong phân tử Nitrogen\label{fig:phantuN2}}
	\end{center}
	\begin{hoivadap}
		Từ hình \ref{fig:phantuN2}, cho biết mối nguyên tử nitrogen đã đạt được cấu hình electron bến vững của nguyên tử khí hiếm nào.
		\loigiai{Mỗi nguyên tử nitrogen đã đạt đuọc cấu hình elctron bền vững của nguyên tố khí hiếm Ne}
	\end{hoivadap}
	\begin{hoivadap}
		Nguyên tử của các nguyên tố hydrogen và fluorine có xu huớng cho đi, nhận thêm hay góp chung các electron hoá trị khi tham gia liên kết hình thành phân tử hydrogen fluoride (HF)?
		\loigiai{Nguyên tử H và F đều cần thêm 1 electron nữa để đạt cấu hình bền của khí hiếm nên mỗi nguyên tử H và F sẽ chia sẻ 1 electron để góp chung}
	\end{hoivadap}
\Noibat[\maunhan][][\faArrowCircleORight][]{Tìm hiểu cách vận dụng quy tắc octet trong sự hình thành ion dương, ion âm}
	\begin{center}
		\includegraphics[width=9cm]{Images/Tikz/suhinhthanhionNa.pdf}
		\captionof{figure}{Sự tạo thành ion $Na^+$}
		\includegraphics[width=9cm]{Images/Tikz/suhinhthanhionF.pdf}
		\captionof{figure}{Sự tạo thành ion $F^-$}
	\end{center}
	\begin{hoivadap}
		Biết phân tử magnesium oxide hình thành bởi các ion $\mathrm{Mg}^{2+}$ và $\mathrm{O}^{2-}$. Vận dụng quy tắc octet, trình bày sự hình thành các ion trên từ những nguyên tử tương ứng.
		\loigiai{%
			\begin{center}
				\begin{tabular}{cccccc}
				&$Mg$&$\rightarrow$&$Mg^{2+}$&$+$&$2e$\\
				Cấu hình electron& $[Ne]3s^2$&$\rightarrow$&$[Ne]:$&&\\
				&$O$&$+$&$2e$&$\rightarrow$&$O^{2-}$\\
				Cấu hình electron& $[He]2s^22p^4$&$\rightarrow$&$[Ne]:$&&\\
		    	\end{tabular}
			\end{center}
		    \\
			Phương trình phân tử: $2Mg + O_2 \rightarrow 2MgO$}
	\end{hoivadap}
