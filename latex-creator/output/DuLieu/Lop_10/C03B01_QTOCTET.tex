\section{Quy tắc octet}
\begin{Muctieu}
	\begin{itemize}
		\item Trình bày được quy tắc octet với các nguyên tố nhóm $A$.
		\item Vận dụng được quy tắc octet trong quá trình hình thành liên kết hoá học ở các nguyên tố nhóm $A$.
	\end{itemize}
\end{Muctieu}
\begin{kd}
	\hinhphai{\lq\lq Hãy quan sát hai hình ảnh bên:
		\begin{enumerate}
			\item  Một quả bóng đang lăn từ đỉnh đồi xuống chân đồi (hình a)
			\item  Electron ở lớp vỏ ngoài cùng của nguyên tử Natri (Na) (hình b)
		\end{enumerate}
		Theo em, hai hiện tượng này có điểm gì giống nhau về mặt xu hướng năng lượng (muốn trở về trạng thái năng lượng thấp hơn hay cao hơn)?\rq\rq}{\tikz{
			\node[name=ball] at (0,0) {\includegraphics[width=4cm]{Images/anhhoahoc10/anhminhoa/ball_on_hill.png}};
			\node[right=0.5cm of ball, name = na] {\includegraphics[width=4cm]{Images/Tikz/Na.pdf}};
			\node [below= 2mm of ball, font=\scriptsize] {\textbf{Hình a}};
			\node [below= 2mm of na,font=\scriptsize] {\textbf{Hình b}};
		}
	}
\end{kd}
\subsection{Nội dung bài học}
	\subsubsection{Liên kết hóa học}
\Noibat[\maunhan][][\faArrowCircleORight][]{Tìm hiểu sự hình thành liên kết hóa học}
	\begin{center}
		\includegraphics[width=12cm]{Images/Tikz/suhinhthanhlienketion.pdf}
		\includegraphics[width=12cm]{Images/Tikz/suhinhthanhlienketF2.pdf}
		\captionof{figure}{Sự hình thành liên kết trong phân tử Sodium chloride và Flourine \label{fig:Suhinhthanhphantu}}
	\end{center}
\begin{hoivadap}
	\begin{cauhoi}
		Dựa vào hình \ref{fig:Suhinhthanhphantu}các em có nhận xét gì về cấu hình e lớp ngoài cùng ( số e xung quanh mỗi nguyên tử) của các nguyên tử sau khi tham gia tạo thành liên kết?
	\end{cauhoi}
	\loigiai{%
		Sau khi tham gia liên kết hóa học các nguyên tử đều có 8 electron ở lớp ngoài cùng
	}
\end{hoivadap}

\begin{hopdongian}
	Theo thuyết cấu tạo hoá học, sự liên kết giữa các nguyên tử tạo thành phân tử hay tinh thể được giải thích bằng sự giảm năng lượng khi các nguyên tử kết hợp lại với nhau. Khi tạo liên kết hoá học thì nguyên tử có xu hướng đạt tới cấu hình electron bền vững của khí hiếm (2 e hoặc 8 e ở lớp ngoài cùng).
\end{hopdongian}
\begin{ghinho}
	\textit{\textbf{Liên kết hoá học} là sự kết hợp giữa các nguyên tử tạo thành phân tử hay tinh thể bền vững hơn.}
\end{ghinho}
\subsubsection{Quy tắc Octet}
\Noibat[\maunhan][][\faArrowCircleORight][]{Tìm hiểu quy tắc octet (bát tử)}
	\begin{ghinho}
	\indam[\maunhan]{Quy tắc octet (bát tử):}
	Trong quá trình hình thành liên kết hoá học, nguyên tử của các nguyên tố nhóm A có xu hướng tạo thành lớp vỏ ngoài cùng có 8 electron tương ứng với khí hiếm gẩn nhất (hoặc 2 electron với khí hiếm helium). 
	\end{ghinho}
\Noibat[\maunhan][][\faArrowCircleORight][]{Tìm hiểu cách vận dụng quy tắc Octet  trong hình thành phân tử Nitrogen ($\textbf{N}_\text{2}$)}
	\begin{center}
		\includegraphics[width=9cm]{Images/Tikz/suhinhthanhlienketN2.pdf}
		\captionof{figure}{Sự hình thành liên kết trong phân tử Nitrogen\label{fig:phantuN2}}
	\end{center}
	\begin{hoivadap}
		Từ hình \ref{fig:phantuN2}, cho biết mối nguyên tử nitrogen đã đạt được cấu hình electron bến vững của nguyên tử khí hiếm nào.
		\loigiai{Mỗi nguyên tử nitrogen đã đạt đuọc cấu hình elctron bền vững của nguyên tố khí hiếm Ne}
	\end{hoivadap}
	\begin{hoivadap}
		Nguyên tử của các nguyên tố hydrogen và fluorine có xu huớng cho đi, nhận thêm hay góp chung các electron hoá trị khi tham gia liên kết hình thành phân tử hydrogen fluoride (HF)?
		\loigiai{Nguyên tử H và F đều cần thêm 1 electron nữa để đạt cấu hình bền của khí hiếm nên mỗi nguyên tử H và F sẽ chia sẻ 1 electron để góp chung}
	\end{hoivadap}
\Noibat[\maunhan][][\faArrowCircleORight][]{Tìm hiểu cách vận dụng quy tắc octet trong sự hình thành ion dương, ion âm}
	\begin{center}
		\includegraphics[width=9cm]{Images/Tikz/suhinhthanhionNa.pdf}
		\captionof{figure}{Sự tạo thành ion $Na^+$}
		\includegraphics[width=9cm]{Images/Tikz/suhinhthanhionF.pdf}
		\captionof{figure}{Sự tạo thành ion $F^-$}
	\end{center}
	\begin{hoivadap}
		Biết phân tử magnesium oxide hình thành bởi các ion $\mathrm{Mg}^{2+}$ và $\mathrm{O}^{2-}$. Vận dụng quy tắc octet, trình bày sự hình thành các ion trên từ những nguyên tử tương ứng.
		\loigiai{%
			\begin{center}
				\begin{tabular}{cccccc}
				&$Mg$&$\rightarrow$&$Mg^{2+}$&$+$&$2e$\\
				Cấu hình electron& $[Ne]3s^2$&$\rightarrow$&$[Ne]:$&&\\
				&$O$&$+$&$2e$&$\rightarrow$&$O^{2-}$\\
				Cấu hình electron& $[He]2s^22p^4$&$\rightarrow$&$[Ne]:$&&\\
		    	\end{tabular}
			\end{center}
		    \\
			Phương trình phân tử: $2Mg + O_2 \rightarrow 2MgO$}
	\end{hoivadap}

\subsection{Các dạng bài tập}
	\phan{Trắc nghiệm nhiều lựa chọn}
%%%=============SOẠN EX===============%%%
\Opensolutionfile{ansex}[Ans/LGEX-C03B01_QTOCTET_01.tex]
\Opensolutionfile{ans}[Ans/Ans-C03B01_QTOCTET_01.tex]
%%%=============EX_1=============%%%
\begin{ex}
	Công thức cấu tạo nào sau đây không đủ electron theo quy tắc octet?
	\choice
	{\setcharge{{.style={fill=\mycolor!50!black,draw=none}}}
		\chemfig{
			H
			-[,0.48,,,draw=none]
			\charge{[.radius=0.2ex]
				90:2pt=\:,
				180:2pt=\:,
				-90:2pt=\:,
				0:2pt=\:
			}{N}
			(-[-90,0.48,,,draw=none]H)-[,0.48,,,draw=none]H
		}
		\resetcharge}
	%%%
	{\True \setcharge{{.style={fill=\mycolor!50!black,draw=none}}}
		\chemfig{
			H
			-[,0.48,,,draw=none]
			\charge{[.radius=0.2ex]
				180:2pt=\:,
				-90:2pt=\:,
				0:2pt=\:
			}{B}
			(-[-90,0.48,,,draw=none]H)-[,0.48,,,draw=none]H
		}
		\resetcharge}
	%%%
	{\setcharge{{.style={fill=\mycolor!50!black,draw=none}}}
		\chemfig{
			\charge{[.radius=0.2ex]
				0:0.4pt=\:,
				120:0.5pt=\:,
				-120:0.5pt=\:
			}{O}
			-[,0.51,,,draw=none]
			\charge{[.radius=0.2ex]
				180:2pt=\:,
				0:2pt=\:
			}{C}
			-[,0.51,,,draw=none]
			\charge{[.radius=0.2ex]
				180:0.4pt=\:,
				60:0.5pt=\:,
				-60:0.5pt=\:
			}{O}
		}
		\resetcharge}
	%%%
	{\setcharge{{.style={fill=\mycolor!50!black,draw=none}}}
		\chemfig{
			\charge{[.radius=0.2ex]
				0:0.5pt=\:,
				180:0.5pt=\:,
				90:0.5pt=\:,
				-90:0.5pt=\:
			}{Cl}
			-[,0.45,,,draw=none]
			\charge{[.radius=0.2ex]
				0:0.5pt=\:,
				90:0.5pt=\:,
				-90:0.5pt=\:
			}{Cl}
		}
		\resetcharge}
	%%%
	\loigiai{Trong BH$_3$, nguyên tử B chỉ có 6 electron lớp ngoài cùng, chưa đạt octet.}
\end{ex}
%%%=============EX_2=============%%%
\begin{ex}
	Liên kết hoá học là
	\choice
	{sự kết hợp của các hạt cơ bản hình thành nguyên tử bền vững}
	{\True sự kết hợp giữa các nguyên tử tạo thành phân tử hay tinh thể bền vững hơn}
	{sự kết hợp của các phân tử hình thành các chất bền vững}
	{sự kết hợp của chất tạo thành vật thể bền vững}
	\loigiai{Liên kết hóa học là sự kết hợp giữa các nguyên tử tạo thành phân tử hay tinh thể bền vững hơn. Sự kết hợp này làm giảm năng lượng của hệ, tạo ra hệ bền vững hơn.}
\end{ex}
%%%=============EX_3=============%%%
\begin{ex}
	Theo quy tắc octet, khi hình thành liên kết hoá học, các nguyên tử có xu hướng nhường, nhận hoặc góp chung electron để đạt tới cấu hình electron bền vững giống như
	\choice
	{kim loại kiềm gần kề}
	{kim loại kiềm thổ gần kề}
	{nguyên tử halogen gần kề}
	{\True nguyên tử khí hiếm gần kề}
	\loigiai{Theo quy tắc octet, khi hình thành liên kết hoá học, các nguyên tử có xu hướng nhường, nhận hoặc góp chung electron để đạt tới cấu hình electron bền vững của khí hiếm gần nhất.}
\end{ex}
%%%=============EX_4=============%%%
\begin{ex}
	Khi hình thành liên kết hoá học, nguyên tử có số hiệu nào sau đây có xu hướng nhường 2 electron để đạt cấu hình electron bền vững theo quy tắc octet?
	\choice
	{\True $(Z=12)$}
	{$(Z=9)$}
	{$(Z=11)$}
	{$(Z=10)$}
	\loigiai{Nguyên tử có Z = 12 (Mg) có cấu hình electron là [Ne]3s$^2$. Khi hình thành liên kết hóa học, Mg có xu hướng nhường 2 electron để đạt cấu hình electron bền vững của khí hiếm Ne.}
\end{ex}
%%%=============EX_5=============%%%
\begin{ex}
	Trong công thức $CS_2$, tổng số cặp electron lớp ngoài cùng của C và S chưa tham gia liên kết là
	\choice
	{$2$}
	{$3$}
	{\True $4$}
	{$5$}
	\loigiai{Cấu hình electron của C là [He]2s$^2$2p$^2$ (2 cặp electron chưa liên kết). Cấu hình electron của S là [Ne]3s$^2$3p$^4$ (2 cặp electron chưa liên kết). Vậy tổng số cặp electron lớp ngoài cùng của C và 2 nguyên tử S chưa tham gia liên kết là 2 + 2$\times$2 = 4.}
\end{ex}
%%%=============EX_6=============%%%
\begin{ex}
	Phân tử nào sau đây có các nguyên tử đều đã đạt cấu hình electron bão hoà theo quy tắc octet?
	\choice
	{$\mathrm{BeH}_2$}
	{$\mathrm{AlCl}_3$}
	{$\mathrm{PCl}_5$}
	{\True $\mathrm{SiF}_4$}
	\loigiai{Trong phân tử SiF$_4$, nguyên tử Si có 8 electron lớp ngoài cùng và 4 nguyên tử F đều có 8 electron lớp ngoài cùng.}
\end{ex}
%%%=============EX_7=============%%%
\begin{ex}
	Quy tắc octet không đúng với trường hợp phân tử chất nào sau đây?
	\choice
	{$H_2O$}
	{\True $NO_2$}
	{$CO_2$}
	{$\mathrm{Cl}_2$}
	\loigiai{Phân tử NO$_2$ có nguyên tử N có 7 electron lớp ngoài cùng, không tuân theo quy tắc octet.}
\end{ex}
%%%=============EX_8=============%%%
\begin{ex}
	Vì sao các nguyên tử lại liên kết với nhau thành phân tử?
	\choice
	{\True Để mỗi nguyên tử trong phân tử đạt được cơ cấu electron ổn định, bền vững}
	{Đề mỗi nguyên tử trong phân tử đều đạt 8 electron ở lớp ngoài cùng}
	{Để tổng số electron ngoài cùng của các nguyên tử trong phân tử là 8}
	{Để lớp ngoài củng của mỗi nguyên tử trong phân tử có nhiều electron độc thân nhất}
	\loigiai{Các nguyên tử liên kết với nhau thành phân tử để đạt được cấu hình electron bền vững hơn, làm giảm năng lượng của hệ.}
\end{ex}
%%%=============EX_9=============%%%
\begin{ex}
	Nguyên tử nào sau đây có khuynh hướng đạt cấu hình electron bền của khí hiếm neon khi tham gia hình thành liên kết hoá học?
	\choice
	{\True Chlorine}
	{Sulfur}
	{Oxygen}
	{Hydrogen}
	\loigiai{Nguyên tử chlorine (Cl) có 7 electron lớp ngoài cùng. Khi tham gia liên kết hóa học, Cl có xu hướng nhận thêm 1 electron để đạt cấu hình electron bền vững của khí hiếm neon.}
\end{ex}
%%%=============EX_10=============%%%
\begin{ex}
	Sodium hydride $(\mathrm{NaH})$ là một hợp chất được sử dụng như một chất lưu trữ hydrogen trong các phương tiện chạy bằng pin nhiên liệu do khả năng giải phóng hydrogen của nó. Trong sodium hydride, nguyên tử sodium có cấu hình electron bền của khí hiếm
	\choice
	{helium}
	{argon}
	{krypton}
	{\True neon}
	\loigiai{Sodium (Na) có cấu hình electron [Ne]3s$^1$. Khi tham gia liên kết hóa học, Na có xu hướng nhường 1 electron để đạt cấu hình electron bền vững của khí hiếm neon.}
\end{ex}
%%%=============EX_11=============%%%
\begin{ex}
	Khi tham gia hình thành liên kết hoá học, các nguyên tử lithium và chlorine có khuynh hướng đạt cấu hình electron bền của lần lượt các khí hiếm nào dưới đây?
	\choice
	{Helium và argon}
	{\True Helium và neon}
	{Neon và argon}
	{Argon và helium}
	\loigiai{Li có cấu hình electron [He]2s$^1$ có xu hướng nhường 1e để đạt cấu hình của He. Cl có cấu hình electron [Ne]3s$^2$3p$^5$ có xu hướng nhận 1e để đạt cấu hình của Ne.}
\end{ex}
%%%=============EX_12=============%%%
\begin{ex}
	Trong phân tử HBr, nguyên tử hydrogen và bromine đã lần lượt đạt cấu hình electron bền của các khi hiếm nào dưới đây?
	\choice
	{Neon và argon}
	{\True Helium và krypton}
	{Helium và radon}
	{Helium và argon}
	\loigiai{Trong phân tử HBr, H đạt cấu hình bền của He, Br đạt cấu hình bền của Kr.}
\end{ex}
%%%=============EX_13=============%%%
\begin{ex}
	Trong các hợp chất, nguyên tử magnesium đã đạt được cấu hình bền của khí hiếm gần nhất bằng cách
	\choice
	{\True cho đi 2 electron}
	{nhận vào 1 electron}
	{cho đi 3 electron}
	{nhận vào 2 electron}
	\loigiai{Magnesium (Mg) có cấu hình electron [Ne]3s$^2$. Trong các hợp chất, Mg đạt cấu hình bền vững của khí hiếm neon bằng cách nhường đi 2 electron.}
\end{ex}
%%%=============EX_14=============%%%
\begin{ex}
	Cho các phân tử sau: $\mathrm{Cl}_2, H_2O, \mathrm{NaF}$ và $CH_4$. Có bao nhiêu nguyên tử trong các phân tử trên đạt cấu hình electron bền của khi hiếm neon?
	\choice
	{\True $3$}
	{$2$}
	{$5$}
	{$4$}
	\loigiai{
		\begin{itemize}
			\item Trong $\mathrm{Cl}_2$:
			\begin{itemize}
				\item Mỗi nguyên tử Cl $(Z=17)$ có cấu hình $[Ne]3s^23p^5$
				\item Khi liên kết, mỗi Cl dùng chung 1 electron $\rightarrow$ đạt cấu hình $[Ar]$ chứ không phải $[Ne]$
			\end{itemize}
			\item Trong $\mathrm{H_2O}$:
			\begin{itemize}
				\item O $(Z=8)$ có cấu hình $1s^22s^22p^4$
				\item Khi liên kết với 2H, O nhận thêm 2 electron $\rightarrow$ đạt cấu hình $[Ne]$
			\end{itemize}
			\item Trong $\mathrm{NaF}$:
			\begin{itemize}
				\item F $(Z=9)$ nhận 1 electron từ Na $\rightarrow$ đạt cấu hình $[Ne]$
				\item Na $(Z=11)$ cho đi 1 electron $\rightarrow$ đạt cấu hình $[Ne]$
			\end{itemize}
			\item Trong $\mathrm{CH_4}$:
			\begin{itemize}
				\item C $(Z=6)$ có cấu hình $1s^22s^22p^2$
				\item Khi liên kết với 4H, C dùng chung 4 electron $\rightarrow$ đạt cấu hình $[Ne]$
			\end{itemize}
			
		\end{itemize}
		Vậy có 3 nguyên tử đạt cấu hình electron của Ne là: O (trong $\mathrm{H_2O}$), F và Na (trong $\mathrm{NaF}$).}
\end{ex}
%%%=============EX_15=============%%%
\begin{ex}
	Nguyên tử trong phân tử nào dưới đây ngoại lệ với quy tắc octet?
	\choice
	{$H_2O$}
	{$NH_3$}
	{HCl}
	{\True $BF_3$}
	\loigiai{Trong $BF_3$, nguyên tử B chỉ có 6 electron ở lớp ngoài cùng.}
\end{ex}
%%%=============EX_16=============%%%
\begin{ex}
	Nguyên tử oxygen $(Z=8)$ có xu hướng nhường hay nhận bao nhiêu electron để đạt lớp vỏ thoả mãn quy tắc octet? Chọn phương án đúng.
	\choice
	{Nhường $6$ electron}
	{\True Nhận $2$ electron}
	{Nhường $8$ electron}
	{Nhận $6$ electron}
	\loigiai{%
	Oxygen có cấu hình e lớp ngoài cùng là $2s^22p^4$ $\Rightarrow$ có 6 e lớp ngoài cùng , theo quy tắc octet Oxygen có xu hướng nhận thêm 2 e để trở thành cấu hình bền của khí hiếm
	}
\end{ex}
%%%=============EX_17=============%%%
\begin{ex}
	Nguyên tử lithium $(Z=3)$ có xu hướng nhường hay nhận bao nhiêu electron để lớp vỏ thoả mãn quy tắc octet? Chọn phương án đúng.
	\choice
	{\True Nhường $1$ electron}
	{Nhận $7$ electron}
	{Nhường $11$ electron}
	{Nhận $1$ electron}
	\loigiai{
	Lithium có cấu hình e là $1s^22s^1$ $\Rightarrow$ có 1 e lớp ngoài cùng , theo quy tắc octet Oxygen có xu hướng nhường đi 1 e để đạt cấu hình e bền của khí hiếm He $1s^2$.
	}
\end{ex}
%%%=============EX_18=============%%%
\begin{ex}
	Nguyên tử nào sau đầy có thể nhường hoặc nhận $4$ electron để đạt cấu hình electron bền vững?
	\choice
	{\True Silicon}
	{Beryllium}
	{Nitrogen}
	{Selenium}
	\loigiai{%
		Silicon có cấu hình e là $1s^22s^22p^63s^23p^2$ $\Rightarrow$ có 4 e lớp ngoài cùng , theo quy tắc octet Oxygen có xu hướng nhường đi 4 e để đạt cấu hình e bền của khí hiếm Ne $1s^22s^22p^6$ hoặc cũng có thể nhận thêm 4 e để đạt cấu hình e bền của khí hiếm Ar $1s^22s^22p^63s^23p^6$.
	}
\end{ex}
%%%=============EX_19=============%%%
\begin{ex}
	Nguyên tử nào sau đây không có xu hướng nhường hoặc nhận electron để đạt được lớp vỏ thoả mãn quy tắc octet?
	\choice
	{Nitrogen}
	{Oxygen}
	{Sodium}
	{\True Hydrogen}
	\loigiai{%
	Hydrogen không thể đạt được lớp vỏ thoả mãn quy tắc octet mà chỉ có thể đạt được lớp vỏ của khí hiếm gần nó nhất là helium (2 electron).
	}
\end{ex}
%%%=============EX_20=============%%%
\begin{ex}
	Nguyên tử nào trong các nguyên tử sau đây không có xu hướng nhường electron để đạt lớp vỏ thoả mãn quy tắc octet?
	\choice
	{Calcium}
	{Magnesium}
	{Potassium}
	{\True Chlorine}
	\loigiai{%
	Chlorine có 17 e và có 7 e lớp ngoài cùng nên có xu hướng nhận thêm 1 e để đạt cấu hình bền của khí hiếm Ar.
	}
\end{ex}
%%%=============EX_21=============%%%
\begin{ex}
	Mô hình mô tả quá trình tạo liên kết hoá học sau đây phù hợp với xu hướng tạo liên kết hoá học của nguyên tử nào?
	\begin{center}
		\includegraphics[height=3cm]{Images/Tikz/xuhuongnhane_ofphotphorus.pdf}
	\end{center}
	\choice
	{Aluminium}
	{Nitrogen}
	{\True Phosphorus}
	{Oxygen}
	\loigiai{%
	Dựa vào hình vẽ ta thấy nguyên tử có 15 e  và có 5 lớp ngoài cùng và nhận thêm 3 electron để đạt cấu hình e bền của khí hiếm Ar $\Rightarrow$ đây là nguyên tử Phosphorus.
	}
\end{ex}
%%%=============EX_22=============%%%
\begin{ex}
	Nguyên tử có mô hình cấu tạo sau đây có xu hướng nhường hoặc nhận electron như thế nào khi hình thành liên kết hoá học?
	\begin{center}
		\includegraphics[height=4cm]{Images/Tikz/xuhuongnhane_ofpotasium.pdf}
	\end{center}
	\choice[2]
	{Nhận 1 electron}
	{\True Nhường 1 electron}
	{Nhận 7 electron}
	{Không có xu hướng nhường hoặc nhận electron}
	\loigiai{%
	Nguyen tử có 19 electron và 1 electron ở lớp ngoài cùng nên có xu hướng nhường đi 1 electron để đạt cấu hình bền của khí hiếm Ar
	}
\end{ex}
%%%=============EX_23=============%%%
\begin{ex}
	Nguyên tử có mô hình cấu tạo sau sẽ có xu hướng tạo thành ion mang điện tích nào khi nó thoả mãn quy tắc octet?
	\begin{center}
		\includegraphics[height=3cm]{Images/Tikz/xuhuongnhane_ofAluminium.pdf}
	\end{center}
	\choice
	{\True $3+$}
	{$5+$}
	{3-}
	{$5-$}
	\loigiai{%
	Dựa vào hình vẽ ta thấy nguyên tử có 13 electron và có 3 electron ở lớp ngoài cùng nên có xu hướng nhường đi 3 e để đạt cấu hình bền của khí hiếm Ne. Khi nguyên tử mất đi 3 e sẽ trở thành cation mang điện tích $+3$.
	}
\end{ex}
%%%=============EX_24=============%%%
\begin{ex}
	Phân tử nào sau đây KHÔNG tuân theo quy tắc octet?
	\choice
	{$H_2O$}
	{$CO_2$}
	{\True $BH_3$}
	{$CH_4$}
	\loigiai{Trong $BH_3$, nguyên tử B chỉ có 6 electron ở lớp ngoài cùng, không đạt được octet.}
\end{ex}
%%%=============EX_25=============%%%
\begin{ex}
	Nguyên tử F (Z=9) khi nhận thêm 1 electron sẽ có cấu hình electron giống với:
	\choice
	{$O^{2-}$}
	{$Na^+$}
	{\True Ne}
	{He}
	\loigiai{Khi F nhận thêm 1 electron, nó sẽ có 10 electron, giống với cấu hình electron của khí hiếm Ne.}
\end{ex}
%%%=============EX_26=============%%%
\begin{ex}
	Để đạt được cấu hình bền vững của khí hiếm, nguyên tử Al (Z=13) có xu hướng:
	\choice
	{Nhận 3 electron}
	{\True Nhường 3 electron}
	{Góp chung 3 electron}
	{Không tham gia liên kết}
	\loigiai{Al thuộc nhóm IIIA, có 3 electron ở lớp ngoài cùng. Để đạt cấu hình bền vững, Al có xu hướng nhường 3 electron, tạo ra ion $Al^{3+}$.}
\end{ex}
%%%=============EX_27=============%%%
\begin{ex}
	Trong ion $F^-$, số electron ở lớp ngoài cùng là:
	\choice
	{7 electron}
	{\True 8 electron}
	{9 electron}
	{10 electron}
	\loigiai{Ion $F^-$ được tạo thành khi nguyên tử F nhận thêm 1 electron. Lúc này, $F^-$ có 8 electron ở lớp ngoài cùng.}
\end{ex}
%%%=============EX_28=============%%%
\begin{ex}
	Những nguyên tử nào sau đây có xu hướng nhường electron?
	\choice
	{\True Na, K, Ca}
	{F, Cl, Br}
	{N, P, As}
	{O, S, Se}
	\loigiai{%
		$Na, K, Ca $là các kim loại kiềm và kiềm thổ, có xu hướng nhường electron để đạt cấu hình khí hiếm.}
\end{ex}
%%%=============EX_29=============%%%
\begin{ex}
	Trong phân tử $O_2$, số cặp electron dùng chung là:
	\choice
	{1 cặp}
	{\True 2 cặp}
	{3 cặp}
	{4 cặp}
	\loigiai{Trong phân tử $O_2$, hai nguyên tử O góp chung 2 cặp electron để đạt cấu hình bền vững.}
\end{ex}
%%%=============EX_30=============%%%
\begin{ex}
	Khi tham gia liên kết hóa học, nguyên tử H có xu hướng:
	\choice
	{Nhường electron}
	{Nhận electron}
	{\True Góp chung electron}
	{Không tham gia liên kết}
	\loigiai{Nguyên tử H có 1 electron. Khi tham gia liên kết hóa học, nó thường góp chung electron này để đạt cấu hình bền vững của He.}
\end{ex}
%%%=============EX_31=============%%%
\begin{ex}
	Trong các phân tử sau, phân tử nào tuân theo quy tắc octet?
	\choice
	{$PCl_5$}
	{$SF_6$}
	{\True $H_2O$}
	{NO}
	\loigiai{Trong $H_2O$, cả O và H đều đạt được cấu hình bền vững theo quy tắc octet.}
\end{ex}
%%%=============EX_32=============%%%
\begin{ex}
	Ion $Na^+$ có cấu hình electron giống với khí hiếm nào?
	\choice
	{He}
	{\True Ne}
	{Ar}
	{Kr}
	\loigiai{Khi mất 1 electron, $Na^+$ có 10 electron, giống với cấu hình electron của Ne.}
\end{ex}
%%%=============EX_33=============%%%
\begin{ex}
	Nguyên tử của nguyên tố nhóm VIIA (17) có xu hướng:
	\choice
	{Nhường 7 electron}
	{\True Nhận 1 electron}
	{Góp chung 7 electron}
	{Không tham gia liên kết}
	\loigiai{Các nguyên tố nhóm VIIA có 7 electron ở lớp ngoài cùng, chúng có xu hướng nhận 1 electron để đạt cấu hình khí hiếm.}
\end{ex}
%%%=============EX_34=============%%%
\begin{ex}
	Trong phân tử $CH_4$, nguyên tử C:
	\choice
	{Nhường 4 electron}
	{Nhận 4 electron}
	{\True Góp chung 4 electron}
	{Không tham gia liên kết}
	\loigiai{Trong phân tử $CH_4$, nguyên tử C góp chung 4 electron với 4 nguyên tử H để tạo thành 4 liên kết C-H.}
\end{ex}
%%%=============EX_35=============%%%
\begin{ex}
	Quy tắc octet không áp dụng cho:
	\choice
	{Các nguyên tố nhóm A}
	{Các nguyên tố khí hiếm}
	{Các nguyên tố phi kim}
	{\True Các nguyên tố nhóm B}
	\loigiai{Quy tắc octet thường không áp dụng cho các nguyên tố nhóm B (nguyên tố chuyển tiếp) vì chúng có phân lớp d và f tham gia liên kết, có thể tạo ra nhiều loại liên kết phức tạp hơn.}
\end{ex}
%%%=============EX_36=============%%%
\begin{ex}
	Vì sao các nguyên tử lại liên kết với nhau thành phân tử?
	\choice
	{\True Để mối nguyên tử trong phân tử đạt được cơ cấu electron ổn định, bền vững}
	{Để mỗi nguyên tử trong phân tử đều đạt 8 electron ở lớp ngoài cùng}
	{Để tổng số electron ngoài cùng của các nguyên tử trong phân tử là 8 }
	{Để lớp ngoài cúng của mỗi nguyên tử trong phân tử có nhiều electron độc thân nhất}
	\loigiai{}
\end{ex}

%%%=============EX_37=============%%%
\begin{ex}
	Nguyên tử nào sau đây có khuynh hướng đạt cấu hình electron bền của khí hiếm neon khi tham gia hình thành liên kêt hoá học?
	\choice
	{Chlorine}
	{Sulfur}
	{\True Oxygen}
	{Hydrogen}
	\loigiai{}
\end{ex}

%%%=============EX_38=============%%%
\begin{ex}
	Sodium hydride $(\mathrm{NaH})$ là một hợp chất được sử dụng như một chất lưu trữ hydrogen trong các phương tiện chạy bằng pin nhiên liệu do khả năng giải phóng hydrogen của nó. Trong sodium hydride, nguyên tử sodium có cấu hình electron bền của khí hiếm
	\choice
	{helium}
	{argon}
	{krypton}
	{\True neon}
	\loigiai{}
\end{ex}

%%%=============EX_39=============%%%
\begin{ex}
	Khi tham gia hình thành liên kết hoá học, các nguyên tử lithium và chlorine có khuynh hướng đạt cấu hình electron bền của lần lượt các khí hiếm nào dưới đây?
	\choice
	{Helium và argon}
	{Helium và neon}
	{Neon và argon}
	{\True Argon và helium}
	\loigiai{}
\end{ex}

%%%=============EX_40=============%%%
\begin{ex}
	Trong phân tử HBr , nguyên tử hydrogen và bromine đã lần lượt đạt cấu hình electron bền của các khí hiếm nào dưới đây?
	\choice
	{Neon và argon}
	{Helium và xenon}
	{Helium và radon}
	{\True Helium và krypton}
	\loigiai{}
\end{ex}

%%%=============EX_41=============%%%
\begin{ex}
	Trong các hợp chất, nguyên tử magnesium đã đạt được cấu hình bền của khí hiếm gần nhất bằng cách
	\choice
	{\True cho đi 2 electron}
	{nhận vào 1 electron}
	{cho đi 3 electron}
	{nhận vào 2 electron}
	\loigiai{Trong quá trình hình thành phân tử magnesium oxide MgO , nguyên tử magnesium đã đạt được cấu hình bền của khí hiếm gần nhất bằng cách cho đi 2 electron.}
\end{ex}

%%%=============EX_42=============%%%
\begin{ex}
	Cho các phân tử sau: $\mathrm{Cl}_2, \mathrm{H}_2 \mathrm{O}, \mathrm{NaF}$ và $\mathrm{CH}_4$. Có bao nhiêu nguyên tử trong các phân tử trên đạt cấu hình electron bền của khí hiếm neon?
	\choice
	{3 }
	{2}
	{5 }
	{\True 4 }
	\loigiai{Có 4 nguyên tử trong các phân tử đã cho đạt cấu hình electron bền của khí hiếm neon là $\mathrm{O}, \mathrm{Na}, \mathrm{F}$ và C .}
\end{ex}

%%%=============EX_43=============%%%
\begin{ex}
	Nguyên tử trong phân tử nào dưới đây ngoại lệ với quy tắc octet?
	\choice
	{$\mathrm{H}_2 \mathrm{O}$}
	{$\mathrm{NH}_3$}
	{HCl }
	{\True $\mathrm{BF}_3$}
	\loigiai{Trong phân tử $\mathrm{BF}_3$, nguyên tử B mới chỉ có 6 electron ở lớp ngoài cùng, chưa đạt được cơ cấu bền của khí hiếm gần nhất.}
\end{ex}
%%%=============EX_44=============%%%
\begin{ex}
	Theo quy tắc octet, xu hướng chung của các nguyên tử nguyên tố nhóm IA là nhường
	\choice
	{2 electron}
	{3 electron}
	{\True 1 electron}
	{4 electron}
	\loigiai{}
\end{ex}

%%%=============EX_45=============%%%
\begin{ex}
	Để thỏa mãn quy tắc octet, nguyên tử chlorine ($Z=17$) có xu hướng
	\choice
	{nhường 1 electron}
	{\True nhận 1 electron}
	{nhường 3 electron}
	{nhận 3 electron}
	\loigiai{}
\end{ex}

%%%=============EX_46=============%%%
\begin{ex}
	Khi hình thành liên kết hóa học, nguyên tử có số hiệu nào sau đây có xu hướng nhường 2 electron để đạt
	cấu hình electron bền vững theo quy tắc octet?
	\choice
	{$Z=11$}
	{$Z=9$}
	{\True $Z=12$}
	{$Z=10$}
	\loigiai{}
\end{ex}

%%%=============EX_47=============%%%
\begin{ex}
	Theo quy tắc octet nguyên tử nào sau đây nhận 1 electron để đạt cấu trúc ion bền?
	\choice
	{X ($Z=8$)}
	{\True Y ($Z=9$)}
	{T ($Z=11$)}
	{Q ($Z=12$)}
	\loigiai{}
\end{ex}

%%%=============EX_48=============%%%
\begin{ex}
	Quy tắc octet \textbf{không đúng} với trường hợp phân tử chất nào sau đây?
	\choice
	{$H_2O$}
	{\True$NO_2$}
	{$CO_2$}
	{$Cl_2$}
	\loigiai{}
\end{ex}

%%%=============EX_49=============%%%
\begin{ex}
	Phân tử nào dưới đây các nguyên tử liên kết \textbf{không} tuân theo quy tắc octet?
	\choice
	{$H_2O$}
	{$NH_3$}
	{$CH_4$}
	{\True $NO$}
	\loigiai{}
\end{ex}

%%%=============EX_50=============%%%
\begin{ex}
	Quy tắc octet không đúng với phân tử nào sau đây?
	\choice
	{$H_2O$}
	{$NH_3$}
	{$CO_2$}
	{\True $PCl_5$}
	\loigiai{}
\end{ex}
\Closesolutionfile{ans}
\Closesolutionfile{ansex}
%\bangdapan{Ans-C03B01_QTOCTET_01.tex}
\phan{Trắc nghiệm đúng sai}
%%%=============SOẠN EXTF===============%%%
\Opensolutionfile{ansex}[Ans/LGTF-C03B01QTOT.tex]
\Opensolutionfile{ansbook}[Ansbook/AnsTF-C03B01QTOT.tex]
\Opensolutionfile{ans}[Ans/Tempt-C03B01QTOT.tex]
%%%=============EX_1=============%%%
\begin{ex}
	Xét các phát biểu về quy tắc octet:
	\choiceTF
	{\True Nguyên tử các nguyên tố có xu hướng nhận hoặc nhường electron để đạt cấu hình electron của khí hiếm gần nhất}
	{Quy tắc octet chỉ áp dụng cho các nguyên tố phi kim}
	{\True Các nguyên tử có thể đạt được cấu hình octet bằng cách chia sẻ các electron hóa trị}
	{\True Trong phân tử, các nguyên tử thường có xu hướng đạt được 8 electron ở lớp ngoài cùng}
	\loigiai{
		\begin{itemchoice}[T1,F2,T3,T4]
			\itemch Đây là nguyên lý cơ bản của quy tắc octet
			\itemch Quy tắc octet áp dụng cho cả kim loại và phi kim
			\itemch Việc chia sẻ electron hóa trị tạo thành liên kết cộng hóa trị
			\itemch Cấu hình electron bền vững thường có 8 electron lớp ngoài cùng
		\end{itemchoice}
	}
\end{ex}
%%%=============EX_2=============%%%
\begin{ex}
	Về quy tắc bát tử :
	\choiceTF
	{\True Các nguyên tử trong phân tử thường đạt 8 electron lớp ngoài cùng}
	{Tất cả các nguyên tố chu kỳ 2 đều tuân theo quy tắc bát tử}
	{\True Cấu hình electron của khí hiếm là cấu hình bền vững}
	{Nguyên tử Li có 8 electron ở lớp vỏ ngoài cùng}
	\loigiai{
		\begin{itemchoice}[T1,F2,T3,F4]
			\itemch Đây là quy luật phổ biến trong tự nhiên khi hình thành liên kết
			\itemch Be và B trong chu kỳ 2 là những ngoại lệ của quy tắc bát tử
			\itemch Khí hiếm có cấu hình electron đặc biệt bền vững
			\itemch Li có 1 electron lớp ngoài cùng, không phải 8 electron
		\end{itemchoice}
	}
\end{ex}
%%%=============EX_3=============%%%
\begin{ex}
	Về các trường hợp đặc biệt:
	\choiceTF
	{\True Hiđro chỉ cần 1 electron để đạt cấu hình bền của Heli}
	{\True Các nguyên tử có thể đạt được octet bằng cách nhận, nhường hoặc dùng chung electron}
	{Mọi nguyên tử đều phải đạt đủ 8 electron để tạo thành phân tử bền}
	{\True Một số nguyên tử có thể bền với ít hơn 8 electron ở lớp ngoài cùng}
	\loigiai{
		\begin{itemchoice}[T1,T2,F3,T4]
			\itemch H là trường hợp đặc biệt vì nó thuộc chu kỳ 1
			\itemch Có nhiều cách để nguyên tử đạt được cấu hình bền
			\itemch Có những trường hợp ngoại lệ như H (2e), B (6e), Be (4e)
			\itemch Be trong hợp chất của nó chỉ có 4 electron vẫn bền
		\end{itemchoice}
	}
\end{ex}
%%%=============EX_4=============%%%
\begin{ex}
	Về mối liên hệ với bảng tuần hoàn:
	\choiceTF
	{\True Số electron tối đa ở lớp ngoài cùng của các nguyên tử trong một chu kỳ luôn bằng số thứ tự của nhóm A}
	{Tất cả các nguyên tố họ p đều tuân theo quy tắc bát tử}
	{\True Các electron lớp ngoài cùng quyết định khả năng tham gia phản ứng của nguyên tử}
	{\True Các nguyên tố nhóm A có số electron hóa trị bằng số thứ tự nhóm}
	\loigiai{
		\begin{itemchoice}[T1,F2,T3,T4]
			\itemch Đây là quy luật quan trọng trong bảng tuần hoàn
			\itemch B thuộc họ p nhưng là ngoại lệ của quy tắc bát tử
			\itemch Electron lớp ngoài quyết định tính chất hóa học
			\itemch Số electron hóa trị tương ứng với số thứ tự nhóm A
		\end{itemchoice}
	}
\end{ex}
%%%=============EX_5=============%%%
\begin{ex}
	Về năng lượng và cấu hình electron
	\choiceTF
	{\True Cấu hình electron của khí hiếm có năng lượng thấp nhất trong cùng chu kỳ}
	{\True Sự bền vững của cấu hình bát tử liên quan đến năng lượng ion hóa cao}
	{Các nguyên tử luôn đạt được cấu hình bát tử bằng cách nhận thêm electron}
	{\True Độ bền của cấu hình bát tử liên quan đến sự đối xứng của các orbital}
	\loigiai{
		\begin{itemchoice}[T1,T2,F3,T4]
			\itemch Năng lượng thấp thể hiện tính bền vững cao
			\itemch Năng lượng ion hóa cao của khí hiếm chứng tỏ độ bền vững
			\itemch Nguyên tử có thể đạt bát tử bằng nhiều cách khác nhau
			\itemch Orbital đầy và đối xứng tạo nên độ bền cao của cấu hình
		\end{itemchoice}
	}
\end{ex}
%%%=============EX_6=============%%%
\begin{ex}
	Về quan hệ giữa cấu trúc electron và quy tắc bát tử
	\choiceTF
	{\True Orbital p chỉ chứa tối đa 6 electron nên cần thêm 2 electron từ orbital s để đạt cấu hình bát tử}
	{Mọi nguyên tử đều cần đủ 8 electron để tạo thành phân tử bền}
	{\True Cấu hình bát tử liên quan đến sự lấp đầy hoàn toàn các orbital s và p}
	{\True Độ bền của cấu hình bát tử liên quan đến sự đối xứng của các orbital s và p}
	\loigiai{
		\begin{itemchoice}[T1,F2,T3,T4]
			\itemch Cấu hình bát tử gồm 2 electron s và 6 electron p
			\itemch H, He và một số nguyên tố khác là ngoại lệ
			\itemch Orbital s và p đầy tạo nên cấu hình electron bền
			\itemch Sự đối xứng của orbital làm tăng độ bền của nguyên tử
		\end{itemchoice}
	}
\end{ex}
%%%=============EX_7=============%%%
\begin{ex}
	Cho các phát biểu sau về quy tắc octet:
	\choiceTF
	{\True Nguyên tử của hầu hết các nguyên tố nhóm A có xu hướng đạt cấu hình 8 electron lớp ngoài cùng khi hình thành liên kết hóa học.}
	{Nguyên tử của nguyên tố nhóm B luôn có xu hướng đạt cấu hình 8 electron lớp ngoài cùng.}
	{Quy tắc octet áp dụng cho tất cả các nguyên tố.}
	{Nguyên tử H có xu hướng đạt 8 electron lớp ngoài cùng khi tham gia liên kết hóa học.}
	\loigiai{
		\begin{itemchoice}[T1,F2,F3,F4]
			\itemch Nguyên tử của hầu hết các nguyên tố nhóm A có xu hướng đạt cấu hình 8 electron lớp ngoài cùng khi hình thành liên kết hóa học. Đây là nội dung của quy tắc octet.
			\itemch Nguyên tử của nguyên tố nhóm B không nhất định đạt cấu hình 8 electron lớp ngoài cùng.
			\itemch Quy tắc octet không áp dụng cho tất cả các nguyên tố, ví dụ như $H$, $Li$, $Be$, $B$,\ldots
			\itemch Nguyên tử H có xu hướng đạt 2 electron lớp ngoài cùng khi tham gia liên kết hóa học.
		\end{itemchoice}
	}
\end{ex}
%%%=============EX_8=============%%%
\begin{ex}
	Cho các nhận định sau về việc áp dụng quy tắc octet:
	\choiceTF
	{Quy tắc octet không phải lúc nào cũng đúng với mọi hợp chất.}
	{\True Có thể dựa vào quy tắc octet để dự đoán công thức của một số hợp chất.}
	{Quy tắc octet chỉ áp dụng cho hợp chất ion.}
	{\True Quy tắc octet có thể áp dụng cho cả hợp chất cộng hóa trị.}
	\loigiai{
		\begin{itemchoice}[T1,T2,F3,T4]
			\itemch Quy tắc octet không phải lúc nào cũng đúng với mọi hợp chất. Có những trường hợp ngoại lệ.
			\itemch Có thể dựa vào quy tắc octet để dự đoán công thức của một số hợp chất.
			\itemch Quy tắc octet không chỉ áp dụng cho hợp chất ion mà còn áp dụng cho hợp chất cộng hóa trị.
			\itemch Quy tắc octet có thể áp dụng cho cả hợp chất cộng hóa trị.
		\end{itemchoice}
	}
\end{ex}
%%%=============EX_9=============%%%
\begin{ex}
	Cho các phát biểu sau về phân tử $BF_3$:
	\choiceTF
	{\True $BF_3$ không tuân theo quy tắc octet.}
	{$B$ trong $BF_3$ đạt cấu hình bền của khí hiếm $He$.}
	{\True Mỗi nguyên tử $F$ trong $BF_3$ đạt cấu hình bền của $Ne$.}
	{\True $BF_3$ là phân tử bền vững.}
	\loigiai{
		\begin{itemchoice}[T1,F2,T3,T4]
			\itemch $BF_3$ không tuân theo quy tắc octet vì $B$ chỉ có 6 electron lớp ngoài cùng.
			\itemch $B$ trong $BF_3$ có 6 electron lớp ngoài cùng, không đạt cấu hình bền của khí hiếm $He$ (2 electron).
			\itemch Mỗi nguyên tử $F$ trong $BF_3$ đạt cấu hình bền của $Ne$ (8 electron lớp ngoài cùng).
			\itemch $BF_3$ là phân tử bền vững mặc dù không tuân theo quy tắc octet.
		\end{itemchoice}
	}
\end{ex}
%%%=============EX_10=============%%%
\begin{ex}
	Cho các phát biểu sau về phân tử $N_2$:
	\choiceTF
	{\True Mỗi nguyên tử $N$ có 8 electron lớp ngoài cùng.}
	{ $N_2$ có liên kết đôi.}
	{ $N_2$ không tuân theo quy tắc octet.}
	{\True Mỗi nguyên tử $N$ góp 3 electron tạo liên kết.}
	\loigiai{
		\begin{itemchoice}[T1,F2,F3,T4]
			\itemch Mỗi nguyên tử $N$ có 8 electron lớp ngoài cùng (đạt cấu hình bền vững).
			\itemch $N_2$ có liên kết ba.
			\itemch $N_2$ tuân theo quy tắc octet.
			\itemch Mỗi nguyên tử $N$ góp 3 electron tạo liên kết ba.
		\end{itemchoice}
	}
\end{ex}
%%%=============EX_11=============%%%
\begin{ex}
	Cho các phát biểu sau về ion $Mg^{2+}$:
	\choiceTF
	{\True Có cấu hình electron giống khí hiếm $Ne$.}
	{Có 12 electron.}
	{ $Mg$ có xu hướng nhận 2 electron để tạo thành $Mg^{2+}$.}
	{\True $Mg$ thuộc nhóm IIA.}
	\loigiai{
		\begin{itemchoice}[T1,F2,F3,T4]
			\itemch $Mg^{2+}$ có cấu hình electron giống khí hiếm $Ne$.
			\itemch $Mg^{2+}$ có 10 electron.
			\itemch $Mg$ có xu hướng nhường 2 electron để tạo thành $Mg^{2+}$.
			\itemch $Mg$ thuộc nhóm IIA.
		\end{itemchoice}
	}
\end{ex}
%%%=============EX_12=============%%%
\begin{ex}
	Cho các phát biểu sau về $CCl_4$:
	\choiceTF
	{\True $CCl_4$ tuân theo quy tắc octet.}
	{\True Mỗi nguyên tử $Cl$ đạt cấu hình 8e lớp ngoài cùng.}
	{\True $C$ trung tâm đạt cấu hình 8e lớp ngoài cùng.}
	{$CCl_4$ có 3 liên kết cộng hóa trị.}
	\loigiai{
		\begin{itemchoice}[T1,T2,T3,F4]
			\itemch $CCl_4$ tuân theo quy tắc octet.
			\itemch Mỗi nguyên tử $Cl$ đạt cấu hình 8e lớp ngoài cùng.
			\itemch $C$ trung tâm đạt cấu hình 8e lớp ngoài cùng.
			\itemch $CCl_4$ có 4 liên kết cộng hóa trị.
		\end{itemchoice}
	}
\end{ex}
%%%=============EX_13=============%%%
\begin{ex}
	Cho các phát biểu sau về $SF_6$:
	\choiceTF
	{$SF_6$ tuân theo quy tắc octet.}
	{$S$ có 8 electron lớp ngoài cùng.}
	{\True Mỗi $F$ có 8 electron lớp ngoài cùng.}
	{\True $SF_6$ có 6 liên kết cộng hóa trị.}
	\loigiai{
		\begin{itemchoice}[F1,F2,T3,T4]
			\itemch $SF_6$ không tuân theo quy tắc octet, $S$ có 12 electron lớp ngoài cùng.
			\itemch $S$ có 12 electron lớp ngoài cùng.
			\itemch Mỗi $F$ có 8 electron lớp ngoài cùng.
			\itemch $SF_6$ có 6 liên kết cộng hóa trị.
		\end{itemchoice}
	}
\end{ex}
%%%=============EX_14=============%%%
\begin{ex}
	Cho các nhận định liên quan đến quy tắc octet
	\choiceTF
	{\True Quy tắc Octet nói rằng các nguyên tử có xu hướng nhận, nhường hoặc góp chung electron để đạt được 8 electron ở lớp ngoài cùng. (trừ Helium)}
	{Quy tắc Octet áp dụng cho tất cả các nguyên tố trong bảng tuần hoàn}
	{Các nguyên tố nhóm A luôn tuân theo quy tắc octet khi tạo liên kết hóa học}
	{\True Quy tắc Octet không thể giải thích được cấu hình của các phân tử thuộc nhóm B}
	{}
	\loigiai{}
\end{ex}

%%%=============EX_15=============%%%
\begin{ex}
	Về ion
	\choiceTF
	{\True Khi sodium (Na) mất 1 electron, nó trở thành ion $Na^+$}
	{\True Khi fluorine nhận thêm 1 electron, nó trở thành ion $F^-$}
	{Nguyên tử helium (He) tuân theo quy tắc Octet}
	{\True Nguyên tử nitrogen trong phân tử $N_2$ tuân theo quy tắc Octet}
	\loigiai{}
\end{ex}

%%%=============EX_16=============%%%
\begin{ex}
	Phân tích đặc điểm các nguyên tử, phân tử \textbf{không} theo quy tắc octet
	\choiceTF
	{Nguyên tử fluorine có cấu hình electron bền vững sau khi nhận thêm 1 electron}
	{\True Phân tử $SF_6$ tuân theo quy tắc Octet}
	{Phân tử nitrogen ($N_2$) được tạo thành bởi 3 cặp electron chung}
	{Nguyên tử oxygen trong phân tử $O_2$ có 2 cặp electron chưa liên kết}
	\loigiai{}
\end{ex}

%%%=============EX_17=============%%%
\begin{ex}
	Khi hình thành liên kết hoá học trong phân tử $CCl_4$:
	\choiceTF
	{\True Mỗi nguyên tử chlorine đều có 7 electron ở lớp ngoài cùng}
	{\True Mỗi nguyên tử chlorine cần góp chung thêm 1 electron để đạt cấu hình bền vững}
	{\True Nguyên tử carbon có 4 electron hóa trị, nên nguyên tử carbon sẽ góp chung với mỗi nguyên tử chlorine 1 electron}
	{Nguyên tử carbon và chlorine sau khi góp chung electron đều sẽ đạt cấu hình bền vững của khí hiếm neon}
	\loigiai{}
\end{ex}
\Closesolutionfile{ans}
\Closesolutionfile{ansbook}
\Closesolutionfile{ansex}
%\bangdapanTF{AnsTF-C03B01QTOT.tex}
\phan{Bài tập tự luận}
%%%=============SOẠN BT===============%%%
\Opensolutionfile{ansbth}[Ans/LGBT-C03B01_QTOCTET_01.tex]
\Opensolutionfile{ansbt}[Ans/AnsBT-C03B01_QTOCTET_01.tex]
	%%%=============BT_1=============%%%
	\begin{bt}
		Hãy ghép mỗi nguyên tử ở cột A với nội dung được mô tả ở cột B cho phù hợp.
		\\
		\begin{tabular}{L{0.35\linewidth}L{0.65\linewidth}}
			\textbf{Cột A}
			\begin{enumerate}[a)]
				\item $\mathrm{Ne}(\mathrm{Z}=10)$
				\item $\mathrm{F}(\mathrm{Z}=9)$
				\item $\mathrm{Mg}(\mathrm{Z}=12)$
				\item $\mathrm{He}(\mathrm{Z}=2)$
			\end{enumerate}
			&
			\textbf{Cột B}
			\begin{enumerate}[1.]
				\item có xu hướng nhận thêm 1 electron.
				\item có cấu hình lớp vỏ ngoài cùng 8 electron bền vững.
				\item có $x u$ hướng nhường đi 2 electron.
				\item có cấu hình lớp vỏ ngoài cùng 2 electron bền vững.
			\end{enumerate}\\
		\end{tabular}
		\loigiai{\begin{tabular}{cccc}
				a) -- 2.& b) -- 1.&c) -- 3.&d) -- 4.
		\end{tabular}}
	\end{bt}
	%%%=============BT_2=============%%%
	\begin{bt}
		Em hãy vẽ mô hình mô tả quá trình tạo lớp vỏ thoả mãn quy tắc octet trong các trường hợp sau đây:
		\begin{enumerate}[a)]
			\item Nguyên tử $\mathrm{O}(\mathrm{Z}=8)$ nhận 2 electron để tạo anion $\mathrm{O}^{2-}$.
			\item Nguyên tử $\mathrm{Ca}(\mathrm{Z}=20)$ nhường 2 electron để tạo cation $\mathrm{Ca}^{2+}$.
			\item Hai nguyên tử fluorine "góp chung electron" để đạt được lớp vỏ thoả mãn quy tắc octet.
		\end{enumerate}
		\loigiai{
		\begin{enumerate}[a)]
			\item Nguyên tử $\mathrm{O}(\mathrm{Z}=8)$ nhận 2 electron để tạo anion $\mathrm{O}^{2-}$.
			\begin{center}
				\includegraphics[height=3.5cm]{Images/Tikz/xuhuongnhanelectron-Oxigen.pdf}
			\end{center}
			\item Nguyên tử $\mathrm{Ca}(\mathrm{Z}=20)$ nhường 2 electron để tạo cation $\mathrm{Ca}^{2+}$.
			\begin{center}
				\includegraphics[height=4.5cm]{Images/Tikz/xuhuongnhuongelectron-Calcium.pdf}
			\end{center}
			\item Hai nguyên tử fluorine "góp chung electron" để đạt được lớp vỏ thoả mãn quy tắc octet.
			\begin{center}
				\includegraphics[height=3.5cm]{Images/Tikz/xuhuonggopchungelectron-Flourine.pdf}
			\end{center}
		\end{enumerate}
		}
	\end{bt}
	%%%=============BT_3=============%%%
	\begin{bt}
		Trong tự nhiên, các khí hiếm tồn tại dưới dạng nguyên tử tự do. Các nguyên tử của khí hiếm không liên kết với nhau tạo thành phân tử và rất khó liên kết với các nguyên tử của các nguyên tố khác. Ngược lại nguyên tử các nguyên tố khác lại liên kết với nhau tạo thành phân tử hay tinh thể. Giải thích
		\loigiai{\begin{itemize}
				\item Nguyên tử khí hiếm đều có cấu hình electron bão hoà là $n s^2 n p^6$ (trừ helium có cấu hình $1 \mathrm{~s}^2$ ) làm cho nguyên tử khí hiếm rất bền vững nên các nguyên tử khí hiếm rất khó tham gia phản ứng hoá học. Trong tự nhiên, các khí hiếm đều tồn tại ở trạng thái nguyên tử (hay còn gọi là phân tử một nguyên tử) tự do, bền vững (nên còn gọi là các khí trơ).
				\item Nguyên tử của các nguyên tố khác có xu hướng liên kết với nhau để đạt được cấu hình electron bền vững của khí hiếm, ví dự: $\mathrm{H}_2, \mathrm{Cl}_2, \mathrm{HCl}, \mathrm{CO}_2, \ldots$ hay tự tập hợp lại thành các khối tinh thể, ví dụ: tỉnh thể $\mathrm{NaCl}_2, \ldots$
		\end{itemize}}
	\end{bt}
	%%%=============BT_4=============%%%
	\begin{bt}
		Cấu hình electron lớp ngoài cùng của nguyên tử potassium (kali) là $4\mathrm{s}^1$, cấu hình electron lớp ngoài cùng của nguyên tử bromine là $4\mathrm{s}^24p^5$. Làm thế nào các nguyên tử potassium và bromine có được cấu hình electron của nguyên tử khí hiếm theo quy tắc octet
		\loigiai{\begin{itemize}
				\item Nguyên tử potassium chỉ có 1 electron ở lớp ngoài cùng nên dễ dàng nhường đi electron này để tạo thành ion dương. Ion dương $\left(\mathrm{K}^{+}\right)$có cấu hình electron lớp ngoài cùng giống với khi hiếm argon $\left(3 \mathrm{~s}^2 3 \mathrm{p}^6\right)$ đứng trước potassium trong bảng tuần hoàn.
				\item Nguyên tử bromine có 7 electron ở lớp electron ngoài cùng nên dễ dàng nhận thêm 1 electron tạo ra anion bromide $\left(\mathrm{Br}^{-}\right)$có cấu hình electron lớp ngoài cùng giống với khí hiếm krypton $\left(4 \mathrm{~s}^2 4 \mathrm{p}^6\right)$, đứng sau bromine trong bảng tuần hoàn.
		\end{itemize}}
	\end{bt}
	%%%=============BT_5=============%%%
	\begin{bt}
		Khi hình thành liên kết $H+\mathrm{Cl} \rightarrow \mathrm{HCl}$ và khi phá vỡ liên kết $\mathrm{HCl} \rightarrow H+\mathrm{Cl}$ thì hệ thu năng lượng hay toả năng lượng. Năng lượng phân tử HCl lớn hơn hay nhỏ hơn năng lượng hệ hai nguyên tử H và Cl riêng rẽ? Trong hai hệ đó thì hệ nào bền hơn?
		\loigiai{%
			\begin{itemize}
				\item  Khi hình thành liên kết $\mathrm{H}+\mathrm{Cl} \rightarrow \mathrm{H}-\mathrm{Cl}$ thì hệ toả ra năng lượng và ngược lại khi phá vỡ liên kết $\mathrm{H}-\mathrm{Cl} \rightarrow \mathrm{H}+\mathrm{Cl}$ thì hệ thu thêm năng lượng.
				\item  Xét về mặt năng lượng thì phân tử $\mathrm{H}-\mathrm{Cl}$ có năng lượng nhỏ hơn hệ hai nguyên tử H và Cl riêng rẽ. Trong hai hệ đó thì hệ $\mathrm{H}-\mathrm{Cl}$ bền hơn hệ H và Cl .
			\end{itemize}
		}
	\end{bt}
	%%%=============BT_6=============%%%
	\begin{bt}
		Trong phân tử $\mathrm{Na}_2\mathrm{S}$, cấu hình electron của các nguyên tử có tuân theo quy tắc octet không?
		\loigiai{%
			Cấu hình electron của Na:
			$\underset{1s^2}{\squarerow[2ud][0.5][\maunhan]{1}}$ $\underset{2s^2}{\squarerow[2ud][0.5][\maunhan]{1}}$
			$\underset{2p^6}{\squarerow[2ud,2ud,2ud][0.5][\maunhan]{3}}$
			$\underset{3s^1}{\squarerow[1u][0.5][\maunhan]{1}}$
			\\
			Cấu hình electron của S:
			$\underset{1s^2}{\squarerow[2ud][0.5][\maunhan]{1}}$ $\underset{2s^2}{\squarerow[2ud][0.5][\maunhan]{1}}$
			$\underset{2p^6}{\squarerow[2ud,2ud,2ud][0.5][\maunhan]{3}}$
			$\underset{3s^2}{\squarerow[2ud][0.5][\maunhan]{1}}$
			$\underset{3p^6}{\squarerow[2ud,2ud,2ud][0.5][\maunhan]{3}}$
			$\underset{4s^2}{\squarerow[2ud][0.5][\maunhan]{1}}$
			\begin{itemize}
				\item Khi Na kết hợp với S , mỗi nguyên tử Na nhường đi 1 electron hoá trị duy nhất để tạo thành cation $\mathrm{Na}^{+}$có 8 electron ở vỏ nguyên tử giống với khí hiếm neon. Nguyên tử S có 6 electron hoá trị nhận thêm 2 electron từ hai nguyên tử Na tạo thành ion sulfide $\mathrm{S}^{2-}$ có 8 electron ở vỏ nguyên tử giống với khí hiếm argon.
				\item Hai nguyên tử Na và S đều đạt cấu hình electron bão hoà theo quy tắc octet trong phân tử sodium sulfide $\mathrm{Na}_2 \mathrm{~S}$.
			\end{itemize}
		}
	\end{bt}
	%%%=============BT_7=============%%%
	\begin{bt}
		Vận dụng quy tắc octet để giải thích sự hình thành liên kết trong các phân tử: $O_2, CO_2, \mathrm{CaCl}_2, \mathrm{KBr}$
		\loigiai{%
			\begin{enumerate}
				\item Phân tử $O_2$:\\
				\schemestart
					\chemfig{\charge{[.radius=0.2ex]0:2pt=\:,90:2pt=\:,180:2pt=\:}{O}} 
					\+ 
					\chemfig{\charge{[.radius=0.2ex]0:2pt=\:,90:2pt=\:,180:2pt=\:}{O}}%
					\arrow{->}[,,,-stealth]
					\chemfig{%
						\charge{[.radius=0.2ex]0:2pt=\:,90:2pt=\:,180:2pt=\:}{O}
						-[,0.6,,,draw=none]
						\charge{[.radius=0.2ex]0:2pt=\:,90:2pt=\:,180:2pt=\:}{O}
					}%
				\schemestop \quad hay \quad \chemfig{%
					\charge{[.radius=0.2ex]90:2pt=\:,180:2pt=\:}{O}
					=[,0.6]
					\charge{[.radius=0.2ex]0:2pt=\:,90:2pt=\:}{O}
				}%
				\item Phân tử $CO_2$:\\
				\schemestart
					\chemfig{\charge{[.radius=0.2ex]0:2pt=\:,90:2pt=\:,180:2pt=\:}{O}} 
					\+ 
					\chemfig{\charge{[.radius=0.2ex]180:2pt=\:,0:2pt=\:}{C}} 
					\+
					\chemfig{\charge{[.radius=0.2ex]0:2pt=\:,90:2pt=\:,180:2pt=\:}{O}}%
					\arrow{->}[,,,-stealth]
					\chemfig{%
						\charge{[.radius=0.2ex]0:2pt=\:,90:2pt=\:,180:2pt=\:}{O}
						-[,0.6,,,draw=none]
						\charge{[.radius=0.2ex]0:2pt=\:,180:2pt=\:}{C}
						-[,0.6,,,draw=none]
						\charge{[.radius=0.2ex]0:2pt=\:,90:2pt=\:,180:2pt=\:}{O}
					}%
				\schemestop\quad hay \quad \chemfig{%
					\charge{[.radius=0.2ex]90:2pt=\:,180:2pt=\:}{O}
					=[,0.6]C=[,0.6]\charge{[.radius=0.2ex]0:2pt=\:,90:2pt=\:}{O}
				}%
				\item Phân tử $\mathrm{CaCl}_2$:\\
				\schemestart
				\chemfig{\charge{[.radius=0.2ex]0:2pt=\.,90:2pt=\:,180:2pt=\:,-90:2pt=\:}{Cl}} 
				\+ 
				\chemfig{\charge{[.radius=0.2ex]0:2pt=\.,180:2pt=\.}{Ca}}
				\+
				\chemfig{\charge{[.radius=0.2ex]0:2pt=\:,90:2pt=\:,180:2pt=\.,-90:2pt=\:}{Cl}}
				\arrow{->}[,,,-stealth]
				\khungion[-]{\chemfig{\charge{[.radius=0.2ex]0:2pt=\:,90:2pt=\:,180:2pt=\:,-90:2pt=\:}{Cl}}} 
				\+ 
				\khungion[2+]{\chemfig{Ca}}
				\+
				\khungion[-]{\chemfig{\charge{[.radius=0.2ex]0:2pt=\:,90:2pt=\:,180:2pt=\:,-90:2pt=\:}{Cl}}}
				\schemestop
				\item Phân tử $\mathrm{KBr}$:\\
				\schemestart
				\chemfig{\charge{[.radius=0.2ex]0:2pt=\.}{K}} 
				\+ 
				\chemfig{\charge{[.radius=0.2ex]180:2pt=\.,0:2pt=\:,90:2pt=\:,-90:2pt=\:}{Br}}
				\arrow{->}[,,,-stealth]
				\khungion{\chemfig{K}}\+\khungion[-]{\chemfig{\charge{[.radius=0.2ex]180:2pt=\:,0:2pt=\:,90:2pt=\:,-90:2pt=\:}{Br}}}
				\schemestop
			\end{enumerate}
		}
	\end{bt}

	%%%=============BT_8=============%%%
	\begin{bt}
		Đá vôi (thành phần chính là $\mathrm{CaCO}_3$) được dùng để sản xuất vôi, trong lĩnh vực xây dựng, $\ldots$ Barium nitrate $\mathrm{Ba}\left(NO_3\right)_2$ có trong thành phần của kính quang học, gốm, men,\ldots Phèn đơn aluminium sulfate (thành phần chính là $\mathrm{Al}_2\left(SO_4\right)_3$) được sử dụng rộng rãi trong xử lí nước thải, trong công nghệ sản xuất giấy, công nghệ nhuộm vải và công nghệ lọc nước và nuôi trồng thuỷ sản,\ldots Dựa vào quy tắc octet, đề xuất công thức cấu tạo của các chất trên
		\loigiai{%
			\begin{enumerate}
				\item $CaCO_3$:
				\chemfig{Ca?[a]-[:45]O-[:-45]C(-[:-135]O?[a])=O}
				\item $Ba(NO_3)_2$
				\chemfig{O=[:-45]N(-[:-135,,,,-stealth]O)-O-Ba-O-N(-[:-45,,,,-stealth]O)=[:45]O}
				\item $Al_2(SO_4)_3$
				\chemfig{S(-[:-135,,,,-stealth]O)(-[:135,,,,-stealth]O)(-[:-45]O?[a])-[:45]O-[:-45]Al?[a]-O-[:-45]S(-[:-135,,,,-stealth]O)(-[:-45,,,,-stealth]O)-[:45]O-Al?[b]-[:45]O-[:-45]S(-[:-135]O?[b])(-[:-45,,,,-stealth]O)-[:45,,,,-stealth]O}
			\end{enumerate}
		}
	\end{bt}
	%%%=============BT_9=============%%%
	\begin{bt}
		Hợp chất X tạo bởi hai nguyên tố $A$, $D$ có khối lượng phân tử là 76. X là dung môi không phân cực, thường được sử dụng làm nguyên liệu trong tồng hợp chất hữu cơ chứa lưu huỳnh và được sử dụng rộng rãi trong sản xuất vải viscoza mềm. A có công thức hydride dạng $AH_4$ và D có công thức oxide ứng với hoá trị cao nhất dạng $DO_3$.
		\begin{enumerate}
			\item  Hãy thiết lập công thức phân tử của X. Biết rằng A có số oxi hoá cao nhất trong X.
			\item  Đề xuất công thức cấu tạo của X và cho biết các nguyên tử thành phần của X khi liên kết có đủ electron theo quy tắc octet không?
		\end{enumerate}
		\loigiai{
			\begin{enumerate}
				\item A thuộc nhóm IVA và D thuộc nhóm VIA $\Rightarrow$ số oxi hoá cao nhất của A trong X là +4 còn số oxi hoá của D trong X là -2 .
				Công thức phân tử X có dạng $\mathrm{AD}_2$. Ta có: $\mathrm{A}+2 \mathrm{D}=76$.
				\\
				$\Rightarrow$ Nguyên tử khối trung bình của $\mathrm{A}, \mathrm{D}$ là: $\dfrac{76}{3}=25,33$.
				\\
				$\Rightarrow A$ và $D$ thuộc chu kì $2,3 \Rightarrow$ Có các cặp nguyên tố sau:
				$\mathrm{C}=12$ và $\mathrm{O}=16 ; \mathrm{C}=12$ và $\mathrm{S}=32 ; \mathrm{Si}=28$ và $\mathrm{O}=16 ; \mathrm{Si}=28$ và $\mathrm{S}=32$.
				$\mathrm{C}=12$ và $\mathrm{S}=32$ thoả mãn $\mathrm{A}+2 \mathrm{D}=76$
				\\
				$\Rightarrow$ Công thức $\mathrm{X}: \mathrm{CS}_2$.
				\item Đề xuất công thức cấu tạo:\,\, \chemfig{\charge{120:1pt=\:,-120:1pt=\:}{S}=C=\charge{60:1pt=\:,-60:1pt=\:}{S}}\,\, $\mathrm{CS}_2$ có cấu trúc thẳng giống $\mathrm{CO}_2$. Các nguyên tử C và S đều có 8 electron lớp ngoài cùng theo quy tắc octet.
			\end{enumerate}
		}
	\end{bt}
	%%%=============BT_10=============%%%
	\begin{bt}
		Em hãy nêu tên và công thức hoá học của 1 chất ở thể rắn, 1 chất ở thể lỏng và 1 chất ở thể khí (trong điều kiện thường), trong đó nguyên tử oxygen đạt được cấu hình bền của khí hiếm neon.
		\loigiai{
			Để đạt cấu hình electron bền của Ne (1s$^2$2s$^2$2p$^6$), nguyên tử O cần nhận thêm 2 electron để có 8 electron lớp ngoài cùng.
			\begin{enumerate}
				\item Chất rắn: Natri oxit ($\mathrm{Na}_2\mathrm{O}$)
				\begin{itemize}
					\item O nhận 2 electron từ 2 nguyên tử Na để tạo thành ion O$^{2-}$
					\item Mỗi nguyên tử Na cho đi 1 electron để tạo thành ion Na$^+$
					\item Liên kết ion hình thành giữa các ion Na$^+$ và O$^{2-}$
				\end{itemize}
				\item Chất lỏng: Nước ($\mathrm{H}_2\mathrm{O}$)
				\begin{itemize}
					\item O chia sẻ electron với 2 nguyên tử H
					\item Mỗi liên kết O-H là liên kết đơn (1 cặp electron được chia sẻ)
					\item O đạt cấu hình octet nhờ 2 cặp electron liên kết và 2 cặp electron độc thân
				\end{itemize}
				\item Chất khí: Carbon dioxide ($\mathrm{CO}_2$)
				\begin{itemize}
					\item Mỗi nguyên tử O tạo liên kết đôi với nguyên tử C trung tâm
					\item Mỗi O chia sẻ 2 cặp electron với C để đạt cấu hình octet
					\item Phân tử có cấu trúc thẳng O=C=O
				\end{itemize}
			\end{enumerate}
		}
	\end{bt}
	%%%=============BT_11=============%%%
	\begin{bt}
		Potassium iodide $(KI)$ được  sử dụng như một loại thuốc long đờm, giúp làm lỏng và phá vỡ chất nhầy trong đường thở, thường dùng cho các bệnh nhân hen suyễn, viêm phế quản mãn tính. Trong trường hợp bị nhiễm phóng xạ, KI còn giúp ngăn tuyến giáp hấp thụ iodine phóng xạ, bảo vệ và giảm nguy cơ ung thư tuyến giáp. Trong phân tử KI, các nguyên tử K và I đều đã đạt được cơ cấu bền của khí hiếm gần nhất. Đó lần lượt là những khí hiếm nào?
		\loigiai{%
			\begin{enumerate}
				\item Xác định cấu hình electron của K và I:
				\begin{itemize}
					\item K (Z = 19): 1s$^2$2s$^2$2p$^6$3s$^2$3p$^6$4s$^1$
					\item I (Z = 53): 1s$^2$2s$^2$2p$^6$3s$^2$3p$^6$3d$^{10}$4s$^2$4p$^6$4d$^{10}$5s$^2$5p$^5$
				\end{itemize}
				\item Trong KI:
				\begin{itemize}
					\item K cho đi 1 electron ($4s^1$) để tạo thành ion K$^+$
					\item I nhận thêm 1 electron để tạo thành ion I$^-$
				\end{itemize}
				\item Cấu hình electron của các ion:
				\begin{itemize}
					\item K$^+$: 1s$^2$2s$^2$2p$^6$3s$^2$3p$^6$ (giống Ar)
					\item I$^-$: 1s$^2$2s$^2$2p$^6$3s$^2$3p$^6$3d$^{10}$4s$^2$4p$^6$4d$^{10}$5s$^2$5p$^6$ (giống Xe)
				\end{itemize}
				Vậy:
				\begin{itemize}
					\item K đạt cấu hình electron của khí hiếm Argon (Ar)
					\item I đạt cấu hình electron của khí hiếm Xenon (Xe)
				\end{itemize}
			\end{enumerate}
		}
	\end{bt}
	%%%=============BT_12=============%%%
	\begin{bt}
		Em hãy nêu tên và công thức hoá học của 1 chất ở thể rắn, 1 chất ở thể lỏng và 1 chất ở thể khí (trong điều kiện thường), trong đó nguyên tử oxygen đạt được cấu hình bền của khí hiếm neon.
		\loigiai{Nguyên tử oxygen đạt được cấu hình bền của khí hiếm neon trong MgO (chất rắn), $\mathrm{H}_2\mathrm{O}$ (chất lỏng) và $\mathrm{O}_2$ (chất khí).}
	\end{bt}
	
	%%%=============BT_13=============%%%
	\begin{bt}
		Potassium iodide $(KI)$ được sử dụng như một loại thuốc long đờm, giúp làm Iỏng và phá vỡ chất nhầy trong đường thở, thường dùng cho các bệnh nhân hen suyễn, viêm phế quản mãn tính. Trong trường hợp bị nhiễm phóng xạ, KI còn giúp ngăn tuyến giáp hấp thụ iodine phóng xạ, bảo vệ và giảm nguy cơ ung thư tuyến giáp. Trong phân tử KI, các nguyên tử K và I đều đã đạt được cơ cấu bền của khí hiếm gần nhất. Đó lần lượt là những khí hiếm nào?
		\loigiai{Trong phân tử potassium iodide (KI), nguyên tử K và I Iần lượt đạt được cơ cấu bền của khí hiếm gần nhất là argon (Ar) và xenon (Xe).}
	\end{bt}
	%%%=============BT_14=============%%%
	\begin{bt}
		Biểu diễn công thức electron, công thức Lewis và CTCT của các phân tử sau:
		$\text{H}_2\text{O}$ ; $\text{NH}_3$ ; $\text{CH}_4$ ; $\text{CO}_2$ ; $\text{CCl}_4$ ; $\text{H}_2\text{S}$ ; $\text{CS}_2$ ; $\text{N}_2$ ; $\text{O}_2$ ; $\text{HCl}$ ; $\text{BF}_3$ ; $\text{PCl}_5$ ; $\text{SF}_6$ ; $\text{BCl}_3$ ; $\text{AlCl}_3$ ; $\text{PF}_5$ ; $\text{HF}$ ; $\text{H}_2\text{CO}$ ; $\text{HNO}_3$; $\text{SO}_2$; $\text{CO}$; $\text{NO}_2$; $\text{NO}$;$\text{CH}_4$; $\text{C}_2\text{H}_4$; $\text{C}_2\text{H}_2$,
		\loigiai{}
	\end{bt}
\Closesolutionfile{ansbt}
\Closesolutionfile{ansbth}
%\bangdapanSA{AnsBT-C03B01_QTOCTET_01.tex}
