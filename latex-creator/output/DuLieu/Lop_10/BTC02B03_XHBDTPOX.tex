%%%%===============Dạng 1============%%%
\begin{dang}{Lý thuyết về xu hướng biến đổi thành phần và một số tính chất của hợp chất trong một chu kì}\end{dang}
\begin{pp}
	Trong một chu kì, theo chiều tăng của điện tích hạt nhân, tính kim loại giảm, dẫn đến tính bazo của oxide và hidroxide tương ứng giảm dần, đòng thời tính acid của chúng tăng dần
\end{pp}
%%%Ví dụ mẫu dạng 1%%%
\Noibat[\maunhan][][\faBookmark]{Ví dụ mẫu}
%%%=============VDM 1=============%%%
%%%==============Cau_VDEX1==============%%%
\begin{vdex}
	Dãy nào sau đây sắp xếp theo thứ tự tăng dần tính acid?
	\choice
	{$\mathrm{Cl}_2O_7, \mathrm{Al}_2O_3; SO_3; P_2O_5$}
	{\True $\mathrm{Al}_2O_3; P_2O; SO_3; \mathrm{Cl}_2O_7$}
	{$P_2O_5; SO_3; \mathrm{Al}_2O_3; \mathrm{Cl}_2O_7$}
	{$\mathrm{Al}_2O_3; SO_3; P_2O_5; \mathrm{Cl}_2O_7$}
	\loigiai{Trong một chu kì theo chiều tăng điện tích hạt nhân tính acid của các oxide tăng dần.Do đó thứ tự tăng dần tính acid là: $Al_2O_3 < P_2O_5 < SO_3 < Cl_2O_7$ 
	}
\end{vdex}
%%%==============HetCau_VDEX1==============%%%

%%%==============Cau_VDEX2==============%%%
\begin{vdex}
	Trong các hydroxide của các nguyên tố chu kì 3, acid mạnh nhất là
	\choice
	{$H_2SO_4$}
	{\True $\mathrm{HClO}_4$}
	{$H_2\mathrm{SiO}_3$}
	{$H_3PO_4$}
	\loigiai{
		Trong một chu kì theo chiều tăng điện tích hạt nhân tính acid của các hydroxide tăng dần.Do đó acid mạnh  nhất là  $\mathrm{HClO}_4$.
	}
\end{vdex}
%%%==============HetCau_VDEX2==============%%%
%%%==============Cau_VDEX3==============%%%
\begin{vdex}
	Đâu là xu hướng biến đổi tính acid, base của các oxide và hydroxide trong một chu kì
	\choice
	{Trong một chu kì tính acid và tính base của các oxide và hydroxide tương ứng tăng dần }
	{Trong một chu kì tính acid và tính base của các oxide và hydroxide tương ứng giảm dần }
	{\True Trong một chu kì tính acid của các oxide và hydroxide tương ứng tăng dần còn tính base thì giảm dần }
	{Trong một chu kì tính acid của các oxide và hydroxide tương ứng giảm dần còn tính base thì tăng dần }
	\loigiai{
		Trong một chu kì tính acid của các oxide và hydroxide tương ứng tăng dần còn tính base thì giảm dần.
	}
\end{vdex}
%%%==============HetCau_VDEX3==============%%%
\Noibat[][][\faBank]{Bài tập tự luyện dạng \thedang}
\phan[\maudam]{Trắc nghiệm nhiều lựa chọn}
%%%=============SOẠN EX===============%%%
\Opensolutionfile{ansex}[Ans/LGEX-C02B03-XHBDTPOX01.tex]
\Opensolutionfile{ans}[Ans/Ans-C02B03-XHBDTPOX01.tex]
%%%=============EX_1=============%%%
\begin{ex}%[0H2N1-2]
	Chất nào sau đây là oxit lưỡng tính?
	\choice
	{\True $\mathrm{Al}_2O_3$}
	{$\mathrm{Na}_2O$}
	{$\mathrm{SO}_2$}
	{$\mathrm{MgO}$}
	\loigiai{$\mathrm{Al}_2O_3$ là oxit lưỡng tính vì nó có thể phản ứng với cả acid và base.}
\end{ex}
%%%=============EX_2=============%%%
\begin{ex}%[0H2H1-2]
	Dãy nào sau đây sắp xếp theo thứ tự giảm dần tính bazơ của các hydroxide?
	\choice
	{$\mathrm{Fe(OH)}_3 > \mathrm{Al(OH)}_3 > \mathrm{NaOH} > \mathrm{Ca(OH)}_2$}
	{\True $\mathrm{NaOH} > \mathrm{Ca(OH)}_2 > \mathrm{Al(OH)}_3 > \mathrm{Fe(OH)}_3$}
	{$\mathrm{Al(OH)}_3 > \mathrm{Fe(OH)}_3 > \mathrm{NaOH} > \mathrm{Ca(OH)}_2$}
	{$\mathrm{Ca(OH)}_2 > \mathrm{NaOH} > \mathrm{Fe(OH)}_3 > \mathrm{Al(OH)}_3$}
	\loigiai{Tính bazơ của các hydroxide kim loại kiềm và kiềm thổ mạnh hơn so với hydroxide của các kim loại chuyển tiếp.}
\end{ex}
%%%=============EX_3=============%%%
\begin{ex}%[0H2H1-2]
	Trong các oxit sau, oxit nào có tính acid mạnh nhất?
	\choice
	{\True $\mathrm{Cl}_2O_7$}
	{$\mathrm{SO}_2$}
	{$\mathrm{CO}2$}
	{$\mathrm{P_4O{10}}$}
	\loigiai{$\mathrm{Cl}_2O_7$ là oxit acid mạnh nhất vì clo có độ âm điện rất lớn, làm tăng tính acid của oxit.}
\end{ex}
%%%=============EX_4=============%%%
\begin{ex}%[0H2H1-2]
	Trong các acid sau, acid nào có tính acid yếu nhất?
	\choice
	{\True $H_2\mathrm{CO}_3$}
	{$\mathrm{HNO}_3$}
	{$\mathrm{HClO}_4$}
	{$\mathrm{H_2SO}_4$}
	\loigiai{$H_2\mathrm{CO}_3$ là acid yếu, phân ly rất ít trong nước so với các acid mạnh như $HNO_3$ hay $H_2SO_4$.}
\end{ex}
%%%=============EX_5=============%%%
\begin{ex}%[0H2H1-2]
	Trong các cặp oxit sau, cặp nào là oxit kiềm và oxit acid?
	\choice {\True $\mathrm{Na}_2O$ và $\mathrm{SO}_3$}
	{$\mathrm{MgO}$ và $\mathrm{CO}_2$}
	{$\mathrm{CaO}$ và $\mathrm{P_2O_5}$}
	{$\mathrm{K_2O}$ và $\mathrm{NO}_2$}
	\loigiai{$\mathrm{Na}_2O$ là oxit kiềm, còn $\mathrm{SO}_3$ là oxit acid.}
\end{ex}
%%%=============EX_6=============%%%
\begin{ex}%[0H2H1-2]
	Dung dịch nào sau đây có tính bazơ mạnh nhất?
	\choice {\True $\mathrm{NaOH}$}
	{$\mathrm{Mg(OH)}_2$}
	{$\mathrm{Ca(OH)}_2$}
	{$\mathrm{Al(OH)}_3$}
	\loigiai{$\mathrm{NaOH}$ là bazơ mạnh vì phân ly hoàn toàn trong dung dịch nước, tạo ion $OH^-$.}
\end{ex}
%%%=============EX_7=============%%%
\begin{ex}%[0H2H1-2]
	Hydroxide nào có tính base yếu nhất trong các hydroxide sau đây?
	\choice
	{\True Magnesium hydroxide}
	{Calcium hydroxide}
	{Strontium hydroxide}
	{Barium hydroxide}
	\loigiai{Magnesium hydroxide có tính base yếu nhất vì Mg(OH)$_2$ ít tan trong nước và có tính base yếu hơn các hydroxide của nhóm kim loại kiềm thổ khác.}
\end{ex}
%%%=============EX_8=============%%%
\begin{ex}%[0H2H1-2]
	Hydroxide nào có tính base mạnh nhất trong các nguyên tố thuộc nhóm $IIA$?
	\choice
	{Magnesium hydroxide}
	{Calcium hydroxide}
	{Strontium hydroxide}
	{\True Barium hydroxide}
	\loigiai{Barium hydroxide (Ba(OH)$_2$) có tính base mạnh nhất trong các hydroxide của nhóm kim loại kiềm thổ.}
\end{ex}
%%%=============EX_9=============%%%
\begin{ex}%[0H2H1-2]
	Trong các hydroxide sau, chất nào có tính base yếu nhất?
	\choice
	{Strontium hydroxide}
	{\True Aluminum hydroxide}
	{Magnesium hydroxide}
	{Calcium hydroxide}
	\loigiai{Aluminum hydroxide (Al(OH)$_3$) có tính lưỡng tính, vừa có tính base yếu vừa có tính acid yếu.}
\end{ex}
%%%=============EX_10=============%%%
\begin{ex}%[0H2H1-2]
	Thứ tự tăng dần tính base của các hydroxide sau: $Al(OH)_3, Be(OH)_2, Mg(OH)_2, Ca(OH)_2$ là
	\choice
	{Ca(OH)$_2$, Mg(OH)$_2$, Be(OH)$_2$, Al(OH)$_3$}
	{Al(OH)$_3$, Mg(OH)$_2$, Be(OH)$_2$, Ca(OH)$_2$}
	{\True Al(OH)$_3$, Be(OH)$_2$, Mg(OH)$_2$, Ca(OH)$_2$}
	{Be(OH)$_2$, Al(OH)$_3$, Mg(OH)$_2$, Ca(OH)$_2$}
	\loigiai{Thứ tự tăng dần tính base là Al(OH)$_3$, Be(OH)$_2$, Mg(OH)$_2$, Ca(OH)$_2$ }
\end{ex}
%%%=============EX_11=============%%%
\begin{ex}%[0H2H1-2]
	Nguyên tử nguyên tố nào sau đây có bán kính nhỏ nhất?
	\choice
	{Li}
	{Na}
	{K}
	{\True F}
	\loigiai{Trong cùng một chu kì, đi từ trái sang phải bán kính nguyên tử giảm dần. Flo ở cuối chu kì 2 nên có bán kính nhỏ nhất.}
\end{ex}
%%%=============EX_12=============%%%
\begin{ex}%[0H2H1-2]
	Nguyên tố nào sau đây có tính kim loại mạnh nhất?
	\choice
	{\True Cs}
	{K}
	{Na}
	{Li}
	\loigiai{Trong cùng một nhóm A, đi từ trên xuống dưới tính kim loại tăng dần. Cs ở vị trí thấp nhất trong nhóm IA nên có tính kim loại mạnh nhất.}
\end{ex}
%%%=============EX_13=============%%%
\begin{ex}%[0H2H1-2]
	Nguyên tố nào sau đây có ái lực electron lớn nhất (tính theo giá trị tuyệt đối)?
	\choice
	{Si}
	{P}
	{S}
	{\True Cl}
	\loigiai{Trong cùng một chu kì, đi từ trái sang phải ái lực electron (theo giá trị tuyệt đối) tăng dần. Clo ở cuối chu kì 3 (trừ khí hiếm) nên có ái lực electron lớn nhất.}
\end{ex}
%%%=============EX_14=============%%%
\begin{ex}%[0H2H1-2]
	Nguyên tử của nguyên tố nào sau đây có năng lượng ion hóa thứ nhất lớn nhất?
	\choice
	{Na}
	{Mg}
	{\True Al}
	{Si}
	\loigiai{Trong cùng một chu kì, đi từ trái sang phải năng lượng ion hóa thứ nhất tăng dần. Nhôm (Al) ở vị trí thứ 3 trong chu kì 3 nên có năng lượng ion hóa thứ nhất lớn nhất trong dãy.}
\end{ex}
%%%=============EX_15=============%%%
\begin{ex}%[0H2V1-4]
	Cấu hình electron của ion $Na^+$ là:
	\choice
	{$1s^22s^22p^63s^1$}
	{$1s^22s^22p^63s^2$}
	{\True $1s^22s^22p^6$}
	{$1s^22s^22p^5$}
	\loigiai{Natri (Na) có cấu hình electron là $1s^22s^22p^63s^1$. Khi mất 1 electron lớp ngoài cùng, ion $Na^+$ có cấu hình electron là $1s^22s^22p^6$.}
\end{ex}
%%%=============EX_16=============%%%
\begin{ex}%[0H2V2-1]
	Nguyên tố X có cấu hình electron lớp ngoài cùng là $ns^2np^4$. X thuộc nhóm:
	\choice
	{IVA}
	{\True VIA}
	{VA}
	{VIIA}
	\loigiai{X có 6 electron lớp ngoài cùng ($ns^2np^4$) nên thuộc nhóm VIA.}
\end{ex}
%%%=============EX_17=============%%%
\begin{ex}%[0H2H1-2]
	Cho các nguyên tố: $_{11}$Na, $_{12}$Mg, $_{13}$Al, $_{19}$K. Nguyên tố có tính kim loại yếu nhất là:
	\choice
	{K}
	{Na}
	{\True Al}
	{Mg}
	\loigiai{Trong cùng một chu kì, đi từ trái sang phải tính kim loại yếu dần. Trong các nguyên tố trên, Al ở vị trí cuối cùng của chu kì 3 nên có tính kim loại yếu nhất.}
\end{ex}
%%%=============EX_18=============%%%
\begin{ex}%[0H2V1-2]
	Cho các ion: $Na^+$, $Mg^{2+}$, $Al^{3+}$, $F^-$. Ion có bán kính lớn nhất là:
	\choice
	{$Na^+$}
	{$Mg^{2+}$}
	{$Al^{3+}$}
	{\True $F^-$}
	\loigiai{Các ion trên đều có cấu hình electron giống khí hiếm Ne ($1s^22s^22p^6$). Ion $F^-$ có số proton trong hạt nhân ít hơn số electron lớp vỏ nên lực hút giữa hạt nhân và electron yếu hơn, bán kính lớn hơn.}
\end{ex}
%%%=============EX_19=============%%%
\begin{ex}%[0H2V2-2]
	Nguyên tử nguyên tố Y có tổng số electron ở phân lớp p là 11. Y là nguyên tố:
	\choice
	{Si}
	{\True Cl}
	{S}
	{P}
	\loigiai{Phân lớp p chứa tối đa 6 electron. Y có 11 electron ở phân lớp p nghĩa là có 2 phân lớp p đã được lấp đầy (2 x 6 = 12). Cấu hình electron của Y là $1s^22s^22p^63s^23p^5$. Vậy Y là Clo (Cl).}
\end{ex}
%%%=============EX_20=============%%%
\begin{ex}%[0H2N1-2]
	Nguyên tố nào sau đây có độ âm điện lớn nhất?
	\choice
	{O}
	{\True F}
	{N}
	{Cl}
	\loigiai{Trong cùng một chu kì, đi từ trái sang phải độ âm điện tăng dần. Flo (F) ở vị trí cuối cùng của chu kì 2 nên có độ âm điện lớn nhất.}
\end{ex}
%%%=============EX_21=============%%%
\begin{ex}%[0H2H1-2]
	Cho các nguyên tố: $_{3}$Li, $_{9}$F, $_{11}$Na, $_{17}$Cl. Nguyên tố có năng lượng ion hóa thứ nhất nhỏ nhất là:
	\choice
	{\True Na}
	{Li}
	{F}
	{Cl}
	\loigiai{Trong cùng một nhóm A, đi từ trên xuống dưới năng lượng ion hóa thứ nhất giảm dần. Natri (Na) ở vị trí thấp nhất trong nhóm IA nên có năng lượng ion hóa thứ nhất nhỏ nhất.}
\end{ex}
%%%=============EX_22=============%%%
\begin{ex}%[0H2H1-2]
	Nguyên tử của nguyên tố nào sau đây có khuynh hướng nhận thêm 2 electron khi tham gia phản ứng hóa học?
	\choice
	{Na}
	{\True O}
	{F}
	{Cl}
	\loigiai{Oxi (O) thuộc nhóm VIA, có 6 electron lớp ngoài cùng, có khuynh hướng nhận thêm 2 electron để đạt cấu hình electron bền vững của khí hiếm gần nhất.}
\end{ex}
%%%=============EX_23=============%%%
\begin{ex}%[0H2H1-2]
	Cho các nguyên tố: N, O, F, Ne. Nguyên tố có bán kính nguyên tử lớn nhất là:
	\choice
	{F}
	{O}
	{\True N}
	{Ne}
	\loigiai{Trong cùng một chu kì, đi từ trái sang phải bán kính nguyên tử giảm dần. Nito (N) ở vị trí đầu tiên của chu kì 2 nên có bán kính nguyên tử lớn nhất.}
\end{ex}
%%%=============EX_24=============%%%
\begin{ex}%[0H2V2-5]
	Nguyên tố X thuộc chu kì 4, nhóm IIA. Cấu hình electron của ion $X^{2+}$ là:
	\choice
	{$1s^22s^22p^63s^23p^63d^2$}
	{$1s^22s^22p^63s^23p^64s^2$}
	{$1s^22s^22p^63s^23p^6$}
	{\True $1s^22s^22p^63s^23p^64s^0$}
	\loigiai{X thuộc chu kì 4, nhóm IIA nên có cấu hình electron là $1s^22s^22p^63s^23p^64s^2$. Khi mất 2 electron lớp ngoài cùng, ion $X^{2+}$ có cấu hình electron là $1s^22s^22p^63s^23p^64s^0$.}
\end{ex}
%%%=============EX_25=============%%%
\begin{ex}%[0H2H1-2]
	Cho các nguyên tố: $_{12}$Mg, $_{13}$Al, $_{14}$Si, $_{15}$P. Nguyên tố có tính phi kim mạnh nhất là:
	\choice
	{Mg}
	{Al}
	{Si}
	{\True P}
	\loigiai{Trong cùng một chu kì, đi từ trái sang phải tính phi kim tăng dần. Photpho (P) ở vị trí cuối cùng của chu kì 3 nên có tính phi kim mạnh nhất.}
\end{ex}
%%%=============EX_26=============%%%
\begin{ex}%[0H2V2-6]
	Nguyên tử của nguyên tố nào sau đây có 3 electron độc thân?
	\choice
	{C}
	{\True N}
	{O}
	{F}
	\loigiai{Nito (N) có cấu hình electron là $1s^22s^22p^3$. Phân lớp p có 3 orbital, mỗi orbital chứa tối đa 2 electron. 3 electron ở phân lớp p của N chiếm 3 orbital riêng biệt, tạo thành 3 electron độc thân.}
\end{ex}
%%%=============EX_27=============%%%
\begin{ex}%[0H2H1-2]
	Cho các nguyên tố: $_{4}$Be, $_{12}$Mg, $_{20}$Ca, $_{38}$Sr. Nguyên tố có năng lượng ion hóa thứ hai nhỏ nhất là:
	\choice
	{Be}
	{Mg}
	{\True Sr}
	{Ca}
	\loigiai{Trong cùng một nhóm A, đi từ trên xuống dưới năng lượng ion hóa thứ hai giảm dần. Stronti (Sr) ở vị trí thấp nhất trong nhóm IIA nên có năng lượng ion hóa thứ hai nhỏ nhất.}
\end{ex}
%%%=============EX_28=============%%%
\begin{ex}%[0H2V2-2]
	Nguyên tố R có cấu hình electron lớp ngoài cùng là $ns^2np^3$. Công thức hợp chất khí với hidro của R là:
	\choice
	{RH}
	{RH$_2$}
	{\True RH$_3$}
	{RH$_4$}
	\loigiai{R có 5 electron lớp ngoài cùng ($ns^2np^3$) nên có hóa trị 3 khi tạo hợp chất với hidro. Công thức hợp chất khí với hidro của R là RH$_3$.}
\end{ex}
%%%=============EX_29=============%%%
\begin{ex}%[0H2V2-3]
	Nguyên tử nguyên tố X có tổng số electron ở phân lớp s là 7. X là nguyên tố thuộc nhóm:
	\choice
	{IA}
	{\True IIA}
	{IIIA}
	{IVA}
	\loigiai{Phân lớp s chứa tối đa 2 electron. X có 7 electron ở phân lớp s nghĩa là có 4 phân lớp s đã được lấp đầy ($4 \times 2 = 8$). Cấu hình electron của X là $1s^22s^22p^63s^23p^64s^2$. Vậy X thuộc nhóm IIA.}
\end{ex}
%%%=============EX_29=============%%%
\begin{ex}%[0H2H2-2]
	X là nguyên tố nhóm IIIA. Công thức oxide ứng với hoá trị cao nhất của X là
	\choice
	{XO}
	{$XO_2$}
	{$X_2O$}
	{\True $X_2O_3$}
	\loigiai{Nguyên tố nhóm IIIA có hoá trị cao nhất là III. Vì vậy, công thức oxide ứng với hóa trị cao nhất của X là $X_2O_3$.}
\end{ex}
%%%=============EX_30=============%%%
\begin{ex}%[0H2H1-2]
	Cho các oxide sau: $Na_2O$, $Al_2O_3$, MgO, $SiO_2$.
	Thứ tự giảm dần tính base là
	\choice
	{$Na_2O$ > $Al_2O_3$ > MgO > $SiO_2$}
	{$Al_2O_3$ > $SiO_2$ > MgO > $Na_2O$}
	{\True $Na_2O$ > MgO > $Al_2O_3$ > $SiO_2$}
	{ MgO > $Na_2O$ > $Al_2O_3$ > $SiO_2$}
	\loigiai{Trong một chu kì, tính base của oxide giảm dần khi đi từ trái sang phải. Trong một nhóm A, tính base của oxide mạnh dần khi đi từ trên xuống dưới. Vậy, thứ tự giảm dần tính base là: $Na_2O$ > MgO > $Al_2O_3$ > $SiO_2$.}
\end{ex}
%%%=============EX_31=============%%%
\begin{ex}%[0H2H1-2]
	Dãy nào sau đây sắp xếp theo thứ tự tăng dần tính acid?
	\choice
	{$Cl_2O_7$; $Al_2O_3$; $SO_3$; $P_2O_5$}
	{\True $Al_2O_3$; $P_2O_5$; $SO_3$; $Cl_2O_7$}
	{$P_2O_5$; $SO_3$; $Al_2O_3$; $Cl_2O_7$}
	{$Al_2O_3$; $SO_3$; $P_2O_5$; $Cl_2O_7$}
	\loigiai{Trong một chu kì, tính acid của oxide tăng dần khi đi từ trái sang phải. Trong một nhóm A, tính acid của oxide yếu dần khi đi từ trên xuống dưới. Vậy, thứ tự tăng dần tính acid là: $Al_2O_3$ < $P_2O_5$ < $SO_3$ < $Cl_2O_7$.}
\end{ex}
%%%=============EX_32=============%%%
\begin{ex}%[0H2V1-2]
	Ba nguyên tố với số hiệu nguyên tử $Z = 11$, $Z = 12$, $Z = 13$ có hydroxide tương ứng là X, Y, T. Chiều tăng dần tính base của các hydroxide này là
	\choice
	{X, Y, T}
	{X, T, Y}
	{\True T, X, Y}
	{T, Y, X}
	\loigiai{Ba nguyên tố với số hiệu nguyên tử $Z = 11$, $Z = 12$, $Z = 13$  lần lượt là Na, Mg, Al. Trong một chu kì, tính base của hydroxide giảm dần. Vậy, chiều tăng dần tính base là: $Al(OH)_3$ < $NaOH$  < $Mg(OH)_2$ }
\end{ex}
%%%=============EX_33=============%%%
\begin{ex}%[0H2V1-2]
	Trong các hydroxide của các nguyên tố chu kì 3, acid mạnh nhất là 
	\choice
	{$H_2SO_4$}
	{\True $HClO_4$}
	{$H_2SiO_3$}
	{$H_3PO_4$}
	\loigiai{Trong một chu kì, tính acid của hydroxide  tăng dần khi đi từ trái sang phải. Vậy, acid mạnh nhất trong các hydroxide của các nguyên tố chu kì 3 là $HClO_4$}
\end{ex}
%%%=============EX_34=============%%%
\begin{ex}%[0H2H1-2]
	Dãy nào sau đây sắp xếp theo thứ tự giảm dần tính base?
	\choice
	{$Al(OH)_3$; $NaOH$; $Mg(OH)_2$; $Si(OH)_4$}
	{$NaOH$;  $Mg(OH)_2$; $Si(OH)_4$; $Al(OH)_3$}
	{\True $NaOH$;  $Mg(OH)_2$; $Al(OH)_3$; $Si(OH)_4$}
	{$Si(OH)_4$; $NaOH$; $Mg(OH)_2$; $Al(OH)_3$}
	\loigiai{Trong một chu kì, tính base của hydroxide giảm dần khi đi từ trái sang phải. Trong một nhóm A, tính base của hydroxide mạnh dần khi đi từ trên xuống dưới. Vậy, thứ tự giảm dần tính base là: $NaOH$ >  $Mg(OH)_2$ > $Al(OH)_3$ > $Si(OH)_4$ }
\end{ex}
%%%=============EX_35=============%%%
\begin{ex}%[0H2H1-2]
	Dãy nào sau đây sắp xếp theo thứ tự tăng dần tính acid?
	\choice
	{$H_3PO_4$; $H_2SO_4$; $H_3AsO_4$}
	{$H_2SO_4$; $H_3AsO_4$; $H_3PO_4$}
	{\True $H_3PO_4$; $H_3AsO_4$; $H_2SO_4$}
	{$H_3AsO_4$; $H_3PO_4$; $H_2SO_4$}
	\loigiai{Trong một chu kì, tính acid của oxide tăng dần khi đi từ trái sang phải. Trong một nhóm A, tính acid của oxide yếu dần khi đi từ trên xuống dưới. Vậy, thứ tự tăng dần tính acid là: $H_3PO_4$ <  $H_3AsO_4$ <  $H_2SO_4$ }
\end{ex}
%%%=============EX_36=============%%%
\begin{ex}%[0H2V2-2]
	Nguyên tố R có cấu hình electron: $1s^22s^22p^3$. Công thức hợp chất oxide ứng với hoá trị cao nhất của R và hydride (hợp chất của R với hydrogen) tương ứng là
	\choice
	{$RO_2$ và $RH_4$}
	{\True $R_2O_5$ và $RH_3$}
	{$RO_3$ và $RH_2$}
	{$R_2O_3$ và $RH_3$}
	\loigiai{Nguyên tố R có cấu hình electron: $1s^22s^22p^3$  => R thuộc nhóm VA.  Công thức hợp chất oxide ứng với hoá trị cao nhất của R và hydride (hợp chất của R với hydrogen) tương ứng là $R_2O_5$ và $RH_3$}
\end{ex}
%%%=============EX_37=============%%%
\begin{ex}%[0H2V1-4]
	Nguyên tố X ở ô thứ 17 của bảng tuần hoàn. 
	Có các phát biểu sau:
	(1) X có độ âm điện lớn và là một phi kim mạnh.
	(2) X có thể tạo thành ion bền có dạng $X^+$.
	(3) Oxide cao nhất của X có công thức $X_2O_7$ và là acidic oxide. 
	(4) Hydroxide của X có công thức $HClO_4$ và là acid mạnh.
	Trong các phát biểu trên, số phát biểu đúng là 
	\choice
	{1}
	{2}
	{\True 3}
	{4}
	\loigiai{Nguyên tố X ở ô thứ 17 là Cl. 
		(1) X có độ âm điện lớn và là một phi kim mạnh. => đúng
		(2) X có thể tạo thành ion bền có dạng $X^-$. => sai 
		(3) Oxide cao nhất của X có công thức $X_2O_7$ và là acidic oxide.  => đúng
		(4) Hydroxide của X có công thức $HClO_4$ và là acid mạnh. => đúng
		Vậy, có 3 phát biểu đúng}
\end{ex}
\Closesolutionfile{ans}
\Closesolutionfile{ansex}
%\bangdapan{Ans-C02B03-XHBDTPOX01.tex}
\phan[\maunhan]{Trắc nghiệm đúng - sai}
%%%=============SOẠN EXTF===============%%%
\Opensolutionfile{ansex}[Ans/LGTF-C02B03-XHBDTPOX01.tex]
\Opensolutionfile{ansbook}[Ansbook/AnsTF-C02B03-XHBDTPOX01.tex]
\Opensolutionfile{ans}[Ans/Tempt-C02B03-XHBDTPOX01.tex]
	%%%=============EX_1=============%%%
	\begin{ex}%[0H2H1-2]
		Về xu hướng tính acid của oxide của các nguyên tố trong một chu kỳ, nhận định nào sau đây đúng?
		\choiceTF[t]
		{\True Tính acid của oxide tăng dần từ trái sang phải trong một chu kỳ}
		{Tính acid của oxide giảm dần từ trái sang phải trong một chu kỳ}
		{\True Các oxide của kim loại kiềm thổ có tính base mạnh hơn các oxide của kim loại kiềm}
		{Các oxide của phi kim đều có tính base}
		\loigiai{
			\begin{itemchoice}[T1,F2,T3,F4]
				\itemch Trong một chu kỳ, khi đi từ trái sang phải, độ âm điện của nguyên tố tăng dần nên tính acid của oxide tăng dần
				\itemch Điều này không đúng vì tính acid tăng dần từ trái sang phải trong chu kỳ
				\itemch Kim loại kiềm thổ có điện tích hạt nhân lớn hơn và bán kính nguyên tử nhỏ hơn kim loại kiềm cùng chu kỳ, nên oxide của chúng có tính base mạnh hơn
				\itemch Các oxide của phi kim có tính acid, không có tính base
			\end{itemchoice}
		}
	\end{ex}
	%%%=============EX_2=============%%%
	\begin{ex}%[0H2H1-2]
		Về hydroxide của các nguyên tố trong một chu kỳ, những phát biểu nào sau đây đúng?
		\choiceTF[t]
		{\True Tính base của hydroxide giảm dần từ trái sang phải trong một chu kỳ}
		{\True Hydroxide của kim loại kiềm $(MOH)$ có tính base mạnh nhất trong chu kỳ}
		{Tất cả các nguyên tố trong một chu kỳ đều tạo được hydroxide bền}
		{Hydroxide của các phi kim đều có tính base yếu}
		\loigiai{
			\begin{itemchoice}[T1,T2,F3,F4]
				\itemch Do độ âm điện tăng dần từ trái sang phải trong chu kỳ, liên kết M-OH yếu dần nên tính base giảm
				\itemch Kim loại kiềm có độ âm điện thấp nhất nên tạo hydroxide có tính base mạnh nhất
				\itemch Các phi kim ở cuối chu kỳ không tạo được hydroxide bền
				\itemch Hydroxide của phi kim có tính acid, không có tính base
			\end{itemchoice}
		}
	\end{ex}
	%%%=============EX_3=============%%%
	\begin{ex}%[0H2H1-2]
		Về oxide của các nguyên tố nhóm VIIA (halogen), những nhận định nào sau đây đúng?
		\choiceTF[t]
		{\True $Cl_2O_7$ có tính acid mạnh hơn $Br_2O_7$}
		{\True Các oxide của halogen đều là oxide acid}
		{$I_2O_5$ có tính acid mạnh hơn $Cl_2O_5$}
		{\True Trong nhóm VIIA, khi đi từ trên xuống dưới, tính acid của oxide giảm dần}
		\loigiai{
			\begin{itemchoice}[T1,T2,F3,T4]
				\itemch Do Cl có độ âm điện lớn hơn Br nên $Cl_2O_7$ có tính acid mạnh hơn $Br_2O_7$
				\itemch Các halogen là phi kim mạnh nên oxide của chúng đều có tính acid
				\itemch Do I có độ âm điện nhỏ hơn Cl nên $I_2O_5$ có tính acid yếu hơn $Cl_2O_5$
				\itemch Khi đi từ trên xuống dưới nhóm VIIA, độ âm điện giảm dần nên tính acid của oxide cũng giảm dần
			\end{itemchoice}
		}
	\end{ex}
	%%%=============EX_4=============%%%
	\begin{ex}%[0H2H1-2]
		Về sự biến đổi tính chất của oxide theo chu kỳ, nhận định nào sau đây đúng?
		\choiceTF[t]
		{\True Trong một chu kỳ, nhiệt độ nóng chảy của oxide thường tăng dần từ trái sang phải}
		{Tất cả các oxide trong một chu kỳ đều tan được trong nước}
		{\True Số oxi hóa cao nhất của nguyên tố trong oxide thường tăng dần từ trái sang phải trong chu kỳ}
		{\True $SiO_2$ có nhiệt độ nóng chảy cao hơn $Al_2O_3$}
		\loigiai{
			\begin{itemchoice}[T1,F2,T3,T4]
				\itemch Liên kết trong oxide thường mạnh dần từ trái sang phải trong chu kỳ nên nhiệt độ nóng chảy tăng
				\itemch Một số oxide trung tính như $SiO_2$ không tan trong nước
				\itemch Số electron hóa trị tăng dần từ trái sang phải trong chu kỳ nên số oxi hóa cao nhất cũng tăng
				\itemch $SiO_2$ có cấu trúc mạng tinh thể khổng lồ với liên kết Si-O rất bền nên có nhiệt độ nóng chảy cao hơn $Al_2O_3$
			\end{itemchoice}
		}
	\end{ex}
	%%%=============EX_5=============%%%
	\begin{ex}%[0H2H1-2]
		Về hydroxide của các nguyên tố trong một chu kỳ, phát biểu nào sau đây đúng?
		\choiceTF[t]
		{\True Độ tan trong nước của hydroxide giảm dần từ trái sang phải trong chu kỳ}
		{Tất cả các hydroxide trong một chu kỳ đều bị phân hủy bởi nhiệt}
		{\True $Be(OH)_2$ và $Al(OH)_3$ có tính lưỡng tính}
		{\True Trong một chu kỳ, độ bền nhiệt của hydroxide giảm dần từ trái sang phải}
		\loigiai{
			\begin{itemchoice}[T1,F2,T3,T4]
				\itemch Tính base giảm dần từ trái sang phải trong chu kỳ nên độ tan của hydroxide cũng giảm
				\itemch Hydroxide của kim loại kiềm rất bền với nhiệt, không bị phân hủy
				\itemch $Be(OH)_2$ và $Al(OH)_3$ có thể phản ứng được với cả acid và base
				\itemch Liên kết M-OH yếu dần từ trái sang phải trong chu kỳ nên độ bền nhiệt giảm
			\end{itemchoice}
		}
	\end{ex}
	%%%=============EX_6=============%%%
	\begin{ex}%[0H2H1-2]
		Về xu hướng biến đổi tính acid-base của oxide và hydroxide theo chu kì, nhận định nào sau đây đúng?
		\choiceTF[t]
		{\True Tính acid của oxide tăng dần từ trái sang phải trong một chu kì}
		{Tính base của hydroxide tăng dần từ trái sang phải trong một chu kì}
		{\True Tính base của oxide giảm dần từ trái sang phải trong một chu kì}
		{\True Tính acid của hydroxide tăng dần từ trái sang phải trong một chu kì}
		\loigiai{
			\begin{itemchoice}[T1,F2,T3,T4]
				\itemch Khi đi từ trái sang phải trong một chu kì, tính kim loại giảm dần nên tính acid của oxide tăng dần.
				\itemch Tính base của hydroxide giảm dần từ trái sang phải trong một chu kì do tính kim loại giảm.
				\itemch Tính base của oxide giảm dần từ trái sang phải trong một chu kì do tính kim loại giảm.
				\itemch Tính acid của hydroxide tăng dần từ trái sang phải trong một chu kì do tính phi kim tăng.
			\end{itemchoice}
		}
	\end{ex}
	%%%=============EX_7=============%%%
	\begin{ex}%[0H2V1-2]
		Xét các nguyên tố trong chu kì 3 của bảng tuần hoàn, điều nào sau đây là đúng?
		\choiceTF[t]
		{\True $\text{Na}_2\text{O}$ có tính base mạnh hơn $\text{MgO}$}
		{$\text{Al}_2\text{O}_3$ có tính acid mạnh hơn $\text{SiO}_2$}
		{\True $\text{P}_2\text{O}_5$ có tính acid mạnh hơn $\text{SO}_3$}
		{\True $\text{NaOH}$ có tính base mạnh hơn $\text{Mg(OH)}_2$}
		\loigiai{
			\begin{itemchoice}[T1,F2,F3,T4]
				\itemch $\text{Na}_2\text{O}$ có tính base mạnh hơn $\text{MgO}$ do Na có tính kim loại mạnh hơn Mg.
				\itemch $\text{SiO}_2$ có tính acid mạnh hơn $\text{Al}_2\text{O}_3$ vì Si có tính phi kim mạnh hơn Al.
				\itemch $\text{SO}_3$ có tính acid mạnh hơn $\text{P}_2\text{O}_5$ do S nằm bên phải P trong chu kì.
				\itemch $\text{NaOH}$ có tính base mạnh hơn $\text{Mg(OH)}_2$ vì Na có tính kim loại mạnh hơn Mg.
			\end{itemchoice}
		}
	\end{ex}
	%%%=============EX_8=============%%%
	\begin{ex}%[0H2H1-2]
		Về sự biến đổi bán kính nguyên tử trong một chu kì, nhận định nào sau đây đúng?
		\choiceTF[t]
		{\True Bán kính nguyên tử giảm dần từ trái sang phải trong một chu kì}
		{Bán kính nguyên tử tăng dần từ trái sang phải trong một chu kì}
		{\True Nguyên tố kim loại kiềm có bán kính nguyên tử lớn nhất trong chu kì}
		{\True Nguyên tố khí hiếm có bán kính nguyên tử nhỏ nhất trong chu kì}
		\loigiai{
			\begin{itemchoice}[T1,F2,T3,T4]
				\itemch Bán kính nguyên tử giảm dần từ trái sang phải do điện tích hạt nhân tăng và số lớp electron không đổi.
				\itemch Bán kính nguyên tử không tăng dần mà giảm dần từ trái sang phải trong một chu kì.
				\itemch Nguyên tố kim loại kiềm ở đầu chu kì có bán kính nguyên tử lớn nhất do có ít electron và lực hút hạt nhân yếu nhất.
				\itemch Nguyên tố khí hiếm ở cuối chu kì có bán kính nguyên tử nhỏ nhất do có nhiều electron và lực hút hạt nhân mạnh nhất.
			\end{itemchoice}
		}
	\end{ex}
	%%%=============EX_9=============%%%
	\begin{ex}%[0H2N1-2]
		Về xu hướng biến đổi độ âm điện trong một chu kì, điều nào sau đây là đúng?
		\choiceTF[t]
		{\True Độ âm điện tăng dần từ trái sang phải trong một chu kì}
		{Nguyên tố kim loại kiềm có độ âm điện lớn nhất trong chu kì}
		{\True Nguyên tố halogen có độ âm điện lớn nhất trong chu kì}
		{\True Độ âm điện của các nguyên tố liên quan đến khả năng hút electron khi tạo liên kết}
		\loigiai{
			\begin{itemchoice}[T1,F2,T3,T4]
				\itemch Độ âm điện tăng dần từ trái sang phải do tính phi kim tăng dần.
				\itemch Nguyên tố kim loại kiềm có độ âm điện nhỏ nhất, không phải lớn nhất trong chu kì.
				\itemch Nguyên tố halogen (trừ Flo) có độ âm điện lớn nhất trong chu kì do có khả năng hút electron mạnh nhất.
				\itemch Độ âm điện phản ánh khả năng hút electron của nguyên tử khi tạo liên kết hóa học.
			\end{itemchoice}
		}
	\end{ex}
	%%%=============EX_10=============%%%
	\begin{ex}%[0H2H1-2]
		Về sự biến đổi năng lượng ion hóa thứ nhất trong một chu kì, nhận định nào sau đây đúng?
		\choiceTF[t]
		{\True Năng lượng ion hóa thứ nhất tăng dần từ trái sang phải trong một chu kì}
		{Nguyên tố kim loại kiềm có năng lượng ion hóa thứ nhất lớn nhất trong chu kì}
		{\True Nguyên tố khí hiếm có năng lượng ion hóa thứ nhất lớn nhất trong chu kì}
		{\True Năng lượng ion hóa thứ nhất liên quan đến khả năng giữ electron ngoài cùng của nguyên tử}
		\loigiai{
			\begin{itemchoice}[T1,F2,T3,T4]
				\itemch Năng lượng ion hóa thứ nhất tăng dần từ trái sang phải do lực hút hạt nhân tăng.
				\itemch Nguyên tố kim loại kiềm có năng lượng ion hóa thứ nhất nhỏ nhất, không phải lớn nhất trong chu kì.
				\itemch Nguyên tố khí hiếm có cấu hình electron bền vững nên có năng lượng ion hóa thứ nhất lớn nhất trong chu kì.
				\itemch Năng lượng ion hóa thứ nhất phản ánh khả năng giữ electron ngoài cùng của nguyên tử.
			\end{itemchoice}
		}
	\end{ex}
	%%%=============EX_11=============%%%
	\begin{ex}%[0H2H1-2]
		Về xu hướng biến đổi tính kim loại trong một chu kì, điều nào sau đây là đúng?
		\choiceTF[t]
		{\True Tính kim loại giảm dần từ trái sang phải trong một chu kì}
		{Tính kim loại tăng dần từ trái sang phải trong một chu kì}
		{\True Nguyên tố kim loại kiềm có tính kim loại mạnh nhất trong chu kì}
		{\True Tính kim loại liên quan đến khả năng nhường electron của nguyên tử}
		\loigiai{
			\begin{itemchoice}[T1,F2,T3,T4]
				\itemch Tính kim loại giảm dần từ trái sang phải do khả năng nhường electron giảm.
				\itemch Tính kim loại không tăng dần mà giảm dần từ trái sang phải trong một chu kì.
				\itemch Nguyên tố kim loại kiềm ở đầu chu kì có tính kim loại mạnh nhất do dễ nhường electron nhất.
				\itemch Tính kim loại phản ánh khả năng nhường electron của nguyên tử khi tham gia phản ứng hóa học.
			\end{itemchoice}
		}
	\end{ex}
	%%%=============EX_12=============%%%
	\begin{ex}%[0H2H1-2]
		Về sự biến đổi tính chất của oxide trong chu kì 3, nhận định nào sau đây đúng?
		\choiceTF[t]
		{\True $\text{Na}_2\text{O}$ và $\text{MgO}$ là oxide base}
		{\True $\text{Al}_2\text{O}_3$ là oxide lưỡng tính}
		{\True $\text{SiO}_2$, $\text{P}_4\text{O}_{10}$, $\text{SO}_3$, và $\text{Cl}_2\text{O}_7$ là oxide acid}
		{$\text{SiO}_2$ có tính acid mạnh hơn $\text{P}_4\text{O}_{10}$}
		\loigiai{
			\begin{itemchoice}[T1,T2,T3,F4]
				\itemch $\text{Na}_2\text{O}$ và $\text{MgO}$ là oxide của kim loại nên có tính base.
				\itemch $\text{Al}_2\text{O}_3$ là oxide lưỡng tính, có thể phản ứng với cả acid và base.
				\itemch $\text{SiO}_2$, $\text{P}_4\text{O}_{10}$, $\text{SO}_3$, và $\text{Cl}_2\text{O}_7$ là oxide của phi kim nên có tính acid.
				\itemch $\text{P}_4\text{O}_{10}$ có tính acid mạnh hơn $\text{SiO}_2$ do P nằm bên phải Si trong chu kì.
			\end{itemchoice}
		}
	\end{ex}
	%%%=============EX_13=============%%%
	\begin{ex}%[0H2H1-2]
		Về sự biến đổi tính chất của hydroxide trong chu kì 3, điều nào sau đây là đúng?
		\choiceTF[t]
		{\True $\text{NaOH}$ và $\text{Mg(OH)}_2$ là hydroxide base mạnh}
		{\True $\text{Al(OH)}_3$ là hydroxide lưỡng tính}
		{$\text{Si(OH)}_4$ là hydroxide base}
		{\True $\text{H}_3\text{PO}_4$, $\text{H}_2\text{SO}_4$, và $\text{HClO}_4$ là acid mạnh}
		\loigiai{
			\begin{itemchoice}[T1,T2,F3,T4]
				\itemch $\text{NaOH}$ và $\text{Mg(OH)}_2$ là hydroxide của kim loại kiềm và kiềm thổ nên là base mạnh.
				\itemch $\text{Al(OH)}_3$ là hydroxide lưỡng tính, có thể phản ứng với cả acid và base.
				\itemch $\text{Si(OH)}_4$ (hay $\text{H}_4\text{SiO}_4$) là một acid yếu, không phải hydroxide base.
				\itemch $\text{H}_3\text{PO}_4$, $\text{H}_2\text{SO}_4$, và $\text{HClO}_4$ là acid mạnh của các phi kim P, S, và Cl.
			\end{itemchoice}
		}
	\end{ex}
	Tôi sẽ tiếp tục tạo các câu hỏi trắc nghiệm theo yêu cầu của bạn.
	
	%%%=============EX_14=============%%%
	\begin{ex}%[0H2H1-2]
		Về sự biến đổi tính acid-base của các hợp chất trong chu kì 3, nhận định nào sau đây đúng?
		\choiceTF[t]
		{\True Tính acid của hydroxide tăng dần theo thứ tự: $\text{NaOH} < \text{Mg(OH)}_2 < \text{Al(OH)}_3 < \text{H}_4\text{SiO}_4 < \text{H}_3\text{PO}_4 < \text{H}_2\text{SO}_4 < \text{HClO}_4$}
		{\True Tính base của oxide giảm dần theo thứ tự: $\text{Na}_2\text{O} > \text{MgO} > \text{Al}_2\text{O}_3 > \text{SiO}_2 > \text{P}_4\text{O}_{10} > \text{SO}_3 > \text{Cl}_2\text{O}_7$}
		{$\text{Al}_2\text{O}_3$ và $\text{Al(OH)}_3$ đều là hợp chất có tính base mạnh}
		{\True $\text{SiO}_2$ là oxide acid yếu, trong khi $\text{P}_4\text{O}_{10}$ là oxide acid mạnh}
		\loigiai{
			\begin{itemchoice}[T1,T2,F3,T4]
				\itemch Tính acid của hydroxide tăng dần từ trái sang phải trong chu kì 3, tương ứng với sự giảm dần tính kim loại.
				\itemch Tính base của oxide giảm dần từ trái sang phải trong chu kì 3, tương ứng với sự tăng dần tính phi kim.
				\itemch $\text{Al}_2\text{O}_3$ và $\text{Al(OH)}_3$ là hợp chất lưỡng tính, không phải là hợp chất có tính base mạnh.
				\itemch $\text{SiO}_2$ là oxide acid yếu vì Si có tính phi kim yếu, trong khi $\text{P}_4\text{O}_{10}$ là oxide acid mạnh do P có tính phi kim mạnh hơn.
			\end{itemchoice}
		}
	\end{ex}
	%%%=============EX_15=============%%%
	\begin{ex}%[0H2H1-2]
		Về sự biến đổi của các đơn chất trong chu kì 3, điều nào sau đây là đúng?
		\choiceTF[t]
		{\True Tính kim loại giảm dần theo thứ tự: $\text{Na} > \text{Mg} > \text{Al} > \text{Si}$}
		{\True Tính phi kim tăng dần theo thứ tự: $\text{Si} < \text{P} < \text{S} < \text{Cl}$}
		{Argon (Ar) có tính phi kim mạnh nhất trong chu kì 3}
		{\True Silicon (Si) có thể thể hiện cả tính chất kim loại và phi kim}
		\loigiai{
			\begin{itemchoice}[T1,T2,F3,T4]
				\itemch Tính kim loại giảm dần từ Na đến Si do số proton tăng và lực hút electron tăng.
				\itemch Tính phi kim tăng dần từ Si đến Cl do khả năng hút electron tăng.
				\itemch Argon (Ar) là khí hiếm, không thể hiện tính phi kim. Cl có tính phi kim mạnh nhất trong chu kì 3.
				\itemch Silicon (Si) là á kim, có thể thể hiện cả tính chất kim loại và phi kim tùy theo điều kiện.
			\end{itemchoice}
		}
	\end{ex}
	%%%=============EX_16=============%%%
	\begin{ex}%[0H2H1-2]
		Về sự biến đổi số oxi hóa cao nhất của các nguyên tố trong chu kì 3, nhận định nào sau đây đúng?
		\choiceTF[t]
		{\True Số oxi hóa cao nhất tăng dần từ $\text{Na}$ đến $\text{P}$}
		{\True Số oxi hóa cao nhất của $\text{Na}$, $\text{Mg}$, $\text{Al}$ lần lượt là $+1$, $+2$, $+3$}
		{Số oxi hóa cao nhất của $\text{S}$ và $\text{Cl}$ đều là $+6$}
		{\True Số oxi hóa cao nhất của $\text{Si}$, $\text{P}$, $\text{S}$, $\text{Cl}$ lần lượt là $+4$, $+5$, $+6$, $+7$}
		\loigiai{
			\begin{itemchoice}[T1,T2,F3,T4]
				\itemch Số oxi hóa cao nhất tăng dần từ $\text{Na}$ ($+1$) đến $\text{P}$ ($+5$) do số electron hóa trị tăng.
				\itemch Số oxi hóa cao nhất của $\text{Na}$, $\text{Mg}$, $\text{Al}$ tương ứng với số electron hóa trị của chúng.
				\itemch Số oxi hóa cao nhất của $\text{S}$ là $+6$, nhưng của $\text{Cl}$ là $+7$.
				\itemch Số oxi hóa cao nhất của $\text{Si}$, $\text{P}$, $\text{S}$, $\text{Cl}$ tương ứng với số electron hóa trị tối đa của chúng.
			\end{itemchoice}
		}
	\end{ex}
	%%%=============EX_17=============%%%
	\begin{ex}%[0H2H1-2]
		Về sự biến đổi tính chất của các hiđride trong chu kì 3, điều nào sau đây là đúng?
		\choiceTF[t]
		{\True Tính acid của hiđride tăng dần theo thứ tự: $\text{NaH} < \text{MgH}_2 < \text{AlH}_3 < \text{SiH}_4 < \text{PH}_3 < \text{H}_2\text{S} < \text{HCl}$}
		{$\text{NaH}$ và $\text{MgH}_2$ là những hiđride có tính acid}
		{\True $\text{AlH}_3$ là hiđride lưỡng tính}
		{\True $\text{HCl}$ là hiđride có tính acid mạnh nhất trong chu kì 3}
		\loigiai{
			\begin{itemchoice}[T1,F2,T3,T4]
				\itemch Tính acid của hiđride tăng dần từ trái sang phải trong chu kì 3, tương ứng với sự giảm tính kim loại và tăng tính phi kim.
				\itemch $\text{NaH}$ và $\text{MgH}_2$ là những hiđride có tính base, không phải tính acid.
				\itemch $\text{AlH}_3$ là hiđride lưỡng tính, có thể thể hiện cả tính acid và base tùy theo điều kiện.
				\itemch $\text{HCl}$ là hiđride của Cl - nguyên tố có tính phi kim mạnh nhất trong chu kì 3, nên có tính acid mạnh nhất.
			\end{itemchoice}
		}
	\end{ex}
	%%%=============EX_18=============%%%
	\begin{ex}%[0H2V1-2]
		Về sự biến đổi nhiệt độ nóng chảy của các nguyên tố trong chu kì 3, nhận định nào sau đây đúng?
		\choiceTF[t]
		{\True Nhiệt độ nóng chảy tăng dần từ $\text{Na}$ đến $\text{Si}$}
		{\True $\text{Si}$ có nhiệt độ nóng chảy cao nhất trong chu kì 3}
		{Nhiệt độ nóng chảy tăng đều từ $\text{Na}$ đến $\text{Ar}$}
		{\True Nhiệt độ nóng chảy giảm mạnh từ $\text{Si}$ đến $\text{Ar}$}
		\loigiai{
			\begin{itemchoice}[T1,T2,F3,T4]
				\itemch Nhiệt độ nóng chảy tăng dần từ $\text{Na}$ đến $\text{Si}$ do liên kết kim loại và liên kết cộng hóa trị mạnh dần.
				\itemch $\text{Si}$ có cấu trúc mạng tinh thể cộng hóa trị 3D nên có nhiệt độ nóng chảy cao nhất trong chu kì 3.
				\itemch Nhiệt độ nóng chảy không tăng đều mà có xu hướng tăng từ $\text{Na}$ đến $\text{Si}$, sau đó giảm từ $\text{P}$ đến $\text{Ar}$.
				\itemch Nhiệt độ nóng chảy giảm mạnh từ $\text{Si}$ đến $\text{Ar}$ do chuyển từ cấu trúc mạng sang phân tử và nguyên tử.
			\end{itemchoice}
		}
	\end{ex}
	%%%=============EX_19=============%%%
	\begin{ex}%[0H2H1-2]
		Về sự biến đổi độ âm điện của các nguyên tố trong chu kì 3, điều nào sau đây là đúng?
		\choiceTF[t]
		{\True Độ âm điện tăng dần từ $\text{Na}$ đến $\text{Cl}$}
		{$\text{Ar}$ có độ âm điện lớn nhất trong chu kì 3}
		{\True $\text{Cl}$ có độ âm điện lớn nhất trong các nguyên tố của chu kì 3}
		{\True Sự chênh lệch độ âm điện giữa hai nguyên tố liên tiếp lớn nhất là giữa $\text{Na}$ và $\text{Mg}$}
		\loigiai{
			\begin{itemchoice}[T1,F2,T3,T4]
				\itemch Độ âm điện tăng dần từ $\text{Na}$ đến $\text{Cl}$ do tính phi kim tăng dần.
				\itemch $\text{Ar}$ là khí hiếm, không có giá trị độ âm điện. $\text{Cl}$ có độ âm điện lớn nhất trong chu kì 3.
				\itemch $\text{Cl}$ có độ âm điện lớn nhất trong chu kì 3 do có khả năng hút electron mạnh nhất.
				\itemch Sự chênh lệch độ âm điện giữa $\text{Na}$ và $\text{Mg}$ lớn nhất do có sự thay đổi đáng kể về cấu hình electron.
			\end{itemchoice}
		}
	\end{ex}
	%%%=============EX_20=============%%%
	\begin{ex}%[0H2H1-2]
		Về sự biến đổi bán kính nguyên tử của các nguyên tố trong chu kì 3, nhận định nào sau đây đúng?
		\choiceTF[t]
		{\True Bán kính nguyên tử giảm dần từ $\text{Na}$ đến $\text{Ar}$}
		{Bán kính nguyên tử của $\text{Mg}$ lớn hơn $\text{Na}$}
		{\True $\text{Na}$ có bán kính nguyên tử lớn nhất trong chu kì 3}
		{\True Sự giảm bán kính nguyên tử chậm lại khi đi từ $\text{P}$ đến $\text{Ar}$}
		\loigiai{
			\begin{itemchoice}[T1,F2,T3,T4]
				\itemch Bán kính nguyên tử giảm dần từ $\text{Na}$ đến $\text{Ar}$ do điện tích hạt nhân tăng và số lớp electron không đổi.
				\itemch Bán kính nguyên tử của $\text{Mg}$ nhỏ hơn $\text{Na}$ do $\text{Mg}$ có điện tích hạt nhân lớn hơn.
				\itemch $\text{Na}$ có bán kính nguyên tử lớn nhất trong chu kì 3 do có điện tích hạt nhân nhỏ nhất và lực hút electron yếu nhất.
				\itemch Sự giảm bán kính nguyên tử chậm lại từ $\text{P}$ đến $\text{Ar}$ do sự tăng đẩy electron giữa các electron trong cùng phân lớp $3p$.
			\end{itemchoice}
		}
	\end{ex}
	%%%=============EX_21=============%%%
	\begin{ex}%[0H2H1-2]
		Về sự biến đổi năng lượng ion hóa thứ nhất của các nguyên tố trong chu kì 3, điều nào sau đây là đúng?
		\choiceTF[t]
		{\True Năng lượng ion hóa thứ nhất tăng dần từ $\text{Na}$ đến $\text{Ar}$}
		{Năng lượng ion hóa thứ nhất của $\text{Mg}$ lớn hơn $\text{Al}$}
		{\True $\text{Ar}$ có năng lượng ion hóa thứ nhất lớn nhất trong chu kì 3}
		{\True Năng lượng ion hóa thứ nhất của $\text{Al}$ nhỏ hơn $\text{Mg}$ do electron bị ion hóa ở $\text{Al}$ nằm ở phân lớp $3p$}
		\loigiai{
			\begin{itemchoice}[T1,F2,T3,T4]
				\itemch Năng lượng ion hóa thứ nhất tăng dần từ $\text{Na}$ đến $\text{Ar}$ do lực hút hạt nhân tăng dần.
				\itemch Năng lượng ion hóa thứ nhất của $\text{Mg}$ lớn hơn $\text{Al}$ do cấu hình electron bền vững hơn của $\text{Mg}$ ($3s^2$).
				\itemch $\text{Ar}$ có năng lượng ion hóa thứ nhất lớn nhất trong chu kì 3 do có cấu hình electron đặc biệt bền vững (khí hiếm).
				\itemch Năng lượng ion hóa thứ nhất của $\text{Al}$ nhỏ hơn $\text{Mg}$ vì electron bị ion hóa ở $\text{Al}$ nằm ở phân lớp $3p$, xa hạt nhân hơn so với electron ở phân lớp $3s$ của $\text{Mg}$.
			\end{itemchoice}
		}
	\end{ex}
	%%%=============EX_22=============%%%
	\begin{ex}%[0H2H1-2]
		Về sự biến đổi tính chất của các halogenua trong chu kì 3, nhận định nào sau đây đúng?
		\choiceTF[t]
		{\True Tính ion của liên kết trong các halogenua giảm dần từ $\text{NaCl}$ đến $\text{PCl}_3$}
		{\True $\text{NaCl}$ và $\text{MgCl}_2$ là những hợp chất ion điển hình}
		{$\text{AlCl}_3$ là hợp chất có liên kết cộng hóa trị không phân cực}
		{\True $\text{SiCl}_4$ và $\text{PCl}_3$ là những hợp chất có liên kết cộng hóa trị phân cực}
		\loigiai{
			\begin{itemchoice}[T1,T2,F3,T4]
				\itemch Tính ion của liên kết giảm dần từ $\text{NaCl}$ đến $\text{PCl}_3$ do sự giảm chênh lệch độ âm điện giữa các nguyên tố.
				\itemch $\text{NaCl}$ và $\text{MgCl}_2$ là hợp chất ion điển hình do chênh lệch độ âm điện lớn giữa kim loại và Cl.
				\itemch $\text{AlCl}_3$ là hợp chất có liên kết cộng hóa trị phân cực, không phải không phân cực.
				\itemch $\text{SiCl}_4$ và $\text{PCl}_3$ có liên kết cộng hóa trị phân cực do chênh lệch độ âm điện giữa Si, P và Cl.
			\end{itemchoice}
		}
	\end{ex}
	%%%=============EX_23=============%%%
	\begin{ex}%[0H2H1-2]
		Về sự biến đổi tính chất của các oxit cao nhất trong chu kì 3, điều nào sau đây là đúng?
		\choiceTF[t]
		{\True Tính base của oxit giảm dần theo thứ tự: $\text{Na}_2\text{O} > \text{MgO} > \text{Al}_2\text{O}_3$}
		{\True Tính acid của oxit tăng dần theo thứ tự: $\text{SiO}_2 < \text{P}_4\text{O}_{10} < \text{SO}_3 < \text{Cl}_2\text{O}_7$}
		{\True $\text{Al}_2\text{O}_3$ là oxit lưỡng tính}
		{$\text{SiO}_2$ là oxit base yếu}
		\loigiai{
			\begin{itemchoice}[T1,T2,T3,F4]
				\itemch Tính base của oxit giảm dần từ $\text{Na}_2\text{O}$ đến $\text{Al}_2\text{O}_3$ do tính kim loại giảm dần.
				\itemch Tính acid của oxit tăng dần từ $\text{SiO}_2$ đến $\text{Cl}_2\text{O}_7$ do tính phi kim tăng dần.
				\itemch $\text{Al}_2\text{O}_3$ là oxit lưỡng tính, có thể phản ứng với cả acid và base.
				\itemch $\text{SiO}_2$ là oxit acid yếu, không phải oxit base yếu.
			\end{itemchoice}
		}
	\end{ex}
	%%%=============EX_24=============%%%
	\begin{ex}%[0H2H1-2]
		Về sự biến đổi của các hợp chất khí với hidro trong chu kì 3, nhận định nào sau đây đúng?
		\choiceTF[t]
		{\True Nhiệt độ sôi của các hợp chất khí với hidro tăng dần theo thứ tự: $\text{PH}_3 < \text{H}_2\text{S} < \text{HCl}$}
		{\True Tính acid của các hợp chất khí với hidro tăng dần theo thứ tự: $\text{PH}_3 < \text{H}_2\text{S} < \text{HCl}$}
		{$\text{PH}_3$ có tính khử mạnh hơn $\text{H}_2\text{S}$}
		{\True $\text{HCl}$ là acid mạnh nhất trong số các hợp chất khí với hidro của chu kì 3}
		\loigiai{
			\begin{itemchoice}[T1,T2,F3,T4]
				\itemch Nhiệt độ sôi tăng dần do khối lượng phân tử và lực tương tác giữa các phân tử tăng dần.
				\itemch Tính acid tăng dần do tính phi kim của nguyên tố trung tâm tăng dần từ P đến Cl.
				\itemch $\text{H}_2\text{S}$ có tính khử mạnh hơn $\text{PH}_3$ do S có độ âm điện thấp hơn P.
				\itemch $\text{HCl}$ là acid mạnh nhất do Cl có tính phi kim mạnh nhất trong chu kì 3.
			\end{itemchoice}
		}
	\end{ex}
	%%%=============EX_25=============%%%
	\begin{ex}%[0H2H1-2]
		Về sự biến đổi của các hợp chất clo trong chu kì 3, điều nào sau đây là đúng?
		\choiceTF[t]
		{\True Nhiệt độ nóng chảy của các clorua giảm dần theo thứ tự: $\text{NaCl} > \text{MgCl}_2 > \text{AlCl}_3 > \text{SiCl}_4 > \text{PCl}_3$}
		{\True $\text{NaCl}$ và $\text{MgCl}_2$ tan tốt trong nước, trong khi $\text{AlCl}_3$ bị thủy phân một phần}
		{$\text{SiCl}_4$ và $\text{PCl}_3$ không phản ứng với nước}
		{\True Tính oxi hóa của clo trong các hợp chất tăng dần theo thứ tự: $\text{NaCl} < \text{MgCl}_2 < \text{AlCl}_3 < \text{SiCl}_4 < \text{PCl}_3$}
		\loigiai{
			\begin{itemchoice}[T1,T2,F3,T4]
				\itemch Nhiệt độ nóng chảy giảm dần do chuyển từ liên kết ion sang liên kết cộng hóa trị và cấu trúc phân tử.
				\itemch $\text{NaCl}$ và $\text{MgCl}_2$ là hợp chất ion nên tan tốt trong nước. $\text{AlCl}_3$ bị thủy phân một phần do tính acid yếu của $\text{Al}^{3+}$.
				\itemch $\text{SiCl}_4$ và $\text{PCl}_3$ đều phản ứng mạnh với nước, tạo thành các oxit tương ứng và $\text{HCl}$.
				\itemch Tính oxi hóa của clo tăng dần do số oxi hóa của nguyên tố trung tâm tăng dần từ $\text{Na}$ đến $\text{P}$.
			\end{itemchoice}
		}
	\end{ex}
	%%%=============EX_26=============%%%
	\begin{ex}%[0H2H1-2]
		Hydroxide nào có tính base yếu nhất trong các hydroxide sau đây? Cho biết hợp chất này được sử dụng trong thuốc nhuộm và điều chế các chất hóa học khác.
		\choice
		{Potassium hydroxide}
		{Sodium hydroxide}
		{Calcium hydroxide}
		{\True Aluminium hydroxide}
		\loigiai{Trong các hydroxide đã cho, nhôm hydroxide (Al(OH)$_3$) có tính base yếu nhất.}
	\end{ex}
	%%%=============EX_27=============%%%
	\begin{ex}%[0H2H1-2]
		Hydroxide nào có tính acid yếu nhất trong các hydroxide sau đây? Cho biết hợp chất này được dùng để phân huỷ các quặng phức tạp; phân tích khoáng vật hoặc làm chất xúc tác.
		\choice
		{Calcium hydroxide}
		{Sodium hydroxide}
		{Potassium hydroxide}
		{\True Lithium hydroxide}
		\loigiai{Tính acid yếu nhất thuộc về lithium hydroxide (LiOH).}
	\end{ex}
	%%%=============EX_28=============%%%
	\begin{ex}
		Tính axit của các oxit phi kim trong cùng một chu kì:
		\choiceTF[t]
		{\True Tăng dần từ trái sang phải.}
		{\True Không đổi vì tất cả các oxit đều có tính axit.}
		{Giảm dần từ trái sang phải.}
		{Giảm mạnh ở các oxit của nhóm cuối cùng do tính phi kim yếu hơn.}
		\loigiai{Tính axit của các oxit phi kim tăng dần từ trái sang phải trong cùng một chu kì do độ âm điện của các nguyên tố tăng.}
	\end{ex}
\Closesolutionfile{ans}
\Closesolutionfile{ansbook}
\Closesolutionfile{ansex}
%\bangdapanTF{AnsTF-C02B03-XHBDTPOX01.tex}
\phan[\mycolor]{Bài tập tự luận}
%%%=============SOẠN BT===============%%%
\Opensolutionfile{ansbth}[Ans/LGBT-C02B03-XHBDTPOX01.tex]
\Opensolutionfile{ansbt}[Ans/AnsBT-C02B03-XHBDTPOX01.tex]
	%%%==============Bai_BT7_10==============%%%
	\begin{bt}%[0H2H2-6]
		\begin{enumerate}[a)]
			\item Nêu quan hệ giữa hóa trị của các nguyên tố hóa học với thành phần của các oxide và hydroxide của chúng.
			\item Nêu sự biến đổi hóa trị của các nguyên tố hóa học trong chu kì 3.
		\end{enumerate}
		\loigiai{
			\begin{enumerate}[a)]
				\item Hóa trị của nguyên tố hóa học với oxi sẽ quyết định đến công thức hóa học của oxide và hydroxide của chúng. Ví dụ: Nguyên tố Na có hóa trị I với oxi do đó oxide tương ứng là $Na_2O$, hydroxide tương ứng là NaOH. Nguyên tố S có hóa trị cao nhất với oxi là VI do đó oxide tương ứng là $SO_3$, hydroxide tương ứng là $H_2SO_4$.
				\item Sự biến đổi hóa trị của các nguyên tố hóa học trong chu kì 3 được thể hiện ở bảng sau:
				\begin{center}
					\begin{tabular}{|c|c|c|c|c|c|c|c|}
						\hline
						Nguyên tố & Na & Mg & Al & Si & P & S & Cl \\
						\hline
						Hóa trị cao nhất với oxi & I & II & III & IV & V & VI & VII \\
						\hline
					\end{tabular}
				\end{center}
			\end{enumerate}
		}
	\end{bt}
	
	%%%==============Bai_BT7_11==============%%%
	\begin{bt}%[0H2H1-3]
		Hãy nêu sự biến đổi tính chất acid – base của các oxide và hydroxide của các nguyên tố trong chu kì 3 khi đi từ trái sang phải.
		\loigiai{
			Tính acid của các oxide và hydroxide của các nguyên tố trong chu kì 3 tăng dần khi đi từ trái sang phải. Ngược lại, tính base của chúng giảm dần.
		}
	\end{bt}
	
	%%%==============Bai_BT7_12==============%%%
	\begin{bt}%[0H2H1-2]
		Cho các hợp chất sau: $Al_2O_3$, $Na_2O$, $SiO_2$, MgO, $SO_3$, $P_2O_5$, $Cl_2O_7$.
		Hãy sắp xếp theo xu hướng biến đổi tính acid – base. Giải thích.
		\loigiai{
			Trật tự sắp xếp theo xu hướng biến đổi tính acid - base:
			$$Na_2O < MgO < Al_2O_3 < SiO_2 < P_2O_5 < SO_3 < Cl_2O_7$$
			Giải thích:
			- Tính acid của oxide càng mạnh khi phi kim thuộc oxide càng mạnh.
			- Trong một chu kì, tính phi kim tăng dần khi đi từ trái sang phải.
		}
	\end{bt}
	
	%%%==============Bai_BT7_13==============%%%
	\begin{bt}%[0H2H1-2]
		Sắp xếp các hợp chất sau theo xu hướng biến đổi tính acid – base: NaOH, $H_2SiO_3$, $HClO_4$, $Mg(OH)_2$, $Al(OH)_3$, $H_2SO_4$.
		\loigiai{
			Trật tự sắp xếp theo xu hướng biến đổi tính acid - base:
			$$NaOH < Mg(OH)_2 < Al(OH)_3 < H_2SiO_3 < H_2SO_4 < HClO_4$$
			Giải thích:
			\begin{itemize}
				\item  Tính acid của hydroxide càng mạnh khi phi kim thuộc hydroxide càng mạnh. Ngược lại, tính base của hydroxide càng mạnh khi kim loại thuộc hydroxide càng mạnh.
				\item  Trong một chu kì, tính kim loại giảm dần và tính phi kim tăng dần khi đi từ trái sang phải.
			\end{itemize}
		}
	\end{bt}
	
	%%%==============Bai_BT7_14==============%%%
	\begin{bt}%[0H2V1-2]
		So sánh tính base của các hydroxide trong mỗi dãy sau và giải thích ngắn gọn:
		\begin{enumerate}[a)]
			\item Calcium hydroxide, strontium hydroxide và barium hydroxide;
			\item Sodium hydroxide và aluminium hydroxide;
			\item Calcium hydroxide và caesium hydroxide.
		\end{enumerate}
		\loigiai{
			\begin{enumerate}[a)]
				\item Tính base tăng dần theo dãy: $Ca(OH)_2 < Sr(OH)_2 < Ba(OH)_2$. Trong một nhóm A, tính kim loại tăng dần khi đi từ trên xuống dưới.
				\item $NaOH$ có tính base mạnh hơn $Al(OH)_3$. Trong một chu kì, tính kim loại giảm dần khi đi từ trái sang phải.
				\item $CsOH$ có tính base mạnh hơn $Ca(OH)_2$. $Cs$ ở chu kì sau $Ca$ nên có tính kim loại mạnh hơn.
			\end{enumerate}
		}
	\end{bt}
	
	%%%==============Bai_BT7_15==============%%%
	\begin{bt}%[0H2H1-2]
		Hãy so sánh tính acid của các chất trong mỗi dãy sau và giải thích ngắn gọn:
		\begin{enumerate}[a)]
			\item Carbonic acid và silicic acid.
			\item Sulfuric acid, selenic acid và teluric acid.
			\item Silicic acid, phosphoric acid và sulfuric acid.
		\end{enumerate}
		\loigiai{
			\begin{enumerate}[a)]
				\item $H_2CO_3$ có tính acid mạnh hơn $H_2SiO_3$. Trong một chu kì, tính phi kim tăng dần khi đi từ trái sang phải.
				\item Tính acid giảm dần theo dãy: $H_2SO_4 > H_2SeO_4 > H_2TeO_4$. Trong một nhóm A, tính phi kim giảm dần khi đi từ trên xuống dưới.
				\item Tính acid tăng dần theo dãy: $H_2SiO_3 < H_3PO_4 < H_2SO_4$. Trong một chu kì, tính phi kim tăng dần khi đi từ trái sang phải.
			\end{enumerate}
		}
	\end{bt}
	
	%%%==============Bai_BT7_16==============%%%
	\begin{bt}%[0H2V1-2]
		Cho các oxide sau: $Na_2O$, $SO_3$, $Cl_2O_7$, $CO_2$, $CaO$, $N_2O_5$.
		Viết các phương trình hóa học biểu diễn phản ứng với nước (nếu có) của các oxide trên và nhận xét về tính chất acid – base của chúng.
		\loigiai{
			- Các oxide phản ứng với nước:
			\begin{align*}
				Na_2O + H_2O &\longrightarrow 2NaOH \\
				SO_3 + H_2O &\longrightarrow H_2SO_4 \\
				Cl_2O_7 + H_2O &\longrightarrow 2HClO_4 \\
				CO_2 + H_2O &\rightleftharpoons H_2CO_3 \\
				CaO + H_2O &\longrightarrow Ca(OH)_2 \\
				N_2O_5 + H_2O &\longrightarrow 2HNO_3
			\end{align*}
			- Nhận xét:
			\begin{itemize}
				\item $Na_2O$, $CaO$ là oxide base.
				\item $SO_3$, $Cl_2O_7$, $CO_2$, $N_2O_5$ là oxide acid.
			\end{itemize}
		}
	\end{bt}
	
\Closesolutionfile{ansbt}
\Closesolutionfile{ansbth}
%\bangdapanSA{AnsBT-C02B03-XHBDTPOX01.tex}
%%%%%%%========================Dạng 2=============================%%%%
\begin{dang}{Bài toán xác định công thức oxide cao nhất và công thức hợp chất khí với Hidro}
\end{dang}
\begin{pp}
	\Noibat[\maunhan][\myfont{qag}][\faAndroid]{Kiến thức cần nhớ:}
		\begin{enumerate}
			\item  \textbf{Quy tắc bát tử:} Nguyên tử của nguyên tố có xu hướng kết hợp với nguyên tử khác để đạt được cấu hình electron bền vững của khí hiếm gần nhất (8 electron lớp ngoài cùng).
			\item  \textbf{Hóa trị của nguyên tố:}
			\begin{itemize}
				\item  Trong oxit cao nhất, nguyên tố phi kim có hóa trị bằng số thứ tự của nhóm A.
				\item  Trong hợp chất khí với hidro, nguyên tố phi kim có hóa trị bằng (8 - số thứ tự của nhóm A).
			\end{itemize}
			\item  \textbf{Ngoại lệ:}
			
			Oxi (O) chỉ có hóa trị II.
			Flo (F) chỉ có hóa trị I trong các hợp chất.
			Một số nguyên tố nhóm B có nhiều hóa trị.
		\end{enumerate}
	\Noibat[\maunhan][\myfont{qag}][\faAndroid]{Cách viết công thức oxit cao nhất và hợp chất khí với hidro:}
		\begin{cacbuoc}
			\item Xác định số thứ tự nhóm của nguyên tố trong bảng tuần hoàn:

			Từ số hiệu nguyên tử, tên, kí hiệu hóa học của nguyên tố, cấu hình e xác định số thứ tự nhóm của các nguyên tố trong bảng tuần hoàn.
			\item  Viết công thức chung:
			\begin{itemize}
				\item  Công thức oxit cao nhất: $R_2O_n$ (R là nguyên tố phi kim, n là hóa trị của R trong oxit cao nhất).
				\item  Công thức hợp chất khí với hidro: $RH_{8-n}$ (n là hóa trị của R trong oxit cao nhất).
			\end{itemize}
			\item  Thay giá trị:
			\begin{itemize}
				\item  Thay giá trị của n (hóa trị của R) vào công thức chung.
				\item  Lưu ý trường hợp ngoại lệ của O và F.
			\end{itemize}
			\item  Rút gọn công thức (nếu có thể):
			
			Rút gọn chỉ số của các nguyên tố về tỉ lệ tối giản.
		\end{cacbuoc}
\end{pp}
	\Noibat[\maunhan][][\faBookmark]{Ví dụ mẫu}
	%%%==============Vidu_EX1==============%%%
	\begin{vdex}
		Nguyên tố $R$ có cấu hình electron: $1\mathrm{s}^22\mathrm{s}^22p^3$. Công thức hợp chất oxide ứng với hoá trị cao nhất của $R$ và hydride (hợp chất của $R$ với hydrogen) tương ứng là 
		\choice
		{$RO_2$ và $RH_4$}
		{$R_2O_5$ và $RH_3$}
		{$RO_3$ và $RH_2$}
		{$R_2O_3$ và $RH_3$}
		\loigiai{
		$R$ có cấu hình $1\mathrm{s}^22\mathrm{s}^22p^3$ $\Rightarrow$ $R$ thuộc nhóm $VA$
		do đó ta có:
		\begin{itemize}
			\item  Hóa trị cao nhất của R trong hợp chất với $O$ là $V$ $\Rightarrow$ công thức oxide ứng với hóa trị cao nhất của $R$ là $R_2O_5$
			\item  Hóa trị của R trong hợp chất với $H$ là $III$ $\Rightarrow$ công thức hợp chất khí với $H$  là $RH_3$
		\end{itemize}
		}
	\end{vdex}
	%%%==============HetVd_EX1==============%%%
	
%%%==============Vidu_EX2==============%%%
\begin{vdex}
	Nguyên tố X ở ô thứ 17 của bảng tuần hoàn.
	Có các phát biểu sau:
	\begin{enumerate}
		\item  X có độ âm điện lớn và là một phi kim mạnh.
		\item  X có thể tạo thành ion bền có dạng $X^{+}$.
		\item  Oxide cao nhất của X có công thức $X_2O_5$ và là acidic oxide.
		\item  Hydroxide của X có công thức $HXO_4$ và là acid mạnh.
	\end{enumerate}
	Trong các phát biểu trên, số phát biểu đúng là
	\choice
	{$1$}
	{\True $2$}
	{$3$}
	{$4$}
	\loigiai{
		Nguyên tố X ở ô thứ 17 của bảng tuần hoàn là Clo (Cl).
		
		Xét từng phát biểu:
		\begin{enumerate}
			\item Đúng. Clo có độ âm điện lớn (3.16 trên thang Pauling) và là một phi kim mạnh.
			\item Sai. Clo tạo thành ion bền có dạng $Cl^-$, không phải $X^+$.
			\item Sai. Oxide cao nhất của Cl có công thức $Cl_2O_7$, không phải $X_2O_5$. Tuy nhiên, đúng là nó là một acidic oxide.
			\item Đúng. Hydroxide của Cl có công thức $HClO_4$ (acid percloric) và là một acid mạnh.
		\end{enumerate}
		
		Vậy có 2 phát biểu đúng (1 và 4).
	}
\end{vdex}
	
	%%%=============BT_1=============%%%
	\begin{bt}%[0H2V2-2]
		Nguyên tố R thuộc nhóm VIA trong bảng tuần hoàn.
		\begin{enumerate}[a)]
			\item Viết công thức oxide cao nhất và hợp chất khí với hydrogen của R.
			\item Oxide cao nhất của R có tính acid hay base?
		\end{enumerate}
		\loigiai{
			\begin{enumerate}[a)]
				\item R thuộc nhóm VIA nên có hóa trị cao nhất là VI. Công thức oxide cao nhất là RO$_3$, hợp chất khí với hydrogen là H$_2$R.
				\item Oxide cao nhất của R là RO$_3$ là oxide acid.
			\end{enumerate}
		}
	\end{bt}
	%%%=============BT_2=============%%%
	\begin{bt}%[0H2V2-2]
		Oxide cao nhất của nguyên tố Y chứa 40\% khối lượng nguyên tố Y. Y tạo được hợp chất khí với hydrogen.
		\begin{enumerate}[a)]
			\item Xác định nguyên tố Y.
			\item Viết công thức hydroxide của Y và cho biết tính acid - base của nó.
		\end{enumerate}
		\loigiai{
			\begin{enumerate}[a)]
				\item Gọi công thức oxide cao nhất của Y là Y$_2$O$_n$ (n là hóa trị của Y).
				Ta có: $\dfrac{2M_Y}{2M_Y + 16n} = 0.4$
				Biện luận với n = 1, 2, \ldots, 7, ta tìm được n = 6 và $M_Y = 32$ (g/mol) phù hợp. Vậy Y là lưu huỳnh (S).
				\item Công thức hydroxide của S là H$_2$SO$_4$, là một acid mạnh.
			\end{enumerate}
		}
	\end{bt}
	%%%=============BT_3=============%%%
	\begin{bt}%[0H2V2-2]
		Nguyên tử của nguyên tố X có cấu hình electron lớp ngoài cùng là $ns^2 np^4$. Trong hợp chất hydride (hợp chất của X với hydrogen), nguyên tố X chiếm 94,12\% khối lượng.
		\begin{enumerate}[a)]
			\item Xác định phần trăm khối lượng của X trong oxide cao nhất.
			\item Viết công thức oxide ứng với hoá trị cao nhất của X, hydroxide tương ứng và nêu tính chất acid - base của chúng.
		\end{enumerate}
		\loigiai{
			\begin{enumerate}[a)]
				\item Cấu hình electron lớp ngoài cùng của X là $ns^2 np^4$ nên X thuộc nhóm VIA. Hợp chất khí với hydrogen của X có công thức H$_2$X.
				Ta có: $\dfrac{M_X}{M_X + 2} = 0.9412$
				Giải phương trình, ta tìm được $M_X = 32$ (g/mol). Vậy X là lưu huỳnh (S).
				Công thức oxide cao nhất của S là SO$_3$.
				Phần trăm khối lượng của S trong SO$_3$ là: $\dfrac{32}{32+16\times3} \times 100\% = 40\%$
				\item Công thức oxide cao nhất của S là SO$_3$ (lưu huỳnh trioxide), hydroxide tương ứng là H$_2$SO$_4$ (acid sulfuric). Cả hai đều có tính acid mạnh.
			\end{enumerate}
		}
	\end{bt}
	%%%=============BT_4=============%%%
	\begin{bt}%[0H2V1-2]
		Hai nguyên tố X và Y thuộc hai chu kì liên tiếp và đều thuộc nhóm A. Tổng số electron trong hai nguyên tử X và Y bằng 20.
		\begin{enumerate}[a)]
			\item Xác định X và Y.
			\item So sánh tính acid của oxide cao nhất và tính base của hydroxide tương ứng của X và Y. Giải thích.
		\end{enumerate}
		\loigiai{
			\begin{enumerate}[a)]
				\item Gọi $Z_X$ và $Z_Y$ là số hiệu nguyên tử của X và Y. Ta có: $Z_X + Z_Y = 20$.
				Vì X và Y thuộc hai chu kì liên tiếp và đều thuộc nhóm A, nên $|Z_X - Z_Y| = 8$.
				Giải hệ phương trình, ta tìm được $Z_X = 6$ (cacbon - C) và $Z_Y = 14$ (silic - Si).
				\item C và Si đều thuộc nhóm IVA, trong đó C thuộc chu kì 2, Si thuộc chu kì 3. Trong một nhóm A, tính phi kim giảm dần khi đi từ trên xuống dưới, nên tính acid của oxide cao nhất giảm dần, tính base của hydroxide tương ứng tăng dần. Do đó, tính acid của CO$_2$ mạnh hơn SiO$_2$, tính base của H$_2$SiO$_3$ mạnh hơn H$_2$CO$_3$.
			\end{enumerate}
		}
	\end{bt}
	%%%=============BT_5=============%%%
	\begin{bt}%[0H2V2-6]
		Hòa tan hoàn toàn 4,8 gam một kim loại M thuộc nhóm IIA vào 200 ml dung dịch HCl 1M, thu được dung dịch A và V lít khí H$_2$ (đktc). Để trung hòa axit còn dư trong A cần dùng vừa đủ 50 ml dung dịch NaOH 1M.
		\begin{enumerate}[a)]
			\item Xác định kim loại M.
			\item Tính V.
			\item Viết công thức oxide cao nhất của M và cho biết tính chất của nó.
		\end{enumerate}
		\loigiai{
			\begin{enumerate}[a)]
				\item Số mol HCl ban đầu là: $0.2 \times 1 = 0.2$ (mol)
				Số mol NaOH là: $0.05 \times 1 = 0.05$ (mol)
				Phương trình phản ứng:
				M + 2HCl $\rightarrow$ MCl$_2$ + H$_2$ (1)
				HCl + NaOH $\rightarrow$ NaCl + H$_2$O (2)
				Theo phương trình (2): $n_\text{HCl dư} = n_\text{NaOH} = 0.05$ (mol)
				Theo phương trình (1): $n_M = \dfrac{1}{2} (n_\text{HCl ban đầu} - n_\text{HCl dư}) = 0.075$ (mol)
				Khối lượng mol của M là: $M_M = \dfrac{4.8}{0.075} = 64$ (g/mol). Vậy M là đồng (Cu).
				\item Theo phương trình (1): $n_\text{H2} = \dfrac{1}{2} n_\text{HCl phản ứng} = 0.075$ (mol)
				Vậy $V = 0.075 \times 22.4 = 1.68$ (lít)
				\item Công thức oxide cao nhất của Cu là CuO. CuO là oxide base.
			\end{enumerate}
		}
	\end{bt}
	
	%%%=========bt_1=========%%%
	\begin{bt}%[0H2H1-2]
		Sắp xếp các nguyên tố sau theo chiều tăng dần tính phi kim và giải thích:
		\begin{enumerate}[a)]
			\item N, O, F
			\item P, As, Sb
			\item S, Cl, Br
		\end{enumerate}
		\loigiai{
			\begin{enumerate}[a)]
				\item Tính phi kim tăng dần theo dãy: N $<$ O $<$ F. Trong một chu kì, tính phi kim tăng dần khi đi từ trái sang phải do điện tích hạt nhân tăng, bán kính nguyên tử giảm, sức hút của hạt nhân với electron lớp ngoài cùng tăng. 
				\item Tính phi kim tăng dần theo dãy: Sb $<$ As $<$ P. Trong một nhóm A, tính phi kim giảm dần khi đi từ trên xuống dưới do điện tích hạt nhân tăng, bán kính nguyên tử tăng, sức hút của hạt nhân với electron lớp ngoài cùng giảm.
				\item Tính phi kim tăng dần theo dãy: S $<$ Br $<$ Cl. So sánh tính phi kim của nguyên tố trong cùng một chu kỳ, theo qui luật tăng dần từ trái sang phải, so sánh tính phi kim của nguyên tố trong cùng một nhóm, theo qui luật giảm dần từ trên xuống dưới. 
			\end{enumerate}
		}
	\end{bt}
	
	%%%=========bt_2=========%%%
	\begin{bt}%[0H2H2-6]
		Cho các nguyên tố sau:  $_3$Li, $_7$N, $_9$F, $_{11}$Na, $_{17}$Cl. 
		\begin{enumerate}[a)]
			\item Viết cấu hình electron, xác định vị trí (ô, chu kì, nhóm) của các nguyên tố trong bảng tuần hoàn.
			\item Sắp xếp các nguyên tố theo chiều tăng dần bán kính nguyên tử và giải thích.
			\item Sắp xếp các nguyên tố theo chiều tăng dần độ âm điện và giải thích.
		\end{enumerate}
		\loigiai{
			\begin{enumerate}[a)]
				\item 
				\begin{itemize}
					\item $_3$Li: $1s^22s^1$, Ô số 3, chu kì 2, nhóm IA
					\item $_7$N: $1s^22s^22p^3$, Ô số 7, chu kì 2, nhóm VA
					\item $_9$F: $1s^22s^22p^5$, Ô số 9, chu kì 2, nhóm VIIA
					\item $_{11}$Na: $1s^22s^22p^63s^1$, Ô số 11, chu kì 3, nhóm IA
					\item $_{17}$Cl: $1s^22s^22p^63s^23p^5$, Ô số 17, chu kì 3, nhóm VIIA
				\end{itemize}
				\item  Bán kính nguyên tử tăng dần theo dãy: F $<$ Cl $<$ N $<$ Li $<$ Na.
				\begin{itemize}
					\item  Trong một chu kì, bán kính nguyên tử giảm dần khi đi từ trái sang phải.
					\item  Trong một nhóm A, bán kính nguyên tử tăng dần khi đi từ trên xuống dưới.
				\end{itemize}
				\item Độ âm điện tăng dần theo dãy: Na $<$ Li $<$ Cl $<$ N $<$ F.
				\begin{itemize}
					\item  Trong một chu kì, độ âm điện tăng dần khi đi từ trái sang phải.
					\item  Trong một nhóm A, độ âm điện giảm dần khi đi từ trên xuống dưới.
				\end{itemize}
			\end{enumerate}
		}
	\end{bt}
	
	%%%=========bt_3=========%%%
	\begin{bt}%[0H2V2-6]
		Nguyên tố R thuộc chu kỳ 3, nhóm VIIA của bảng tuần hoàn.
		\begin{enumerate}[a)]
			\item Viết cấu hình electron, xác định số hiệu nguyên tử của R.
			\item Nguyên tố R là kim loại, phi kim hay khí hiếm.
			\item Viết công thức oxide cao nhất và hydroxide tương ứng của R, cho biết oxide, hydroxide đó có tính acid hay base.
			\item So sánh tính phi kim của R với các nguyên tố lân cận trong cùng chu kỳ và cùng nhóm.
		\end{enumerate}
		\loigiai{
			\begin{enumerate}[a)]
				\item R thuộc chu kỳ 3, nhóm VIIA $\Rightarrow$ R có 3 lớp electron và 7 electron lớp ngoài cùng.
				Cấu hình electron của R là $1s^22s^22p^63s^23p^5$.
				Số hiệu nguyên tử của R là 17.
				\item R có 7 electron lớp ngoài cùng $\Rightarrow$ R là phi kim
				\item Oxide cao nhất của R là $R_2O_7$, là acidic oxide.
				Hydroxide tương ứng của R là $HRO_4$, là acid.
				\item 
				\begin{itemize}
					\item  Trong cùng một chu kỳ 3, tính phi kim tăng dần theo dãy: Na $<$ Mg $<$ Al $<$ Si $<$ P $<$ S $<$ Cl.
					\item  Trong cùng một nhóm VIIA, tính phi kim giảm dần theo dãy: F $>$ Cl $>$ Br $>$ I $>$ At. 
				\end{itemize}
			\end{enumerate}
		}
	\end{bt}
	
	%%%=========bt_4=========%%%
	\begin{bt}%[0H2V2-6]
		Cho các nguyên tố X, Y, Z có số hiệu nguyên tử lần lượt là 9, 11, 17.
		\begin{enumerate}[a)]
			\item Viết cấu hình electron, xác định vị trí (chu kì, nhóm) của các nguyên tố trong bảng tuần hoàn.
			\item So sánh  tính acid của  hydroxide tương ứng của X, Y, Z và giải thích.
		\end{enumerate}
		\loigiai{
			\begin{enumerate}[a)]
				\item 
				\begin{itemize}
					\item X (Z = 9): $1s^22s^22p^5$, chu kì 2, nhóm VIIA.
					\item Y (Z = 11): $1s^22s^22p^63s^1$, chu kì 3, nhóm IA.
					\item  Z (Z = 17): $1s^22s^22p^63s^23p^5$, chu kì 3, nhóm VIIA.
				\end{itemize}
				\item Tính acid tăng dần theo dãy: $Y(OH) < Z(OH) < X(OH)$
				\begin{itemize}
					\item  Trong một chu kì, tính acid của hydroxide tương ứng tăng dần khi đi từ trái sang phải, do độ âm điện của nguyên tố trung tâm tăng, khả năng hút electron của nguyên tố trung tâm từ nhóm OH tăng, làm cho liên kết O-H phân cực hơn, dễ dàng bị phân li thành $H^+$  trong dung dịch.
					\item  Trong một nhóm A, tính acid của hydroxide tương ứng tăng dần khi đi từ trên xuống dưới, do bán kính nguyên tử của nguyên tố trung tâm tăng,  làm cho liên kết O-H phân cực hơn, dễ dàng bị phân li thành $H^+$  trong dung dịch. 
				\end{itemize}
			\end{enumerate}
		}
	\end{bt}
	
	%%%=========bt_5=========%%%
	\begin{bt}%[0H2V2-6]
		Nguyên tố  X có electron cuối cùng được điền vào phân lớp $3p^5$.
		\begin{enumerate}[a)]
			\item Viết cấu hình electron, xác định vị trí (ô, chu kì, nhóm) của X trong bảng tuần hoàn.
			\item X là kim loại, phi kim hay khí hiếm.
			\item Viết công thức hợp chất khí với hydrogen (nếu có) của X.
			\item So sánh  bán kính nguyên tử của X với các nguyên tố lân cận trong cùng chu kỳ.
		\end{enumerate}
		\loigiai{
			\begin{enumerate}[a)]
				\item Cấu hình electron của X: $1s^22s^22p^63s^23p^5$. 
				X có 17 electron $\Rightarrow$  Z = 17, X ở ô số 17, chu kì 3, nhóm VIIA.
				\item X có 7 electron lớp ngoài cùng  $\Rightarrow$ X là phi kim.
				\item X thuộc nhóm VIIA nên công thức hợp chất khí với hydrogen là HX.
				\item 
				Trong cùng một chu kỳ 3, bán kính nguyên tử giảm dần theo dãy: Na $>$ Mg $>$ Al $>$ Si $>$ P $>$ S $>$ Cl (X). 
			\end{enumerate}
		}
	\end{bt}
	%%%==============Bai_BT7_17==============%%%
	\begin{bt}%[0H2V2-6]
		Nguyên tố X nằm ở chu kì 3 của bảng tuần hoàn và M là nguyên tố s có electron lớp ngoài cùng là $ns^1$. X có công thức oxide ứng với hoá trị cao nhất là $XO_3$. Một hợp chất của M và X, trong đó M chiếm 58,97\% về khối lượng, là một hoá chất công nghiệp quan trọng, được sử dụng trong sản xuất giấy Kraft, thuốc nhuộm, thuộc da, dầu mỗ, xử lí ô nhiễm kim loại nặng,...
		\begin{enumerate}[a)]
			\item Xác định công thức hoá học của hợp chất giữa M và X. 
			\item Viết công thức oxide ứng với hoá trị cao nhất và hydroxide tương ứng của M, của X và nêu tính acid – base của chúng.
		\end{enumerate}
		\loigiai{
			\begin{enumerate}[a)]
				\item X có công thức oxide ứng với hoá trị cao nhất là $XO_3$ nên X thuộc nhóm VIA, chu kì 3. Do đó, X là S.
				Gọi công thức hóa học của hợp chất giữa M và X là $M_aX_b$.
				Ta có:
				$$\dfrac{aM}{bX} = \dfrac{58,97}{100 - 58,97} = 1,44$$
				$$\Rightarrow \dfrac{a}{b} = 1,44\dfrac{X}{M} = 1,44\dfrac{32}{M}$$
				Do M là nguyên tố s có electron lớp ngoài cùng là $ns^1$ nên M thuộc nhóm IA. Thử lần lượt với các nguyên tố Li, Na, K ta thấy với M là Na thì:
				$$\dfrac{a}{b} = 1,44\dfrac{32}{23} = 2$$
				Chọn a = 2, b = 1, công thức hóa học của hợp chất là $Na_2S$.
				\item Công thức oxide ứng với hoá trị cao nhất và hydroxide tương ứng của Na, S:
				\begin{itemize}
					\item $Na_2O$: oxide base; $NaOH$: base mạnh.
					\item $SO_3$: oxide acid; $H_2SO_4$: acid mạnh.
				\end{itemize}
			\end{enumerate}
		}
	\end{bt}
	
	%%%==============Bai_BT7_18==============%%%
	\begin{bt}%[0H2V2-6]
		Nguyên tử của nguyên tố X có cấu hình electron lớp ngoài cùng là $ns^2np^4$. Trong hợp chất hydride (hợp chất của X với hydrogen), nguyên tố X chiếm 94,12\% khối lượng.
		\begin{enumerate}[a)]
			\item Xác định phần trăm khối lượng của X trong oxide cao nhất.
			\item Viết công thức oxide ứng với hoá trị cao nhất của X, hydroxide tương ứng và nêu tính chất acid – base của chúng.
		\end{enumerate}
		\loigiai{
			\begin{enumerate}[a)]
				\item X có cấu hình electron lớp ngoài cùng là $ns^2np^4$ nên X thuộc nhóm VIA.
				Gọi số hiệu nguyên tử của X là Z.
				X chiếm 94,12\% khối lượng trong hợp chất khí với hydrogen nên:
				$$\dfrac{Z}{Z + 2} = 0,9412$$
				$$\Rightarrow Z = 32$$
				Vậy X là S.
				Phần trăm khối lượng của S trong oxide cao nhất là:
				$$\dfrac{32}{32 + 16 \times 3} \times 100\% = 40\%$$
				\item Công thức oxide ứng với hoá trị cao nhất của S là $SO_3$, hydroxide tương ứng là $H_2SO_4$. Cả hai hợp chất này đều có tính acid.
			\end{enumerate}
		}
	\end{bt}
	
	%%%==============Bai_BT7_19==============%%%
	\begin{bt}
		Hai nguyên tố X và Y ở hai nhóm A liên tiếp trong bảng tuần hoàn. Ở trạng thái đơn chất, X và Y không phản ứng với nhau. Tổng số proton trong hạt nhân X và Y bằng 23. 
		\begin{enumerate}[a)]
			\item Xác định X, Y. 
			\item Viết công thức các hợp chất oxide ứng với hóa trị cao nhất, hydroxide tương ứng của X, Y và nêu tính acid – base của chúng.
		\end{enumerate}
		\loigiai{
			\begin{enumerate}[a)]
				\item Gọi số hiệu nguyên tử của X, Y lần lượt là $Z_X$, $Z_Y$.
				Ta có:
				$$\begin{cases}
					Z_X + Z_Y &= 23 \\
					|Z_X - Z_Y| &= 1
				\end{cases}$$
				$$\Rightarrow \begin{cases}
					Z_X = 12 \\
					Z_Y = 11
				\end{cases}$$
				Vậy X là Mg, Y là Na.
				\item Công thức các hợp chất oxide ứng với hóa trị cao nhất, hydroxide tương ứng của Mg, Na:
				\begin{itemize}
					\item $MgO$: oxide base; $Mg(OH)_2$: base yếu.
					\item $Na_2O$: oxide base; $NaOH$: base mạnh.
				\end{itemize}
			\end{enumerate}
		}
	\end{bt}

	%%%==============Bai_BT7_20==============%%%
	\begin{bt}%[0H2V2-6]
		Nguyên tố X có electron phân lớp ngoài cùng là $np^2$, nguyên tố Y có electron phân lớp ngoài cùng là $np^3$. Hợp chất khí với hydrogen của X chứa a\% khối lượng X, oxide ứng với hóa trị cao nhất của Y chứa b\% khối lượng Y. Tỉ số a : b = 3,365. Hợp chất A tạo bởi X và Y có nhiều ứng dụng chính hình trong lĩnh vực y khoa, vật liệu này cũng là một sự thay thế cho PEEK (polyether ether ketone) và titan, được sử dụng cho các thiết bị tổng hợp tủy sống. Khối lượng mol của A là 140 g/mol.
		\begin{enumerate}[a)]
			\item Xác định X, Y. 
			\item Viết công thức hợp chất khí với hydrogen của X, oxide ứng với hóa trị cao nhất, hydroxide tương ứng của X, Y và nêu tính acid – base của chúng.
		\end{enumerate}
		\loigiai{
			\begin{enumerate}[a)]
				\item X có electron phân lớp ngoài cùng là $np^2$ nên X thuộc nhóm IVA.
				Y có electron phân lớp ngoài cùng là $np^3$ nên Y thuộc nhóm VA.
				Gọi số hiệu nguyên tử của X, Y lần lượt là $Z_X$, $Z_Y$.
				Ta có:
				$$\dfrac{Z_X}{Z_X + 4} : \dfrac{Z_Y}{Z_Y + 16 \times 3} = 3,365$$
				$$\Rightarrow \dfrac{Z_X(Z_Y + 48)}{Z_Y(Z_X + 4)} = 3,365$$
				Thử lần lượt các giá trị của $Z_X$, $Z_Y$ ta thấy với $Z_X = 14$, $Z_Y = 15$ thỏa mãn.
				Vậy X là Si, Y là P.
				Gọi công thức hóa học của A là $Si_aP_b$.
				Ta có:
				$$28a + 31b = 140$$
				Thử lần lượt các giá trị của a, b ta thấy với a = 3, b = 2 thỏa mãn.
				Vậy công thức hóa học của A là $Si_3P_2$.
				\item Công thức hợp chất khí với hydrogen của Si là $SiH_4$, oxide ứng với hóa trị cao nhất là $SiO_2$, hydroxide tương ứng là $H_2SiO_3$. $SiO_2$ là oxide acid, $H_2SiO_3$ là acid yếu.
				Công thức oxide ứng với hóa trị cao nhất của P là $P_2O_5$, hydroxide tương ứng là $H_3PO_4$. Cả hai hợp chất này đều có tính acid.
			\end{enumerate}
		}
	\end{bt}
