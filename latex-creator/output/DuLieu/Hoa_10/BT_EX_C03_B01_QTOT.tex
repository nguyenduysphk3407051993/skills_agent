\subsection{Phần trắc nghiệm nhiều lựa chọn}
%%%=============SOẠN EX===============%%%
\Opensolutionfile{ansex}[Ans/LGEX-EX_CO3_B01_QTOT01.tex]
\Opensolutionfile{ans}[Ans/Ans-EX_CO3_B01_QTOT01.tex]
	%%%%=================EX_1====================%%%
	\begin{ex}
		Phát biểu nào sau đây là đúng?
		\choice
		{Lớp electron ngoài cùng của nguyên tử khí hiếm luôn có $8$ electron}
		{Theo quy tắc octet, các nguyên tử có xu hướng tạo thành lớp vỏ bền vững giống khí hiếm}
		{\True Trong quá trình hình thành liên kết hóa học, nguyên tử có xu hướng đạt được cấu hình electron bền vững của khí hiếm}
		{Trong phân tử, các nguyên tử luôn có $8$ electron lớp ngoài cùng}
		\loigiai{Trong quá trình hình thành liên kết hóa học, nguyên tử có xu hướng đạt được cấu hình electron bền vững của khí hiếm bằng cách góp chung, nhận hoặc nhường electron.}
	\end{ex}
	
	%%%%$=================EX_2====================$%%%
	\begin{ex}
		Nguyên tử nguyên tố nào sau đây luôn có $8$ electron ở lớp ngoài cùng?
		\choice
		{Cl}
		{O}
		{\True Ne}
		{Na}
		\loigiai{Nguyên tử các nguyên tố khí hiếm(trừ He)có $8$ electron lớp ngoài cùng, thỏa mãn quy tắc octet.}
	\end{ex}
	
	%%%%$=================EX_3====================$%%%
	\begin{ex}
		Cấu hình electron lớp ngoài cùng của nguyên tử potassium(kali)là $4s^1$, cấu hình electron lớp ngoài cùng của bromine là $4s^24p^5$. Trong phân tử KBr, ion nào sau đây được hình thành?
		\choice
		{\True $K^+$ và $Br^-$}
		{$K^-$ và $Br^+$}
		{$K^{2+}$ và $Br^{2-}$}
		{$K^+$ và $Br^{2-}$}
		\loigiai{Nguyên tử $K(Z=19)$: $1s^22s^22p^63s^2$3p^$6$4s^$1 có 1 electron ở lớp ngoài cùng, dễ nhường đi 1 electron để tạo thành cation K^+ có cấu hình electron bền vững của khí hiếm Ar: $1s^$2$2s^$2$2p^$6$3s^$23p^6$ Nguyên tử Br $(Z=35)$: $1s^22s^22p^63s^2$3p^$6$3d^{$10$}4s^$2$4p^$5 có 7 electron ở lớp ngoài cùng, dễ nhận thêm 1 electron để tạo thành anion $Br^-$ có cấu hình electron bền vững của khí hiếm Kr: $1s^$2$2s^$2$2p^$6$3s^$23p^63d^{10}4s^24p^6$}
	\end{ex} 
	
	%%%%=================EX_4====================%%%
	\begin{ex}
		Theo quy tắc octet, khi hình thành liên kết hóa học, nguyên tử chlorine có xu hướng?
		\choice
		{Nhường 7 electron}
		{Nhường 1 electron}
		{Nhận 2 electron}
		{\True Nhận 1 electron}
		\loigiai{Nguyên tử Cl $(Z=17)$: $1s^22s^22p^63s^2$3p^5$ có 7 electron ở lớp ngoài cùng, dễ nhận thêm 1 electron để tạo thành anion $Cl^-$ có cấu hình electron bền vững của khí hiếm Ar: $1s^22s^22p^63s^2$3p^6$ }
	\end{ex}
	
	%%%%=================EX_5====================%%%
	\begin{ex}
		Theo quy tắc octet, khi hình thành liên kết hóa học, nguyên tử oxygen có xu hướng?
		\choice
		{Nhường $6$ electron}
		{\True Nhận $2$ electron}
		{Nhường $2$ electron}
		{Nhận $6$ electron}
		\loigiai{Nguyên tử O$(Z=8)$: $1s^22s^22p^4$ có $6$ electron ở lớp ngoài cùng, dễ nhận thêm $2$ electron để tạo thành anion $O^{2-}$ có cấu hình electron bền vững của khí hiếm Ne: $1s^22s^22p^6$}
	\end{ex}
	
	%%%%$=================EX_6====================$%%%
	\begin{ex}
		Khi hình thành liên kết hóa học, nguyên tử nitrogen có xu hướng?
		\choice
		{Nhường $5$ electron}
		{Nhường $3$ electron}
		{Nhận $5$ electron}
		{\True Nhận $3$ electron}
		\loigiai{Nguyên tử $N$ $(Z=7)$: $1s^22s^22p^3$ có $5$ electron ở lớp ngoài cùng, dễ nhận thêm $3$ electron để tạo thành anion $N^{3-}$ có cấu hình electron bền vững của khí hiếm Ne: $1s^22s^22p^6$}
	\end{ex}
	
	%%%%=================EX_7====================%%%
	\begin{ex}
		Liên kết ion là loại liên kết được hình thành từ?
		\choice
		{Sự góp chung electron giữa hai nguyên tử}
		{\True Lực hút tĩnh điện giữa ion dương (hình thành từ kim loại) và ion âm (hình thành từ phi kim)}
		{Lực hút tĩnh điện giữa các phân tử}
		{Sự cho - nhận electron giữa hai nguyên tử}
		\loigiai{Liên kết ion là loại liên kết được hình thành từ lực hút tĩnh điện giữa ion dương (hình thành từ kim loại) và ion âm (hình thành từ phi kim).}
	\end{ex}
	
	%%%%=================EX_8====================%%%
	\begin{ex}
		Liên kết hóa học trong phân tử NaCl là?
		\choice
		{Liên kết cộng hóa trị}
		{Liên kết kim loại}
		{Liên kết hydrogen}
		{\True Liên kết ion}
		\loigiai{Liên kết hóa học trong phân tử NaCl được hình thành do lực hút tĩnh điện giữa ion dương $Na^+$ và ion âm $Cl^-$ nên là liên kết ion.}
	\end{ex}
	
	%%%%$=================EX_9====================$%%%
	\begin{ex}
		Nguyên tử $X$ có $Z=11$, nguyên tử $Y$ có $Z=19$. Kiểu liên kết hóa học được tạo thành giữa hai nguyên tử $X$ và $Y$ là?
		\choice
		{Liên kết cộng hóa trị}
		{\True Liên kết ion}
		{Liên kết kim loại}
		{Liên kết hydrogen}
		\loigiai{Nguyên tử $X(Z=11)$: $1s^22s^22p^63s^1$ là kim loại điển hình, dễ nhường đi $1$ electron để tạo thành cation $X^+$ có cấu hình electron bền vững của khí hiếm Ne: $1s^22s^22p^6$ Nguyên tử $Y(Z=19)$: $1s^22s^22p^63s^2$3p^$6$4s^$1 là kim loại điển hình, dễ nhường đi 1 electron để tạo thành cation Y^+ có cấu hình electron bền vững của khí hiếm Ar: $1s^$2$2s^$2$2p^$6$3s^$23p^6$ Liên kết hóa học được hình thành giữa hai nguyên tử $X$ và $Y$ là liên kết kim loại.}
	\end{ex} 
	
	%%%%=================EX_10====================%%%
	\begin{ex}
		Nguyên tố R thuộc nhóm VIIA trong bảng tuần hoàn. Trong hợp chất khí với hydrogen, nguyên tố R chiếm bao nhiêu phần trăm theo khối lượng?
		\choice
		{$98{,}76$\%}
		{\True $97{,}26$\%}
		{$2{,}74$\%}
		{$1{,}23$\%}
		\loigiai{R thuộc nhóm VIIA trong bảng tuần hoàn \Rightarrow Hợp chất khí với hydrogen của R là HR.
			\%$R=$ $\dfrac{A_R}{A_R+1}$.$100$\% $=$ $\dfrac{A_R}{A_R+1}$.$100$\% $=97{,}26$\%
			
		\end{ex}
		
		%%%%=================EX_11====================%%%
		\begin{ex}
			Hai nguyên tố X và Y thuộc 2 nhóm A liên tiếp trong cùng một chu kì. Tổng số proton trong hai nguyên tử X và Y là 15. Công thức phân tử và liên kết trong hợp chất tạo thành từ X và Y là?
			\choice
			{$XY_2$ - liên kết cộng hóa trị}
			{$X_2Y$ - liên kết ion}
			{\True $XY$ - liên kết ion}
			{$X_3Y$ - liên kết ion}
			\loigiai{Hai nguyên tố X và Y thuộc 2 nhóm A liên tiếp trong cùng một chu kì \Rightarrow $Z_Y - Z_X = 1$ (1)
				
				Theo bài ra ta có: $Z_X + Z_Y = 15$ (2)
				
				Từ (1) và (2) \Rightarrow $Z_X = 7$; $Z_Y = 8$
				
				\Rightarrow X là N; Y là O
				
				Nguyên tử N (Z = 7): $1s^22s^22p^3$ có 5 electron ở lớp ngoài cùng, dễ nhận thêm 3 electron để tạo thành anion $N^{3-}$ có cấu hình electron bền vững của khí hiếm Ne: $1s^22s^22p^6$ Nguyên tử O (Z = 8): $1s^22s^22p^4$ có 6 electron ở lớp ngoài cùng, dễ nhận thêm 2 electron để tạo thành anion $O^{2-}$ có cấu hình electron bền vững của khí hiếm Ne: $1s^22s^22p^6$ \Rightarrow Công thức phân tử là NO và liên kết trong hợp chất tạo thành từ X và Y là liên kết ion.}
		\end{ex}
		
		%%%%=================EX_12====================%%%
		\begin{ex}
			Cho biết cấu hình electron lớp ngoài cùng của nguyên tử các nguyên tố: X ($1s^22s^22p^63s^2$3p^3$), Y ($1s^22s^22p^63s^2$3p^64s^1$), Z ($1s^22s^22p^5$). Công thức hóa học của hợp chất ion được tạo thành từ các nguyên tố này là?
			\choice
			{$X_2Y_3$ và $XY_4$}
			{$X_3Y$ và $X_4Y$}
			{$Y_3Z$ và $YZ_4$}
			{\True $Y_3X$ và $XZ$}
			\loigiai{X ($1s^22s^22p^63s^2$3p^3$) là phi kim, dễ nhận thêm 3 electron để tạo thành anion $X^{3-}$ có cấu hình electron bền vững của khí hiếm Ar: $1s^22s^22p^63s^2$3p^6$ Y ($1s^22s^22p^63s^2$3p^64s^1$) là kim loại, dễ nhường đi 1 electron để tạo thành cation $Y^{+}$ có cấu hình electron bền vững của khí hiếm Ar: $1s^22s^22p^63s^2$3p^6$ Z ($1s^22s^22p^5$) là phi kim, dễ nhận thêm 1 electron để tạo thành anion $Z^{-}$ có cấu hình electron bền vững của khí hiếm Ne: $1s^22s^22p^6$ \Rightarrow Công thức hóa học của hợp chất ion được tạo thành từ các nguyên tố này là $Y_3X$ và $XZ$}
		\end{ex}
		
		%%%%=================EX_13====================%%%
		\begin{ex}
			Cho các nguyên tố X, Y, Z với số hiệu nguyên tử lần lượt là $9$, $17$, $35$. Công thức hóa học của hợp chất tạo thành từ các nguyên tố này là?
			\choice
			{$X_2Y$, $XY_2$, $XZ$, $YZ_2$}
			{$XY$, $XZ$, $YZ_3$}
			{\True $XY$, $XZ$}
			{$X_2Y$, $XY_2$, $XZ$, $XYZ_3$}
			\loigiai{X (Z = 9): $1s^22s^22p^5$ là phi kim, dễ nhận thêm 1 electron để tạo thành anion $X^{-}$ có cấu hình electron bền vững của khí hiếm Ne: $1s^22s^22p^6$ $Y$ $(Z=17)$: $1s^22s^22p^63s^2$3p^5$ là phi kim, dễ nhận thêm 1 electron để tạo thành anion $Y^{-}$ có cấu hình electron bền vững của khí hiếm Ar: $1s^22s^22p^63s^2$3p^6$ Z (Z = 35): $1s^22s^22p^63s^2$3p^63d^{10}4s^24p^5$ là phi kim, dễ nhận thêm 1 electron để tạo thành anion $Z^{-}$ có cấu hình electron bền vững của khí hiếm Kr: $1s^22s^22p^63s^2$3p^63d^{10}4s^24p^6$ \Rightarrow Công thức hóa học của hợp chất ion được tạo thành từ các nguyên tố này là $XY$, $XZ$}
		\end{ex}
		
		%%%%=================EX_14====================%%%
		\begin{ex}
			Nguyên tố X thuộc nhóm IA, nguyên tố Y thuộc nhóm VIA. Công thức hóa học của hợp chất tạo thành từ hai nguyên tố trên là?
			\choice
			{$XY_2$}
			{\True $X_2Y$}
			{$XY$}
			{$X_2Y_3$}
			\loigiai{Nguyên tố X thuộc nhóm IA \Rightarrow X là kim loại, dễ nhường đi 1 electron để tạo thành cation $X^{+}$
				
				Nguyên tố Y thuộc nhóm VIA \Rightarrow Y là phi kim, dễ nhận thêm 2 electron để tạo thành anion $Y^{2-}$
				
				\Rightarrow Công thức hóa học của hợp chất ion được tạo thành từ hai nguyên tố trên là $X_2Y$}
		\end{ex}
		
		%%%%=================EX_15====================%%%
		\begin{ex}
			Hợp chất khí của hydrogen với nguyên tố R là $RH_3$. Trong oxit cao nhất, R chiếm 43,66\% về khối lượng. Nguyên tố R là?
			\choice
			{Nitrogen}
			{\True Phosphorus}
			{Sulfur}
			{Carbon}
			\loigiai{Hợp chất khí của hydrogen với nguyên tố R là $RH_3$ \Rightarrow R thuộc nhóm VA. 
				
				Công thức oxide cao nhất của R là $R_2O_5$
				
				Theo bài ra ta có: $\dfrac{2.A_R}{2.A_R + 5.16} = 43,66\%$ \Rightarrow $A_R = 31$ \Rightarrow R là nguyên tố Phosphorus (P)}
		\end{ex}
		
		%%%%=================EX_16====================%%%
		\begin{ex}
			Nguyên tử X có 7 electron ở lớp ngoài cùng, trong hợp chất khí với hydrogen, X chiếm $97{,}26\%$ về khối lượng. X là nguyên tố nào sau đây?
			\choice
			{N}
			{O}
			{F}
			{\True Cl}
			\loigiai{Nguyên tử X có 7 electron ở lớp ngoài cùng \Rightarrow X thuộc nhóm VIIA \Rightarrow Công thức hóa học của hợp chất khí với hydrogen là HX.
				Trong HX, $\% X=\dfrac{A_X}{A_X+1}\cdot100\%=97{,}26\%$
				\Rightarrow $A_X=35{,}5$ \Rightarrow X là nguyên tố Chlorine (Cl)}
		\end{ex}
		
		%%%%=================EX_17====================%%%
		\begin{ex}
			Trong các chất sau, chất nào có liên kết ion?
			\choice
			{$N_2$}
			{CO}
			{\True KF}
			{$NH_3$}
			\loigiai{KF là hợp chất được tạo thành từ kim loại điển hình K và phi kim điển hình F nên có liên kết ion.}
		\end{ex}
		
		%%%%=================EX_18====================%%%
		\begin{ex}
			Nguyên tố X có cấu hình electron là $1s^22s^22p^63s^2$, nguyên tố Y có cấu hình electron là $1s^22s^22p^5$. Hợp chất tạo thành từ X và Y có tính chất nào sau đây?
			\choice
			{Dẫn điện tốt ở trạng thái rắn}
			{Có nhiệt độ nóng chảy và nhiệt độ sôi rất cao}
			{\True Dễ tan trong nước}
			{Là chất khí ở điều kiện thường}
			\loigiai{Nguyên tố X có cấu hình electron là $1s^22s^22p^63s^2$ \Rightarrow X là kim loại, dễ nhường đi 2 electron để tạo thành cation $X^{2+}$ Nguyên tố Y có cấu hình electron là $1s^22s^22p^5$ \Rightarrow Y là phi kim, dễ nhận thêm 1 electron để tạo thành anion $Y^{-}$ 
				
				\Rightarrow Hợp chất tạo thành từ X và Y là hợp chất ion nên dễ tan trong nước.}
		\end{ex}
		
		%%%%=================EX_19====================%%%
		\begin{ex}
			Muối ăn (NaCl) là hợp chất có vai trò quan trọng trong đời sống con người. Liên kết hóa học trong tinh thể NaCl là?
			\choice
			{Liên kết cộng hóa trị không cực}
			{Liên kết cộng hóa trị có cực}
			{Liên kết kim loại}
			{\True Liên kết ion}
			\loigiai{Muối ăn (NaCl) là hợp chất được tạo thành từ kim loại điển hình Na và phi kim điển hình Cl nên có liên kết ion.}
		\end{ex}
		
		%%%%=================EX_20====================%%%
		\begin{ex}
			Nguyên tố X có Z = 11, nguyên tố Y có Z = 16. Hợp chất tạo thành từ hai nguyên tố này có liên kết?
			\choice
			{Cộng hóa trị có cực}
			{\True Ion}
			{Cộng hóa trị không cực}
			{Cho - nhận}
			\loigiai{Nguyên tố X (Z = 11): $1s^22s^22p^63s^1$ là kim loại điển hình, dễ nhường đi 1 electron để tạo thành cation $X^{+}$ có cấu hình electron bền vững của khí hiếm Ne: $1s^22s^22p^6$ Nguyên tử Y (Z = 16): $1s^22s^22p^63s^23p^4$ là phi kim điển hình, dễ nhận thêm 2 electron để tạo thành anion $Y^{2-}$ có cấu hình electron bền vững của khí hiếm Ar: $1s^22s^22p^63s^23p^6$ \Rightarrow Hợp chất tạo thành từ hai nguyên tố này có liên kết ion.}
		\end{ex}
		%%%%=================EX_19====================%%%
		\begin{ex}
			Phát biểu nào sau đây là đúng về cấu hình electron bền vững?
			\choice
			{\True Các nguyên tử thường có xu hướng đạt được cấu hình electron giống khí hiếm}
			{Nguyên tử kim loại có xu hướng nhận thêm electron để đạt cấu hình bền vững}
			{Nguyên tử phi kim có xu hướng nhường electron để đạt cấu hình bền vững}
			{Tất cả các nguyên tử đều có 8 electron ở lớp ngoài cùng để bền vững}
			\loigiai{Theo quy tắc octet, các nguyên tử thường có xu hướng đạt được cấu hình electron giống khí hiếm để đạt độ bền cao.}
		\end{ex}
		
		%%%%=================EX_20====================%%%
		\begin{ex}
			Cấu hình electron lớp ngoài cùng của nguyên tử Na (Z = 11) là gì?
			\choice
			{$2s^1$}
			{\True $3s^1$}
			{$3p^1$}
			{$2p^6$}
			\loigiai{Nguyên tử Na có cấu hình electron: $1s^22s^22p^63s^1$, với electron lớp ngoài cùng là $3s^1$.}
		\end{ex}
		
		%%%%=================EX_21====================%%%
		\begin{ex}
			Nguyên tử nào sau đây có xu hướng nhường đi 2 electron để đạt được cấu hình bền vững?
			\choice
			{Cl}
			{S}
			{\True Mg}
			{F}
			\loigiai{Nguyên tử Mg (Z = 12) có cấu hình electron $1s^22s^22p^63s^2$, dễ nhường 2 electron để đạt cấu hình bền vững giống khí hiếm Ne: $1s^22s^22p^6$.}
		\end{ex}
		
		%%%%=================EX_22====================%%%
		\begin{ex}
			Nguyên tử nào sau đây có cấu hình electron tương tự như khí hiếm sau khi nhận thêm 1 electron?
			\choice
			{Mg}
			{\True Cl}
			{K}
			{Na}
			\loigiai{Nguyên tử Cl (Z = 17) có 7 electron lớp ngoài cùng, dễ nhận thêm 1 electron để đạt cấu hình bền vững giống khí hiếm Ar.}
		\end{ex}
		
		%%%%=================EX_23====================%%%
		\begin{ex}
			Hợp chất nào sau đây có liên kết cộng hóa trị?
			\choice
			{\True $H_2O$}
			{$NaCl$}
			{$CaO$}
			{$MgF_2$}
			\loigiai{Liên kết cộng hóa trị được hình thành giữa hai nguyên tử phi kim. Trong phân tử $H_2O$, liên kết giữa H và O là cộng hóa trị.}
		\end{ex}
		
		%%%%=================EX_24====================%%%
		\begin{ex}
			Nguyên tố X có Z = 6, nguyên tố Y có Z = 8. Công thức hợp chất và kiểu liên kết trong hợp chất tạo thành từ X và Y là gì?
			\choice
			{\True $CO_2$ - liên kết cộng hóa trị}
			{$C_2O$ - liên kết ion}
			{$C_2O_2$ - liên kết kim loại}
			{$CO$ - liên kết kim loại}
			\loigiai{C (Z = 6) và O (Z = 8) đều là phi kim, nên liên kết giữa chúng trong $CO_2$ là cộng hóa trị.}
		\end{ex}
		
		%%%%=================EX_25====================%%%
		\begin{ex}
			Phát biểu nào sau đây là đúng khi so sánh liên kết ion và liên kết cộng hóa trị?
			\choice
			{Liên kết ion yếu hơn liên kết cộng hóa trị}
			{\True Liên kết ion được hình thành từ sự hút tĩnh điện giữa các ion trái dấu}
			{Liên kết cộng hóa trị là sự cho - nhận electron giữa hai nguyên tử}
			{Liên kết ion tồn tại chủ yếu giữa các phân tử phi kim}
			\loigiai{Liên kết ion hình thành từ lực hút tĩnh điện giữa các ion trái dấu, thường là giữa kim loại và phi kim.}
		\end{ex}
		
		%%%%=================EX_26====================%%%
		\begin{ex}
			Công thức hóa học của hợp chất giữa nguyên tố kim loại Mg và phi kim Cl là gì?
			\choice
			{\True $MgCl_2$}
			{$Mg_2Cl$}
			{$MgCl$}
			{$Mg_2Cl_3$}
			\loigiai{Nguyên tử Mg nhường 2 electron tạo $Mg^{2+}$, còn nguyên tử Cl nhận 1 electron tạo $Cl^-$. Công thức hóa học là $MgCl_2$.}
		\end{ex}
		
		%%%%=================EX_27====================%%%
		\begin{ex}
			Công thức hóa học của hợp chất giữa các nguyên tố nhóm IA và VIIA thường có dạng gì?
			\choice
			{$XY_2$}
			{$X_2Y$}
			{\True $XY$}
			{$X_3Y$}
			\loigiai{Nguyên tố nhóm IA thường có xu hướng nhường 1 electron và nhóm VIIA nhận 1 electron, nên hợp chất sẽ có dạng $XY$.}
		\end{ex}
		
		%%%%=================EX_28====================%%%
		\begin{ex}
			Nguyên tử nào sau đây có thể hình thành cation $Al^{3+}$?
			\choice
			{Mg}
			{Na}
			{\True Al}
			{Cl}
			\loigiai{Nguyên tử Al (Z = 13) có 3 electron lớp ngoài cùng và dễ nhường đi 3 electron để tạo cation $Al^{3+}$.}
		\end{ex}
		
		%%%%=================EX_29====================%%%
		\begin{ex}
			Tổng số proton của hai nguyên tử X và Y là 18. Biết X có Z = 8, nguyên tố Y là gì?
			\choice
			{C}
			{O}
			{\True F}
			{Ne}
			\loigiai{Tổng số proton của hai nguyên tử là 18. Nếu X có Z = 8 (Oxy), thì Y phải có Z = 10, tương ứng với nguyên tố Neon.}
		\end{ex}
		
		%%%%=================EX_30====================%%%
		\begin{ex}
			Nguyên tố nào có cấu hình electron bền vững tương tự Ne sau khi nhường đi 1 electron?
			\choice
			{\True Na}
			{Mg}
			{Cl}
			{Ar}
			\loigiai{Nguyên tử Na (Z = 11) có cấu hình electron $1s^22s^22p^63s^1$, khi nhường 1 electron, nó trở thành $Na^+$ với cấu hình giống Ne: $1s^22s^22p^6$.}
		\end{ex}
		
		%%%%=================EX_31====================%%%
		\begin{ex}
			Phân tử $Na_2S$ được hình thành từ liên kết gì?
			\choice
			{\True Liên kết ion}
			{Liên kết cộng hóa trị}
			{Liên kết kim loại}
			{Liên kết hydrogen}
			\loigiai{Trong phân tử $Na_2S$, liên kết ion được hình thành do lực hút giữa ion $Na^+$ và $S^{2-}$.}
		\end{ex}
		
		%%%%=================EX_32====================%%%
		\begin{ex}
			Khi tạo hợp chất với oxygen, nguyên tử nào có xu hướng nhường 2 electron?
			\choice
			{N}
			{\True Mg}
			{S}
			{F}
			\loigiai{Nguyên tử Mg (Z = 12) có cấu hình electron $1s^22s^22p^63s^2$, dễ nhường 2 electron để tạo thành $Mg^{2+}$.}
		\end{ex}
		
		%%%%=================EX_33====================%%%
		\begin{ex}
			Nguyên tố nào thuộc nhóm VIIIA trong bảng tuần hoàn và không phản ứng với các nguyên tố khác trong điều kiện thường?
			\choice
			{O}
			{\True Ne}
			{Cl}
			{Na}
			\loigiai{Ne (Neon) thuộc nhóm khí hiếm (nhóm VIIIA), với cấu hình electron bền vững nên không dễ dàng tham gia phản ứng hóa học.}
		\end{ex}
		
		%%%%=================EX_34====================%%%
		\begin{ex}
			Cấu hình electron của ion $Ca^{2+}$ là gì?
			\choice
			{$1s^22s^22p^63s^2$3p^64s^2$}
			{\True $1s^22s^22p^63s^2$3p^6$}
			{$1s^22s^22p^6$}
			{$1s^22s^22p^63s^2$3p^64s^1$}
			\loigiai{Ion $Ca^{2+}$ mất 2 electron từ lớp ngoài cùng của $Ca$, cấu hình electron sẽ giống như khí hiếm Ar: $1s^22s^22p^63s^2$3p^6$.}
		\end{ex}
		
		%%%%=================EX_35====================%%%
		\begin{ex}
			Nguyên tử nào có năng lượng ion hóa thứ nhất cao nhất trong nhóm II của bảng tuần hoàn?
			\choice
			{Ca}
			{Mg}
			{\True Be}
			{Ba}
			\loigiai{Trong nhóm II, năng lượng ion hóa giảm dần từ trên xuống dưới. Be có năng lượng ion hóa cao nhất trong nhóm này.}
		\end{ex}
		
		%%%%=================EX_36====================%%%
		\begin{ex}
			Năng lượng ion hóa là gì?
			\choice
			{Năng lượng cần thiết để một nguyên tử nhận electron}
			{\True Năng lượng cần thiết để tách electron khỏi một nguyên tử ở trạng thái khí}
			{Năng lượng cần thiết để một nguyên tử đạt được trạng thái ion âm}
			{Năng lượng cần thiết để chuyển nguyên tử từ trạng thái rắn sang khí}
			\loigiai{Năng lượng ion hóa là năng lượng cần thiết để tách 1 electron khỏi 1 nguyên tử ở trạng thái khí, giúp hình thành ion dương.}
		\end{ex}
\Closesolutionfile{ans}
\Closesolutionfile{ansex}
%\bangdapan{Ans-EX_CO3_B01_QTOT01.tex}