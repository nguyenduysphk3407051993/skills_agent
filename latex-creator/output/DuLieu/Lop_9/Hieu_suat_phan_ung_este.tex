\begin{tcolorbox}[
colback=\mycolor!10,
frame empty,
fontupper =\LARGE\bfseries\color{\maudam},
halign =center
]
BÀI TẬP VỀ HIỆU SUẤT PHẢN ỨNG ESTE HÓA
\end{tcolorbox}

%%%=============BT_1=============%%%
\begin{bt}
Cho $10.2$ gam $CH_3COOH$ tác dụng với lượng dư $C_2H_5OH$ (xúc tác ${H_2SO_4}_{\text{đặc}}$, đun nóng) thu được $10.472$ gam $CH_3COOC_2H_5$. Viết phương trình hóa học xảy ra và tính hiệu suất của phản ứng
\loigiai{
\vphantom{x}\hfill\textbf{Đáp số: }$70\%$
}
\end{bt}

%%%=============BT_2=============%%%
\begin{bt}
Cho $2.4$ gam $CH_3COOH$ tác dụng với lượng dư $C_2H_5OH$ (xúc tác ${H_2SO_4}_{\text{đặc}}$, đun nóng) thu được $2.64$ gam $CH_3COOC_2H_5$. Viết phương trình hóa học xảy ra và tính hiệu suất của phản ứng
\loigiai{
\vphantom{x}\hfill\textbf{Đáp số: }$75\%$
}
\end{bt}

%%%=============BT_3=============%%%
\begin{bt}
Cho $9.0$ gam $CH_3COOH$ tác dụng với lượng dư $C_2H_5OH$ (xúc tác ${H_2SO_4}_{\text{đặc}}$, đun nóng) thu được $9.9$ gam $CH_3COOC_2H_5$. Viết phương trình hóa học xảy ra và tính hiệu suất của phản ứng
\loigiai{
\vphantom{x}\hfill\textbf{Đáp số: }$75\%$
}
\end{bt}

%%%=============BT_4=============%%%
\begin{bt}
Cho $4.2$ gam $CH_3COOH$ tác dụng với lượng dư $C_2H_5OH$ (xúc tác ${H_2SO_4}_{\text{đặc}}$, đun nóng) thu được $3.696$ gam $CH_3COOC_2H_5$. Viết phương trình hóa học xảy ra và tính hiệu suất của phản ứng
\loigiai{
\vphantom{x}\hfill\textbf{Đáp số: }$60\%$
}
\end{bt}

%%%=============BT_5=============%%%
\begin{bt}
Cho $15.6$ gam $CH_3COOH$ tác dụng với lượng dư $C_2H_5OH$ (xúc tác ${H_2SO_4}_{\text{đặc}}$, đun nóng) thu được $13.728$ gam $CH_3COOC_2H_5$. Viết phương trình hóa học xảy ra và tính hiệu suất của phản ứng
\loigiai{
\vphantom{x}\hfill\textbf{Đáp số: }$60\%$
}
\end{bt}

%%%=============BT_6=============%%%
\begin{bt}
Cho $11.4$ gam $CH_3COOH$ tác dụng với lượng dư $C_2H_5OH$ (xúc tác ${H_2SO_4}_{\text{đặc}}$, đun nóng) thu được $12.54$ gam $CH_3COOC_2H_5$. Viết phương trình hóa học xảy ra và tính hiệu suất của phản ứng
\loigiai{
\vphantom{x}\hfill\textbf{Đáp số: }$75\%$
}
\end{bt}

%%%=============BT_7=============%%%
\begin{bt}
Cho $2.4$ gam $CH_3COOH$ tác dụng với lượng dư $C_2H_5OH$ (xúc tác ${H_2SO_4}_{\text{đặc}}$, đun nóng) thu được $2.464$ gam $CH_3COOC_2H_5$. Viết phương trình hóa học xảy ra và tính hiệu suất của phản ứng
\loigiai{
\vphantom{x}\hfill\textbf{Đáp số: }$70\%$
}
\end{bt}

%%%=============BT_8=============%%%
\begin{bt}
Cho $7.2$ gam $CH_3COOH$ tác dụng với lượng dư $C_2H_5OH$ (xúc tác ${H_2SO_4}_{\text{đặc}}$, đun nóng) thu được $6.864$ gam $CH_3COOC_2H_5$. Viết phương trình hóa học xảy ra và tính hiệu suất của phản ứng
\loigiai{
\vphantom{x}\hfill\textbf{Đáp số: }$65\%$
}
\end{bt}

%%%=============BT_9=============%%%
\begin{bt}
Cho $13.2$ gam $CH_3COOH$ tác dụng với lượng dư $C_2H_5OH$ (xúc tác ${H_2SO_4}_{\text{đặc}}$, đun nóng) thu được $14.52$ gam $CH_3COOC_2H_5$. Viết phương trình hóa học xảy ra và tính hiệu suất của phản ứng
\loigiai{
\vphantom{x}\hfill\textbf{Đáp số: }$75\%$
}
\end{bt}

%%%=============BT_10=============%%%
\begin{bt}
Cho $11.4$ gam $CH_3COOH$ tác dụng với lượng dư $C_2H_5OH$ (xúc tác ${H_2SO_4}_{\text{đặc}}$, đun nóng) thu được $11.704$ gam $CH_3COOC_2H_5$. Viết phương trình hóa học xảy ra và tính hiệu suất của phản ứng
\loigiai{
\vphantom{x}\hfill\textbf{Đáp số: }$70\%$
}
\end{bt}

%%%=============BT_11=============%%%
\begin{bt}
Cho $15.0$ gam $CH_3COOH$ tác dụng với lượng dư $C_2H_5OH$ (xúc tác ${H_2SO_4}_{\text{đặc}}$, đun nóng) thu được $17.6$ gam $CH_3COOC_2H_5$. Viết phương trình hóa học xảy ra và tính hiệu suất của phản ứng
\loigiai{
\vphantom{x}\hfill\textbf{Đáp số: }$80\%$
}
\end{bt}

%%%=============BT_12=============%%%
\begin{bt}
Cho $3.6$ gam $CH_3COOH$ tác dụng với lượng dư $C_2H_5OH$ (xúc tác ${H_2SO_4}_{\text{đặc}}$, đun nóng) thu được $3.432$ gam $CH_3COOC_2H_5$. Viết phương trình hóa học xảy ra và tính hiệu suất của phản ứng
\loigiai{
\vphantom{x}\hfill\textbf{Đáp số: }$65\%$
}
\end{bt}

%%%=============BT_13=============%%%
\begin{bt}
Cho $6.0$ gam $CH_3COOH$ tác dụng với lượng dư $C_2H_5OH$ (xúc tác ${H_2SO_4}_{\text{đặc}}$, đun nóng) thu được $6.6$ gam $CH_3COOC_2H_5$. Viết phương trình hóa học xảy ra và tính hiệu suất của phản ứng
\loigiai{
\vphantom{x}\hfill\textbf{Đáp số: }$75\%$
}
\end{bt}

%%%=============BT_14=============%%%
\begin{bt}
Cho $8.4$ gam $CH_3COOH$ tác dụng với lượng dư $C_2H_5OH$ (xúc tác ${H_2SO_4}_{\text{đặc}}$, đun nóng) thu được $9.856$ gam $CH_3COOC_2H_5$. Viết phương trình hóa học xảy ra và tính hiệu suất của phản ứng
\loigiai{
\vphantom{x}\hfill\textbf{Đáp số: }$80\%$
}
\end{bt}

%%%=============BT_15=============%%%
\begin{bt}
Cho $15.6$ gam $CH_3COOH$ tác dụng với lượng dư $C_2H_5OH$ (xúc tác ${H_2SO_4}_{\text{đặc}}$, đun nóng) thu được $16.016$ gam $CH_3COOC_2H_5$. Viết phương trình hóa học xảy ra và tính hiệu suất của phản ứng
\loigiai{
\vphantom{x}\hfill\textbf{Đáp số: }$70\%$
}
\end{bt}

%%%=============BT_16=============%%%
\begin{bt}
Cho $3.6$ gam $CH_3COOH$ tác dụng với lượng dư $C_2H_5OH$ (xúc tác ${H_2SO_4}_{\text{đặc}}$, đun nóng) thu được $3.168$ gam $CH_3COOC_2H_5$. Viết phương trình hóa học xảy ra và tính hiệu suất của phản ứng
\loigiai{
\vphantom{x}\hfill\textbf{Đáp số: }$60\%$
}
\end{bt}

%%%=============BT_17=============%%%
\begin{bt}
Cho $1.2$ gam $CH_3COOH$ tác dụng với lượng dư $C_2H_5OH$ (xúc tác ${H_2SO_4}_{\text{đặc}}$, đun nóng) thu được $1.32$ gam $CH_3COOC_2H_5$. Viết phương trình hóa học xảy ra và tính hiệu suất của phản ứng
\loigiai{
\vphantom{x}\hfill\textbf{Đáp số: }$75\%$
}
\end{bt}

%%%=============BT_18=============%%%
\begin{bt}
Cho $1.2$ gam $CH_3COOH$ tác dụng với lượng dư $C_2H_5OH$ (xúc tác ${H_2SO_4}_{\text{đặc}}$, đun nóng) thu được $1.144$ gam $CH_3COOC_2H_5$. Viết phương trình hóa học xảy ra và tính hiệu suất của phản ứng
\loigiai{
\vphantom{x}\hfill\textbf{Đáp số: }$65\%$
}
\end{bt}

%%%=============BT_19=============%%%
\begin{bt}
Cho $1.8$ gam $CH_3COOH$ tác dụng với lượng dư $C_2H_5OH$ (xúc tác ${H_2SO_4}_{\text{đặc}}$, đun nóng) thu được $1.584$ gam $CH_3COOC_2H_5$. Viết phương trình hóa học xảy ra và tính hiệu suất của phản ứng
\loigiai{
\vphantom{x}\hfill\textbf{Đáp số: }$60\%$
}
\end{bt}

%%%=============BT_20=============%%%
\begin{bt}
Cho $1.8$ gam $CH_3COOH$ tác dụng với lượng dư $C_2H_5OH$ (xúc tác ${H_2SO_4}_{\text{đặc}}$, đun nóng) thu được $1.716$ gam $CH_3COOC_2H_5$. Viết phương trình hóa học xảy ra và tính hiệu suất của phản ứng
\loigiai{
\vphantom{x}\hfill\textbf{Đáp số: }$65\%$
}
\end{bt}