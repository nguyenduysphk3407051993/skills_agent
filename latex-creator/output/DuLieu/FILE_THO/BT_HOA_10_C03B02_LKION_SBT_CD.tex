10.1. Phân loại các hợp chất ion dưới đây vào các nhóm sau: hợp chất tạo nên bởi các ion đơn nguyên tử, hợp chất tạo nên bởi ion đơn nguyên tử và đa nguyên tử, hợp chất tạo nên bởi các ion đa nguyên tử.$KCl$, $Na_2CO_3$, $(NH_4)_2SO_4$, $BaCO_3$, $AgCl$, $BaSO_4$, $KMnO_4$

10.2. Cho các ion: $Na^+$, $Ca^{2+}$, $F^-$, $CO_3^{2-}$. Số lượng các hợp chất chứa hai loại ion có thể tạo thành từ các ion này là
A. 2.
B. 3.
C. 4.
D. vô số hợp chất.

10.3. Cặp nguyên tố nào sau đây có khả năng tạo thành liên kết ion trong hợp chất của chúng?
A. Nitrogen, và oxygen.
B. Carbon, và hydrogen.
C. Sulfur, và oxygen.
D. Calcium, và oxygen.

10.4. Những đặc điểm nào sau đây là đúng khi nói về hợp chất tạo thành giữa $Na^+$ và $O^{2-}$?
A. Là hợp chất ion.
B. Có công thức hóa học là $NaO$.
C. Trong điều kiện thường, tồn tại ở thể khí.
D. Trong điều kiện thường, tồn tại ở thể rắn.
E. Có nhiệt độ nóng chảy và nhiệt độ sôi cao.
G. Có nhiệt độ nóng chảy và nhiệt độ sôi thấp.
H. Lực tương tác giữa $Na^+$ và $O^{2-}$ là lực tĩnh điện.

10.5. $ZnO$ là một hợp chất ion được sử dụng nhiều trong kem chống nắng. Bản kính của nguyên tử $O$ như thế nào so với bản kính của anion $O^{2-}$ trong tinh thể $ZnO$?
A. Bằng nhau.
B. Bán kính của $O$ lớn hơn của $O^{2-}$.
C. Bán kính của $O$ nhỏ hơn của $O^{2-}$.
D. Không dự đoán được.

10.6. Bán kính của nguyên tử $Al$ như thế nào so với bán kính của cation $Al^{3+}$ trong tinh thể $AlCl_3$?
A. Bằng nhau.
B. Bán kính của $Al$ lớn hơn của $Al^{3+}$.
C. Bán kính của $Al$ nhỏ hơn của $Al^{3+}$.
D. Không dự đoán được.

10.7. Ghép mỗi nguyên tử ở cột A với các giá trị điện tích của ion mà nguyên tử có thể tạo thành ở cột B.
Cột A                      	Cột B
a) S                         	1. điện tích $2^{+}$
b) Al                       	2. điện tích $3^{+}$
c) F                         	3. điện tích $2^{-}$
d) Mg                      	4. điện tích $1^{-}$

10.8. Chọn phương án đúng để hoàn thành câu sau:
Khi hình thành các hợp chất ion, $,$...(1)... mất các electron hoá trị của chúng để tạo thành $,$...(2)... mang điện tích dương và $,$...(3)... nhận các electron hoá trị để tạo thành $,$...(4)... mang điện tích âm.

A. (1) kim loại$,$ (2) anion$,$ (3) phi kim$,$ (4) cation.
B. (1) phi kim$,$ (2) cation$,$ (3) kim loại$,$ (4) anion.
C. (1) kim loại$,$ (2) ion đa nguyên tử$,$ (3) phi kim$,$ (4) anion.
D. (1) phi kim$,$ (2) anion$,$ (3) kim loại$,$ (4) cation.
E. (1) kim loại$,$ (2) cation$,$ (3) phi kim$,$ (4) anion.


10.9. Điền từ thích hợp vào chỗ trống:
Barium thuộc nhóm IIA$,$ iodine thuộc nhóm VIIA$,$ hợp chất của hai nguyên tố này là hợp chất ...(1)... Ở điều kiện thường$,$ hợp chất này tồn tại ở thể ...(2)... với cấu trúc tinh thể tạo nên bởi ...(3)... và ...(4)...

10.10. Viết hai giai đoạn của sự hình thành $CaF_2$ từ các nguyên tử tương ứng (kèm theo cấu hình electron).

10.11. Cho biết sự tạo thành $NaCl(s)$ từ $Na(s)$ và $Cl_2(g)$ giải phóng nhiều năng lượng. Hãy cho biết năng lượng giải phóng có nguồn gốc từ đâu.
*Gợi ý: Nếu các tiểu phân hút nhau sẽ giải phóng năng lượng$,$ đẩy nhau sẽ hấp thu năng lượng.*

10.12. Biết rằng năng lượng toả ra khi hình thành các hợp chất ion từ các cation và anion tỉ lệ thuận với điện tích của mỗi ion và tỉ lệ nghịch với bán kính của chúng. Dựa trên cơ sở này$,$ hãy cho biết khi hình thành hợp chất nào trong mỗi cặp chất sau đây từ các ion tương ứng thì năng lượng toả ra là nhiều hơn.
a) $LiCl$ và $NaCl$.
b) $Na_2O$ và $MgO$.





