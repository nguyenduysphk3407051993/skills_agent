%%%%=================EX_1====================%%%
\begin{ex}
Liên kết cộng hóa trị là liên kết được hình thành giữa hai nguyên tử bằng cách
\choice
{chuyển electron từ nguyên tử này sang nguyên tử khác.}
{\True dùng chung electron.}
{hút tĩnh điện.}
{cho nhận proton.}
\loigiai{Liên kết cộng hóa trị được hình thành bằng cách dùng chung một hay nhiều cặp electron giữa hai nguyên tử.}
\end{ex}

%%%%=================EX_2====================%%%
\begin{ex}
Nguyên tử Cl có 7 electron lớp ngoài cùng, khi hình thành liên kết với một nguyên tử Cl khác, mỗi nguyên tử Cl có xu hướng
\choice
{nhận thêm 2 electron.}
{nhường đi 1 electron.}
{\True góp chung 1 electron.}
{nhường đi 7 electron.}
\loigiai{Nguyên tử Cl có 7 electron lớp ngoài cùng, để đạt cấu hình electron bền vững của khí hiếm, mỗi nguyên tử Cl sẽ góp chung 1 electron để tạo thành 1 cặp electron chung.}
\end{ex}

%%%%=================EX_3====================%%%
\begin{ex}
Liên kết trong phân tử nào sau đây là liên kết cộng hóa trị không cực?
\choice
{HCl}
{HBr}
{\True Cl$_2$}
{HF}
\loigiai{Liên kết cộng hóa trị không cực được hình thành giữa hai nguyên tử giống nhau. Vậy Cl$_2$ có liên kết cộng hóa trị không cực.}
\end{ex}

%%%%=================EX_4====================%%%
\begin{ex}
Phân tử nào sau đây có liên kết cộng hóa trị phân cực?
\choice
{N$_2$}
{H$_2$}
{\True NH$_3$}
{O$_2$}
\loigiai{Liên kết cộng hóa trị phân cực được hình thành giữa hai nguyên tử khác nhau. Vậy NH$_3$ có liên kết cộng hóa trị phân cực.}
\end{ex}

%%%%=================EX_5====================%%%
\begin{ex}
Trong phân tử HCl, cặp electron liên kết bị lệch về phía nguyên tử nào?
\choice
{H}
{\True Cl}
{Lệch về cả hai phía}
{Không bị lệch}
\loigiai{Trong phân tử HCl, do Cl có độ âm điện lớn hơn H nên cặp electron liên kết bị lệch về phía nguyên tử Cl.}
\end{ex}

%%%%=================EX_6====================%%%
\begin{ex}
Dãy nào sau đây gồm các chất chỉ có liên kết cộng hóa trị?
\choice
{NaCl, MgO, CaF$_2$}
{\True CO$_2$, H$_2$O, NH$_3$}
{NaOH, KOH, Ba(OH)<span class="math-inline">\_2</span>}
{KCl, AlCl$_3$, FeCl$_3$}
\loigiai{CO$_2$, H$_2$O, NH$_3$ là các hợp chất được tạo thành từ các nguyên tử phi kim nên chỉ chứa liên kết cộng hóa trị.}
\end{ex}

%%%%=================EX_7====================%%%
\begin{ex}
Số cặp electron dùng chung trong phân tử CO$_2$ là
\choice
{1}
{2}
{\True 4}
{3}
\loigiai{Trong phân tử CO$_2$, nguyên tử C góp chung 4 electron, mỗi nguyên tử O góp chung 2 electron, hình thành 2 liên kết đôi, tương ứng với 4 cặp electron dùng chung.}
\end{ex}

%%%%=================EX_8====================%%%
\begin{ex}
Cho độ âm điện của H là 2,2 và của O là 3,44. Vậy liên kết O-H trong phân tử H$_2$O là
\choice
{liên kết ion.}
{liên kết cộng hóa trị không phân cực.}
{\True liên kết cộng hóa trị có cực.}
{liên kết kim loại.}
\loigiai{Do H và O là hai phi kim, có độ âm điện chênh lệch nhưng không quá lớn (3,44 - 2,2 = 1,24) nên liên kết O-H là liên kết cộng hóa trị có cực.}
\end{ex}

%%%%=================EX_9====================%%%
\begin{ex}
Liên kết cộng hóa trị được tạo thành do
\choice
{lực hút tĩnh điện giữa các ion.}
{\True sự dùng chung cặp electron giữa hai nguyên tử.}
{sự cho nhận electron giữa hai nguyên tử.}
{lực hút giữa hạt nhân và các electron.}
\loigiai{Liên kết cộng hóa trị được hình thành do sự dùng chung một hay nhiều cặp electron giữa hai nguyên tử.}
\end{ex}

%%%%=================EX_10====================%%%
\begin{ex}
Chất nào sau đây có liên kết cộng hóa trị không cực?
\choice
{H$_2$O}
{\True Br$_2$}
{NH$_3$}
{HCl}
\loigiai{Br$_2$ được tạo thành từ hai nguyên tử Br giống nhau nên liên kết trong phân tử Br$_2$ là liên kết cộng hóa trị không cực.}
\end{ex}

%%%%=================EX_11====================%%%
\begin{ex}
Cặp chất nào sau đây đều chỉ chứa liên kết cộng hóa trị?
\choice
{NaCl và MgO}
{NaOH và KOH}
{\True CH$_4$ và NH$_3$}
{KCl và CaO}
\loigiai{CH$_4$ và NH$_3$ là các hợp chất được tạo thành từ các nguyên tử phi kim nên chỉ chứa liên kết cộng hóa trị.}
\end{ex}

%%%%=================EX_12====================%%%
\begin{ex}
Trong phân tử N$_2$, hai nguyên tử nitơ liên kết với nhau bằng cách
\choice
{mỗi nguyên tử nitơ góp 1 electron.}
{mỗi nguyên tử nitơ góp 2 electron.}
{\True mỗi nguyên tử nitơ góp 3 electron.}
{một nguyên tử nitơ góp 2 electron, nguyên tử còn lại góp 4 electron.}
\loigiai{Trong phân tử N$_2$, mỗi nguyên tử nitơ góp 3 electron để tạo thành 3 cặp electron chung (liên kết ba).}
\end{ex}

%%%%=================EX_13====================%%%
\begin{ex}
Phân tử nào sau đây có liên kết cho - nhận?
\choice
{H$_2$O}
{\True CO}
{NH$_3$}
{Cl$_2$}
\loigiai{Trong phân tử CO, cặp electron liên kết thứ ba là do nguyên tử O cho nguyên tử C. }
\end{ex}

%%%%=================EX_14====================%%%
\begin{ex}
Độ âm điện của một nguyên tố đặc trưng cho
\choice
{khả năng nhường electron của nguyên tử đó khi hình thành liên kết hóa học.}
{\True khả năng hút electron của nguyên tử đó khi hình thành liên kết hóa học.}
{khả năng tham gia phản ứng hóa học của nguyên tử đó.}
{khả năng tạo thành liên kết ion của nguyên tử đó.}
\loigiai{Độ âm điện của một nguyên tố đặc trưng cho khả năng hút electron của nguyên tử nguyên tố đó khi hình thành liên kết hóa học.}
\end{ex}

%%%%=================EX_15====================%%%
\begin{ex}
Liên kết trong phân tử nào sau đây là liên kết cộng hóa trị có cực?
\choice
{O$_2$}
{N$_2$}
{\True HF}
{Cl$_2$}
\loigiai{Liên kết cộng hóa trị có cực được hình thành giữa hai nguyên tử phi kim khác nhau. Vậy HF có liên kết cộng hóa trị có cực.}
\end{ex}

%%%%=================EX_16====================%%%
\begin{ex}
Cho các phân tử: H$_2$O, NH$_3$, CO$_2$, CH$_4$. Phân tử có độ phân cực lớn nhất là
\choice
{CO$_2$}
{CH$_4$}
{\True H$_2$O}
{NH$_3$}
\loigiai{H$_2$O có độ phân cực lớn nhất do nguyên tử O có độ âm điện lớn và cấu trúc phân tử dạng góc làm cho mômen lưỡng cực lớn.}
\end{ex}

%%%%=================EX_17====================%%%
\begin{ex}
Liên kết cộng hóa trị trong phân tử nào sau đây có cực nhất?
\choice
{H-Cl}
{H-Br}
{\True H-F}
{H-I}
\loigiai{Trong các halogen, F có độ âm điện lớn nhất nên liên kết H-F có cực nhất.}
\end{ex}

%%%%=================EX_18====================%%%
\begin{ex}
Nguyên tử X có 4 electron lớp ngoài cùng. X có thể hình thành với H
\choice
{1 liên kết cộng hóa trị.}
{2 liên kết cộng hóa trị.}
{3 liên kết cộng hóa trị.}
{\True 4 liên kết cộng hóa trị.}
\loigiai{Nguyên tử X có 4 electron lớp ngoài cùng, mỗi electron sẽ góp chung với 1 electron của nguyên tử H để tạo thành liên kết cộng hóa trị. Vậy X có thể hình thành với H 4 liên kết cộng hóa trị (ví dụ như CH$_4$).}
\end{ex}
%%%%=================EX_19====================%%%
\begin{ex}
Trong phân tử nước (H$_2$O), góc liên kết  $\widehat{HOH}$ xấp xỉ là:
\choice
{180$^\circ$}
{120$^\circ$}
{90$^\circ$}
{\True 104,5$^\circ$}
\loigiai{Trong phân tử nước, nguyên tử O có 2 cặp electron chưa liên kết, đẩy 2 liên kết O-H lại gần nhau, làm cho góc liên kết $\widehat{HOH}$ xấp xỉ 104,5$^\circ$.}
\end{ex}

%%%%=================EX_20====================%%%
\begin{ex}
Số cặp electron chưa liên kết trên nguyên tử trung tâm của phân tử NH$_3$ là
\choice
{0}
{\True 1}
{2}
{3}
\loigiai{Trong phân tử NH$_3$, nguyên tử N có 5 electron lớp ngoài cùng, trong đó có 3 electron tham gia liên kết với 3 nguyên tử H, còn lại 1 cặp electron chưa liên kết.}
\end{ex}

%%%%=================EX_21====================%%%
\begin{ex}
Cho biết độ âm điện của các nguyên tố: H (2,20); O (3,44); Cl (3,16); S (2,58). Liên kết trong phân tử nào sau đây có độ phân cực lớn nhất?
\choice
{H$_2$O}
{\True HCl}
{H$_2$S}
{SO$_2$}
\loigiai{Độ phân cực của liên kết phụ thuộc vào hiệu độ âm điện giữa hai nguyên tử. Hiệu độ âm điện càng lớn thì liên kết càng phân cực. 

* H$_2$O: 3,44 - 2,20 = 1,24
* HCl: 3,16 - 2,20 = 0,96
* H$_2$S: 2,58 - 2,20 = 0,38
* SO$_2$: 3,44 - 2,58 = 0,86

Vậy liên kết trong phân tử HCl có độ phân cực lớn nhất.}
\end{ex}

%%%%=================EX_22====================%%%
\begin{ex}
Dãy gồm các chất trong phân tử chỉ chứa liên kết đơn là
\choice
{N$_2$, O$_2$, F$_2$}
{CO$_2$, SO$_2$, H$_2$O}
{\True CH$_4$, NH$_3$, H$_2$O}
{C$_2$H$_4$, C$_2$H$_2$, CO$_2$}
\loigiai{Liên kết đơn là liên kết được tạo thành bởi 1 cặp electron chung. Trong các chất trên, chỉ có CH$_4$, NH$_3$ và H$_2$O có liên kết đơn.}
\end{ex}

%%%%=================EX_23====================%%%
\begin{ex}
Nguyên tử A có 3 electron ở lớp ngoài cùng, nguyên tử B có 7 electron ở lớp ngoài cùng. Công thức phân tử của hợp chất tạo thành giữa A và B là
\choice
{AB$_2$}
{A$_2$B}
{AB$_3$}
{\True A$_2$B$_3$}
\loigiai{Để đạt cấu hình bền vững, A có xu hướng cho 3 electron, B có xu hướng nhận 1 electron. Vậy 2 nguyên tử A sẽ liên kết với 3 nguyên tử B, tạo thành hợp chất A$_2$B$_3$.}
\end{ex}

%%%%=================EX_24====================%%%
\begin{ex}
Trong phân tử nào sau đây, nguyên tử trung tâm không tuân theo quy tắc bát tử?
\choice
{CO$_2$}
{NH$_3$}
{H$_2$O}
{\True BF$_3$}
\loigiai{Trong phân tử BF$_3$, nguyên tử B chỉ có 6 electron lớp ngoài cùng (tạo 3 liên kết với 3 nguyên tử F).}
\end{ex}

%%%%=================EX_25====================%%%
\begin{ex}
Liên kết đôi gồm
\choice
{hai liên kết $\sigma$}
{hai liên kết $\pi$}
{\True một liên kết $\sigma$ và một liên kết $\pi$}
{hai liên kết ion}
\loigiai{Liên kết đôi gồm một liên kết $\sigma$ (sigma) bền vững và một liên kết $\pi$ (pi) kém bền vững hơn.}
\end{ex}

%%%%=================EX_26====================%%%
\begin{ex}
Cho các phân tử sau: H$_2$, HCl, HF, HBr, HI. Phân tử có năng lượng liên kết lớn nhất là
\choice
{H$_2$}
{HCl}
{\True HF}
{HI}
\loigiai{Năng lượng liên kết phụ thuộc vào độ bền của liên kết. Trong các phân tử trên, liên kết H-F có độ bền lớn nhất do độ âm điện của F lớn nhất, dẫn đến năng lượng liên kết lớn nhất.}
\end{ex}

%%%%=================EX_27====================%%%
\begin{ex}
Ý nào sau đây \textbf{không đúng} khi nói về liên kết cộng hóa trị?
\choice
{Liên kết cộng hóa trị được hình thành do sự dùng chung electron giữa hai nguyên tử.}
{Liên kết cộng hóa trị có thể là liên kết đơn, liên kết đôi hoặc liên kết ba.}
{Liên kết cộng hóa trị được hình thành giữa hai nguyên tử phi kim.}
{\True Liên kết cộng hóa trị luôn là liên kết có cực.}
\loigiai{Liên kết cộng hóa trị có thể là liên kết có cực hoặc không cực. Liên kết cộng hóa trị không cực được hình thành giữa hai nguyên tử giống nhau.}
\end{ex}

%%%%=================EX_28====================%%%
\begin{ex}
Phân tử nào sau đây có chứa cả liên kết cộng hóa trị và liên kết cho - nhận?
\choice
{HCl}
{CO$_2$}
{\True HNO$_3$}
{H$_2$O}
\loigiai{Trong phân tử HNO$_3$, có 2 liên kết cộng hóa trị (N-O) và 1 liên kết cho - nhận (N-O).}
\end{ex}

%%%%=================EX_29====================%%%
\begin{ex}
Để đạt được cấu hình electron bền vững của khí hiếm gần nhất, nguyên tử clo có xu hướng
\choice
{nhường đi 1 electron}
{\True nhận thêm 1 electron}
{góp chung 1 electron}
{nhận thêm 7 electron}
\loigiai{Nguyên tử clo có 7 electron lớp ngoài cùng, để đạt được cấu hình electron bền vững của khí hiếm gần nhất (Argon), clo có xu hướng nhận thêm 1 electron.}
\end{ex}

%%%%=================EX_30====================%%%
\begin{ex}
Cho các chất sau: Cl$_2$, HCl, NaCl, NaF.  Số chất chứa liên kết cộng hóa trị là
\choice
{1}
{\True 2}
{3}
{4}
\loigiai{Cl$_2$ và HCl là các hợp chất được tạo thành từ các phi kim nên chứa liên kết cộng hóa trị.}
\end{ex}