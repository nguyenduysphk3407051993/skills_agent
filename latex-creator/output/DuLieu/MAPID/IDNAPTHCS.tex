-[6] Lớp 6
----[D] Số học
-------[1] Tập hợp các số tự nhiên
----------[1] Tập hợp
----------[2] Cách ghi số tự nhiên
----------[3] Thứ tự trong tập hợp các số tự nhiên
----------[4] Phép cộng và phép trừ số tự nhiên
----------[5] Phép nhân và phép chia số tự nhiên
----------[6] Lũy thừa với số mũ tự nhiên
----------[7] Thứ tự thực hiện các phép tính
-------[2] Tính chia hết trong tập hợp các số tự nhiên
----------[1] Quan hệ chia hết và tính chất
----------[2] Dấu hiệu chia hết
----------[3] Số nguyên tố
----------[4] Ước chung. Ước chung lớn nhất
----------[5] Bội chung. Bội chung nhỏ nhất
-------[3] Số nguyên
----------[1] Tập hợp các số nguyên
----------[2] Phép cộng và phép trừ số nguyên
----------[3] Quy tắc dấu ngoặc
----------[4] Phép nhân số nguyên
----------[5] Phép chia hết. Ước và bội của một số nguyên
-------[6] Phân số
----------[1] Mở rộng phân số. Phân số bằng nhau
----------[2] So sánh phân số. Hỗn số dương
----------[3] Phép cộng và phép trừ phân số
----------[4] Phép nhân và phép chia phân số
----------[5] Hai bài toán về phân số
-------[7] Số thập phân
----------[1] Số thập phân
----------[2] Tính toán với số thập phân
----------[3] Làm tròn và ước lượng
----------[4] Một số bài toán về tỉ số và tỉ số phần trăm
----[H] Hình học
-------[4] Một số hình phẳng trong thực tiễn
----------[1] Hình tam giác đều. Hình vuông. Hình lục giác đều
----------[2] Hình chữ nhật. Hình thoi. Hình bình hành. Hình thang cân
----------[3] Chu vi và diện tích của một số tứ giác đã học
-------[5] Tính đối xứng của hình phẳng trong tự nhiên
----------[1] Hình có trục đối xứng
----------[2] Hình có tâm đối xứng
-------[8] Những hình học cơ bản
----------[1] Điểm và đường thẳng
----------[2] Điểm nằm giữa hai điểm. Tia
----------[3] Đoạn thẳng. Độ dài đoạn thẳng
----------[4] Trung điểm của đoạn thẳng
----------[5] Góc
----------[6] Số đo góc
----[X] Xác suất - Thống kê
-------[9] Dữ liệu và xác suất thực nghiệm
----------[1] Dữ liệu và thu thập dữ liệu
----------[2] Bảng thống kê và biểu đồ tranh
----------[3] Biểu đồ cột
----------[4] Biểu đồ cột kép
----------[5] Kết quả có thể và sự kiện trong trò chơi, thí nghiệm
----------[6] Xác suất thực nghiệm
-[7] Lớp 7
----[D] Đại số
-------[1] Số hữu tỉ
----------[1] Tập hợp các số hữu tỉ
----------[2] Cộng, trừ, nhân, chia số hữu tỉ
----------[3] Lũy thừa với số mũ tự nhiên của một số hữu tỉ
----------[4] Thứ tự thực hiện các phép tính. Quy tắc chuyển vế
-------[2] Số thực
----------[1] Làm quen với số thập phân vô hạn tuần hoàn
----------[2] Số vô tỉ. Căn bậc hai số học
----------[3] Tập hợp các số thực
-------[6] Tỉ lệ thức và đại lượng tỉ lệ
----------[1] Tỉ lệ thức
----------[2] Tính chất của dãy tỉ số bằng nhau
----------[3] Đại lượng tỉ lệ thuận
----------[4] Đại lượng tỉ lệ nghịch
-------[7] Biểu thức đại số và đa thức một biến
----------[1] Biểu thức đại số
----------[2] Đa thức một biến
----------[3] Phép cộng và phép trừ đa thức một biến
----------[4] Phép nhân đa thức một biến
----------[5] Phép chia đa thức một biến
----[H] Hình học
-------[3] Góc và đường thẳng song song
----------[1] Góc ở vị trí đặc biệt. Tia phân giác của một góc
----------[2] Hai đường thẳng song song và dấu hiệu nhận biết
----------[3] Tiên đề Euclid. Tính chất của hai đường thẳng song song
----------[4] Định lí và chứng minh định lí
-------[4] Tam giác bằng nhau
----------[1] Tổng các góc trong một tam giác
----------[2] Hai tam giác bằng nhau. Trường hợp bằng nhau thứ nhất của tam giác
----------[3] Trường hợp bằng nhau thứ hai và thứ ba của tam giác
----------[4] Các trường hợp bằng nhau của tam giác vuông
----------[5] Tam giác cân. Đường trung trực của đoạn thẳng
-------[9] Quan hệ giữa các yếu tố trong một tam giác
----------[1] Quan hệ giữa góc và cạnh đối diện trong một tam giác
----------[2] Quan hệ giữa đường vuông góc và đường xiên
----------[3] Quan hệ giữa ba cạnh của một tam giác
----------[4] Sự đồng quy của ba đường trung tuyến, ba đường phân giác trong một tam giác
----------[5] Sự đồng quy của ba đường trung trực, ba đường cao trong một tam giác
-------[0] Một số hình khối trong thực tiễn
----------[1] Hình hộp chữ nhật và hình lập phương
----------[2] Hình lăng trụ đứng tam giác và hình lăng trụ đứng tứ giác
----[X] Xác suất - Thống kê
-------[5] Thu thập và biểu diễn dữ liệu
----------[1] Thu thập và phân loại dữ liệu
----------[2] Biểu đồ hình quạt tròn
----------[3] Biểu đồ đoạn thẳng
-------[8] Làm quen với biến cố và xác suất của biến cố
----------[1] Làm quen với biến cố
----------[2] Làm quen với xác suất của biến cố
-[8] Lớp 8
----[D] Đại số
-------[1] Đa thức
----------[1] Đơn thức
----------[2] Đa thức
----------[3] Phép cộng và phép trừ đa thức
----------[4] Phép nhân đa thức
----------[5] Phép chia đa thức cho đơn thức
-------[2] Hằng đẳng thức đáng nhớ và ứng dụng
----------[1] Hiệu hai bình phương. Bình phương của một tổng hay một hiệu
----------[2] Lập phương của một tổng hay một hiệu
----------[3] Tổng và hiệu hai lập phương
----------[4] Phân tích đa thức thành nhân tử
-------[6] Phân thức đại số
----------[1] Phân thức đại số
----------[2] Tính chất cơ bản của phân thức đại số
----------[3] Phép cộng và phép trừ phân thức đại số
----------[4] Phép nhân và phép chia phân thức đại số
-------[7] Phương trình bậc nhất và hàm số bậc nhất
----------[1] Phương trình bậc nhất một ẩn
----------[2] Giải bài toán bằng cách lập phương trình
----------[3] Khái niệm hàm số và đồ thị của hàm số
----------[4] Hàm số bậc nhất và đồ thị của hàm số bậc nhất
----------[5] Hệ số góc của đường thẳng
----[H] Hình học
-------[3] Tứ giác
----------[1] Tứ giác
----------[2] Hình thang cân
----------[3] Hình bình hành
----------[4] Hình chữ nhật
----------[5] Hình thoi và hình vuông
-------[4] Định lý Thalès
----------[1] Định lý Thalès trong tam giác
----------[2] Đường trung bình của tam giác
----------[3] Tính chất đường phân giác của tam giác
-------[9] Tam giác đồng dạng
----------[1] Hai tam giác đồng dạng
----------[2] Ba trường hợp đồng dạng của hai tam giác
----------[3] Định lý Pythagore và ứng dụng
----------[4] Các trường hợp đồng dạng của hai tam giác vuông
----------[5] Hình đồng dạng
----[X] Xác suất - Thống kê
-------[5] Dữ liệu và biểu đồ
----------[1] Thu thập và phân loại dữ liệu
----------[2] Biểu diễn dữ liệu bằng bảng, biểu đồ
----------[3] Phân tích số liệu thống kê dựa vào biểu đồ
-------[8] Mở đầu về tính xác suất của biến cố
----------[1] Kết quả có thể và kết quả thuận lợi
----------[2] Cách tính xác suất của biến cố bằng tỉ số
----------[3] Mối liên hệ giữa xác suất thực nghiệm với xác suất và ứng dụng
-[9] Lớp 9
----[D] Đại số
-------[1] Phương trình và hệ hai phương trình bậc nhất hai ẩn
----------[1] Khái niệm phương trình và hệ hai phương trình bậc nhất hai ẩn
----------[2] Giải hệ hai phương trình bậc nhất hai ẩn
----------[3] Giải bài toán bằng cách lập hệ phương trình
-------[2] Phương trình và bất phương trình bậc nhất một ẩn
----------[1] Phương trình quy về phương trình bậc nhất một ẩn
----------[2] Bất đẳng thức và tính chất
----------[3] Bất phương trình bậc nhất một ẩn
-------[3] Căn bậc hai và căn bậc ba
----------[1] Căn bậc hai và căn thức bậc hai
----------[2] Khai căn bậc hai với phép nhân và phép chia
----------[3] Biến đổi đơn giản và rút gọn biểu thức chứa căn thức bậc hai
----------[4] Căn bậc ba và căn thức bậc ba
-------[6] Hàm số $y=ax^2~(a \ne 0)$. Phương trình bậc hai một ẩn
----------[1] Hàm số $y=ax^2~(a \ne 0)$
----------[2] Phương trình bậc hai một ẩn
----------[3] Định lý Viète và ứng dụng
----------[4] Giải bài toán bằng cách lập phương trình
----[H] Hình học
-------[4] Hệ thức lượng trong tam giác vuông
----------[1] Tỉ số lượng giác của góc nhọn
----------[2] Một số hệ thức giữa cạnh, góc trong tam giác vuông và ứng dụng
-------[5] Đường tròn
----------[1] Mở đầu về đường tròn
----------[2] Cung và dây của một đường tròn
----------[3] Độ dài của cung tròn. Diện tích hình quạt tròn và hình vành khuyên
----------[4] Vị trí tương đối của đường thẳng và đường tròn
----------[5] Vị trí tương đối của hai đường tròn
-------[9] Đường tròn ngoại tiếp và đường tròn nội tiếp
----------[1] Góc nội tiếp
----------[2] Đường tròn ngoại tiếp và đường tròn nội tiếp của một tam giác
----------[3] Tứ giác nội tiếp
----------[4] Đa giác đều
-------[0] Một số hình khối trong thực tiễn
----------[1] Hình trụ và hình nón
----------[2] Hình cầu
----[X] Xác suất - Thống kê
-------[7] Tần số và tần số tương đối
----------[1] Bảng tần số và biểu đồ tần số
----------[2] Bảng tần số tương đối và biểu đồ tần số tương đối
----------[3] Bảng tần số, tần số tương đối ghép nhóm và biểu đồ
-------[8] Xác suất của biến cố trong một số mô hình xác suất đơn giản
----------[1] Phép thử ngẫu nhiên và không gian mẫu
----------[2] Xác suất của biến cố liên quan tới phép thử
%Cấu hình nội dung Map ID
%END%