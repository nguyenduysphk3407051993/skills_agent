%
%%=== HỆ THỐNG DẠNG TOÁN HỢP NHẤT - DƯƠNG PHƯỚC SANG ====
%Cấu hình chi tiết ID
%
%%Cấu hình mức độ dùng chung.
[Y] Yếu
[B] Trung bình
[K] Khá
[G] Giỏi
[T] Thực tế
%
-[0] Lớp 10
----[D] Đại số và giải tích
-------[1] Mệnh đề và tập hợp
----------[1] Mệnh đề
-------------[1] Xác định mệnh đề, mệnh đề chứa biến
-------------[2] Tính đúng-sai của mệnh đề (cơ bản)
-------------[3] Phủ định của một mệnh đề (cơ bản)
-------------[4] Mệnh đề kéo theo, mệnh đề đảo, mệnh đề tương đương
-------------[5] Mệnh đề với mọi, tồn tại (và phủ định chúng)
-------------[6] Áp dụng mệnh đề vào suy luận có lí
----------[2] Tập hợp
-------------[1] Tập hợp và phần tử của tập hợp
-------------[2] Tập hợp con. Hai tập hợp bằng nhau
-------------[3] Ký hiệu khoảng, đoạn, nửa khoảng
----------[3] Các phép toán tập hợp
-------------[1] Giao và hợp của hai tập hợp (rời rạc)
-------------[2] Hiệu và phần bù của hai tập hợp (rời rạc)
-------------[3] Toán thực tế ứng dụng tập hợp
-------------[4] Giao và hợp (dùng đoạn, khoảng)
-------------[5] Hiệu và phần bù (dùng đoạn, khoảng)
-------[2] BPT và hệ BPT bậc nhất hai ẩn
----------[1] Bất phương trình bậc nhất hai ẩn
-------------[1] Các khái niệm về BPT bậc I hai ẩn
-------------[2] Miền nghiệm của BPT bậc I hai ẩn
-------------[3] Toán thực tế về BPT bậc I hai ẩn
----------[2] Hệ bất phương trình bậc nhất hai ẩn
-------------[1] Các khái niệm về Hệ BPT bậc I hai ẩn
-------------[2] Miền nghiệm của Hệ BPT bậc I hai ẩn
-------------[3] Toán thực tế về Hệ BPT bậc I hai ẩn
-------[3] Hàm số bậc hai và đồ thị
----------[1] Hàm số và đồ thị
-------------[1] Xác định một hàm số
-------------[2] Tập xác định của hàm số
-------------[3] Giá trị của hàm số
-------------[4] Đồ thị của hàm số
-------------[5] Tính đồng biến, nghịch biến
-------------[6] Toán thực tế về hàm số
-------------[7] [Giảm] Tính chẵn, lẻ
----------[2] Hàm số bậc hai
-------------[1] Xác định hàm số bậc hai
-------------[2] Bảng biến thiên, tính đơn điệu, max, min
-------------[3] Đồ thị của hàm số bậc hai
-------------[4] Bài toán về sự tương giao
-------------[5] Toán thực tế ứng dụng hàm số bậc hai
-------------[6] Hàm số chứa dấu giá trị tuyệt đối
-------[4] Bất phương trình bậc 2 một ẩn
----------[1] Dấu của tam thức bậc 2
-------------[1] Xác định tam thức bậc 2
-------------[2] Dấu của tam thức bậc 2 và ứng dụng
-------------[3] Bài toán xét dấu biết BXD, đồ thị
-------------[4] Toán thực tế, liên môn
-------------[5] [Giảm] Xét dấu biểu thức dạng tích, thương
----------[2] Giải bất phương trình bậc 2 một ẩn
-------------[1] Bất phương trình bậc 2 và ứng dụng
-------------[2] Giải bất phương trình bậc hai biết BXD, đồ thị
-------------[3] Toán thực tế, liên môn
-------------[4] [Giảm] Bất phương trình dạng tích, thương
-------------[5] [Giảm] Hệ bất phương trình BPT bậc 2
-------------[6] [Giảm] Bất phương trình chứa căn, |.|
----------[3] Phương trình quy về phương trình bậc hai
-------------[1] Phương trình căn\{fx\} = căn\{gx\} và mở rộng
-------------[2] Phương trình căn\{fx\} = gx và mở rộng
-------------[3] Phương trình căn thức có tham số
-------------[4] Toán hình, toán thực tế vận dụng PT bậc 2
-------------[5] [Giảm] Phương trình căn thức (dạng khác)
-------------[6] [Giảm] Phương trình khác quy về PT bậc 2
----[H] Hình học và đo lường
-------[1] Hệ thức lượng trong tam giác
----------[1] Giá trị lượng giác của góc (0-180)
-------------[1] Xét dấu của biểu thức lượng giác
-------------[2] Tính các giá trị lượng giác
-------------[3] Biến đổi, rút gọn biểu thức lượng giác
----------[2] Định lý sin và định lý côsin
-------------[1] Bài toán chỉ dùng định lý Sin, Côsin
-------------[2] Bài toán có dùng công thức diện tích
-------------[3] Biến đổi, rút gọn biểu thức
-------------[4] Nhận dạng tam giác
----------[3] Giải tam giác và ứng dụng thực tế
-------------[1] Giải tam giác
-------------[2] Các ứng dụng thực tế
-------[2] Véctơ (chưa xét tọa độ)
----------[1] Khái niệm véctơ
-------------[1] Xác định một véctơ
-------------[2] Xét phương và hướng của các véctơ
-------------[3] Hai véctơ bằng nhau
-------------[4] Hai véctơ đối nhau
-------------[5] Độ dài của một véctơ
-------------[6] Toán thực tế, liên môn dùng véctơ
----------[2] Tổng và hiệu của hai véctơ
-------------[1] Tính toán, thu gọn tổng các véctơ
-------------[2] Tính toán, thu gọn hiệu các véctơ
-------------[3] Tính đúng-sai của 1 đẳng thức véctơ
-------------[4] Tìm điểm nhờ đẳng thức véctơ
-------------[5] Tính độ dài của véctơ tổng, hiệu
-------------[6] Toán thực tế, liên môn dùng véctơ
----------[3] Tích của một số với véctơ
-------------[1] Xác định k.vec\{a\} và độ dài của nó
-------------[2] Biến đổi, thu gọn 1 đẳng thức véctơ
-------------[3] Tìm điểm nhờ đẳng thức véctơ
-------------[4] Sự cùng phương của 2 véctơ và ứng dụng
-------------[5] Phân tích 1 véctơ theo 2 véctơ không cùng phương
-------------[6] Tính độ dài của véctơ tổng, hiệu
-------------[7] Tập hợp điểm
-------------[8] Cực trị hình học
-------------[9] Toán thực tế, liên môn dùng véctơ
----------[4] Tích vô hướng (chưa xét tọa độ)
-------------[1] Tích vô hướng, góc giữa 2 véctơ
-------------[2] Tìm góc nhờ tích vô hướng
-------------[3] Đẳng thức về tích vô hướng hoặc độ dài
-------------[4] Điều kiện vuông góc
-------------[5] Các bài toán tìm điểm và tập hợp điểm
-------------[6] Cực trị và chứng minh bất đẳng thức
-------------[7] Toán thực tế, liên môn
-------[3] Véctơ (trong hệ tọa độ)
----------[1] Toạ độ của véctơ
-------------[1] Tọa độ điểm, độ dài đại số của véctơ trên 1 trục
-------------[2] Phép toán véctơ (tổng, hiệu, tích với số) trong Oxy
-------------[3] Tọa độ điểm và véctơ trên hệ trục Oxy
-------------[4] Sự cùng phương của 2 véctơ và ứng dụng
-------------[5] Phân tích một véctơ theo 2 véctơ không cùng phương
-------------[6] Toán thực tế dùng hệ toạ độ
----------[2] Tích vô hướng (theo tọa độ)
-------------[1] Tích vô hướng, góc giữa 2 véctơ
-------------[2] Tìm góc nhờ tích vô hướng
-------------[3] Đẳng thức về tích vô hướng hoặc độ dài
-------------[4] Điều kiện vuông góc
-------------[5] Các bài toán tìm điểm và tập hợp điểm
-------------[6] Cực trị và chứng minh bất đẳng thức
-------------[7] Toán thực tế, liên môn
-------[4] Phương pháp tọa độ trong mặt phẳng
----------[1] Đường thẳng trong mặt phẳng toạ độ
-------------[1] Điểm, véctơ, hệ số góc của đường thẳng
-------------[2] Phương trình đường thẳng
-------------[3] Vị trí tương đối giữa hai đường thẳng
-------------[4] Bài toán về góc giữa hai đường thẳng
-------------[5] Bài toán về khoảng cách
-------------[6] Bài toán tìm điểm
-------------[7] Bài toán dùng cho tam giác, tứ giác
-------------[8] Bài toán thực tế, PP tọa độ hóa
-------------[9] [Giảm] Bài toán có dùng PT chính tắc
----------[2] Đường tròn trong mặt phẳng toạ độ
-------------[1] Tìm tâm, bán kính và điều kiện là đường tròn
-------------[2] Phương trình đường tròn
-------------[3] Phương trình tiếp tuyến của đường tròn
-------------[4] Vị trí tương đối liên quan đường tròn
-------------[5] Toán tổng hợp đường thẳng và đường tròn
-------------[6] Bài toán dùng cho tam giác, tứ giác
-------------[7] Bài toán thực tế, PP tọa độ hóa
-------------[8] [Giảm] Dạng chính tắc của pt đường thẳng
----------[3] Ba đường conic trong mặt phẳng toạ độ
-------------[1] Xác định các yếu tố của elip
-------------[2] Phương trình chính tắc của elip
-------------[3] Bài toán điểm trên elip
-------------[4] Xác định các yếu tố của hypebol
-------------[5] Phương trình chính tắc của hypebol
-------------[6] Bài toán điểm trên hypebol
-------------[7] Xác định các yếu tố của parabol
-------------[8] Phương trình chính tắc của parabol
-------------[9] Bài toán điểm trên parabol
-------------[0] Bài toán tổng hợp 3 đường conic
-------------[A] Bài toán thực tế, PP tọa độ hóa
-------------[B] [Chuyển] Dạng toán Sách chuyên đề
----[X] Thống kê và xác suất
-------[1] Thống kê
----------[1] Số gần đúng. Sai số
-------------[1] Tính và ước lượng sai số tuyệt đối, tương đối
-------------[2] Tính và xác định độ chính xác của kết quả
-------------[3] Quy tròn số gần đúng
-------------[4] Viết số gần đúng cho số đúng biết độ chính xác
----------[2] Mô tả và biểu diễn dữ liệu trên các bảng và biểu đồ
-------------[1] Đọc và phân tích thông tin trên bảng số liệu
-------------[2] Đọc và phân tích thông tin trên Biểu đồ
-------------[3] Số liệu bất thường trên bảng số liệu
-------------[4] Số liệu bất thường trên Biểu đồ
----------[3] Các số đặc trưng đo xu thế trung tâm của mẫu số liệu
-------------[1] Số trung bình cộng
-------------[2] Số trung vị
-------------[3] Tứ phân vị
-------------[4] Mốt
-------------[5] Câu hỏi lý thuyết
----------[4] Các số đặc trưng đo mức độ phân tán của mẫu số liệu
-------------[1] Khoảng biến thiên, khoảng tứ phân vị
-------------[2] Giá trị bất thường của mẫu số liệu
-------------[3] Phương sai, độ lệch chuẩn của mẫu số liệu
-------------[4] Câu hỏi lý thuyết
-------[2] Đại số tổ hợp
----------[1] Quy tắc cộng-quy tắc nhân
-------------[1] Bài toán chỉ sử dụng quy tắc cộng
-------------[2] Bài toán chỉ sử dụng quy tắc nhân
-------------[3] Bài toán kết hợp quy tắc cộng và quy tắc nhân
-------------[4] Bài toán dùng quy tắc bù trừ
-------------[5] Bài toán đếm số tự nhiên
-------------[6] Sơ đồ hình cây
----------[2] Hoán vị-chỉnh hợp-tổ hợp
-------------[1] Bài toán xếp chỗ (không tròn, không lặp)
-------------[2] Bài toán chọn người
-------------[3] Bài toán chọn đối tượng khác
-------------[4] Bài toán có yếu tố hình học
-------------[5] Bài toán đếm số tự nhiên
-------------[6] Hoán vị bàn tròn
-------------[7] Hoán vị lặp
-------------[8] Bài toán có biểu thức P,C,A
----------[3] Nhị thức Newton
-------------[1] Khai triển một nhị thức Newton
-------------[2] Tìm hệ số, số hạng trong khai triển
-------------[3] Tính tổng nhờ khai triển nhị thức Newton
-------------[4] Toán tổ hợp có dùng nhị thức Newton
-------[3] Xác suất
----------[1] Không gian mẫu và biến cố
-------------[1] Mô tả không gian mẫu, biến cố
-------------[2] Đếm phần tử không gian mẫu, biến cố
-------------[3] Các câu hỏi lý thuyết tổng hợp
----------[2] Xác suất của biến cố
-------------[1] Liên quan xúc xắc, đồng tiền (PP liệt kê)
-------------[2] Liên quan việc sắp xếp chỗ
-------------[3] Liên quan việc chọn người
-------------[4] Liên quan việc chọn đối tượng khác
-------------[5] Liên quan hình học
-------------[6] Liên quan việc đếm số
-------------[7] Liên quan bàn tròn hoặc hoán vị lặp
-------------[8] Liên quan vấn đề khác
-------------[9] Các câu hỏi lý thuyết tổng hợp
-[1] Lớp 11
----[D] Đại số và giải tích
-------[1] Hàm số lượng giác và phương trình lượng giác
----------[1] Góc lượng giác
-------------[1] Chuyển đổi đơn vị độ và radian
-------------[2] Số đo của một góc lượng giác
-------------[3] Độ dài của một cung tròn
-------------[4] Đường tròn lượng giác và ứng dụng
-------------[5] Các bài toán thực tế, liên môn
-------------[6] Câu hỏi lý thuyết
----------[2] Giá trị lượng giác của một góc lượng giác
-------------[1] Xét dấu các giá trị lượng giác
-------------[2] Tính giá trị lượng giác của một góc
-------------[3] Ứng dụng các góc có liên quan đặc biệt
-------------[4] Biến đổi, thu gọn 1 biểu thức lượng giác
-------------[5] Các bài toán có yếu tố thực tế, liên môn
-------------[6] Câu hỏi lý thuyết
----------[3] Các công thức lượng giác
-------------[1] Áp dụng công thức cộng
-------------[2] Áp dụng công thức nhân đôi - hạ bậc
-------------[3] Áp dụng công thức biến đổi tích <-> tổng
-------------[4] Kết hợp nhiều công thức lượng giác
-------------[5] Nhận dạng tam giác
-------------[6] Các bài toán có yếu tố thực tế, liên môn
-------------[7] Câu hỏi lý thuyết
----------[4] Hàm số lượng giác và đồ thị
-------------[1] Tìm tập xác định
-------------[2] Xét tính đơn điệu
-------------[3] Xét tính chẵn, lẻ
-------------[4] Xét tính tuần hoàn, tìm chu kỳ
-------------[5] Tìm tập giá trị và min, max
-------------[6] Bảng biến thiên và đồ thị
----------[5] Phương trình lượng giác cơ bản
-------------[1] Khái niệm phương trình tương đương
-------------[2] Điều kiện có nghiệm
-------------[3] Phương trình cơ bản dùng Radian
-------------[4] Phương trình cơ bản dùng Độ
-------------[5] Phương trình đưa về dạng cơ bản
-------------[6] Toán thực tế, liên môn
----------[6] [Giảm] Phương trình lượng giác thường gặp
-------------[1] Phương trình bậc n theo một hàm số lượng giác
-------------[2] Phương trình đẳng cấp bậc n đối với sinx và cosx
-------------[3] Phương trình bậc nhất đối với sinx và cosx
-------------[4] Phương trình đối xứng, phản đối xứng
-------------[5] Phương trình lượng giác không mẫu mực
-------------[6] Phương trình lượng giác có chứa ẩn ở mẫu số
-------------[7] Phương trình lượng giác có chứa tham số
-------------[8] Bài toán thực tế
-------[2] Dãy số. Cấp số cộng. Cấp số nhân
----------[1] Dãy số
-------------[1] Số hạng tổng quát, biểu diễn dãy số
-------------[2] Tìm số hạng cụ thể của dãy số
-------------[3] Dãy số tăng, dãy số giảm
-------------[4] Dãy số bị chặn
-------------[5] Toán thực tế về dãy số
-------------[6] Câu hỏi lý thuyết
----------[2] Cấp số cộng
-------------[1] Nhận diện cấp số cộng, công sai d
-------------[2] Số hạng tổng quát của cấp số cộng
-------------[3] Tìm số hạng cụ thể trong cấp số cộng
-------------[4] Điều kiện để dãy số là cấp số cộng
-------------[5] Tính tổng của cấp số cộng
-------------[6] Các bài toán thực tế
----------[3] Cấp số nhân
-------------[1] Nhận diện cấp số nhân, công bội q
-------------[2] Số hạng tổng quát của cấp số nhân
-------------[3] Tìm số hạng cụ thể trong cấp số nhân
-------------[4] Điều kiện để dãy số là cấp số nhân
-------------[5] Tính tổng của cấp số nhân
-------------[6] Kết hợp cấp số nhân và cấp số cộng
-------------[7] Các bài toán thực tế
-------[3] Giới hạn. Hàm số liên tục
----------[1] Giới hạn của dãy số
-------------[1] Câu hỏi lý thuyết
-------------[2] Phương pháp đặt thừa số chung (lim hữu hạn)
-------------[3] Phương pháp lượng liên hợp (lim hữu hạn)
-------------[4] Giới hạn vô cực
-------------[5] Cấp số nhân lùi vô hạn
-------------[6] Toán thực tế, liên môn
-------------[7] [Giảm] Nguyên lí kẹp
----------[2] Giới hạn của hàm số
-------------[1] Câu hỏi lý thuyết
-------------[2] Thay số trực tiếp
-------------[3] PP đặt thừa số chung, kết quả hữu hạn
-------------[4] PP đặt thừa số chung, kết quả vô cực
-------------[5] PP lượng liên hợp, kết quả hữu hạn
-------------[6] PP lượng liên hợp, kết quả vô cực
-------------[7] Giới hạn một bên
-------------[8] Toán thực tế, liên môn
----------[3] Hàm số liên tục
-------------[1] Câu hỏi lý thuyết
-------------[2] Tính liên tục thể hiện qua đồ thị
-------------[3] Hàm số liên tục tại một điểm
-------------[4] Hàm số liên tục trên khoảng, đoạn
-------------[5] Bài toán phương trình có nghiệm
-------------[6] Toán thực tế, liên môn
-------[4] Hàm số mũ và hàm số lôgarít
----------[1] Phép tính luỹ thừa
-------------[1] Tính giá trị của biểu thức chứa lũy thừa
-------------[2] Biến đổi, rút gọn biểu thức chứa lũy thừa
-------------[3] Điều kiện cho luỹ thừa, căn thức
-------------[4] So sánh các lũy thừa
----------[2] Phép tính lôgarít
-------------[1] Tính giá trị biểu thức chứa lôgarít
-------------[2] Biến đổi, rút gọn, biểu diễn biểu thức chứa lôgarít
-------------[3] Số e và bài toán lãi kép
-------------[4] Toán thực tế, liên môn
----------[3] Hàm số mũ. Hàm số lôgarít
-------------[1] Tập xác định của hàm số
-------------[2] Sự biến thiên và đồ thị của hàm số mũ, lôgarít
-------------[3] So sánh các luỹ thừa và lôgarít
-------------[4] Bài toán thực tế, liên môn
-------------[5] Lý thuyết tổng hợp hàm số lũy thừa, mũ, lôgarít
----------[4] Phương trình, bất phương trình mũ và lôgarít
-------------[1] Điều kiện có nghiệm
-------------[2] Phương trình mũ, lôgarít cơ bản
-------------[3] Bất phương trình mũ, lôgarít cơ bản
-------------[4] Phương trình mũ, lôgarít đưa về cùng cơ số
-------------[5] Bất phương trình mũ, lôgarít đưa về cùng cơ số
-------------[6] Bài toán thực tế, liên môn
----------[5] [Giảm] Các phương pháp giải được giảm tải
-------------[1] Phương pháp đặt ẩn phụ cho PT mũ, lôgarít
-------------[2] Phương pháp lôgarít hóa, mũ cho PT mũ, lôgarít
-------------[3] Phương pháp hàm số, đánh giá cho PT mũ, lôgarít
-------------[4] Hệ PT mũ, lôgarít
-------------[5] Phương pháp đặt ẩn phụ với BPT mũ, lôgarít
-------------[6] Phương pháp lôgarít hóa, mũ cho BPT mũ, lôgarít
-------------[7] Phương pháp hàm số, đánh giá cho BPT mũ, lôgarít
-------------[8] Hệ BPT mũ, lôgarít
-------[5] Đạo hàm
----------[1] Đạo hàm
-------------[1] Tính đạo hàm bằng định nghĩa
-------------[2] Số gia hàm số, số gia biến số
-------------[3] Ý nghĩa hình học của đạo hàm
-------------[4] Ý nghĩa Vật lý của đạo hàm
-------------[5] Bài toán thực tế, liên môn khác
----------[2] Các quy tắc đạo hàm
-------------[1] Tính đạo hàm
-------------[2] Đẳng thức có y và y'
-------------[3] Tiếp tuyến tại một điểm
-------------[4] Tiếp tuyến biết trước hệ số góc
-------------[5] Tiếp tuyến chưa biết tiếp điểm và hệ số góc
-------------[6] Bài toán thực tế, liên môn
-------------[7] Giới hạn hàm số lượng giác, hàm số mũ, lôgarít
-------------[8] Dùng đạo hàm cho nhị thức Newton
----------[3] Đạo hàm cấp hai
-------------[1] Tính đạo hàm cấp hai
-------------[2] Đẳng thức có y và (y', y'')
-------------[3] Ý nghĩa Vật lý của đạo hàm cấp hai
----[H] Hình học và đo lường
-------[1] Đường thẳng, mặt phẳng. Quan hệ song song trong không gian
----------[1] Điểm, đường thẳng và mặt phẳng
-------------[1] Câu hỏi lý thuyết
-------------[2] Hình biểu diễn của một hình không gian
-------------[3] Tìm giao tuyến của hai mặt phẳng
-------------[4] Tìm giao điểm của đường thẳng và mặt phẳng
-------------[5] Xác định thiết diện
-------------[6] Ba điểm thẳng hàng, ba đường thẳng đồng quy
-------------[7] Bài toán quỹ tích và điểm cố định
-------------[8] Bài toán thực tế
----------[2] Hai đường thẳng song song
-------------[1] Câu hỏi lý thuyết
-------------[2] Hai đường thẳng song song
-------------[3] Tìm giao tuyến bằng cách kẻ song song
-------------[4] Tìm giao điểm của đường thẳng và mặt phẳng
-------------[5] Xác định thiết diện bằng cách kẻ song song
-------------[6] Ba điểm thẳng hàng
-------------[7] Bài toán quỹ tích và điểm cố định
-------------[8] Bài toán thực tế
----------[3] Đường thẳng và mặt phẳng song song
-------------[1] Câu hỏi lý thuyết
-------------[2] Đường thẳng song song với mặt phẳng
-------------[3] Tìm giao tuyến bằng cách kẻ song song
-------------[4] Tìm giao điểm của đường thẳng và mặt phẳng
-------------[5] Xác định thiết diện bằng cách kẻ song song
-------------[6] Ba điểm thẳng hàng
-------------[7] Bài toán quỹ tích và điểm cố định
-------------[8] Bài toán thực tế
----------[4] Hai mặt phẳng song song
-------------[1] Câu hỏi lý thuyết
-------------[2] Hai mặt phẳng song song
-------------[3] Tìm giao tuyến bằng cách kẻ song song
-------------[4] Tìm giao điểm của đường thẳng và mặt phẳng
-------------[5] Xác định thiết diện bằng cách kẻ song song
-------------[6] Bài toán tổng hợp
-------------[7] Bài toán thực tế
----------[5] Hình lăng trụ và hình hộp
-------------[1] Bài toán về hình lăng trụ
-------------[2] Bài toán về hình hộp
----------[6] Phép chiếu song song
-------------[1] Câu hỏi lý thuyết
-------------[2] Hình biểu diễn của một hình không gian
-------------[3] Xác định yế tố song song
-------[2] Quan hệ vuông góc trong không gian
----------[1] Hai đường thẳng vuông góc
-------------[1] Câu hỏi lí thuyết
-------------[2] Xác định hai đường thẳng vuông góc
-------------[3] Tìm góc giữa hai đường thẳng
-------------[4] Các bài toán thực tế
----------[2] Đường thẳng vuông góc với mặt phẳng
-------------[1] Câu hỏi lí thuyết
-------------[2] Xác định đường thẳng và mặt phẳng vuông góc
-------------[3] Xác định hai đường thẳng vuông góc
-------------[4] Dựng mặt phẳng, tìm thiết diện
-------------[5] Tìm góc giữa hai đường thẳng
-------------[6] Các bài toán thực tế
----------[3] Phép chiếu vuông góc
-------------[1] Lý thuyết về phép chiếu vuông góc
-------------[2] Hình chiếu vuông góc của đa giác trên mặt phẳng
-------------[3] Các bài toán thực tế
----------[4] Hai mặt phẳng vuông góc
-------------[x] !!!-> Đề có góc (d,(P)), góc nhị diện, xếp vào bài 6
-------------[x] !!!-> Xếp lăng trụ đứng, chóp (cụt) đều vào bài 7
-------------[x] -----------------------------
-------------[1] Câu hỏi lí thuyết
-------------[2] Xác định quan hệ vuông góc
-------------[3] Dựng mặt phẳng, thiết diện
-------------[4] Xác định góc giữa hai mặt phẳng
-------------[5] Nhận dạng và tính toán liên quan các hình thông dụng
-------------[6] Bài toán cho trước góc giữa d và (P)
-------------[7] Các bài toán thực tế
----------[5] Khoảng cách
-------------[x] !!!-> Đề có góc (d,(P)), góc nhị diện, xếp vào bài 6
-------------[x] !!!-> Xếp lăng trụ đứng, chóp (cụt) đều vào bài 7
-------------[x] !!!-> Xếp toán thể tích cũng vào bài 7
-------------[x] -----------------------------
-------------[1] Câu hỏi lí thuyết
-------------[2] Khoảng cách giữa 2 điểm, từ 1 điểm đến 1 đường thẳng
-------------[3] Khoảng cách từ một điểm đến một mặt phẳng
-------------[4] Khoảng cách giữa hai đường thẳng chéo nhau
-------------[5] Đường vuông góc chung của hai đường thẳng chéo nhau
-------------[6] Các bài toán thực tế
----------[6] Góc giữa đường thẳng và mặt phẳng. Góc nhị diện
-------------[1] Góc giữa đường thẳng và mặt phẳng
-------------[2] Góc nhị diện, góc phẳng nhị diện
-------------[3] Góc giữa 2 mặt phẳng, biết trước góc (d,(P))
-------------[4] Khoảng cách giữa điểm, đường, biết trước góc (d,(P))
-------------[5] Khoảng cách giữa điểm - mặt phẳng, biết trước góc (d,(P))
-------------[6] Khoảng cách giữa 2 đường chéo nhau, biết trước góc (d,(P))
-------------[7] Các bài toán thực tế
----------[7] Hình lăng trụ đứng. Hình chóp đều. Thể tích
-------------[1] Tính góc, cạnh, đường cao, diện tích
-------------[2] Thể tích khối chóp, chóp cụt
-------------[3] Thể tích các khối lăng trụ
-------------[4] Thể tích các khối khác
-------------[5] Các bài toán thực tế
-------------[6] Bài toán cực trị
----[X] Thống kê và xác suất
-------[1] Các số đặc trưng đo xu thế trung tâm cho mẫu số liệu ghép nhóm
----------[1] Số trung bình và mốt của mẫu số liệu ghép nhóm
-------------[1] Mẫu số liệu ghép nhóm
-------------[2] Số trung bình
-------------[3] Mốt
-------------[4] Câu hỏi lý thuyết
----------[2] Trung vị và tứ phân vị của mẫu số liệu ghép nhóm
-------------[1] Trung vị
-------------[2] Tứ phân vị
-------------[3] Câu hỏi lý thuyết
-------[2] Xác suất
----------[1] Các khái niệm về biến cố
-------------[1] Câu hỏi lí thuyết
-------------[2] Mô tả không gian mẫu, biến cố
-------------[3] Xác định các loại biến cố
----------[2] Công thức xác suất
-------------[1] Câu hỏi lí thuyết
-------------[2] Tính xác suất bằng định nghĩa
-------------[3] Sơ đồ hình cây
-------------[4] Tính xác suất bằng quy tắc nhân
-------------[5] Tính xác suất bằng quy tắc cộng
-------------[6] Tính xác suất bằng cách kết hợp quy tắc
-[2] Lớp 12
----[D] Giải tích
-------[1] Ứng dụng đạo hàm để khảo sát hàm số
----------[1] Sự đồng biến và nghịch biến của hàm số
-------------[1] Xét tính đơn điệu của hàm số cho bởi công thức
-------------[2] Xét tính đơn điệu dựa vào bảng biến thiên, đồ thị
-------------[3] Tìm tham số m để hàm số đơn điệu
-------------[4] Ứng dụng tính đơn điệu để chứng minh BĐT, giải PT, BPT, HPT
----------[2] Cực trị của hàm số
-------------[1] Tìm cực trị của hàm số cho bởi công thức
-------------[2] Tìm cực trị dựa vào BBT, đồ thị
-------------[3] Tìm m để hàm số đạt cực trị tại 1 điểm cho trước
-------------[4] Tìm m để hàm số, đồ thị hàm số bậc ba có cực trị thỏa mãn điều kiện
-------------[5] Tìm m để hàm số, đồ thị hàm số trùng phương có cực trị thỏa mãn điều kiện
-------------[6] Tìm m để hàm số, đồ thị hàm số các hàm số khác có cực trị thỏa mãn điều kiện
----------[3] Giá trị lớn nhất và giá trị nhỏ nhất của hàm số
-------------[1] GTLN, GTNN trên đoạn
-------------[2] GTLN, GTNN trên khoảng
-------------[3] Sử dụng các đánh giá, bất đẳng thức cổ điển
-------------[4] Ứng dụng GTNN, GTLN trong bài toán PT, BPT
-------------[5] GTLN, GTNN hàm nhiều biến
-------------[6] Bài toán ứng dụng, tối ưu, thực tế
----------[4] Đường tiệm cận
-------------[1] Xác định các ĐTC của HS (không chứa tham số) hoặc biết BBT, đồ thị
-------------[2] Bài toán xác định các đường tiệm cận của hàm số có chứa tham số
-------------[3] Bài toán liên quan đến đồ thị hàm số và các đường tiệm cận
----------[5] Khảo sát sự biến thiên và vẽ đồ thị hàm số
-------------[1] Nhận dạng đồ thị, bảng biến thiên
-------------[2] Các phép biến đổi đồ thị
-------------[3] Biện luận số giao điểm dựa vào đồ thị, bảng biến thiên
-------------[4] Sự tương giao của hai đồ thị (liên quan đến tọa độ giao điểm)
-------------[5] Đồ thị của hàm đạo hàm
-------------[6] Phương trình tiếp tuyến của đồ thị hàm số
-------------[7] Điểm đặc biệt của đồ thị hàm số
-------[2] Hàm số lũy thừa-Hàm số mũ và Hàm số lô-ga-rít
----------[1] Lũy thừa
-------------[1] Tính giá trị của biểu thức chứa lũy thừa
-------------[2] Biến đổi, rút gọn, biểu diễn các biểu thức chứa lũy thừa
-------------[3] So sánh các lũy thừa
----------[2] Hàm số lũy thừa
-------------[1] Tập xác định của hàm số chứa hàm lũy thừa
-------------[2] Đạo hàm hàm số lũy thừa
-------------[3] Khảo sát sự biến thiên và đồ thị hàm số lũy thừa
-------------[4] Tìm giá trị lớn nhất, giá trị nhỏ nhất của biểu thức chứa hàm lũy thừa
----------[3] Lô-ga-rít
-------------[1] Tính giá trị biểu thức chứa lô-ga-rít
-------------[2] Biến đổi, rút gọn, biểu diễn biểu thức chứa lô-ga-rít
-------------[3] So sánh các biểu thức lô-ga-rít
----------[4] Hàm số mũ. Hàm số lô-ga-rít
-------------[1] Tập xác định của hàm số mũ, hàm số lô-ga-rít
-------------[2] Tính đạo hàm hàm số mũ, hàm số lô-ga-rít
-------------[3] Khảo sát sự biến thiên và đồ thị của hàm số mũ, lô-ga-rít
-------------[4] Tìm giá trị lớn nhất, nhỏ nhất của biểu thức chứa hàm mũ, hàm lô-ga-rít
-------------[5] Bài toán thực tế
-------------[6] Giới hạn, liên tục liên quan hàm số mũ, lô-ga-rít
-------------[7] Lý thuyết tổng hợp hàm số lũy thừa, mũ, lô-ga-rít
----------[5] Phương trình mũ và phương trình lô-ga-rít
-------------[1] Phương trình cơ bản
-------------[2] Phương pháp đưa về cùng cơ số
-------------[3] Phương pháp đặt ẩn phụ
-------------[4] Phương pháp lô-ga-rít hóa, mũ hóa
-------------[5] Phương pháp hàm số, đánh giá
-------------[6] Bài toán thực tế
----------[6] Bất phương trình mũ và lô-ga-rít
-------------[1] Bất phương trình cơ bản
-------------[2] Phương pháp đưa về cùng cơ số
-------------[3] Phương pháp đặt ẩn phụ
-------------[4] Phương pháp lô-ga-rít hóa, mũ hóa
-------------[5] Phương pháp hàm số, đánh giá
-------------[6] Bài toán thực tế
-------[3] Nguyên hàm, tích phân và ứng dụng
----------[1] Nguyên hàm
-------------[1] Định nghĩa, tính chất và nguyên hàm cơ bản
-------------[2] Phương pháp đổi biến số
-------------[3] Phương pháp nguyên hàm từng phần
----------[2] Tích phân
-------------[1] Định nghĩa, tính chất và tích phân cơ bản
-------------[2] Phương pháp đổi biến số
-------------[3] Phương pháp tích phân từng phần
-------------[4] Tích phân của hàm ẩn. Tích phân đặc biệt
----------[3] Ứng dụng của tích phân
-------------[1] Diện tích hình phẳng được giới hạn bởi các đồ thị
-------------[2] Bài toán thực tế sử dụng diện tích hình phẳng
-------------[3] Thể tích giới hạn bởi các đồ thị (tròn xoay)
-------------[4] Thể tích tính theo mặt cắt S(x)
-------------[5] Bài toán thực tế và ứng dụng thể tích
-------------[6] Ứng dụng vào tính tổng khai triển nhị thức
-------------[7] Ứng dụng tích phân vào bài toán liên môn (lý, hóa, sinh, kinh tế)
-------[4] Số phức
----------[1] Khái niệm số phức
-------------[1] Xác định các yếu tố cơ bản của số phức
-------------[2] Biểu diễn hình học cơ bản của số phức
----------[2] Phép cộng, trừ và nhân số phức
-------------[1] Thực hiện phép tính
-------------[2] Xác định các yếu tố cơ bản của số phức qua các phép toán
-------------[3] Bài toán quy về giải phương trình, hệ phương trình nghiệm thực
-------------[4] Bài toán tập hợp điểm
----------[3] Phép chia số phức
-------------[1] Thực hiện phép tính
-------------[2] Xác định các yếu tố cơ bản của số phức qua các phép toán
-------------[3] Bài toán quy về giải phương trình, hệ phương trình nghiệm thực
-------------[4] Bài toán tập hợp điểm
----------[4] Phương trình bậc hai hệ số thực
-------------[1] Giải phương trình. Tính toán biểu thức nghiệm
-------------[2] Định lí Viet và ứng dụng
-------------[3] Phương trình quy về bậc hai
----------[5] Cực trị
-------------[1] Phương pháp hình học
-------------[2] Phương pháp đại số
----[H] Hình học
-------[1] Khối đa diện
----------[1] Khái niệm về khối đa diện
-------------[1] Nhận diện hình đa diện, khối đa diện
-------------[2] Xác định số đỉnh, cạnh, mặt bên của một khối đa diện
-------------[3] Phân chia, lắp ghép các khối đa diện
-------------[4] Phép biến hình trong không gian
----------[2] Khối đa diện lồi và khối đa diện đều
-------------[1] Nhận diện đa diện lồi
-------------[2] Nhận diện loại đa diện đều
-------------[3] Tính chất đối xứng
----------[3] Khái niệm về thể tích của khối đa diện
-------------[1] Diện tích xung quanh, diện tích toàn phần của khối đa diện
-------------[2] Tính thể tích các khối đa diện
-------------[3] Tỉ số thể tích
-------------[4] Các bài toán khác(góc, khoảng cách,...) liên quan đến thể tích khối đa diện
-------------[5] Bài toán thực tế về khối đa diện
-------------[6] Bài toán cực trị
-------[2] Mặt nón, mặt trụ, mặt cầu
----------[1] Khái niệm về mặt tròn xoay
-------------[1] Thể tích khối nón, khối trụ
-------------[2] Sxq, Stp, độ dài đường sinh, chiều cao, bán kính đáy, thiết diện
-------------[3] Khối tròn xoay nội tiếp, ngoại tiếp khối đa diện
-------------[4] Bài toán thực tế về khối nón, khối trụ
-------------[5] Bài toán cực trị về khối nón, khối trụ
-------------[6] Câu hỏi lý thuyết
----------[2] Mặt cầu
-------------[1] Bài toán sử dụng định nghĩa, tính chất, vị trí tương đối
-------------[2] Khối cầu ngoại tiếp khối đa diện
-------------[3] Khối cầu nội tiếp khối đa diện
-------------[4] Bài toán thực tế về khối cầu
-------------[5] Bài toán cực trị về khối cầu
-------------[6] Bài toán tổng hợp về khối nón, khối trụ, khối cầu
-------[3] Phương pháp tọa độ trong không gian
----------[1] Hệ tọa độ trong không gian
-------------[1] Tìm tọa độ điểm, véc-tơ liên quan đến hệ trục Oxyz
-------------[2] Tích vô hướng và ứng dụng
-------------[3] Xác định tâm, bán kính, viết PT mặt cầu đơn giản,...
-------------[4] Các bài toán cực trị
----------[2] Phương trình mặt phẳng
-------------[1] Tích có hướng và ứng dụng
-------------[2] Xác định VTPT
-------------[3] Viết phương trình mặt phẳng
-------------[4] Tìm tọa độ điểm liên quan đến mặt phẳng
-------------[5] Góc
-------------[6] Khoảng cách
-------------[7] Vị trí tương đối giữa hai mặt phẳng, giữa mặt cầu và mặt phẳng
-------------[8] Các bài toán cực trị
----------[3] Phương trình đường thẳng trong không gian
-------------[1] Xác định VTCP
-------------[2] Viết phương trình đường thẳng
-------------[3] Tìm tọa độ điểm liên quan đến đường thẳng
-------------[4] Góc
-------------[5] Khoảng cách
-------------[6] Vị trí tương đối giữa hai đường thẳng, giữa ĐT và MP
-------------[7] Bài toán liên quan giữa đường thẳng - mặt phẳng - mặt cầu
-------------[8] Các bài toán cực trị
----------[4] Ứng dụng của phương pháp tọa độ
-------------[1] Bài toán HHKG
-------------[2] Bài toán đại số