%
%Map ID v2.0.0.2 by MyLT.
%ID 3 bộ sách mới lớp 11 do thầy Dương Phước Sang gửi.
%
%Nếu là ID5 sẽ theo định đạng: %[Tham số 1 Tham số 2 Tham số 3 Tham số 4 Tham số 5]. Ví dụ: %[1D2B3]
%Nếu là ID6 sẽ theo định đạng: %[Tham số 1 Tham số 2 Tham số 3 Tham số 4 Tham số 5 - Tham số 6]. Ví dụ: %[1D2B3-1]
%Có thể thay đổi nội dung của mức độ nhưng vị trí của mức độ trong ID không đổi (thông số thứ 3 từ trái qua).
%Cú pháp của thông số: [Giá trị] Mô tả thông số. Ví dụ: [0] Lớp 10 thì 0 là giá trị để lưu vào ID.
%
%Chú ý: Các dấu gạch ngang không được thay đổi.
%
%Cấu hình tên các tham số
%
Tên tham số 1: Lớp
Tên tham số 2: Môn
Tên tham số 3: Chương
Tên tham số 4: Mức độ
Tên tham số 5: Bài
Tên tham số 6: Dạng
%
%Cấu hình chi tiết ID
%
%%Cấu hình mức độ dùng chung.
[Y] Yếu
[B] Trung bình
[K] Khá
[G] Giỏi
[T] Thực tế
%
%Cấu hình nội dung
%
-[0] Lớp 10
----[C] Cánh Diều
-------[1] Mệnh đề toán học. Tập hợp
----------[1] Mệnh đề toán học
-------------[1] Nhận biết mệnh đề, mệnh đề chứa biến
-------------[2] Xét tính đúng - sai của mệnh đề
-------------[3] Phủ định của một mệnh đề
-------------[4] Mệnh đề kéo theo, mệnh đề đảo, hai mệnh đề tương đương
-------------[5] Mệnh đề với ký hiệu mọi và tồn tại
----------[2] Tập hợp. Các phép toán trên tập hợp
-------------[1] Tập hợp và phần tử của tập hợp
-------------[2] Tập hợp con - Hai tập hợp bằng nhau
-------------[3] Giao và hợp của hai tập hợp
-------------[4] Hiệu và phần bù của hai tập hợp
-------------[5] Toán thực tế ứng dụng của tập hợp
-------------[6] Xác định giao, hợp của các khoảng, đoạn, nửa khoảng
-------------[7] Xác định hiệu và phần bù của các khoảng, đoạn, nửa khoảng
-------[2] Bất phương trình và hệ bất phương trình bậc nhất hai ẩn
----------[1] Bất phương trình bậc nhất hai ẩn
-------------[1] Bất phương trình bậc nhất hai ẩn và các bài toán liên quan
-------------[2] Nghiệm của bất phương trình bậc nhất hai ẩn
----------[2] Hệ bất phương trình bậc nhất hai ẩn
-------------[1] Hệ bất phương trình bậc nhất hai ẩn và các bài toán liên quan
-------------[2] Các bài toán ứng dụng thực tế
-------------[3] Miền nghiệm của hệ bất phương trình bậc nhất hai ẩn
-------[3] Hàm số và đồ thị
----------[1] Hàm số và đồ thị
-------------[1] Tính giá trị của hàm số
-------------[2] Tìm tập xác định của hàm số
-------------[3] Đồ thị của hàm số
-------------[4] Tính đồng biến, nghịch biến của hàm số
----------[2] Hàm số bậc hai. Đồ thị hàm số bậc hai và ứng dụng
-------------[1] Bảng biến thiên, tính đơn điệu, GTLN - GTNN
-------------[2] Xác định hàm số bậc hai
-------------[3] Đồ thị của hàm số bậc hai
-------------[4] Bài toán tương giao
-------------[5] Toán thực tế ứng dụng hàm số bậc hai
-------------[6] Hàm số chứa dấu giá trị tuyệt đối
----------[3] Dấu của tam thức bậc hai
-------------[1] Nhận dạng tam thức và xét dấu biểu thức
-------------[2] Bài toán thực tế
----------[4] Bất phương trình bậc hai một ẩn
-------------[1] Giải và các bài toán liên quan bất phương trình bậc hai
-------------[2] Giải và các bài toán liên quan bất phương trình tích, thương
-------------[3] Giải và các bài toán liên quan hệ bất phương bậc hai
-------------[4] Phương trình và bất phương trình chứa dấu giá trị tuyệt đối
-------------[5] Phương trình và bất phương trình chứa dấu giá trị tuyệt đối có tham số
----------[5] Hai dạng phương trình quy về phương trình bậc hai
-------------[1] Phương trình và bất phương trình chứa căn thức
-------------[2] Phương trình và bất phương trình chứa căn thức có tham số
-------[4] Hệ thức lượng trong tam giác. Vectơ
----------[1] Giá trị lượng giác của một góc từ 0 độ đến 180 độ. Định lý sin và cosin trong tam giác
-------------[1] Xét dấu của các giá trị lượng giác
-------------[2] Tính các giá trị lượng giác
-------------[3] Chứng minh, rút gọn các biểu thức lượng giác
-------------[4] Tính toán các đại lượng trong tam giác
-------------[5] Chứng minh các hệ thức
-------------[6] Nhận dạng tam giác
----------[2] Giải tam giác
-------------[1] Giải tam giác và các ứng dụng thực tế
----------[3] Khái niệm vectơ
-------------[1] Xác định một vectơ
-------------[2] Sự cùng phương và hướng của hai vectơ
-------------[3] Hai vectơ bằng nhau, độ dài của vectơ
----------[4] Tổng và hiệu của hai vectơ
-------------[1] Tổng của hai vectơ, tổng của nhiều vectơ
-------------[2] Chứng minh đẳng thức vectơ
-------------[3] Xác định vị trí của một điểm nhờ đẳng thức vectơ
-------------[4] Tìm vectơ đối, hiệu của hai vectơ
-------------[5] Tính độ dài tổng và hiệu các vectơ
----------[5] Tích của một số với một vectơ
-------------[1] Xác định vectơ k\vec{a}, tính độ dài vectơ
-------------[2] Chứng minh các đẳng thức vectơ, thu gọn biểu thức
-------------[3] Xác định vị trí của một điểm nhờ đẳng thức vectơ
-------------[4] Phân tích một vectơ theo hai vectơ không cùng phương
-------------[5] Chứng minh ba điểm thẳng hàng, hai đường thẳng song song, hai điểm trùng nhau
-------------[6] Tập hợp điểm
-------------[7] Cực trị
----------[6] Tích vô hướng của hai vectơ
-------------[1] Tính tích vô hướng của hai vectơ và xác định góc
-------------[2] Chứng minh đẳng thức về tích vô hướng hoặc độ dài
-------------[3] Điều kiện vuông góc
-------------[4] Các bài toán tìm điểm và tập hợp điểm
-------------[5] Cực trị và chứng minh bất đẳng thức
-------------[6] Xác định góc giữa hai vectơ, góc giữa hai đường thẳng
-------[5] Đại số tổ hợp
----------[1] Quy tắc cộng. Quy tắc nhân. Sơ đồ hình cây
-------------[1] Bài toán sử dụng quy tắc cộng
-------------[2] Bài toán sử dụng quy tắc nhân
-------------[3] Bài toán kết hợp quy tắc cộng và quy tắc nhân
----------[2] Hoán vị. Chỉnh hợp
-------------[1] Bài toán chỉ sử dụng P hoặc A
-------------[2] Bài toán kết hợp P và A
----------[3] Tổ hợp
-------------[1] Bài toán chỉ sử dụng C
-------------[2] Bài toán kết hợp P, C và A
-------------[3] Bài toán liên quan đến hình học
-------------[4] Hoán vị bàn tròn
-------------[5] Hoán vị lặp
-------------[6] Giải PT, BPT, HPT, chứng minh liên quan đến P,C,A
----------[4] Nhị thức Newton
-------------[1] Khai triển một nhị thức Newton
-------------[2] Tìm hệ số, số hạng trong khai triển nhị thức Newton
-------------[3] Chứng minh, tính giá trị của biểu thức đại số tổ hợp có sử dụng nhị thức Newton
-------[6] Một số yếu tố thống kê và xác suất
----------[1] Số gần đúng. Sai số
-------------[1] Tính và ước lượng sai số tuyệt đối
-------------[2] Tính và xác định độ chính xác của kết quả
----------[2] Các số đặc trưng đo xu thế trung tâm cho mẫu số liệu không ghép nhóm
-------------[1] Số trung bình cộng
-------------[2] Số trung vị
-------------[3] Tứ phân vị
-------------[4] Mốt
-------------[5] Câu hỏi lý thuyết
----------[3] Các số đặc trưng đo mức độ phân tán cho mẫu số liệu không ghép nhóm
-------------[1] Khoảng biến thiên, khoảng tứ phân vị
-------------[2] Tính phương sai, độ lệch chuẩn dựa vào bảng số liệu cho trước
-------------[3] Câu hỏi lý thuyết
----------[4] Xác suất của biến cố trong một số trò chơi đơn giản
-------------[1] Trò chơi tung đồng xu
-------------[2] Trò chơi gieo xúc xắc
----------[5] Xác suất của biến cố
-------------[1] Mô tả không gian mẫu, biến cố
-------------[2] Các câu hỏi lý thuyết tổng hợp
-------------[3] Tính xác suất bằng định nghĩa
-------------[4] Tính xác suất bằng công thức cộng
-------------[5] Tính xác suất bằng công thức nhân
-------------[6] Bài toán kết hợp quy tắc cộng và quy tắc nhân xác suất
-------[7] Phương pháp tọa độ trong mặt phẳng
----------[1] Toạ độ của vectơ
-------------[1] Tìm tọa độ của một điểm
-------------[2] Xác định tọa độ của vectơ
----------[2] Biểu thức toạ độ của các phép toán vectơ
-------------[1] Tìm tọa độ các vectơ tổng, hiệu, k\vec{a} và tích vô hướng của 2 vectơ
-------------[2] Độ dài vectơ
-------------[3] Phân tích một vectơ theo hai vectơ không cùng phương
-------------[4] Chứng minh ba điểm thẳng hàng, hai đường thẳng song song
----------[3] Phương trình đường thẳng
-------------[1] Xác định các yếu tố của đường thẳng
-------------[2] Viết phương trình đường thẳng
----------[4] Vi trí tương đối và góc giữa hai đường thẳng. Khoảng cách từ một điểm đến một đường thẳng.
-------------[1] Vị trí tương đối giữa hai đường thẳng
-------------[2] Bài toán liên quan góc giữa hai đường thẳng
-------------[3] Bài toán liên quan công thức khoảng cách
-------------[4] Bài toán liên quan đến tìm điểm
-------------[5] Bài toán thực tế
----------[5] Phương trình đường tròn
-------------[1] Xác định tâm, bán kính và điều kiện là đường tròn
-------------[2] Viết phương trình đường tròn
-------------[3] Viết phương trình đường tiếp tuyến của đường tròn
-------------[4] Vị trí tương đối của đường tròn và đường thẳng, hai đường tròn
-------------[5] Các dạng toán tổng hợp đường thẳng và đường tròn
-------------[6] Bài toán thực tế
----------[6] Ba đường conic trong mặt phẳng toạ độ
-------------[1] Xác định các yếu tố của elip
-------------[2] Viết phương trình chính tắc của elip
-------------[3] Bài toán tìm điểm trên elip
-------------[4] Xác định các yếu tố của hypebol
-------------[5] Viết phương trình chính tắc của hypebol
-------------[6] Bài toán tìm điểm trên hypebol
-------------[7] Xác định các yếu tố của parabol
-------------[8] Viết phương trình chính tắc của parabol
-------------[9] Bài toán tìm điểm trên parabol
-------------[0] Bài toán thực tế
----[T] Chân Trời Sáng Tạo
-------[1] Mệnh đề và tập hợp
----------[1] Mệnh đề
-------------[1] Nhận biết mệnh đề, mệnh đề chứa biến
-------------[2] Xét tính đúng - sai của mệnh đề
-------------[3] Phủ định của một mệnh đề
-------------[4] Mệnh đề kéo theo, mệnh đề đảo, hai mệnh đề tương đương
-------------[5] Mệnh đề với ký hiệu mọi và tồn tại
----------[2] Tập hợp
-------------[1] Tập hợp và phần tử của tập hợp
-------------[2] Tập hợp con - Hai tập hợp bằng nhau
-------------[3] Khoảng, đoạn
----------[3] Các phép toán tập hợp
-------------[1] Giao và hợp của hai tập hợp
-------------[2] Hiệu và phần bù của hai tập hợp
-------------[3] Toán thực tế ứng dụng của tập hợp
-------------[4] Xác định giao, hợp của các khoảng, đoạn, nửa khoảng
-------------[5] Xác định hiệu và phần bù của các khoảng, đoạn, nửa khoảng
-------[2] Bất phương trình và hệ bất phương trình bậc nhất hai ẩn
----------[1] Bất phương trình bậc nhất hai ẩn
-------------[1] Bất phương trình bậc nhất hai ẩn và các bài toán liên quan
-------------[2] Nghiệm của bất phương trình bậc nhất hai ẩn
----------[2] Hệ bất phương trình bậc nhất hai ẩn
-------------[1] Hệ bất phương trình bậc nhất hai ẩn và các bài toán liên quan
-------------[2] Các bài toán ứng dụng thực tế
-------------[3] Miền nghiệm của hệ bất phương trình bậc nhất hai ẩn
-------[3] Hàm số bậc hai và đồ thị
----------[1] Hàm số và đồ thị
-------------[1] Tính giá trị của hàm số
-------------[2] Tìm tập xác định của hàm số
-------------[3] Đồ thị của hàm số
-------------[4] Tính đồng biến, nghịch biến của hàm số
----------[2] Hàm số bậc hai
-------------[1] Bảng biến thiên, tính đơn điệu, GTLN - GTNN
-------------[2] Xác định hàm số bậc hai
-------------[3] Đồ thị của hàm số bậc hai
-------------[4] Bài toán tương giao
-------------[5] Toán thực tế ứng dụng hàm số bậc hai
-------------[6] Hàm số chứa dấu giá trị tuyệt đối
-------[4] Hệ thức lượng trong tam giác
----------[1] Giá trị lượng giác của một góc từ 0 độ đến 180 độ
-------------[1] Xét dấu của các giá trị lượng giác
-------------[2] Tính các giá trị lượng giác
-------------[3] Chứng minh, rút gọn các biểu thức lượng giác
----------[2] Định lý sin và cosin
-------------[1] Tính toán các đại lượng trong tam giác
-------------[2] Chứng minh các hệ thức
-------------[3] Nhận dạng tam giác
----------[3] Giải tam giác và ứng dụng thực tế
-------------[1] Giải tam giác và các ứng dụng thực tế
-------[5] Vectơ
----------[1] Khái niệm vectơ
-------------[1] Xác định một véc-tơ
-------------[2] Sự cùng phương và hướng của hai véc-tơ
-------------[3] Hai véc-tơ bằng nhau, độ dài của véc-tơ
----------[2] Tổng và hiệu của hai véc-tơ
-------------[1] Tổng của hai véc-tơ, tổng của nhiều véc-tơ
-------------[2] Chứng minh đẳng thức véc-tơ
-------------[3] Xác định vị trí của một điểm nhờ đẳng thức véc-tơ
-------------[4] Tìm véc-tơ đối, hiệu của hai véc-tơ
-------------[5] Tính độ dài tổng và hiệu các véc-tơ
----------[3] Tích của một số với vectơ
-------------[1] Xác định véc-tơ k\vec{a}, tính độ dài véc-tơ
-------------[2] Chứng minh các đẳng thức véc-tơ, thu gọn biểu thức
-------------[3] Xác định vị trí của một điểm nhờ đẳng thức véc-tơ
-------------[4] Phân tích một véc-tơ theo hai véc-tơ không cùng phương
-------------[5] Chứng minh ba điểm thẳng hàng, hai đường thẳng song song, hai điểm trùng nhau
-------------[6] Tập hợp điểm
-------------[7] Cực trị
----------[4] Tích vô hướng của hai vectơ
-------------[1] Tính tích vô hướng của hai véc-tơ và xác định góc
-------------[2] Chứng minh đẳng thức về tích vô hướng hoặc độ dài
-------------[3] Điều kiện vuông góc
-------------[4] Các bài toán tìm điểm và tập hợp điểm
-------------[5] Cực trị và chứng minh bất đẳng thức
-------------[6] Xác định góc giữa hai véc-tơ, góc giữa hai đường thẳng
-------[6] Thống kê
----------[1] Số gần đúng. Sai số
-------------[1] Tính và ước lượng sai số tuyệt đối
-------------[2] Tính và xác định độ chính xác của kết quả
----------[2] Mô tả và biểu diễn dữ liệu trên các bảng và biểu đồ
-------------[1] Bảng số liệu
-------------[2] Biểu đồ
----------[3] Các số đặc trưng đo xu thế trung tâm của mẫu số liệu
-------------[1] Số trung bình cộng
-------------[2] Số trung vị
-------------[3] Mốt
-------------[4] Câu hỏi lý thuyết
----------[4] Các số đặc trưng đo mức độ phân tán của mẫu số liệu
-------------[1] Khoảng biến thiên, khoảng tứ phân vị, ngoại lệ
-------------[2] Tính phương sai, độ lệch chuẩn dựa vào bảng số liệu cho trước
-------------[3] Câu hỏi lý thuyết
-------[7] Bất phương trình bậc hai một ẩn
----------[1] Dấu của tam thức bậc hai
-------------[1] Nhận dạng tam thức và xét dấu biểu thức
----------[2] Giải bất phương trình bậc hai một ẩn
-------------[1] Giải và các bài toán liên quan bất phương trình bậc hai
-------------[2] Giải và các bài toán liên quan bất phương trình tích, thương
-------------[3] Giải và các bài toán liên quan hệ bất phương bậc hai
----------[3] Phương trình quy về phương trình bậc hai
-------------[1] Phương trình và bất phương trình chứa dấu giá trị tuyệt đối
-------------[2] Phương trình và bất phương trình chứa căn thức
-------------[3] Phương trình và bất phương trình chứa dấu giá trị tuyệt đối có tham số
-------------[4] Phương trình và bất phương trình chứa căn thức có tham số
-------[8] Đại số tổ hợp
----------[1] Quy tắc cộng-quy tắc nhân
-------------[1] Bài toán sử dụng quy tắc cộng
-------------[2] Bài toán sử dụng quy tắc nhân
-------------[3] Bài toán kết hợp quy tắc cộng và quy tắc nhân
----------[2] Hoán vị-chỉnh hợp-tổ hợp
-------------[1] Bài toán chỉ sử dụng P hoặc C hoặc A
-------------[2] Bài toán kết hợp P, C và A
-------------[3] Bài toán liên quan đến hình học
-------------[4] Hoán vị bàn tròn
-------------[5] Hoán vị lặp
-------------[6] Giải PT, BPT, HPT, chứng minh liên quan đến P,C,A
----------[3] Nhị thức Newton
-------------[1] Khai triển một nhị thức Newton
-------------[2] Tìm hệ số, số hạng trong khai triển nhị thức Newton
-------------[3] Chứng minh, tính giá trị của biểu thức đại số tổ hợp có sử dụng nhị thức Newton
-------[9] Phương pháp tọa độ trong mặt phẳng
----------[1] Toạ độ của vectơ
-------------[1] Tìm tọa độ của một điểm và độ dài đại số của một véc-tơ trên trục
-------------[2] Tìm tọa độ các véc-tơ tổng, hiệu và k\vec{a}
-------------[3] Xác định tọa độ của véc-tơ và của một điểm trên mặt phẳng tọa độ Oxy
-------------[4] Phân tích một véc-tơ theo hai véc-tơ không cùng phương
-------------[5] Chứng minh ba điểm thẳng hàng, hai đường thẳng song song
----------[2] Đường thẳng trong mặt phẳng toạ độ
-------------[1] Xác định các yếu tố của đường thẳng
-------------[2] Viết phương trình đường thẳng
-------------[3] Vị trí tương đối giữa hai đường thẳng
-------------[4] Bài toán liên quan góc giữa hai đường thẳng
-------------[5] Bài toán liên quan công thức khoảng cách
-------------[6] Bài toán liên quan đến tìm điểm
-------------[7] Bài toán thực tế
----------[3] Đường tròn trong mặt phẳng toạ độ
-------------[1] Xác định tâm, bán kính và điều kiện là đường tròn
-------------[2] Viết phương trình đường tròn
-------------[3] Viết phương trình đường tiếp tuyến của đường tròn
-------------[4] Vị trí tương đối của đường tròn và đường thẳng, hai đường tròn
-------------[5] Các dạng toán tổng hợp đường thẳng và đường tròn
-------------[6] Bài toán thực tế
----------[4] Ba đường conic trong mặt phẳng toạ độ
-------------[1] Xác định các yếu tố của elip
-------------[2] Viết phương trình chính tắc của elip
-------------[3] Bài toán tìm điểm trên elip
-------------[4] Xác định các yếu tố của hypebol
-------------[5] Viết phương trình chính tắc của hypebol
-------------[6] Bài toán tìm điểm trên hypebol
-------------[7] Xác định các yếu tố của parabol
-------------[8] Viết phương trình chính tắc của parabol
-------------[9] Bài toán tìm điểm trên parabol
-------------[0] Bài toán thực tế
-------[0] Xác suất
----------[1] Không gian mẫu và biến cố
-------------[1] Mô tả không gian mẫu, biến cố
-------------[2] Các câu hỏi lý thuyết tổng hợp
----------[2] Xác suất của biến cố
-------------[1] Các câu hỏi lý thuyết tổng hợp
-------------[2] Tính xác suất bằng định nghĩa
-------------[3] Tính xác suất bằng công thức cộng
-------------[4] Tính xác suất bằng công thức nhân
-------------[5] Bài toán kết hợp quy tắc cộng và quy tắc nhân xác suất
----[K] Kết Nối Tri Thức
-------[1] Mệnh đề và tập hợp
----------[1] Mệnh đề
-------------[1] Nhận biết mệnh đề, mệnh đề chứa biến
-------------[2] Xét tính đúng - sai của mệnh đề
-------------[3] Phủ định của một mệnh đề
-------------[4] Mệnh đề kéo theo, mệnh đề đảo, hai mệnh đề tương đương
-------------[5] Mệnh đề với ký hiệu mọi và tồn tại
----------[2] Tập hợp và các phép toán trên tập hợp
-------------[1] Tập hợp và phần tử của tập hợp
-------------[2] Tập hợp con - Hai tập hợp bằng nhau
-------------[3] Khoảng, đoạn
-------------[4] Giao và hợp của hai tập hợp
-------------[5] Hiệu và phần bù của hai tập hợp
-------------[6] Toán thực tế ứng dụng của tập hợp
-------------[7] Xác định giao, hợp của các khoảng, đoạn, nửa khoảng
-------------[8] Xác định hiệu và phần bù của các khoảng, đoạn, nửa khoảng
-------[2] Bất phương trình và hệ bất phương trình bậc nhất hai ẩn
----------[3] Bất phương trình bậc nhất hai ẩn
-------------[1] Bất phương trình bậc nhất hai ẩn
-------------[2] Nghiệm của bất phương trình bậc nhất hai ẩn
-------------[3] Biểu diễn miền nghiệm của bất phương trình bậc nhất hai ẩn
----------[4] Hệ bất phương trình bậc nhất hai ẩn
-------------[1] Hệ bất phương trình bậc nhất hai ẩn và các bài toán liên quan
-------------[2] Miền nghiệm của hệ bất phương trình bậc nhất hai ẩn
-------------[3] Các bài toán ứng dụng thực tế
-------[3] Hệ thức lượng trong tam giác
----------[5] Giá trị lượng giác của một góc từ 0 độ đến 180 độ
-------------[1] Xét dấu của các giá trị lượng giác
-------------[2] Tính các giá trị lượng giác
-------------[3] Chứng minh, rút gọn các biểu thức lượng giác
----------[6] Hệ thức lượng trong tam giác
-------------[1] Tính toán các đại lượng trong tam giác
-------------[2] Chứng minh các hệ thức
-------------[3] Nhận dạng tam giác
-------------[4] Giải tam giác và các ứng dụng thực tế
-------[4] Vectơ
----------[7] Các khái niệm mở đầu
-------------[1] Xác định một véc-tơ
-------------[2] Sự cùng phương và hướng của hai véc-tơ
-------------[3] Hai véc-tơ bằng nhau, độ dài của véc-tơ
----------[8] Tổng và hiệu của hai véc-tơ
-------------[1] Tổng của hai véc-tơ, tổng của nhiều véc-tơ
-------------[2] Chứng minh đẳng thức véc-tơ
-------------[3] Xác định vị trí của một điểm nhờ đẳng thức véc-tơ
-------------[4] Tìm véc-tơ đối, hiệu của hai véc-tơ
-------------[5] Tính độ dài tổng và hiệu các véc-tơ
----------[9] Tích của một vectơ với một số
-------------[1] Xác định véc-tơ k\vec{a}, tính độ dài véc-tơ
-------------[2] Chứng minh các đẳng thức véc-tơ, thu gọn biểu thức
-------------[3] Xác định vị trí của một điểm nhờ đẳng thức véc-tơ
-------------[4] Phân tích một véc-tơ theo hai véc-tơ không cùng phương
-------------[5] Chứng minh ba điểm thẳng hàng, hai đường thẳng song song, hai điểm trùng nhau
-------------[6] Tập hợp điểm
-------------[7] Cực trị
----------[0] Vectơ trong mặt phẳng toạ độ
-------------[1] Tìm tọa độ của một điểm và độ dài đại số của một véc-tơ trên trục
-------------[2] Tìm tọa độ các véc-tơ tổng, hiệu và k\vec{a}
-------------[3] Xác định tọa độ của véc-tơ và của một điểm trên mặt phẳng tọa độ Oxy
-------------[4] Phân tích một véc-tơ theo hai véc-tơ không cùng phương
-------------[5] Chứng minh ba điểm thẳng hàng, hai đường thẳng song song
----------[A] Tích vô hướng của hai vectơ
-------------[1] Tính tích vô hướng của hai véc-tơ và xác định góc
-------------[2] Chứng minh đẳng thức về tích vô hướng hoặc độ dài
-------------[3] Điều kiện vuông góc
-------------[4] Các bài toán tìm điểm và tập hợp điểm
-------------[5] Cực trị và chứng minh bất đẳng thức
-------------[6] Xác định góc giữa hai véc-tơ, góc giữa hai đường thẳng
-------[5] Các số đặc trưng của mẫu số liệu không ghép nhóm
----------[B] Số gần đúng. Sai số
-------------[1] Tính và ước lượng sai số tuyệt đối
-------------[2] Tính và xác định độ chính xác của kết quả
-------------[3] Quy tròn số gần đúng
----------[C] Các số đặc trưng đo xu thế trung tâm
-------------[1] Số trung bình cộng
-------------[2] Số trung vị
-------------[3] Tứ phân vị
-------------[4] Mốt
-------------[5] Câu hỏi lý thuyết
----------[D] Các số đặc trưng đo mức độ phân tán
-------------[1] Khoảng biến thiên, khoảng tứ phân vị
-------------[2] Tính phương sai, độ lệch chuẩn dựa vào bảng số liệu cho trước
-------------[3] Câu hỏi lý thuyết
-------[6] Hàm số, đồ thị và ứng dụng
----------[E] Hàm số
-------------[1] Tính giá trị của hàm số
-------------[2] Tìm tập xác định của hàm số
-------------[3] Đồ thị của hàm số
-------------[4] Tính đồng biến, nghịch biến của hàm số
----------[F] Hàm số bậc hai
-------------[1] Bảng biến thiên, tính đơn điệu, GTLN - GTNN
-------------[2] Xác định hàm số bậc hai
-------------[3] Đồ thị của hàm số bậc hai
-------------[4] Bài toán tương giao
-------------[5] Toán thực tế ứng dụng hàm số bậc hai
-------------[6] Hàm số chứa dấu giá trị tuyệt đối
----------[G] Dấu của tam thức bậc hai
-------------[1] Nhận dạng tam thức và xét dấu biểu thức
-------------[2] Giải và các bài toán liên quan bất phương trình bậc hai
-------------[3] Giải và các bài toán liên quan bất phương trình tích, thương
-------------[4] Giải và các bài toán liên quan hệ bất phương bậc hai
----------[H] Phương trình quy về phương trình bậc hai
-------------[1] Phương trình và bất phương trình chứa dấu giá trị tuyệt đối
-------------[2] Phương trình và bất phương trình chứa căn thức
-------------[3] Phương trình và bất phương trình chứa dấu giá trị tuyệt đối có tham số
-------------[4] Phương trình và bất phương trình chứa căn thức có tham số
-------[7] Phương pháp tọa độ trong mặt phẳng
----------[I] Phương trình đường thẳng
-------------[1] Xác định các yếu tố của đường thẳng
-------------[2] Viết phương trình đường thẳng
----------[J] Vị trí tương đối giữa hai đường thẳng. Góc và khoảng cách
-------------[1] Vị trí tương đối giữa hai đường thẳng
-------------[2] Bài toán liên quan góc giữa hai đường thẳng
-------------[3] Bài toán liên quan công thức khoảng cách
-------------[4] Bài toán liên quan đến tìm điểm
-------------[5] Bài toán thực tế
----------[K] Đường tròn trong mặt phẳng toạ độ
-------------[1] Xác định tâm, bán kính và điều kiện là đường tròn
-------------[2] Viết phương trình đường tròn
-------------[3] Viết phương trình đường tiếp tuyến của đường tròn
-------------[4] Vị trí tương đối của đường tròn và đường thẳng, hai đường tròn
-------------[5] Các dạng toán tổng hợp đường thẳng và đường tròn
-------------[6] Bài toán thực tế
----------[L] Ba đường conic trong mặt phẳng toạ độ
-------------[1] Xác định các yếu tố của elip
-------------[2] Viết phương trình chính tắc của elip
-------------[3] Bài toán tìm điểm trên elip
-------------[4] Xác định các yếu tố của hypebol
-------------[5] Viết phương trình chính tắc của hypebol
-------------[6] Bài toán tìm điểm trên hypebol
-------------[7] Xác định các yếu tố của parabol
-------------[8] Viết phương trình chính tắc của parabol
-------------[9] Bài toán tìm điểm trên parabol
-------------[0] Bài toán thực tế
-------[8] Đại số tổ hợp
----------[M] Quy tắc đếm
-------------[1] Bài toán sử dụng quy tắc cộng
-------------[2] Bài toán sử dụng quy tắc nhân
-------------[3] Bài toán kết hợp quy tắc cộng và quy tắc nhân
----------[N] Hoán vị, chỉnh hợp và tổ hợp
-------------[1] Bài toán chỉ sử dụng P hoặc C hoặc A
-------------[2] Bài toán kết hợp P, C và A
-------------[3] Bài toán liên quan đến hình học
-------------[4] Hoán vị bàn tròn
-------------[5] Hoán vị lặp
-------------[6] Giải PT, BPT, HPT, chứng minh liên quan đến P,C,A
----------[O] Nhị thức Newton
-------------[1] Khai triển một nhị thức Newton
-------------[2] Tìm hệ số, số hạng trong khai triển nhị thức Newton
-------------[3] Chứng minh, tính giá trị của biểu thức đại số tổ hợp có sử dụng nhị thức Newton
-------[9] Tính xác suất theo định nghĩa cổ điển
----------[P] Biến cố và định nghĩa cổ điển của xác suất
-------------[1] Mô tả không gian mẫu, biến cố
-------------[2] Các câu hỏi lý thuyết tổng hợp
-------------[3] Tính xác suất bằng định nghĩa
----------[Q] Thực hành tính xác suất theo định nghĩa cổ điển
-------------[1] Các câu hỏi lý thuyết tổng hợp
-------------[3] Tính xác suất bằng công thức cộng
-------------[4] Tính xác suất bằng công thức nhân
-------------[5] Bài toán kết hợp quy tắc cộng và quy tắc nhân xác suất
----[D] Đại số
-------[1] Mệnh đề. Tập hợp
----------[1] Mệnh đề
-------------[1] Nhận biết mệnh đề, mệnh đề chứa biến
-------------[2] Xét tính đúng - sai của mệnh đề
-------------[3] Phủ định của một mệnh đề
-------------[4] Mệnh đề kéo theo, mệnh đề đảo, hai mệnh đề tương đương
-------------[5] Mệnh đề với ký hiệu mọi và tồn tại
----------[2] Tập hợp
-------------[1] Tập hợp và phần tử của tập hợp
-------------[2] Tập hợp con - Hai tập hợp bằng nhau
----------[3] Các phép toán tập hợp
-------------[1] Giao và hợp của hai tập hợp
-------------[2] Hiệu và phần bù của hai tập hợp
-------------[3] Toán thực tế ứng dụng của tập hợp
----------[4] Các tập hợp số
-------------[1] Xác định giao, hợp của các khoảng, đoạn, nửa khoảng
-------------[2] Xác định hiệu và phần bù của các khoảng, đoạn, nửa khoảng
----------[5] Số gần đúng. Sai số
-------------[1] Tính và ước lượng sai số tuyệt đối
-------------[2] Tính và xác định độ chính xác của kết quả
-------[2] Hàm số bậc nhất và bậc hai
----------[1] Hàm số
-------------[1] Tính giá trị của hàm số
-------------[2] Tìm tập xác định của hàm số
-------------[3] Tính đồng biến, nghịch biến của hàm số
-------------[4] Tính chẵn, lẻ của hàm số
----------[2] Hàm số bậc nhất
-------------[1] Tính đồng biến, nghịch biến của hàm số
-------------[2] Xác định hàm số bậc nhất
-------------[3] Đồ thị của hàm số bậc nhất
-------------[4] Bài toán tương giao
-------------[5] Toán thực tế ứng dụng hàm số bậc nhất
----------[3] Hàm số bậc hai
-------------[1] Bảng biến thiên, tính đơn điệu, GTLN - GTNN
-------------[2] Xác định hàm số bậc hai
-------------[3] Đồ thị của hàm số bậc hai
-------------[4] Bài toán tương giao
-------------[5] Toán thực tế ứng dụng hàm số bậc hai
-------------[6] Hàm số chứa dấu giá trị tuyệt đối
-------[3] Phương trình - Hệ phương trình
----------[1] Đại cương về phương trình
-------------[1] Tìm điều kiện của phương trình
-------------[2] Nghiệm của phương trình
-------------[3] Giải phương trình bằng cách biến đổi tương đương hoặc hệ quả
----------[2] Phương trình quy về phương trình bậc nhất, bậc hai
-------------[1] Phương trình tích
-------------[2] Phương trình chứa ẩn trong dấu giá trị tuyệt đối
-------------[3] Phương trình chứa ẩn ở mẫu
-------------[4] Phương trình chứa ẩn dưới dấu căn
-------------[5] Định lí Vi-ét và ứng dụng
-------------[6] Giải và biện luận phương trình
-------------[7] Phương trình bậc cao và các bài toán liên quan
-------------[8] Phương trình hàm ẩn
----------[3] Phương trình và hệ phương trình bậc nhất nhiều ẩn
-------------[1] Giải và biện luận phương trình bậc nhất hai ẩn
-------------[2] Giải và biện luận hệ phương trình bậc nhất hai ẩn
-------------[3] Giải hệ phương trình bậc nhất hai ẩn, ba ẩn
-------------[4] Giải hệ phương trình bậc cao
-------------[5] Toán thực tế giải phương trình, hệ phương trình
-------[4] Bất đẳng thức - Bất phương trình
----------[1] Bất đẳng thức
-------------[1] Chứng minh BĐT dựa vào định nghĩa và tính chất
-------------[2] Chứng minh BĐT dựa vào BĐT Cauchy
-------------[3] Chứng minh BĐT dựa vào BĐT Bunhiacopxki
-------------[4] Bất đẳng thức về giá trị tuyệt đối
-------------[5] Ứng dụng BĐT để giải PT, HPT, BPT, tìm GTLN-GTNN
----------[2] Bất phương trình và hệ bất phương trình một ẩn
-------------[1] Tìm điều kiện xác định của bất phương trình - hệ phương trình
-------------[2] Bất phương trình - hệ bất phương trình tương đương
-------------[3] Giải bất phương trình bậc nhất một ẩn và biểu diễn tập nghiệm
-------------[4] Giải hệ bất phương trình bậc nhất một ẩn và biểu diễn tập nghiệm
-------------[5] Bất phương trình - hệ bất phương trình bậc nhất một ẩn chứa tham số
-------------[6] Toán thực tế giải bất phương trình, hệ bất phương trình
----------[3] Dấu của nhị thức bậc nhất
-------------[1] Nhận dạng nhị thức và xét dấu biểu thức
-------------[2] Bất phương trình tích
-------------[3] Bất phương có ẩn ở mẫu
-------------[4] Dấu nhị thức bậc nhất trên một miền
-------------[5] Giải PT, BPT chứa dấu giá trị tuyệt đối
----------[4] Bất phương trình bậc nhất hai ẩn
-------------[1] Bất phương trình bậc nhất hai ẩn và các bài toán liên quan
-------------[2] Hệ bất phương trình bậc nhất hai ẩn và các bài toán liên quan
-------------[3] Các bài toán ứng dụng thực tế
-------------[4] Miền nghiệm của hệ bất phương trình bậc nhất hai ẩn
----------[5] Dấu của tam thức bậc hai
-------------[1] Nhận dạng tam thức và xét dấu biểu thức
-------------[2] Giải và các bài toán liên quan bất phương trình bậc hai
-------------[3] Giải và các bài toán liên quan bất phương trình tích, thương
-------------[4] Giải và các bài toán liên quan hệ bất phương bậc hai
-------------[5] Phương trình và bất phương trình chứa dấu giá trị tuyệt đối
-------------[6] Phương trình và bất phương trình chứa căn thức
-------------[7] Phương trình và bất phương trình chứa dấu giá trị tuyệt đối có tham số
-------------[8] Phương trình và bất phương trình chứa căn thức có tham số
-------[5] Thống kê
----------[1] Bảng phân bố tần số và tần suất
-------------[1] Bảng phân bố tần số và tần suất
-------------[2] Bảng phân bố tần số và tần suất ghép lớp
-------------[3] Câu hỏi lý thuyết
----------[2] Biểu đồ
-------------[1] Biểu đồ tần số và tần suất hình cột
-------------[2] Biểu đồ đường gấp khúc
-------------[3] Biểu đồ hình quạt
-------------[4] Câu hỏi lý thuyết
----------[3] Số trung bình cộng. Số trung vị. Mốt
-------------[1] Số trung bình cộng
-------------[2] Số trung vị
-------------[3] Mốt
-------------[4] Câu hỏi lý thuyết
----------[4] Phương sai và độ lệch chuẩn
-------------[1] Tính phương sai, độ lệch chuẩn dựa vào bảng số liệu cho trước
-------------[2] Câu hỏi lý thuyết
-------[6] Cung và góc lượng giác. Công thức lượng giác
----------[1] Cung và góc lượng giác
-------------[1] Mối liên hệ giữa độ và radian
-------------[2] Độ dài của một cung tròn
-------------[3] Biểu diễn cung lên đường tròn lượng giác
-------------[4] Các bài toán thực tế, liên môn
-------------[5] Câu hỏi lý thuyết
----------[2] Giá trị lượng giác của một cung
-------------[1] Xét dấu của các giá trị lượng giác
-------------[2] Tính giá trị lượng giác của một cung
-------------[3] Giá trị lượng giác của các cung có liên quan đặc biệt
-------------[4] Tìm giá trị lớn nhất, giá trị nhỏ nhất của biểu thức lượng giác
-------------[5] Rút gọn biểu thức lượng giác. Đẳng thức lượng giác
-------------[6] Các bài toán có yếu tố thực tế, liên môn
-------------[7] Câu hỏi lý thuyết
----------[3] Công thức lượng giác
-------------[1] Áp dụng công thức cộng
-------------[2] Áp dụng công thức nhân đôi - hạ bậc
-------------[3] Áp dụng công thức biến đổi tích thành tổng, tổng thành tích
-------------[4] Kết hợp các công thức lượng giác
-------------[5] Tìm giá trị lớn nhất, giá trị nhỏ nhất của biểu thức lượng giác
-------------[6] Nhận dạng tam giác
-------------[7] Các bài toán có yếu tố thực tế, liên môn
-------------[8] Câu hỏi lý thuyết
----[H] Hình học
-------[1] Véc-tơ
----------[1] Các định nghĩa
-------------[1] Xác định một véc-tơ
-------------[2] Sự cùng phương và hướng của hai véc-tơ
-------------[3] Hai véc-tơ bằng nhau, độ dài của véc-tơ
----------[2] Tổng và hiệu của hai véc-tơ
-------------[1] Tổng của hai véc-tơ, tổng của nhiều véc-tơ
-------------[2] Chứng minh đẳng thức véc-tơ
-------------[3] Xác định vị trí của một điểm nhờ đẳng thức véc-tơ
-------------[4] Tìm véc-tơ đối, hiệu của hai véc-tơ
-------------[5] Tính độ dài tổng và hiệu các véc-tơ
----------[3] Tích của véc-tơ với một số
-------------[1] Xác định véc-tơ k\vec{a}, tính độ dài véc-tơ
-------------[2] Chứng minh các đẳng thức véc-tơ, thu gọn biểu thức
-------------[3] Xác định vị trí của một điểm nhờ đẳng thức véc-tơ
-------------[4] Phân tích một véc-tơ theo hai véc-tơ không cùng phương
-------------[5] Chứng minh ba điểm thẳng hàng, hai đường thẳng song song, hai điểm trùng nhau
-------------[6] Tập hợp điểm
-------------[7] Cực trị
----------[4] Hệ trục toạ độ
-------------[1] Tìm tọa độ của một điểm và độ dài đại số của một véc-tơ trên trục
-------------[2] Tìm tọa độ các véc-tơ tổng, hiệu và k\vec{a}
-------------[3] Xác định tọa độ của véc-tơ và của một điểm trên mặt phẳng tọa độ Oxy
-------------[4] Phân tích một véc-tơ theo hai véc-tơ không cùng phương
-------------[5] Chứng minh ba điểm thẳng hàng, hai đường thẳng song song
-------[2] Tích vô hướng của hai véc-tơ và ứng dụng
----------[1] Giá trị lượng giác của một góc bất kì từ 0 độ đến 180 độ
-------------[1] Xét dấu của các giá trị lượng giác
-------------[2] Tính các giá trị lượng giác
-------------[3] Chứng minh, rút gọn các biểu thức lượng giác
-------------[4] Xác định góc giữa hai véc-tơ, góc giữa hai đường thẳng
----------[2] Tích vô hướng
-------------[1] Tính tích vô hướng của hai véc-tơ và xác định góc
-------------[2] Chứng minh đẳng thức về tích vô hướng hoặc độ dài
-------------[3] Điều kiện vuông góc
-------------[4] Các bài toán tìm điểm và tập hợp điểm
-------------[5] Cực trị và chứng minh bất đẳng thức
----------[3] Các hệ thức lượng trong tam giác
-------------[1] Tính toán các đại lượng trong tam giác
-------------[2] Chứng minh các hệ thức
-------------[3] Nhận dạng tam giác
-------------[4] Giải tam giác và các ứng dụng thực tế
-------[3] Phương pháp tọa độ trong mặt phẳng
----------[1] Phương trình đường thẳng
-------------[1] Xác định các yếu tố của đường thẳng
-------------[2] Viết phương trình đường thẳng
-------------[3] Vị trí tương đối giữa hai đường thẳng
-------------[4] Bài toán liên quan góc giữa hai đường thẳng
-------------[5] Bài toán liên quan công thức khoảng cách
-------------[6] Bài toán liên quan đến tìm điểm
-------------[7] Bài toán thực tế
----------[2] Phương trình đường tròn
-------------[1] Xác định tâm, bán kính và điều kiện là đường tròn
-------------[2] Viết phương trình đường tròn
-------------[3] Viết phương trình đường tiếp tuyến của đường tròn
-------------[4] Vị trí tương đối của đường tròn và đường thẳng, hai đường tròn
-------------[5] Các dạng toán tổng hợp đường thẳng và đường tròn
-------------[6] Bài toán thực tế
----------[3] Phương trình đường elip
-------------[1] Xác định các yếu tố của elip
-------------[2] Viết phương trình chính tắc của elip
-------------[3] Bài toán tìm điểm trên elip
-------------[4] Bài toán thực tế
-[1] Lớp 11
----[C] Cánh Diều
-------[1] Hàm số lượng giác và phương trình lượng giác
----------[1] Góc lượng giác. Giá trị lượng giác của góc lượng giác
-------------[1] Chuyển đổi đơn vị độ và radian
-------------[2] Số đo của một góc lượng giác
-------------[3] Biểu diễn góc lượng giác lên đường tròn lượng giác
-------------[4] Xét dấu các giá trị lượng giác
-------------[5] Tính giá trị lượng giác của một góc
-------------[6] Giá trị lượng giác của các góc có liên quan đặc biệt
-------------[7] Biến đổi, thu gọn biểu thức lượng giác
-------------[8] Các bài toán có yếu tố thực tế, liên môn
-------------[9] Câu hỏi lý thuyết
-------------[0] [Giảm] Độ dài của một cung tròn
----------[2] Các phép biến đổi lượng giác
-------------[1] Áp dụng công thức cộng
-------------[2] Áp dụng công thức nhân đôi - hạ bậc
-------------[3] Áp dụng công thức biến đổi tích <-> tổng
-------------[4] Kết hợp nhiều công thức lượng giác
-------------[5] Nhận dạng tam giác
-------------[6] Các bài toán có yếu tố thực tế, liên môn
-------------[7] Câu hỏi lý thuyết
----------[3] Hàm số lượng giác và đồ thị
-------------[1] Tìm tập xác định
-------------[2] Xét tính đơn điệu
-------------[3] Xét tính chẵn, lẻ
-------------[4] Xét tính tuần hoàn, tìm chu kỳ
-------------[5] Tìm tập giá trị và min-max
-------------[6] Bảng biến thiên và đồ thị
----------[4] Phương trình lượng giác cơ bản
-------------[1] Phương trình tương đương
-------------[2] Điều kiện có nghiệm
-------------[3] Phương trình cơ bản dùng Radian
-------------[4] Phương trình cơ bản dùng Độ
-------------[5] Phương trình đưa về dạng cơ bản
-------------[6] Toán thực tế, liên môn
----------[5] [Giảm] Phương trình lượng giác thường gặp
-------------[1] Phương trình bậc n theo một hàm số lượng giác
-------------[2] Phương trình đẳng cấp bậc n đối với sinx và cosx
-------------[3] Phương trình bậc nhất đối với sinx và cosx
-------------[4] Phương trình đối xứng, phản đối xứng
-------------[5] Phương trình lượng giác không mẫu mực
-------------[6] Phương trình lượng giác có chứa ẩn ở mẫu số
-------------[7] Phương trình lượng giác có chứa tham số
-------------[8] Bài toán thực tế
-------[2] Dãy số. Cấp số cộng. Cấp số nhân
----------[1] Dãy số
-------------[1] Số hạng tổng quát, biểu diễn dãy số
-------------[2] Tìm số hạng cụ thể của dãy số
-------------[3] Dãy số tăng, dãy số giảm
-------------[4] Dãy số bị chặn
-------------[5] Toán thực tế về dãy số
-------------[6] Câu hỏi lý thuyết
----------[2] Cấp số cộng
-------------[1] Nhận diện cấp số cộng, công sai d
-------------[2] Số hạng tổng quát của cấp số cộng
-------------[3] Tìm số hạng cụ thể trong cấp số cộng
-------------[4] Điều kiện để dãy số là cấp số cộng
-------------[5] Tính tổng của cấp số cộng
-------------[6] Các bài toán thực tế
----------[3] Cấp số nhân
-------------[1] Nhận diện cấp số nhân, công bội q
-------------[2] Số hạng tổng quát của cấp số nhân
-------------[3] Tìm số hạng cụ thể trong cấp số nhân
-------------[4] Điều kiện để dãy số là cấp số nhân
-------------[5] Tính tổng của cấp số nhân
-------------[6] Kết hợp cấp số nhân và cấp số cộng
-------------[7] Các bài toán thực tế
-------[3] Giới hạn. Hàm số liên tục
----------[1] Giới hạn của dãy số
-------------[1] Câu hỏi lý thuyết
-------------[2] Phương pháp đặt thừa số chung (lim hữu hạn)
-------------[3] Phương pháp lượng liên hợp (lim hữu hạn)
-------------[4] Giới hạn vô cực
-------------[5] Cấp số nhân lùi vô hạn
-------------[6] Toán thực tế, liên môn liên quan đến giới hạn dãy số
-------------[7] [Giảm] Nguyên lí kẹp
----------[2] Giới hạn của hàm số
-------------[1] Câu hỏi lý thuyết
-------------[2] Thay số trực tiếp
-------------[3] PP đặt thừa số chung, kết quả hữu hạn
-------------[4] PP đặt thừa số chung, kết quả vô cực
-------------[5] PP lượng liên hợp, kết quả hữu hạn
-------------[6] PP lượng liên hợp, kết quả vô cực
-------------[7] Giới hạn một bên
-------------[8] Toán thực tế, liên môn về giới hạn hàm số
----------[3] Hàm số liên tục
-------------[1] Câu hỏi lý thuyết
-------------[2] Tính liên tục thể hiện qua đồ thị
-------------[3] Hàm số liên tục tại một điểm
-------------[4] Hàm số liên tục trên khoảng, đoạn
-------------[5] Bài toán chứa tham số
-------------[6] Toán thực tế, liên môn về hàm số liên tục
-------------[7] [Giảm] Bài toán phương trình có nghiệm
-------[4] Đường thẳng, mặt phẳng trong không gian. Quan hệ song song
----------[1] Đường thẳng và mặt phẳng trong không gian
-------------[1] Câu hỏi lý thuyết
-------------[2] Hình biểu diễn của một hình không gian
-------------[3] Tìm giao tuyến của hai mặt phẳng
-------------[4] Tìm giao điểm của đường thẳng và mặt phẳng
-------------[5] Xác định thiết diện
-------------[6] Ba điểm thẳng hàng, ba đường thẳng đồng quy
-------------[7] Bài toán quỹ tích và điểm cố định
-------------[8] Bài toán thực tế
----------[2] Hai đường thẳng song song trong không gian
-------------[1] Câu hỏi lý thuyết
-------------[2] Hai đường thẳng song song
-------------[3] Tìm giao tuyến bằng cách kẻ song song
-------------[4] Tìm giao điểm của đường thẳng và mặt phẳng
-------------[5] Xác định thiết diện bằng cách kẻ song song
-------------[6] Ba điểm thẳng hàng
-------------[7] Bài toán quỹ tích và điểm cố định
-------------[8] Bài toán thực tế
----------[3] Đường thẳng và mặt phẳng song song
-------------[1] Câu hỏi lý thuyết
-------------[2] Đường thẳng song song với mặt phẳng
-------------[3] Tìm giao tuyến bằng cách kẻ song song
-------------[4] Tìm giao điểm của đường thẳng và mặt phẳng
-------------[5] Xác định thiết diện bằng cách kẻ song song
-------------[6] Ba điểm thẳng hàng
-------------[7] Bài toán quỹ tích và điểm cố định
-------------[8] Bài toán thực tế
----------[4] Hai mặt phẳng song song
-------------[1] Câu hỏi lý thuyết
-------------[2] Hai mặt phẳng song song
-------------[3] Tìm giao tuyến bằng cách kẻ song song
-------------[4] Tìm giao điểm của đường thẳng và mặt phẳng
-------------[5] Xác định thiết diện bằng cách kẻ song song
-------------[6] Bài toán tổng hợp
-------------[7] Bài toán thực tế
----------[5] Hình lăng trụ và hình hộp
-------------[1] Bài toán về hình lăng trụ
-------------[2] Bài toán về hình hộp
----------[6] Phép chiếu song song. Hình biểu diễn của một hình không gian
-------------[1] Câu hỏi lý thuyết
-------------[2] Hình biểu diễn của một hình không gian
-------------[3] Xác định yế tố song song
-------[5] Một số yếu tố thống kê và xác suất
----------[1] Các số đặc trưng đo xu thế trung tâm cho mẫu số liệu ghép nhóm
-------------[1] Bảng dữ liệu ghép nhóm
-------------[2] Số trung bình
-------------[3] Trung vị
-------------[4] Tứ phân vị
-------------[5] Mốt
-------------[6] Câu hỏi lý thuyết
----------[2] Biến cố hợp và biến cố giao. Biến cố độc lập. Các quy tắc tính xác suất
-------------[1] Câu hỏi lý thuyết
-------------[2] Xác định các loại biến cố
-------------[3] Tính xác suất bằng định nghĩa
-------------[4] Tính xác suất bằng quy tắc cộng
-------------[5] Sơ đồ hình cây
-------------[6] Tính xác suất bằng quy tắc nhân
-------------[7] Tính xác suất bằng cách kết hợp quy tắc
-------[6] Hàm số mũ và hàm số lôgarít
----------[1] Phép tính luỹ thừa với số mũ thực
-------------[1] Tính giá trị của biểu thức chứa lũy thừa
-------------[2] Biến đổi, rút gọn biểu thức chứa lũy thừa
-------------[3] So sánh các lũy thừa
-------------[4] Điều kiện cho luỹ thừa, căn thức
----------[2] Phép tính lôgarít
-------------[1] Tính giá trị biểu thức chứa lôgarít
-------------[2] Biến đổi, rút gọn, biểu diễn biểu thức chứa lôgarít
-------------[3] Toán thực tế, liên môn
----------[3] Hàm số mũ. Hàm số lôgarít
-------------[1] Tập xác định của hàm số
-------------[2] Sự biến thiên và đồ thị của hàm số mũ, lôgarít
-------------[3] So sánh các luỹ thừa và lôgarít
-------------[4] Bài toán thực tế, liên môn
-------------[5] Lý thuyết tổng hợp hàm số lũy thừa, mũ, lôgarít
----------[4] Phương trình, bất phương trình mũ và lôgarít
-------------[1] Điều kiện có nghiệm
-------------[2] Phương trình mũ, lôgarít cơ bản
-------------[3] Bất phương trình mũ, lôgarít cơ bản
-------------[4] Phương trình mũ, lôgarít đưa về cùng cơ số
-------------[5] Bất phương trình mũ, lôgarít đưa về cùng cơ số
-------------[6] Bài toán thực tế, liên môn
----------[5] [Giảm] Các phương pháp giải được giảm tải
-------------[1] Phương pháp đặt ẩn phụ cho PT mũ, lôgarít
-------------[2] Phương pháp lôgarít hóa, mũ cho PT mũ, lôgarít
-------------[3] Phương pháp hàm số, đánh giá cho PT mũ, lôgarít
-------------[4] Hệ PT mũ, lôgarít
-------------[5] Phương pháp đặt ẩn phụ với BPT mũ, lôgarít
-------------[6] Phương pháp lôgarít hóa, mũ cho BPT mũ, lôgarít
-------------[7] Phương pháp hàm số, đánh giá cho BPT mũ, lôgarít
-------------[8] Hệ BPT mũ, lôgarít
-------[7] Đạo hàm
----------[1] Định nghĩa đạo hàm. Ý nghĩa hình học của đạo hàm
-------------[1] Tính đạo hàm bằng định nghĩa
-------------[2] Số gia hàm số, số gia biến số
-------------[3] Ý nghĩa hình học của đạo hàm
-------------[4] Ý nghĩa Vật lý của đạo hàm
----------[2] Các quy tắc đạo hàm
-------------[1] Tính đạo hàm
-------------[2] Đẳng thức có y và y'
-------------[3] Tiếp tuyến tại một điểm
-------------[4] Tiếp tuyến biết trước hệ số góc
-------------[5] Tiếp tuyến chưa biết tiếp điểm và hệ số góc
-------------[6] Bài toán thực tế, liên môn
-------------[7] Giới hạn hàm số lượng giác, hàm số mũ, lôgarít
-------------[8] Dùng đạo hàm cho nhị thức Newton
----------[3] Đạo hàm cấp hai
-------------[1] Tính đạo hàm cấp hai
-------------[2] Đẳng thức có y và (y', y'')
-------------[3] Ý nghĩa Vật lý của đạo hàm cấp hai
-------[8] Quan hệ vuông góc trong không gian. Phép chiếu vuông góc
----------[1] Hai đường thẳng vuông góc
-------------[1] Câu hỏi lí thuyết
-------------[2] Xác định góc giữa hai đường thẳng bằng định nghĩa
-------------[3] Xác định hai đường thẳng vuông góc
-------------[4] Các bài toán thực tế
----------[2] Đường thẳng vuông góc với mặt phẳng
-------------[1] Câu hỏi lí thuyết
-------------[2] Xác định đường thẳng và mặt phẳng vuông góc
-------------[3] Xác định hai đường thẳng vuông góc
-------------[4] Góc giữa hai đường thẳng (có d vuông (P))
-------------[5] Phép chiếu vuông góc
-------------[6] Dựng mặt phẳng, tìm thiết diện
-------------[7] Các bài toán thực tế
----------[3] Góc giữa đường thẳng và mặt phẳng. Góc nhị diện
-------------[1] Xác định góc giữa đường thẳng và mặt phẳng
-------------[2] Xác định góc phẳng nhị diện
-------------[3] Các bài toán thực tế
----------[4] Hai mặt phẳng vuông góc
-------------[1] Câu hỏi lí thuyết
-------------[2] Xác định quan hệ vuông góc giữa ĐT và MP, MP và MP
-------------[3] Xác định góc giữa hai mặt phẳng
-------------[4] Dựng mặt phẳng vuông góc với mặt phẳng cho trước. Thiết diện
-------------[5] Hình chiếu vuông góc của đa giác trên mặt phẳng
-------------[6] Các bài toán thực tế
----------[5] Khoảng cách trong không gian
-------------[1] Câu hỏi lí thuyết
-------------[2] Khoảng cách giữa 2 điểm, từ một điểm đến một đường thẳng
-------------[3] Khoảng cách từ một điểm đến một mặt phẳng
-------------[4] Khoảng cách giữa hai đường thẳng chéo nhau
-------------[5] Đường vuông góc chung của hai đường thẳng chéo nhau
-------------[6] Các bài toán thực tế
-------------[7] Bài toán cực trị
----------[6] Hình lăng trụ đứng. Hình chóp đều. Thể tích của một số hình khối
-------------[1] Tính góc, cạnh, đường cao, diện tích các hình thông dụng
-------------[2] Thể tích khối chóp, lăng trụ
-------------[3] Thể tích khối chóp cụt và khối khác
-------------[4] Bài toán vận dụng khái niệm thể tích
-------------[5] Các bài toán thực tế
-------------[6] Bài toán cực trị
----[T] Chân trời sáng tạo
-------[1] Hàm số lượng giác và phương trình lượng giác
----------[1] Góc lượng giác
-------------[1] Chuyển đổi đơn vị độ và radian
-------------[2] Số đo của một góc lượng giác
-------------[3] Biểu diễn góc lượng giác lên đường tròn lượng giác
-------------[4] Các bài toán thực tế, liên môn
-------------[5] Câu hỏi lý thuyết
-------------[6] [Giảm] Độ dài của một cung tròn
----------[2] Giá trị lượng giác của một góc lượng giác
-------------[1] Xét dấu các giá trị lượng giác
-------------[2] Tính giá trị lượng giác của một góc
-------------[3] Giá trị lượng giác của các góc có liên quan đặc biệt
-------------[4] Biến đổi, thu gọn biểu thức lượng giác
-------------[5] Các bài toán có yếu tố thực tế, liên môn
-------------[6] Câu hỏi lý thuyết
----------[3] Các công thức lượng giác
-------------[1] Áp dụng công thức cộng
-------------[2] Áp dụng công thức nhân đôi - hạ bậc
-------------[3] Áp dụng công thức biến đổi tích <-> tổng
-------------[4] Kết hợp nhiều công thức lượng giác
-------------[5] Nhận dạng tam giác
-------------[6] Các bài toán có yếu tố thực tế, liên môn
-------------[7] Câu hỏi lý thuyết
----------[4] Hàm số lượng giác và đồ thị
-------------[1] Tìm tập xác định
-------------[2] Xét tính đơn điệu
-------------[3] Xét tính chẵn, lẻ
-------------[4] Xét tính tuần hoàn, tìm chu kỳ
-------------[5] Tìm tập giá trị và min-max
-------------[6] Bảng biến thiên và đồ thị
----------[5] Phương trình lượng giác cơ bản
-------------[1] Phương trình tương đương
-------------[2] Điều kiện có nghiệm
-------------[3] Phương trình cơ bản dùng Radian
-------------[4] Phương trình cơ bản dùng Độ
-------------[5] Phương trình đưa về dạng cơ bản
-------------[6] Toán thực tế, liên môn
----------[6] [Giảm] Phương trình lượng giác thường gặp
-------------[1] Phương trình bậc n theo một hàm số lượng giác
-------------[2] Phương trình đẳng cấp bậc n đối với sinx và cosx
-------------[3] Phương trình bậc nhất đối với sinx và cosx
-------------[4] Phương trình đối xứng, phản đối xứng
-------------[5] Phương trình lượng giác không mẫu mực
-------------[6] Phương trình lượng giác có chứa ẩn ở mẫu số
-------------[7] Phương trình lượng giác có chứa tham số
-------------[8] Bài toán thực tế
-------[2] Dãy số. Cấp số cộng. Cấp số nhân
----------[1] Dãy số
-------------[1] Số hạng tổng quát, biểu diễn dãy số
-------------[2] Tìm số hạng cụ thể của dãy số
-------------[3] Dãy số tăng, dãy số giảm
-------------[4] Dãy số bị chặn
-------------[5] Toán thực tế về dãy số
-------------[6] Câu hỏi lý thuyết
----------[2] Cấp số cộng
-------------[1] Nhận diện cấp số cộng, công sai d
-------------[2] Số hạng tổng quát của cấp số cộng
-------------[3] Tìm số hạng cụ thể trong cấp số cộng
-------------[4] Điều kiện để dãy số là cấp số cộng
-------------[5] Tính tổng của cấp số cộng
-------------[6] Các bài toán thực tế
----------[3] Cấp số nhân
-------------[1] Nhận diện cấp số nhân, công bội q
-------------[2] Số hạng tổng quát của cấp số nhân
-------------[3] Tìm số hạng cụ thể trong cấp số nhân
-------------[4] Điều kiện để dãy số là cấp số nhân
-------------[5] Tính tổng của cấp số nhân
-------------[6] Kết hợp cấp số nhân và cấp số cộng
-------------[7] Các bài toán thực tế
-------[3] Giới hạn. Hàm số liên tục
----------[1] Giới hạn của dãy số
-------------[1] Câu hỏi lý thuyết
-------------[2] Phương pháp đặt thừa số chung (lim hữu hạn)
-------------[3] Phương pháp lượng liên hợp (lim hữu hạn)
-------------[4] Giới hạn vô cực
-------------[5] Cấp số nhân lùi vô hạn
-------------[6] Toán thực tế, liên môn liên quan đến giới hạn dãy số
-------------[7] [Giảm] Nguyên lí kẹp
----------[2] Giới hạn của hàm số
-------------[1] Câu hỏi lý thuyết
-------------[2] Thay số trực tiếp
-------------[3] PP đặt thừa số chung, kết quả hữu hạn
-------------[4] PP đặt thừa số chung, kết quả vô cực
-------------[5] PP lượng liên hợp, kết quả hữu hạn
-------------[6] PP lượng liên hợp, kết quả vô cực
-------------[7] Giới hạn một bên
-------------[8] Toán thực tế, liên môn về giới hạn hàm số
----------[3] Hàm số liên tục
-------------[1] Câu hỏi lý thuyết
-------------[2] Tính liên tục thể hiện qua đồ thị
-------------[3] Hàm số liên tục tại một điểm
-------------[4] Hàm số liên tục trên khoảng, đoạn
-------------[5] Bài toán chứa tham số
-------------[6] Toán thực tế, liên môn về hàm số liên tục
-------------[7] Bài toán phương trình có nghiệm
-------[4] Đường thẳng, mặt phẳng. Quan hệ song song trong không gian
----------[1] Điểm, đường thẳng và mặt phẳng
-------------[1] Câu hỏi lý thuyết
-------------[2] Hình biểu diễn của một hình không gian
-------------[3] Tìm giao tuyến của hai mặt phẳng
-------------[4] Tìm giao điểm của đường thẳng và mặt phẳng
-------------[5] Xác định thiết diện
-------------[6] Ba điểm thẳng hàng, ba đường thẳng đồng quy
-------------[7] Bài toán quỹ tích và điểm cố định
-------------[8] Bài toán thực tế
----------[2] Hai đường thẳng song song
-------------[1] Câu hỏi lý thuyết
-------------[2] Hai đường thẳng song song
-------------[3] Tìm giao tuyến bằng cách kẻ song song
-------------[4] Tìm giao điểm của đường thẳng và mặt phẳng
-------------[5] Xác định thiết diện bằng cách kẻ song song
-------------[6] Ba điểm thẳng hàng
-------------[7] Bài toán quỹ tích và điểm cố định
-------------[8] Bài toán thực tế
----------[3] Đường thẳng và mặt phẳng song song
-------------[1] Câu hỏi lý thuyết
-------------[2] Đường thẳng song song với mặt phẳng
-------------[3] Tìm giao tuyến bằng cách kẻ song song
-------------[4] Tìm giao điểm của đường thẳng và mặt phẳng
-------------[5] Xác định thiết diện bằng cách kẻ song song
-------------[6] Ba điểm thẳng hàng
-------------[7] Bài toán quỹ tích và điểm cố định
-------------[8] Bài toán thực tế
----------[4] Hai mặt phẳng song song
-------------[1] Câu hỏi lý thuyết
-------------[2] Hai mặt phẳng song song
-------------[3] Tìm giao tuyến bằng cách kẻ song song
-------------[4] Tìm giao điểm của đường thẳng và mặt phẳng
-------------[5] Xác định thiết diện bằng cách kẻ song song
-------------[6] Bài toán tổng hợp
-------------[7] Bài toán thực tế
----------[5] Phép chiếu song song
-------------[1] Câu hỏi lý thuyết
-------------[2] Hình biểu diễn của một hình không gian
-------------[3] Xác định yế tố song song
-------[5] Các số đặc trưng đo xu thế trung tâm cho mẫu số liệu ghép nhóm
----------[1] Số trung bình và mốt của mẫu số liệu ghép nhóm
-------------[1] Bảng dữ liệu ghép nhóm
-------------[2] Số trung bình
-------------[3] Mốt
-------------[4] Câu hỏi lý thuyết
----------[2] Trung vị và tứ phân vị của mẫu số liệu ghép nhóm
-------------[1] Trung vị
-------------[2] Tứ phân vị
-------------[3] Câu hỏi lý thuyết
----------[3] [Lớp 12 mới] Xu thế phân tán cho mẫu số liệu ghép nhóm
-------------[1] Phương sai, độ lệch chuẩn
-------[6] Hàm số mũ và hàm số lôgarít
----------[1] Phép tính luỹ thừa
-------------[1] Tính giá trị của biểu thức chứa lũy thừa
-------------[2] Biến đổi, rút gọn biểu thức chứa lũy thừa
-------------[3] Điều kiện cho luỹ thừa, căn thức
----------[2] Phép tính lôgarít
-------------[1] Tính giá trị biểu thức chứa lôgarít
-------------[2] Biến đổi, rút gọn, biểu diễn biểu thức chứa lôgarít
-------------[3] Toán thực tế, liên môn
----------[3] Hàm số mũ. Hàm số lôgarít
-------------[1] Tập xác định của hàm số
-------------[2] Sự biến thiên và đồ thị của hàm số mũ, lôgarít
-------------[3] So sánh các luỹ thừa và lôgarít
-------------[4] Bài toán thực tế, liên môn
-------------[5] Lý thuyết tổng hợp hàm số lũy thừa, mũ, lôgarít
----------[4] Phương trình, bất phương trình mũ và lôgarít
-------------[1] Điều kiện có nghiệm
-------------[2] Phương trình mũ, lôgarít cơ bản
-------------[3] Bất phương trình mũ, lôgarít cơ bản
-------------[4] Phương trình mũ, lôgarít đưa về cùng cơ số
-------------[5] Bất phương trình mũ, lôgarít đưa về cùng cơ số
-------------[6] Bài toán thực tế, liên môn
----------[5] [Giảm] Các phương pháp giải được giảm tải
-------------[1] Phương pháp đặt ẩn phụ cho PT mũ, lôgarít
-------------[2] Phương pháp lôgarít hóa, mũ cho PT mũ, lôgarít
-------------[3] Phương pháp hàm số, đánh giá cho PT mũ, lôgarít
-------------[4] Hệ PT mũ, lôgarít
-------------[5] Phương pháp đặt ẩn phụ với BPT mũ, lôgarít
-------------[6] Phương pháp lôgarít hóa, mũ cho BPT mũ, lôgarít
-------------[7] Phương pháp hàm số, đánh giá cho BPT mũ, lôgarít
-------------[8] Hệ BPT mũ, lôgarít
-------[7] Đạo hàm
----------[1] Đạo hàm
-------------[1] Tính đạo hàm bằng định nghĩa
-------------[2] Số e và bài toán lãi kép
-------------[3] Ý nghĩa hình học của đạo hàm
-------------[4] Ý nghĩa Vật lý của đạo hàm
-------------[5] Số gia hàm số, số gia biến số
----------[2] Các quy tắc đạo hàm
-------------[1] Tính đạo hàm
-------------[2] Tính đạo hàm cấp hai
-------------[3] Đẳng thức có y và (y', y'')
-------------[4] Tiếp tuyến tại một điểm
-------------[5] Tiếp tuyến biết trước hệ số góc
-------------[6] Tiếp tuyến chưa biết tiếp điểm và hệ số góc
-------------[7] Bài toán quãng đường, vận tốc, gia tốc
-------------[8] Giới hạn hàm số lượng giác, hàm số mũ, lôgarít
-------------[9] Dùng đạo hàm cho nhị thức Newton
-------[8] Quan hệ vuông góc trong không gian
----------[1] Hai đường thẳng vuông góc
-------------[1] Câu hỏi lí thuyết
-------------[2] Xác định góc giữa hai đường thẳng bằng định nghĩa
-------------[3] Xác định hai đường thẳng vuông góc
-------------[4] Các bài toán thực tế
----------[2] Đường thẳng vuông góc với mặt phẳng
-------------[1] Câu hỏi lí thuyết
-------------[2] Xác định đường thẳng và mặt phẳng vuông góc
-------------[3] Xác định hai đường thẳng vuông góc
-------------[4] Góc giữa hai đường thẳng (có d vuông (P))
-------------[5] Phép chiếu vuông góc
-------------[6] Dựng mặt phẳng, tìm thiết diện
-------------[7] Các bài toán thực tế
----------[3] Hai mặt phẳng vuông góc
-------------[1] Câu hỏi lí thuyết
-------------[2] Xác định quan hệ vuông góc giữa ĐT và MP, MP và MP
-------------[3] Xác định góc giữa hai mặt phẳng
-------------[4] Dựng mặt phẳng vuông góc với mặt phẳng cho trước. Thiết diện
-------------[5] Hình chiếu vuông góc của đa giác trên mặt phẳng
-------------[6] Tính góc, cạnh, đường cao, diện tích các hình thông dụng
-------------[7] Các bài toán thực tế
----------[4] Khoảng cách trong không gian
-------------[1] Câu hỏi lí thuyết
-------------[2] Khoảng cách giữa 2 điểm, từ một điểm đến một đường thẳng
-------------[3] Khoảng cách từ một điểm đến một mặt phẳng
-------------[4] Khoảng cách giữa hai đường thẳng chéo nhau
-------------[5] Đường vuông góc chung của hai đường thẳng chéo nhau
-------------[6] Thể tích khối chóp, lăng trụ
-------------[7] Thể tích khối chóp cụt và khối khác
-------------[8] Bài toán vận dụng khái niệm thể tích
-------------[9] Các bài toán thực tế
-------------[0] Bài toán cực trị
----------[5] Góc giữa đường thẳng và mặt phẳng. Góc nhị diện
-------------[1] Xác định góc giữa đường thẳng và mặt phẳng
-------------[2] Xác định góc phẳng nhị diện
-------------[3] Các bài toán thực tế
-------[9] Xác suất
----------[1] Biến cố giao và quy tắc nhân xác suất
-------------[1] Câu hỏi lí thuyết
-------------[2] Xác định các loại biến cố
-------------[3] Sơ đồ hình cây
-------------[4] Tính xác suất bằng quy tắc nhân
-------------[5] Tính xác suất bằng định nghĩa
----------[2] Biến cố hợp và quy tắc cộng xác suất
-------------[1] Câu hỏi lí thuyết
-------------[2] Xác định các loại biến cố
-------------[3] Tính xác suất bằng quy tắc cộng
-------------[4] Tính xác suất bằng cách kết hợp quy tắc
----[K] Kết nối tri thức
-------[1] Hàm số lượng giác và phương trình lượng giác
----------[1] Giá trị lượng giác của góc lượng giác
-------------[1] Chuyển đổi đơn vị độ và radian
-------------[2] Số đo của một góc lượng giác
-------------[3] Độ dài của một cung tròn
-------------[4] Biểu diễn góc lượng giác lên đường tròn lượng giác
-------------[5] Xét dấu các giá trị lượng giác
-------------[6] Tính giá trị lượng giác của một góc
-------------[7] Giá trị lượng giác của các góc có liên quan đặc biệt
-------------[8] Biến đổi, thu gọn biểu thức lượng giác
-------------[9] Các bài toán có yếu tố thực tế, liên môn
-------------[0] Câu hỏi lý thuyết
----------[2] Công thức lượng giác
-------------[1] Áp dụng công thức cộng
-------------[2] Áp dụng công thức nhân đôi - hạ bậc
-------------[3] Áp dụng công thức biến đổi tích <-> tổng
-------------[4] Kết hợp nhiều công thức lượng giác
-------------[5] Nhận dạng tam giác
-------------[6] Các bài toán có yếu tố thực tế, liên môn
-------------[7] Câu hỏi lý thuyết
----------[3] Hàm số lượng giác
-------------[1] Tìm tập xác định
-------------[2] Xét tính đơn điệu
-------------[3] Xét tính chẵn, lẻ
-------------[4] Xét tính tuần hoàn, tìm chu kỳ
-------------[5] Tìm tập giá trị và min-max
-------------[6] Bảng biến thiên và đồ thị
----------[4] Phương trình lượng giác cơ bản
-------------[1] Phương trình tương đương
-------------[2] Điều kiện có nghiệm
-------------[3] Phương trình cơ bản dùng Radian
-------------[4] Phương trình cơ bản dùng Độ
-------------[5] Phương trình đưa về dạng cơ bản
-------------[6] Toán thực tế, liên môn
-------------[7] [Giảm] Phương trình bậc n theo một hàm số lượng giác
-------------[8] [Giảm] Phương trình đẳng cấp bậc n đối với sinx và cosx
-------------[9] [Giảm] Phương trình bậc nhất đối với sinx và cosx
-------------[0] [Giảm] Phương trình đối xứng, phản đối xứng
-------------[A] [Giảm] Phương trình lượng giác không mẫu mực
-------------[B] [Giảm] Phương trình lượng giác có chứa ẩn ở mẫu số
-------------[C] [Giảm] Phương trình thường gặp có chứa tham số
-------[2] Dãy số. Cấp số cộng. Cấp số nhân
----------[5] Dãy số
-------------[1] Số hạng tổng quát, biểu diễn dãy số
-------------[2] Tìm số hạng cụ thể của dãy số
-------------[3] Dãy số tăng, dãy số giảm
-------------[4] Dãy số bị chặn
-------------[5] Toán thực tế về dãy số
-------------[6] Câu hỏi lý thuyết
----------[6] Cấp số cộng
-------------[1] Nhận diện cấp số cộng, công sai d
-------------[2] Số hạng tổng quát của cấp số cộng
-------------[3] Tìm số hạng cụ thể trong cấp số cộng
-------------[4] Điều kiện để dãy số là cấp số cộng
-------------[5] Tính tổng của cấp số cộng
-------------[6] Các bài toán thực tế
----------[7] Cấp số nhân
-------------[1] Nhận diện cấp số nhân, công bội q
-------------[2] Số hạng tổng quát của cấp số nhân
-------------[3] Tìm số hạng cụ thể trong cấp số nhân
-------------[4] Điều kiện để dãy số là cấp số nhân
-------------[5] Tính tổng của cấp số nhân
-------------[6] Kết hợp cấp số nhân và cấp số cộng
-------------[7] Các bài toán thực tế
-------[3] Các số đặc trưng đo xu thế trung tâm cho mẫu số liệu ghép nhóm
----------[8] Mẫu số liệu ghép nhóm
-------------[1] Bảng dữ liệu ghép nhóm
----------[9] Các số đặc trưng đo xu thế trung tâm
-------------[1] Số trung bình
-------------[2] Trung vị
-------------[3] Tứ phân vị
-------------[4] Mốt
-------------[5] Câu hỏi lý thuyết
-------[4] Quan hệ song song trong không gian
----------[0] Đường thẳng và mặt phẳng trong không gian
-------------[1] Câu hỏi lý thuyết
-------------[2] Hình biểu diễn của một hình không gian
-------------[3] Tìm giao tuyến của hai mặt phẳng
-------------[4] Tìm giao điểm của đường thẳng và mặt phẳng
-------------[5] Xác định thiết diện
-------------[6] Ba điểm thẳng hàng, ba đường thẳng đồng quy
-------------[7] Bài toán quỹ tích và điểm cố định
-------------[8] Bài toán thực tế
----------[A] Hai đường thẳng song song
-------------[1] Câu hỏi lý thuyết
-------------[2] Hai đường thẳng song song
-------------[3] Tìm giao tuyến bằng cách kẻ song song
-------------[4] Tìm giao điểm của đường thẳng và mặt phẳng
-------------[5] Xác định thiết diện bằng cách kẻ song song
-------------[6] Ba điểm thẳng hàng
-------------[7] Bài toán quỹ tích và điểm cố định
-------------[8] Bài toán thực tế
----------[B] Đường thẳng và mặt phẳng song song
-------------[1] Câu hỏi lý thuyết
-------------[2] Đường thẳng song song với mặt phẳng
-------------[3] Tìm giao tuyến bằng cách kẻ song song
-------------[4] Tìm giao điểm của đường thẳng và mặt phẳng
-------------[5] Xác định thiết diện bằng cách kẻ song song
-------------[6] Ba điểm thẳng hàng
-------------[7] Bài toán quỹ tích và điểm cố định
-------------[8] Bài toán thực tế
----------[C] Hai mặt phẳng song song
-------------[1] Câu hỏi lý thuyết
-------------[2] Hai mặt phẳng song song
-------------[3] Tìm giao tuyến bằng cách kẻ song song
-------------[4] Tìm giao điểm của đường thẳng và mặt phẳng
-------------[5] Xác định thiết diện bằng cách kẻ song song
-------------[6] Bài toán tổng hợp
-------------[7] Bài toán thực tế
----------[D] Phép chiếu song song
-------------[1] Câu hỏi lý thuyết
-------------[2] Hình biểu diễn của một hình không gian
-------------[3] Xác định yế tố song song
-------[5] Giới hạn. Hàm số liên tục
----------[E] Giới hạn của dãy số
-------------[1] Câu hỏi lý thuyết
-------------[2] Phương pháp đặt thừa số chung (lim hữu hạn)
-------------[3] Phương pháp lượng liên hợp (lim hữu hạn)
-------------[4] Giới hạn vô cực
-------------[5] Cấp số nhân lùi vô hạn
-------------[6] Toán thực tế, liên môn liên quan đến giới hạn dãy số
-------------[7] Nguyên lí kẹp
----------[F] Giới hạn của hàm số
-------------[1] Câu hỏi lý thuyết
-------------[2] Thay số trực tiếp
-------------[3] PP đặt thừa số chung, kết quả hữu hạn
-------------[4] PP đặt thừa số chung, kết quả vô cực
-------------[5] PP lượng liên hợp, kết quả hữu hạn
-------------[6] PP lượng liên hợp, kết quả vô cực
-------------[7] Giới hạn một bên
-------------[8] Toán thực tế, liên môn về giới hạn hàm số
----------[G] Hàm số liên tục
-------------[1] Câu hỏi lý thuyết
-------------[2] Tính liên tục thể hiện qua đồ thị
-------------[3] Hàm số liên tục tại một điểm
-------------[4] Hàm số liên tục trên khoảng, đoạn
-------------[5] Bài toán chứa tham số
-------------[6] Toán thực tế, liên môn về hàm số liên tục
-------------[7] Bài toán phương trình có nghiệm
-------[6] Hàm số mũ và hàm số lôgarít
----------[H] Luỹ thừa với số mũ thực
-------------[1] Tính giá trị của biểu thức chứa lũy thừa
-------------[2] Biến đổi, rút gọn biểu thức chứa lũy thừa
-------------[3] So sánh các lũy thừa
-------------[4] Điều kiện cho luỹ thừa, căn thức
----------[I] Lôgarít
-------------[1] Tính giá trị biểu thức chứa lôgarít
-------------[2] Biến đổi, rút gọn, biểu diễn biểu thức chứa lôgarít
-------------[3] Số e, toán thực tế, liên môn
----------[J] Hàm số mũ và hàm số lôgarít
-------------[1] Tập xác định của hàm số
-------------[2] Sự biến thiên và đồ thị của hàm số mũ, lôgarít
-------------[3] So sánh các luỹ thừa và lôgarít
-------------[4] Bài toán thực tế, liên môn
-------------[5] Lý thuyết tổng hợp hàm số lũy thừa, mũ, lôgarít
----------[K] Phương trình, bất phương trình mũ và lôgarít
-------------[1] Điều kiện có nghiệm
-------------[2] Phương trình mũ, lôgarít cơ bản
-------------[3] Bất phương trình mũ, lôgarít cơ bản
-------------[4] Phương trình mũ, lôgarít đưa về cùng cơ số
-------------[5] Bất phương trình mũ, lôgarít đưa về cùng cơ số
-------------[6] Bài toán thực tế, liên môn
-------------[7] [Giảm] Phương pháp đặt ẩn phụ cho PT mũ, lôgarít
-------------[8] [Giảm] Phương pháp lôgarít hóa, mũ cho PT mũ, lôgarít
-------------[9] [Giảm] Phương pháp hàm số, đánh giá cho PT mũ, lôgarít
-------------[0] [Giảm] Hệ PT mũ, lôgarít
-------------[A] [Giảm] Phương pháp đặt ẩn phụ với BPT mũ, lôgarít
-------------[B] [Giảm] Phương pháp lôgarít hóa, mũ cho BPT mũ, lôgarít
-------------[C] [Giảm] Phương pháp hàm số, đánh giá cho BPT mũ, lôgarít
-------------[D] [Giảm] Hệ BPT mũ, lôgarít
-------[7] Quan hệ vuông góc trong không gian
----------[L] Hai đường thẳng vuông góc
-------------[1] Câu hỏi lí thuyết
-------------[2] Xác định góc giữa hai đường thẳng bằng định nghĩa
-------------[3] Xác định hai đường thẳng vuông góc
-------------[4] Các bài toán thực tế
----------[M] Đường thẳng vuông góc với mặt phẳng
-------------[1] Câu hỏi lí thuyết
-------------[2] Xác định đường thẳng và mặt phẳng vuông góc
-------------[3] Xác định hai đường thẳng vuông góc
-------------[4] Góc giữa hai đường thẳng (có d vuông (P))
-------------[5] Dựng mặt phẳng, tìm thiết diện
-------------[6] Các bài toán thực tế
----------[N] Phép chiếu vuông góc. Góc giữa đường thẳng và mặt phẳng
-------------[1] Câu hỏi lí thuyết
-------------[2] Phép chiếu vuông góc
-------------[3] Xác định góc giữa đường thẳng và mặt phẳng
-------------[4] Các bài toán thực tế
----------[O] Hai mặt phẳng vuông góc
-------------[1] Câu hỏi lí thuyết
-------------[2] Xác định quan hệ vuông góc giữa ĐT và MP, MP và MP
-------------[3] Xác định góc giữa hai mặt phẳng
-------------[4] Xác định góc phẳng nhị diện
-------------[5] Dựng mặt phẳng vuông góc với mặt phẳng cho trước. Thiết diện
-------------[6] Hình chiếu vuông góc của đa giác trên mặt phẳng
-------------[7] Tính góc, cạnh, đường cao, diện tích các hình thông dụng
-------------[8] Các bài toán thực tế
----------[P] Khoảng cách
-------------[1] Câu hỏi lí thuyết
-------------[2] Khoảng cách giữa 2 điểm, từ một điểm đến một đường thẳng
-------------[3] Khoảng cách từ một điểm đến một mặt phẳng
-------------[4] Khoảng cách giữa hai đường thẳng chéo nhau
-------------[5] Đường vuông góc chung của hai đường thẳng chéo nhau
-------------[6] Các bài toán thực tế
-------------[7] Bài toán cực trị
----------[Q] Thể tích
-------------[1] Thể tích khối chóp, lăng trụ
-------------[2] Thể tích khối chóp cụt và khối khác
-------------[3] Bài toán vận dụng khái niệm thể tích
-------------[4] Các bài toán thực tế
-------------[5] Bài toán cực trị
-------[8] Các quy tắc tính xác suất
----------[R] Biến cố hợp, biến cố giao, biến cố độc lập
-------------[1] Câu hỏi lí thuyết
-------------[2] Mô tả không gian mẫu, biến cố
-------------[3] Xác định các loại biến cố
----------[S] Công thức cộng xác suất
-------------[1] Câu hỏi lí thuyết
-------------[2] Tính xác suất bằng định nghĩa
-------------[3] Tính xác suất bằng quy tắc cộng
----------[T] Công thức nhân xác suất cho 2 biến cố độc lập
-------------[1] Câu hỏi lí thuyết
-------------[2] Sơ đồ hình cây
-------------[3] Tính xác suất bằng quy tắc nhân
-------------[4] Tính xác suất bằng cách kết hợp quy tắc
-------[9] Đạo hàm
----------[U] Định nghĩa và ý nghĩa của đạo hàm
-------------[1] Tính đạo hàm bằng định nghĩa
-------------[2] Ý nghĩa hình học của đạo hàm
-------------[3] Ý nghĩa Vật lý của đạo hàm
-------------[4] Bài toán thực tế
-------------[5] [Giảm] Số gia hàm số, số gia biến số
----------[V] Các quy tắc tính đạo hàm
-------------[1] Tính đạo hàm
-------------[2] Đẳng thức có y và y'
-------------[3] Tiếp tuyến tại một điểm
-------------[4] Tiếp tuyến biết trước hệ số góc
-------------[5] Tiếp tuyến chưa biết tiếp điểm và hệ số góc
-------------[6] Bài toán thực tế, liên môn
-------------[7] Giới hạn hàm số lượng giác, hàm số mũ, lôgarít
-------------[8] Dùng đạo hàm cho nhị thức Newton
----------[W] Đạo hàm cấp hai
-------------[1] Tính đạo hàm cấp hai
-------------[2] Đẳng thức có y và (y', y'')
-------------[3] Ý nghĩa Vật lý của đạo hàm cấp hai
----[D] Đại số và giải tích
-------[1] Hàm số lượng giác. Phương trình lượng giác
----------[1] Các hàm số lượng giác
-------------[1] Tìm tập xác định
-------------[2] Xét tính đơn điệu
-------------[3] Xét tính chẵn, lẻ
-------------[4] Xét tính tuần hoàn, tìm chu kỳ
-------------[5] Tìm tập giá trị và min-max
-------------[6] Bảng biến thiên và đồ thị
----------[2] Phương trình lượng giác cơ bản
-------------[1] Phương trình lượng giác cơ bản
----------[3] Phương trình lượng giác thường gặp
-------------[1] Phương trình bậc n theo một hàm số lượng giác
-------------[2] Phương trình đẳng cấp bậc n đối với sinx và cosx
-------------[3] Phương trình bậc nhất đối với sinx và cosx
-------------[4] Phương trình đối xứng, phản đối xứng
-------------[5] Phương trình lượng giác không mẫu mực
-------------[6] Phương trình lượng giác có chứa ẩn ở mẫu số
-------------[7] Phương trình lượng giác có chứa tham số
-------------[8] Bài toán thực tế
-------[2] Tổ hợp. Xác suất. Nhị thức Newton
----------[1] Quy tắc cộng-quy tắc nhân
-------------[1] Bài toán sử dụng quy tắc cộng
-------------[2] Bài toán sử dụng quy tắc nhân
-------------[3] Bài toán kết hợp quy tắc cộng và quy tắc nhân
----------[2] Hoán vị-chỉnh hợp-tổ hợp
-------------[1] Bài toán chỉ sử dụng P hoặc C hoặc A
-------------[2] Bài toán kết hợp P, C và A
-------------[3] Bài toán liên quan đến hình học
-------------[4] Hoán vị bàn tròn
-------------[5] Hoán vị lặp
-------------[6] Giải PT, BPT, HPT, chứng minh liên quan đến P,C,A
----------[3] Nhị thức Newton
-------------[1] Khai triển một nhị thức Newton
-------------[2] Tìm hệ số, số hạng trong khai triển nhị thức Newton
-------------[3] Chứng minh, tính giá trị của biểu thức đại số tổ hợp có sử dụng nhị thức Newton
----------[4] Phép thử và biến cố
-------------[1] Mô tả không gian mẫu, biến cố
-------------[2] Các câu hỏi lý thuyết tổng hợp
----------[5] Xác suất của biến cố
-------------[1] Các câu hỏi lý thuyết tổng hợp
-------------[2] Tính xác suất bằng định nghĩa
-------------[3] Tính xác suất bằng công thức cộng
-------------[4] Tính xác suất bằng công thức nhân
-------------[5] Bài toán kết hợp quy tắc cộng và quy tắc nhân xác suất
-------[3] Dãy số - Cấp số cộng- Cấp số nhân
----------[1] Phương pháp quy nạp
-------------[1] Các dạng toán áp dụng trực tiếp phương pháp quy nạp
----------[2] Dãy số
-------------[1] Biểu diễn dãy số, tìm công thức tổng quát dãy số
-------------[2] Tìm hạng tử trong dãy số
-------------[3] Dãy số tăng, dãy số giảm
-------------[4] Dãy số bị chặn trên, bị chặn dưới
-------------[5] Tìm giới hạn của dãy số
----------[3] Cấp số cộng
-------------[1] Nhận diện cấp số cộng
-------------[2] Tìm công thức của cấp số cộng
-------------[3] Tìm hạng tử trong cấp số cộng
-------------[4] Tìm điều kiện và chứng minh một dãy số là cấp số cộng
-------------[5] Tính tổng của dãy nhiều số hạng liên quan đến CSC, tổng các hạng tử của CSC
-------------[6] Các bài toán thực tế
----------[4] Cấp số nhân
-------------[1] Nhận diện cấp số nhân
-------------[2] Tìm công thức của cấp số nhân
-------------[3] Tìm hạng tử trong cấp số nhân
-------------[4] Tìm điêu kiện và chứng minh một dãy số là cấp số nhân
-------------[5] Tính tổng của dãy nhiều số hạng liên quan đến CSN, tổng các hạng tử của CSN
-------------[6] Kết hợp cấp số nhân và cấp số cộng
-------------[7] Các bài toán thực tế
-------[4] Giới hạn
----------[1] Giới hạn của dãy số
-------------[1] Câu hỏi lý thuyết
-------------[2] Nguyên lí kẹp
-------------[3] Dùng phương pháp đặt thừa số
-------------[4] Dùng lượng liên hợp
-------------[5] Cấp số nhân lùi vô hạn
-------------[6] Toán thực tế, liên môn liên quan đến giới hạn dãy số
----------[2] Giới hạn của hàm số
-------------[1] Câu hỏi lý thuyết
-------------[2] Thay số trực tiếp
-------------[3] Dạng 0/0, 0 nhân vô cùng
-------------[4] Dạng vô cùng trừ vô cùng
-------------[5] Giới hạn một bên
-------------[6] Giới hạn bằng vô cùng
-------------[7] Dạng vô cùng chia vô cùng, số chia vô cùng
-------------[8] Toán thực tế, liên môn về giới hạn hàm số
----------[3] Hàm số liên tục
-------------[1] Câu hỏi lý thuyết
-------------[2] Xét tính liên tục bằng đồ thị
-------------[3] Hàm số liên tục tại một điểm
-------------[4] Hàm số liên tục trên khoảng, đoạn
-------------[5] Bài toán chứa tham số
-------------[6] Chứng minh phương trình có nghiệm
-------------[7] Toán thực tế, liên môn về hàm số liên tục
-------[5] Đạo hàm
----------[1] Đạo hàm và ý nghĩa của đạo hàm
-------------[1] Tính đạo hàm bằng định nghĩa
----------[2] Quy tắc tính đạo hàm
-------------[1] Tính đạo hàm và bài toán liên quan
-------------[2] Tiếp tuyến tại điểm
-------------[3] Tiếp tuyến cho sẵn hệ số góc, song song - vuông góc
-------------[4] Tiếp tuyến đi qua một điểm
-------------[5] Tổng hợp về tiếp tuyến và các kiến thức liên quan
-------------[6] Bài toán quãng đường, vận tốc, gia tốc
----------[3] Đạo hàm của các hàm số lượng giác
-------------[1] Tính đạo hàm và bài toán liên quan
-------------[2] Giới hạn hàm số lượng giác
----------[4] Vi phân
-------------[1] Tính vi phân và bài toán liên quan
----------[5] Đạo hàm cấp hai
-------------[1] Tính đạo hàm các cấp
-------------[2] Mối liên hệ giữa hàm số và đạo hàm các cấp
-------------[3] Ứng dụng vào tính tổng khai triển nhị thức và giới hạn
----[H] Hình học
-------[1] Phép dời hình và phép đồng dạng trong mặt phẳng
----------[1] Phép biến hình
-------------[1] Câu hỏi lý thuyết
-------------[2] Bài toán xác định một phép đặt tương ứng có là phép dời hình hay không
----------[2] Phép tịnh tiến
-------------[1] Câu hỏi lý thuyết
-------------[2] Tìm ảnh hoặc tạo ảnh khi thực hiện phép tịnh tiến
-------------[3] Ứng dụng phép tịnh tiến
----------[3] Phép đối xứng trục
-------------[1] Câu hỏi lý thuyết
-------------[2] Tìm ảnh hoặc tạo ảnh khi thực hiện phép đối xứng trục
-------------[3] Xác định trục đối xứng và số trục đối xứng của một hình
-------------[4] Ứng dụng phép đối xứng trục
----------[4] Phép đối xứng tâm
-------------[1] Câu hỏi lý thuyết
-------------[2] Tìm ảnh, tạo ảnh khi thực hiện phép đối xứng tâm
-------------[3] Xác định hình có tâm đối xứng
-------------[4] Ứng dụng phép đối xứng tâm
----------[5] Phép quay
-------------[1] Câu hỏi lý thuyết
-------------[2] Xác định vị trí ảnh của điểm, hình khi thực hiện phép quay cho trước
-------------[3] Tìm tọa độ ảnh của điểm, phương trình của một đường thẳng
----------[6] Khái niệm về phép dời hình và hai hình bằng nhau
-------------[1] Câu hỏi lý thuyết
-------------[2] Xác định ảnh khi thực hiện phép dời hình
----------[7] Phép vị tự
-------------[1] Câu hỏi lý thuyết
-------------[2] Xác định ảnh, tạo ảnh khi thực hiện phép vị tự
-------------[3] Tìm tâm vị tự của hai đường tròn
-------------[4] Ứng dụng phép vị tự
----------[8] Phép đồng dạng
-------------[1] Câu hỏi lý thuyết
-------------[2] Xác định ảnh, tạo ảnh khi thực hiện phép đồng dạng
-------[2] Quan hệ song song trong không gian
----------[1] Đại cương về đường thẳng và mặt phẳng
-------------[1] Câu hỏi lý thuyết
-------------[2] Xác định giao tuyến của hai mặt phẳng
-------------[3] Tìm giao điểm của đường thẳng và mặt phẳng
-------------[4] Xác định thiết diện
-------------[5] Chứng minh ba điểm thẳng hàng đồng quy và ba đường thẳng đồng quy
-------------[6] Bài toán điểm cố định và quỹ tích của một điểm
----------[2] Hai đường thẳng chéo nhau và hai đường thẳng song song
-------------[1] Câu hỏi lý thuyết
-------------[2] Chứng minh hai đường thẳng song song
-------------[3] Tìm giao điểm của đường thẳng và mặt phẳng
-------------[4] Tìm giao tuyến, thiết diện bằng cách kẻ song song
-------------[5] Chứng minh ba điểm thẳng hàng
-------------[6] Xác định quỹ tích và các yếu tố định
----------[3] Đường thẳng và mặt phẳng song song
-------------[1] Câu hỏi lý thuyết
-------------[2] Đường thẳng song song với mặt phẳng
-------------[3] Giao tuyến của hai mặt phẳng
-------------[4] Thiết diện
-------------[5] Giao điểm
----------[4] Hai mặt phẳng song song
-------------[1] Câu hỏi lý thuyết
-------------[2] Hai mặt phẳng song song
-------------[3] Giao tuyến của hai mặt phẳng
-------------[4] Thiết diện
-------------[5] Giao điểm
-------------[6] Các bài toán tổng hợp
----------[5] Phép chiếu song song. Hình biểu diễn của một hình không gian
-------------[1] Câu hỏi lý thuyết
-------------[2] Vẽ hình biểu diễn
-------------[3] Xác định song song
-------[3] Véc-tơ trong không gian. Quan hệ vuông góc trong không gian
----------[1] Véc-tơ trong không gian
-------------[1] Câu hỏi lý thuyết
-------------[2] Đẳng thức véc-tơ
-------------[3] Phân tích véc-tơ theo các véc-tơ cho trước
-------------[4] Điều kiện đồng phẳng của ba véc-tơ
-------------[5] Ba điểm thẳng hàng, hai đường thẳng song song
----------[2] Hai đường thẳng vuông góc
-------------[1] Câu hỏi lí thuyết
-------------[2] Xác định góc giữa hai véc-tơ bằng định nghĩa
-------------[3] Xác định góc giữa hai đường thẳng bằng định nghĩa
-------------[4] Ứng dụng tích vô hướng của hai véc-tơ
----------[3] Đường thẳng vuông góc với mặt phẳng
-------------[1] Câu hỏi lí thuyết
-------------[2] Xác định quan hệ vuông góc giữa ĐT và ĐT, ĐT và MP
-------------[3] Xác định góc giữa mặt phẳng và đường thẳng
-------------[4] Dựng mặt phẳng vuông góc với đường thẳng cho trước. Thiết diện
----------[4] Hai mặt phẳng vuông góc
-------------[1] Câu hỏi lí thuyết
-------------[2] Xác định quan hệ vuông góc giữa ĐT và MP, MP và MP
-------------[3] Xác định góc giữa hai mặt phẳng, góc giữa đường thẳng và mặt phẳng
-------------[4] Dựng mặt phẳng vuông góc với mặt phẳng cho trước. Thiết diện
-------------[5] Hình chiếu vuông góc của đa giác trên mặt phẳng
-------------[6] Góc giữa hai véc-tơ, hai đường thẳng trong các hình lăng trụ, lập phương
----------[5] Khoảng cách
-------------[1] Câu hỏi lí thuyết
-------------[2] Tính độ dài đoạn thẳng và tính khoảng cách từ một điểm đến một đường thẳng
-------------[3] Khoảng cách từ một điểm đến một mặt phẳng
-------------[4] Khoảng cách giữa hai đường thẳng chéo nhau
-------------[5] Xác định đường vuông góc chung của hai đường thẳng chéo nhau
-[2] Lớp 12
----[D] Giải tích
-------[1] Ứng dụng đạo hàm để khảo sát hàm số
----------[1] Sự đồng biến và nghịch biến của hàm số
-------------[1] Xét tính đơn điệu của hàm số cho bởi công thức
-------------[2] Xét tính đơn điệu dựa vào bảng biến thiên, đồ thị
-------------[3] Tìm tham số m để hàm số đơn điệu
-------------[4] Ứng dụng tính đơn điệu để chứng minh BĐT, giải PT, BPT, HPT
----------[2] Cực trị của hàm số
-------------[1] Tìm cực trị của hàm số cho bởi công thức
-------------[2] Tìm cực trị dựa vào BBT, đồ thị
-------------[3] Tìm m để hàm số đạt cực trị tại 1 điểm cho trước
-------------[4] Tìm m để hàm số, đồ thị hàm số bậc ba có cực trị thỏa mãn điều kiện
-------------[5] Tìm m để hàm số, đồ thị hàm số trùng phương có cực trị thỏa mãn điều kiện
-------------[6] Tìm m để hàm số, đồ thị hàm số các hàm số khác có cực trị thỏa mãn điều kiện
----------[3] Giá trị lớn nhất và giá trị nhỏ nhất của hàm số
-------------[1] GTLN, GTNN trên đoạn
-------------[2] GTLN, GTNN trên khoảng
-------------[3] Sử dụng các đánh giá, bất đẳng thức cổ điển
-------------[4] Ứng dụng GTNN, GTLN trong bài toán PT, BPT
-------------[5] GTLN, GTNN hàm nhiều biến
-------------[6] Bài toán ứng dụng, tối ưu, thực tế
----------[4] Đường tiệm cận
-------------[1] Xác định các ĐTC của HS (không chứa tham số) hoặc biết BBT, đồ thị
-------------[2] Bài toán xác định các đường tiệm cận của hàm số có chứa tham số
-------------[3] Bài toán liên quan đến đồ thị hàm số và các đường tiệm cận
----------[5] Khảo sát sự biến thiên và vẽ đồ thị hàm số
-------------[1] Nhận dạng đồ thị, bảng biến thiên
-------------[2] Các phép biến đổi đồ thị
-------------[3] Biện luận số giao điểm dựa vào đồ thị, bảng biến thiên
-------------[4] Sự tương giao của hai đồ thị (liên quan đến tọa độ giao điểm)
-------------[5] Đồ thị của hàm đạo hàm
-------------[6] Phương trình tiếp tuyến của đồ thị hàm số
-------------[7] Điểm đặc biệt của đồ thị hàm số
-------[2] Hàm số lũy thừa-Hàm số mũ và Hàm số lô-ga-rít
----------[1] Lũy thừa
-------------[1] Tính giá trị của biểu thức chứa lũy thừa
-------------[2] Biến đổi, rút gọn, biểu diễn các biểu thức chứa lũy thừa
-------------[3] So sánh các lũy thừa
----------[2] Hàm số lũy thừa
-------------[1] Tập xác định của hàm số chứa hàm lũy thừa
-------------[2] Đạo hàm hàm số lũy thừa
-------------[3] Khảo sát sự biến thiên và đồ thị hàm số lũy thừa
-------------[4] Tìm giá trị lớn nhất, giá trị nhỏ nhất của biểu thức chứa hàm lũy thừa
----------[3] Lô-ga-rít
-------------[1] Tính giá trị biểu thức chứa lô-ga-rít
-------------[2] Biến đổi, rút gọn, biểu diễn biểu thức chứa lô-ga-rít
-------------[3] So sánh các biểu thức lô-ga-rít
----------[4] Hàm số mũ. Hàm số lô-ga-rít
-------------[1] Tập xác định của hàm số mũ, hàm số lô-ga-rít
-------------[2] Tính đạo hàm hàm số mũ, hàm số lô-ga-rít
-------------[3] Khảo sát sự biến thiên và đồ thị của hàm số mũ, lô-ga-rít
-------------[4] Tìm giá trị lớn nhất, nhỏ nhất của biểu thức chứa hàm mũ, hàm lô-ga-rít
-------------[5] Bài toán thực tế
-------------[6] Giới hạn, liên tục liên quan hàm số mũ, lô-ga-rít
-------------[7] Lý thuyết tổng hợp hàm số lũy thừa, mũ, lô-ga-rít
----------[5] Phương trình mũ và phương trình lô-ga-rít
-------------[1] Phương trình cơ bản
-------------[2] Phương pháp đưa về cùng cơ số
-------------[3] Phương pháp đặt ẩn phụ
-------------[4] Phương pháp lô-ga-rít hóa, mũ hóa
-------------[5] Phương pháp hàm số, đánh giá
-------------[6] Bài toán thực tế
----------[6] Bất phương trình mũ và lô-ga-rít
-------------[1] Bất phương trình cơ bản
-------------[2] Phương pháp đưa về cùng cơ số
-------------[3] Phương pháp đặt ẩn phụ
-------------[4] Phương pháp lô-ga-rít hóa, mũ hóa
-------------[5] Phương pháp hàm số, đánh giá
-------------[6] Bài toán thực tế
-------[3] Nguyên hàm, tích phân và ứng dụng
----------[1] Nguyên hàm
-------------[1] Định nghĩa, tính chất và nguyên hàm cơ bản
-------------[2] Phương pháp đổi biến số
-------------[3] Phương pháp nguyên hàm từng phần
----------[2] Tích phân
-------------[1] Định nghĩa, tính chất và tích phân cơ bản
-------------[2] Phương pháp đổi biến số
-------------[3] Phương pháp tích phân từng phần
-------------[4] Tích phân của hàm ẩn. Tích phân đặc biệt
-------------[5] Kỹ thuật bình phương
----------[3] Ứng dụng của tích phân
-------------[1] Diện tích hình phẳng được giới hạn bởi các đồ thị
-------------[2] Bài toán thực tế sử dụng diện tích hình phẳng
-------------[3] Thể tích giới hạn bởi các đồ thị (tròn xoay)
-------------[4] Thể tích tính theo mặt cắt S(x)
-------------[5] Bài toán thực tế và ứng dụng thể tích
-------------[6] Ứng dụng vào tính tổng khai triển nhị thức
-------------[7] Ứng dụng tích phân vào bài toán liên môn (lý, hóa, sinh, kinh tế)
-------[4] Số phức
----------[1] Khái niệm số phức
-------------[1] Xác định các yếu tố cơ bản của số phức
-------------[2] Biểu diễn hình học cơ bản của số phức
----------[2] Phép cộng, trừ và nhân số phức
-------------[1] Thực hiện phép tính
-------------[2] Xác định các yếu tố cơ bản của số phức qua các phép toán
-------------[3] Bài toán quy về giải phương trình, hệ phương trình nghiệm thực
-------------[4] Bài toán tập hợp điểm
----------[3] Phép chia số phức
-------------[1] Thực hiện phép tính
-------------[2] Xác định các yếu tố cơ bản của số phức qua các phép toán
-------------[3] Bài toán quy về giải phương trình, hệ phương trình nghiệm thực
-------------[4] Bài toán tập hợp điểm
----------[4] Phương trình bậc hai hệ số thực
-------------[1] Giải phương trình. Tính toán biểu thức nghiệm
-------------[2] Định lí Viet và ứng dụng
-------------[3] Phương trình quy về bậc hai
----------[5] Cực trị
-------------[1] Phương pháp hình học
-------------[2] Phương pháp đại số
----[H] Hình học
-------[1] Khối đa diện
----------[1] Khái niệm về khối đa diện
-------------[1] Nhận diện hình đa diện, khối đa diện
-------------[2] Xác định số đỉnh, cạnh, mặt bên của một khối đa diện
-------------[3] Phân chia, lắp ghép các khối đa diện
-------------[4] Phép biến hình trong không gian
----------[2] Khối đa diện lồi và khối đa diện đều
-------------[1] Nhận diện đa diện lồi
-------------[2] Nhận diện loại đa diện đều
-------------[3] Tính chất đối xứng
----------[3] Khái niệm về thể tích của khối đa diện
-------------[1] Diện tích xung quanh, diện tích toàn phần của khối đa diện
-------------[2] Tính thể tích các khối đa diện
-------------[3] Tỉ số thể tích
-------------[4] Các bài toán khác(góc, khoảng cách,...) liên quan đến thể tích khối đa diện
-------------[5] Bài toán thực tế về khối đa diện
-------------[6] Bài toán cực trị
-------[2] Mặt nón, mặt trụ, mặt cầu
----------[1] Khái niệm về mặt tròn xoay
-------------[1] Thể tích khối nón, khối trụ
-------------[2] Sxq, Stp, độ dài đường sinh, chiều cao, bán kính đáy, thiết diện
-------------[3] Khối tròn xoay nội tiếp, ngoại tiếp khối đa diện
-------------[4] Bài toán thực tế về khối nón, khối trụ
-------------[5] Bài toán cực trị về khối nón, khối trụ
-------------[6] Câu hỏi lý thuyết
----------[2] Mặt cầu
-------------[1] Bài toán sử dụng định nghĩa, tính chất, vị trí tương đối
-------------[2] Khối cầu ngoại tiếp khối đa diện
-------------[3] Khối cầu nội tiếp khối đa diện
-------------[4] Bài toán thực tế về khối cầu
-------------[5] Bài toán cực trị về khối cầu
-------------[6] Bài toán tổng hợp về khối nón, khối trụ, khối cầu
-------[3] Phương pháp tọa độ trong không gian
----------[1] Hệ tọa độ trong không gian
-------------[1] Tìm tọa độ điểm, véc-tơ liên quan đến hệ trục Oxyz
-------------[2] Tích vô hướng và ứng dụng
-------------[3] Xác định tâm, bán kính, viết PT mặt cầu đơn giản,...
-------------[4] Các bài toán cực trị
----------[2] Phương trình mặt phẳng
-------------[1] Tích có hướng và ứng dụng
-------------[2] Xác định VTPT
-------------[3] Viết phương trình mặt phẳng
-------------[4] Tìm tọa độ điểm liên quan đến mặt phẳng
-------------[5] Góc
-------------[6] Khoảng cách
-------------[7] Vị trí tương đối giữa hai mặt phẳng, giữa mặt cầu và mặt phẳng
-------------[8] Các bài toán cực trị
----------[3] Phương trình đường thẳng trong không gian
-------------[1] Xác định VTCP
-------------[2] Viết phương trình đường thẳng
-------------[3] Tìm tọa độ điểm liên quan đến đường thẳng
-------------[4] Góc
-------------[5] Khoảng cách
-------------[6] Vị trí tương đối giữa hai đường thẳng, giữa ĐT và MP
-------------[7] Bài toán liên quan giữa đường thẳng - mặt phẳng - mặt cầu
-------------[8] Các bài toán cực trị
----------[4] Ứng dụng của phương pháp tọa độ
-------------[1] Bài toán HHKG
-------------[2] Bài toán đại số
-[6] Lớp 6
----[C] Cánh Diều
-------[1] Số tự nhiên
----------[1] Tập hợp
----------[2] Tập hợp các số tự nhiên
----------[3] Phép cộng, phép trừ các số tự nhiên
----------[4] Phép nhân, phép chia các số tự nhiên
----------[5] Phép tính lũy thừa với số mũ tự nhiên
----------[6] Thứ tự thực hiện các phép tính
----------[7] Quan hệ chia hết. Tính chất chia hết
----------[8] Dấu hiệu chia hết cho 2, cho 5
----------[9] Dấu hiệu chia hết cho 3, cho 9
----------[0] Số nguyên tố. Hợp số
----------[A] Phân tích một số ra thừa số nguyên tố
----------[B] Ước chung và ước chung lớn nhất
----------[C] Bội chung và bội chung nhỏ nhất
-------[2] Số nguyên
----------[1] Số nguyên âm
----------[2] Tập hợp các số nguyên
----------[3] Phép cộng các số nguyên
----------[4] Phép trừ số nguyên. Quy tắc dấu ngoặc
----------[5] Phép nhân các số nguyên
----------[6] Phép chia hết hai số nguyên. Quan hệ chia hết trong tập hợp số nguyên
-------[3] Hình học trực quan
----------[1] Tam giác đều. Hình vuông. Lục giác đều
----------[2] Hình chữ nhật. Hình thoi
----------[3] Hình bình hành
----------[4] Hình thang cân
----------[5] Hình có trục đối xứng
----------[6] Hình có tâm đối xứng
----------[7] Đối xứng trong thực tiễn
-------[4] Một số yếu tố thống kê và xác suất
----------[1] Thu thập, tổ chức, biểu diễn, phân tích và xử lí dữ liệu
----------[2] Biểu đồ cột kép
----------[3] Mô hình xác suất trong một số trò chơi và thí nghiệm đơn giản
----------[4] Xác suất thực nghiệm trong một trò chơi và thí nghiệm đơn giản
-------[5] Phân số và số thập phân
----------[1] Phân số với tử và mẫu là số nguyên
----------[2] So sánh các phân số. Hỗn số dương
----------[3] Phép cộng. Phép trừ phân số
----------[4] Phép nhân, phép chia phân số
----------[5] Số thập phân
----------[6] Phép cộng, phép trừ số thập phân
----------[7] Phép nhân, phép chia số thập phân
----------[8] Ước lượng và làm tròn số
----------[9] Tỉ số. Tỉ số phần trăm
----------[0] Hai bài toán về phân số
-------[6] Hình học phẳng
----------[1] Điểm. Đường thẳng
----------[2] Hai đường thẳng cắt nhau. Hai đường thẳng song song
----------[3] Đoạn thẳng
----------[4] Tia
----------[5] Góc
----[T] Chân Trời Sáng Tạo
-------[1] Số tự nhiên
----------[1] Tập hợp. Phần tử của tập hợp
----------[2] Tập hợp số tự nhiên. Ghi số tự nhiên
----------[3] Các phép tính trong tập hợp số tự nhiên
----------[4] Lũy thừa với số mũ tự nhiên
----------[5] Thứ tự thực hiện các phép tính
----------[6] Chia hết và chia có dư. Tính chất chia hết của một tổng
----------[7] Dấu hiệu chia hết cho 2, cho 5
----------[8] Dấu hiệu chia hết cho 3, cho 9
----------[9] Ước và bội
----------[0] Số nguyên tố. Hợp số. Phân tích một số ra thừa số nguyên tố
----------[A] Hoạt động thực hành và trải nghiệm
----------[B] Ước chung. Ước chung lớn nhất
----------[C] Bội chung. Bội chung nhỏ nhất
----------[D] Hoạt động thực hành và trải nghiệm
-------[2] Số nguyên
----------[1] Số nguyên âm và tập hợp các số nguyên
----------[2] Thứ tự trong tập hợp số nguyên
----------[3] Phép cộng và phép trừ hai số nguyên
----------[4] Phép nhân và phép chia hết hai số nguyên
----------[5] Hoạt động thực hành và trải nghiệm
-------[3] Hình học trực quan và hình phẳng trong thực tiễn
----------[1] Hình vuông - Tam giác đều - Lục giác đều
----------[2] Hình chữ nhật - Hình thoi - Hình bình hành - Hình thang cân
----------[3] Chu vi và diện tích của một số hình trong thực tiễn
----------[4] Hoạt động thực hành và trải nghiệm
-------[4] Một số yếu tố thống kê
----------[1] Thu thập và phân loại dữ liệu
----------[2] Biểu diễn dữ liệu trên bảng
----------[3] Biểu đồ tranh
----------[4] Biểu đồ cột - Biểu đồ cột kép
----------[5] Hoạt động thực hành và trải nghiệm
-------[5] Phân số
----------[1] Phân số với tử số và mẫu số là số nguyên
----------[2] Tính chất cơ bản của phân số
----------[3] So sánh phân số
----------[4] Phép cộng và phép trừ phân số
----------[5] Phép nhân và phép chia phân số
----------[6] Giá trị phân số của một số
----------[7] Hỗn số
----------[8] Hoạt động thực hành và trải nghiệm
-------[6] Số thập phân
----------[1] Số thập phân
----------[2] Các phép tính với số thập phân
----------[3] Làm tròn số thập phân và ước lượng kết quả
----------[4] Tỉ số và tỉ số phần trăm
----------[5] Bài toán về tỉ số phần trăm
----------[6] Hoạt động thực hành và trải nghiệm
-------[7] Hình học trực quan
----------[1] Hình có trục đối xứng
----------[2] Hình có tâm đối xứng
----------[3] Vai trò của tính đối xứng trong thế giới tự nhiên
----------[4] Hoạt động thực hành và trải nghiệm
-------[8] Hình học phẳng và các hình hình học cơ bản
----------[1] Điểm. Đường thẳng
----------[2] Ba điểm thẳng hàng. Ba điểm không thẳng
----------[3] Hai đường thẳng cắt nhau, song song. Tia
----------[4] Đoạn thẳng. Độ dài đoạn thẳng
----------[5] Trung điểm của đoạn thẳng
----------[6] Góc
----------[7] Số đo góc. Các góc đặc biệt
----------[8] Hoạt động thực hành và trải nghiệm
-------[9] Một số yếu tố xác suất
----------[1] Phép thử nghiệm. Sự kiện
----------[2] Xác suất thực nghiệm
----------[3] Hoạt động thực hành và trải nghiệm
----[K] Kết Nối Tri Thức
-------[1] Tập hợp các số tự nhiên
----------[1] Tập hợp
----------[2] Cách ghi số tự nhiên
----------[3] Thứ tự trong tập hợp các số tự nhiên
----------[4] Phép cộng và phép trừ số tự nhiên
----------[5] Phép nhân và phép chia số tự nhiên
----------[6] Lũy thừa với số mũ tự nhiên
----------[7] Thứ tự thực hiện các phép tính
-------[2] Tính chia hết trong tập hợp các số tự nhiên
----------[8] Quan hệ chia hết và tính chất
----------[9] Dấu hiệu chia hết
----------[0] Số nguyên tố
----------[A] Ước chung. Ước chung lớn nhất
----------[B] Bội chung. Bội chung nhỏ nhất
-------[3] Số nguyên
----------[C] Tập hợp các số nguyên
----------[D] Phép cộng và phép trừ số nguyên
----------[E] Quy tắc dấu ngoặc
----------[F] Phép nhân số nguyên
----------[G] Phép chia hết. Ước và bội của một số nguyên
-------[4] Một số hình phẳng trong thực tiễn
----------[H] Hình tam giác đều. Hình vuông. Hình lục giác đều
----------[I] Hình chữ nhật. Hình thoi hình bình hành. Hình thang cân
----------[J] Chu vi và diện tích của một số tứ giác đã học
-------[5] Tính đối xứng của hình phẳng trong tự nhiên
----------[K] Hình có trục đối xứng
----------[L] Hình có tâm đối xứng
-------[6] Phân số
----------[M] Mở rộng phân số. Phân số bằng nhau
----------[N] So sánh phân số. Hỗn số dương
----------[O] Phép cộng và phép trừ phân số
----------[P] Phép nhân và phép chia phân số
----------[Q] Hai bài toán về phân số
-------[7] Số thập phân
----------[R] Số thập phân
----------[S] Tính toán với số thập phân
----------[T] Làm tròn và ước lượng
----------[U] Một số bài toán về tỉ số và tỉ số phần trăm
-------[8] Những hình học cơ bản
----------[V] Điểm và đường thẳng
----------[W] Điểm nằm giữa hai điểm. Tia
----------[X] Đoạn thẳng. Độ dài đoạn thẳng
----------[Y] Trung điểm của đoạn thẳng
----------[Z] Góc
----------[1] Số đo góc
-------[9] Dữ liệu và xác suất thực nghiệm
----------[1] Dữ liệu và thu thập dữ liệu
----------[2] Bảng thống kê và biểu đồ tranh
----------[3] Biểu đồ cột
----------[4] Biểu đồ cột kép
----------[5] Kết quả có thể và sự kiện trong trò chơi, thí nghiệm
----------[6] Xác suất thực nghiệm
----[D] Số học
-------[1] Ôn tập và bổ túc về số tự nhiên
----------[1] Tập hợp. Phần tử của tập hợp
----------[2] Tập hợp các số tự nhiên
----------[3] Ghi số tự nhiên
----------[4] Số phần tử của một tập hợp. Tập hợp con
----------[5] Phép cộng và phép nhân
----------[6] Phép trừ và phép chia
----------[7] Lũy thừa với số mũ tự nhiên. Nhân hai lũy thừa cùng cơ số.
----------[8] Chia hai lũy thừa cùng cơ số
----------[9] Thứ tự thực hiện các phép tính
----------[0] Tính chất chia hết của một tổng
----------[A] Dấu hiệu chia hết cho 2, cho 5
----------[B] Dấu hiệu chia hết cho 3, cho 9
----------[C] Ước và bội
----------[D] Số nguyên tố. Hợp số. Bảng số nguyên tố
----------[E] Phân tích một số ra thừa số nguyên tố
----------[F] Ước chung và bội chung
----------[G] Ước chung lớn nhất
----------[H] Bội chung nhỏ nhất
-------[2] Số nguyên
----------[1] Làm quen với số nguyên âm
----------[2] Tập hợp các số nguyên
----------[3] Thứ tự trong tập hợp các số nguyên
----------[4] Cộng hai số nguyên cùng dấu
----------[5] Cộng hai số nguyên khác dấu
----------[6] Tính chất của phép cộng các số nguyên
----------[7] Phép trừ hai số nguyên
----------[8] Quy tắc dấu ngoặc
----------[9] Quy tắc chuyển vế
----------[0] Nhân hai số nguyên khác dấu
----------[A] Nhân hai số nguyên cùng dấu
----------[B] Tính chất của phép nhân
----------[C] Bội và ước của một số nguyên
-------[3] Phân số
----------[1] Mở rộng khái niệm về phân số
----------[2] Phân số bằng nhau
----------[3] Tính chất cơ bản của phân số
----------[4] Rút gọn phân số
----------[5] Quy đồng mẫu số nhiều phân số
----------[6] So sánh phân số
----------[7] Phép cộng phân số
----------[8] Tính chất cơ bản của phép cộng phân số
----------[9] Phép trừ phân số
----------[0] Phép nhân phân số
----------[A] Tính chất cơ bản của phép nhân phân số
----------[B] Phép chia phân số
----------[C] Hỗn số. Số thập phân. Phần trăm
----------[D] Tìm giá trị phân số của một số cho trước
----------[E] Tìm một số biết giá trị một phân số của nó
----------[F] Tìm tỉ số của hai số
----------[G] Biểu đồ phần trăm
----[H] Hình học
-------[1] Đoạn thẳng
----------[1] Điểm. Đường thẳng
----------[2] Ba điểm thẳng hàng
----------[3] Đường thẳng đi qua hai điểm
----------[4] Thực hành: trồng cây thẳng hàng
----------[5] Tia
----------[6] Đoạn thẳng
----------[7] Độ dài đoạn thẳng
----------[8] Khi nào thì AM + MB = AB?
----------[9] Vẽ đoạn thẳng cho biết độ dài
----------[0] Trung điểm của đoạn thẳng
-------[2] Góc
----------[1] Nửa mặt phẳng
----------[2] Góc
----------[3] Số đo góc
----------[4] Khi nào góc xOy + góc yOz = góc xOz?
----------[5] Vẽ góc cho biết số đo
----------[6] Tia phân giác của góc
----------[7] Thực hành đo góc trên mặt đất
----------[8] Đường tròn
----------[9] Tam giác
-[7] Lớp 7
----[C] Cánh Diều
-------[1] Số hữu tỉ
----------[1] Tập hợp Q các số hữu tỉ
----------[2] Cộng, trừ, nhân, chia số hữu tỉ
----------[3] Phép tính luỹ thừa với số mũ tự nhiên của một số hữu tỉ
----------[4] Thứ tự thực hiện các phép tính. Quy tắc dấu ngoặc
----------[5] Biễu diễn thập phân của số hữu tỉ
-------[2] Số thực
----------[1] Số vô tỉ. Căn bậc hai số học
----------[2] Tập hợp R các số thực
----------[3] Giá trị tuyệt đối của một số thực
----------[4] Làm tròn và ước lượng
----------[5] Tỉ lệ thức
----------[6] Dãy tỉ số bằng nhau
----------[7] Đại lượng tỉ lệ thuận
----------[8] Đại lượng tỉ lệ nghịch
-------[3] Hình học trực quan
----------[1] Hình hộp chữ nhật. Hình lập phương
----------[2] Hình lăng trụ đứng tam giác. Hình lăng trụ đứng tứ giác
-------[4] Góc. Đường thẳng song song
----------[1] Góc ở vị trí đặc biệt
----------[2] Tia phân giác của một góc
----------[3] Hai đường thẳng song song
----------[4] Định lý
-------[5] Một số yếu tố thống kê và xác suất
----------[1] Thu thập, phân loại và biểu diễn dữ liệu
----------[2] Phân tích và xử lí dữ liệu
----------[3] Biểu đồ đoạn thẳng
----------[4] Biểu đồ hình quạt tròn
----------[5] Biến cố trong một số trò chơi đơn giản
----------[6] Xác suất của biến cố ngẫu nhiên trong một số trò chơi đơn giản
-------[6] Biểu thức đại số
----------[1] Biểu thức số. Biểu thức đại số
----------[2] Đa thức một biến. Nghiệm của đa thức một biến
----------[3] Phép cộng, phép trừ đa thức một biến
----------[4] Phép nhân đa thức một biến
----------[5] Phép chia đa thức một biến
-------[7] Tam giác
----------[1] Tổng các góc của một tam giác
----------[2] Quan hệ giữa góc và cạnh đối diện. Bất đẳng thức tam giác
----------[3] Hai tam giác bằng nhau
----------[4] Trường hợp bằng nhau thứ nhất của tam giác: cạnh - cạnh - cạnh
----------[5] Trường hợp bằng nhau thứ hai của tam giác: cạnh - góc - cạnh
----------[6] Trường hợp bằng nhau thứ ba của tam giác: góc - cạnh - góc
----------[7] Tam giác cân
----------[8] Đường vuông góc và đường xiên
----------[9] Đường trung trực của một đoạn thẳng
----------[0] Tính chất ba đường trung tuyến của tam giác
----------[A] Tính chất ba đường phân giác của tam giác
----------[B] Tính chất ba đường trung trực của tam giác
----------[C] Tính chất ba đường cao của tam giác
----[T] Chân Trời Sáng Tạo
-------[1] Số hữu tỉ
----------[1] Tập hợp các số hữu tỉ
----------[2] Các phép tính với số hữu tỉ
----------[3] Lũy thừa của một số hữu tỉ
----------[4] Quy tắc dấu ngoặc và quy tắc chuyển vế
----------[5] Hoạt động thực hành và trải nghiệm: Thực hành tính tiền điện
-------[2] Số thực
----------[1] Số vô tỉ. Căn bậc hai số học
----------[2] Số thực. Giá trị tuyệt đối của một số thực
----------[3] Làm tròn số và ước lượng kết quả
----------[4] Hoạt động thực hành và trải nghiệm: Tính chỉ số đánh giá thể trạng BMI (Body mass index)
-------[3] Các hình khối trong thực tiễn
----------[1] Hình hộp chữ nhật - Hình lập phương
----------[2] Diện tích xung quanh và thể tích của hình hộp chữ nhật, hình lập phương
----------[3] Hình lăng trụ đứng tam giác - Hình lăng trụ đứng tứ giác
----------[4] Diện tích xung quanh và thể tích của hình lăng trụ đứng tam giác, lăng trụ đứng tứ giác
----------[5] Hoạt động thực hành và trải nghiệm: Các bài toán về đo đạc và gấp hình
-------[4] Góc và đường thẳng song song
----------[1] Các góc ở vị trí đặc biệt
----------[2] Tia phân giác
----------[3] Hai đường thẳng song song
----------[4] Định lí và chứng minh một định lí
----------[5] Hoạt động thực hành và trải nghiệm: Vẽ hai đường thẳng song song và đo góc bằng phần mềm GeoGebra
-------[5] Một số yếu tố thống kê
----------[1] Thu thập và phân loại dữ liệu
----------[2] Biểu đồ hình quạt tròn
----------[3] Biểu đồ đoạn thẳng
----------[4] Hoạt động thực hành và trải nghiệm: Dùng biểu đồ để phân tích kết quả học tập môn Toán của lớp
-------[6] Các đại lượng tỉ lệ
----------[1] Tỉ lệ thức - Dãy tỉ số bằng nhau
----------[2] Đại lượng tỉ lệ thuận
----------[3] Đại lượng tỉ lệ nghịch
----------[4] Hoạt động thực hành và trải nghiệm: Các đại lượng tỉ lệ trong thực tế
-------[7] Biểu thức đại số
----------[1] Biểu thức số, biểu thức đại số
----------[2] Đa thức một biến
----------[3] Phép cộng và phép trừ đa thức một biến
----------[4] Phép nhân và phép chia đa thức một biến
----------[5] Hoạt động thực hành và trải nghiệm: Cách tính điểm trung bình môn học kì
-------[8] Tam giác
----------[1] Góc và cạnh của một tam giác
----------[2] Tam giác bằng nhau
----------[3] Tam giác cân
----------[4] Đường vuông góc và đường xiên
----------[5] Đường trung trực của một đoạn thẳng
----------[6] Tính chất ba đường trung trực của tam giác
----------[7] Tính chất ba đường trung tuyến của tam giác
----------[8] Tính chất ba đường cao của tam giác
----------[9] Tính chất ba đường phân giác của tam giác
----------[0] Hoạt động thực hành và trải nghiệm: Làm giàn hoa tam giác để trang trí lớp học
-------[9] Một số yếu tố xác suất
----------[1] Làm quen với biến cố ngẫu nhiên
----------[2] Làm quen với xác suất của biến cố ngẫu nhiên
----[K] Kết Nối Tri Thức
-------[1] Số hữu tỉ
----------[1] Tập hợp các số hữu tỉ
----------[2] Cộng, trừ, nhân, chia số hữu tỉ
----------[3] Lũy thừa với số mũ tự nhiên của một số hữu tỉ
----------[4] Thứ tự thực hiện các phép tính. Quy tắc chuyển vế
-------[2] Số thực
----------[5] Làm quen với số thập phân vô hạn tuần hoàn
----------[6] Số vô tỉ. Căn bậc hai số học
----------[7] Tập hợp các số thực
-------[3] Góc và đường thẳng song song
----------[8] Góc ở vị trí đặc biệt. Tia phân giác của một góc
----------[9] Hai đường thẳng song song và dấu hiệu nhận biết
----------[0] Tiên đề Euclid. Tính chất của hai đường thẳng song song
----------[A] Định lí và chứng minh định lí
-------[4] Tam giác bằng nhau
----------[B] Tổng các góc trong một tam giác
----------[C] Hai tam giác bằng nhau. Trường hợp bằng nhau thứ nhất của tam giác
----------[D] Trường hợp bằng nhau thứ hai và thứ ba của tam giác
----------[E] Các trường hợp bằng nhau của tam giác vuông
----------[F] Tam giác cân. Đường trung trực của đoạn thẳng
-------[5] Thu thập và biểu diễn dữ liệu
----------[G] Thu thập và phân loại dữ liệu
----------[H] Biểu đồ hình quạt tròn
----------[I] Biểu đồ đoạn thẳng
-------[6] Tỉ lệ thức và đại lượng tỉ lệ
----------[J] Tỉ lệ thức
----------[K] Tính chất của dãy tỉ số bằng nhau
----------[L] Đại lượng tỉ lệ thuận
----------[M] Đại lượng tỉ lệ nghịch
-------[7] Biểu thức đại số và đa thức một biến
----------[N] Biểu thức đại số
----------[O] Đa thức một biến
----------[P] Phép cộng và phép trừ đa thức một biến
----------[Q] Phép nhân đa thức một biến
----------[R] Phép chia đa thức một biến
-------[8] Làm quen với biến cố và xác suất của biến cố
----------[S] Làm quen với biến cố
----------[T] Làm quen với xác suất của biến cố
-------[9] Quan hệ giữa các yếu tố trong một tam giác
----------[U] Quan hệ giữa góc và cạnh đối diện trong một tam giác
----------[V] Quan hệ giữa đường vuông góc và đường xiên
----------[W] Quan hệ giữa ba cạnh của một tam giác
----------[X] Sự đồng quy của ba đường trung tuyến, ba đường phân giác trong một tam giác
----------[Y] Sự đồng quy của ba đường trung trực, ba đường cao trong một tam giác
-------[0] Một số hình khối trong thực tiễn
----------[1] Hình hộp chữ nhật và hình lập phương
----------[2] Hình lăng trụ đứng tam giác và hình lăng trụ đứng tứ giác
----[D] Đại số
-------[1] Số hữu tỉ. Số thực
----------[1] Tập hợp Q các số hữu tỉ
----------[2] Cộng, trừ số hữu tỉ
----------[3] Nhân, chia số hữu tỉ
----------[4] Giá trị tuyệt đối của một số hữu tỉ. Cộng, trừ, nhân, chia số thập phân
----------[5] Lũy thừa của một số hữu tỉ
----------[6] Lũy thừa của một số hữu tỉ ( tiếp theo)
----------[7] Tỉ lệ thức
----------[8] Tính chất của dãy tỉ số bằng nhau
----------[9] Số thập phân hữu hạn. Số thập phân vô hạn tuần hoàn
----------[0] Làm tròn số
----------[A] Số vô tỉ. Khái niệm về căn bậc hai
----------[B] Số thực
-------[2] Hàm số và đồ thị
----------[1] Đại lượng tỉ lệ thuận
----------[2] Một số bài toán về đại lượng tỉ lệ thuận
----------[3] Đại lượng tỉ lệ nghịch
----------[4] Một số bài toán về đại lượng tỉ lệ nghịch
----------[5] Hàm số
----------[6] Mặt phẳng toạ độ
----------[7] Đồ thị hàm số y = ax (a # 0)
-------[3] Thống kê
----------[1] Thu thập số liệu thống kê, tần số
----------[2] Bảng tần số các giá trị của dấu hiệu
----------[3] Biểu đồ
----------[4] Số trung bình cộng
-------[4] Biểu thức đại số
----------[1] Khái niệm về biểu thức đại số
----------[2] Giá trị của một biểu thức đại số
----------[3] Đơn thức
----------[4] Đơn thức đồng dạng
----------[5] Đa thức
----------[6] Cộng, trừ đa thức
----------[7] Đa thức một biến
----------[8] Cộng, trừ đa thức một biến
----------[9] Nghiệm của đa thức một biến
----[H] Hình học
-------[1] Đường thẳng vuông góc. Đường thẳng song song
----------[1] Hai góc đối đỉnh
----------[2] Hai đường thẳng vuông góc
----------[3] Các góc tạo bởi một đường thẳng cắt hai đường thẳng
----------[4] Hai đường thẳng song song
----------[5] Tiên đề Ơ-clit về đường thẳng song song
----------[6] Từ vuông góc đến song song
----------[7] Định lí
-------[2] Tam giác
----------[1] Tổng ba góc của một tam giác
----------[2] Hai tam giác bằng nhau
----------[3] Trường hợp bằng nhau thứ nhất của tam giác cạnh - cạnh - cạnh (c.c.c)
----------[4] Trường hợp bằng nhau thứ hai của tam giác cạnh - góc - cạnh (c.g.c)
----------[5] Trường hợp bằng nhau thứ ba của tam giác góc - cạnh - góc (g.c.g)
----------[6] Tam giác cân
----------[7] Định lí Py-ta-go
----------[8] Các trường hợp bằng nhau của tam giác vuông
-------[3] Quan hệ giữa các yểu tố trong tam giác. Các đường đồng quy của tam giác
----------[1] Quan hệ giữa góc và cạnh đối diện trong một tam giác
----------[2] Quan hệ giữa đường vuông góc và đường xiên, đường xiên và hình chiếu
----------[3] Quan hệ giữa ba cạnh của một tam giác, bất đẳng thức tam giác
----------[4] Tính chất ba đường trung tuyến của tam giác
----------[5] Tính chất tia phân giác của một góc
----------[6] Tính chất ba đường phân giác của tam giác
----------[7] Tính chất đường trung trực của một đoạn thẳng
----------[8] Tính chất ba đường trung trực của tam giác
----------[9] Tính chất ba đường cao của tam giác
-[8] Lớp 8
----[C] Cánh Diều
-------[1] Đa thức nhiều biến
----------[1] Đơn thức nhiều biến. Đa thức nhiều biến
----------[2] Các phép tính với đa thức nhiều biến
----------[3] Hằng đẳng thức đáng nhớ
----------[4] Vận dụng hằng đẳng thức vào phân tích đa thức thành nhân tử
-------[2] Phân thức đại số
----------[1] Phân thức đại số
----------[2] Phép cộng, phép trừ phân thức đại số
----------[3] Phép nhân, phép chia phân thức đại số
-------[3] Hàm số và đồ thị
----------[1] Hàm số
----------[2] Mặt phẳng tọa độ. Đồ thị của hàm số
----------[3] Hàm số bậc nhất y = ax + b (a khác 0)
----------[4] Đồ thị của hàm số bậc nhất y = ax + b (a khác 0)
-------[4] Hình học trực quan
----------[1] Hình chóp tam giác đều
----------[2] Hình chóp tứ giác đều
-------[5] Tam giác. Tứ giác
----------[1] Định lí Pythagore
----------[2] Tứ giác
----------[3] Hình thang cân
----------[4] Hình bình hành
----------[5] Hình chữ nhật
----------[6] Hình thoi
----------[7] Hình vuông
-------[6] Một số yếu tố thống kê và xác suất
----------[1] Thu thập và phân loại dữ liệu
----------[2] Mô tả và biểu diễn dữ liệu trên các bảng, biểu đồ
----------[3] Phân tích và xử lí dữ liệu thu được ở dạng bảng, biểu đồ
----------[4] Xác suất của biến cố ngẫu nhiên trong một số trò chơi đơn giản
----------[5] Xác suất thực nghiệm của một biến cố trong một số trò chơi đơn giản
-------[7] Phương trình bậc nhất một ẩn
----------[1] Phương trình bậc nhất một ẩn
----------[2] Ứng dụng của phương trình bậc nhất một ẩn
-------[8] Tam giác đồng dạng. Hình đồng dạng
----------[1] Định lí Thalès trong tam giác
----------[2] Ứng dụng của định lí Thalès trong tam giác
----------[3] Đường trung bình của tam giác
----------[4] Tính chất đường phân giác của tam giác
----------[5] Tam giác đồng dạng
----------[6] Trường hợp đồng dạng thứ nhất của tam giác
----------[7] Trường hợp đồng dạng thứ hai của tam giác
----------[8] Trường hợp đồng dạng thứ ba của tam giác
----------[9] Hình đồng dạng
----------[0] Hình đồng dạng trong thực tiễn
----[T] Chân Trời Sáng Tạo
-------[1] Biểu thức đại số
----------[1] Đơn thức và đa thức nhiều biến
----------[2] Các phép toán với đa thức nhiều biến
----------[3] Hằng đẳng thức đáng nhớ
----------[4] Phân tích đa thức thành nhân tử
----------[5] Phân thức đại số
----------[6] Cộng, trừ phân thức
----------[7] Nhân, chia phân thức
-------[2] Các hình khối trong thực tiễn
----------[1] Hình chóp tam giác đều – Hình chóp tứ giác đều
----------[2] Diện tích xung quanh và thể tích của hình chóp tam giác đều, hình chóp tứ giác đều
-------[3] Định lí Pythagore. Các loại tứ giác thường gặp
----------[1] Định lí Pythagore
----------[2] Tứ giác
----------[3] Hình thang – Hình thang cân
----------[4] Hình bình hành – Hình thoi
----------[5] Hình chữ nhật – Hình vuông
-------[4] Một số yếu tố thống kê
----------[1] Thu thập và phân loại dữ liệu
----------[2] Lựa chọn dạng biểu đồ để biểu diễn dữ liệu
----------[3] Phân tích dữ liệu
-------[5] Hàm số và đồ thị
----------[1] Khái niệm hàm số
----------[2] Toạ độ của một điểm và đồ thị của hàm số
----------[3] Hàm số bậc nhất y = ax + b (a khác 0)
----------[4] Hệ số góc của đường thẳng
-------[6] Phương trình
----------[1] Phương trình bậc nhất một ẩn
----------[2] Giải bài toán bằng cách lập phương trình bậc nhất
-------[7] Định lí Thalès
----------[1] Định lí Thalès trong tam giác
----------[2] Đường trung bình của tam giác
----------[3] Tính chất đường phân giác của tam giác
-------[8] Hình đồng dạng
----------[1] Hai tam giác đồng dạng
----------[2] Các trường hợp đồng dạng của hai tam giác
----------[3] Các trường hợp đồng dạng của hai tam giác vuông
----------[4] Hai hình đồng dạng
-------[9] Một số yếu tố xác suất
----------[1] Mô tả xác suất bằng tỉ số
----------[2] Xác suất lí thuyết và xác suất thực nghiệm
----[K] Kết Nối Tri Thức
-------[1] Đa thức
----------[1] Đơn thức
----------[2] Đa thức
----------[3] Phép cộng và phép trừ đa thức
----------[4] Phép nhân đa thức
----------[5] Phép chia đa thức cho đơn thức
-------[2] Hằng đẳng thức đáng nhớ và ứng dụng
----------[6] Hiệu hai bình phương. Bình phương của một tổng hay một hiệu
----------[7] Lập phương của một tổng hay một hiệu
----------[8] Tổng và hiệu hai lập hương
----------[9] Phân tích đa thức thành nhân tử
-------[3] Tứ giác
----------[0] Tứ giác
----------[A] Hình thang cân
----------[B] Hình bình hành
----------[C] Hình chữ nhật
----------[D] Hình thoi và hình vuông
-------[4] Định lí Thalès
----------[E] Định lí Thalès trong tam giác
----------[F] Đường trung bình của tam giác
----------[G] Tính chất đường phân giác của tam giác
-------[5] Dữ liệu và biểu đồ
----------[H] Thu thập và phân loại dữ liệu
----------[I] Biểu diễn dữ liệu bằng bảng, biểu đồ
----------[J] Phân tích số liệu thống kê dựa vào biểu đồ
-------[6] Phân thức đại số
----------[K] Phân thức đại số
----------[L] Tính chất cơ bản của phân thức đại số
----------[M] Phép cộng và phép trừ phân thức đại số
----------[N] Phép nhân và phép chia phân thức đại số
-------[7] Phương trình bậc nhất và hàm số bậc nhất
----------[O] Phương trình bậc nhất một ẩn
----------[P] Giải bài toán bằng cách lập phương trình
----------[Q] Khái niệm hàm số và đồ thị của hàm số
----------[R] Hàm số bậc nhất và đồ thị của hàm số bậc nhất
----------[S] Hệ số góc của đường thẳng
-------[8] Mở đầu về tính xác suất của biến cố
----------[T] Kết quả có thể và kết quả thuận lợi
----------[U] Cách tính xác suất của biến cố bằng tỉ số
----------[V] Mối liên hệ giữa xác suất thực nghiệm với xác suất và ứng dụng
-------[9] Tam giác đồng dạng
----------[W] Hai tam giác đồng dạng
----------[X] Ba trường hợp đồng dạng của hai tam giác
----------[Y] Định lí Pythagore và ứng dụng
----------[Z] Các trường hợp đồng dạng của hai tam giác vuông
----------[1] Hình đồng dạng
-------[0] Một số hình khối trong thực tiễn
----------[2] Hình chóp tam giác đều
----------[3] Hình chóp tứ giác đều
----[D] Đại số
-------[1] Phép nhân và phép chia đa thức
----------[1] Nhân đơn thức với đa thức
----------[2] Nhân đa thức với đa thức
----------[3] Những hằng đẳng thức đáng nhớ
----------[4] Những hằng đẳng thức đáng nhớ (tiếp)
----------[5] Những hằng đẳng thức đáng nhớ (tiếp)
----------[6] Phân tích đa thức thành nhân tử bằng phương pháp đặt nhân tử chung
----------[7] Phân tích đa thức thành nhân tử bằng phương pháp dùng hằng đẳng thức
----------[8] Phân tích đa thức thành nhân tử bằng phương pháp nhóm hạng tử
----------[9] Phân tích đa thức thành nhân tử bằng cách phối hợp nhiều phương pháp
----------[0] Chia đơn thức cho đơn thức
----------[A] Chia đa thức cho đơn thức
----------[B] Chia đa thức một biến đã sắp xếp
-------[2] Phân thức đại số
----------[1] Phân thức đại số
----------[2] Tính chất cơ bản của phân thức
----------[3] Rút gọn phân thức
----------[4] Quy đồng mẫu thức nhiều phân thức
----------[5] Phép cộng các phân thức đại số
----------[6] Phép trừ các phân thức đại số
----------[7] Phép nhân các phân thức đại số
----------[8] Phép chia các phân thức đại số
----------[9] Biến đổi các biểu thức hữu tỉ. Giá trị của phân thức
-------[3] Phương trình bậc nhất một ẩn
----------[1] Mở đầu về phương trình
----------[2] Phương trình bậc nhất một ẩn và cách giải
----------[3] Phương trình đưa được về dạng ax + b = 0
----------[4] Phương trình tích
----------[5] Phương trình chứa ẩn ở mẫu
----------[6] Giải bài toán bằng cách lập phương trình
----------[7] Giải bài toán bằng cách lập phương trình (tiếp)
-------[4] Bất phương trình bậc nhất một ẩn
----------[1] Liên hệ giữa thứ tự và phép cộng
----------[2] Liên hệ giữa thứ tự và phép nhân
----------[3] Bất phương trình một ẩn
----------[4] Bất phương trình bậc nhất một ẩn
----------[5] Phương trình chứa dấu giá trị tuyệt đối
----[H] Hình học
-------[1] Tứ giác
----------[1] Tứ giác
----------[2] Hình thang
----------[3] Hình thang cân
----------[4] Đường trung bình của tam giác, của hình thang
----------[5] Dựng hình bằng thước và compa. Dựng hình thang
----------[6] Đối xứng trục
----------[7] Hình bình hành
----------[8] Đối xứng tâm
----------[9] Hình chữ nhật
----------[0] Đường thẳng song song với một đường thẳng cho trước
----------[A] Hình thoi
----------[B] Hình vuông
-------[2] Đa giác, diện tích đa giác
----------[1] Đa giác. Đa giác đều
----------[2] Diện tích hình chữ nhật
----------[3] Diện tích tam giác
----------[4] Diện tích hình thang
----------[5] Diện tích hình thoi
----------[6] Diện tích đa giác]
-------[3] Tam giác đồng dạng
----------[1] Định lí Ta-lét trong tam giác
----------[2] Định lí đảo và hệ quả của định lí Ta-lét
----------[3] Tính chất đường phân giác của tam giác
----------[4] Khái niệm hai tam giác đồng dạng
----------[5] Trường hợp đồng dạng thứ nhất
----------[6] Trường hợp đồng dạng thứ hai
----------[7] Trường hợp đồng dạng thứ ba
----------[8] Các trường hợp đồng dạng của tam giác vuông
----------[9] Ứng dụng thực tế của tam giác đồng dạng
-------[4] Hình lăng trụ đứng. Hình chóp đều
----------[1] Hình hộp chữ nhật
----------[2] Hình hộp chữ nhật (tiếp)
----------[3] Thể tích của hình hộp chữ nhật
----------[4] Hình lăng trụ đứng
----------[5] Diện tích xung quanh của hình lăng trụ đứng
----------[6] Thể tích của hình lăng trụ đứng
----------[7] Hình chóp đều và hình chóp cụt đều
----------[8] Diện tích xung quanh của hình chóp
----------[9] Thể tích của hình chóp đều
-[9] Lớp 9
----[D] Đại số
-------[1] Căn bậc hai. Căn bậc ba
----------[1] Căn bậc hai
----------[2] Căn thức bậc hai và hằng đẳng thức
----------[3] Liên hệ giữa phép nhân và phép khai phương
----------[4] Liên hệ giữa phép chia và phép khai phương
----------[5] Bảng căn bậc hai
----------[6] Biến đổi đơn giản biểu thức chứa căn thức bậc hai
----------[7] Biến đổi đơn giản biểu thức chứa căn thức bậc hai (tiếp theo)
----------[8] Rút gọn biểu thức chứa căn bậc hai
----------[9] Căn bậc ba
-------[2] Hàm số bậc nhất
----------[1] Nhắc lại và bổ sung các khái niệm về hàm số
----------[2] Hàm số bậc nhất
----------[3] Đồ thị của hàm số y = ax + b (a ≠ 0)
----------[4] Đường thẳng song song và đường thẳng cắt nhau
----------[5] Hệ số góc của đường thẳng y = ax + b (a ≠ 0)
-------[3] Hệ hai phương trình bậc nhất hai ẩn
----------[1] Phương trình bậc nhất hai ẩn
----------[2] Hệ hai phương trình bậc nhất hai ẩn
----------[3] Giải hệ phương trình bằng phương pháp thế
----------[4] Giải hệ phương trình bằng phương pháp cộng đại số
----------[5] Giải bài toán bằng cách lập hệ phương trình
----------[6] Giải bài toán bằng cách lập hệ phương trình (tiếp theo)
-------[4] Hàm số y = ax^2 (a ≠ 0). Phương trình bậc hai một ẩn
----------[1] Hàm số y = ax^2 (a ≠ 0)
----------[2] Đồ thị của hàm số y = ax^2 (a ≠ 0)
----------[3] Phương trình bậc hai một ẩn
----------[4] Công thức nghiệm của phương trình bậc hai
----------[5] Công thức nghiệm thu gọn
----------[6] Hệ thức vi-ét và ứng dụng
----------[7] Phương trình quy về phương trinh bậc hai
----------[8] Giải bài toán bằng cách lập phương trình
----[H] Hình học
-------[1] Hệ thức lượng trong tam giác vuông
----------[1] Một số hệ thức về cạnh và đường cao trong tam giác vuông
----------[2] Tỉ số lượng giác của góc nhọn
----------[3] Bảng lượng giác
----------[4] Một số hệ thức về cạnh và góc trong tam giác vuông
----------[5] Ứng dụng thực tế các tỉ số lượng giác của góc nhọn. Thực hành ngoài trời
-------[2] Đường tròn
----------[1] Sự xác định của đường tròn. Tính chất đối xứng của đường tròn
----------[2] Đường kính và dây của đường tròn
----------[3] Liên hệ giữa dây và khoảng cách từ tâm đến dây
----------[4] Vị trí tương đối của đường thẳng và đường tròn
----------[5] Dấu hiệu nhận biết tiếp tuyến của đường tròn
----------[6] Tính chất của hai tiếp tuyến cắt nhau
----------[7] Vị trí tương đối của hai đường tròn
----------[8] Vị trí tương đối của hai đường tròn (tiếp theo)
-------[3] Góc với đường tròn
----------[1] Góc ở tâm. Số đo cung
----------[2] Liên hệ giữa cung và dây
----------[3] Góc nội tiếp
----------[4] Góc tạo bởi tia tiếp tuyến và dây cung
----------[5] Góc có đỉnh ở bên trong đường tròn. Góc có đỉnh ở bên ngoài đường tròn
----------[6] Cung chứa góc
----------[7] Tứ giác nội tiếp
----------[8] Đường tròn ngoại tiếp. Đường tròn nội tiếp
----------[9] Độ dài đường tròn, cung tròn
----------[0] Diện tích hình tròn, hình quạt tròn
-------[4] Hình trụ - hình nón - hình cầu
----------[1] Hình trụ - diện tích xung quanh và thể tích hình trụ
----------[2] Hình nón - hình nón cụt. Diện tích xung quanh và thể tích của hình nón, hình nón cụt
----------[3] Hình cầu. Diện tích hình cầu và thể tích hình cầu
%Kết thúc nội dung ID