%%%=============BT_1=============%%%
\begin{bt}
	Cho các giá trị năng lượng liên kết trung bình ở điều kiện chuẩn $25^\circ C$, $1$ bar (kJ/liên kết):
	\begin{center}
		\begin{tabular}{|c|c|c|c|c|c|}
			\hline
			Liên kết & C-H & C=O & O-H & H-H & O=O\\
			\hline
			Năng lượng & 413 & 745 & 467 & 432 & 495\\
			\hline
		\end{tabular}
	\end{center}
	Tính hiệu ứng nhiệt của phản ứng đốt cháy $1$ mol khí methane.
	\loigiai{
		\\
		Phản ứng đốt cháy 1 mol khí methane:
		\\
		$CH_4 (g) + 2O_2 (g) \longrightarrow CO_2 (g) + 2H_2O (l)$
		\\
		Trong phản ứng:
		\\
		- Cần phá vỡ $4$ liên kết C-H và $2$ liên kết O=O. Năng lượng cần cung cấp là: $4 \times 413 + 2 \times 495 = 2642$ (kJ)
		\\
		- Hình thành $2$ liên kết C=O và $4$ liên kết O-H. Năng lượng tỏa ra là: $2 \times 745 + 4 \times 467 = 3374$ (kJ)
		\\
		Hiệu ứng nhiệt của phản ứng là: $-3374 + 2642 = -732$ (kJ).
		\\
		Vậy hiệu ứng nhiệt của phản ứng đốt cháy $1$ mol khí methane là $-732$ kJ.
	}
\end{bt}
%%%=============BT_2=============%%%
\begin{bt}
	Tính nhiệt lượng tỏa ra (theo đơn vị kJ) khi đốt cháy hoàn toàn $1$ kg than đá có chứa $90\%$ carbon, biết than đá cháy trong không khí sinh ra khí $CO_2$ và giá trị enthalpy tạo thành chuẩn của $CO_2$ là  $-393,5$ kJ/mol.
	\loigiai{
		\\
		Phản ứng hóa học xảy ra khi than đá cháy trong không khí:
		\\
		$C(s) + O_2(g) \longrightarrow CO_2(g)$ $\Delta _f H_{298}^\circ = -393,5$ kJ/mol
		\\
		Khối lượng carbon trong $1$ kg than đá:
		\\
		$m_C = 1000 \times 90\% = 900$ (g)
		\\
		Số mol carbon trong $1$ kg than đá:
		\\
		$n_C = \dfrac{m_C}{M_C} = \dfrac{900}{12} = 75$ (mol)
		\\
		Nhiệt lượng tỏa ra khi đốt cháy hoàn toàn $1$ kg than đá:
		\\
		$Q = n_C \times \Delta _f H_{298}^\circ(CO_2) = 75 \times (-393,5) = -29512,5$ (kJ)
		\\
		Vậy nhiệt lượng tỏa ra khi đốt cháy hoàn toàn $1$ kg than đá là $29512,5$ kJ.
	}
\end{bt}
%%%=============BT_3=============%%%
\begin{bt}
	Trong quá trình sản xuất axit nitric, giai đoạn oxi hóa khí ammonia được thực hiện theo phương trình hóa học (1):
	\[(1)\hspace{1cm}4NH_3(g) + 5O_2(g) \xrightarrow[Pt, 800^{\circ}C][] 4NO(g) + 6H_2O(g)\]
	Phản ứng (1) là phản ứng tỏa nhiệt. Một phần nhiệt tỏa ra được sử dụng để gia nhiệt cho phản ứng tổng hợp ammonia từ nitrogen và hydrogen theo phương trình (2):
	\[(2)\hspace{1cm}N_2(g) + 3H_2(g) \rightleftharpoons 2NH_3(g)\]
	Xét các phản ứng ở điều kiện chuẩn và hiệu suất chuyển hóa của ammonia trong phản ứng (1) là 95%. Tính khối lượng ammonia (tính theo kg, làm tròn đến hàng đơn vị) cần thiết để sản xuất 126 kg NO(g) trong giai đoạn trên. Biết 80% lượng nhiệt tỏa ra từ phản ứng (1) được cung cấp cho phản ứng (2) và các giá trị nhiệt tạo thành $\left(\Delta_f H_{298}^\circ \right)$ của các chất ở điều kiện chuẩn được cho trong bảng sau:
	\begin{center}
		\begin{tabular}{|c|c|c|c|}
			\hline
			Chất    & $NH_3(g)$    & $NO(g)$    & $H_2O(g)$ \\
			\hline
			$\Delta_fH_{298}^{\circ}\left(\mathrm{kJ} \mathrm{mol}^{-1}\right)$    & $-46,1$    & $90,3$    & $-241,8$ \\
			\hline
		\end{tabular}
	\end{center}
	\loigiai{
		Bước 1: Tính số mol NO cần sản xuất
		\[n_{NO} = \frac{m_{NO}}{M_{NO}} = \frac{126}{30} = 4,2 \text{ mol}\]
		
		Bước 2: Tính số mol NH_3 cần thiết (theo phương trình)
		\[n_{NH_3} = n_{NO} = 4,2 \text{ mol}\]
		
		Bước 3: Tính số mol NH_3 thực tế cần dùng (do hiệu suất 95%)
		\[n_{NH_3 \text{ thực tế}} = \frac{n_{NH_3}}{0,95} = \frac{4,2}{0,95} = 4,42 \text{ mol}\]
		
		Bước 4: Tính enthalpy của phản ứng (1)
		\[\Delta H_1 = 4 \times 90,3 + 6 \times (-241,8) - 4 \times (-46,1) = -904,5 \text{ kJ}\]
		
		Bước 5: Tính khối lượng NH_3 cần dùng
		\[m_{NH_3} = n_{NH_3 \text{ thực tế}} \times M_{NH_3} = 4,42 \times 17 = 75,14 \text{ g} \approx 75 \text{ g} = 0,075 \text{ kg}\]
		
		Vậy, cần 75 kg NH_3 để sản xuất 126 kg NO.
	}
\end{bt}
%%%=============BT_4=============%%%
\begin{bt}
	Trong quá trình sản xuất axit sulfuric, giai đoạn oxi hóa sulfur dioxide thành sulfur trioxide được thực hiện theo phương trình hóa học (1):
	\[(1)\hspace{1cm}2SO_2(g) + O_2(g) \rightleftharpoons 2SO_3(g)\]
	Phản ứng (1) là phản ứng tỏa nhiệt. Một phần nhiệt tỏa ra được sử dụng để gia nhiệt cho phản ứng tổng hợp sulfur dioxide từ sulfur và oxygen theo phương trình (2):
	\[(2)\hspace{1cm}S(s) + O_2(g) \rightarrow SO_2(g)\]
	Xét các phản ứng ở điều kiện chuẩn và hiệu suất chuyển hóa của SO_2 trong phản ứng (1) là 98%. Tính thể tích O_2 (đo ở đktc, tính theo lít, làm tròn đến hàng thập phân) cần thiết để sản xuất 160 g SO_3(g) trong giai đoạn trên. Biết 85% lượng nhiệt tỏa ra từ phản ứng (1) được cung cấp cho phản ứng (2) và các giá trị nhiệt tạo thành $\left(\Delta_f H_{298}^\circ \right)$ của các chất ở điều kiện chuẩn được cho trong bảng sau:
	\begin{center}
		\begin{tabular}{|c|c|c|}
			\hline
			Chất    & $SO_2(g)$    & $SO_3(g)$ \\
			\hline
			$\Delta_fH_{298}^{\circ}\left(\mathrm{kJ} \mathrm{mol}^{-1}\right)$    & $-296,8$    & $-395,7$ \\
			\hline
		\end{tabular}
	\end{center}
	\loigiai{
		Bước 1: Tính số mol SO_3 cần sản xuất
		\[n_{SO_3} = \frac{m_{SO_3}}{M_{SO_3}} = \frac{160}{80} = 2 \text{ mol}\]
		
		Bước 2: Tính số mol O_2 cần thiết (theo phương trình)
		\[n_{O_2} = \frac{1}{2} n_{SO_3} = 1 \text{ mol}\]
		
		Bước 3: Tính số mol O_2 thực tế cần dùng (do hiệu suất 98%)
		\[n_{O_2 \text{ thực tế}} = \frac{n_{O_2}}{0,98} = \frac{1}{0,98} = 1,02 \text{ mol}\]
		
		Bước 4: Tính enthalpy của phản ứng (1)
		\[\Delta H_1 = 2 \times (-395,7) - 2 \times (-296,8) = -197,8 \text{ kJ}\]
		
		Bước 5: Tính thể tích O_2 cần dùng (ở đktc)
		\[V_{O_2} = n_{O_2 \text{ thực tế}} \times V_m = 1,02 \times 22,4 = 22,85 \text{ L} \approx 22,9 \text{ L}\]
		
		Vậy, cần 22,9 L O_2 (đo ở đktc) để sản xuất 160 g SO_3.
	}
\end{bt}