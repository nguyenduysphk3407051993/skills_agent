\begin{dangntd}{Phản ứng este hóa}
	\nhanmanh{Bài toán 1: Tính hiệu suất phản ứng este hóa}
	\begin{ntdppg}{Phương pháp giải}
		\begin{enumerate}[label=\itshape\sffamily{Bước \arabic*},leftmargin=6pt,wide=3pt]
		\item Xác định hiệu suất tính theo chất nào?
		\item Tính $ n_{\text{este lý thuyết}} $ (tính theo phương trình hóa học)
		\item Tính hiệu suất
		\tcbox{$ H = \dfrac{ m_{\text{este thực tế}} }{m_{\text{este lý thuyết}}}\cdot 100 \% = \dfrac{ n_{\text{este thực tế}} }{n_{\text{este lý thuyết}}}\cdot 100 \%$}
		\end{enumerate}
	\end{ntdppg}
\begin{notegsnd}
	$ n_{\text{thực tế}} $ và $ m_{\text{thực tế}} $ đề bài cho
\end{notegsnd}
\nhanmanh{Bài toán 2: Tính toán lượng chất khi biết hiệu suất phản ứng}
\begin{ntdppg}{Phương pháp giải}
	\begin{enumerate}[label=\itshape\sffamily{Bước \arabic*},leftmargin=6pt,wide=3pt]
		\item Tính theo hiệu suát $ 100 \% $
		\item Tính toán theo yêu cầu
		\tcbox} : H\% $
	}
		
	\tcbox} x H\% $
		}
		
		
		
	\end{enumerate}
\end{ntdppg}
\end{dangntd}
\begin{vdm}{Ví dụ mẫu}
\end{vdm}
%%%%%%%%%%%%%%%%%%%%%%%%%%%%%%%%Bắt đầu ví dụ 1%%%%%%%%%%%%%%%%%%%%%%%%%%%%%%%%%%%%%%%%%%%%%
\begin{vdex}[1][Tính hiệu suất phản ứng este hóa]
	Đun $ 3.0~\mathrm {gam} CH_3COOH $ với $ C_2H_5OH $ dư (xúc tác $ H_2SO_4 $ đặc), thu được $ 2.2~\mathrm{gam} $ $ CH_3COOC_2H_5 .$ Hiệu suất của phản ứng este hóa là
	\choice
	{%
		\True	$ 50 \% $
	}
	{%
		$ 25 \% $
	}
	{%
		$ 36,67 \% $
	}
	{%
		$ 20,75 \% $
	}
	\huongdan
	{%
	Do \chemfig{C_2H_5OH} dư nên hiệu suất phản ứng este hóa tính theo axit.\\
		$ n_{CH_3COOH}=0.05~\mathrm{mol} $;  $n_{CH_3COOC_2H_5}=0.025~\mathrm{mol} $.\\
	Phương trình phản ứng:\\
	\schemestart
	\chemfig{CH_3COOH}
	\+
	\chemfig{C_2H_5OH}
	\arrow{<=>[\scriptsize\chemfig{H_2SO_4}đặc][][]}[,.8,,,]
	\chemfig{CH_3COOC_2H_5}
	\+
	\chemfig{H_2O}
	\schemestop\\
Theo phương trình ta có: $ n_{\scriptsize\chemfig{CH_3COOC_2H_5}\text{theo lý thuyết}} =n_{\scriptsize\chemfig{CH_3COOH}}=0,05 ~\mathrm{mol}$	\\
$\Rightarrow$ $ m_{\scriptsize\chemfig{CH_3COOC_2H_5}\text{theo lý thuyết}} =0,05\cdot88=4,4~\mathrm{gam}$	\\	
Hiệu suất phản ứng este hóa trên là:
$ H=\dfrac{m_{\text{thực tế}}}{m_{\text{lý thuyết}}}\cdot100\%=\dfrac{2,2}{4,4}\cdot100\%=50\% $
	}
\end{vdex}
%%%%%%%%%%%%%%%%%%%%%%%%%%%%%%%%Kết thúc ví dụ 1%%%%%%%%%%%%%%%%%%%%%%%%%%%%%%%%%%%%%%%
%%%%%%%%%%%%%%%%%%%%%%%%%%%%%%%%Bắt đầu ví dụ 2%%%%%%%%%%%%%%%%%%%%%%%%%%%%%%%%%%%%%%%
\begin{vdex}[1][Tính hiệu suất phản ứng este hóa]
		Cho $ 45~\mathrm{gam} $ axit axetic phản ứng với $ 69~\mathrm{gam} $ etanol (có $ H_2SO_4 $ đặc làm chất xúc tác) đun nóng, thu được $ 41,25~\mathrm{gam} $ este. Hiệu suất của phản ứng este hóa là:
	\choice
	{%
		$ 31,25 \% $
	}
	{%
		$ 40,00 \% $
	}
	{%
		$ 50,00 \% $
	}
	{%
	\True	$ 62,50 \% $
	}
	\huongdan
	{%
	Ta có:	$ n_{CH_3COOH}=0.75~\mathrm{mol} $;  $n_{C_2H_5OH}=1,5~\mathrm{mol} $.\\
		Phương trình phản ứng:\\
		\schemestart
		\chemfig{CH_3COOH}
		\+
		\chemfig{C_2H_5OH}
		\arrow{<=>[\scriptsize\chemfig{H_2SO_4}đặc][][]}[,.8,,,]
		\chemfig{CH_3COOC_2H_5}
		\+
		\chemfig{H_2O}
		\schemestop\\
		
		Ta thấy: $ \dfrac{0,75}{1} < \dfrac{1,5}{1}$ $\Rightarrow$ Hiệu suất tính theo axit.\\
		Theo phương trình ta có: $ n_{\scriptsize\chemfig{CH_3COOC_2H_5}\text{theo lý thuyết}} =n_{\scriptsize\chemfig{CH_3COOH}}=0,75 ~\mathrm{mol}$	\\
		$\Rightarrow$ $ m_{\scriptsize\chemfig{CH_3COOC_2H_5}\text{theo lý thuyết}} =0,75\cdot88=66~\mathrm{gam}$	\\	
		Hiệu suất phản ứng este hóa trên là:
		$ H=\dfrac{m_{\text{thực tế}}}{m_{\text{lý thuyết}}}\cdot100\%=\dfrac{41,25}{66}\cdot100\%=62,5\% $
		
	}
\end{vdex}
%%%%%%%%%%%%%%%%%%%%%%%%%%%%%%%% Kết thúc ví dụ 2 %%%%%%%%%%%%%%%%%%%%%%%%%%%%%%%%%%%%%%%%%%%%%%%%%%%%%%
\begin{vdm}{Ví dụ mẫu}
\end{vdm}
%%%%%%%%%%%%%%%%%%%%%%%%%%%%%%%% Bắt đầu ví dụ 3 %%%%%%%%%%%%%%%%%%%%%%%%%%%%%%%%%%%%%%%%%%%%%%%%%%%%%%%
\begin{vdex}[1][Tính lượng chất khi biết hiệu suất]
	Thực hiện phản ứng este hóa $ m~\mathrm{gam} $ \chemfig{CH_3COOH} bằng \chemfig{C_2H_5OH} thu được $ 0,02~\mathrm{mol} $ este. Biết hiệu suất phản ứng đuọc tính theo axit bằng $ 80 \% $.Giá trị của m là:
	\choice
	{%
		$ 2,00  $
	}
	{%
		$ 0,72 $
	}
	{%
		$ 1,20 $
	}
	{%
		\True $1,50 $
	}
	\huongdan
	{%
	Phương trình hóa học:
	
	\schemestart
	\chemfig{CH_3COOH}
	\+
	\chemfig{C_2H_5OH}
	\arrow{<=>[\scriptsize\chemfig{H_2SO_4}đặc][][]}[,.8,,,]
	\chemfig{CH_3COOC_2H_5}
	\+
	\chemfig{H_2O}
	\schemestop\\
	Coi $ H=100 \% $	thì $ n_{CH_3COOH} =n_{este} =0,02~\mathrm{mol} $\\
	$ \Rightarrow m_{CH_3COOH} = 0,02 \cdot 60 = 1,2~\mathrm {gam}  $\\
	Với $ H= 80 \% $ thì $ m_{CH_3COOH ~\text{ban đầu}} = 1,2 : 80 \% =1,5~\mathrm{gam} $
	}
\end{vdex}
%%%%%%%%%%%%%%%%%%%%%%%%%%%%%%%% Kết thúc ví dụ 3 %%%%%%%%%%%%%%%%%%%%%%%%%%%%%%%%%%%%%%%%%%%%%%%%%%%%%%
%%%%%%%%%%%%%%%%%%%%%%%%%%%%%%%% Bắt đầu ví dụ 4 %%%%%%%%%%%%%%%%%%%%%%%%%%%%%%%%%%%%%%%%%%%%%%%%%%%%%%%
\begin{vdex}[2][Tính lượng chất khi biết hiệu suất]
	Đun nóng $ 6,0~\mathrm{gam} $ $ CH_3COOH $ với $ 6,0~\mathrm{gam} $ \chemfig{C_2H_5OH} ( có $ H_2SO_4 $ làm chất xúc tác, hiệu suất phản ứng este hóa bằng 50 \%). Khối lượng este tạo thành là:
	\choice
	{%
		$ 6,0~\mathrm{gam}  $
	}
	{%
		\True	$ 4,4~\mathrm{gam}  $
	}
	{%
			$ 8,8~\mathrm{gam}  $
	}
	{%
			$ 5,2~\mathrm{gam}  $
	}
	\huongdan
	{%
		Phương trình hóa học:
		
		\begin{tabular}{*8{c}}	
\chemfig{CH_3COOH}& $ + $ & \chemfig{C_2H_5OH}&    

\begin{tikzpicture}[declare function={d=2cm;}]
	\path (0,0) coordinate (A)
	(d,0) coordinate (B)
	(d,-0.03) coordinate (C)
	(0,-0.05) coordinate (D)
	(D)--(C)--([turn]0:3pt) coordinate (E)
	(E)--(B)--([turn]0:5pt) coordinate (F)
	(B)--(A)--([turn]0:3pt) coordinate (Et)
	(Et)--(D)--([turn]0:3pt) coordinate (Ft)
	;
	\tikzset{%
		
		muiten/.pic={%
			\begin{scope}[transform canvas={yshift=-1pt}]
				
				\path (D)--(C) node[sloped,yshift=-5pt,pos=.5] {\scriptsize{\chemfig{H_2SO_4}đặc}}
				;
				\fill[dnvang!70!black]
				(A) rectangle (C)
				(Et)--(Ft)--(A)--cycle
				;
			\end{scope}
			
			\begin{scope}[transform canvas={yshift=1pt}]
				\path (A)--(B) node[sloped,yshift=5pt,pos=.5] {$ t^\circ $};
				\fill[dnvang!70!black]
				(A) rectangle (C)
				(E)--(F)--(C)--cycle
				;
			\end{scope}
		}
	}     
	\path pic [local bounding box=A1] at (0,0) {muiten};
\end{tikzpicture}
& \chemfig{CH_3COOC_2H_5} & $ + $ &  \chemfig{H_2O}  &   \\
$ 0.1~\mathrm{mol} $ &    &   $ 0.13~\mathrm{mol} $  &    &     &     &      & $ \mathrm{mol} $ \\

		\end{tabular}\\
Ta thấy: $ \dfrac{0,1}{1}<\dfrac{0,13}{1} $ $ \Rightarrow $ Tính theo số mol \chemfig{CH_3COOH}.\\
Coi $ H=100 \% $	thì $ n_{CH_3COOC_2H_5} =n_{CH_3COOH} =0,1~\mathrm{mol} $\\
$ \Rightarrow m_{CH_3COOC_2H_5} = 0.1 \cdot 88 = 8.8~\mathrm {gam}  $\\
Với $ H= 50 \% $ thì $ m_{\text{este thu được}} = 8.8 \cdot 50 \% =4,4~\mathrm{gam} $

	}
\end{vdex}
%%%%%%%%%%%%%%%%%%%%%%%%%%%%%%%% Kết thúc ví dụ 4 %%%%%%%%%%%%%%%%%%%%%%%%%%%%%%%%%%%%%%%%%%%%%%%%%%%%%%
%%%%%%%%%%%%%%%%%%%%%%%%%%%%%%%% Bắt đầu ví dụ 5 %%%%%%%%%%%%%%%%%%%%%%%%%%%%%%%%%%%%%%%%%%%%%%%%%%%%%%%
\begin{bttl}{BÀI TẬP TỰ LUYỆN}

\end{bttl}

%%%%%%%%%%%%%%%%%%%%%%%%%%%%%%%% Bắt đầu câu 1 %%%%%%%%%%%%%%%%%%%%%%%%%%%%%%%%%%%%%%%%%%%%%%%%%%%%%%%
\Opensolutionfile{ans}[DAPAN/BTTL03]


\begin{ex}[2][Tính hiệu suất phản ứng este hóa]
	Đun nóng  $ 24~\mathrm{gam} $ axit axetic với lượng dư ancol etylic ( xúc tác $ H_2SO_4 $ đặc), thu được $ 26,4~\mathrm{gam} $ este.Hiệu suất của phản ứng este hóa là:
	\choice
{%
	$ 44 \% $
}
{%
	$ 55 \% $
}
{%
	$ 60 \% $
}
{%
\True	$ 75 \% $
}
\sodongkeex[4]
	\loigiai
	{%
		
}
\end{ex}
%%%%%%%%%%%%%%%%%%%%%%%%%%%%%%%% Kết thúc câu 1 %%%%%%%%%%%%%%%%%%%%%%%%%%%%%%%%%%%%%%%%%%%%%%%%%%%%%%
%%%%%%%%%%%%%%%%%%%%%%%%%%%%%%%% Bắt đầu câu 2 %%%%%%%%%%%%%%%%%%%%%%%%%%%%%%%%%%%%%%%%%%%%%%%%%%%%%%%
\begin{ex}[2][Tính lượng chất khi biết hiệu suất phản ứng]
	Đun nóng  $ m~\mathrm{gam} $ axit axetic với lượng dư ancol etylic ( xúc tác $ H_2SO_4 $ đặc), thu được $ 44~\mathrm{gam} $ este.(Biết hiệu suất của phản ứng este hóa là:$ 80 \% $). Tìm giá trị của $ m $
	\choice
	{%
		$ 35,2  $
	}
	{%
		\True $ 55  $
	}
	{%
		$ 60  $
	}
	{%
		$ 75  $
	}
	\sodongkeex[4]
	\loigiai
	{%
		
	}
\end{ex}
%%%%%%%%%%%%%%%%%%%%%%%%%%%%%%%% Kết thúc câu 2 %%%%%%%%%%%%%%%%%%%%%%%%%%%%%%%%%%%%%%%%%%%%%%%%%%%%%%
%%%%%%%%%%%%%%%%%%%%%%%%%%%%%%%% Bắt đầu câu 3 %%%%%%%%%%%%%%%%%%%%%%%%%%%%%%%%%%%%%%%%%%%%%%%%%%%%%%%
\begin{ex}[2][Tính lượng chất khi biết hiệu suất phản ứng]
	Đun nóng  $ 240~\mathrm{gam} $ axit axetic với $ 400~\mathrm{gam} $ ancol isoamylic \chemfig{{(CH_3)}_2-CH-CH_2-CH_2-OH} ( xúc tác $ H_2SO_4 $ đặc),(biết hiệu suất của phản ứng este hóa là:$ 58 \% $). Khối lượng isoamyl axetat (dầu chuối) thu được là:
	\choice
	{%
		$ 896,55 ~\mathrm{gam}  $
	}
	{%
		$ 342,73 ~\mathrm{gam} $
	}
	{%
		\True $ 301,6 ~\mathrm{gam} $
	}
	{%
		$ 1018,81 ~\mathrm{gam} $
	}
	\sodongkeex[4]
	\loigiai
	{%
		
	}
\end{ex}
%%%%%%%%%%%%%%%%%%%%%%%%%%%%%%%% Kết thúc câu 3 %%%%%%%%%%%%%%%%%%%%%%%%%%%%%%%%%%%%%%%%%%%%%%%%%%%%%%
%%%%%%%%%%%%%%%%%%%%%%%%%%%%%%%% Bắt đầu câu 4 %%%%%%%%%%%%%%%%%%%%%%%%%%%%%%%%%%%%%%%%%%%%%%%%%%%%%%%
%%%%%%%%%%%%%%%%%%%%%%%%%%%%%%%%%%%%%%%%%%%%%%%%%%%%%%%
\begin{ex}[3][Xác định công thức cấu tạo este]
	Cho $ 0,1\mathrm{mol} $ axit đơn chức $ X $ phản ứng với $ 0,15\mathrm{mol} $ ancol etylic thu được $ 4,664\mathrm{gam} $ este với hiệu suất $ 53 \% $
	\choice
	{%
		\chemfig{C_2H_5COOCH_3}
	}
	{%
		\chemfig{CH_3COOCH_3}
	}
	{%
		\True \chemfig{CH_3COOC_2H_5}
	}
	{%
		\chemfig{HCOOCH_2CH_2CH_3}
	}
	\sodongkeex[4]
	\loigiai
	{%
		
	}
\end{ex}
%%%%%%%%%%%%%%%%%%%%%%%%%%%%%%%% Kết thúc câu 4 %%%%%%%%%%%%%%%%%%%%%%%%%%%%%%%%%%%%%%%%%%%%%%%%%%%%%%
%%%%%%%%%%%%%%%%%%%%%%%%%%%%%%%% Bắt đầu câu 5 %%%%%%%%%%%%%%%%%%%%%%%%%%%%%%%%%%%%%%%%%%%%%%%%%%%%%%%
\begin{ex}[3][Xác định lượng este thu được]
	Chia $ 23,6\mathrm{gam} $ hỗn hợp $ X $ gồm axit axetic và ancol etylic thành hai phần bằng nhau. Cho phần một tác dụng với $ Na $ dư, thu được $ 2,464\mathrm{mol} $
	khí $ H_2 $ ở (đktc) . Đun phần hai với $ H_2SO_4 $ đặc, thu được $ m $ gam este với hiệu suất $ 65\% $. Giá trị của m là:
	\choice
	{%
		$ 6,864 \mathrm{gam}$
	}
	{%
		$ 13,538 \mathrm{gam}$
	}
	{%
		$ 16,246 \mathrm{gam}$
	}
	{%
		\True $ 5,720 \mathrm{gam}$
	}
	\sodongkeex[6]
	\loigiai
	{%
		
	}
\end{ex}
%%%%%%%%%%%%%%%%%%%%%%%%%%%%%%%% Kết thúc câu 5 %%%%%%%%%%%%%%%%%%%%%%%%%%%%%%%%%%%%%%%%%%%%%%%%%%%%%%
%%%%%%%%%%%%%%%%%%%%%%%%%%%%%%%% Bắt đầu câu 6 %%%%%%%%%%%%%%%%%%%%%%%%%%%%%%%%%%%%%%%%%%%%%%%%%%%%%%%
\Closesolutionfile{ans}


\begin{dangntd}{Phản ứng thủy phân của este đặc biệt}
	\begin{ntdppg}{Phương pháp giải}
		\begin{itemize}
			\item Một số este đặc biệt: Este của phenol; este khi bị thủy phân sinh ra sản phẩm có phản ứng tráng bạc; este tạo bởi các axit cacboxylic và ancol nhỏ nhất trong dãy đồng đẳng;...
			\item Viết sơ đồ phản ứng ( thường không phải cân bằng phương trình hóa học)
		\begin{itemize}
			\item 
		\schemestart
		$ \left(-COO-\right) $
		\+
		\chemfig{NaOH}
		\arrow{->[][][]}[,.6,,,]
	     $ \left(-COONa-\right) $
		\+
		$ \left(-OH\right) $
		\schemestop
		
			\item 
		\schemestart
		$ \left(-COO-C_6H_4R\right) $
		\+
		\chemfig{2NaOH}
		\arrow{->[][][]}[,.6,,,]
		$ \left(-COONa-\right) $
		\+
		$ \left(RC_6H_4ONa\right) $
		\+
		$ H_2O $
		\schemestop
		
			\item 
		\schemestart
		$ 2\left(-OH\right) $
		\+
		$ 2Na $
		\arrow{->[][][]}[,.6,,,]
		$ 2\left(-ONa-\right) $
		\+
		$ H_2 $
		\schemestop
		
		\end{itemize}
	\item Khai thác tất cả dữ kiện đề cho như: cấu trúc mạch cacbon; giới hạn phân tử khối; đặc điểm của sản phẩm tạo thành;...
	\item Vận dụng linh hoạt định luật bảo toàn nguyên tố, định luật bảo toàn khối lượng.
		\end{itemize}
	\end{ntdppg}
\end{dangntd}
\begin{notegsnd}
	\begin{enumerate}
		\item Các quan hệ giải toán:
		
		\begin{tabular}{*9{c}}
		$ -COOR $	& $ + $ & $ NaOH $ 
		&   
		\begin{tikzpicture}[declare function={d=1.2cm;}]
			\path (0,0) coordinate (A)
			(d,0) coordinate (B)
			;
			\tikzset{%
				
				muiten/.pic={%
					\begin{scope}[transform canvas={yshift=4pt}]
						\draw[->,>=stealth] (A)--(B)
						;
					\end{scope}
				}
			}     
			\path pic [local bounding box=A1] at (0,0) {muiten};
		\end{tikzpicture}
		
		& $ -COONa $  & $ + $ & $ ROH $ &  &  \\
	$ a $	&  & $ a $ & & $ a $ & & $ a $&   & (mol)\\

		$ -COOC_6H_4-R' $	& $ + $ & $ 2NaOH $ 
	&   
	\begin{tikzpicture}[declare function={d=1.2cm;}]
		\path (0,0) coordinate (A)
		(d,0) coordinate (B)
		;
		\tikzset{%
			
			muiten/.pic={%
				\begin{scope}[transform canvas={yshift=4pt}]
					\draw[->,>=stealth] (A)--(B)
					;
				\end{scope}
			}
		}     
		\path pic [local bounding box=A1] at (0,0) {muiten};
	\end{tikzpicture}
	
	& $ -COONa $  & $ + $ & $ R'C_6H_4ONa $ & $ + $ & $ H_2O $ \\
	
	$ b $	&  & $ 2b $ & & $ b $ & & $ b $&   & $ b  ~(mol)$\\	
		\end{tabular}
	\item Công thức thường gặp:\begin{multicols}{4}
	\tcbox[width=3cm]{$ a+b = n_{-COO-} $}
	\columnbreak
	\tcbox[width=3cm]{$ a+2b =  n_{NaOH} $}
	\columnbreak
	\tcbox[width=3cm]{$ n_{ancol}=a~ mol $}
	\columnbreak
	\tcbox[width=3cm]{$ n_{H_2O}=b~ mol $}
	\end{multicols}
    \item Dấu hiệu nhận biết:
    \begin{itemize}
    	\item hỗn hợp este, không nói đến mạch, có vòng benzen
    	\item Công thức phân tử của este $ C_xH_yO_z (x\geq7) $ 
    	\item Xà phòng hóa tạo ra nước
    \end{itemize}
    
	\end{enumerate}
\end{notegsnd}
\begin{vdm}{Ví dụ mẫu}
\end{vdm}
\begin{vdex}[2]
	Cho $ 7,2~\mathrm{gam} $ vinyl fomat tác dụng vừa đủ với dung dịch $ NaOH $ thu đuọc hỗn hợp $ X $ .Cho hỗn hợp X tác dụng với $ AgNO_3 $ trong $ NH_3 $ dư thu được $ a~\mathrm{gam}~Ag$ .Giá trị của $ a $ là:
\choice
{%
 $ 21.6 $
}
{%
	$ 10.8 $
}
{%
\True 	$ 43.2 $
}
{%
	$ 16.2 $
}	
\huongdan{%
Phương trình hóa học:

\begin{tabular}{*8{c}}	
	\chemfig{HCOOCH=CH_2}& $ + $ & \chemfig{KOH}&    
	
	\begin{tikzpicture}[declare function={d=1.2cm;}]
		\path (0,0) coordinate (A)
		(d,0) coordinate (B)
		;
		\tikzset{%
			
			muiten/.pic={%
				\begin{scope}[transform canvas={yshift=4pt}]
					\draw[->,>=stealth] (A)--(B)
					;
				\end{scope}
			}
		}     
		\path pic [local bounding box=A1] at (0,0) {muiten};
	\end{tikzpicture}
	& \chemfig{HCOOK} & $ + $ &  \chemfig{CH_3CHO}  &   \\
	$ 0.1 $ &    &     &    &   $\rightarrow 0.1$  &     &   $\rightarrow 0.1 $   & $ \mathrm{mol} $ \\
\end{tabular}\\
Hỗn hợp $ X $ gồm \chemfig{HCOOK} và \chemfig{CH_3CH=O}\\
Khi $ X $ tác dụng với dung dịch $ AgNO_3/{NH_3}~_\text{dư} $ ta có sơ đồ phản ứng sau:\\

\begin{tabular}{*4{c}}
\chemfig{HCOOK}	&	\begin{tikzpicture}[declare function={d=1.2cm;}]
	\path (0,0) coordinate (A)
	(d,0) coordinate (B)
	;
	\tikzset{%
		
		muiten/.pic={%
			\begin{scope}[transform canvas={yshift=4pt}]
				\draw[->,>=stealth] (A)--(B)
				;
			\end{scope}
		}
	}     
	\path pic [local bounding box=A1] at (0,0) {muiten};
\end{tikzpicture} &2Ag &\\
$ 0,1 $	
& 
\begin{tikzpicture}[declare function={d=1.2cm;}]
	\path (0,0) coordinate (A)
	(d,0) coordinate (B)
	;
	\tikzset{%
		
		muiten/.pic={%
			\begin{scope}[transform canvas={yshift=4pt}]
				\draw[->,>=stealth] (A)--(B)
				;
			\end{scope}
		}
	}     
	\path pic [local bounding box=A1] at (0,0) {muiten};
\end{tikzpicture}
& $ 0,2 $ & $ \mathrm{mol} $\\

\chemfig{CH_3CH=O}	&	\begin{tikzpicture}[declare function={d=1.2cm;}]
	\path (0,0) coordinate (A)
	(d,0) coordinate (B)
	;
	\tikzset{%
		
		muiten/.pic={%
			\begin{scope}[transform canvas={yshift=4pt}]
				\draw[->,>=stealth] (A)--(B)
				;
			\end{scope}
		}
	}     
	\path pic [local bounding box=A1] at (0,0) {muiten};
\end{tikzpicture} &2Ag &\\
$ 0,1 $	
& 
\begin{tikzpicture}[declare function={d=1.2cm;}]
	\path (0,0) coordinate (A)
	(d,0) coordinate (B)
	;
	\tikzset{%
		
		muiten/.pic={%
			\begin{scope}[transform canvas={yshift=4pt}]
				\draw[->,>=stealth] (A)--(B)
				;
			\end{scope}
		}
	}     
	\path pic [local bounding box=A1] at (0,0) {muiten};
\end{tikzpicture}
& $ 0,2 $ & $ \mathrm{mol} $\\
\end{tabular}\\
$ \Rightarrow \Sigma n_Ag=0.4~\mathrm{mol} $\\
$ \Rightarrow  m_Ag=0.4\cdot108=43.2~\mathrm{gam} $
}
\end{vdex}


\begin{vdex}[3]
Đun nóng $ 14,64~\mathrm{gam} $ este $ E $ có công thức phân tử $ C_7H_6O_2 $ cần dùng vừa đủ $ 80~\mathrm{gam}$ dung dịch  NaOH $12 \% $. Cô cạn dung dịch được $ x $ gam muối khan.Giá trị của $ x $ là
\choice
{%
\True $ 22,08 $
}
{%
	$ 28,08 $
}
{%
	$ 24,04 $
}
{%
	$ 25,82 $
}

\huongdan
{%
Theo đề bài ta có:$ n_{C_7H_6O_2} =0.12~\mathrm{mol}$\\
$ m_{NaOH} =80\cdot12\%=9,6~\mathrm{gam}$\\
$\Rightarrow  n_{NaOH}  =\dfrac{9,6}{40} =0,24 ~\mathrm{mol} $\\
Vì $ n_{NaOH}=2n_{\text{este}}$ nên Este E là este của phenol $ (RCOOC_6H_5 )$\\
Phương trình hóa học:

\begin{tabular}{*9{c}}
	$ HCOOC_6H_5 $	& $ + $ & $ 2NaOH $ 
&   
\begin{tikzpicture}[declare function={d=1.2cm;}]
	\path (0,0) coordinate (A)
	(d,0) coordinate (B)
	;
	\tikzset{%
		
		muiten/.pic={%
			\begin{scope}[transform canvas={yshift=4pt}]
				\draw[->,>=stealth] (A)--(B)
				;
			\end{scope}
		}
	}     
	\path pic [local bounding box=A1] at (0,0) {muiten};
\end{tikzpicture}

& $ HCOONa $  & $ + $ & $ C_6H_5ONa $ & $ + $ & $ H_2O $ \\

$ b $	&  & $ 2b $ & & $ b $ & & $ b $&   & $ b~(\mathrm{mol})$\\	
\end{tabular}\\
Ta có: $ m_{\text{muối khan}} = m_{HCOONa}+ m_{C_6H_5ONa} =0,12\cdot68 + 0,12\cdot116=22,08~\mathrm{gam} $
}
\end{vdex}

\begin{bttl}{Bài tập tự luyện }
\end{bttl}
\Opensolutionfile{ans}[DAPAN/BTTL04]
%%%%%%%%%%%%%%%%%%%%%%%%%%%%%% Bắt đầu câu 1 %%%%%%%%%%%%%%%%%%%%%%%%%%%%%%
\begin{ex}[3]
	Cho $ 5,16~\mathrm{gam} $ một este đơn chức mạch hở $ X $ phản ứng hoàn toàn với lượng dư $ AgNO_3/NH_3 $ thì thu được $ 12,96~\mathrm{gam} Ag $.Biết $ M_X<150. $ Số đồng phân cấu tạo phù hợp của $ X $ là:
	\choice
	{%
	$ 4 $
}
	{%
	$ 2 $
}
	{%
	$ 5 $
}
	{%
\True	$ 3 $
}
\sodongkeex[5]
\end{ex}

%%%%%%%%%%%%%%%%%%%%%%%%%%%%%% Kết thúc câu 1 %%%%%%%%%%%%%%%%%%%%%%%%%%%%%%
%%%%%%%%%%%%%%%%%%%%%%%%%%%%%% Bắt đầu câu 2 %%%%%%%%%%%%%%%%%%%%%%%%%%%%%%
\begin{ex}[2][][(Đề MH - 2020)]
	: Hỗn hợp $\mathrm{X}$ gồm hai este có cùng công thức phân tử $\mathrm{C}_8 \mathrm{H}_8 \mathrm{O}_2$ và đều chứa vòng benzen. Để phản ứng hết với $0,25 \mathrm{~mol} \mathrm{X}$ cần tối đa $0,35 \mathrm{~mol} \mathrm{NaOH}$ trong dung dịch, thu được $\mathrm{m}$ gam hỗn hợp hai muối. Giá trị của m là
	\choice
	{%
	$ 20,5 $
}
	{%
	$ 17,0 $
}
	{%
\True	$ 30,0 $
}
	{%
	$ 13,0 $
}
\sodongkeex[6]
\end{ex}
%%%%%%%%%%%%%%%%%%%%%%%%%%%%%% Kết thúc câu 2 %%%%%%%%%%%%%%%%%%%%%%%%%%%%%%
%%%%%%%%%%%%%%%%%%%%%%%%%%%%%% Bắt đầu câu 3 %%%%%%%%%%%%%%%%%%%%%%%%%%%%%%
\begin{ex}[2][][(Đề TSĐH A - 2011)]
	 Cho axit salixylic (axit o-hiđroxibenzoic) phản ứng với anhiđrit axetic, thu được axit axetylsalixylic (o- $\left.\mathrm{CH}_3 \mathrm{COO}-\mathrm{C}_6 \mathrm{H}_4-\mathrm{COOH}\right)$ dùng làm thuốc cảm (aspirin). Để phản ứng hoàn toàn với 43,2 gam axit axetylsalixylic cần vừa đủ $\mathrm{V}$ lít dung dịch $\mathrm{KOH} 1 \mathrm{M}$. Giá trị của $\mathrm{V}$ là
	\choice
	{%
		$ 0,24 $
	}
	{%
		$ 0,96 $
	}
	{%
		\True	$ 0,72 $
	}
	{%
		$  0,48 $
	}
	\sodongkeex[6]
\end{ex}
%%%%%%%%%%%%%%%%%%%%%%%%%%%%%% Kết thúc câu 3 %%%%%%%%%%%%%%%%%%%%%%%%%%%%%%
%%%%%%%%%%%%%%%%%%%%%%%%%%%%%% Bắt đầu câu 4 %%%%%%%%%%%%%%%%%%%%%%%%%%%%%%
\begin{ex}[2][][(Đề TSĐH B - 2014)]
	 Hai este $X, Y$ có cùng công thức phân tử $\mathrm{C}_8 \mathrm{H}_8 \mathrm{O}_2$ và chứa vòng benzen trong phân từ. Cho 6,8 gam hỗn hợp gồm $\mathrm{X}$ và $\mathrm{Y}$ tác dụng với dung dịch $\mathrm{NaOH}$ dư, đun nóng, lượng $\mathrm{NaOH}$ phản ứng tối đa là $0,06 \mathrm{~mol}$, thu được dung dịch $\mathrm{Z}$ chứa 4,7 gam ba muối. Khối lượng muối của axit cacboxýlic có phân tử khối lớn hơn trong $\mathrm{Z}$ là
	\choice
	{%
		3,40 gam
	}
	{%
		\True 0,82 gam
	}
	{%
		0,68 gam
	}
	{%
		2,72 gam
	}
	\sodongkeex[6]
\end{ex}
%%%%%%%%%%%%%%%%%%%%%%%%%%%%%% Kết thúc câu 4 %%%%%%%%%%%%%%%%%%%%%%%%%%%%%%
%%%%%%%%%%%%%%%%%%%%%%%%%%%%%% Bắt đầu câu 5 %%%%%%%%%%%%%%%%%%%%%%%%%%%%%%

\begin{ex}[3][][(Đề THPT QG - 2017)]
	 Hỗn hợp X gồm phenyl axetat, metyl benzoat, benzyl fomat và etyl phenyl oxalat. Thủy phân hoàn toàn 36,9 gam $\mathrm{X}$ trong dung dịch $\mathrm{NaOH}$ (dư, đun nóng), có 0,4 mol $\mathrm{NaOH}$ phản ứng, thu được $\mathrm{m}$ gam hỗn hợp muối và 10,9 gam hỗn hợp Y gồm các ancol. Cho toàn bộ $\mathrm{Y}$ tác dụng với $\mathrm{Na}$ dư, thu được 2,24 lít khí $\mathrm{H}_2$ (đktc). Giá trị của m là
	
	\choice
	{%
	\True	$ 40,2 $
	}
	{%
		$ 49,3 $
    }	
	{%
		$ 42,0 $
	}
	{%
	  $ 38,4 $
	}
	\sodongkeex[6]
\end{ex}
%%%%%%%%%%%%%%%%%%%%%%%%%%%%%% Kết thúc câu 6 %%%%%%%%%%%%%%%%%%%%%%%%%%%%%%
%%%%%%%%%%%%%%%%%%%%%%%%%%%%%% Bắt đầu câu 7 %%%%%%%%%%%%%%%%%%%%%%%%%%%%%%
\begin{ex}[3][][(Đề THPT QG - 2018)]
	Hỗn hợp E gồm bốn este đều có công thức $\mathrm{C}_8 \mathrm{H}_8 \mathrm{O}_2$ và có vòng benzen. Cho $\mathrm{m}$ gam $\mathrm{E}$ tác dụng tối đa với $200 \mathrm{ml}$ dung dịch $\mathrm{NaOH} 1 \mathrm{M}$ (đun nóng), thu được hỗn hợp $\mathrm{X}$ gồm các ancol và 20,5 gam hỗn hợp muối. Cho toàn bộ $\mathrm{X}$ vào bình đựng kim loại $\mathrm{Na}$ dư, sau khi phản ứng kết thúc khối lượng chất rắn trong bình tăng 6,9 gam so với ban đầu. Giá trị của $\mathrm{m}$ là 
	
	\choice
	{%
	$ 13,60$
	}
	{%
	$ 8,16 $
	}	
	{%
	$ 16,32 $
	}
	{%
	\True $ 20,40 $
	}
	\sodongkeex[6]
\end{ex}
%%%%%%%%%%%%%%%%%%%%%%%%%%%%%% Kết thúc câu 7 %%%%%%%%%%%%%%%%%%%%%%%%%%%%%%
%%%%%%%%%%%%%%%%%%%%%%%%%%%%%% Bắt đầu câu 8  %%%%%%%%%%%%%%%%%%%%%%%%%%%%%%
\begin{ex}[3][][(Đề THPT QG - 2018)]
	 Cho $\mathrm{m}$ gam hỗn hợp $\mathrm{X}$ gồm ba este đều đơn chức tác dụng tối đa với $350 \mathrm{ml}$ dung dịch $\mathrm{NaOH} 1 \mathrm{M}$, thu được hỗn hợp $\mathrm{Y}$ gồm hai ancol cùng dãy đồng đẳng và 28,6 gam hỗn hợp muối $\mathrm{Z}$. Đốt cháy hoàn toàn $\mathrm{Y}$, thu được 4,48 lít khí $\mathrm{CO}_2$ (đktc) và 6,3 gam $\mathrm{H}_2 \mathrm{O}$. Giá trị của m là
	\choice
	{%
	\True $ 21,9$
	}
	{%
		$ 30,4 $
	}	
	{%
		$ 20,1 $
	}
	{%
		$ 22,8 $
	}
	\sodongkeex[5]
\end{ex}
%%%%%%%%%%%%%%%%%%%%%%%%%%%%%% Kết thúc câu 8 %%%%%%%%%%%%%%%%%%%%%%%%%%%%%%
%%%%%%%%%%%%%%%%%%%%%%%%%%%%%% Bắt đầu câu 9  %%%%%%%%%%%%%%%%%%%%%%%%%%%%%%
\begin{ex}[3][][(Đề THPT QG - 2018)]
	Hỗn hợp $E$ gồm bốn este đều có công thức $\mathrm{C}_8 \mathrm{H}_8 \mathrm{O}_2$ và có vòng benzen. Cho 16,32 gam $\mathrm{E}$ tác dụng tối đa với $\mathrm{V} \mathrm{ml}$ dung dịch $\mathrm{NaOH} 1 \mathrm{M}$ (đun nóng), thu được hỗn hợp $\mathrm{X}$ gồm các ancol và 18,78 gam hỗn hợp muối. Cho toàn bộ $\mathrm{X}$ vào bình đựng kim loại $\mathrm{Na}$ dư, sau khi phản ứng kết thúc khối lượng chất rắn trong bình tăng 3,83 gam so với ban đầu. Giá trị của $\mathrm{V}$ là
	\choice
	{%
	\True $190$
	}
	{%
		$100$
	}	
	{%
		$120$
	}
	{%
		$240$
	}
	\sodongkeex[8]
\end{ex}
%%%%%%%%%%%%%%%%%%%%%%%%%%%%%% Kết thúc câu 9 %%%%%%%%%%%%%%%%%%%%%%%%%%%%%%
%%%%%%%%%%%%%%%%%%%%%%%%%%%%%% Bắt đầu câu 10  %%%%%%%%%%%%%%%%%%%%%%%%%%%%%%
\begin{ex}[3][][(Đề THPT QG - 2017)]
	Cho 0,3 mol hỗn hợp $\mathrm{X}$ gồm hai este đơn chức tác dụng vừa đủ với 250 $\mathrm{ml}$ dung dịch $\mathrm{KOH} 2 \mathrm{M}$, thu được chất hữu cơ $\mathrm{Y}$ (no, đơn chức, mạch hở, có tham gia phản ứng tráng bạc) và 53 gam hỗn hợp muối. Đốt cháy toàn bộ $\mathrm{Y}$ cần vừa đủ 5,6 lít khí $\mathrm{O}_2$ (đktc). Khối lượng của 0,3 mol X là
	\choice
	{%
		$ 29,4 $ gam
	}
	{%
		$ 31,0 $ gam
	}	
	{%
		\True $ 33,0 $ gam
	}
	{%
		$ 41,0 $ gam
	}
	\sodongkeex[8]
\end{ex}
%%%%%%%%%%%%%%%%%%%%%%%%%%%%%% Kết thúc câu 10 %%%%%%%%%%%%%%%%%%%%%%%%%%%%%%
%%%%%%%%%%%%%%%%%%%%%%%%%%%%%% Bắt đầu câu 11  %%%%%%%%%%%%%%%%%%%%%%%%%%%%%%






\Closesolutionfile{ans}





