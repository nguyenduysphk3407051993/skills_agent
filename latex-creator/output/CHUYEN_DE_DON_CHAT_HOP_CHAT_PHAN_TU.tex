\documentclass[Main.tex]{subfiles}
\begin{document}

\chapter{Phân tử - Liên kết hóa~học}
\section{Đơn chất - Hợp chất - Phân tử}

\begin{Muctieu}
    \begin{itemize}
        \item Nêu được khái niệm đơn chất, hợp chất và phân tử.
        \item Phân biệt được đơn chất và hợp chất dựa vào công thức hóa học.
        \item Tính được khối lượng phân tử theo đơn vị amu.
        \item Giải được các bài toán liên quan đến thành phần phần trăm nguyên tố và lập công thức hóa học.
    \end{itemize}
\end{Muctieu}

\begin{kd}
    \immini{
        Trong không khí, khí oxygen ($O_2$) giúp duy trì sự sống, nhưng khí ozone ($O_3$) ở tầng thấp lại là chất gây ô nhiễm. Nước ($H_2O$) là hợp chất thiết yếu, trong khi muối ăn ($NaCl$) là gia vị không thể thiếu. Làm thế nào để phân loại và hiểu rõ cấu tạo của những chất này từ các hạt vô cùng nhỏ bé? Bài học này sẽ giúp chúng ta khám phá thế giới vi mô của các chất.
    }{

    }
\end{kd}

\subsection{Nội dung bài học}

\subsubsection{Đơn chất}
\begin{ghinho}
    \textbf{Đơn chất} là những chất được tạo nên từ một nguyên tố hóa học.
    \begin{itemize}
        \item Đơn chất kim loại: Thường dẫn điện, dẫn nhiệt, có ánh kim (VD: Đồng, Sắt, Nhôm...).
        \item Đơn chất phi kim: Thường không dẫn điện, không dẫn nhiệt (VD: Khí Hidro, Than, Lưu huỳnh...).
        \item Khí hiếm: Tồn tại dưới dạng nguyên tử độc lập (VD: Helium, Neon...).
    \end{itemize}
\end{ghinho}

\subsubsection{Hợp chất}
\begin{ghinho}
    \textbf{Hợp chất} là những chất được tạo nên từ hai hay nhiều nguyên tố hóa học.
    \begin{itemize}
        \item Ví dụ: Nước ($H_2O$) gồm H và O; Muối ăn ($NaCl$) gồm Na và Cl.
    \end{itemize}
\end{ghinho}

\subsubsection{Phân tử - Khối lượng phân tử}
\begin{ghinho}
    \begin{itemize}
        \item \textbf{Phân tử} là hạt đại diện cho chất, gồm một số nguyên tử liên kết với nhau và thể hiện đầy đủ tính chất hóa học của chất.
        \item \textbf{Khối lượng phân tử} (Phân tử khối) bằng tổng khối lượng của các nguyên tử có trong phân tử. Đơn vị: amu.
    \end{itemize}
\end{ghinho}

%=============================================================
% DẠNG 1
%=============================================================
\begin{dang}{Phân biệt đơn chất và hợp chất}
    \begin{phuongphap}
	\begin{itemize}
	    \item \textbf{Đơn chất:} 1 KHHH. 
        \item \textbf{Hợp chất:} $\ge 2$ KHHH.
	\end{itemize}
    \end{phuongphap}

    \phan{Trắc nghiệm nhiều lựa chọn}
    \Opensolutionfile{ansex}[Ans/LGEX_D1]
    \Opensolutionfile{ans}[Ans/AnsEX_D1]
    
    %%%%%===========EX_1=======%%%%%
    \begin{ex}
        Chất nào sau đây là đơn chất?
        \choice
        {$H_2O$}
        {\True $O_2$}
        {$NaCl$}
        {$HCl$}
        \loigiai{$O_2$ được tạo từ 1 nguyên tố O.}
    \end{ex}
    
    %%%%%===========EX_2=======%%%%%
    \begin{ex}
        Chất nào sau đây là hợp chất?
        \choice
        {$Cu$}
        {$Fe$}
        {\True $CO_2$}
        {$N_2$}
        \loigiai{$CO_2$ tạo từ C và O.}
    \end{ex}

    %%%%%===========EX_3=======%%%%%
    \begin{ex}
        Dãy gồm toàn đơn chất là:
        \choice
        {$Fe, N_2, HCl$}
        {$Mg, K_2O, C$}
        {\True $S, Cl_2, Al$}
        {$O_3, H_2O, Na$}
        \loigiai{$S, Cl_2, Al$ đều là đơn chất.}
    \end{ex}

    %%%%%===========EX_4=======%%%%%
    \begin{ex}
        Trong các chất: Kim cương ($C$), Đá vôi ($CaCO_3$), Khí Metan ($CH_4$), Khí Clo ($Cl_2$). Số lượng hợp chất là:
        \choice
        {1}
        {\True 2}
        {3}
        {4}
        \loigiai{$CaCO_3$ và $CH_4$.}
    \end{ex}

    %%%%%===========EX_5=======%%%%%
    \begin{ex}
        Hợp chất $X$ được tạo bởi Na, C và O. Công thức nào sau đây có thể là $X$?
        \choice
        {$NaCl$}
        {$NaOH$}
        {\True $Na_2CO_3$}
        {$NaHCO_3$}
        \loigiai{$Na_2CO_3$ chứa Na, C, O.}
    \end{ex}
    %%%%%===========EX_6=======%%%%%
    \begin{ex}
        Đơn chất phi kim ở thể rắn là:
        \choice
        {$O_2$}
        {\True $P$}
        {$H_2$}
        {$N_2$}
        \loigiai{Phosphorus (P) là phi kim rắn.}
    \end{ex}
    %%%%%===========EX_7=======%%%%%
    \begin{ex}
        Hợp chất khí gây hiệu ứng nhà kính chính là:
        \choice
        {$O_2$}
        {\True $CO_2$}
        {$N_2$}
        {$H_2$}
        \loigiai{$CO_2$ là nguyên nhân chính gây hiệu ứng nhà kính.}
    \end{ex}
    %%%%%===========EX_8=======%%%%%
    \begin{ex}
        Trong các công thức: $H_2O, NaCl, Cl_2, KOH$. Đơn chất là:
        \choice
        {$H_2O$}
        {$NaCl$}
        {\True $Cl_2$}
        {$KOH$}
        \loigiai{$Cl_2$ chỉ gồm nguyên tố Cl.}
    \end{ex}
    %%%%%===========EX_9=======%%%%%
    \begin{ex}
        Kim loại nào sau đây là đơn chất lỏng ở điều kiện thường?
        \choice
        {Sắt ($Fe$)}
        {Đồng ($Cu$)}
        {\True Thủy ngân ($Hg$)}
        {Nhôm ($Al$)}
        \loigiai{Thủy ngân (Mercury) là kim loại lỏng.}
    \end{ex}
    %%%%%===========EX_10=======%%%%%
    \begin{ex}
        Chất nào là hợp chất vô cơ?
        \choice
        {$CH_4$}
        {\True $CaCO_3$}
        {$C_2H_5OH$}
        {$C_{12}H_{22}O_{11}$}
        \loigiai{$CaCO_3$ là muối vô cơ (đá vôi).}
    \end{ex}
    \Closesolutionfile{ans}
    \Closesolutionfile{ansex}

    \phan{Trắc nghiệm Đúng/Sai}
    \Opensolutionfile{ansex}[Ans/LGTF_D1]
    \Opensolutionfile{ansbook}[Ansbook/AnsTF_D1]
    \Opensolutionfile{ans}[Ans/Tempt_D1]

    %%%%%===========TF_1=======%%%%%
    \begin{ex}
        Phát biểu về đơn chất:
        \choiceTF
        {\True Đơn chất do một nguyên tố tạo nên.}
        {Đơn chất luôn dẫn điện.}
        {\True Kim loại đồng là đơn chất.}
        {Đơn chất không thể bị phân hủy hóa học.}
        \loigiai{
            \begin{itemchoice}[T1,F2,T3,F4]
                \itemch Đơn chất do một nguyên tố tạo nên.
                \itemch Phi kim thường không dẫn điện.
                \itemch Kim loại đồng là đơn chất.
                \itemch Đơn chất $O_3$ có thể phân hủy thành $O_2$.
            \end{itemchoice}
        }
    \end{ex}
    
    %%%%%===========TF_2=======%%%%%
    \begin{ex}
        Về hợp chất:
        \choiceTF
        {\True Nước là hợp chất.}
        {Hợp chất do 1 nguyên tố tạo nên.}
        {\True Muối ăn (NaCl) là hợp chất.}
        {Không khí là hợp chất.}
        \loigiai{
            \begin{itemchoice}[T1,F2,T3,F4]
                \itemch Nước ($H_2O$) gồm H và O.
                \itemch Hợp chất do 2 nguyên tố trở lên tạo nên.
                \itemch Muối ăn ($NaCl$) gồm Na và Cl.
                \itemch Không khí là hỗn hợp nhiều chất.
            \end{itemchoice}
        }
    \end{ex}
    
    %%%%%===========TF_3=======%%%%%
    \begin{ex}
        Phân loại chất:
        \choiceTF
        {\True $O_3$ là đơn chất.}
        {$HCl$ là đơn chất.}
        {\True $KMnO_4$ là hợp chất.}
        {$P$ (Phosphorus) là hợp chất.}
        \loigiai{
            \begin{itemchoice}[T1,F2,T3,F4]
                \itemch $O_3$ chỉ gồm nguyên tố O.
                \itemch $HCl$ gồm H và Cl nên là hợp chất.
                \itemch Gồm K, Mn, O.
                \itemch $P$ chỉ gồm nguyên tố P nên là đơn chất.
            \end{itemchoice}
        }
    \end{ex}

    %%%%%===========TF_4=======%%%%%
    \begin{ex}
        Xét các chất $H_2, H_2S, S, SO_2$:
        \choiceTF
        {\True Có 2 đơn chất trong dãy.}
        {\True Có 2 hợp chất trong dãy.}
        {Tất cả đều là khí ở điều kiện thường.}
        {$H_2S$ là đơn chất.}
        \loigiai{
            \begin{itemchoice}[T1,T2,F3,F4]
                \itemch ($H_2, S$).
                \itemch ($H_2S, SO_2$).
                \itemch S là chất rắn.
                \itemch $H_2S$ là hợp chất.
            \end{itemchoice}
        }
    \end{ex}

    %%%%%===========TF_5=======%%%%%
    \begin{ex}
        Cho công thức $C_2H_6O$ (Ethanol) và $O_2$:
        \choiceTF
        {\True $C_2H_6O$ là hợp chất hữu cơ.}
        {$O_2$ là hợp chất.}
        {\True $O_2$ duy trì sự cháy.}
        {$C_2H_6O$ chứa 2 nguyên tố.}
        \loigiai{
            \begin{itemchoice}[T1,F2,T3,F4]
                \itemch Ethanol là chất hữu cơ.
                \itemch $O_2$ là đơn chất.
                \itemch
                \itemch Chứa 3 nguyên tố (C, H, O).
            \end{itemchoice}
        }
    \end{ex}
    %%%%%===========TF_6=======%%%%%
    \begin{ex}
        Than chì và Kim cương:
        \choiceTF
        {\True Đều tạo từ nguyên tố Carbon.}
        {Là hợp chất của Carbon.}
        {\True Kim cương rất cứng.}
        {Than chì không dẫn điện.}
        \loigiai{
            \begin{itemchoice}[T1,F2,T3,F4]
                \itemch Đều là dạng thù hình của C.
                \itemch Là đơn chất.
                \itemch Cứng nhất trong tự nhiên.
                \itemch Than chì dẫn điện được.
            \end{itemchoice}
        }
    \end{ex}
    %%%%%===========TF_7=======%%%%%
    \begin{ex}
        Về kim loại Mercury (Thủy ngân):
        \choiceTF
        {\True Là kim loại duy nhất ở thể lỏng (đktc).}
        {Là hợp chất.}
        {\True Kí hiệu là Hg.}
        {Rất độc.}
        \loigiai{
            \begin{itemchoice}[T1,F2,T3,F4]
                \itemch
                \itemch Là đơn chất.
                \itemch
                \itemch
            \end{itemchoice}
        }
    \end{ex}
    %%%%%===========TF_8=======%%%%%
    \begin{ex}
        Khí hiếm (Helium, Neon, Argon):
        \choiceTF
        {\True Tồn tại dạng nguyên tử tự do.}
        {Là hợp chất.}
        {\True Rất kém hoạt động hóa học.}
        {Thường ở thể lỏng.}
        \loigiai{
            \begin{itemchoice}[T1,F2,T3,F4]
                \itemch
                \itemch Là đơn chất.
                \itemch Khó phản ứng.
                \itemch Là chất khí.
            \end{itemchoice}
        }
    \end{ex}
    %%%%%===========TF_9=======%%%%%
    \begin{ex}
        Nước tự nhiên và Nước cất:
        \choiceTF
        {Đều là chất tinh khiết.}
        {\True Nước cất là chất tinh khiết ($H_2O$).}
        {\True Nước tự nhiên là hỗn hợp.}
        {Nước cất dẫn điện tốt.}
        \loigiai{
            \begin{itemchoice}[F1,T2,T3,F4]
                \itemch Nước tự nhiên có tạp chất.
                \itemch
                \itemch
                \itemch Nước tinh khiết dẫn điện kém.
            \end{itemchoice}
        }
    \end{ex}
    %%%%%===========TF_10=======%%%%%
    \begin{ex}
        Phân loại $Cl_2, HCl, NaCl, O_3$:
        \choiceTF
        {\True 2 đơn chất.}
        {2 hợp chất.}
        {\True $O_3$ là đơn chất.}
        {$NaCl$ là khí.}
        \loigiai{
            \begin{itemchoice}[T1,F2,T3,F4]
                \itemch $Cl_2, O_3$.
                \itemch $HCl, NaCl$.
                \itemch
                \itemch Muối ăn là rắn.
            \end{itemchoice}
        }
    \end{ex}
    \Closesolutionfile{ans}
    \Closesolutionfile{ansbook}
    \Closesolutionfile{ansex}
    %%\bangdapanTF{AnsTF_D1}

    \phan{Trả lời ngắn}
    \Opensolutionfile{ansbth}[Ans/LGSA_D1]
    \Opensolutionfile{ansbt}[Ans/AnsSA_D1]

    %%%%%===========SA_1=======%%%%%
    \begin{bt}
        Số nguyên tố hóa học tạo nên phân tử $H_2SO_4$ là bao nhiêu?
        \shortans{3}
        \loigiai{Gồm H, S, O.}
    \end{bt}

    %%%%%===========SA_2=======%%%%%
    \begin{bt}
        Trong các chất: $O_2, O_3, CO, CO_2, C$. Số lượng \textbf{đơn chất} là?
        \shortans{3}
        \loigiai{$O_2, O_3, C$.}
    \end{bt}

    %%%%%===========SA_3=======%%%%%
    \begin{bt}
        Than chì (Graphite) được tạo nên từ bao nhiêu loại nguyên tố hóa học?
        \shortans{1}
        \loigiai{Chỉ tạo từ Carbon.}
    \end{bt}

    %%%%%===========SA_4=======%%%%%
    \begin{bt}
        Phân tử Ozone ($O_3$) được tạo nên từ bao nhiêu nguyên tử Oxygen?
        \shortans{3}
        \loigiai{3 nguyên tử O.}
    \end{bt}

    %%%%%===========SA_5=======%%%%%
    \begin{bt}
        Muối ăn ($NaCl$) được tạo nên từ bao nhiêu nguyên tố hóa học?
        \shortans{2}
        \loigiai{Tạo từ 2 nguyên tố Na và Cl.}
    \end{bt}
    %%%%%===========SA_6=======%%%%%
    \begin{bt}
        Số nguyên tố trong phân tử Axit photphoric ($H_3PO_4$)?
        \shortans{3}
        \loigiai{H, P, O.}
    \end{bt}
    %%%%%===========SA_7=======%%%%%
    \begin{bt}
        Số nguyên tử Nitơ trong phân tử $N_2$?
        \shortans{2}
        \loigiai{2 nguyên tử.}
    \end{bt}
    %%%%%===========SA_8=======%%%%%
    \begin{bt}
        Số lượng đơn chất trong dãy: $Cu, Al, H_2O, NaCl$?
        \shortans{2}
        \loigiai{$Cu$ và $Al$.}
    \end{bt}
    %%%%%===========SA_9=======%%%%%
    \begin{bt}
        Số lượng hợp chất trong dãy: $O_3, CO_2, SO_2, N_2$?
        \shortans{2}
        \loigiai{$CO_2$ và $SO_2$.}
    \end{bt}
    %%%%%===========SA_10=======%%%%%
    \begin{bt}
        Số nguyên tố tạo nên đường Glucose ($C_6H_{12}O_6$)?
        \shortans{3}
        \loigiai{C, H, O.}
    \end{bt}
    \Closesolutionfile{ansbt}
    \Closesolutionfile{ansbth}

    \phan{Tự luận}
    \Opensolutionfile{ansbth}[Ans/LGBT_D1]
    \Opensolutionfile{ansbt}[Ans/AnsBT_D1]

    %%%%%===========BT_1=======%%%%%
    \begin{bt}
        Kể tên 3 đơn chất kim loại và 3 đơn chất phi kim thường gặp.
        \loigiai{Kim loại: Fe, Al, Cu. Phi kim: O, H, C.}
    \end{bt}

    %%%%%===========BT_2=======%%%%%
    \begin{bt}
        Giải thích tại sao nước ($H_2O$) là hợp chất, còn khí Hydro ($H_2$) là đơn chất?
        \loigiai{Nước tạo từ 2 nguyên tố. $H_2$ từ 1 nguyên tố.}
    \end{bt}

    %%%%%===========BT_3=======%%%%%
    \begin{bt}
        Phân loại các chất sau vào bảng đơn chất/hợp chất: $Mg(OH)_2, Zn, Cl_2, K_2SO_4, Ar$.
        \loigiai{Đơn chất: $Zn, Cl_2, Ar$. Hợp chất: $Mg(OH)_2, K_2SO_4$.}
    \end{bt}

    %%%%%===========BT_4=======%%%%%
    \begin{bt}
        Kim cương và than chì đều được tạo từ nguyên tố Carbon. Chúng có phải là cùng một chất không?
        \loigiai{Không. Chúng là các dạng thù hình khác nhau.}
    \end{bt}

    %%%%%===========BT_5=======%%%%%
    \begin{bt}
        Methane ($CH_4$) là đơn chất hay hợp chất?
        \loigiai{Hợp chất, vì tạo từ C và H.}
    \end{bt}
    %%%%%===========BT_6=======%%%%%
    \begin{bt}
        Phân biệt khí $Cl_2$ và khí $HCl$ về mặt phân loại đơn/hợp chất.
        \loigiai{$Cl_2$ là đơn chất (chỉ có Cl). $HCl$ là hợp chất (H và Cl).}
    \end{bt}
    %%%%%===========BT_7=======%%%%%
    \begin{bt}
        Trong không khí có các khí: $O_2, N_2, Ar, CO_2$. Hãy liệt kê các đơn chất.
        \loigiai{$O_2, N_2, Ar$.}
    \end{bt}
    %%%%%===========BT_8=======%%%%%
    \begin{bt}
        Giải thích sự khác nhau cơ bản trong cấu tạo của kim loại Cu và hợp kim Đồng thau (Cu-Zn).
        \loigiai{Cu là đơn chất tinh khiết. Đồng thau là hỗn hợp của Cu và Zn.}
    \end{bt}
    %%%%%===========BT_9=======%%%%%
    \begin{bt}
        Cho các chất: Vàng (Au), Bạc (Ag), Thạch anh ($SiO_2$). Chất nào là hợp chất?
        \loigiai{Thạch anh ($SiO_2$).}
    \end{bt}
    %%%%%===========BT_10=======%%%%%
    \begin{bt}
        Liệt kê 3 hợp chất có trong gian bếp nhà em.
        \loigiai{Muối ($NaCl$), Đường ($C_{12}H_{22}O_{11}$), Giấm ($CH_3COOH$), Nước ($H_2O$),...}
    \end{bt}
    \Closesolutionfile{ansbt}
    \Closesolutionfile{ansbth}
\end{dang}

%=============================================================
% DẠNG 2
%=============================================================
\begin{dang}{Xác định thành phần nguyên tố}
    \phan{Trắc nghiệm nhiều lựa chọn}
    \Opensolutionfile{ansex}[Ans/LGEX_D2]
    \Opensolutionfile{ans}[Ans/AnsEX_D2]
    %%%%%===========EX_1=======%%%%%
    \begin{ex}
        Phân tử $CaCO_3$ gồm những nguyên tố nào?
        \choice
        {Ca, C}
        {Ca, O}
        {\True Ca, C, O}
        {C, O}
        \loigiai{Calcium, Carbon, Oxygen.}
    \end{ex}
    %%%%%===========EX_2=======%%%%%
    \begin{ex}
        Số nguyên tử Hydrogen trong phân tử $CH_4$ là:
        \choice
        {1}
        {2}
        {3}
        {\True 4}
        \loigiai{4.}
    \end{ex}
    %%%%%===========EX_3=======%%%%%
    \begin{ex}
        Số nguyên tử Oxygen trong $Fe_2(SO_4)_3$ là:
        \choice
        {4}
        {7}
        {\True 12}
        {10}
        \loigiai{12.}
    \end{ex}
    %%%%%===========EX_4=======%%%%%
    \begin{ex}
        $KMnO_4$. Nó chứa bao nhiêu loại nguyên tố?
        \choice
        {2}
        {\True 3}
        {4}
        {1}
        \loigiai{3.}
    \end{ex}
    %%%%%===========EX_5=======%%%%%
    \begin{ex}
        Trong $H_2SO_4$, tỉ lệ số nguyên tử H : S : O là:
        \choice
        {1:1:4}
        {\True 2:1:4}
        {2:1:2}
        {1:2:4}
        \loigiai{2:1:4.}
    \end{ex}
    %%%%%===========EX_6=======%%%%%
    \begin{ex}
        Số nguyên tử Hydrogen trong phân tử $NH_3$ là:
        \choice
        {1}
        {2}
        {\True 3}
        {4}
        \loigiai{3 H.}
    \end{ex}
    %%%%%===========EX_7=======%%%%%
    \begin{ex}
        Phân tử $Ca(OH)_2$ có bao nhiêu loại nguyên tố?
        \choice
        {2}
        {\True 3}
        {4}
        {5}
        \loigiai{Ca, O, H.}
    \end{ex}
    %%%%%===========EX_8=======%%%%%
    \begin{ex}
        Tỉ lệ số nguyên tử các nguyên tố Ca:C:O trong đá vôi ($CaCO_3$) là:
        \choice
        {1:2:3}
        {\True 1:1:3}
        {2:1:3}
        {1:1:2}
        \loigiai{1 Ca, 1 C, 3 O.}
    \end{ex}
    %%%%%===========EX_9=======%%%%%
    \begin{ex}
        Tổng số nguyên tử trong phân tử khí cacbonic ($CO_2$) là:
        \choice
        {2}
        {\True 3}
        {4}
        {1}
        \loigiai{$1+2=3$.}
    \end{ex}
    %%%%%===========EX_10=======%%%%%
    \begin{ex}
        Số nguyên tử O trong $Al_2(SO_4)_3$ là:
        \choice
        {4}
        {7}
        {\True 12}
        {10}
        \loigiai{$4 \times 3 = 12$.}
    \end{ex}
    \Closesolutionfile{ans}
    \Closesolutionfile{ansex}

    \phan{Trắc nghiệm Đúng/Sai}
    \Opensolutionfile{ansex}[Ans/LGTF_D2]
    \Opensolutionfile{ansbook}[Ansbook/AnsTF_D2]
    \Opensolutionfile{ans}[Ans/Tempt_D2]
    %%%%%===========TF_1=======%%%%%
    \begin{ex}
        Về phân tử $NH_3$:
        \choiceTF
        {\True Chứa N và H.}
        {Chứa 3 nguyên tử N.}
        {Tổng số nguyên tử là 3.}
        {\True Tỉ lệ N:H là 1:3.}
        \loigiai{
            \begin{itemchoice}[T1,F2,F3,T4]
                \itemch Gồm N và H.
                \itemch Chỉ có 1 nguyên tử N.
                \itemch Tổng là 4 nguyên tử ($1+3$).
                \itemch Tỉ lệ 1:3.
            \end{itemchoice}
        }
    \end{ex}
    %%%%%===========TF_2=======%%%%%
    \begin{ex}
        $C_2H_5OH$:
        \choiceTF
        {\True Có 2 nguyên tử C.}
        {\True Có 6 nguyên tử H.}
        {Chỉ có 1 nguyên tố.}
        {Có 2 nguyên tử O.}
        \loigiai{
            \begin{itemchoice}[T1,T2,F3,F4]
                \itemch $C_2$ có 2 nguyên tử C.
                \itemch $H_5 + H = 6$ nguyên tử H.
                \itemch Có 3 nguyên tố C, H, O.
                \itemch Chỉ có 1 nguyên tử O.
            \end{itemchoice}
        }
    \end{ex}
    %%%%%===========TF_3=======%%%%%
    \begin{ex}
        $ZnCl_2$:
        \choiceTF
        {Có 2 nguyên tử Zn.}
        {\True Có 2 nguyên tử Cl.}
        {\True Tổng số nguyên tử là 3.}
        {Gồm nguyên tố Zn, C và I.}
        \loigiai{
            \begin{itemchoice}[F1,T2,T3,F4]
                \itemch Có 1 nguyên tử Zn.
                \itemch Có 2 nguyên tử Cl.
                \itemch $1+2=3$.
                \itemch Gồm nguyên tố Zn và Cl.
            \end{itemchoice}
        }
    \end{ex}
    %%%%%===========TF_4=======%%%%%
    \begin{ex}
        $C_6H_{12}O_6$:
        \choiceTF
        {\True Là hợp chất hữu cơ.}
        {Số nguyên tử H gấp đôi số nguyên tử C.}
        {\True Số nguyên tử O bằng số nguyên tử C.}
        {Chứa 3 nguyên tố.}
        \loigiai{
            \begin{itemchoice}[T1,F2,T3,F4]
                \itemch
                \itemch $H=12, C=6 \Rightarrow 12 = 2 \times 6$.
                \itemch $O=6, C=6$.
                \itemch Gồm C, H, O.
            \end{itemchoice}
        }
    \end{ex}
    %%%%%===========TF_5=======%%%%%
    \begin{ex}
        $Fe_3O_4$:
        \choiceTF
        {\True Có 3 nguyên tử Fe.}
        {Có 3 nguyên tử O.}
        {\True Tổng số nguyên tử là 7.}
        {Là đơn chất.}
        \loigiai{
            \begin{itemchoice}[T1,F2,T3,F4]
                \itemch Có 3 nguyên tử Fe.
                \itemch Có 4 nguyên tử O.
                \itemch $3+4=7$.
                \itemch Là hợp chất.
            \end{itemchoice}
        }
    \end{ex}
    %%%%%===========TF_6=======%%%%%
    \begin{ex}
        Phân tử $CH_4$ (Methane):
        \choiceTF
        {\True Là hợp chất hữu cơ.}
        {Có 5 nguyên tử trong phân tử.}
        {\True Tỉ lệ C:H là 1:4.}
        {Chứa oxy.}
        \loigiai{
            \begin{itemchoice}[T1,F2,T3,F4]
                \itemch
                \itemch $1+4=5$.
                \itemch
                \itemch Chỉ có C và H.
            \end{itemchoice}
        }
    \end{ex}
    %%%%%===========TF_7=======%%%%%
    \begin{ex}
        Muối ăn ($NaCl$):
        \choiceTF
        {\True Hợp chất vô cơ.}
        {Chứa nguyên tố Natri và Clo.}
        {\True Tỉ lệ nguyên tử 1:1.}
        {Là chất khí.}
        \loigiai{
            \begin{itemchoice}[T1,F2,T3,F4]
                \itemch
                \itemch
                \itemch
                \itemch Là chất rắn.
            \end{itemchoice}
        }
    \end{ex}
    %%%%%===========TF_8=======%%%%%
    \begin{ex}
        Oxy già ($H_2O_2$):
        \choiceTF
        {\True Có 2 nguyên tử O.}
        {Giống nước ($H_2O$).}
        {\True Tổng số nguyên tử là 4.}
        {Là đơn chất.}
        \loigiai{
            \begin{itemchoice}[T1,F2,T3,F4]
                \itemch
                \itemch Khác số nguyên tử O.
                \itemch $2+2=4$.
                \itemch Là hợp chất.
            \end{itemchoice}
        }
    \end{ex}
    %%%%%===========TF_9=======%%%%%
    \begin{ex}
        Về $Fe_2(SO_4)_3$:
        \choiceTF
        {\True Có 2 nguyên tử Fe.}
        {Có 4 nguyên tử S.}
        {\True Có 12 nguyên tử O.}
        {Tổng số nguyên tử là 15.}
        \loigiai{
            \begin{itemchoice}[T1,F2,T3,F4]
                \itemch
                \itemch Có 3 nguyên tử S.
                \itemch $4 \times 3 = 12$.
                \itemch $2 + 3 + 12 = 17$.
            \end{itemchoice}
        }
    \end{ex}
    %%%%%===========TF_10=======%%%%%
    \begin{ex}
        Đá vôi $CaCO_3$:
        \choiceTF
        {\True Gồm 3 nguyên tố: Ca, C, O.}
        {Có 2 nguyên tử C.}
        {\True Tổng số nguyên tử là 5.}
        {Là chất lỏng.}
        \loigiai{
            \begin{itemchoice}[T1,F2,T3,F4]
                \itemch
                \itemch Có 1 C.
                \itemch $1+1+3=5$.
                \itemch Là chất rắn.
            \end{itemchoice}
        }
    \end{ex}
    \Closesolutionfile{ans}
    \Closesolutionfile{ansbook}
    \Closesolutionfile{ansex}
    %%\bangdapanTF{AnsTF_D2}

    \phan{Trả lời ngắn}
    \Opensolutionfile{ansbth}[Ans/LGSA_D2]
    \Opensolutionfile{ansbt}[Ans/AnsSA_D2]
    %%%%%===========SA_1=======%%%%%
    \begin{bt}
        Số nguyên tử O trong $Mg(NO_3)_2$ là bao nhiêu?
        \shortans{6}
        \loigiai{6.}
    \end{bt}
    %%%%%===========SA_2=======%%%%%
    \begin{bt}
        Tổng số nguyên tử trong $C_{12}H_{22}O_{11}$?
        \shortans{45}
        \loigiai{45.}
    \end{bt}
    %%%%%===========SA_3=======%%%%%
    \begin{bt}
        Trong $CuSO_4 \cdot 5H_2O$, có tất cả bao nhiêu nguyên tử O?
        \shortans{9}
        \loigiai{9.}
    \end{bt}
    %%%%%===========SA_4=======%%%%%
    \begin{bt}
        Hợp chất $A_2B_5$ có tổng bao nhiêu nguyên tử?
        \shortans{7}
        \loigiai{7.}
    \end{bt}
    %%%%%===========SA_5=======%%%%%
    \begin{bt}
        Phân tử $Cl_2$ có bao nhiêu nguyên tử?
        \shortans{2}
        \loigiai{2.}
    \end{bt}
    %%%%%===========SA_6=======%%%%%
    \begin{bt}
        Số nguyên tử Carbon trong $C_2H_4$?
        \shortans{2}
        \loigiai{2 nguyên tử C.}
    \end{bt}
    %%%%%===========SA_7=======%%%%%
    \begin{bt}
        Tổng số nguyên tố trong thuốc tím ($KMnO_4$)?
        \shortans{3}
        \loigiai{K, Mn, O.}
    \end{bt}
    %%%%%===========SA_8=======%%%%%
    \begin{bt}
        Tổng số nguyên tử trong phân tử nước ($H_2O$)?
        \shortans{3}
        \loigiai{$2+1=3$.}
    \end{bt}
    %%%%%===========SA_9=======%%%%%
    \begin{bt}
        Trong $CaCl_2$ có bao nhiêu nguyên tử Clo?
        \shortans{2}
        \loigiai{2 nguyên tử Cl.}
    \end{bt}
    %%%%%===========SA_10=======%%%%%
    \begin{bt}
        Trong một phân tử Ozone ($O_3$) có bao nhiêu nguyên tử?
        \shortans{3}
        \loigiai{3 nguyên tử O.}
    \end{bt}
    \Closesolutionfile{ansbt}
    \Closesolutionfile{ansbth}

    \phan{Tự luận}
    \Opensolutionfile{ansbth}[Ans/LGBT_D2]
    \Opensolutionfile{ansbt}[Ans/AnsBT_D2]
    %%%%%===========BT_1=======%%%%%
    \begin{bt}
        Mô tả thành phần của $Al_2O_3$.
        \loigiai{Gồm 2 Al, 3 O.}
    \end{bt}
    %%%%%===========BT_2=======%%%%%
    \begin{bt}
        Đếm số lượng nguyên tử mỗi loại trong $Ca_3(PO_4)_2$.
        \loigiai{Ca: 3. P: 2. O: 8.}
    \end{bt}
    %%%%%===========BT_3=======%%%%%
    \begin{bt}
        So sánh số nguyên tử trong $O_2$ và $O_3$.
        \loigiai{2 và 3.}
    \end{bt}
    %%%%%===========BT_4=======%%%%%
    \begin{bt}
        Phân tích thành phần của Urea ($CO(NH_2)_2$).
        \loigiai{C: 1, O: 1, N: 2, H: 4.}
    \end{bt}
    %%%%%===========BT_5=======%%%%%
    \begin{bt}
        Thành phần của $CH_3COOH$.
        \loigiai{C: 2, H: 4, O: 2.}
    \end{bt}
    %%%%%===========BT_6=======%%%%%
    \begin{bt}
        Mô tả thành phần nguyên tử của phân tử $NH_3$ (Ammmonia).
        \loigiai{Gồm 1 nguyên tử N và 3 nguyên tử H.}
    \end{bt}
    %%%%%===========BT_7=======%%%%%
    \begin{bt}
        Phân tích số lượng nguyên tử mỗi loại trong $Ca(OH)_2$.
        \loigiai{Ca: 1. O: 2. H: 2.}
    \end{bt}
    %%%%%===========BT_8=======%%%%%
    \begin{bt}
        So sánh thành phần số nguyên tử của Ethylene ($C_2H_4$) và Acetylene ($C_2H_2$).
        \loigiai{Giống nhau số C (2). Khác nhau số H (4 so với 2).}
    \end{bt}
    %%%%%===========BT_9=======%%%%%
    \begin{bt}
        Aspirin có công thức $C_9H_8O_4$. Hãy cho biết số lượng nguyên tử của từng nguyên tố.
        \loigiai{C: 9, H: 8, O: 4.}
    \end{bt}
    %%%%%===========BT_10=======%%%%%
    \begin{bt}
        Vitamin C có công thức $C_6H_8O_6$. Tính tổng số nguyên tử trong một phân tử Vitamin C.
        \loigiai{$6+8+6=20$ nguyên tử.}
    \end{bt}
    \Closesolutionfile{ansbt}
    \Closesolutionfile{ansbth}
\end{dang}

%=============================================================
% DẠNG 3: TÍNH PHÂN TỬ KHỐI
%=============================================================
\begin{dang}{Tính phân tử khối}
    \phan{Trắc nghiệm nhiều lựa chọn}
    \Opensolutionfile{ansex}[Ans/LGEX_D3]
    \Opensolutionfile{ans}[Ans/AnsEX_D3]
    %%%%%===========EX_1=======%%%%%
    \begin{ex}
        KLPT của $H_2O$:
        \choice
        {17}
        {\True 18}
        {16}
        {20}
        \loigiai{18.}
    \end{ex}
    %%%%%===========EX_2=======%%%%%
    \begin{ex}
        KLPT của $H_2SO_4$:
        \choice
        {96}
        {49}
        {\True 98}
        {97}
        \loigiai{98.}
    \end{ex}
    %%%%%===========EX_3=======%%%%%
    \begin{ex}
        KLPT của $CaCO_3$:
        \choice
        {\True 100}
        {90}
        {110}
        {56}
        \loigiai{100.}
    \end{ex}
    %%%%%===========EX_4=======%%%%%
    \begin{ex}
        PTK nặng nhất:
        \choice
        {$H_2O$}
        {$CO_2$}
        {\True $H_2SO_4$}
        {$NaCl$}
        \loigiai{$H_2SO_4$.}
    \end{ex}
    %%%%%===========EX_5=======%%%%%
    \begin{ex}
        PTK của $CuSO_4$:
        \choice
        {150}
        {\True 160}
        {110}
        {180}
        \loigiai{160.}
    \end{ex}
    %%%%%===========EX_6=======%%%%%
    \begin{ex}
        Phân tử khối của khí Clo ($Cl_2$) là:
        \choice
        {35.5}
        {\True 71}
        {70}
        {35}
        \loigiai{$35.5 \times 2 = 71$.}
    \end{ex}
    %%%%%===========EX_7=======%%%%%
    \begin{ex}
        Phân tử khối của $NH_3$ là:
        \choice
        {\True 17}
        {16}
        {15}
        {18}
        \loigiai{$14 + 3 = 17$.}
    \end{ex}
    %%%%%===========EX_8=======%%%%%
    \begin{ex}
        Hợp chất nào có PTK là 44?
        \choice
        {$SO_2$}
        {\True $CO_2$}
        {$NO_2$}
        {$H_2S$}
        \loigiai{$12 + 32 = 44$.}
    \end{ex}
    %%%%%===========EX_9=======%%%%%
    \begin{ex}
        Muối ăn ($NaCl$) có phân tử khối là:
        \choice
        {58}
        {59}
        {\True 58.5}
        {60}
        \loigiai{$23 + 35.5 = 58.5$.}
    \end{ex}
    %%%%%===========EX_10=======%%%%%
    \begin{ex}
        PTK của Ozone ($O_3$) là:
        \choice
        {32}
        {16}
        {\True 48}
        {64}
        \loigiai{$16 \times 3 = 48$.}
    \end{ex}
    \Closesolutionfile{ans}
    \Closesolutionfile{ansex}

    \phan{Trắc nghiệm Đúng/Sai}
    \Opensolutionfile{ansex}[Ans/LGTF_D3]
    \Opensolutionfile{ansbook}[Ansbook/AnsTF_D3]
    \Opensolutionfile{ans}[Ans/Tempt_D3]
    %%%%%===========TF_1=======%%%%%
    \begin{ex}
        So sánh PTK:
        \choiceTF
        {\True $O_3 > O_2$.}
        {\True $CO_2 > N_2$.}
        {$H_2$ nhẹ nhất.}
        {$Cl_2 < KK$.}
        \loigiai{
            \begin{itemchoice}[T1,T2,F3,F4]
                \itemch $48 > 32$.
                \itemch $44 > 28$.
                \itemch $H_2$ (2) nhẹ nhất.
                \itemch $Cl_2=71 > 29$.
            \end{itemchoice}
        }
    \end{ex}
    %%%%%===========TF_2=======%%%%%
    \begin{ex}
        $NaCl$:
        \choiceTF
        {Na=23, Cl=35.}
        {\True Cle=35.5.}
        {\True PTK = 58.5.}
        {PTK = 58.}
        \loigiai{
            \begin{itemchoice}[F1,T2,T3,F4]
                \itemch Na=23.
                \itemch Cl=35.5.
                \itemch $23+35.5=58.5$.
                \itemch
            \end{itemchoice}
        }
    \end{ex}
    %%%%%===========TF_3=======%%%%%
    \begin{ex}
        $SO_2$:
        \choiceTF
        {\True S=32.}
        {PTK = 48.}
        {\True PTK = 64.}
        {Nặng hơn KK.}
        \loigiai{
            \begin{itemchoice}[T1,F2,T3,F4]
                \itemch S=32.
                \itemch $32+32=64$.
                \itemch
                \itemch $64 > 29$.
            \end{itemchoice}
        }
    \end{ex}
    %%%%%===========TF_4=======%%%%%
    \begin{ex}
        $Fe_2O_3$:
        \choiceTF
        {112.}
        {48.}
        {\True 160.}
        {72.}
        \loigiai{
            \begin{itemchoice}[F1,F2,T3,F4]
                \itemch $Fe(56) \times 2 = 112$.
                \itemch
                \itemch $112 + 48 = 160$.
                \itemch
            \end{itemchoice}
        }
    \end{ex}
    %%%%%===========TF_5=======%%%%%
    \begin{ex}
        Tính toán:
        \choiceTF
        {\True $NH_3=17$.}
        {$CH_4=18$.}
        {$NO_2=46$.}
        {$H_2S=34$.}
        \loigiai{
            \begin{itemchoice}[T1,F2,F3,F4]
                \itemch $14+3=17$.
                \itemch $12+4=16$.
                \itemch $14+32=46$.
                \itemch $2+32=34$.
            \end{itemchoice}
        }
    \end{ex}
    %%%%%===========TF_6=======%%%%%
    \begin{ex}
        So sánh $H_2SO_4$ và $H_3PO_4$:
        \choiceTF
        {\True Chúng có cùng phân tử khối.}
        {$H_2SO_4$ nhẹ hơn.}
        {\True $PTK = 98$.}
        {Cùng số nguyên tử O.}
        \loigiai{
            \begin{itemchoice}[T1,F2,T3,F4]
                \itemch Đều bằng 98.
                \itemch Bằng nhau.
                \itemch
                \itemch Đều có 4 Oxy.
            \end{itemchoice}
        }
    \end{ex}
    %%%%%===========TF_7=======%%%%%
    \begin{ex}
        $O_2$ và $N_2$:
        \choiceTF
        {\True $O_2$ nặng hơn $N_2$.}
        {PTK bằng nhau.}
        {\True $O_2=32, N_2=28$.}
        {$N_2$ nặng hơn không khí.}
        \loigiai{
            \begin{itemchoice}[T1,F2,T3,F4]
                \itemch $32 > 28$.
                \itemch
                \itemch
                \itemch $28 < 29$.
            \end{itemchoice}
        }
    \end{ex}
    %%%%%===========TF_8=======%%%%%
    \begin{ex}
        $CaCO_3$ và $KHCO_3$:
        \choiceTF
        {\True Cùng PTK là 100.}
        {$CaCO_3$ nặng hơn.}
        {\True $Ca=40, K=39, H=1$.}
        {Cả 2 đều chứa C.}
        \loigiai{
            \begin{itemchoice}[T1,F2,T3,F4]
                \itemch
                \itemch
                \itemch $39+1=40$.
                \itemch
            \end{itemchoice}
        }
    \end{ex}
    %%%%%===========TF_9=======%%%%%
    \begin{ex}
        Axit HCl:
        \choiceTF
        {\True Phân tử khối là 36.5.}
        {Nặng hơn $CO_2$.}
        {\True Nhẹ hơn khí clo ($Cl_2$).}
        {Nặng bằng $F_2$ (38).}
        \loigiai{
            \begin{itemchoice}[T1,F2,T3,F4]
                \itemch $1+35.5$.
                \itemch $36.5 < 44$.
                \itemch $36.5 < 71$.
                \itemch $36.5 \ne 38$.
            \end{itemchoice}
        }
    \end{ex}
    %%%%%===========TF_10=======%%%%%
    \begin{ex}
        Bazơ KOH:
        \choiceTF
        {\True PTK = 56.}
        {Nặng bằng $CaO$.}
        {\True Nặng hơn $NaOH$ (40).}
        {Nhẹ hơn $H_2O$.}
        \loigiai{
            \begin{itemchoice}[T1,F2,T3,F4]
                \itemch $39+16+1$.
                \itemch $CaO=40+16=56$.
                \itemch $56 > 40$.
                \itemch $56 > 18$.
            \end{itemchoice}
        }
    \end{ex}
    \Closesolutionfile{ans}
    \Closesolutionfile{ansbook}
    \Closesolutionfile{ansex}
    %%\bangdapanTF{AnsTF_D3}

    \phan{Trả lời ngắn}
    \Opensolutionfile{ansbth}[Ans/LGSA_D3]
    \Opensolutionfile{ansbt}[Ans/AnsSA_D3]
    %%%%%===========SA_1=======%%%%%
    \begin{bt}
        PTK của $KNO_3$?
        \shortans{101}
        \loigiai{101.}
    \end{bt}
    %%%%%===========SA_2=======%%%%%
    \begin{bt}
        PTK của $Al_2(SO_4)_3$?
        \shortans{342}
        \loigiai{342.}
    \end{bt}
    %%%%%===========SA_3=======%%%%%
    \begin{bt}
        PTK của $C_{12}H_{22}O_{11}$?
        \shortans{342}
        \loigiai{342.}
    \end{bt}
    %%%%%===========SA_4=======%%%%%
    \begin{bt}
        Chất $X$ có PTK bằng 64 và gồm 1 nguyên tử S, 2 nguyên tử O. Tổng số nguyên tử trong một phân tử X là bao nhiêu?
        \shortans{3}
        \loigiai{$SO_2$ có $1+2=3$ nguyên tử.}
    \end{bt}
    %%%%%===========SA_5=======%%%%%
    \begin{bt}
        Tổng PTK $CO + CO_2$?
        \shortans{72}
        \loigiai{72.}
    \end{bt}
    %%%%%===========SA_6=======%%%%%
    \begin{bt}
        Tính phân tử khối của $NaOH$?
        \shortans{40}
        \loigiai{40.}
    \end{bt}
    %%%%%===========SA_7=======%%%%%
    \begin{bt}
        Tính phân tử khối của $SO_2$?
        \shortans{64}
        \loigiai{64.}
    \end{bt}
    %%%%%===========SA_8=======%%%%%
    \begin{bt}
        Tính PTK của khí Metan ($CH_4$)?
        \shortans{16}
        \loigiai{16.}
    \end{bt}
    %%%%%===========SA_9=======%%%%%
    \begin{bt}
        Nguyên tử khối của Sắt ($Fe$)?
        \shortans{56}
        \loigiai{56.}
    \end{bt}
    %%%%%===========SA_10=======%%%%%
    \begin{bt}
        Tính PTK của vôi sống ($CaO$)?
        \shortans{56}
        \loigiai{56.}
    \end{bt}
    \Closesolutionfile{ansbt}
    \Closesolutionfile{ansbth}

    \phan{Tự luận}
    \Opensolutionfile{ansbth}[Ans/LGBT_D3]
    \Opensolutionfile{ansbt}[Ans/AnsBT_D3]
    %%%%%===========BT_1=======%%%%%
    \begin{bt}
        Tính PTK: $HCl, NaOH, BaCO_3$.
        \loigiai{36.5, 40, 197.}
    \end{bt}
    %%%%%===========BT_2=======%%%%%
    \begin{bt}
        $X$ nặng gấp đôi $O_2$. Tìm X.
        \loigiai{64.}
    \end{bt}
    %%%%%===========BT_3=======%%%%%
    \begin{bt}
        Tổng khối lượng 1 $H_2SO_4$ và 2 $H_2O$.
        \loigiai{134.}
    \end{bt}
    %%%%%===========BT_4=======%%%%%
    \begin{bt}
        $R_2O$ có PTK=62. Tìm R.
        \loigiai{Na.}
    \end{bt}
    %%%%%===========BT_5=======%%%%%
    \begin{bt}
        Sắp xếp PTK: $H_2, N_2, O_2, Cl_2$.
        \loigiai{Tăng dần.}
    \end{bt}
    %%%%%===========BT_6=======%%%%%
    \begin{bt}
        Tính và so sánh PTK của $K_2O$ và $MgO$.
        \loigiai{$K_2O=94, MgO=40. K_2O$ nặng hơn.}
    \end{bt}
    %%%%%===========BT_7=======%%%%%
    \begin{bt}
        So sánh phân tử khối của $CO_2$ và $SO_2$.
        \loigiai{$CO_2=44, SO_2=64. SO_2$ nặng hơn 1.45 lần.}
    \end{bt}
    %%%%%===========BT_8=======%%%%%
    \begin{bt}
        Tính tổng phân tử khối của 1 phân tử $NaOH$ và 1 phân tử $HCl$.
        \loigiai{$40 + 36.5 = 76.5$ đvC.}
    \end{bt}
    %%%%%===========BT_9=======%%%%%
    \begin{bt}
        Tìm nguyên tố X biết hợp chất $XO$ có PTK bằng 40.
        \loigiai{$X + 16 = 40 \Rightarrow X = 24$ (Mg).}
    \end{bt}
    %%%%%===========BT_10=======%%%%%
    \begin{bt}
        Sắp xếp theo chiều tăng dần PTK: $LiOH, NaOH, KOH$.
        \loigiai{$LiOH(24) < NaOH(40) < KOH(56)$.}
    \end{bt}
    \Closesolutionfile{ansbt}
    \Closesolutionfile{ansbth}
\end{dang}

%=============================================================
% DẠNG 4: SO SÁNH KHỐI LƯỢNG PHÂN TỬ
%=============================================================
\begin{dang}{So sánh khối lượng phân tử}
    \phan{Trắc nghiệm nhiều lựa chọn}
    \Opensolutionfile{ansex}[Ans/LGEX_D4]
    \Opensolutionfile{ans}[Ans/AnsEX_D4]
    %%%%%===========EX_1=======%%%%%
    \begin{ex}
        Khí nặng gấp 22 lần $H_2$ là:
        \choice
        {$O_2$}
        {\True $CO_2$}
        {$N_2$}
        {$CH_4$}
        \loigiai{$CO_2$.}
    \end{ex}
    %%%%%===========EX_2=======%%%%%
    \begin{ex}
        Tỉ khối $O_2/KK$ là:
        \choice
        {0.9}
        {\True 1.1}
        {1.5}
        {2.0}
        \loigiai{1.1.}
    \end{ex}
    %%%%%===========EX_3=======%%%%%
    \begin{ex}
        Khí nhẹ nhất:
        \choice
        {$CO$}
        {$CO_2$}
        {\True $H_2$}
        {$N_2$}
        \loigiai{$H_2$.}
    \end{ex}
    %%%%%===========EX_4=======%%%%%
    \begin{ex}
        So sánh $CO_2$ và $N_2O$:
        \choice
        {$CO_2 > $}
        {$N_2O > $}
        {\True Bằng nhau}
        {Không so được}
        \loigiai{Bằng.}
    \end{ex}
    %%%%%===========EX_5=======%%%%%
    \begin{ex}
        PTK bằng $O_2$ (32), X là:
        \choice
        {$N_2$}
        {\True $CH_3OH$}
        {$CH_4$}
        {$H_2O$}
        \loigiai{Methanol (32) hoặc S (32).}
    \end{ex}
    %%%%%===========EX_6=======%%%%%
    \begin{ex}
        Tỉ khối của khí Helium so với khí Hydro ($d_{He/H_2}$) là:
        \choice
        {4}
        {\True 2}
        {8}
        {0.5}
        \loigiai{$4/2 = 2$.}
    \end{ex}
    %%%%%===========EX_7=======%%%%%
    \begin{ex}
        Tỉ khối của không khí so với khí Hidro xấp xỉ bằng:
        \choice
        {29}
        {14}
        {\True 14.5}
        {15}
        \loigiai{$29/2 = 14.5$.}
    \end{ex}
    %%%%%===========EX_8=======%%%%%
    \begin{ex}
        Cho biết $d_{CO/X} = 1$. Khí X là:
        \choice
        {$O_2$}
        {$CO_2$}
        {\True $N_2$}
        {$H_2$}
        \loigiai{$CO$ (28) và $N_2$ (28) nặng bằng nhau.}
    \end{ex}
    %%%%%===========EX_9=======%%%%%
    \begin{ex}
        Khí nào sau đây nhẹ hơn không khí ($M_{KK} \approx 29$)?
        \choice
        {$O_2$}
        {$Cl_2$}
        {\True $NH_3$}
        {$CO_2$}
        \loigiai{$NH_3 (17) < 29$.}
    \end{ex}
    %%%%%===========EX_10=======%%%%%
    \begin{ex}
        Khí nào nặng hơn không khí?
        \choice
        {$H_2$}
        {$CH_4$}
        {$N_2$}
        {\True $O_2$}
        \loigiai{$O_2 (32) > 29$.}
    \end{ex}
    \Closesolutionfile{ans}
    \Closesolutionfile{ansex}

    \phan{Trắc nghiệm Đúng/Sai}
    \Opensolutionfile{ansex}[Ans/LGTF_D4]
    \Opensolutionfile{ansbook}[Ansbook/AnsTF_D4]
    \Opensolutionfile{ans}[Ans/Tempt_D4]
    %%%%%===========TF_1=======%%%%%
    \begin{ex}
        $CO_2$:
        \choiceTF
        {\True Nặng hơn KK.}
        {Nhẹ hơn Hydro.}
        {\True Nặng gấp 1.5 lần KK.}
        {Nhẹ hơn Clo.}
        \loigiai{
            \begin{itemchoice}[T1,F2,T3,F4]
                \itemch $44 > 29$.
                \itemch $44 > 2$.
                \itemch $44 \approx 1.5 \times 29$.
                \itemch $44 < 71$.
            \end{itemchoice}
        }
    \end{ex}
    %%%%%===========TF_2=======%%%%%
    \begin{ex}
        Tỉ khối:
        \choiceTF
        {\True $d>1$: A nặng hơn B.}
        {$d<1$: Úp bình.}
        {\True $d_{O2/H2}=16$.}
        {$d_{N2/KK} \approx 1$.}
        \loigiai{
            \begin{itemchoice}[T1,F2,T3,F4]
                \itemch
                \itemch Phải đặt ngửa bình.
                \itemch $32/2=16$.
                \itemch $28 \approx 29$.
            \end{itemchoice}
        }
    \end{ex}
    %%%%%===========TF_3=======%%%%%
    \begin{ex}
        $CH_4$ và $SO_2$:
        \choiceTF
        {\True $CH_4 < SO_2$.}
        {$CH_4/SO_2 = 1/4$.}
        {\True $CH_4 < KK$.}
        {$SO_2 < KK$.}
        \loigiai{
            \begin{itemchoice}[T1,F2,T3,F4]
                \itemch $16 < 64$.
                \itemch $16/64 = 0.25$.
                \itemch $16 < 29$.
                \itemch $64 > 29$.
            \end{itemchoice}
        }
    \end{ex}
    %%%%%===========TF_4=======%%%%%
    \begin{ex}
        $CH_4$ trong hầm cầu:
        \choiceTF
        {\True Nhẹ hơn KK.}
        {Nặng hơn KK.}
        {Bằng KK.}
        {Độc.}
        \loigiai{
            \begin{itemchoice}[T1,F2,F3,F4]
                \itemch $16 < 29$.
                \itemch $16 > 29$.
                \itemch
                \itemch
            \end{itemchoice}
        }
    \end{ex}
    %%%%%===========TF_5=======%%%%%
    \begin{ex}
        $Cl_2$:
        \choiceTF
        {\True Nặng hơn KK.}
        {\True $d=35.5$ so với H2.}
        {Tích tụ chỗ thấp.}
        {Nhẹ hơn CO2.}
        \loigiai{
            \begin{itemchoice}[T1,T2,F3,F4]
                \itemch $71 > 29$.
                \itemch $71/2 = 35.5$.
                \itemch Nặng hơn không khí nên tích tụ bên dưới.
                \itemch $71 > 44$ nên nặng hơn $CO_2$.
            \end{itemchoice}
        }
    \end{ex}
    %%%%%===========TF_6=======%%%%%
    \begin{ex}
        So sánh $SO_2$ và $NO_2$:
        \choiceTF
        {\True $SO_2$ nặng hơn.}
        {Hai khí đều nhẹ hơn không khí.}
        {\True $SO_2$ gây mưa axit.}
        {$d_{SO_2/NO_2} > 1.5$.}
        \loigiai{
            \begin{itemchoice}[T1,F2,T3,F4]
                \itemch $64 > 46$.
                \itemch Đều nặng hơn KK.
                \itemch
                \itemch $64/46 \approx 1.39$.
            \end{itemchoice}
        }
    \end{ex}
    %%%%%===========TF_7=======%%%%%
    \begin{ex}
        Khí Helium ($He$):
        \choiceTF
        {\True Nặng gấp 2 lần khí Hidro.}
        {Nhẹ hơn không khí.}
        {\True Dùng bơm bóng bay an toàn hơn Hidro.}
        {Dễ cháy nổ.}
        \loigiai{
            \begin{itemchoice}[T1,F2,T3,F4]
                \itemch $4/2=2$.
                \itemch $4 < 29$.
                \itemch Helium là khí trơ, không cháy.
                \itemch Hidro mới dễ nổ.
            \end{itemchoice}
        }
    \end{ex}
    %%%%%===========TF_8=======%%%%%
    \begin{ex}
        Khí $N_2$ (Nitơ) và $CO$ (Cacbon monoxit):
        \choiceTF
        {\True Có cùng phân tử khối.}
        {Có cùng tính chất hóa học.}
        {\True Đều nhẹ hơn không khí.}
        {$CO$ rất độc.}
        \loigiai{
            \begin{itemchoice}[T1,F2,T3,F4]
                \itemch Đều bằng 28.
                \itemch Khác nhau.
                \itemch $28 < 29$.
                \itemch
            \end{itemchoice}
        }
    \end{ex}
    %%%%%===========TF_9=======%%%%%
    \begin{ex}
        Khí $H_2S$ (Hidro sunfua):
        \choiceTF
        {\True Nặng hơn không khí.}
        {Có mùi trứng thối.}
        {\True Nặng hơn khí Oxi.}
        {Thu khí bằng cách đặt úp bình.}
        \loigiai{
            \begin{itemchoice}[T1,F2,T3,F4]
                \itemch $34 > 29$.
                \itemch
                \itemch $34 > 32$.
                \itemch Phải ngửa bình vì nặng hơn KK.
            \end{itemchoice}
        }
    \end{ex}
    %%%%%===========TF_10=======%%%%%
    \begin{ex}
        Khí Clo ($Cl_2$) và Flo ($F_2$):
        \choiceTF
        {\True $Cl_2$ nặng hơn $F_2$.}
        {Cả hai đều nhẹ hơn không khí.}
        {\True $d_{Cl_2/KK} \approx 2.45$.}
        {$F_2$ có PTK là 19.}
        \loigiai{
            \begin{itemchoice}[T1,F2,T3,F4]
                \itemch $71 > 38$.
                \itemch Đều nặng hơn.
                \itemch $71/29 \approx 2.45$.
                \itemch $F_2 = 19 \times 2 = 38$.
            \end{itemchoice}
        }
    \end{ex}
    \Closesolutionfile{ans}
    \Closesolutionfile{ansbook}
    \Closesolutionfile{ansex}
    %%\bangdapanTF{AnsTF_D4}

    \phan{Trả lời ngắn}
    \Opensolutionfile{ansbth}[Ans/LGSA_D4]
    \Opensolutionfile{ansbt}[Ans/AnsSA_D4]
    %%%%%===========SA_1=======%%%%%
    \begin{bt}
        Tỉ khối $N_2/H_2$?
        \shortans{14}
        \loigiai{14.}
    \end{bt}
    %%%%%===========SA_2=======%%%%%
    \begin{bt}
        Chất $X$ có $d_{X/H_2} = 8$. Phân tử khối của X là bao nhiêu?
        \shortans{16}
        \loigiai{$8 \times 2 = 16$.}
    \end{bt}
    %%%%%===========SA_3=======%%%%%
    \begin{bt}
        $SO_3$ nặng gấp bao nhiêu lần $O_2$?
        \shortans{2.5}
        \loigiai{2.5.}
    \end{bt}
    %%%%%===========SA_4=======%%%%%
    \begin{bt}
        M trung bình $O_2, N_2$ (1:1)?
        \shortans{30}
        \loigiai{30.}
    \end{bt}
    %%%%%===========SA_5=======%%%%%
    \begin{bt}
        M không khí?
        \shortans{29}
        \loigiai{29.}
    \end{bt}
    %%%%%===========SA_6=======%%%%%
    \begin{bt}
        Tỉ khối của khí Oxi so với khí Heli ($O_2/He$)?
        \shortans{8}
        \loigiai{$32/4 = 8$.}
    \end{bt}
    %%%%%===========SA_7=======%%%%%
    \begin{bt}
        Tỉ khối hơi của $SO_2$ so với $O_2$?
        \shortans{2}
        \loigiai{$64/32 = 2$.}
    \end{bt}
    %%%%%===========SA_8=======%%%%%
    \begin{bt}
        Tỉ khối của khí Metan ($CH_4$) so với khí Heli?
        \shortans{4}
        \loigiai{$16/4 = 4$.}
    \end{bt}
    %%%%%===========SA_9=======%%%%%
    \begin{bt}
        Tỉ khối của khí $CO$ so với khí $N_2$?
        \shortans{1}
        \loigiai{$28/28 = 1$.}
    \end{bt}
    %%%%%===========SA_10=======%%%%%
    \begin{bt}
        Tỉ khối của khí $NO_2$ so với khí $H_2$?
        \shortans{23}
        \loigiai{$46/2 = 23$.}
    \end{bt}
    \Closesolutionfile{ansbt}
    \Closesolutionfile{ansbth}

    \phan{Tự luận}
    \Opensolutionfile{ansbth}[Ans/LGBT_D4]
    \Opensolutionfile{ansbt}[Ans/AnsBT_D4]
    %%%%%===========BT_1=======%%%%%
    \begin{bt}
        Khí nào bay lên, khí nào chìm: $H_2, Cl_2$.
        \loigiai{H2 bay, Cl2 chìm.}
    \end{bt}
    %%%%%===========BT_2=======%%%%%
    \begin{bt}
        Tính M biết d.
        \loigiai{Tính theo công thức.}
    \end{bt}
    %%%%%===========BT_3=======%%%%%
    \begin{bt}
        Tại sao $CO_2$ tích tụ đáy giếng?
        \loigiai{Nặng hơn không khí.}
    \end{bt}
    %%%%%===========BT_4=======%%%%%
    \begin{bt}
        Tìm X trong $XO_2$ biết d.
        \loigiai{S.}
    \end{bt}
    %%%%%===========BT_5=======%%%%%
    \begin{bt}
        So sánh $NH_3$ và $H_2O$.
        \loigiai{Nước nặng hơn.}
    \end{bt}
    %%%%%===========BT_6=======%%%%%
    \begin{bt}
        Tính tỉ khối của khí Amoniac ($NH_3$) so với không khí. Nặng hay nhẹ hơn?
        \loigiai{$17/29 \approx 0.59$. Nhẹ hơn không khí.}
    \end{bt}
    %%%%%===========BT_7=======%%%%%
    \begin{bt}
        Trong các khí: $H_2, He, CH_4, N_2$. Khí nào nặng nhất, khí nào nhẹ nhất?
        \loigiai{Nặng nhất: $N_2$ (28). Nhẹ nhất: $H_2$ (2).}
    \end{bt}
    %%%%%===========BT_8=======%%%%%
    \begin{bt}
        Tính tỉ khối của khí Clo ($Cl_2$) so với khí Oxi ($O_2$).
        \loigiai{$71/32 \approx 2.22$.}
    \end{bt}
    %%%%%===========BT_9=======%%%%%
    \begin{bt}
        Tìm phân tử khối của khí X, biết tỉ khối của X so với $O_2$ là 2.
        \loigiai{$M_X = 2 \times 32 = 64$ ($SO_2$).}
    \end{bt}
    %%%%%===========BT_10=======%%%%%
    \begin{bt}
        Vì sao người ta dùng Helium để bơm bóng bay thay vì Hidro dù Hidro nhẹ hơn?
        \loigiai{Vì Hidro dễ gây cháy nổ, còn Helium là khí trơ an toàn.}
    \end{bt}
    \Closesolutionfile{ansbt}
    \Closesolutionfile{ansbth}
\end{dang}

%=============================================================
% DẠNG 5
%=============================================================
\begin{dang}{Lập công thức hóa học}
    \phan{Trắc nghiệm nhiều lựa chọn}
    \Opensolutionfile{ansex}[Ans/LGEX_D5]
    \Opensolutionfile{ans}[Ans/AnsEX_D5]
    %%%%%===========EX_1=======%%%%%
    \begin{ex}
        75\% C, 25\% H. Công thức?
        \choice
        {$C_2H_2$}
        {\True $CH_4$}
        {$C_2H_6$}
        {$C_3H_8$}
        \loigiai{CH4.}
    \end{ex}
    %%%%%===========EX_2=======%%%%%
    \begin{ex}
        50\% S. Oxit?
        \choice
        {$SO$}
        {\True $SO_2$}
        {$SO_3$}
        {$S_2O$}
        \loigiai{SO2.}
    \end{ex}
    %%%%%===========EX_3=======%%%%%
    \begin{ex}
        Ca, C, O (40\% Ca). PTK=100.
        \choice
        {$CaO$}
        {$CaC_2$}
        {\True $CaCO_3$}
        {$Ca(OH)_2$}
        \loigiai{CaCO3.}
    \end{ex}
    %%%%%===========EX_4=======%%%%%
    \begin{ex}
        Na:Cl = 1:1.
        \choice
        {$Na_2Cl$}
        {\True $NaCl$}
        {$NaCl_2$}
        {$Na_2Cl_2$}
        \loigiai{NaCl.}
    \end{ex}
    %%%%%===========EX_5=======%%%%%
    \begin{ex}
        \%H trong nước?
        \choice
        {2\%}
        {89\%}
        {\True 11\%}
        {50\%}
        \loigiai{11\%.}
    \end{ex}
    %%%%%===========EX_6=======%%%%%
    \begin{ex}
        Thành phần phần trăm khối lượng của C trong $CO_2$ là:
        \choice
        {50\%}
        {\True 27.3\%}
        {72.7\%}
        {30\%}
        \loigiai{$12/44 \times 100\% \approx 27.3\%$.}
    \end{ex}
    %%%%%===========EX_7=======%%%%%
    \begin{ex}
        Thành phần phần trăm khối lượng của O trong $Fe_2O_3$ là:
        \choice
        {70\%}
        {\True 30\%}
        {40\%}
        {60\%}
        \loigiai{$(16 \times 3)/160 = 30\%$.}
    \end{ex}
    %%%%%===========EX_8=======%%%%%
    \begin{ex}
        Lưu huỳnh (S) hóa trị VI. Công thức oxit tương ứng là:
        \choice
        {$SO_2$}
        {\True $SO_3$}
        {$S_2O_3$}
        {$SO$}
        \loigiai{S(VI) và O(II) $\Rightarrow SO_3$.}
    \end{ex}
    %%%%%===========EX_9=======%%%%%
    \begin{ex}
        Công thức hóa học của Đồng(II) oxit là:
        \choice
        {$Cu_2O$}
        {\True $CuO$}
        {$CuO_2$}
        {$Cu_2O_3$}
        \loigiai{Cu(II) và O(II) $\Rightarrow CuO$.}
    \end{ex}
    %%%%%===========EX_10=======%%%%%
    \begin{ex}
        \%H trong phân tử $CH_4$ là:
        \choice
        {\True 25\%}
        {75\%}
        {20\%}
        {80\%}
        \loigiai{$4/16 = 25\%$.}
    \end{ex}
    \Closesolutionfile{ans}
    \Closesolutionfile{ansex}

    \phan{Trắc nghiệm Đúng/Sai}
    \Opensolutionfile{ansex}[Ans/LGTF_D5]
    \Opensolutionfile{ansbook}[Ansbook/AnsTF_D5]
    \Opensolutionfile{ans}[Ans/Tempt_D5]
    %%%%%===========TF_1=======%%%%%
    \begin{ex}
        Hợp chất PTK=44, C và O:
        \choiceTF
        {\True 27\% C.}
        {\True 73\% O.}
        {CTHH là CO.}
        {\True CTHH là $CO_2$.}
        \loigiai{
            \begin{itemchoice}[T1,T2,F3,T4]
                \itemch $12/44 \approx 27.27\%$.
                \itemch $32/44 \approx 72.72\%$.
                \itemch CO có phân tử khối là 28.
                \itemch $CO_2$ có PTK là 44 và thỏa mãn thành phần phần trăm.
            \end{itemchoice}
        }
    \end{ex}
    %%%%%===========TF_2=======%%%%%
    \begin{ex}
        $CuO$:
        \choiceTF
        {Cu (I).}
        {\True PTK80.}
        {\True 80\% Cu.}
        {Oxit acid.}
        \loigiai{
            \begin{itemchoice}[F1,T2,T3,F4]
                \itemch Trong $CuO$, Cu có hóa trị II.
                \itemch $64 + 16 = 80$.
                \itemch $\%Cu = \frac{64}{80} \times 100\% = 80\%$.
                \itemch Oxit kim loại thường là oxit bazơ.
            \end{itemchoice}
        }
    \end{ex}
    %%%%%===========TF_3=======%%%%%
    \begin{ex}
        $H_2SO_4$:
        \choiceTF
        {PTK 96.}
        {\True \%S 32.6\%.}
        {\True \%O max.}
        {4 H.}
        \loigiai{
            \begin{itemchoice}[F1,T2,T3,F4]
                \itemch $M(H_2SO_4) = 2 + 32 + 64 = 98$.
                \itemch $\%S = \frac{32}{98} \times 100\% \approx 32.65\%$.
                \itemch $\%O = \frac{64}{98} \times 100\% \approx 65.3\%$ (lớn nhất).
                \itemch Có 2 nguyên tử H.
            \end{itemchoice}
        }
    \end{ex}
    %%%%%===========TF_4=======%%%%%
    \begin{ex}
        Lập CTHH:
        \choiceTF
        {Cần hóa trị.}
        {Cần \%.}
        {\True Có thể từ \% và PTK.}
        {Cần trạng thái.}
        \loigiai{
            \begin{itemchoice}[F1,F2,T3,F4]
                \itemch Thiếu thông tin.
                \itemch
                \itemch Có PTK và phần trăm tìm được số nguyên tử.
                \itemch
            \end{itemchoice}
        }
    \end{ex}
    %%%%%===========TF_5=======%%%%%
    \begin{ex}
        $N_xO_y$ (46).
        \choiceTF
        {\True x1 y2.}
        {NO.}
        {\True NO2.}
        {\%N > \%O.}
        \loigiai{
            \begin{itemchoice}[T1,F2,T3,F4]
                \itemch $14x + 16y = 46 \Rightarrow x=1, y=2$ (vì $NO_2=46$).
                \itemch $NO = 30$.
                \itemch $NO_2$.
                \itemch $N=14 < 2 \times 16$ nên $\%N < \%O$.
            \end{itemchoice}
        }
    \end{ex}
    %%%%%===========TF_6=======%%%%%
    \begin{ex}
        Cho CTHH của khí Cacbon monoxit ($CO$) và khí Cacbonic ($CO_2$):
        \choiceTF
        {\True Đều tạo từ C và O.}
        {Thành phần \%C như nhau.}
        {\True $CO_2$ giàu Oxi hơn.}
        {Đều là oxit bazơ.}
        \loigiai{
            \begin{itemchoice}[T1,F2,T3,F4]
                \itemch
                \itemch $\%C$(CO) > $\%C$($CO_2$).
                \itemch
                \itemch Oxit phi kim thường là oxit axit hoặc trung tính.
            \end{itemchoice}
        }
    \end{ex}
    %%%%%===========TF_7=======%%%%%
    \begin{ex}
        Hợp chất $Fe_2O_3$ (Sắt(III) oxit):
        \choiceTF
        {\True Sắt có hóa trị III.}
        {Oxi chiếm 70\% khối lượng.}
        {\True PTK = 160.}
        {Là hợp chất hữu cơ.}
        \loigiai{
            \begin{itemchoice}[T1,F2,T3,F4]
                \itemch
                \itemch 30\% O (48/160).
                \itemch
                \itemch Vô cơ.
            \end{itemchoice}
        }
    \end{ex}
    %%%%%===========TF_8=======%%%%%
    \begin{ex}
        Công thức muối ăn $NaCl$:
        \choiceTF
        {\True Na hóa trị I, Cl hóa trị I.}
        {\%Na > \%Cl về khối lượng.}
        {\True PTK = 58.5.}
        {Trong phân tử có 3 nguyên tử.}
        \loigiai{
            \begin{itemchoice}[T1,F2,T3,F4]
                \itemch
                \itemch $23 < 35.5$.
                \itemch
                \itemch Có 2 nguyên tử.
            \end{itemchoice}
        }
    \end{ex}
    %%%%%===========TF_9=======%%%%%
    \begin{ex}
        Nhôm oxit $Al_2O_3$:
        \choiceTF
        {\True Al hóa trị III.}
        {PTK = 100.}
        {\True \%Al = 52.9\%.}
        {Oxi chiếm khối lượng lớn hơn Al.}
        \loigiai{
            \begin{itemchoice}[T1,F2,T3,F4]
                \itemch
                \itemch 102.
                \itemch $54/102 \approx 52.9\%$.
                \itemch $48 < 54$.
            \end{itemchoice}
        }
    \end{ex}
    %%%%%===========TF_10=======%%%%%
    \begin{ex}
        So sánh $FeO$ và $Fe_2O_3$:
        \choiceTF
        {\True $FeO$: Sắt hóa trị II.}
        {$Fe_2O_3$ giàu sắt hơn $FeO$ (theo \% khối lượng).}
        {\True $FeO$ có PTK nhỏ hơn.}
        {Đều là oxit sắt.}
        \loigiai{
            \begin{itemchoice}[T1,F2,T3,F4]
                \itemch
                \itemch $\%Fe(FeO) \approx 77.7\% > \%Fe(Fe_2O_3) = 70\%$.
                \itemch $72 < 160$.
                \itemch
            \end{itemchoice}
        }
    \end{ex}
    \Closesolutionfile{ans}
    \Closesolutionfile{ansbook}
    \Closesolutionfile{ansex}
    %%\bangdapanTF{AnsTF_D5}

    \phan{Trả lời ngắn}
    \Opensolutionfile{ansbth}[Ans/LGSA_D5]
    \Opensolutionfile{ansbt}[Ans/AnsSA_D5]
    %%%%%===========SA_1=======%%%%%
    \begin{bt}
        Hợp chất $XY_2$ có PTK=44. X=12. Tính nguyên tử khối của Y.
        \shortans{16}
        \loigiai{$12+2Y=44 \rightarrow 2Y=32 \rightarrow Y=16$ (O).}
    \end{bt}
    %%%%%===========SA_2=======%%%%%
    \begin{bt}
        $Fe_xO_y$ (160). x=2. y?
        \shortans{3}
        \loigiai{3.}
    \end{bt}
    %%%%%===========SA_3=======%%%%%
    \begin{bt}
        \%H trong $CH_4$?
        \shortans{25}
        \loigiai{25.}
    \end{bt}
    %%%%%===========SA_4=======%%%%%
    \begin{bt}
        1N, xO. PTK 46. x?
        \shortans{2}
        \loigiai{2.}
    \end{bt}
    %%%%%===========SA_5=======%%%%%
    \begin{bt}
        \%K trong $K_2O$?
        \shortans{83}
        \loigiai{83.}
    \end{bt}
    %%%%%===========SA_6=======%%%%%
    \begin{bt}
        \%O trong $MgO$?
        \shortans{40}
        \loigiai{$16/40 = 40\%$.}
    \end{bt}
    %%%%%===========SA_7=======%%%%%
    \begin{bt}
        \%C trong $C_2H_4$?
        \shortans{86}
        \loigiai{$24/28 \approx 85.7\%$.}
    \end{bt}
    %%%%%===========SA_8=======%%%%%
    \begin{bt}
        Tìm x trong $SO_x$ biết PTK = 80?
        \shortans{3}
        \loigiai{$32 + 16x = 80 \Rightarrow x=3$.}
    \end{bt}
    %%%%%===========SA_9=======%%%%%
    \begin{bt}
        Tìm y trong $N_yO$ biết PTK = 44?
        \shortans{2}
        \loigiai{$14y + 16 = 44 \Rightarrow y=2$.}
    \end{bt}
    %%%%%===========SA_10=======%%%%%
    \begin{bt}
        Thành phần \% khối lượng của Fe trong $FeO$?
        \shortans{78}
        \loigiai{$56/72 \approx 77.8\%$.}
    \end{bt}
    \Closesolutionfile{ansbt}
    \Closesolutionfile{ansbth}

    \phan{Tự luận}
    \Opensolutionfile{ansbth}[Ans/LGBT_D5]
    \Opensolutionfile{ansbt}[Ans/AnsBT_D5]
    %%%%%===========BT_1=======%%%%%
    \begin{bt}
        S(40), O(60). PTK 80.
        \loigiai{SO3.}
    \end{bt}
    %%%%%===========BT_2=======%%%%%
    \begin{bt}
        Fe(70). PTK 160.
        \loigiai{Fe2O3.}
    \end{bt}
    %%%%%===========BT_3=======%%%%%
    \begin{bt}
        Na, O. PTK 62.
        \loigiai{Na2O.}
    \end{bt}
    %%%%%===========BT_4=======%%%%%
    \begin{bt}
        \%N trong $NH_4NO_3$.
        \loigiai{35\%.}
    \end{bt}
    %%%%%===========BT_5=======%%%%%
    \begin{bt}
        80\%C, 20\%H. PTK 30.
        \loigiai{C2H6.}
    \end{bt}
    %%%%%===========BT_6=======%%%%%
    \begin{bt}
        Lập công thức hóa học của hợp chất gồm P(V) và O.
        \loigiai{P(V) và O(II) $\Rightarrow P_2O_5$.}
    \end{bt}
    %%%%%===========BT_7=======%%%%%
    \begin{bt}
        Lập CTHH của Nhôm clorua (Al và Cl hóa trị I).
        \loigiai{Al(III) và Cl(I) $\Rightarrow AlCl_3$.}
    \end{bt}
    %%%%%===========BT_8=======%%%%%
    \begin{bt}
        Một hiđrocacbon chứa 75\% Cacbon, 25\% Hiđro về khối lượng. Phân tử khối là 16. Tìm CTHH.
        \loigiai{$C_xH_y$: $12x = 16 \times 0.75 = 12 \Rightarrow x=1$. $y=4 \Rightarrow CH_4$.}
    \end{bt}
    %%%%%===========BT_9=======%%%%%
    \begin{bt}
        Oxit của kim loại M (hóa trị III) có phân tử khối là 102. Xác định tên kim loại M.
        \loigiai{$M_2O_3$: $2M + 48 = 102 \Rightarrow 2M = 54 \Rightarrow M=27$ (Al).}
    \end{bt}
    %%%%%===========BT_10=======%%%%%
    \begin{bt}
        Tính thành phần phần trăm khối lượng của các nguyên tố trong HCl.
        \loigiai{H: $2.7\%$, Cl: $97.3\%$.}
    \end{bt}
    \Closesolutionfile{ansbt}
    \Closesolutionfile{ansbth}
\end{dang}

\end{document}
