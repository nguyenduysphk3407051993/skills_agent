\documentclass{standalone}
\usepackage{tikz,tikz-3dplot}
\usetikzlibrary{calc,fadings,decorations.pathreplacing}
\begin{document}
	\tdplotsetmaincoords{70}{120}
	\begin{tikzpicture}[%
		% điều chỉnh góc nhìn
		tdplot_main_coords,
		% vẽ đường nét tròn ở đầu mút
		line cap=round,
		line join=round,
		% gán giá trị cho biến r <bán kính>, w <độ dày liên kết>
		declare function={
			r=8mm;
			w=6pt;
		}] 
		%vòng lặp vẽ các nguyên tử
		\foreach \x in {0,5,10}{
			\foreach \y in {0,5,10}{
				\foreach \z in {0,6,12}{		
					% Sử dụng điều kiện để kiểm tra tính chẵn lẻ của tổng các chỉ số
					\pgfmathparse{int((\x/5) + (\y/5) + (\z/6))}
					\ifodd\pgfmathresult
						\shade[ball color=cyan!50!violet] (\x,\y,\z) circle (1.2*r);
					\else
						\shade[ball color=red!50!pink] (\x,\y,\z) circle (r);
					\fi
					%Sử dụng điều kiện để nối các hình cầu
					\ifnum\x<10
						\path[line width=w,shorten <={r/2},draw=purple!30!black!30,fill=purple!60!black!30,rounded corners=1.2pt] (\x,\y,\z) -- ++(5,0,0);
					\fi
					\ifnum\y<10
						\path[line width=w,shorten <={r/2},draw=purple!50!black!50,fill=purple!60!black!30] (\x,\y,\z) -- ++(0,5,0);
					\fi
					\ifnum\z<12
						\path[line width=w,shorten <={r/2},draw=purple!70!black!70,fill=purple!60!black!30] (\x,\y,\z) -- ++(0,0,6);
					\fi
				}
			}
		}
		\shade[ball color=cyan!50!violet] (0,14,12) circle (1.2*r);
		\path (0,16,12.3) node[font=\bfseries\sffamily] {\LARGE Cl$\mathsf{^-}$};
		\shade[ball color=red!50!pink] (0,14,9) circle (r);
		\path (0,16,9.3) node[font=\bfseries\sffamily] {\LARGE Na$\mathsf{^+}$};
	\end{tikzpicture}
\end{document}
