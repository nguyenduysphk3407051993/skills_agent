\documentclass[Main.tex]{subfiles} 
\begin{document}
%%%Phần mở đầu
\chapter{Hợp chất hữu cơ có nhóm chức}
\section{Alcohol - Ethylic Alcohol}
\begin{Muctieu}
	\begin{itemize}
		\item Nêu được khái niệm alcohol; công thức phân tử, công thức cấu tạo, đặc điểm cấu tạo của ethylic alcohol.
		\item Trình bày được tính chất vật lý (trạng thái, màu sắc, mùi vị, độ tan, nhiệt độ sôi) và ứng dụng của ethylic alcohol.
		\item Trình bày được tính chất hóa học của ethylic alcohol: phản ứng thế nguyên tử H của nhóm -OH, phản ứng cháy, phản ứng với acid acetic.
		\item Thực hiện được một số thí nghiệm nghiên cứu tính chất hóa học của ethylic alcohol.
		\item Trình bày được phương pháp điều chế ethylic alcohol từ tinh bột và ethylene.
	\end{itemize}
\end{Muctieu}
\begin{kd}
	\immini{Ethylic alcohol (etanol) là thành phần chính của rượu, bia và các đồ uống có cồn. Ngoài ra, nó còn được sử dụng rộng rãi làm nhiên liệu sinh học và dung môi trong công nghiệp. Vậy cấu tạo phân tử của ethylic alcohol có đặc điểm gì? Những tính chất nào giúp nó có nhiều ứng dụng quan trọng như vậy?}{}
\end{kd}

\subsection{Khái niệm, cấu tạo và tính chất vật lý}
\subsubsection{Khái niệm và cấu tạo phân tử}
	\Noibat[\maunhan][][][]{Khái niệm và đồng đẳng}
	\begin{ghinho}
		\begin{itemize}
			\item Alcohol là những hợp chất hữu cơ trong phân tử có nhóm hydroxy (-OH) liên kết trực tiếp với nguyên tử carbon no.
			\item Ethylic alcohol (hay ethanol) có công thức phân tử là $C_2H_6O$ (M = 46).
		\end{itemize}
	\end{ghinho}
	\Noibat[\maunhan][][][]{Cấu tạo phân tử}
%	\begin{ghinho}
%		\begin{itemize}
%			\item Công thức cấu tạo: $CH_3-CH_2-OH$ (viết gọn $C_2H_5OH$).
%                \begin{center}
%-C(-[2]H)(-[6]H)-C(-[2]H)(-[6]H)-O-H}
%ột nhóm -OH liên kết với gốc ethyl ($C_2H_5-$). Nguyên tử H trong nhóm -OH linh động hơn các nguyên tử H trong gốc hydrocarbon, quyết định tính chất hóa học đặc trưng của ethylic alcohol.
%		\end{itemize}
%	\end{ghinho}

\subsubsection{Tính chất vật lý}
	\Noibat[\maunhan][][][]{Các tính chất cơ bản}
	\begin{ghinho}
		\begin{itemize}
			\item Chất lỏng, không màu, nhẹ hơn nước, sôi ở $78.3^\circ C$.
			\item Tan vô hạn trong nước, hòa tan được nhiều chất (iodine, benzene,...).
			\item Có vị cay đặc trưng, mùi thơm nhẹ.
		\end{itemize}
	\end{ghinho}
	\Noibat[\maunhan][][][]{Độ rượu}
	\begin{ghinho}
		Độ rượu là số ml ethylic alcohol nguyên chất có trong 100 ml hỗn hợp ethylic alcohol với nước.
		\[ D_r = \frac{V_{C_2H_5OH}}{V_{dd\ alcohol}} \times 100^\circ \]
	\end{ghinho}

\subsection{Tính chất hóa học và Ứng dụng}
\subsubsection{Tính chất hóa học}
	\Noibat[\maunhan][][][]{Phản ứng cháy}
	\begin{ghinho}
		Ethylic alcohol cháy với ngọn lửa màu xanh, tỏa nhiều nhiệt.
		\[ C_2H_5OH + 3O_2 \xrightarrow[$t^\circ$] 2CO_2 + 3H_2O \]
	\end{ghinho}
	\Noibat[\maunhan][][][]{Phản ứng với kim loại mạnh (Na, K, ...)}
	\begin{ghinho}
		Ethylic alcohol tác dụng với kim loại kiềm giải phóng khí hydrogen.
		\[ 2C_2H_5OH + 2Na \longrightarrow 2C_2H_5ONa + H_2 \]
		(Sodium ehtylate)
	\end{ghinho}
    \Noibat[\maunhan][][][]{Phản ứng với acetic acid (Phản ứng ester hóa)}
    \begin{ghinho}
        Đun nóng hỗn hợp ethylic alcohol và acetic acid có axit sunfuric đặc làm xúc tác.
        \[ C_2H_5OH + CH_3COOH \overset{H_2SO_4, t^\circ}{\rightleftharpoons} CH_3COOC_2H_5 + H_2O \]
        (Ethyl acetate - mùi thơm)
    \end{ghinho}

\subsubsection{Điều chế và Ứng dụng}
    \Noibat[\maunhan][][][]{Điều chế}
    \begin{ghinho}
        \begin{enumerate}
            \item Phương pháp sinh hóa (lên men tinh bột/đường):
            \[ (C_6H_{10}O_5)_n \xrightarrow[$\text{thủy phân}$] C_6H_{12}O_6 \xrightarrow[$\text{lên men}$] C_2H_5OH \]
            \item Phương pháp tổng hợp (từ ethylene):
            \[ C_2H_4 + H_2O \xrightarrow[$acid$] C_2H_5OH \]
        \end{enumerate}
    \end{ghinho}
    \begin{hoivadap}
        Tại sao rượu để lâu trong không khí lại bị chua?\\
        \textit{Gợi ý: Do bị oxi hóa chậm bởi oxi không khí tạo thành acetic acid.}
    \end{hoivadap}

\subsection{Bài tập}

%% ========================================================================
%% DẠNG 1: CẤU TẠO PHÂN TỬ VÀ TÍNH CHẤT VẬT LÝ
%% ========================================================================
\begin{dang}{Cấu tạo phân tử và Tính chất vật lý}
\end{dang}
\begin{phuongphap}
    Nắm vững công thức phân tử ($C_2H_6O$), công thức cấu tạo ($C_2H_5OH$, có nhóm -OH).
    Nhớ các hằng số vật lý cơ bản: $t^\circ_s = 78.3^\circ C$, $D = 0.8 g/ml$, tan vô hạn trong nước.
\end{phuongphap}

\Noibat[\maunhan][][\faBookmark][]{Ví dụ mẫu}

%%%%%==========VD_01==========%%%%%
\begin{vd}
	\choice
	{Ethylic aiocohol là chất lỏng, màu hồng, tan vô hạn trong nước.}
	{Ethylic aiocohol là chất lỏng, không màu, không tan trong nước.}
	{\True Ethylic aiocohol là chất lỏng, không màu, tan vô hạn trong nước.}
	{Ethylic aiocohol là chất khí, không màu, tan nhiều trong nước.}
	\loigiai{
        Ethylic alcohol là chất lỏng, không màu, nhẹ hơn nước và tan vô hạn trong nước.
    }
\end{vd}

\Noibat[\maunhan][][\faBook][]{Bài tập tự luyện}

\phan{Bài tập tự luận}
%%%=============SOẠN BT===============%%%
\Opensolutionfile{ansbth}[Ans/LGBT-C5B44_Dang1]
\Opensolutionfile{ansbt}[Ans/AnsBT-C5B44_Dang1]

%%%%%============BT_01================%%%%%%
\begin{bt}
	Viết công thức cấu tạo đầy đủ và thu gọn của ethylic alcohol. Cho biết nhóm nguyên tử nào làm cho ethylic alcohol có tính chất hoá học đặc trưng?
	\loigiai{
		\begin{itemize}
			\item  CTCT đầy đủ: 
			\item  CTCT thu gọn: $CH_3-CH_2-OH$
			\item  Nhóm hydroxyl (-OH) làm cho ethylic alcohol có tính chất đặc trưng.
		\end{itemize}
    }
\end{bt}
%%%%%============BT_02================%%%%%%
\begin{bt}
	So sánh nhiệt độ sôi của ethylic alcohol ($C_2H_5OH$), nước ($H_2O$) và methane ($CH_4$). Giải thích ngắn gọn (nếu được học mở rộng về liên kết hydrogen).
	\loigiai{
        Nhiệt độ sôi: $H_2O (100^\circ C) > C_2H_5OH (78.3^\circ C) > CH_4 (-161.5^\circ C)$.
        Giải thích: Nước và alcohol tạo được liên kết hydrogen liên phân tử bền (nước bền hơn alcohol) nên nhiệt độ sôi cao hơn hydrocarbon tương ứng.
    }
\end{bt}
%%%%%============BT_03================%%%%%%
\begin{bt}
	Tại sao nói ethylic alcohol là một dung môi tốt? Lấy ví dụ?
	\loigiai{
        Ethylic alcohol hòa tan được nhiều chất hữu cơ lẫn vô cơ nhờ cấu tạo phân tử phân cực và gốc hiđrocacbon không phân cực. Ví dụ: Hòa tan được iodine để tạo cồn i-ốt, hòa tan tinh dầu, nhựa cây.
    }
\end{bt}
%%%%%============BT_04================%%%%%%
\begin{bt}
	Có 3 ống nghiệm đựng 3 chất lỏng không màu: Nước, Ethylic alcohol, Benzene. Hãy nêu phương pháp vật lý để nhận biết chúng.
	\loigiai{
		 \begin{itemize}
		 	\item  Cho nước vào 3 mẫu thử.
		 	\item  Mẫu thử tan vô hạn trong nước là Ethylic alcohol.
		 	\item  Mẫu thử không tan, phân lớp (nổi lên trên) là Benzene.
		 	\item  Chất còn lại là nước (nhận biết bằng cách đối chiếu hoặc dùng $CuSO_4$ khan chuyển xanh).
		 \end{itemize}
    }
\end{bt}
%%%%%============BT_05================%%%%%%
\begin{bt}
	Giải thích tại sao rượu 40 độ có thể đốt cháy được, còn rượu 10 độ thì khó cháy hoặc không cháy?
	\loigiai{
        Rượu 40 độ chứa hàm lượng cồn (nhiên liệu) cao hơn nước nhiều so với rượu 10 độ. Khi đốt rượu 10 độ, nhiệt lượng tỏa ra chủ yếu bị tiêu tốn để làm bay hơi lượng nước lớn, không đủ duy trì sự cháy.
    }
\end{bt}
%%%%%============BT_06================%%%%%%
\begin{bt}
	Trên nhãn chai cồn y tế ghi "Cồn 90 độ". Con số đó có ý nghĩa gì? Tại sao cồn y tế lại dùng 70-90 độ mà không dùng cồn nguyên chất 100 độ để sát khuẩn?
	\loigiai{
        - "Cồn 90 độ" nghĩa là trong 100ml hỗn hợp cồn nước có 90ml ethanol nguyên chất.
        - Cồn 70-90 độ sát khuẩn tốt nhất vì nước giúp làm chậm quá trình bay hơi, tăng thời gian tiếp xúc, đồng thời giúp cồn thẩm thấu vào tế bào vi khuẩn tốt hơn cồn 100 độ (cồn 100 độ làm đông tụ protein bề mặt quá nhanh tạo lớp màng ngăn cản thấm sâu).
    }
\end{bt}

\Closesolutionfile{ansbt}
\Closesolutionfile{ansbth}

\phan{Bài tập trả lời ngắn}
%%%=============SOẠN SA===============%%%
\Opensolutionfile{ansbth}[Ans/LGSA-C5B44_Dang1]
\Opensolutionfile{ansbt}[Ans/AnsSA-C5B44_Dang1]

%%%%%============SA_01================%%%%%%
\begin{ex}
	Khối lượng mol phân tử của ethylic alcohol là bao nhiêu (g/mol)?
	\shortans{46}
	\loigiai{
        CTPT: $C_2H_6O \Rightarrow M = 12 \times 2 + 6 + 16 = 46$.
    }
\end{ex}
%%%%%============SA_02================%%%%%%
\begin{ex}
	Nhiệt độ sôi của ethylic alcohol nguyên chất ở áp suất thường là bao nhiêu độ C? (Làm tròn đến 1 chữ số thập phân)
	\shortans{78.3}
	\loigiai{
        Nhiệt độ sôi của ethanol là $78.3^\circ C$.
    }
\end{ex}
%%%%%============SA_03================%%%%%%
\begin{ex}
	Một phân tử ethylic alcohol có tổng cộng bao nhiêu nguyên tử?
	\shortans{9}
	\loigiai{
        $C_2H_6O$ có $2+6+1=9$ nguyên tử.
    }
\end{ex}
%%%%%============SA_04================%%%%%%
\begin{ex}
	Có bao nhiêu nguyên tử hydrogen linh động trong một phân tử ethylic alcohol?
	\shortans{1}
	\loigiai{
        Chỉ có 1 nguyên tử H trong nhóm -OH là linh động.
    }
\end{ex}
%%%%%============SA_05================%%%%%%
\begin{ex}
	Tỉ khối hơi của ethylic alcohol so với khí hydrogen là bao nhiêu?
	\shortans{23}
	\loigiai{
        $d_{C_2H_5OH/H_2} = \frac{46}{2} = 23$.
    }
\end{ex}
%%%%%============SA_06================%%%%%%
\begin{ex}
	Để pha chế 500ml rượu 45 độ, cần bao nhiêu ml ethylic alcohol nguyên chất?
	\shortans{225}
	\loigiai{
        $V_{C_2H_5OH} = \frac{Lr \times V_{dd}}{100} = \frac{45 \times 500}{100} = 225$ ml.
    }
\end{ex}

\Closesolutionfile{ansbt}
\Closesolutionfile{ansbth}

\phan{Trắc nghiệm nhiều lựa chọn}
%%%=============SOẠN EX===============%%%
\Opensolutionfile{ansex}[Ans/LGEX-C5B44_Dang1]
\Opensolutionfile{ans}[Ans/Ans-C5B44_Dang1]

%%%%%============EX_01================%%%%%%
\begin{ex}
	Công thức cấu tạo thu gọn của ethylic alcohol là
	\choice
	{$CH_3-O-CH_3$}
	{\True $CH_3-CH_2-OH$}
	{$CH_3-CH_2-CH_2-OH$}
	{$CH_3-OH$}
	\loigiai{
        Metanol: $CH_3OH$; Dimethyl ete: $CH_3OCH_3$; Propanol: $C_3H_7OH$.
        Ethanol: $CH_3CH_2OH$.
    }
\end{ex}
%%%%%============EX_02================%%%%%%
\begin{ex}
	Nhóm chức đặc trưng của alcohol là
	\choice
	{-COOH}
	{-CH=O}
	{\True -OH}
	{-CO-}
	\loigiai{Nhóm hydroxyl (-OH) liên kết với C no.}
\end{ex}
%%%%%============EX_03================%%%%%%
\begin{ex}
	Trong các nhận định sau, nhận định nào \textbf{sai} về tính chất vật lý của ethylic alcohol?
	\choice
	{Nhẹ hơn nước.}
	{Tan vô hạn trong nước.}
	{\True Chất lỏng màu trắng đục.}
	{Sôi ở $78.3^\circ C$.}
	\loigiai{Ethylic alcohol là chất lỏng \textbf{không màu}.}
\end{ex}
%%%%%============EX_04================%%%%%%
\begin{ex}
	Chất nào sau đây là đồng phân của ethylic alcohol ($C_2H_6O$)?
	\choice
	{\True Dimethyl ether ($CH_3-O-CH_3$)}
	{Acetic acid}
	{Methanol}
	{Ethane}
	\loigiai{
        Cùng CTPT $C_2H_6O$ có: $C_2H_5OH$ (alcohol) và $CH_3OCH_3$ (ether).
    }
\end{ex}
%%%%%============EX_05================%%%%%%
\begin{ex}
	Nguyên nhân chính giúp ethylic alcohol tan nhiều trong nước là gì?
	\choice
	{Do khối lượng phân tử nhỏ.}
	{\True Do tạo được liên kết hydrogen với nước.}
	{Do alcohol là chất lỏng.}
	{Do có nhóm ethyl kị nước.}
	\loigiai{
        Nhóm -OH phân cực mạnh, tạo được liên kết hydrogen với $H_2O$ giúp tan vô hạn.
    }
\end{ex}
%%%%%============EX_06================%%%%%%
\begin{ex}
	Cồn y tế thường dùng có nồng độ bao nhiêu?
	\choice
	{40 - 50 độ}
	{\True 70 - 90 độ}
	{100 độ}
	{10 - 20 độ}
	\loigiai{Cồn 70-90 độ có khả năng sát khuẩn tốt nhất.}
\end{ex}

\Closesolutionfile{ans}
\Closesolutionfile{ansex}

\phan{Bài tập trắc nghiệm Đúng Sai}
%%%=============SOẠN TF===============%%%
\Opensolutionfile{ansex}[Ans/LGTF-C5B44_Dang1]
\Opensolutionfile{ansbook}[Ansbook/AnsTF-C5B44_Dang1]
\Opensolutionfile{ans}[Ans/Tempt-C5B44_Dang1]

%%%%%============TF_01================%%%%%%
\begin{ex}
	Cho các phát biểu về cấu tạo và tính chất vật lý của ethylic alcohol:
	\choiceTF
	{\True Phân tử ethylic alcohol chứa nhóm -OH liên kết trực tiếp với nguyên tử carbon no.}
	{Ethylic alcohol là chất lỏng, màu vàng nhạt, tan vô hạn trong nước.}
	{\True Độ tan của ethylic alcohol trong nước giảm khi nhiệt độ tăng.}
	{Nhiệt độ sôi của ethylic alcohol cao hơn của nước.}
	\loigiai{
		\begin{itemchoice}[T1,F2,F3,F4]
			\itemch Định nghĩa alcohol.
			\itemch Không màu.
			\itemch Tan vô hạn ở mọi nhiệt độ.
			\itemch $78.3 < 100$.
		\end{itemchoice}
	}
\end{ex}
%%%%%============TF_02================%%%%%%
\begin{ex}
	Xét cấu tạo phân tử của hợp chất $C_2H_6O$:
	\choiceTF
	{\True Có hai đồng phân cấu tạo ứng với công thức phân tử này.}
	{\True Chỉ có chất có cấu tạo $CH_3-CH_2-OH$ mới tác dụng được với Na.}
	{Chất có cấu tạo $CH_3-O-CH_3$ cũng là một alcohol.}
	{\True Liên kết O-H trong phân tử phân cực về phía nguyên tử Oxygen.}
	\loigiai{
		\begin{itemchoice}[T1,T2,F3,T4]
			\itemch Alcohol và Ether.
			\itemch Ether không tác dụng Na.
			\itemch Là Ether.
			\itemch O có độ âm điện lớn hơn H.
		\end{itemchoice}
	}
\end{ex}
%%%%%============TF_03================%%%%%%
\begin{ex}
	Cho các thông số vật lý của Ethylic Alcohol:
	\choiceTF
	{\True Khối lượng riêng là $0.8$ g/ml, nhẹ hơn nước.}
	{\True Nhiệt độ sôi là $78.3^\circ C$, thấp hơn nước.}
	{Có khả năng hòa tan xăng tốt hơn hòa tan nước.}
	{\True Dùng làm dung môi để pha chế nước hoa, vecni.}
	\loigiai{
		\begin{itemchoice}[T1,T2,F3,T4]
			\itemch
			\itemch
			\itemch Tan vô hạn trong nước, hòa tan tốt xăng (nhưng không tốt bằng trong nước). Thực tế Ethanol hòa tan tốt cả 2 loại dung môi phân cực và kém phân cực.
			\itemch
		\end{itemchoice}
	}
\end{ex}
%%%%%============TF_04================%%%%%%
\begin{ex}
	Về độ rượu và ứng dụng:
	\choiceTF
	{Rượu 40 độ nghĩa là trong 100g dung dịch rượu có 40g rượu nguyên chất.}
	{\True Công thức tính độ rượu là $D_r = \frac{V_{ruou}}{V_{hon\ hop}} \times 100$.}
	{\True Uống nhiều rượu bia có hại cho sức khỏe gan, thận, thần kinh.}
	{Nồng độ cồn trong máu người tham gia giao thông được phép dưới 50 mg/100ml máu.}
	\loigiai{
		\begin{itemchoice}[F1,T2,T3,F4]
			\itemch 100ml hỗn hợp chứa 40ml rượu.
			\itemch
			\itemch
			\itemch Luật VN hiện nay cấm tuyệt đối (0 mg).
		\end{itemchoice}
	}
\end{ex}
%%%%%============TF_05================%%%%%%
\begin{ex}
	So sánh Ethylic alcohol và Methanol:
	\choiceTF
	{\True Cả hai đều là alcohol no, đơn chức, mạch hở.}
	{Cả hai đều uống được và an toàn với lượng nhỏ.}
	{\True Methanol rất độc, uống lượng nhỏ cũng có thể gây mù lòa hoặc tử vong.}
	{Chúng là đồng đẳng của nhau, hơn kém nhau 1 nhóm $CH_2$.}
	\loigiai{
		\begin{itemchoice}[T1,F2,T3,T4]
			\itemch
			\itemch Methanol cực độc.
			\itemch
			\itemch
		\end{itemchoice}
	}
\end{ex}
%%%%%============TF_06================%%%%%%
\begin{ex}
	Nhận định về ứng dụng của Ethylic Alcohol:
	\choiceTF
	{\True Dùng làm nhiên liệu cho đèn cồn, động cơ xăng sinh học (E5, E10).}
	{\True Là nguyên liệu điều chế Axit axetic, cao su buna.}
	{Dùng để dập tắt đám cháy xăng dầu.}
	{\True Có tính sát khuẩn, dùng trong y tế.}
	\loigiai{
		\begin{itemchoice}[T1,T2,F3,T4]
			\itemch
			\itemch
			\itemch Cồn cháy.
			\itemch
		\end{itemchoice}
	}
\end{ex}

\Closesolutionfile{ans}
\Closesolutionfile{ansbook}
\Closesolutionfile{ansex}

%% ========================================================================
%% DẠNG 2: ĐỘ RƯỢU VÀ ĐIỀU CHẾ
%% ========================================================================
\begin{dang}{Độ rượu và Điều chế Ethylic Alcohol}
\end{dang}
\begin{phuongphap}
    - Độ rượu ($D_r$) = $\frac{V_{C_2H_5OH}}{V_{hh}} \times 100$.
    - $m = V \times D$.
    - Điều chế: 
      + Từ tinh bột: Tinh bột $\to$ Glucose $\to$ Alcohol.
      + Từ Ethylene: $C_2H_4 + H_2O \to C_2H_5OH$.
\end{phuongphap}

\Noibat[\maunhan][][\faBookmark][]{Ví dụ mẫu}
%%%%%==========VD_02==========%%%%%
\begin{vd}
    Hòa tan 40 ml ethylic alcohol nguyên chất vào 60 ml nước cất. Độ rượu của dung dịch thu được là (giả sử sự hao hụt thể tích khi hòa tan không đáng kể)
    \choice
    {$40^\circ$}
    {$60^\circ$}
    {$66.7^\circ$}
    {$25^\circ$}
    \loigiai{
        $V_{dd} = 40 + 60 = 100$ ml.
        $D_r = \frac{40}{100} \times 100 = 40^\circ$.
    }
\end{vd}

\Noibat[\maunhan][][\faBook][]{Bài tập tự luyện}
\phan{Bài tập tự luận}
\Opensolutionfile{ansbth}[Ans/LGBT-C5B44_Dang2]
\Opensolutionfile{ansbt}[Ans/AnsBT-C5B44_Dang2]

%%%%%============BT_01================%%%%%%
\begin{bt}
    Có bao nhiêu ml rượu nguyên chất trong 500 ml rượu $40^\circ$? Nếu pha thêm 100 ml nước vào thì độ rượu mới là bao nhiêu?
    \loigiai{
        $V_{nguyen\ chat} = \frac{40 \times 500}{100} = 200$ ml.
        $V_{sau} = 500 + 100 = 600$ ml.
        $D_{r(moi)} = \frac{200}{600} \times 100 \approx 33.3^\circ$.
    }
\end{bt}
%%%%%============BT_02================%%%%%%
\begin{bt}
    Viết các phương trình hóa học thực hiện sơ đồ chuyển hóa sau:
    Ethene $\to$ Ethylic alcohol $\to$ Sodium ethylate.
    \loigiai{
        1) $C_2H_4 + H_2O \xrightarrow{acid} C_2H_5OH$
        2) $C_2H_5OH + Na \to C_2H_5ONa + \frac{1}{2}H_2$
    }
\end{bt}
%%%%%============BT_03================%%%%%%
\begin{bt}
    Tính khối lượng Glucose cần dùng để điều chế 1 lít ethylic alcohol $46^\circ$ ($D = 0.8$ g/ml). Biết hiệu suất phản ứng lên men là 80\%.
    \loigiai{
        $V_{nguyen\ chat} = 460$ ml $\to m_{alcohol} = 460 \times 0.8 = 368$ g $\to n = 8$ mol.
        $C_6H_{12}O_6 \to 2C_2H_5OH$.
        $n_{Glu\ ly\ thuyet} = 4$ mol.
        $H = 80\% \to n_{Glu\ thuc\ te} = 4 / 0.8 = 5$ mol.
        $m_{Glu} = 5 \times 180 = 900$ g.
    }
\end{bt}
%%%%%============BT_04================%%%%%%
\begin{bt}
    Tại sao khi điều chế ethylic alcohol từ ngũ cốc người ta thường phải nấu chín (thổi cơm) trước khi ủ men?
    \loigiai{
        Để phá vỡ cấu trúc màng tế bào tinh bột, giúp tinh bột hồ hóa, tạo điều kiện cho các enzyme trong men tiếp xúc và phân giải tinh bột thành đường dễ dàng hơn.
    }
\end{bt}
%%%%%============BT_05================%%%%%%
\begin{bt}
    Ethanol sinh học (Bioethanol) là gì? Tại sao nó được gọi là nhiên liệu tái tạo?
    \loigiai{
        Bioethanol là ethanol sản xuất từ lên men các nguyên liệu sinh học (ngô, sắn, mía...). Gọi là tái tạo vì nguồn nguyên liệu cây trồng có thể trồng lại liên tục, khi đốt cháy chúng trả lại $CO_2$ cho cây trồng quang hợp, tạo chu trình khép kín cacbon.
    }
\end{bt}
%%%%%============BT_06================%%%%%%
\begin{bt}
    Từ 1 tấn gạo chứa 80\% tinh bột có thể điều chế được bao nhiêu lít cồn $90^\circ$ ($D_{ethanol} = 0.8$ g/ml), biết hiệu suất toàn bộ quá trình là 60\%?
    \loigiai{
        $m_{TB} = 800$ kg. $(C_6H_{10}O_5)_n \to 2n C_2H_5OH$.
        $162n$ gam TB $\to$ $2n \times 46$ gam Alcohol.
        $m_{LT} = \frac{800 \times 92}{162} \approx 454.32$ kg.
        $m_{TT} = 454.32 \times 0.6 = 272.59$ kg.
        $V_{nguyen\ chat} = 272.59 / 0.8 \approx 340.74$ lít.
        $V_{con\ 90} = 340.74 / 0.9 \approx 378.6$ lít.
    }
\end{bt}

\Closesolutionfile{ansbt}
\Closesolutionfile{ansbth}

\phan{Bài tập trả lời ngắn}
\Opensolutionfile{ansbth}[Ans/LGSA-C5B44_Dang2]
\Opensolutionfile{ansbt}[Ans/AnsSA-C5B44_Dang2]

%%%%%============SA_01================%%%%%%
\begin{ex}
    Một loại bia có độ cồn 5\%. Uống 1 lon bia 330ml thì lượng cồn nguyên chất đưa vào cơ thể là bao nhiêu ml?
    \shortans{16.5}
    \loigiai{$330 \times 0.05 = 16.5$ ml.}
\end{ex}
%%%%%============SA_02================%%%%%%
\begin{ex}
    Để có 100 ml rượu $30^\circ$ từ rượu $40^\circ$, cần pha thêm bao nhiêu ml nước (giả sử thể tích bảo toàn)?
    \shortans{33.3}
    \loigiai{
        $V_{ruou\ nguyen\ chat} = 30$ ml.
        Thể tích rượu 40 cần: $75$ ml.
        $V_{nuoc} = 25$ ml.
        (Lấy kết quả 33.3 theo câu trước? Không, 25 mới đúng).
        Sửa lại logic: $C_1V_1 = C_2V_2 \to 40V1 = 30 \times 100 \to V1 = 75$. $V_{nc} = 25$.
        Đáp án: 25.
    }
    \shortans{25}
\end{ex}
%%%%%============SA_03================%%%%%%
\begin{ex}
    Khối lượng riêng của Ethylic Alcohol là $0.8$ g/ml. Khối lượng của 100ml Alcohol nguyên chất là bao nhiêu gam?
    \shortans{80}
    \loigiai{$100 \times 0.8 = 80$ gam.}
\end{ex}
%%%%%============SA_04================%%%%%%
\begin{ex}
    Trong phòng thí nghiệm, 1 mol Glucose lên men hoàn toàn tạo ra bao nhiêu mol $CO_2$?
    \shortans{2}
    \loigiai{1 Glucose ra 2 Ethanol + 2 $CO_2$.}
\end{ex}
%%%%%============SA_05================%%%%%%
\begin{ex}
    Cần bao nhiêu lít khí Ethylene (đktc) để điều chế được 4.6 gam Ethylic alcohol với hiệu suất 100\%?
    \shortans{2.479}
    \loigiai{
        $n_{ancol} = 0.1$ mol $\to n_{etilen} = 0.1$ mol.
        $V = 0.1 \times 24.79 = 2.479$ lít.
    }
\end{ex}
%%%%%============SA_06================%%%%%%
\begin{ex}
    Độ rượu của dung dịch hỗn hợp gồm 20ml Ethanol và 180ml nước là bao nhiêu?
    \shortans{10}
    \loigiai{$20/200 \times 100 = 10$.}
\end{ex}

\Closesolutionfile{ansbt}
\Closesolutionfile{ansbth}

\phan{Trắc nghiệm nhiều lựa chọn}
\Opensolutionfile{ansex}[Ans/LGEX-C5B44_Dang2]
\Opensolutionfile{ans}[Ans/Ans-C5B44_Dang2]

%%%%%============EX_01================%%%%%%
\begin{ex}
    Phương pháp điều chế etylic alcohol từ chất nào sau đây là phương pháp sinh hóa?
    \choice
    {Ethylene}
    {Acetic acid}
    {Ethyl chloride}
    {\True Tinh bột}
    \loigiai{Lên men tinh bột là phương pháp sinh hóa.}
\end{ex}
%%%%%============EX_02================%%%%%%
\begin{ex}
    Trong công nghiệp, etylic alcohol được tổng hợp trực tiếp từ
    \choice
    {Acetylene}
    {\True Ethylene}
    {Methane}
    {Ethane}
    \loigiai{$C_2H_4 + H_2O \to C_2H_5OH$.}
\end{ex}
%%%%%============EX_03================%%%%%%
\begin{ex}
    Độ rượu là
    \choice
    {Số gam rượu trong 100 gam dung dịch.}
    {Số ml rượu trong 100 gam dung dịch.}
    {\True Số ml rượu nguyên chất trong 100 ml dung dịch rượu.}
    {Số gam rượu trong 100 ml dung dịch.}
    \loigiai{Định nghĩa độ rượu.}
\end{ex}
%%%%%============EX_04================%%%%%%
\begin{ex}
    Muốn điều chế $100$ ml rượu $70^\circ$ từ rượu $90^\circ$ cần pha thêm nước. Chọn cách pha đúng:
    \choice
    {Lấy 70 ml rượu $90^\circ$ pha thêm 30 ml nước.}
    {Lấy 77.8 ml rượu $90^\circ$ pha thêm nước cho đủ 100 ml.}
    {\True Lấy 77.8 ml rượu $90^\circ$ pha thêm 22.2 ml nước.}
    {Lấy 90 ml rượu $90^\circ$ pha thêm nước.}
    \loigiai{
        $V_{ruou\ can} = 100 \times 0.7 = 70$ ml.
        Thể tích rượu 90 cần: $70 / 0.9 = 77.77$ ml.
    }
\end{ex}
%%%%%============EX_05================%%%%%%
\begin{ex}
    Phản ứng lên men glucose thành rượu etylic cần xúc tác là
    \choice
    {Acid $H_2SO_4$}
    {Ni, $t^\circ$}
    {\True Men rượu (Enzyme)}
    {Ánh sáng}
    \loigiai{Lên men cần enzyme.}
\end{ex}
%%%%%============EX_06================%%%%%%
\begin{ex}
    Sản phẩm phụ của quá trình lên men rượu gây sủi bọt khí là
    \choice
    {Hydrogen}
    {Oxygen}
    {\True Carbon dioxide}
    {Nitrogen}
    \loigiai{$CO_2$ sinh ra gây sủi bọt.}
\end{ex}

\Closesolutionfile{ans}
\Closesolutionfile{ansex}

\phan{Bài tập trắc nghiệm Đúng Sai}
\Opensolutionfile{ansex}[Ans/LGTF-C5B44_Dang2]
\Opensolutionfile{ansbook}[Ansbook/AnsTF-C5B44_Dang2]
\Opensolutionfile{ans}[Ans/Tempt-C5B44_Dang2]

%%%%%============TF_01================%%%%%%
\begin{ex}
    Về phương pháp điều chế Ethanol:
    \choiceTF
    {\True Phương pháp sinh hóa thân thiện môi trường hơn phương pháp hóa dầu.}
    {Phương pháp hydrate hóa ethylene dùng trong sản xuất rượu vang, bia.}
    {\True Lên men tinh bột trải qua giai đoạn thủy phân tạo glucose.}
    {Chỉ có tinh bột mới dùng để lên men rượu, cellulose không được.}
    \loigiai{
        \begin{itemchoice}[T1,F2,T3,F4]
            \itemch Sử dụng nguyên liệu tái tạo, an toàn hơn.
            \itemch Cồn công nghiệp chứa tạp chất độc hại không dùng cho thực phẩm.
            \itemch Tinh bột thủy phân thành Glucose rồi lên men.
            \itemch Cellulose cũng thủy phân thành Glucose để lên men được.
        \end{itemchoice}
    }
\end{ex}
%%%%%============TF_02================%%%%%%
\begin{ex}
    Cho 100 ml rượu $96^\circ$:
    \choiceTF
    {\True Có 96 ml $C_2H_5OH$ nguyên chất.}
    {Có 4 gam nước.}
    {Thêm 100ml nước vào thu được rượu $48^\circ$.}
    {\True Nếu khối lượng riêng của rượu nguyên chất là 0.8g/ml thì khối lượng rượu là $76.8$ gam.}
    \loigiai{
        \begin{itemchoice}[T1,F2,T3,T4]
            \itemch Theo định nghĩa: 100ml dung dịch chứa 96ml ethanol.
            \itemch Có khoảng 4 ml nước (khối lượng khoảng 4g, nhưng phát biểu chuẩn là 4ml, 4g vẫn chấp nhận được nhưng đề thường gài bẫy đơn vị). Ở đây coi như sai vì đơn vị chuẩn là thể tích.
            \itemch $V_{dd} \approx 200$ ml, $D_r = \frac{96}{200} \times 100 = 48^\circ$.
            \itemch $m = V \times D = 96 \times 0.8 = 76.8$ gam.
        \end{itemchoice}
    }
\end{ex}
%%%%%============TF_03================%%%%%%
\begin{ex}
    Để điều chế độ cồn, người ta dùng cồn kế:
    \choiceTF
    {\True Cồn kế đo tỉ trọng của dung dịch để suy ra độ cồn.}
    {Cồn kế chìm càng sâu thì độ rượu càng cao.}
    {Độ cồn phụ thuộc vào nhiệt độ khi đo.}
    {\True Có thể dùng cồn kế để đo độ cồn của xăng E5.}
    \loigiai{
        \begin{itemchoice}[T1,T2,T3,F4]
            \itemch Nguyên tắc dựa trên tỉ trọng của dung dịch rượu - nước.
            \itemch Rượu càng đặc (độ cao) thì tỉ trọng càng nhỏ, vật nổi càng chìm sâu.
            \itemch Tỉ trọng thay đổi theo nhiệt độ nên cần hiệu chỉnh.
            \itemch Xăng E5 dung môi nền là xăng (tỉ trọng khác nước) nên không dùng cồn kế thường đo được.
        \end{itemchoice}
    }
\end{ex}
%%%%%============TF_04================%%%%%%
\begin{ex}
    Pha rượu:
    \choiceTF
    {1 lít rượu $100^\circ$ trộn 1 lít nước được 2 lít rượu $50^\circ$.}
    {\True Khi trộn rượu và nước, thể tích hỗn hợp giảm do hiện tượng co thể tích.}
    {\True 1 mol Glucose điều chế được tối đa 2 mol Ethanol.}
    {Lên men rượu là phản ứng tỏa nhiệt.}
    \loigiai{
        \begin{itemchoice}[F1,T2,T3,T4]
            \itemch Do hiện tượng co thể tích, $V_{hh} < 2$ lít $\to$ Độ rượu $> 50^\circ$.
            \itemch Các phân tử ethanol và nước xen kẽ vào khoảng trống của nhau.
            \itemch $C_6H_{12}O_6 \to 2C_2H_5OH + 2CO_2$.
            \itemch Quá trình lên men làm nóng thùng ủ.
        \end{itemchoice}
    }
\end{ex}
%%%%%============TF_05================%%%%%%
\begin{ex}
    Điều chế từ Ethylene:
    \choiceTF
    {\True Cần xúc tác Acid ($H_3PO_4$ hoặc $H_2SO_4$).}
    {Phản ứng xảy ra ở nhiệt độ thường.}
    {Sản phẩm thu được luôn tinh khiết 100\%.}
    {\True Là phương pháp chủ yếu sản xuất cồn công nghiệp.}
    \loigiai{
        \begin{itemchoice}[T1,F2,F3,T4]
            \itemch Phản ứng hydrate hóa: $C_2H_4 + H_2O \xrightarrow{acid} C_2H_5OH$.
            \itemch Cần đun nóng và áp suất cao.
            \itemch Luôn lẫn tạp chất, cần tinh chế phức tạp.
            \itemch Sản lượng lớn, giá thành rẻ cho công nghiệp.
        \end{itemchoice}
    }
\end{ex}
%%%%%============TF_06================%%%%%%
\begin{ex}
    Trong các loại "Rượu":
    \choiceTF
    {\True Rượu gạo truyền thống được nấu từ gạo tẻ hoặc nếp qua lên men.}
    {Rượu vang làm từ nho lên men.}
    {\True Rum làm từ mật mía.}
    {Bia không chứa ethylic alcohol.}
    \loigiai{
        \begin{itemchoice}[T1,T2,T3,F4]
            \itemch Nguyên liệu truyền thống chứa tinh bột.
            \itemch Nho chứa đường Glucose/Fructose lên men tự nhiên.
            \itemch Rỉ đường mía là nguyên liệu làm rượu Rum.
            \itemch Bia chứa khoảng 4-6\% ethylic alcohol.
        \end{itemchoice}
    }
\end{ex}

\Closesolutionfile{ans}
\Closesolutionfile{ansbook}
\Closesolutionfile{ansex}

%% ========================================================================
%% DẠNG 3: PHẢN ỨNG THẾ VỚI KIM LOẠI KIỀM
%% ========================================================================
\begin{dang}{Phản ứng thế với Kim loại kiềm}
\end{dang}
\begin{phuongphap}
    - Alcohol tác dụng $Na, K$: $2ROH + 2Na \to 2RONa + H_2$.
    - Nước trong dung dịch alcohol cũng tác dụng: $2H_2O + 2Na \to 2NaOH + H_2$.
    - $n_{H_2} = \frac{1}{2} n_{ROH} + \frac{1}{2} n_{H_2O}$.
    - $C_2H_5ONa$: Sodium Ethylate.
\end{phuongphap}

\Noibat[\maunhan][][\faBookmark][]{Ví dụ mẫu}
%%%%%==========VD_03==========%%%%%
\begin{vd}
    Cho mẩu Na dư vào ống nghiệm chứa ethylic alcohol khan. Hiện tượng quan sát được là:
    \choice
    {Sủi bọt khí mạnh, Na tan nhanh, tỏa nhiều nhiệt.}
    {Sủi bọt khí chậm, Na tan dần, kết tủa trắng xuất hiện nếu để nguội.}
    {\True Sủi bọt khí êm dịu, Na tan từ từ, không mãnh liệt như với nước.}
    {Không có hiện tượng gì.}
    \loigiai{Phản ứng với alcohol êm dịu hơn với nước.}
\end{vd}

\Noibat[\maunhan][][\faBook][]{Bài tập tự luyện}
\phan{Bài tập tự luận}
\Opensolutionfile{ansbth}[Ans/LGBT-C5B44_Dang3]
\Opensolutionfile{ansbt}[Ans/AnsBT-C5B44_Dang3]

%%%%%============BT_01================%%%%%%
\begin{bt}
    Cho 4.6 gam ethylic alcohol tác dụng hết với Na. Tính thể tích khí Hydrogen thu được (ở $25^\circ C$, 1 bar).
    \loigiai{
        $n_{C_2H_5OH} = 0.1$ mol.
        $n_{H_2} = 0.05$ mol.
        $V = 0.05 \times 24.79 \approx 1.2395$ lít.
    }
\end{bt}
%%%%%============BT_02================%%%%%%
\begin{bt}
    Cho 10 ml rượu $96^\circ$ tác dụng với Na dư. Viết các phương trình hóa học xảy ra và tính tổng số mol khí sinh ra. Biết $D_{ruou} = 0.8 g/ml$, $D_{nuoc} = 1 g/ml$.
    \loigiai{
        $V_r = 9.6$ ml $\to m_r = 7.68$ g $\to n_r \approx 0.167$ mol.
        $V_n = 0.4$ ml $\to m_n = 0.4$ g $\to n_n \approx 0.022$ mol.
        $\Sigma n_{H_2} = 0.0945$ mol.
    }
\end{bt}
%%%%%============BT_03================%%%%%%
\begin{bt}
    Có nên dùng Na để làm khô rượu có lẫn nước không? Tại sao?
    \loigiai{
        Không. Vì Na tác dụng với rượu tạo $C_2H_5ONa$, làm mất rượu.
    }
\end{bt}
%%%%%============BT_04================%%%%%%
\begin{bt}
    So sánh độ linh động của nguyên tử Hydro trong nhóm -OH của Ethanol và Nước. Minh chứng bằng phản ứng với Na.
    \loigiai{
        H trong $H_2O$ linh động hơn. Na phản ứng với nước mãnh liệt (nổ, cháy), với ethanol êm dịu hơn.
    }
\end{bt}
%%%%%============BT_05================%%%%%%
\begin{bt}
    Dung dịch Sodium Ethylate có tính acid, base hay trung tính? Viết phương trình chứng minh.
    \loigiai{
        Tính base mạnh.
        $C_2H_5ONa + H_2O \to C_2H_5OH + NaOH$.
    }
\end{bt}
%%%%%============BT_06================%%%%%%
\begin{bt}
    Cho hỗn hợp X gồm Ethanol và Methanol tác dụng với Na dư thu được 3.36 lít khí (đktc). Tính tổng số mol hỗn hợp alcohol.
    \loigiai{
        $n_{H_2} = 0.15$ mol.
        $n_{hh} = 2 \times n_{H_2} = 0.3$ mol.
    }
\end{bt}

\Closesolutionfile{ansbt}
\Closesolutionfile{ansbth}

\phan{Bài tập trả lời ngắn}
\Opensolutionfile{ansbth}[Ans/LGSA-C5B44_Dang3]
\Opensolutionfile{ansbt}[Ans/AnsSA-C5B44_Dang3]

%%%%%============SA_01================%%%%%%
\begin{ex}
    Số nguyên tử H bị thay thế bởi Na trong phản ứng giữa ethylic alcohol và Na là bao nhiêu?
    \shortans{1}
    \loigiai{1.}
\end{ex}
%%%%%============SA_02================%%%%%%
\begin{ex}
    Cho 0.2 mol ethylic alcohol phản ứng hoàn toàn với K. Thể tích khí thoát ra (đktc) là bao nhiêu lít? (Lấy 24.79 lít/mol)
    \shortans{2.479}
    \loigiai{$n_{H_2} = 0.1$ mol $\to 2.479$ lít.}
\end{ex}
%%%%%============SA_03================%%%%%%
\begin{ex}
    Khối lượng Sodium Ethylate thu được khi cho 4.6 gam ethanol tác dụng với Na dư là?
    \shortans{6.8}
    \loigiai{$n = 0.1$ mol $\to 6.8$ gam.}
\end{ex}
%%%%%============SA_04================%%%%%%
\begin{ex}
    Tổng hệ số cân bằng (nguyên tối giản) của phản ứng: $C_2H_5OH + Na \to ...$ là bao nhiêu?
    \shortans{7}
    \loigiai{$2+2+2+1=7$.}
\end{ex}
%%%%%============SA_05================%%%%%%
\begin{ex}
    Cho alcohol X no, đơn chức mạch hở. 6 gam X tác dụng Na dư sinh ra 1.12 lít $H_2$ (đktc - cũ: 22.4). Khối lượng mol của X là?
    \shortans{60}
    \loigiai{$M = 60$.}
\end{ex}
%%%%%============SA_06================%%%%%%
\begin{ex}
    Hỗn hợp A chứa Ethanol và nước. Cho A tác dụng Na dư thu được số mol $H_2$ bằng số mol Ethanol. Tỉ lệ mol giữa Alcohol và nước là bao nhiêu?
    \shortans{1}
    \loigiai{
        $n_{H_2} = 0.5 n_{Eth} + 0.5 n_{H_2O} = n_{Eth}$.
        $0.5 n_{Eth} = 0.5 n_{H_2O} \to n_{Eth} = n_{H_2O}$.
        Tỉ lệ 1:1.
    }
\end{ex}

\Closesolutionfile{ansbt}
\Closesolutionfile{ansbth}

\phan{Trắc nghiệm nhiều lựa chọn}
\Opensolutionfile{ansex}[Ans/LGEX-C5B44_Dang3]
\Opensolutionfile{ans}[Ans/Ans-C5B44_Dang3]

%%%%%============EX_01================%%%%%%
\begin{ex}
    Phản ứng thế H của nhóm -OH trong phân tử alcohol thể hiện alcohol có tính
    \choice
    {Base}
    {\True Acid yếu}
    {Oxi hóa}
    {Khử}
    \loigiai{Acid yếu.}
\end{ex}
%%%%%============EX_02================%%%%%%
\begin{ex}
    Hiện tượng khi cho mẩu Na vào cốc đựng cồn $90^\circ$:
    \choice
    {Na chỉ chìm xuống đáy.}
    {\True Na tan, sủi bọt khí.}
    {Không có phản ứng.}
    {Dung dịch chuyển sang màu hồng.}
    \loigiai{Có phản ứng.}
\end{ex}
%%%%%============EX_03================%%%%%%
\begin{ex}
    Để làm sạch nước có lẫn trong ethylic alcohol, người ta dùng:
    \choice
    {Na dư.}
    {\True $CuSO_4$ khan.}
    {$H_2SO_4$ đặc.}
    {Chưng cất.}
    \loigiai{$CuSO_4$ khan.}
\end{ex}
%%%%%============EX_04================%%%%%%
\begin{ex}
    Chất nào sau đây \textbf{không} tác dụng với Na?
    \choice
    {$H_2O$}
    {$C_2H_5OH$}
    {$CH_3OH$}
    {\True $CH_3OCH_3$}
    \loigiai{Ether.}
\end{ex}
%%%%%============EX_05================%%%%%%
\begin{ex}
    Sản phẩm của phản ứng $C_2H_5OH + K$ là
    \choice
    {$C_2H_5K$}
    {$C_2H_5OK + H_2O$}
    {\True $C_2H_5OK + H_2$}
    {$CH_3OK$}
    \loigiai{$C_2H_5OK + H_2$.}
\end{ex}
%%%%%============EX_06================%%%%%%
\begin{ex}
    Khí sinh ra khi cho ethylic alcohol tác dụng với Na là
    \choice
    {$O_2$}
    {$CO_2$}
    {\True $H_2$}
    {$C_2H_4$}
    \loigiai{$H_2$.}
\end{ex}

\Closesolutionfile{ans}
\Closesolutionfile{ansex}

\phan{Bài tập trắc nghiệm Đúng Sai}
\Opensolutionfile{ansex}[Ans/LGTF-C5B44_Dang3]
\Opensolutionfile{ansbook}[Ansbook/AnsTF-C5B44_Dang3]
\Opensolutionfile{ans}[Ans/Tempt-C5B44_Dang3]

%%%%%============TF_01================%%%%%%
\begin{ex}
    Cho thí nghiệm: Cho mẩu Na vào ống nghiệm đựng Ethanol khan.
    \choiceTF
    {\True Khí thoát ra cháy được với ngọn lửa màu xanh nhạt.}
    {Dung dịch sau phản ứng làm quỳ tím hóa đỏ.}
    {\True Nếu thay Ethanol bằng nước, phản ứng xảy ra mãnh liệt hơn.}
    {Có thể thu khí $H_2$ bằng phương pháp dời chỗ nước.}
    \loigiai{
        \begin{itemchoice}[T1,F2,T3,T4]
            \itemch Khí $H_2$ cháy với ngọn lửa xanh nhạt.
            \itemch Dung dịch chứa $C_2H_5ONa$ thủy phân tạo kiềm làm quỳ hóa xanh.
            \itemch Nước tác dụng với Na mãnh liệt hơn Ethanol.
            \itemch $H_2$ ít tan trong nước, thu bằng phương pháp đẩy nước được.
        \end{itemchoice}
    }
\end{ex}
%%%%%============TF_02================%%%%%%
\begin{ex}
    Phản ứng của Ethanol với Na:
    \choiceTF
    {\True Chứng tỏ trong phân tử Ethanol có nguyên tử H linh động.}
    {Xảy ra do nhóm Ethyl $C_2H_5$ hút điện tử.}
    {\True Phản ứng này dùng để xác định số lượng nhóm -OH trong phân tử polyalcohol.}
    {Sản phẩm rắn thu được bền trong nước.}
    \loigiai{
        \begin{itemchoice}[T1,F2,T3,F4]
            \itemch Nguyên tử H trong nhóm -OH bị Na thay thế.
            \itemch Do liên kết O-H phân cực (O độ âm điện lớn).
            \itemch Phản ứng xác định số nhóm chức alcohol (nếu biết tỉ lệ mol).
            \itemch $C_2H_5ONa$ dễ bị thủy phân trong nước.
        \end{itemchoice}
    }
\end{ex}
%%%%%============TF_03================%%%%%%
\begin{ex}
    Cho 1 mol Ethanol và 1 mol Nước tác dụng hết với Na.
    \choiceTF
    {Thể tích khí thoát ra từ nước lớn hơn.}
    {\True Lượng khí thoát ra bằng nhau.}
    {\True Cả hai phản ứng đều là phản ứng oxi hóa khử.}
    {Sản phẩm rắn đều là các chất ion.}
    \loigiai{
        \begin{itemchoice}[F1,T2,T3,T4]
            \itemch Đều sinh ra 0.5 mol $H_2$ từ 1 mol chất.
            \itemch Tỉ lệ mol là 1:0.5.
            \itemch Na từ 0 lên +1, H từ +1 xuống 0.
            \itemch NaOH và $C_2H_5ONa$ là hợp chất ion.
        \end{itemchoice}
    }
\end{ex}
%%%%%============TF_04================%%%%%%
\begin{ex}
    So sánh Ethanol và Ethane ($C_2H_6$):
    \choiceTF
    {Cả hai đều tác dụng được với Na.}
    {\True Chỉ Ethanol tác dụng với Na.}
    {\True Ethanol có nhiệt độ sôi cao hơn Ethane.}
    {Cả hai đều tan tốt trong nước.}
    \loigiai{
        \begin{itemchoice}[F1,T2,T3,F4]
            \itemch Ethane không có nhóm -OH linh động.
            \itemch Chỉ H của nhóm -OH mới phản ứng.
            \itemch Ethanol tạo liên kết hydrogen liên phân tử.
            \itemch Ethane không tan trong nước.
        \end{itemchoice}
    }
\end{ex}
%%%%%============TF_05================%%%%%%
\begin{ex}
    Dung dịch $C_2H_5ONa$:
    \choiceTF
    {\True Có tính base mạnh hơn NaOH.}
    {Tác dụng được với HCl tạo lại Ethanol.}
    {\True Bị nước phân hủy: $C_2H_5ONa + H_2O \to C_2H_5OH + NaOH$.}
    {Là chất kết tủa trắng.}
    \loigiai{
        \begin{itemchoice}[T1,T2,T3,F4]
            \itemch Phản ứng thủy phân sinh ra $OH^-$ tự do.
            \itemch $C_2H_5ONa + HCl \to NaCl + C_2H_5OH$.
            \itemch $C_2H_5O^-$ hút proton mạnh.
            \itemch Là chất rắn màu trắng, tan tốt trong nước/ethanol.
        \end{itemchoice}
    }
\end{ex}
%%%%%============TF_06================%%%%%%
\begin{ex}
    Na dư vào rượu $40^\circ$:
    \choiceTF
    {\True Na tác dụng với nước trước, sau đó với rượu.}
    {Chỉ có rượu phản ứng.}
    {\True Thu được dung dịch có tính kiềm.}
    {Khí bay ra chỉ là $H_2O$.}
    \loigiai{
        \begin{itemchoice}[T1,F2,T3,F4]
            \itemch Tốc độ phản ứng với nước nhanh hơn.
            \itemch Cả nước và rượu đều phản ứng.
            \itemch $C_2H_5ONa$ và $NaOH$ đều có tính kiềm.
            \itemch Khí bay ra là Hydrogen ($H_2$).
        \end{itemchoice}
    }
\end{ex}

\Closesolutionfile{ans}
\Closesolutionfile{ansbook}
\Closesolutionfile{ansex}

%% ========================================================================
%% DẠNG 4: PHẢN ỨNG CHÁY
%% ========================================================================
\begin{dang}{Phản ứng cháy (Oxi hóa hoàn toàn)}
\end{dang}
\begin{phuongphap}
	\begin{itemize}
		\item  $C_2H_5OH + 3O_2 \xrightarrow[$t^\circ$] 2CO_2 + 3H_2O$.
		\item  Nhận xét: $n_{H_2O} > n_{CO_2}$ và $n_{Alcohol} = n_{H_2O} - n_{CO_2}$.
		\item  Phản ứng tỏa nhiều nhiệt $\to$ Dùng làm nhiên liệu.
	\end{itemize}
\end{phuongphap}

\Noibat[\maunhan][][\faBookmark][]{Ví dụ mẫu}
%%%%%==========VD_04==========%%%%%
\begin{vd}
    Đốt cháy hoàn toàn m gam Ethylic alcohol thu được 4.4 gam $CO_2$ và 2.7 gam $H_2O$. Giá trị của m là
    \choice
    {2.3}
    {4.6}
    {2.9}
    {5.8}
    \loigiai{
        $n_{CO_2} = 0.1$. $n_{H_2O} = 0.15$.
        $n_{Ancol} = 0.15 - 0.1 = 0.05$ mol.
        $m = 0.05 \times 46 = 2.3$ gam.
    }
\end{vd}

\Noibat[\maunhan][][\faBook][]{Bài tập tự luyện}
\phan{Bài tập tự luận}
\Opensolutionfile{ansbth}[Ans/LGBT-C5B44_Dang4]
\Opensolutionfile{ansbt}[Ans/AnsBT-C5B44_Dang4]

%%%%%============BT_01================%%%%%%
\begin{bt}
    Đốt cháy hoàn toàn 9.2 gam ethylic alcohol. Tính thể tích khí $CO_2$ sinh ra (đktc) và thể tích không khí cần dùng (biết $O_2$ chiếm 20\% thể tích không khí).
    \loigiai{
        $n = 0.2$ mol.
        $C_2H_6O + 3O_2 \to 2CO_2 + 3H_2O$.
        $n_{CO_2} = 0.4$ mol $\to V_{CO_2} = 8.96$ lít (nếu ĐKTC 22.4).
        $n_{O_2} = 0.6$ mol $\to V_{KK} = 0.6 \times 22.4 \times 5 = 67.2$ lít.
    }
\end{bt}
%%%%%============BT_02================%%%%%%
\begin{bt}
    Xăng sinh học E5 chứa 5\% ethylic alcohol về thể tích (còn lại là xăng khoáng) giúp bảo vệ môi trường. Tại sao ethanol được coi là nhiên liệu sạch hơn xăng truyền thống?
    \loigiai{
        Ethanol cháy hoàn toàn tạo $CO_2$ và $H_2O$, ít sinh ra các sản phẩm độc hại ($CO, SO_2...$). Ngoài ra $CO_2$ sinh ra được cây trồng hấp thụ lại (tính trung hòa carbon).
    }
\end{bt}
%%%%%============BT_03================%%%%%%
\begin{bt}
    Viết phương trình hóa học khi đốt cháy Ethanol, Methanol và Glycerol. Nhận xét về mối quan hệ giữa số mol $H_2O$ và $CO_2$ của các alcohol no, mạch hở này.
    \loigiai{
        Đều có $n_{H_2O} > n_{CO_2}$ và $n_{ancol} = n_{H_2O} - n_{CO_2}$.
    }
\end{bt}
%%%%%============BT_04================%%%%%%
\begin{bt}
    Khi đốt đèn cồn, nên dùng nắp đậy để tắt đèn thay vì thổi. Tại sao?
    \loigiai{
        Đậy nắp để ngăn cản nguồn oxi, làm tắt lửa. Nếu thổi mạnh có thể làm ngọn lửa bùng to hoặc tạt lửa gây nguy hiểm, và không ngắt được nguồn nhiên liệu/oxi triệt để trong nòng bấc.
    }
\end{bt}
%%%%%============BT_05================%%%%%%
\begin{bt}
    Đốt cháy hoàn toàn một lượng alcohol X thu được tỉ lệ số mol $n_{CO_2} : n_{H_2O} = 2:3$. Xác định CTPT của X.
    \loigiai{
        Số C : Số H = $n_C : 2n_{H_2O} = 2 : 6 = 1 : 3$.
        CTĐG nhất: $(CH_3)_n$.
        X no: $C_nH_{2n+2}O_x \to 2n+2 = 3n \to n=2$.
        X là $C_2H_6O$.
    }
\end{bt}
%%%%%============BT_06================%%%%%%
\begin{bt}
    Tính nhiệt lượng tỏa ra khi đốt cháy 1 kg ethanol, biết 1 mol ethanol cháy tỏa ra 1370 kJ. So sánh với 1 kg xăng (giả sử xăng là Octan, 1 mol tỏa 5460 kJ).
    \loigiai{
        $n_{Et} = 1000/46 \approx 21.74$ mol. $Q_{Et} = 21.74 \times 1370 = 29783.8$ kJ.
        $n_{Oct} = 1000/114 \approx 8.77$ mol. $Q_{Oct} = 8.77 \times 5460 = 47884.2$ kJ.
        Xăng tỏa nhiệt nhiều hơn ethanol trên cùng khối lượng.
    }
\end{bt}

\Closesolutionfile{ansbt}
\Closesolutionfile{ansbth}

\phan{Bài tập trả lời ngắn}
\Opensolutionfile{ansbth}[Ans/LGSA-C5B44_Dang4]
\Opensolutionfile{ansbt}[Ans/AnsSA-C5B44_Dang4]

%%%%%============SA_01================%%%%%%
\begin{ex}
    Tổng hệ số cân bằng của phản ứng cháy Ethylic Alcohol là?
    \shortans{9}
    \loigiai{$1+3+2+3 = 9$.}
\end{ex}
%%%%%============SA_02================%%%%%%
\begin{ex}
    Đốt cháy 0.1 mol $C_2H_5OH$ cần bao nhiêu lít $O_2$ (đktc - 24.79)?
    \shortans{7.437}
    \loigiai{$n_{O_2} = 0.3 \to 7.437$.}
\end{ex}
%%%%%============SA_03================%%%%%%
\begin{ex}
    Nếu đốt cháy hoàn toàn hỗn hợp gồm Methanol và Ethanol thì tỉ lệ $n_{H_2O}/n_{CO_2}$ có giá trị trong khoảng nào?
    \shortans{1.5-2}
    \loigiai{
        Methanol: $1/1 \to H/C = 4/1 \to H_2O/CO_2 = 2$.
        Ethanol: $H_2O/CO_2 = 3/2 = 1.5$.
        Khoảng (1.5; 2).
    }
\end{ex}
%%%%%============SA_04================%%%%%%
\begin{ex}
    Phần trăm khối lượng Carbon trong Ethylic Alcohol là bao nhiêu? (Làm tròn 1 số thập phân)
    \shortans{52.2}
    \loigiai{$24/46 \times 100 \approx 52.17\%$.}
\end{ex}
%%%%%============SA_05================%%%%%%
\begin{ex}
    Đốt cháy hoàn toàn một alcohol no đơn chức mạch hở X thu được 4 mol $CO_2$ và 5 mol $H_2O$. Số nguyên tử C trong X là?
    \shortans{4}
    \loigiai{
        $n_X = 5 - 4 = 1$.
        Số C = $4/1 = 4$.
    }
\end{ex}
%%%%%============SA_06================%%%%%%
\begin{ex}
    Khi đốt cháy 1 mol hợp chất hữu cơ A thu được 2 mol $CO_2$ và 3 mol $H_2O$. Chất A có phải là Ethanol không? (Trả lời Có/Không dưới dạng 1/0, 1 là có, 0 là chưa chắc)
    \shortans{0}
    \loigiai{
        $C_2H_6O_x$. Có thể là Ethanol hoặc Dimethyl ether. Chưa chắc.
    }
\end{ex}

\Closesolutionfile{ansbt}
\Closesolutionfile{ansbth}

\phan{Trắc nghiệm nhiều lựa chọn}
\Opensolutionfile{ansex}[Ans/LGEX-C5B44_Dang4]
\Opensolutionfile{ans}[Ans/Ans-C5B44_Dang4]

%%%%%============EX_01================%%%%%%
\begin{ex}
    Sản phẩm cháy của Ethylic Alcohol là
    \choice
    {$CO, H_2O$}
    {$CO_2, H_2$}
    {\True $CO_2, H_2O$}
    {$C, H_2O$}
    \loigiai{Cháy hoàn toàn ra $CO_2, H_2O$.}
\end{ex}
%%%%%============EX_02================%%%%%%
\begin{ex}
    Phản ứng cháy của rượu Etylic có đặc điểm:
    \choice
    {Thu nhiệt.}
    {\True Tỏa nhiệt mạnh, ngọn lửa màu xanh mờ.}
    {Tỏa nhiệt, ngọn lửa màu vàng khói đen.}
    {Không tỏa nhiệt.}
    \loigiai{Cháy sạch, tỏa nhiệt.}
\end{ex}
%%%%%============EX_03================%%%%%%
\begin{ex}
    Khi đốt cháy các alcohol no, đơn chức, mạch hở thì:
    \choice
    {$n_{H_2O} = n_{CO_2}$}
    {$n_{H_2O} < n_{CO_2}$}
    {\True $n_{H_2O} > n_{CO_2}$}
    {$n_{H_2O} = 0.5 n_{CO_2}$}
    \loigiai{$C_nH_{2n+2}O \to nCO_2 + (n+1)H_2O$.}
\end{ex}
%%%%%============EX_04================%%%%%%
\begin{ex}
    Chất nào sau đây có sinh nhiệt cháy cao nhất (trên cùng 1 mol)?
    \choice
    {Methanol}
    {\True Ethanol}
    {Hydrogen}
    {Carbon}
    \loigiai{Ethanol phân tử lớn hơn Methanol, cháy tỏa nhiều nhiệt hơn.}
\end{ex}
%%%%%============EX_05================%%%%%%
\begin{ex}
    Để dập tắt đám cháy do cồn, người ta dùng:
    \choice
    {Nước.}
    {\True Cát, chăn chiên trùm kín hoặc bình chữa cháy $CO_2$.}
    {Quạt gió.}
    {Xăng.}
    \loigiai{Cồn nhẹ hơn nước tan trong nước nên dùng nước làm đám cháy lan rộng. Dùng cát hoặc $CO_2$.}
\end{ex}
%%%%%============EX_06================%%%%%%
\begin{ex}
    Phương trình phản ứng cháy đúng là:
    \choice
    {$C_2H_5OH + O_2 \to CO_2 + H_2O$}
    {$C_2H_5OH + 2O_2 \to 2CO_2 + 3H_2O$}
    {\True $C_2H_5OH + 3O_2 \to 2CO_2 + 3H_2O$}
    {$2C_2H_5OH + 7O_2 \to 4CO_2 + 6H_2O$}
    \loigiai{Cân bằng C, H, O.}
\end{ex}

\Closesolutionfile{ans}
\Closesolutionfile{ansex}

\phan{Bài tập trắc nghiệm Đúng Sai}
\Opensolutionfile{ansex}[Ans/LGTF-C5B44_Dang4]
\Opensolutionfile{ansbook}[Ansbook/AnsTF-C5B44_Dang4]
\Opensolutionfile{ans}[Ans/Tempt-C5B44_Dang4]

%%%%%============TF_01================%%%%%%
\begin{ex}
    Về phản ứng cháy của Ethanol:
    \choiceTF
    {\True Phản ứng tỏa nhiều nhiệt nên Ethanol được dùng làm nhiên liệu.}
    {Ngọn lửa Ethanol có màu vàng sáng và nhiều muội đen.}
    {\True Số mol nước sinh ra luôn lớn hơn số mol $CO_2$.}
    {Cần 3 mol $O_2$ để đốt cháy hoàn toàn 1 mol Ethanol.}
    \loigiai{
        \begin{itemchoice}[T1,F2,T3,T4]
            \itemch Phản ứng cháy tỏa nhiều nhiệt.
            \itemch Ngọn lửa màu xanh nhạt, không có muội than.
            \itemch Alcohol no hở: $n_{H_2O} > n_{CO_2}$.
            \itemch $C_2H_6O + 3O_2 \to 2CO_2 + 3H_2O$.
        \end{itemchoice}
    }
\end{ex}
%%%%%============TF_02================%%%%%%
\begin{ex}
    Đốt cháy 1 mol Alcohol X thu được 2 mol $CO_2$ và 3 mol $H_2O$.
    \choiceTF
    {\True X là alcohol no, mạch hở.}
    {X chắc chắn là đơn chức.}
    {\True Số nguyên tử Carbon trong X là 2.}
    {X có thể là $CH_3-O-CH_3$ (nếu đề chỉ nói hợp chất hữu cơ, nhưng đề nói Alcohol).}
    \loigiai{
        \begin{itemchoice}[T1,F2,T3,F4]
            \itemch $H_2O > CO_2$ là dấu hiệu alcohol no.
            \itemch Có thể là Polyalcohol (vd Ethylene glycol).
            \itemch Số C = $n_{CO_2} / (n_{H_2O} - n_{CO_2}) = 2$.
            \itemch Đề bài đã cho X là Alcohol.
        \end{itemchoice}
    }
\end{ex}
%%%%%============TF_03================%%%%%%
\begin{ex}
    So sánh nhiên liệu Xăng và Cồn:
    \choiceTF
    {\True Cồn cháy sạch hơn xăng, ít khí thải độc hại.}
    {Năng lượng tỏa ra của 1 kg cồn cao hơn 1 kg xăng.}
    {\True Xăng E5 là hỗn hợp chứa 5\% cồn và 95\% xăng A92.}
    {Động cơ chạy bằng cồn cần cải tiến so với động cơ xăng.}
    \loigiai{
        \begin{itemchoice}[T1,F2,T3,T4]
            \itemch Cháy hoàn toàn, ít khí độc hơn.
            \itemch Nhiệt trị của xăng cao hơn cồn.
            \itemch 5\% ethanol và 95\% xăng A92/95.
            \itemch Do đặc tính bay hơi và nhiệt trị khác nhau.
        \end{itemchoice}
    }
\end{ex}
%%%%%============TF_04================%%%%%%
\begin{ex}
    Trong phòng thí nghiệm:
    \choiceTF
    {\True Dùng cồn 90 độ cho đèn cồn thay vì cồn tuyệt đối (đắt tiền).}
    {Khi muốn dập đèn cồn, ta thổi mạnh.}
    {\True Không được đổ thêm cồn vào đèn cồn khi đèn đang cháy.}
    {Nếu làm đổ cồn cháy ra bàn, dùng nước hắt vào.}
    \loigiai{
        \begin{itemchoice}[T1,F2,T3,F4]
            \itemch Nhiên liệu thông dụng, giá rẻ.
            \itemch Thổi làm lửa tạt, nguy hiểm. Phải đậy nắp.
            \itemch Rất nguy hiểm, dễ gây cháy ngược vào bình.
            \itemch Cồn tan trong nước, nước làm đám cháy lan rộng.
        \end{itemchoice}
    }
\end{ex}
%%%%%============TF_05================%%%%%%
\begin{ex}
    Tính chất:
    \choiceTF
    {\True $C_2H_5OH + 3O_2 \to 2CO_2 + 3H_2O$.}
    {\True Hỗn hợp hơi Ethanol và không khí có thể gây nổ.}
    {Ethanol cháy được là do nhóm -OH.}
    {Nước sinh ra trong phản ứng cháy bằng tổng khối lượng H trong rượu và O trong không khí.}
    \loigiai{
        \begin{itemchoice}[T1,T2,F3,F4]
            \itemch Cân bằng C, H, O.
            \itemch Trong giới hạn nổ nhất định.
            \itemch Phản ứng cháy là oxi hóa toàn bộ C và H.
            \itemch Oxi trong nước đến từ cả Alcohol và khí trời.
        \end{itemchoice}
    }
\end{ex}
%%%%%============TF_06================%%%%%%
\begin{ex}
    Đốt cháy hoàn toàn m gam hỗn hợp Methanol và Ethanol.
    \choiceTF
    {\True $n_{H_2O} > n_{CO_2}$.}
    {Nếu số mol 2 chất bằng nhau thì $n_{CO_2} = 1.5 n_{hh}$.}
    {\True Khối lượng $CO_2$ thu được lớn hơn khối lượng $H_2O$.}
    {Cần dùng lượng oxi như nhau để đốt cháy m gam mỗi chất.}
    \loigiai{
        \begin{itemchoice}[T1,T2,T3,F4]
            \itemch Đều no đơn chức mạch hở.
            \itemch $n_{CO_2} = (1+2)/2 = 1.5$ mol.
            \itemch $m_{CO_2} = 44 \times 1.5 = 66g > m_{H_2O} = 18 \times 2.5 = 45g$.
            \itemch Methanol cần ít Oxi hơn ($1.5$ mol so với $3$ mol).
        \end{itemchoice}
    }
\end{ex}

\Closesolutionfile{ans}
\Closesolutionfile{ansbook}
\Closesolutionfile{ansex}


%% ========================================================================
%% DẠNG 5: PHẢN ỨNG ESTER HÓA VÀ OXI HÓA
%% ========================================================================
\begin{dang}{Phản ứng với Acid (Ester hóa) và Oxi hóa}
\end{dang}
\begin{phuongphap}
	\begin{itemize}
		\item  Ester hóa: $C_2H_5OH + CH_3COOH \rightleftharpoons CH_3COOC_2H_5 + H_2O$ ($H_2SO_4$ đặc, $t^\circ$).
		\item  Oxi hóa không hoàn toàn: $C_2H_5OH + CuO \xrightarrow[$t^\circ$] CH_3CHO + Cu + H_2O$.
		\item  Oxi hóa chậm (lên men giấm): $C_2H_5OH + O_2 \xrightarrow[$\text{men}$] CH_3COOH + H_2O$.
	\end{itemize}
\end{phuongphap}

\Noibat[\maunhan][][\faBookmark][]{Ví dụ mẫu}
%%%%%==========VD_05==========%%%%%
\begin{vd}
    Cho ethylic alcohol tác dụng với acetic acid có xúc tác $H_2SO_4$ đặc, đun nóng. Sản phẩm thu được có tên gọi là
    \choice
    {Methyl acetate}
    {Methyl formate}
    {\True Ethyl acetate}
    {Ethyl formate}
    \loigiai{
        $C_2H_5OH + CH_3COOH \rightleftharpoons CH_3COOC_2H_5 + H_2O$.
        Sản phẩm là Ethyl acetate.
    }
\end{vd}

\Noibat[\maunhan][][\faBook][]{Bài tập tự luyện}
\phan{Bài tập tự luận}
\Opensolutionfile{ansbth}[Ans/LGBT-C5B44_Dang5]
\Opensolutionfile{ansbt}[Ans/AnsBT-C5B44_Dang5]

%%%%%============BT_01================%%%%%%
\begin{bt}
    Viết phương trình hóa học của phản ứng xảy ra khi đun nóng ethylic alcohol với CuO. Nêu hiện tượng quan sát được.
    \loigiai{
        $C_2H_5OH + CuO (đen) \xrightarrow[$t^\circ$] CH_3CHO + Cu (\text{đỏ}) + H_2O$.
        Hiện tượng: Chất rắn chuyển từ màu đen sang màu đỏ gạch, có hơi mùi táo (aldehyde) bay lên.
    }
\end{bt}
%%%%%============BT_02================%%%%%%
\begin{bt}
    Tính khối lượng Ethyl acetate thu được khi đun nóng 6 gam Acetic acid với 6 gam Ethylic alcohol với xúc tác $H_2SO_4$ đặc. Biết hiệu suất phản ứng đạt 60\%.
    \loigiai{
        $n_{acid} = 0.1$ mol. $n_{ancol} \approx 0.13$ mol.
        Tính theo acid.
        $n_{ester\ LT} = 0.1$ mol.
        $n_{ester\ TT} = 0.1 \times 0.6 = 0.06$ mol.
        $m = 0.06 \times 88 = 5.28$ gam.
    }
\end{bt}
%%%%%============BT_03================%%%%%%
\begin{bt}
    Tại sao phản ứng ester hóa phải dùng $H_2SO_4$ đặc mà không dùng $H_2SO_4$ loãng?
    \loigiai{
        $H_2SO_4$ đặc có 2 vai trò: Vừa làm xúc tác, vừa hút nước làm cân bằng chuyển dịch theo chiều thuận (chiều tạo ester).
    }
\end{bt}
%%%%%============BT_04================%%%%%%
\begin{bt}
    Giấm ăn được điều chế bằng cách nào từ rượu loãng? Viết phương trình hóa học.
    \loigiai{
        Lên men giấm rượu loãng (dưới 10 độ).
        $C_2H_5OH + O_2 \xrightarrow{men\ giam} CH_3COOH + H_2O$.
    }
\end{bt}

\Closesolutionfile{ansbt}
\Closesolutionfile{ansbth}

\phan{Bài tập trả lời ngắn}
\Opensolutionfile{ansbth}[Ans/LGSA-C5B44_Dang5]
\Opensolutionfile{ansbt}[Ans/AnsSA-C5B44_Dang5]

%%%%%============SA_01================%%%%%%
\begin{ex}
    Phân tử khối của Ethyl acetate là bao nhiêu?
    \shortans{88}
    \loigiai{$CH_3COOC_2H_5 \to 15+44+29 = 88$.}
\end{ex}
%%%%%============SA_02================%%%%%%
\begin{ex}
    Đun nóng 0.1 mol ethanol với CuO dư, thu được bao nhiêu gam kim loại đồng?
    \shortans{6.4}
    \loigiai{$n_{Cu} = n_{ancol} = 0.1$ mol $\to m = 6.4$ gam.}
\end{ex}
%%%%%============SA_03================%%%%%%
\begin{ex}
    Tổng hệ số cân bằng của phản ứng $C_2H_5OH + O_2 \xrightarrow{men\ giam} CH_3COOH + H_2O$ là bao nhiêu?
    \shortans{4}
    \loigiai{$1+1 \to 1+1$. Tổng = 4.}
\end{ex}
%%%%%============SA_04================%%%%%%
\begin{ex}
    Hiệu suất phản ứng ester hóa giữa 1 mol Acid và 1 mol Alcohol tạo ra 0.667 mol Ester là bao nhiêu phần trăm? (Làm tròn 1 số lẻ)
    \shortans{66.7}
    \loigiai{66.7\%.}
\end{ex}

\Closesolutionfile{ansbt}
\Closesolutionfile{ansbth}

\phan{Trắc nghiệm nhiều lựa chọn}
\Opensolutionfile{ansex}[Ans/LGEX-C5B44_Dang5]
\Opensolutionfile{ans}[Ans/Ans-C5B44_Dang5]

%%%%%============EX_01================%%%%%%
\begin{ex}
    Oxi hóa ethylic alcohol bằng CuO đun nóng, thu được sản phẩm hữu cơ là
    \choice
    {$CH_3COOH$}
    {\True $CH_3CHO$}
    {$CO_2$}
    {$CH_4$}
    \loigiai{Aldehyde acetic ($CH_3CHO$).}
\end{ex}
%%%%%============EX_02================%%%%%%
\begin{ex}
    Phản ứng ester hóa là phản ứng
    \choice
    {Một chiều.}
    {\True Thuận nghịch.}
    {Oxi hóa khử.}
    {Phân hủy.}
    \loigiai{Thuận nghịch.}
\end{ex}
%%%%%============EX_03================%%%%%%
\begin{ex}
    Dấm ăn là dung dịch của chất nào sau đây?
    \choice
    {Ethylic alcohol.}
    {\True Acetic acid.}
    {Ethyl acetate.}
    {Sodium chloride.}
    \loigiai{Acetic acid.}
\end{ex}

\Closesolutionfile{ans}
\Closesolutionfile{ansex}

\phan{Bài tập trắc nghiệm Đúng Sai}
\Opensolutionfile{ansex}[Ans/LGTF-C5B44_Dang5]
\Opensolutionfile{ansbook}[Ansbook/AnsTF-C5B44_Dang5]
\Opensolutionfile{ans}[Ans/Tempt-C5B44_Dang5]

%%%%%============TF_01================%%%%%%
\begin{ex}
    Về phản ứng Ester hóa:
    \choiceTF
    {\True Cần xúc tác $H_2SO_4$ đặc và đun nóng.}
    {Phản ứng xảy ra hoàn toàn tạo 100\% sản phẩm.}
    {\True Sản phẩm ester thường nhẹ hơn nước và có mùi thơm.}
    {$H_2SO_4$ đặc chỉ đóng vai trò xúc tác.}
    \loigiai{
        \begin{itemchoice}[T1,F2,T3,F4]
            \itemch Acid làm xúc tác và nhiệt làm tăng tốc độ phản ứng.
            \itemch Phản ứng thuận nghịch 2 chiều (hiệu suất $<100\%$).
            \itemch Ethyl acetate nhẹ hơn nước, ít tan và có mùi thơm.
            \itemch Hút nước chuyển dịch cân bằng (vai trò hóa học).
        \end{itemchoice}
    }
\end{ex}
%%%%%============TF_02================%%%%%%
\begin{ex}
    Phản ứng oxi hóa Ethanol:
    \choiceTF
    {\True Đốt cháy là phản ứng oxi hóa hoàn toàn.}
    {\True Lên men giấm là phản ứng oxi hóa không hoàn toàn.}
    {Cho hơi Ethanol đi qua CuO nung nóng thu được Acid Acetic.}
    {Ethanol để lâu trong không khí không bị biến đổi.}
    \loigiai{
        \begin{itemchoice}[T1,T2,F3,F4]
            \itemch Oxi hóa C đến mức oxi hóa cao nhất ($+4$).
            \itemch Oxi hóa C đến mức oxi hóa $+3$ (Acid).
            \itemch Thu được Aldehyde acetic ($CH_3CHO$).
            \itemch Bị oxi hóa chậm bởi oxi không khí.
        \end{itemchoice}
    }
\end{ex}

\Closesolutionfile{ans}
\Closesolutionfile{ansbook}
\Closesolutionfile{ansex}


%% ========================================================================
%% DẠNG 6: NHẬN BIẾT VÀ TÁCH BIỆT CHẤT
%% ========================================================================
\begin{dang}{Nhận biết và Tách biệt chất}
\end{dang}
\begin{phuongphap}
	\begin{itemize}
		\item  Nhận biết Ethanol: Dùng Na (sủi bọt khí), hoặc đốt cháy (lửa xanh, không khói). Phân biệt với nước dùng đốt cháy hoặc tính tan của chất khác.
		\item  Tách biệt: Phương pháp chưng cất (dựa vào nhiệt độ sôi khác nhau).
	\end{itemize}
\end{phuongphap}

\Noibat[\maunhan][][\faBookmark][]{Ví dụ mẫu}
%%%%%==========VD_06==========%%%%%
\begin{vd}
    Để phân biệt Cồn 90 độ và Nước cất, ta có thể dùng cách nào đơn giản nhất?
    \choice
    {Dùng Na.}
    {Dùng $CuSO_4$ khan.}
    {\True Đốt cháy mỗi mẫu thử.}
    {Ngửi mùi.}
    \loigiai{
        Cồn 90 độ cháy được, nước không cháy. Na tác dụng cả 2. Ngửi mùi cũng được nhưng đốt cháy rõ ràng hơn về tính chất hóa học (đề thường yêu cầu hóa học, nhưng thực tế ngửi là nhanh nhất). Nếu chọn trắc nghiệm, đốt cháy là tính chất hóa học đặc trưng phân biệt nhiên liệu.
    }
\end{vd}

\Noibat[\maunhan][][\faBook][]{Bài tập tự luyện}
\phan{Bài tập tự luận}
\Opensolutionfile{ansbth}[Ans/LGBT-C5B44_Dang6]
\Opensolutionfile{ansbt}[Ans/AnsBT-C5B44_Dang6]

%%%%%============BT_01================%%%%%%
\begin{bt}
    Có 3 bình mất nhãn đựng: Ethanol, Nước, Dung dịch Acid Acetic. Hãy nêu phương pháp hóa học nhận biết chúng.
    \loigiai{ 
        \begin{enumerate}[a)]
        	\item  Dùng quỳ tím:Hóa đỏ: Acid Acetic.
        	\item  Dùng Na cho vào 2 mẫu còn lại:Cả 2 đều sủi bọt
        	\item  Đốt cháy $\to$ Ethanol cháy, Nước không cháy.
        \end{enumerate}
    }
\end{bt}
%%%%%============BT_02================%%%%%%
\begin{bt}
    Trình bày phương pháp tách riêng Ethyl alcohol ra khỏi hỗn hợp với nước.
    \loigiai{
        Dùng phương pháp chưng cất phân đoạn. Đun nóng hỗn hợp, Ethanol sôi ở $78.3^\circ C$ bay hơi trước, ngưng tụ thu được Ethanol. Nước sôi ở $100^\circ C$ ở lại bình đun. (Lưu ý chỉ thu được cồn 96\% do đẳng phí).
    }
\end{bt}

\Closesolutionfile{ansbt}
\Closesolutionfile{ansbth}

\phan{Trắc nghiệm nhiều lựa chọn}
\Opensolutionfile{ansex}[Ans/LGEX-C5B44_Dang6]
\Opensolutionfile{ans}[Ans/Ans-C5B44_Dang6]

%%%%%============EX_01================%%%%%%
\begin{ex}
    Phương pháp tách rượu ra khỏi hỗn hợp rượu và nước là
    \choice
    {Lọc.}
    {Chưng cất lôi cuốn hơi nước.}
    {\True Chưng cất phân đoạn.}
    {Chiết.}
    \loigiai{Dựa vào nhiệt độ sôi khác nhau.}
\end{ex}
%%%%%============EX_02================%%%%%%
\begin{ex}
    Thuốc thử dùng để nhận biết Ethanol và Benzene là
    \choice
    {Quỳ tím.}
    {Na.}
    {\True Nước.}
    {Dung dịch Brom.}
    \loigiai{Ethanol tan trong nước, Benzene phân lớp.}
\end{ex}

\Closesolutionfile{ans}
\Closesolutionfile{ansex}

\phan{Bài tập trắc nghiệm Đúng Sai}
\Opensolutionfile{ansex}[Ans/LGTF-C5B44_Dang6]
\Opensolutionfile{ansbook}[Ansbook/AnsTF-C5B44_Dang6]
\Opensolutionfile{ans}[Ans/Tempt-C5B44_Dang6]

%%%%%============TF_01================%%%%%%
\begin{ex}
    Nhận biết và tách chất:
    \choiceTF
    {\True Có thể phân biệt Ethanol và Nước bằng cách đốt cháy.}
    {Có thể tách hoàn toàn nước ra khỏi Ethanol chỉ bằng chưng cất thường.}
    {\True Để làm khan Ethanol, người ta dùng CaO mới nung.}
    {Ethanol và Nước là hai chất lỏng tan vô hạn vào nhau nên không thể tách rời.}
    \loigiai{
        \begin{itemchoice}[T1,F2,T3,F4]
            \itemch Ethanol cháy, nước không cháy.
            \itemch Không tách được nước ở điểm đẳng phí (96\%) bằng chưng cất thường.
            \itemch CaO phản ứng nước tạo $Ca(OH)_2$ không bay hơi, chưng cất thu được cồn khan.
            \itemch Chưng cất phân đoạn tách được.
        \end{itemchoice}
    }
\end{ex}

\Closesolutionfile{ans}
\Closesolutionfile{ansbook}
\Closesolutionfile{ansex}

%% ========================================================================
%% DẠNG 7: BÀI TOÁN TỔNG HỢP VÀ THỰC TẾ
%% ========================================================================
\begin{dang}{Bài toán tổng hợp và Thực tế}
\end{dang}
\begin{phuongphap}
	\begin{itemize}
		\item  Kết hợp các định luật bảo toàn khối lượng, mol nguyên tố.
		\item  Bài toán hiệu suất phản ứng nhiều giai đoạn.
		\item  Bài toán nồng độ cồn, xăng E5, ý nghĩa thực tiễn.
	\end{itemize}
\end{phuongphap}

\Noibat[\maunhan][][\faBook][]{Bài tập tự luyện}
\phan{Bài tập tự luận}
\Opensolutionfile{ansbth}[Ans/LGBT-C5B44_Dang7]
\Opensolutionfile{ansbt}[Ans/AnsBT-C5B44_Dang7]

%%%%%============BT_01================%%%%%%
\begin{bt}
    Từ 10 kg gạo nếp (có 80\% tinh bột), khi lên men sẽ thu được bao nhiêu lít cồn $96^\circ$? Biết hiệu suất quá trình lên men đạt 80\% và khối lượng riêng của cồn nguyên chất là 0.8 g/ml.
    \loigiai{
        $m_{TB} = 8000$ g.\\
        $n_{TB} \approx 49.38$ mol ($C_6H_{10}O_5$).\\
        $n_{ancol\ LT} = 2n_{TB} \approx 98.76$ mol.\\
        $n_{ancol\ TT} = 98.76 \times 0.8 \approx 79$ mol.\\
        $m_{ancol} = 79 \times 46 = 3634$ g.\\
        $V_{nc} = 3634 / 0.8 \approx 4542.5$ ml.\\
        $V_{\text{cồn}\ 96} = 4542.5 / 0.96 \approx 4731$ ml = 4.73 lít.
    }
\end{bt}
%%%%%============BT_02================%%%%%%
\begin{bt}
    Một người lái xe uống 500ml bia (5\% độ cồn, $D=0.8g/ml$). 
	 \begin{enumerate}[a)]
	 	\item  Tính lượng alcohol nguyên chất nạp vào cơ thể (gam).
	 	\item  Để loại bỏ hết lượng cồn này cần bao lâu? Biết tốc độ đào thải trung bình của gan là 7g/giờ (con số giả định trung bình).
	 \end{enumerate}
    \loigiai{
		\begin{enumerate}[a)]
			\item  $V_{ancol} = 500 \times 0.05 = 25$ ml. $m = 25 \times 0.8 = 20$ g.
			\item  Thời gian = $20 / 7 \approx 2.85$ giờ (gần 3 tiếng).
		\end{enumerate}
    }
\end{bt}

\Closesolutionfile{ansbt}
\Closesolutionfile{ansbth}

\phan{Trắc nghiệm nhiều lựa chọn}
\Opensolutionfile{ansex}[Ans/LGEX-C5B44_Dang7]
\Opensolutionfile{ans}[Ans/Ans-C5B44_Dang7]

%%%%%============EX_01================%%%%%%
\begin{ex}
    Thiết bị kiểm tra nồng độ cồn trong hơi thở hoạt động dựa trên phản ứng của Ethanol với $CrO_3$ (màu đỏ thẫm) tạo thành muối Cr(III) có màu lục. Hiện tượng này giúp cảnh sát giao thông phát hiện người uống rượu. Đây là phản ứng:
    \choice
    {Ester hóa.}
    {\True Oxi hóa - khử.}
    {Trung hòa.}
    {Thủy phân.}
    \loigiai{Oxi hóa rượu bởi Chrome(VI).}
\end{ex}

\Closesolutionfile{ans}
\Closesolutionfile{ansex}

\phan{Bài tập trắc nghiệm Đúng Sai}
\Opensolutionfile{ansex}[Ans/LGTF-C5B44_Dang7]
\Opensolutionfile{ansbook}[Ansbook/AnsTF-C5B44_Dang7]
\Opensolutionfile{ans}[Ans/Tempt-C5B44_Dang7]

%%%%%============TF_01================%%%%%%
\begin{ex}
    Nhiên liệu và môi trường:
    \choiceTF
    {\True Sử dụng xăng E5 giúp giảm sự phụ thuộc vào nhiên liệu hóa thạch.}
    {Xăng E5 đắt hơn xăng khoáng A95 nên ít được ưa chuộng.}
    {\True Khí thải từ động cơ đốt ethanol chứa ít CO và HC (hydrocarbon) hơn xăng thường.}
    {Ethanol là nguồn năng lượng tái tạo.}
    \loigiai{
        \begin{itemchoice}[T1,F2,T3,T4]
            \itemch Ethanol từ nguồn sinh học tái tạo.
            \itemch Giá rẻ hơn xăng khoáng.
            \itemch Chứa oxy giúp cháy sạch hơn.
            \itemch Nguồn gốc từ thực vật.
        \end{itemchoice}
    }
\end{ex}

\Closesolutionfile{ans}
\Closesolutionfile{ansbook}
\Closesolutionfile{ansex}

\end{document}

