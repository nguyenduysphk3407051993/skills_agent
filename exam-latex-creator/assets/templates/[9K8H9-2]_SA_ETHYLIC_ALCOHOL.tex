%%%%%============SA_1================%%%%%%
\begin{ex}%[9K8H9-2]
	Trong phản ứng thế giữa ethylic alcohol và kim loại sodium ($Na$), tỉ lệ số mol giữa ethylic alcohol tham gia phản ứng và khí hydrogen ($H_2$) sinh ra là bao nhiêu?
	\shortans{2}
	\loigiai{
		Phương trình hóa học:
		\[ 2C_2H_5OH + 2Na \longrightarrow 2C_2H_5ONa + H_2 \]
		Dựa vào phương trình, cứ 2 mol $C_2H_5OH$ phản ứng sẽ sinh ra 1 mol $H_2$. Vậy tỉ lệ là $2:1$ hay giá trị là 2.
	}
\end{ex}

%%%%%============SA_2================%%%%%%
\begin{ex}%[9K8H9-2]
	Khi đốt cháy hoàn toàn 1 mol ethylic alcohol, số mol hơi nước ($H_2O$) thu được là bao nhiêu?
	\shortans{3}
	\loigiai{
		Phương trình hóa học của phản ứng cháy:
		\[ C_2H_5OH + 3O_2 \xrightarrow[t^\circ][] 2CO_2 + 3H_2O \]
		Từ phương trình ta thấy: đốt cháy 1 mol $C_2H_5OH$ sinh ra 3 mol $H_2O$.
	}
\end{ex}

%%%%%============SA_3================%%%%%%
\begin{ex}%[9K8H9-2]
	Trong quá trình lên men rượu từ glucose ($C_6H_{12}O_6$), nếu hiệu suất phản ứng là $100\%$, từ 1 phân tử glucose sẽ thu được bao nhiêu phân tử ethylic alcohol?
	\shortans{2}
	\loigiai{
		Sơ đồ phản ứng lên men rượu:
		\[ C_6H_{12}O_6 \xrightarrow[\text{men rượu}][30-32^\circ C] 2C_2H_5OH + 2CO_2 \]
		Theo tỉ lệ phương trình, 1 phân tử glucose tạo ra 2 phân tử ethylic alcohol.
	}
\end{ex}

%%%%%============SA_4================%%%%%%
\begin{ex}%[9K8H9-2]
	Tổng hệ số cân bằng (là các số nguyên, tối giản) của các chất tham gia phản ứng (ethylic alcohol và oxygen) trong phương trình đốt cháy hoàn toàn ethylic alcohol là bao nhiêu?
	\shortans{4}
	\loigiai{
		Phương trình hóa học:
		\[ 1C_2H_5OH + 3O_2 \xrightarrow[t^\circ][] 2CO_2 + 3H_2O \]
		Tổng hệ số các chất tham gia = Hệ số $C_2H_5OH$ + Hệ số $O_2$ = $1 + 3 = 4$.
	}
\end{ex}

%%%%%============SA_5================%%%%%%
\begin{ex}%[9K8H9-2]
	Trong phản ứng của ethylic alcohol với kim loại sodium ($Na$), nguyên tử của nguyên tố nào trong nhóm chức hydroxyl ($-OH$) đã bị nguyên tử sodium thay thế?
	\shortans{Hydrogen}
	\loigiai{
		Phản ứng giữa ethylic alcohol và sodium là phản ứng thế. Nguyên tử sodium ($Na$) đã thay thế nguyên tử hydrogen ($H$) trong nhóm $-OH$ để tạo thành muối sodium ethoxide ($C_2H_5ONa$) và giải phóng khí $H_2$.
	}
\end{ex}