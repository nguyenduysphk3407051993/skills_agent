%%%%%============TF_1================%%%%%%
\begin{ex}%[9K8H9-2]
	Tiến hành thí nghiệm cho mẩu sodium (Na) vào ống nghiệm chứa ethylic alcohol khan. Các phát biểu sau đây về thí nghiệm này:
	\choiceTF
	{\True Mẩu sodium tan dần và có bọt khí không màu thoát ra}
	{Phản ứng xảy ra mãnh liệt hơn so với phản ứng của sodium với nước}
	{\True Khí thoát ra là hydrogen ($H_2$)}
	{Sản phẩm thu được trong ống nghiệm là $NaOH$ và $H_2$}
	\loigiai{
		\begin{itemchoice}[T1,F2,T3,F4]
			\itemch Đúng. Hiện tượng quan sát được là Na tan dần, sủi bọt khí.
			\itemch Sai. Phản ứng của ethylic alcohol với Na êm dịu hơn (chậm hơn) so với phản ứng của nước với Na.
			\itemch Đúng. Phương trình: $2C_2H_5OH + 2Na \longrightarrow 2C_2H_5ONa + H_2$.
			\itemch Sai. Sản phẩm là sodium ethoxide ($C_2H_5ONa$), không phải $NaOH$.
		\end{itemchoice}
	}
\end{ex}

%%%%%============TF_2================%%%%%%
\begin{ex}%[9K8H9-2]
	Về phản ứng đốt cháy ethylic alcohol:
	\choiceTF
	{\True Phản ứng tỏa nhiều nhiệt, nên ethylic alcohol được dùng làm nhiên liệu}
	{Khi đốt cháy $1$ mol ethylic alcohol thu được số mol $CO_2$ bằng số mol $H_2O$}
	{\True Ngọn lửa khi đốt cháy ethylic alcohol có màu xanh mờ, không gây muội than}
	{Cần cung cấp nhiệt độ ban đầu để phản ứng cháy xảy ra}
	\loigiai{
		\begin{itemchoice}[T1,F2,T3,T4]
			\itemch Đúng. Phản ứng cháy của cồn tỏa nhiệt lượng lớn.
			\itemch Sai. Tỉ lệ mol $CO_2 : H_2O$ là $2 : 3$ (vì $C_2H_5OH + 3O_2 \xrightarrow[t^\circ][] 2CO_2 + 3H_2O$).
			\itemch Đúng. Cồn cháy sạch, ngọn lửa màu xanh.
			\itemch Đúng. Cần mồi lửa để bắt đầu phản ứng.
		\end{itemchoice}
	}
\end{ex}

%%%%%============TF_3================%%%%%%
\begin{ex}%[9K8H9-2]
	Cho các nhận định về nguyên nhân gây ra tính chất hóa học của ethylic alcohol:
	\choiceTF
	{Ethylic alcohol phản ứng được với Na do trong phân tử có chứa nguyên tử oxygen}
	{\True Nguyên tử hydrogen trong nhóm $-OH$ linh động hơn nguyên tử hydrogen trong gốc hydrocarbon nên dễ bị thay thế bởi Na}
	{Ethylic alcohol cháy được là do có thành phần carbon và hydrogen tương tự hydrocarbon}
	{\True Nhóm $-OH$ là trung tâm phản ứng giúp phân biệt ethylic alcohol với các hydrocarbon như ethane ($C_2H_6$)}
	\loigiai{
		\begin{itemchoice}[F1,T2,T3,T4]
			\itemch Sai. Hợp chất chứa O như ether ($CH_3OCH_3$) không phản ứng với Na. Phản ứng do nhóm $-OH$ quyết định.
			\itemch Đúng. Liên kết $O-H$ phân cực làm H linh động.
			\itemch Đúng. Các hợp chất hữu cơ chứa C, H thường dễ cháy.
			\itemch Đúng. Hydrocarbon không có nhóm $-OH$ nên không tác dụng với Na.
		\end{itemchoice}
	}
\end{ex}

%%%%%============TF_4================%%%%%%
\begin{ex}%[9K8H9-2]
	Quá trình lên men tinh bột để sản xuất ethylic alcohol (rượu) thường diễn ra theo sơ đồ: Tinh bột $\longrightarrow$ Glucose $\longrightarrow$ Rượu.
	\choiceTF
	{\True Cần sử dụng men rượu (enzyme) làm xúc tác cho quá trình chuyển hóa glucose thành rượu}
	{Trong quá trình lên men, ngoài thu được ethylic alcohol còn có khí oxygen thoát ra}
	{\True Từ $1$ phân tử glucose ($C_6H_{12}O_6$) qua lên men thu được $2$ phân tử ethylic alcohol}
	{Quá trình lên men rượu xảy ra tốt nhất ở nhiệt độ rất cao (trên $100^\circ C$)}
	\loigiai{
		\begin{itemchoice}[T1,F2,T3,F4]
			\itemch Đúng. Enzyme là xúc tác sinh học đặc hiệu.
			\itemch Sai. Khí thoát ra là carbon dioxide ($CO_2$).
			\itemch Đúng. Phương trình: $C_6H_{12}O_6 \xrightarrow[\text{men}][] 2C_2H_5OH + 2CO_2$.
			\itemch Sai. Nhiệt độ quá cao sẽ làm chết men rượu, nhiệt độ thích hợp khoảng $20-30^\circ C$.
		\end{itemchoice}
	}
\end{ex}

%%%%%============TF_5================%%%%%%
\begin{ex}%[9K8H9-2]
	Cho mẩu Na dư vào cốc chứa $100$ mL dung dịch ethylic alcohol $46^\circ$.
	\choiceTF
	{Chỉ có phản ứng thế của ethylic alcohol với Na}
	{\True Tổng thể tích khí $H_2$ thu được bao gồm khí sinh ra từ phản ứng của nước và phản ứng của ethylic alcohol}
	{Natri sẽ chìm xuống đáy cốc rồi mới phản ứng}
	{\True Dung dịch sau phản ứng có tính kiềm (base)}
	\loigiai{
		\begin{itemchoice}[F1,T2,F3,T4]
			\itemch Sai. Trong dung dịch cồn $46^\circ$ có nước, nước phản ứng mãnh liệt với Na.
			\itemch Đúng. Cả 2 chất đều phản ứng sinh ra $H_2$.
			\itemch Sai. Na ($D \approx 0,97$) nhẹ hơn nước và nặng hơn cồn một chút nhưng do sủi bọt khí nên thường chạy trên bề mặt chất lỏng.
			\itemch Đúng. Sản phẩm là $NaOH$ và $C_2H_5ONa$ (khi thủy phân tạo $OH^-$) đều có tính kiềm.
		\end{itemchoice}
	}
\end{ex}