%%%%%============BT_1================%%%%%%
\begin{bt}%[9K8H9-4]
	Cho $9{,}2$ gam ethylic alcohol nguyên chất tác dụng hết với một lượng dư kim loại sodium ($Na$).
	\begin{enumerate}
		\item Viết phương trình hóa học của phản ứng xảy ra.
		\item Tính thể tích khí hydrogen thu được (ở điều kiện chuẩn: $25^\circ C$, $1$ bar).
	\end{enumerate}
	\loigiai{
		\begin{enumerate}
			\item Phương trình hóa học:
			\[ 2C_2H_5OH + 2Na \longrightarrow 2C_2H_5ONa + H_2 \]
			\item Số mol của ethylic alcohol là:
			\[ n_{C_2H_5OH} = \frac{9{,}2}{46} = 0{,}2 \text{ (mol)} \]
			Theo phương trình hóa học, số mol khí $H_2$ là:
			\[ n_{H_2} = \frac{1}{2} n_{C_2H_5OH} = \frac{1}{2} \times 0{,}2 = 0{,}1 \text{ (mol)} \]
			Thể tích khí hydrogen thu được ở điều kiện chuẩn là:
			\[ V_{H_2} = 0{,}1 \times 24{,}79 = 2{,}479 \text{ (lít)} \]
		\end{enumerate}
	}
\end{bt}

%%%%%============BT_2================%%%%%%
\begin{bt}%[9K8H9-4]
	Đốt cháy hoàn toàn $13{,}8$ gam ethylic alcohol trong không khí.
	\begin{enumerate}
		\item Tính thể tích khí carbon dioxide ($CO_2$) sinh ra ở điều kiện chuẩn ($25^\circ C$, $1$ bar).
		\item Tính khối lượng nước tạo thành sau phản ứng.
	\end{enumerate}
	\loigiai{
		Số mol của ethylic alcohol là: $n_{C_2H_5OH} = \frac{13{,}8}{46} = 0{,}3$ (mol).
		Phương trình hóa học của phản ứng cháy:
		\[ C_2H_5OH + 3O_2 \xrightarrow[t^\circ][] 2CO_2 + 3H_2O \]
		\begin{enumerate}
			\item Theo phương trình, số mol $CO_2$ là:
			\[ n_{CO_2} = 2 \times n_{C_2H_5OH} = 2 \times 0{,}3 = 0{,}6 \text{ (mol)} \]
			Thể tích khí $CO_2$ thu được là:
			\[ V_{CO_2} = 0{,}6 \times 24{,}79 = 14{,}874 \text{ (lít)} \]
			\item Theo phương trình, số mol $H_2O$ là:
			\[ n_{H_2O} = 3 \times n_{C_2H_5OH} = 3 \times 0{,}3 = 0{,}9 \text{ (mol)} \]
			Khối lượng nước tạo thành là:
			\[ m_{H_2O} = 0{,}9 \times 18 = 16{,}2 \text{ (gam)} \]
		\end{enumerate}
	}
\end{bt}

%%%%%============BT_3================%%%%%%
\begin{bt}%[9K8H9-4]
	Lên men $180$ gam glucose ($C_6H_{12}O_6$) để điều chế ethylic alcohol. Biết hiệu suất của quá trình lên men đạt $80\%$.
	\begin{enumerate}
		\item Viết phương trình hóa học của phản ứng lên men rượu.
		\item Tính khối lượng ethylic alcohol thực tế thu được.
	\end{enumerate}
	\loigiai{
		\begin{enumerate}
			\item Phương trình hóa học:
			\[ C_6H_{12}O_6 \xrightarrow[\text{men rượu}][30-32^\circ C] 2C_2H_5OH + 2CO_2 \]
			\item Số mol glucose ban đầu là: $n_{C_6H_{12}O_6} = \frac{180}{180} = 1$ (mol). \\
			Theo phương trình lí thuyết, số mol ethylic alcohol thu được là:
			\[ n_{C_2H_5OH \text{ (lt)}} = 2 \times n_{C_6H_{12}O_6} = 2 \times 1 = 2 \text{ (mol)} \]
			Khối lượng ethylic alcohol lí thuyết là:
			\[ m_{lt} = 2 \times 46 = 92 \text{ (gam)} \]
			Vì hiệu suất phản ứng đạt $80\%$ nên khối lượng thực tế thu được là:
			\[ m_{tt} = 92 \times \frac{80}{100} = 73{,}6 \text{ (gam)} \]
		\end{enumerate}
	}
\end{bt}

%%%%%============BT_4================%%%%%%
\begin{bt}%[9K8H9-4]
	Đốt cháy hoàn toàn một lượng ethylic alcohol, dẫn toàn bộ sản phẩm cháy qua bình đựng dung dịch nước vôi trong ($Ca(OH)_2$) dư, thu được $20$ gam kết tủa trắng ($CaCO_3$).
	\begin{enumerate}
		\item Tính số mol khí $CO_2$ sinh ra.
		\item Tính khối lượng ethylic alcohol đã đem đốt cháy.
	\end{enumerate}
	\loigiai{
		\begin{enumerate}
			\item Phương trình phản ứng của sản phẩm cháy với nước vôi trong dư:
			\[ CO_2 + Ca(OH)_2 \longrightarrow CaCO_3 \downarrow + H_2O \]
			Số mol kết tủa $CaCO_3$ là: $n_{CaCO_3} = \frac{20}{100} = 0{,}2$ (mol). \\
			Theo phương trình, số mol $CO_2$ bằng số mol kết tủa: $n_{CO_2} = 0{,}2$ mol.
			\item Phương trình phản ứng cháy của ethylic alcohol:
			\[ C_2H_5OH + 3O_2 \xrightarrow[t^\circ][] 2CO_2 + 3H_2O \]
			Theo phương trình cháy, số mol ethylic alcohol là:
			\[ n_{C_2H_5OH} = \frac{1}{2} n_{CO_2} = \frac{1}{2} \times 0{,}2 = 0{,}1 \text{ (mol)} \]
			Khối lượng ethylic alcohol đã đốt cháy là:
			\[ m = 0{,}1 \times 46 = 4{,}6 \text{ (gam)} \]
		\end{enumerate}
	}
\end{bt}

%%%%%============BT_5================%%%%%%
\begin{bt}%[9K8H9-4]
	Cho $10$ mL ethylic alcohol nguyên chất ($D = 0{,}8$ g/mL) tác dụng hết với kim loại Na dư. Tính thể tích khí $H_2$ thu được (ở điều kiện chuẩn $25^\circ C$, $1$ bar).
	\loigiai{
		Khối lượng của ethylic alcohol là:
		\[ m_{C_2H_5OH} = V \times D = 10 \times 0{,}8 = 8 \text{ (gam)} \]
		Số mol ethylic alcohol là:
		\[ n_{C_2H_5OH} = \frac{8}{46} \approx 0{,}174 \text{ (mol)} \]
		Phương trình hóa học:
		\[ 2C_2H_5OH + 2Na \longrightarrow 2C_2H_5ONa + H_2 \]
		Theo phương trình, số mol khí $H_2$ là:
		\[ n_{H_2} = \frac{1}{2} n_{C_2H_5OH} \approx \frac{0{,}174}{2} = 0{,}087 \text{ (mol)} \]
		Thể tích khí $H_2$ thu được là:
		\[ V_{H_2} = 0{,}087 \times 24{,}79 \approx 2{,}157 \text{ (lít)} \]
	}
\end{bt}