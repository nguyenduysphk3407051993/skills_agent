%%%%%============EX_1================%%%%%%
\begin{ex}%[9K8H9-2]
	Khi đốt cháy hoàn toàn một lượng ethylic alcohol ($C_2H_5OH$), số mol khí carbon dioxide ($CO_2$) sinh ra bằng $a$ mol và số mol hơi nước ($H_2O$) sinh ra bằng $b$ mol. Mối quan hệ giữa $a$ và $b$ là
	\choice
	{$a = b$}
	{$a > b$}
	{\True $a < b$}
	{$a = 2b$}
	\loigiai{
		Phương trình hóa học:
		\[ C_2H_5OH + 3O_2 \xrightarrow[t^\circ][] 2CO_2 + 3H_2O \]
		Từ phương trình ta thấy tỉ lệ số mol $n_{CO_2} : n_{H_2O} = 2 : 3$. Do đó $n_{CO_2} < n_{H_2O}$ hay $a < b$.
	}
\end{ex}

%%%%%============EX_2================%%%%%%
\begin{ex}%[9K8H9-2]
	Cho mẩu sodium (Na) dư vào ống nghiệm đựng $10$ mL ethylic alcohol $90^\circ$. Hiện tượng hóa học xảy ra trong ống nghiệm là
	\choice
	{Chỉ có phản ứng của ethylic alcohol với Na, sủi bọt khí $H_2$}
	{Chỉ có phản ứng của nước với Na, sủi bọt khí $H_2$}
	{\True Cả ethylic alcohol và nước đều phản ứng với Na, sủi bọt khí $H_2$}
	{Không có phản ứng nào xảy ra}
	\loigiai{
		Ethylic alcohol $90^\circ$ là hỗn hợp gồm ethylic alcohol và nước. Cả hai chất này đều tác dụng được với Na giải phóng khí hydrogen.
		\[ 2H_2O + 2Na \longrightarrow 2NaOH + H_2 \]
		\[ 2C_2H_5OH + 2Na \longrightarrow 2C_2H_5ONa + H_2 \]
	}
\end{ex}

%%%%%============EX_3================%%%%%%
\begin{ex}%[9K8H9-2]
	Trong phản ứng lên men rượu từ đường glucose, điều kiện cần thiết để phản ứng xảy ra là
	\choice
	{Nhiệt độ rất cao ($> 100^\circ C$)}
	{Có xúc tác acid $H_2SO_4$ đặc}
	{\True Có xúc tác là men rượu (enzyme) và nhiệt độ thích hợp}
	{Có ánh sáng mặt trời chiếu trực tiếp}
	\loigiai{
		Quá trình lên men rượu từ glucose cần có xúc tác là men rượu và nhiệt độ thích hợp (thường khoảng $20 - 30^\circ C$).
		\[ C_6H_{12}O_6 \xrightarrow[\text{men rượu}][30-32^\circ C] 2C_2H_5OH + 2CO_2 \]
	}
\end{ex}

%%%%%============EX_4================%%%%%%
\begin{ex}%[9K8H9-2]
	Ethylic alcohol phản ứng được với kim loại sodium (Na) giải phóng khí hydrogen là do đặc điểm cấu tạo nào sau đây?
	\choice
	{Trong phân tử có chứa nguyên tử carbon}
	{Trong phân tử có chứa nguyên tử oxygen}
	{\True Trong phân tử có nhóm $-OH$}
	{Trong phân tử có nhóm $-CH_3$}
	\loigiai{
		Phản ứng thế với Na đặc trưng cho các hợp chất có nhóm hydroxyl ($-OH$). Nguyên tử H trong nhóm $-OH$ linh động nên bị nguyên tử Na thay thế.
		Hydrocarbon (như $C_2H_6$) không có nhóm này nên không phản ứng với Na.
	}
\end{ex}

%%%%%============EX_5================%%%%%%
\begin{ex}%[9K8H9-2]
	Tại sao ethylic alcohol được sử dụng phổ biến làm nhiên liệu (cồn khô, đèn cồn, phối trộn xăng E5)?
	\choice
	{Vì ethylic alcohol tan vô hạn trong nước}
	{\True Vì phản ứng cháy của ethylic alcohol tỏa nhiều nhiệt và không sinh ra khí độc hại như $SO_2$}
	{Vì ethylic alcohol là chất lỏng không màu}
	{Vì ethylic alcohol có khả năng sát trùng}
	\loigiai{
		Ethylic alcohol dễ cháy, khi cháy tỏa nhiều nhiệt nên được ứng dụng làm nhiên liệu.
		\[ C_2H_5OH + 3O_2 \xrightarrow[t^\circ][] 2CO_2 + 3H_2O \]
	}
\end{ex}