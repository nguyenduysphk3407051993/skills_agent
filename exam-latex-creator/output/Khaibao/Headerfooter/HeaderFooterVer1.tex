%%%======================Thông tin nhóm học tập ===============================%%%
\def\qrcodeytb{\maqr[\maunhan][0.65]{https://www.youtube.com/@hochoakhongkho}}
%\def\qrcodezalo{\maqr[\maunhan][0.65]{https://zalo.me/g/sqemzd726}}%lớp 10
\def\qrcodezalo{\maqr[\maunhan][0.65]{https://zalo.me/g/ogytdc722}}%lớp 11
\def\myytb{\canhdong{\href{https://www.youtube.com/@hochoakhongkho}{\color{\maunhan} \bfseries\sffamily\faAngleDoubleLeft\ \faYoutubePlay\ Học hóa không khó\qrcodeytb \faAngleDoubleRight}}}

\def\tacgia{\canhdong{{\color{\maunhan} \bfseries\sffamily\faAngleDoubleLeft\ Biên soạn Nguyễn Tường Duy \faAngleDoubleRight}}}

%\def\Groupzalo{\canhdong{\color{\maunhan}\small\sffamily\bfseries \faAngleDoubleLeft \qrcodezalo Tham gia group học tập trên zalo\ \faAngleDoubleRight}}
\def\Groupzalo{\canhdong{\color{\maunhan}\small\sffamily\bfseries \faAngleDoubleLeft\  Tài liệu dạy học hóa học 9\ \faAngleDoubleRight}}
\def\ycquetma{\fmmfamily Dùng ~\rotatebox{-15}{\faMobile}~quét mã QR}
\usepackage{fancyhdr}
\usepackage{textcase}
\pagestyle{fancy}
\fancyhead[RE]{\leftmark}
\fancyhead[LE]{}
\fancyhead[LO]{\rightmark}
\fancyhead[RO]{}
\cfoot{}
\fancyhead[LE,RO]{\color{\mycolor!30!black}{\LARGE\fmmfamily\itshape Nên người -- Tiến bộ -- Thành đạt}}
\renewcommand{\headrulewidth}{0pt}
\renewcommand{\chaptermark}[1]{\markboth{\LARGE\fmmfamily\textbf{Chuyên đề \arabic{chapter}.}\ \LARGE\fmmfamily\textbf{#1}}{}%
}
\renewcommand{\sectionmark}[1]{\markright{\LARGE\fmmfamily\textbf{Bài \arabic{section}.}\LARGE\fmmfamily\textbf{#1}}}{}
\usepackage{lastpage}
\renewcommand{\headrule}{\hbox to\headwidth{%
	\color{\mycolor!80!black}\leaders\hrule height \headrulewidth\hfill}}
%%%%%%%%%%%%%%%%%%%%%%%%%%%%%%%%%%%%%%%%%%%%%%%%%%%%%%%%%%%%%%%%%%%%
\usepackage{eso-pic}
\usepackage{qrcode}
\AddToShipoutPicture{%
\hypersetup{urlcolor=\mycolor}
	\ifodd\value{page}
     \begin{tikzpicture}[overlay,remember picture]
     	%%% Đầu trang lẻ
%     	\path[fill=\mycolor] ($(current page.north west)+(2,-1.6)$) coordinate (A) rectangle ++(\linewidth,-0.05) coordinate (AT);
     	%%% Chân trang lẻ
     	\path ($(current page.south west)+(2,1.2)$) coordinate (B) -- ++(\linewidth,0) coordinate (BT) node [midway](ChanelYT){\tacgia};
     	\path (BT) node[anchor=east](ST){\shapepage{\thepage}};
     	\path[draw=\mycolor,line width=1pt](B)--(ChanelYT)--(ST);
%     	\path(ChanelYT) node[below=3pt]{\qrcodeytb};
     	%%%Số trang lẻ
     \end{tikzpicture}
	\else
	 \begin{tikzpicture}[overlay,remember picture]
	 	%%% Đầu trang chẵn
%	 	\path[fill=\mycolor] ($(current page.north west)+(1,-1.6)$) coordinate (A) rectangle ++(\linewidth,-0.05) coordinate (AT);
	 	%%% Chân trang chẵn
	 	\path ($(current page.south west)+(1,1.2)$) coordinate (BH) --++(\linewidth,0) coordinate (BTH) node [midway](GroupMXH){\Groupzalo};
	 	\path (BH)node[anchor=west](STH){\shapepageH{\thepage}};
	 	\path[draw=\mycolor,line width=1pt](STH)--(GroupMXH)--(BTH);
%	 	\path(GroupMXH) node[below=3pt]{\qrcodezalo};
	 	%%% Số trang chẵn
	 \end{tikzpicture}
	\fi
}
\raggedbottom