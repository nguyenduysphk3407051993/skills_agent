%%%==========================Lệnh mũi tên cập nhật này 02-02-2024========================%%%


%%%============Lệnh căn dòng khi dùng tikZ==============================%%%
\newcommand{\canhdong}[1]{\tikz[baseline=(char.base)]{
		\node[inner sep =0pt,outer sep=0pt,anchor =base, baseline] (char){
			#1
		};
	}
}
\newcommand{\canhdongH}[1]{\tikz[baseline=(current bounding box.base)]{
		\node [inner sep=0pt,outer sep=0pt]{
			#1
		};
	}
}
%%%==========Các lệnh Kiểu mũi tên của thư viện Arrow.meta==============%%%
\newcommand{\arrowS}{-{Stealth[width=3.65pt,length=4pt]}}
\newcommand{\arrowL}{-{Latex[width=3.65pt,length=4pt]}}
\gdef\arrowTN{-{Stealth[left]}} %%% tùy chọn khi dùng phản ứng thuận nghịch
\newcommand{\explus}{\;+\;}
%%%============Mũi tên chất khí=============%%%
\NewDocumentCommand{\MuiTenU}{O{}O{}O{}O{}}{%
	\ifblank{#1}{\def\tuychonone{3}}{\def\tuychonone{#1}}
	\ifblank{#2}{\def\tuychontwo{-3}}{\def\tuychontwo{#2}}
	\ifblank{#3}{\def\tuychonthree{\arrowS}}{\def\tuychonthree{#3}}
	\ifblank{#4}{\def\tuychonfour{black}}{\def\tuychonfour{#4}}
	\ignorespaces\hspace*{-14pt}\canhdong{\begin{tikzpicture}[declare function={dodai=\tuychonone mm;goc=90;}]
	\draw[->,\tuychonthree,\tuychonfour,line width =.8pt,shorten <=\tuychontwo pt] (0,0)--(goc:dodai);
	\end{tikzpicture}}%
}
%%%============Mũi tên kết tủa=============%%%
\NewDocumentCommand{\MuiTenD}{O{}O{}O{}O{}}{
	\ifblank{#1}{\def\tuychonone{2}}{\def\tuychonone{#1}}
	\ifblank{#2}{\def\tuychontwo{-5.65}}{\def\tuychontwo{#2}}
	\ifblank{#3}{\def\tuychonthree{\arrowS}}{\def\tuychonthree{#3}}
	\ifblank{#4}{\def\tuychonfour{black}}{\def\tuychonfour{#4}}
	\ignorespaces\hspace*{-16pt}\canhdong{\begin{tikzpicture}[declare function={dodai=\tuychonone mm;goc=-90;}]
			\draw[->,\tuychonthree,\tuychonfour,line width =.8pt,shorten <=\tuychontwo pt] (0,0)--(goc:dodai);
	\end{tikzpicture}}%
}%


\newcommand{\Tieudegiua}[2][\maunhan]{
	\begin{center}
		\tikz[baseline=(char.base)]{
			\node[inner sep =6pt,outer sep=6pt,anchor =base, baseline,shape=rectangle,rounded corners=6pt,fill=#1!10,font=\color{#1!80!black}\Huge\bfseries] (char){
				#2
			}; 
		}
	\end{center}
	{\noindent\large\bfseries Họ và tên học sinh:}\makebox[10.5 cm]{\dotfill}\hfill
	{\large\bfseries Lớp:}\makebox[2.5cm]{\dotfill}
}

%%%=========Tạo lệnh tập con  2 cận 6 tùy chọn  có 4 tùy chọn đầu đầu bỏ trống===============%%%%
%%Tùy chọn 1 dấu ngoặc mở  cho cận a với tham số [t] ngoặc tròn, [v] ngoặc vuông
%%tùy chọn 2 dấu ngoặc đóng cho cận b với tham số [t] ngoặc tròn, [v] ngoặc vuông
%%tùy chọn 3 dấu gạch chéo với tham số [X] gạch chéo xuôi, [N] gạch chéo ngược
%%tùy chọn 4 độ dài phần gạch chéo
%%tùy chọn 5 độ dày gạch chéo
%%tùy chọn 6 kích thước  hình vẽ
\NewDocumentCommand{\tapconRH}{O{}O{}O{}O{}O{15}O{5}mm}{%
	% Xử lý cho tùy chọn ngoặc đầu tiên
	\ifblank{#1}{\def\tuychonone{ngoactron}}{%
		\IfEq{#1}{t}{\def\tuychonone{ngoactron}}{}%
		\IfEq{#1}{v}{\def\tuychonone{ngoacvuong}}{}%
	}
	% Xử lý cho tùy chọn ngoặc thứ hai
	\ifblank{#2}{\def\tuychontwo{ngoactron}}{%
		\IfEq{#2}{t}{\def\tuychontwo{ngoactron}}{}%
		\IfEq{#2}{v}{\def\tuychontwo{ngoacvuong}}{}%
	}
	% Xử lý cho tùy chọn dấu gạch chéo
	\ifblank{#3}{\def\tuychonthree{gachcheoN}}{%
		\IfEq{#3}{X}{\def\tuychonthree{gachcheoX}}{}%
		\IfEq{#3}{N}{\def\tuychonthree{gachcheoN}}{}%
	}
	% Xử lý  độ dài phần gạch chéo
	\ifblank{#4}{\def\tuychonfour{2.5}}{\def\tuychonfour{#4}}
	\resizebox{!}{#6 cm}{
		\begin{tikzpicture}[declare function={a=#7;b=#8;d=\tuychonfour;r=0.5cm;},line join=round,line cap=round]
			%%%Tạo pic các dấu ngoặc%%%
			\tikzset{
				ngoactron/.pic={
					\path[pic actions] (0:0) arc (180:140:r) (0:0) arc (180:220:r);
				},
				ngoacvuong/.pic={
					\path[pic actions] (0.15,r*0.6) --(0,r*0.6)--(0,-r*0.6)--(0.15,-r*0.6);
				},
				gachcheoN/.pic={
					\path[pic actions](45:r*0.35)--(0:0)--(-135:r*0.35);
				},
				gachcheoX/.pic={
					\path[pic actions](135:r*0.35)--(0:0)--(-45:r*0.35);
				},
			}
			\draw [-stealth,\maudam,line width=1.1pt,shorten >=-6pt] (a-1.05*d,0)--(b+1.05*d,0);
			% Gán giá trị của a và b vào biến macro mới
			\pgfmathsetmacro{\a}{int(a)}
			\pgfmathsetmacro{\b}{int(b)}
			% Hiển thị các giá trị của a và b
			\path (a,0) node[below= 0.65*r,font=\sffamily\bfseries]{\a};
			\path (b,0) node[below= 0.65*r,font=\sffamily\bfseries]{\b};
			% Hiển thị các dấu ngoặc vuông, ngoặc tròn
			\path (a,0) pic [draw=\maunhan,line width =1.2pt] {\tuychonone};
			\path (b,0) pic [draw=\maunhan,line width =1.2pt,rotate=180] {\tuychontwo};
			%%%Hiển thị các dấu gạch
			\pgfmathsetmacro{\n}{#5}%%thay đổi độ dày dấu gạch
			% Thực hiện vòng lặp theo tùy chọn #1
			\IfEq{#1}{t}{%
				\foreach \i in {0,1,2,...,\n}{%
					\path (a-1/\n*\i*d,0) pic [draw=\maudam,line width =1.1pt] {\tuychonthree};
				}
			}{%
				\foreach \i in {1,2,...,\n}{%
					\path (a-1/\n*\i*d,0) pic [draw=\maudam,line width =1.1pt] {\tuychonthree};
				}
			}
			% Thực hiện vòng lặp theo tùy chọn #2
			\IfEq{#2}{t}{%
				\foreach \i in {0,1,2,...,\n}{%
					\path (b+1/\n*\i*d,0) pic [draw=\maudam,line width =1.1pt] {\tuychonthree};
				}
			}{%
				\foreach \i in {1,2,...,\n}{%
					\path (b+1/\n*\i*d,0) pic [draw=\maudam,line width =1.1pt] {\tuychonthree};
				}
			}
		\end{tikzpicture}
	}
}

%%%=========Tạo lệnh tập con  1 cận 6 tùy chọn  có 4 tùy chọn đầu đầu bỏ trống===============%%%%
%%Tùy chọn 1 dấu ngoặc với tham số [t] ngoặc tròn, [v] ngoặc vuông
%%tùy chọn 2 dấu ngoặc đóng với tham số [0] mở, [180] đóng
%%tùy chọn 3 dấu gạch chéo với tham số [X] gạch chéo xuôi, [N] gạch chéo ngược
%%tùy chọn 4 độ dài phần gạch chéo
%%tùy chọn 5 độ dày gạch chéo
%%tùy chọn 6 kích thước  hình vẽ
\NewDocumentCommand{\tapconRM}{O{}O{}O{}O{}O{15}O{5}m}{%
	% Xử lý cho tùy chọn dấu ngoặc
	\ifblank{#1}{\def\tuychonone{ngoactron}}{%
		\IfEq{#1}{t}{\def\tuychonone{ngoactron}}{}%
		\IfEq{#1}{v}{\def\tuychonone{ngoacvuong}}{}%
	}
	% Xử lý cho tùy chọn mở đóng dấu ngoặc
	\ifblank{#2}{\def\tuychontwo{0}}{\def\tuychontwo{#2}}
	% Xử lý cho tùy chọn dấu gạch chéo
	\ifblank{#3}{\def\tuychonthree{gachcheoN}}{%
		\IfEq{#3}{X}{\def\tuychonthree{gachcheoX}}{}%
		\IfEq{#3}{N}{\def\tuychonthree{gachcheoN}}{}%
	}
	% Xử lý  độ dài phần gạch chéo
	\ifblank{#4}{\def\tuychonfour{2.5}}{\def\tuychonfour{#4}}
	\resizebox{!}{#6 cm}{
		\begin{tikzpicture}[declare function={a=#7;d=\tuychonfour;r=0.5cm;},line join=round,line cap=round]
			%%%Tạo pic các dấu ngoặc%%%
			\tikzset{
				ngoactron/.pic={
					\path[pic actions] (0:0) arc (180:140:r) (0:0) arc (180:220:r);
				},
				ngoacvuong/.pic={
					\path[pic actions] (0.15,r*0.6) --(0,r*0.6)--(0,-r*0.6)--(0.15,-r*0.6);
				},
				gachcheoN/.pic={
					\path[pic actions](45:r*0.35)--(0:0)--(-135:r*0.35);
				},
				gachcheoX/.pic={
					\path[pic actions](135:r*0.35)--(0:0)--(-45:r*0.35);
				},
			}
			\draw[-stealth,\maudam,line width=1.1pt,shorten >=-6pt] (a-1.05*d,0) --+(2.1*d,0);
			% Gán giá trị của a vào biến macro mới
			\pgfmathsetmacro{\a}{int(a)}
			% Hiển thị các giá trị của a 
			\path (a,0) node[below= 0.65*r,font=\sffamily\bfseries]{\a};
			% Hiển thị các dấu ngoặc vuông, ngoặc tròn
			\path (a,0) pic [draw=\maunhan,line width =1.2pt,rotate=\tuychontwo] {\tuychonone};
			%%%Hiển thị các dấu gạch
			\pgfmathsetmacro{\n}{#5}%%thay đổi độ dày dấu gạch
			\IfEq{#1}{t}{%
				\IfEq{\tuychontwo}{0}{%
					\foreach \i in {0,1,2,...,\n}{%
						\path (a-1/\n*\i*d,0) pic [draw=\maudam,line width =1.1pt] {\tuychonthree};
					}
				}{%
					\foreach \i in {0,1,2,...,\n}{%
						\path (a+1/\n*\i*d,0) pic [draw=\maudam,line width =1.1pt] {\tuychonthree};
					}
				}
				
			}{%
				\IfEq{\tuychontwo}{0}{%
					\foreach \i in {1,2,...,\n}{%
						\path (a-1/\n*\i*d,0) pic [draw=\maudam,line width =1.1pt] {\tuychonthree};
					}
				}{%
					\foreach \i in {1,2,...,\n}{%
						\path (a+1/\n*\i*d,0) pic [draw=\maudam,line width =1.1pt] {\tuychonthree};
					}
				}
				
			}
			
		\end{tikzpicture}
	}
}




