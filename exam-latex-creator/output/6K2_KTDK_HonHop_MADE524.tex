\documentclass[FileMain.tex]{subfiles}
\gdef\sophong{Sở GD \& ĐT Gia Lai}
\gdef\truong{Trường THCS \& THPT}
\gdef\monhoc{Khoa học tự nhiên 6}
\gdef\ngaykt{04/02/2026}
\gdef\nh{2025 - 2026}
\gdef\thoigian{45}
\gdef\made{524}
\setcounter{section}{0}
\begin{document}
\section[Kiểm tra định kì - KHTN 6 - Mã đề \made]{Kiểm tra định kì}

%%%==============Phần trắc nghiệm nhiều lựa chọn==============%%%
\subsection{Bài tập trắc nghiệm nhiều lựa chọn}\textit{\large Thí sinh trả lời từ câu 1 đến câu 12. Mỗi câu thí sinh chỉ chọn một phương án}
\Opensolutionfile{ansex}[Ans/LGEX-6K2_HonHop_MADE524]
\Opensolutionfile{ans}[Ans/Ans-6K2_HonHop_MADE524]

%%%%%============EX_1================%%%%%%
\begin{ex}%[6K2NB1-1]
	Chất nào sau đây là chất tinh khiết?
	\choice
	{Nước biển}
	{Không khí}
	{\True Nước cất}
	{Nước khoáng}
	\loigiai{
		Nước cất là chất tinh khiết vì chỉ chứa một loại phân tử $H_2O$, không lẫn chất khác.
	}
\end{ex}

%%%%%============EX_2================%%%%%%
\begin{ex}%[6K2NB1-1]
	Đặc điểm nào sau đây KHÔNG phải của hỗn hợp?
	\choice
	{Gồm hai hay nhiều chất trộn lẫn}
	{Mỗi chất giữ nguyên tính chất của nó}
	{\True Có thành phần và tính chất xác định}
	{Có thể tách riêng các chất bằng phương pháp vật lý}
	\loigiai{
		Chất tinh khiết mới có thành phần và tính chất xác định. Hỗn hợp có thành phần thay đổi tùy theo tỉ lệ các chất trong hỗn hợp.
	}
\end{ex}

%%%%%============EX_3================%%%%%%
\begin{ex}%[6K2TH1-2]
	Hỗn hợp nào sau đây là hỗn hợp đồng nhất?
	\choice
	{Nước lẫn cát}
	{Dầu ăn lẫn nước}
	{\True Nước muối}
	{Nước cam ép có tép}
	\loigiai{
		Nước muối là hỗn hợp đồng nhất vì muối tan hoàn toàn trong nước, không nhìn thấy ranh giới giữa các thành phần.
	}
\end{ex}

%%%%%============EX_4================%%%%%%
\begin{ex}%[6K2TH1-2]
	Dung dịch khác với huyền phù ở điểm nào?
	\choice
	{Dung dịch là hỗn hợp, huyền phù là chất tinh khiết}
	{\True Dung dịch là hỗn hợp đồng nhất, huyền phù là hỗn hợp không đồng nhất}
	{Dung dịch chỉ chứa chất lỏng, huyền phù chứa chất rắn}
	{Dung dịch không có màu, huyền phù có màu}
	\loigiai{
		Dung dịch là hỗn hợp đồng nhất (chất tan phân bố đều trong dung môi), còn huyền phù là hỗn hợp không đồng nhất (chất rắn lơ lửng trong chất lỏng).
	}
\end{ex}

%%%%%============EX_5================%%%%%%
\begin{ex}%[6K2NB2-1]
	Ví dụ nào sau đây là huyền phù?
	\choice
	{Nước đường}
	{\True Nước phù sa}
	{Sữa tươi}
	{Giấm ăn}
	\loigiai{
		Nước phù sa là huyền phù vì chứa các hạt phù sa (chất rắn) lơ lửng trong nước.
	}
\end{ex}

%%%%%============EX_6================%%%%%%
\begin{ex}%[6K2NB2-1]
	Ví dụ nào sau đây là nhũ tương?
	\choice
	{Nước muối}
	{Nước bùn}
	{\True Mayonnaise}
	{Nước chanh}
	\loigiai{
		Mayonnaise là nhũ tương vì gồm dầu ăn (chất lỏng) phân tán trong giấm và trứng (chất lỏng). Hai chất lỏng này không tan vào nhau.
	}
\end{ex}

%%%%%============EX_7================%%%%%%
\begin{ex}%[6K2TH2-2]
	Trong các hỗn hợp sau, hỗn hợp nào là dung dịch?
	\choice
	{Nước bột mì khuấy đều}
	{Dầu giấm}
	{Sữa tươi}
	{\True Rượu pha nước}
	\loigiai{
		Rượu (ethanol) tan hoàn toàn trong nước tạo thành dung dịch đồng nhất.
	}
\end{ex}

%%%%%============EX_8================%%%%%%
\begin{ex}%[6K2NB3-1]
	Phương pháp cô cạn dùng để tách
	\choice
	{chất rắn không tan ra khỏi nước}
	{\True chất rắn tan ra khỏi dung dịch}
	{hai chất lỏng không tan vào nhau}
	{chất lỏng có nhiệt độ sôi thấp}
	\loigiai{
		Phương pháp cô cạn dùng để tách chất rắn tan (không bay hơi) ra khỏi dung dịch bằng cách làm bay hơi dung môi.
	}
\end{ex}

%%%%%============EX_9================%%%%%%
\begin{ex}%[6K2NB3-1]
	Phương pháp chiết dùng để tách
	\choice
	{chất rắn tan ra khỏi dung dịch}
	{chất rắn không tan ra khỏi chất lỏng}
	{\True hai chất lỏng không tan vào nhau}
	{hai chất rắn có kích thước khác nhau}
	\loigiai{
		Phương pháp chiết dùng để tách hai chất lỏng không tan vào nhau dựa vào sự khác nhau về khối lượng riêng.
	}
\end{ex}

%%%%%============EX_10================%%%%%%
\begin{ex}%[6K2TH3-2]
	Để thu được nước cất từ nước máy, người ta sử dụng phương pháp
	\choice
	{lọc}
	{cô cạn}
	{chiết}
	{\True chưng cất}
	\loigiai{
		Chưng cất là phương pháp phù hợp vì nước bay hơi ở $100^\circ$C, khi ngưng tụ thu được nước cất tinh khiết.
	}
\end{ex}

%%%%%============EX_11================%%%%%%
\begin{ex}%[6K2TH3-3]
	Để tách dầu ăn ra khỏi nước, ta dùng phương pháp
	\choice
	{lọc}
	{\True chiết}
	{cô cạn}
	{chưng cất}
	\loigiai{
		Dầu ăn không tan trong nước và nhẹ hơn nước nên nổi lên trên. Dùng phương pháp chiết để tách riêng hai lớp chất lỏng.
	}
\end{ex}

%%%%%============EX_12================%%%%%%
\begin{ex}%[6K2VD3-4]
	Một mẫu nước giếng có lẫn đất cát và muối hòa tan. Để thu được nước tinh khiết từ mẫu nước này, ta cần thực hiện theo thứ tự
	\choice
	{cô cạn rồi lọc}
	{chưng cất rồi lọc}
	{\True lọc rồi chưng cất}
	{chiết rồi cô cạn}
	\loigiai{
		Lọc để tách đất cát (chất rắn không tan), sau đó chưng cất để tách nước tinh khiết ra khỏi muối hòa tan.
	}
\end{ex}

\Closesolutionfile{ans}
\Closesolutionfile{ansex}

%%%==============Phần trắc nghiệm đúng sai==============%%%
\subsection{Trắc nghiệm đúng sai}\textit{\large Thí sinh trả lời từ câu 1 đến câu 4. Trong mỗi ý a), b), c), d) ở mỗi câu thí sinh chọn đúng hoặc sai}
\Opensolutionfile{ansex}[Ans/LGTF-6K2_HonHop_MADE524]
\Opensolutionfile{ansbook}[Ansbook/AnsTF-6K2_HonHop_MADE524]
\Opensolutionfile{ans}[Ans/Tempt-6K2_HonHop_MADE524]
\setcounter{ex}{0}

%%%%%============TF_1================%%%%%%
\begin{ex}%[6K2TH1-2]
	Cho các phát biểu sau về hỗn hợp:
	\choiceTF
	{\True Trong hỗn hợp, mỗi chất vẫn giữ nguyên tính chất của nó}
	{Hỗn hợp luôn có thành phần và tính chất xác định}
	{\True Có thể tách các chất trong hỗn hợp bằng phương pháp vật lý}
	{\True Nước mưa là hỗn hợp vì có hòa tan các khí trong không khí}
	\loigiai{
		\begin{itemchoice}[T1,F2,T3,T4]
			\itemch Trong hỗn hợp, các chất chỉ trộn lẫn với nhau, không xảy ra phản ứng hóa học nên giữ nguyên tính chất
			\itemch Hỗn hợp có thành phần thay đổi tùy theo tỉ lệ các chất trộn vào, chỉ chất tinh khiết mới có thành phần xác định
			\itemch Các phương pháp như lọc, cô cạn, chưng cất, chiết là phương pháp vật lý dùng để tách chất
			\itemch Nước mưa hòa tan $CO_2$, $O_2$, $N_2$ và các chất khác từ không khí nên là hỗn hợp
		\end{itemchoice}
	}
\end{ex}

%%%%%============TF_2================%%%%%%
\begin{ex}%[6K2TH2-2]
	Cho các phát biểu sau về phân loại hỗn hợp:
	\choiceTF
	{Nước đường để lâu sẽ thành huyền phù}
	{\True Huyền phù để lâu sẽ có hiện tượng lắng đọng}
	{\True Nhũ tương là hỗn hợp của hai chất lỏng không tan vào nhau}
	{Tất cả các dung dịch đều trong suốt, không có màu}
	\loigiai{
		\begin{itemchoice}[F1,T2,T3,F4]
			\itemch Đường tan hoàn toàn trong nước tạo thành dung dịch, không thể thành huyền phù
			\itemch Huyền phù là hỗn hợp không bền, các hạt rắn sẽ lắng xuống đáy theo thời gian
			\itemch Nhũ tương gồm các giọt chất lỏng phân tán trong chất lỏng khác mà hai chất không tan vào nhau
			\itemch Dung dịch có thể có màu (như nước trà, dung dịch muối đồng màu xanh)
		\end{itemchoice}
	}
\end{ex}

%%%%%============TF_3================%%%%%%
\begin{ex}%[6K2VD3-3]
	Cho các phát biểu sau về phương pháp tách chất:
	\choiceTF
	{\True Phương pháp lọc có thể tách được cát ra khỏi nước}
	{Phương pháp cô cạn có thể tách được dầu ăn ra khỏi nước}
	{\True Phương pháp chưng cất dựa vào sự khác nhau về nhiệt độ sôi của các chất}
	{Phương pháp chiết chỉ áp dụng được khi hai chất có cùng khối lượng riêng}
	\loigiai{
		\begin{itemchoice}[T1,F2,T3,F4]
			\itemch Cát không tan trong nước, kích thước hạt lớn hơn lỗ giấy lọc nên có thể dùng phương pháp lọc
			\itemch Dầu ăn không tan trong nước, dùng phương pháp chiết chứ không phải cô cạn
			\itemch Chưng cất tách các chất dựa vào sự khác nhau về nhiệt độ sôi
			\itemch Phương pháp chiết áp dụng khi hai chất lỏng không tan vào nhau và có khối lượng riêng khác nhau
		\end{itemchoice}
	}
\end{ex}

%%%%%============TF_4================%%%%%%
\begin{ex}%[6K2VD3-4]
	Cho các phát biểu sau về ứng dụng tách chất trong thực tế:
	\choiceTF
	{\True Sàng gạo để loại bỏ sạn là ứng dụng dựa vào kích thước hạt}
	{Lọc cà phê bằng phin là ứng dụng của phương pháp chưng cất}
	{\True Làm nước mắm bằng cách lọc qua vải là ứng dụng của phương pháp lọc}
	{\True Chưng cất dầu mỏ để thu được xăng, dầu là ứng dụng trong công nghiệp}
	\loigiai{
		\begin{itemchoice}[T1,F2,T3,T4]
			\itemch Sàng gạo tách hạt gạo và sạn dựa vào kích thước hạt khác nhau
			\itemch Lọc cà phê bằng phin là ứng dụng của phương pháp lọc, không phải chưng cất
			\itemch Lọc nước mắm qua vải để tách bã cá, giữ lại phần dung dịch
			\itemch Chưng cất dầu mỏ ở các nhiệt độ khác nhau để thu các sản phẩm như xăng, dầu diesel, dầu hỏa
		\end{itemchoice}
	}
\end{ex}

\Closesolutionfile{ans}
\Closesolutionfile{ansbook}
\Closesolutionfile{ansex}

%%%==============Phần bài tập trả lời ngắn==============%%%
\subsection{Bài tập trả lời ngắn}\textit{\large Thí sinh trả lời từ câu 1 đến câu 4}
\Opensolutionfile{ansex}[Ans/LGSA-6K2_HonHop_MADE524]
\Opensolutionfile{ansexh}[Ans/AnsSA-6K2_HonHop_MADE524]
\setcounter{ex}{0}

%%%%%============SA_1================%%%%%%
\begin{ex}%[6K2TH2-2]
	Hòa tan $25$ g muối ăn vào $475$ g nước. Tính khối lượng (theo đơn vị gam) của dung dịch nước muối thu được.
	\shortans{$500$}
	\loigiai{
		Khối lượng dung dịch:
		\[ m_{\text{dd}} = m_{\text{muối}} + m_{\text{nước}} = 25 + 475 = 500 \text{ g} \]
	}
\end{ex}

%%%%%============SA_2================%%%%%%
\begin{ex}%[6K2VD2-3]
	Hòa tan $30$ g đường vào nước được $200$ g dung dịch nước đường. Tính nồng độ phần trăm (theo đơn vị $\%$) của dung dịch nước đường.
	\shortans{$15$}
	\loigiai{
		Nồng độ phần trăm:
		\[ C\% = \dfrac{m_{\text{đường}}}{m_{\text{dd}}} \times 100\% = \dfrac{30}{200} \times 100\% = 15\% \]
	}
\end{ex}

%%%%%============SA_3================%%%%%%
\begin{ex}%[6K2VD3-3]
	Cô cạn $400$ g dung dịch nước muối có nồng độ $5\%$. Tính khối lượng muối (theo đơn vị gam) thu được sau khi cô cạn hoàn toàn.
	\shortans{$20$}
	\loigiai{
		Khối lượng muối:
		\[ m_{\text{muối}} = \dfrac{C\% \times m_{\text{dd}}}{100\%} = \dfrac{5 \times 400}{100} = 20 \text{ g} \]
	}
\end{ex}

%%%%%============SA_4================%%%%%%
\begin{ex}%[6K2VD3-4]
	Trong không khí, oxygen chiếm khoảng $21\%$ về thể tích. Tính thể tích khí oxygen (theo đơn vị lít) có trong $1000$ lít không khí.
	\shortans{$210$}
	\loigiai{
		Thể tích khí oxygen:
		\[ V_{O_2} = \dfrac{21\% \times 1000}{100\%} = \dfrac{21 \times 1000}{100} = 210 \text{ lít} \]
	}
\end{ex}

\Closesolutionfile{ansexh}
\Closesolutionfile{ansex}

%%%==============Phần bài tập tự luận==============%%%
\subsection{Bài tập tự luận}\textit{\large Thí sinh trả lời từ bài 1 đến bài 3}
\Opensolutionfile{ansbth}[Ans/LGBT-6K2_HonHop_MADE524]
\Opensolutionfile{ansbt}[Ans/AnsBT-6K2_HonHop_MADE524]

%%%%%============BT_1================%%%%%%
\begin{bt}%[6K2TH2-2]
	So sánh dung dịch, huyền phù và nhũ tương theo các tiêu chí: trạng thái các thành phần, tính đồng nhất, độ bền. Cho mỗi loại một ví dụ minh họa.
	\loigiai{
		\begin{center}
		\begin{tabular}{|p{2.5cm}|p{3.5cm}|p{3.5cm}|p{3.5cm}|}
		\hline
		\textbf{Tiêu chí} & \textbf{Dung dịch} & \textbf{Huyền phù} & \textbf{Nhũ tương} \\
		\hline
		Trạng thái các thành phần & Chất tan (rắn, lỏng, khí) tan trong dung môi (lỏng) & Chất rắn lơ lửng trong chất lỏng & Chất lỏng phân tán trong chất lỏng khác \\
		\hline
		Tính đồng nhất & Đồng nhất, không nhìn thấy ranh giới & Không đồng nhất, nhìn thấy hạt rắn & Không đồng nhất, có thể nhìn thấy các giọt \\
		\hline
		Độ bền & Bền, không tách lớp & Không bền, để lâu sẽ lắng đọng & Không bền, để lâu sẽ tách lớp \\
		\hline
		Ví dụ & Nước muối, nước đường & Nước bùn, nước phù sa & Sữa tươi, dầu giấm \\
		\hline
		\end{tabular}
		\end{center}
	}
\end{bt}

%%%%%============BT_2================%%%%%%
\begin{bt}%[6K2VD3-3]
	Nêu nguyên tắc và trình bày các bước thực hiện phương pháp chưng cất. Cho một ví dụ ứng dụng của phương pháp này trong đời sống.
	\loigiai{
		\textbf{Nguyên tắc:}
		\\
		Phương pháp chưng cất dựa trên sự khác nhau về nhiệt độ sôi của các chất trong hỗn hợp. Khi đun nóng, chất có nhiệt độ sôi thấp hơn sẽ bay hơi trước, hơi được làm lạnh và ngưng tụ thành chất lỏng tinh khiết.

		\textbf{Các bước thực hiện:}
		\begin{enumerate}
		\item Cho hỗn hợp cần tách vào bình chưng cất
		\item Đun nóng bình chưng cất, chất có nhiệt độ sôi thấp hơn bay hơi trước
		\item Hơi đi qua ống sinh hàn (ống làm lạnh), ngưng tụ thành chất lỏng
		\item Thu chất lỏng ngưng tụ vào bình hứng
		\end{enumerate}

		\textbf{Ví dụ ứng dụng:}
		\\
		Nấu rượu: Cơm rượu sau khi lên men chứa ethanol (nhiệt độ sôi $78{,}3^\circ$C) và nước (nhiệt độ sôi $100^\circ$C). Khi đun nóng, ethanol bay hơi trước, ngưng tụ lại thành rượu có nồng độ cao hơn.
	}
\end{bt}

%%%%%============BT_3================%%%%%%
\begin{bt}%[6K2VC3-4]
	Một nhà máy sản xuất muối từ nước biển. Nước biển có nồng độ muối trung bình $3{,}5\%$.
	\begin{enumerate}
	\item Trình bày quy trình sản xuất muối từ nước biển.
	\item Tính khối lượng muối (theo kg) thu được khi cô cạn hoàn toàn $2000$ kg nước biển.
	\end{enumerate}
	\loigiai{
		\begin{enumerate}
		\item \textbf{Quy trình sản xuất muối từ nước biển:}
		\begin{itemize}
		\item Bước 1: Dẫn nước biển vào các ruộng muối (ô phơi)
		\item Bước 2: Phơi nắng để nước bay hơi dần, nồng độ muối tăng lên
		\item Bước 3: Khi nước bay hơi gần hết, muối kết tinh lại thành các tinh thể
		\item Bước 4: Thu hoạch muối, rửa sạch và phơi khô
		\item Bước 5: Đóng gói và bảo quản
		\end{itemize}
		\textit{Phương pháp này là phương pháp cô cạn, dựa vào sự bay hơi của nước dưới tác dụng của nhiệt độ và gió.}

		\item \textbf{Tính khối lượng muối thu được:}
		\\
		Khối lượng nước biển: $m_{\text{nước biển}} = 2000$ kg
		\\
		Nồng độ muối trong nước biển: $C\% = 3{,}5\%$
		\\
		Khối lượng muối thu được:
		\[ m_{\text{muối}} = \dfrac{C\% \times m_{\text{nước biển}}}{100\%} = \dfrac{3{,}5 \times 2000}{100} = 70 \text{ kg} \]
		\\
		Vậy khối lượng muối thu được là $70$ kg.
		\end{enumerate}
	}
\end{bt}

\Closesolutionfile{ansbt}
\Closesolutionfile{ansbth}

\begin{center}
 \rule[4pt]{2cm}{1pt}\,\large\bfseries Hết\,\rule[4pt]{2cm}{1pt}
\end{center}
\label{x}
\end{document}
