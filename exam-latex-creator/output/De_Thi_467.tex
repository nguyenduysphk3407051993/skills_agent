\documentclass[FileMain.tex]{subfiles}
\gdef\sophong{Sở GD \& ĐT Gia Lai} 
\gdef\truong{Trường THPT Chi Lăng} 
\gdef\truongh{Trường Mầm non, THCS, THPT Sao Việt} 
\gdef\monhoc{Khoa học tự nhiên 6} 
\gdef\ngaykt{04/02/2026} 
\gdef\nh{2025 - 2026} 
\gdef\thoigian{45 phút}
\gdef\made{467} 
\setcounter{section}{0}
%\tatloigiai
%\hienthiloigiai
%\dongkeloigiai
\begin{document}
\section[Truy bài định kì - Mã đề \made]{Truy bài định kì} 
%\Tieudegiua{Kiểm tra chủ đề Lương thực - Thực phẩm - Mã đề \made}

%%%==============Phần trắc nghiệm nhiều lựa chọn==============%%% 
\subsection{Bài tập trắc nghiệm nhiều lựa chọn}\textit{\large Thí sinh trả lời từ câu 1 đến 12. Mỗi câu thí sinh chỉ chọn một phương án}
\Opensolutionfile{ansex}[Ans/LGEX-LTTP_KHTN6_467_MADE467]
\Opensolutionfile{ans}[Ans/Ans-LTTP_KHTN6_467_MADE467]

%%%============EX_1=============%%%
\begin{ex}%[6K3N4-1]
	Chất khoáng nào sau đây có vai trò quan trọng trong việc tạo nên hệ xương và răng chắc khỏe?
	\choice
	{Kẽm}
	{Sắt}
	{{\True Canxi}}
	{Iốt}
	\loigiai{
		Canxi (Calcium) là thành phần chính cấu tạo nên xương và răng. Thiếu canxi sẽ dẫn đến còi xương, loãng xương.
	}
\end{ex}

%%%============EX_2=============%%%
\begin{ex}%[6K3N4-2]
	Thực phẩm nào sau đây là nguồn cung cấp chính Vitamin A, giúp sáng mắt?
	\choice
	{Gạo trắng}
	{Thịt nạc}
	{{\True Cà rốt, bí đỏ}}
	{Dầu ăn}
	\loigiai{
		Các loại củ quả có màu đỏ, cam, vàng như cà rốt, bí đỏ, gấc chứa nhiều tiền vitamin A (Beta-carotene) tốt cho mắt.
	}
\end{ex}

%%%============EX_3=============%%%
\begin{ex}%[6K3H4-3]
	Để giữ cho rau củ tươi lâu mà không bị mất chất dinh dưỡng, cách tốt nhất là gì?
	\choice
	{{\True Bảo quản trong ngăn mát tủ lạnh ở nhiệt độ thích hợp}}
	{Phơi nắng thật khô}
	{Ngâm trong nước muối mặn}
	{Để ở nơi ẩm ướt, không thoáng khí}
	\loigiai{
		Rau củ chứa nhiều nước, để bảo quản tươi lâu cần nhiệt độ mát để giảm hô hấp và ức chế vi khuẩn nhưng không được để đóng đá (trừ một số loại).
	}
\end{ex}

%%%============EX_4=============%%%
\begin{ex}%[6K3V4-4]
	Hành động nào sau đây dễ gây ngộ độc thực phẩm nhất?
	\choice
	{Rửa tay sạch trước khi ăn}
	{Ăn chín, uống sôi}
	{{\True Ăn tiết canh, gỏi cá sống}}
	{Che đậy thức ăn cẩn thận}
	\loigiai{
		Tiết canh, gỏi cá sống chứa nhiều ký sinh trùng (sán, giun) và vi khuẩn chưa bị tiêu diệt bởi nhiệt độ, nguy cơ gây ngộ độc rất cao.
	}
\end{ex}

%%%============EX_5=============%%%
\begin{ex}%[6K3N4-1]
	Ngũ cốc (như lúa mì, yến mạch) thuộc nhóm thực phẩm nào?
	\choice
	{{\True Lương thực}}
	{Rau xanh}
	{Thực phẩm giàu đạm}
	{Thực phẩm giàu chất béo}
	\loigiai{
		Ngũ cốc chứa hàm lượng tinh bột cao, được xếp vào nhóm lương thực cung cấp năng lượng.
	}
\end{ex}

%%%============EX_6=============%%%
\begin{ex}%[6K3H4-2]
	Tại sao người bị bệnh bướu cổ thường được khuyên dùng muối I-ốt?
	\choice
	{Vì I-ốt giúp xương chắc khỏe}
	{{\True Vì thiếu I-ốt là nguyên nhân chính gây bệnh bướu cổ}}
	{Vì I-ốt cung cấp nhiều năng lượng}
	{Vì I-ốt giúp da dẻ hồng hào}
	\loigiai{
		I-ốt là vi chất cần thiết cho tuyến giáp hoạt động. Thiếu I-ốt dẫn đến phì đại tuyến giáp (bướu cổ) và đần độn.
	}
\end{ex}

%%%============EX_7=============%%%
\begin{ex}%[6K3V4-3]
	Trong quy trình làm sữa chua, yếu tố nào giúp sữa chuyển từ dạng lỏng sang dạng sệt và có vị chua?
	\choice
	{Nhiệt độ sôi của nước}
	{{\True Hoạt động của vi khuẩn Lactic lên men}}
	{Ánh sáng mặt trời}
	{Lượng đường trong sữa}
	\loigiai{
		Vi khuẩn Lactic lên men đường trong sữa thành Acid Lactic, làm protein trong sữa đông tụ lại và tạo vị chua đặc trưng.
	}
\end{ex}

%%%============EX_8=============%%%
\begin{ex}%[6K3N4-4]
	Loại lương thực nào phổ biến nhất ở Châu Âu và Châu Mỹ?
	\choice
	{Lúa gạo}
	{{\True Lúa mì}}
	{Sắn}
	{Khoai lang}
	\loigiai{
		Người phương Tây sử dụng Lúa mì (làm bánh mì, mì ống\dots) làm lương thực chính thay vì lúa gạo như người Châu Á.
	}
\end{ex}

%%%============EX_9=============%%%
\begin{ex}%[6K3H4-1]
	Vitamin D có nhiều trong thực phẩm nào và có vai trò gì?
	\choice
	{Trong rau xanh, giúp sáng mắt}
	{Trong cam chanh, tăng đề kháng}
	{{\True Trong dầu gan cá, lòng đỏ trứng; giúp hấp thụ Canxi}}
	{Trong gạo, cung cấp năng lượng}
	\loigiai{
		Vitamin D giúp cơ thể hấp thụ Canxi vào xương. Nó có nhiều trong dầu cá, trứng, sữa và được cơ thể tổng hợp dưới ánh nắng mặt trời.
	}
\end{ex}

%%%============EX_10=============%%%
\begin{ex}%[6K3V4-2]
	Khi luộc rau, để giữ được vitamin nhiều nhất, ta nên làm gì?
	\choice
	{Cho rau vào khi nước còn lạnh}
	{Nấu thật lâu cho rau mềm nhũn}
	{{\True Đun nước thật sôi rồi mới cho rau vào, nấu vừa chín tới}}
	{Mở vung nồi thật lớn trong suốt quá trình nấu}
	\loigiai{
		Cho rau vào nước sôi giúp vitamin ít bị phá hủy bởi nhiệt độ trong thời gian dài (enzyme bị phân hủy nhanh). Nấu lâu làm mất vitamin C và vitamin nhóm B.
	}
\end{ex}

%%%============EX_11=============%%%
\begin{ex}%[6K3N4-3]
	Phương pháp bảo quản nào thường dùng cho các loại mứt tết?
	\choice
	{Đông lạnh}
	{{\True Ướp đường (cô đặc với đường)}}
	{Muối chua}
	{Ngâm giấm}
	\loigiai{
		Mứt tết được bảo quản bằng cách sên với đường nồng độ cao (áp suất thẩm thấu cao) làm vi khuẩn không phát triển được.
	}
\end{ex}

%%%============EX_12=============%%%
\begin{ex}%[6K3H4-4]
	Biểu hiện nào sau đây **không** phải là dấu hiệu của ngộ độc thực phẩm?
	\choice
	{Đau bụng dữ dội}
	{Nôn mửa liên tục}
	{Tiêu chảy}
	{{\True Cảm thấy sảng khoái, hưng phấn}}
	\loigiai{
		Ngộ độc thực phẩm luôn gây khó chịu cho cơ thể (đau bụng, nôn, sốt, tiêu chảy), không bao giờ gây sảng khoái.
	}
\end{ex}

\Closesolutionfile{ans}
\Closesolutionfile{ansex}
%\bangdapan{Ans-LTTP_KHTN6_467_MADE467}

%%%==============Phần trắc nghiệm đúng sai==============%%% 
\subsection{Trắc nghiệm đúng sai}\textit{\large Thí sinh trả lời từ câu 1 đến câu 4. Trong mỗi ý a), b), c), d) ở mỗi câu thí sinh chọn đúng hoặc sai}
\Opensolutionfile{ansex}[Ans/LGTF-LTTP_KHTN6_467_MADE467]
\Opensolutionfile{ansbook}[Ansbook/AnsTF-LTTP_KHTN6_467_MADE467]
\Opensolutionfile{ans}[Ans/Tempt-LTTP_KHTN6_467_MADE467]
\setcounter{ex}{0}

%%%=============TF_1=============%%%
\begin{ex}%[6K3H4-1]
	Đánh giá các phát biểu về thực phẩm và sức khỏe:
	\choiceTF
	{\True Chất xơ trong rau xanh không cung cấp năng lượng nhưng giúp hệ tiêu hóa hoạt động tốt.}
	{Chỉ cần ăn thịt cá là đủ, không cần ăn rau vì trong thịt cũng có vitamin.}
	{\True Thiếu sắt (Iron) có thể dẫn đến bệnh thiếu máu, da xanh xao.}
	{Ăn nhiều đồ ngọt (bánh kẹo, nước ngọt) giúp trí não thông minh hơn.}
	\loigiai{
		\begin{itemchoice}[T1,F2,T3,F4]
			\itemch Chất xơ giúp nhuận tràng, cuốn trôi chất thải.
			\itemch Thịt ít chất xơ và thiếu một số vitamin quan trọng như C.
			\itemch Sắt là thành phần cấu tạo Hemoglobin trong máu.
			\itemch Ăn nhiều đường gây béo phì, sâu răng và không tốt cho trí não về lâu dài.
		\end{itemchoice}
	}
\end{ex}

%%%=============TF_2=============%%%
\begin{ex}%[6K3V4-2]
	Về việc lựa chọn thực phẩm an toàn:
	\choiceTF
	{\True Nên chọn thịt có màu hồng tươi, ấn vào có độ đàn hồi tốt.}
	{Cá có mắt đục, mang thâm đen là cá còn rất tươi ngon.}
	{\True Không mua các loại đồ hộp bị phồng, gỉ sét hoặc móp méo.}
	{Rau củ càng bóng mượt, to bất thường thì càng tốt và an toàn.}
	\loigiai{
		\begin{itemchoice}[T1,F2,T3,F4]
			\itemch Đó là dấu hiệu thịt tươi.
			\itemch Đó là dấu hiệu cá ươn. Cá tươi mắt trong, mang đỏ.
			\itemch Hộp phồng là do vi khuẩn sinh khí bên trong, rất nguy hiểm (có thể chứa độc tố Botulinum).
			\itemch Rau quá bóng mượt, to bất thường có thể do dùng thuốc kích thích tăng trưởng.
		\end{itemchoice}
	}
\end{ex}

%%%=============TF_3=============%%%
\begin{ex}%[6K3N4-3]
	Các biện pháp xử lý khi bị ngộ độc thực phẩm nhẹ tại nhà:
	\choiceTF
	{\True Gây nôn (nếu người bệnh còn tỉnh táo) để loại bỏ thức ăn độc ra ngoài.}
	{Uống thuốc cầm tiêu chảy ngay lập tức để không bị mất nước.}
	{\True Bù nước và điện giải bằng dung dịch Oresol hoặc nước cháo muối.}
	{Cho người bệnh ăn thật no để lại sức.}
	\loigiai{
		\begin{itemchoice}[T1,F2,T3,F4]
			\itemch Loại bỏ chất độc càng sớm càng tốt.
			\itemch Tiêu chảy là phản ứng thải độc, cầm ngay sẽ giữ độc tố lại trong ruột. Chỉ uống khi có chỉ định bác sĩ.
			\itemch Quan trọng nhất là chống mất nước.
			\itemch Hệ tiêu hóa đang bị tổn thương, cần ăn lỏng, nhẹ hoặc nhịn ăn tạm thời.
		\end{itemchoice}
	}
\end{ex}

%%%=============TF_4=============%%%
\begin{ex}%[6K3H4-4]
	Về thói quen ăn uống lành mạnh:
	\choiceTF
	{Nên ăn mặn (nhiều muối) để cơ thể cứng cáp hơn.}
	{\True Uống đủ nước (khoảng 1,5 - 2 lít) mỗi ngày giúp cơ thể trao đổi chất tốt.}
	{\True Hạn chế ăn nội tạng động vật vì chứa nhiều cholesterol.}
	{Bữa sáng là bữa phụ, có thể bỏ qua để giảm cân.}
	\loigiai{
		\begin{itemchoice}[F1,T2,T3,F4]
			\itemch Ăn mặn hại thận và gây cao huyết áp.
			\itemch Nước cần thiết cho mọi hoạt động sống.
			\itemch Nội tạng chứa nhiều chất béo xấu và cholesterol.
			\itemch Bữa sáng là bữa quan trọng nhất cung cấp năng lượng cho cả ngày. Bỏ bữa sáng làm cơ thể mệt mỏi và dễ gây đau dạ dày.
		\end{itemchoice}
	}
\end{ex}

\Closesolutionfile{ans}
\Closesolutionfile{ansbook}
\Closesolutionfile{ansex}
%\bangdapanTF{AnsTF-LTTP_KHTN6_467_MADE467}

%%==============Phần bài tập trả lời ngắn==============%%% 
\subsection{Bài tập trả lời ngắn}\textit{\large Thí sinh trả lời từ câu 1 đến câu 4}
\Opensolutionfile{ansex}[Ans/LGSA-LTTP_KHTN6_467_MADE467]
\Opensolutionfile{ansexh}[Ans/AnsSA-LTTP_KHTN6_467_MADE467]
\setcounter{ex}{0}

%%%=============SA_1=============%%%
\begin{ex}%[6K3V4-1]
	Nước chiếm khoảng bao nhiêu phần trăm khối lượng cơ thể người trưởng thành? (Ghi một con số trung bình, ví dụ 70).
	\shortans{$70$}
	\loigiai{
		Nước chiếm khoảng 60-70% trọng lượng cơ thể người trưởng thành.
	}
\end{ex}

%%%=============SA_2=============%%%
\begin{ex}%[6K3V4-2]
	Chỉ số BMI (Body Mass Index) được dùng để đánh giá mức độ gầy hay béo của cơ thể. Công thức tính là: BMI = (Cân nặng) chia cho (Chiều cao x Chiều cao). Một bạn học sinh nặng 45 kg và cao 1,5 mét. Chỉ số BMI của bạn ấy là bao nhiêu?
	\shortans{$20$}
	\loigiai{
		$BMI = \dfrac{45}{1{,}5 \times 1{,}5} = \dfrac{45}{2{,}25} = 20$.
	}
\end{ex}

%%%=============SA_3=============%%%
\begin{ex}%[6K3H4-3]
	Có mấy bước cơ bản trong quy trình rửa tay thường quy của Bộ Y tế để phòng chống dịch bệnh và đảm bảo an toàn thực phẩm?
	\shortans{$6$}
	\loigiai{
		Quy trình rửa tay thường quy gồm 6 bước.
	}
\end{ex}

%%%=============SA_4=============%%%
\begin{ex}%[6K3N4-4]
	Trong 1 gam protein (chất đạm) cung cấp cho cơ thể khoảng bao nhiêu kilocalo (kcal) năng lượng khi bị oxy hóa hoàn toàn? (Nhập số nguyên).
	\shortans{$4$}
	\loigiai{
		1 gam Protein cung cấp khoảng 4 kcal (tương đương với Carbohydrate).
	}
\end{ex}

\Closesolutionfile{ansexh}
\Closesolutionfile{ansex}
%\bangdapanSA{AnsSA-LTTP_KHTN6_467_MADE467}

%%%==============Phần bài tập tự luận==============%%% 
\subsection{Bài tập tự luận}\textit{\large Thí sinh trả lời từ bài 1 đến bài 3}
\Opensolutionfile{ansbth}[Ans/LGBT-LTTP_KHTN6_467_MADE467]
\Opensolutionfile{ansbt}[Ans/AnsBT-LTTP_KHTN6_467_MADE467]

%%%=============BT_1=============%%%
\begin{bt}%[6K3H4-1]
	Thế nào là một bữa ăn cân đối và hợp lý? Hãy kể tên 4 nhóm thực phẩm chính cần có trong một bữa ăn.
	\loigiai{
		1. **Bữa ăn cân đối và hợp lý** là bữa ăn cung cấp đủ năng lượng và đầy đủ các nhóm chất dinh dưỡng theo tỷ lệ thích hợp với nhu cầu của cơ thể.
		2. **4 nhóm thực phẩm chính:**
		   - Nhóm giàu chất bột đường (Carbohydrate): Gạo, ngô, khoai\dots
		   - Nhóm giàu chất đạm (Protein): Thịt, cá, trứng, đậu\dots
		   - Nhóm giàu chất béo (Lipid): Mỡ, dầu ăn, bơ\dots
		   - Nhóm giàu Vitamin và khoáng chất: Rau xanh, hoa quả tươi.
	}
\end{bt}

%%%=============BT_2=============%%%
\begin{bt}%[6K3V4-2]
	Tại sao chúng ta nên hạn chế sử dụng túi nilon và hộp xốp để đựng thức ăn nóng? Theo em nên thay thế bằng vật dụng gì để an toàn và bảo vệ môi trường?
	\loigiai{
		**Lý do hạn chế:**
		- Túi nilon và hộp xốp khi gặp nhiệt độ cao (thức ăn nóng > 70 độ C) có thể thôi nhiễm ra các chất độc hại (như monostyren, DOP) ngấm vào thức ăn, gây hại cho gan, thận và có thể gây ung thư.
		- Chúng khó phân hủy, gây ô nhiễm môi trường nghiêm trọng.
		**Vật dụng thay thế:**
		- Nên dùng bát đĩa sứ, thủy tinh.
		- Dùng cặp lồng inox, hộp thủy tinh chịu nhiệt để đựng thức ăn nóng.
		- Dùng lá chuối, lá sen để gói thực phẩm (truyền thống và an toàn).
	}
\end{bt}

%%%=============BT_3=============%%%
\begin{bt}%[6K3C4-3]
	Gạo lứt (gạo còn nguyên vỏ cám) được khuyên dùng cho người muốn giảm cân hoặc người bị tiểu đường thay vì gạo trắng xát kỹ. Hãy giải thích tại sao dựa trên thành phần dinh dưỡng của lớp vỏ cám.
	\loigiai{
		**Giải thích:**
		- Lớp vỏ cám của gạo lứt chứa rất nhiều **Vitamin nhóm B (đặc biệt là B1)** và **Chất xơ**.
		- Chất xơ giúp làm chậm quá trình tiêu hóa và hấp thu đường vào máu, giúp đường huyết không tăng vọt sau khi ăn (tốt cho người tiểu đường).
		- Chất xơ cũng tạo cảm giác no lâu, giúp người ăn giảm bớt lượng thức ăn nạp vào, hỗ trợ giảm cân hiệu quả.
		- Gạo trắng xát kỹ đã mất đi lớp cám này, chủ yếu chỉ còn lõi tinh bột, hấp thu rất nhanh gây tăng đường huyết và tích mỡ.
	}
\end{bt}

\Closesolutionfile{ansbt}
\Closesolutionfile{ansbth}

\begin{center}
 \rule[4pt]{2cm}{1pt}\,\large\bfseries Hết\,\rule[4pt]{2cm}{1pt}
\end{center}
\label{x}
\end{document}
