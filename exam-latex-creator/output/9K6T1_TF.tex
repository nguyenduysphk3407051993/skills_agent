%%%=============TF_1=============%%%
\begin{ex}%[9K6T1-1]
	Lương thực là nguồn cung cấp năng lượng chính cho cơ thể con người. Gạo, ngô, khoai, sắn là những loại lương thực phổ biến ở Việt Nam. Đánh giá tính đúng/sai của các phát biểu sau:
	\choiceTF
	{\True Thành phần chính của gạo là tinh bột (carbohydrate)}
	{\True Tinh bột khi thủy phân hoàn toàn tạo ra glucose}
	{Protein là thành phần chính chiếm tỉ lệ cao nhất trong gạo}
	{\True Gạo cung cấp năng lượng cho cơ thể dưới dạng carbohydrate}
	\loigiai{
		\begin{itemchoice}[T1,T2,F3,T4]
			\itemch Gạo chứa khoảng $75-80\%$ tinh bột, đây là thành phần chính của gạo. Phát biểu đúng.
			\itemch Tinh bột ${(C_6H_{10}O_5)}_n$ khi thủy phân hoàn toàn tạo ra glucose $C_6H_{12}O_6$: ${(C_6H_{10}O_5)}_n + nH_2O \xrightarrow[]{H^+, t^\circ} nC_6H_{12}O_6$. Phát biểu đúng.
			\itemch Protein chỉ chiếm khoảng $7-8\%$ trong gạo, không phải thành phần chính. Thành phần chính là tinh bột. Phát biểu sai.
			\itemch Carbohydrate (tinh bột) trong gạo được cơ thể chuyển hóa thành năng lượng thông qua quá trình hô hấp tế bào. Phát biểu đúng.
		\end{itemchoice}
	}
\end{ex}

%%%=============TF_2=============%%%
\begin{ex}%[9K6T1-2]
	Thực phẩm cung cấp các chất dinh dưỡng cần thiết cho cơ thể như protein, lipid, vitamin và khoáng chất. Đánh giá tính đúng/sai của các phát biểu sau về thực phẩm giàu protein:
	\choiceTF
	{Protein chỉ có trong thực phẩm có nguồn gốc động vật}
	{\True Trứng, thịt, cá là những thực phẩm giàu protein}
	{\True Đậu nành là thực phẩm thực vật giàu protein}
	{Protein khi thủy phân hoàn toàn tạo ra glucose}
	\loigiai{
		\begin{itemchoice}[F1,T2,T3,F4]
			\itemch Protein có cả trong thực phẩm động vật (thịt, cá, trứng) và thực vật (đậu nành, đậu xanh). Phát biểu sai.
			\itemch Trứng chứa khoảng $12-13\%$ protein, thịt chứa $18-22\%$ protein, cá chứa $15-20\%$ protein. Đây đều là thực phẩm giàu protein. Phát biểu đúng.
			\itemch Đậu nành chứa khoảng $35-40\%$ protein, là nguồn protein thực vật quan trọng. Phát biểu đúng.
			\itemch Protein khi thủy phân hoàn toàn tạo ra các amino acid, không phải glucose. Glucose là sản phẩm thủy phân của tinh bột. Phát biểu sai.
		\end{itemchoice}
	}
\end{ex}

%%%=============TF_3=============%%%
\begin{ex}%[9K6T1-3]
	Chất béo (lipid) là một trong những chất dinh dưỡng quan trọng có trong thực phẩm. Đánh giá tính đúng/sai của các phát biểu sau:
	\choiceTF
	{\True Chất béo cung cấp năng lượng cao hơn carbohydrate khi cùng khối lượng}
	{\True Dầu thực vật chứa nhiều acid béo không no}
	{\True Mỡ động vật chứa nhiều acid béo no}
	{Chất béo không no có nhiệt độ nóng chảy cao hơn chất béo no}
	\loigiai{
		\begin{itemchoice}[T1,T2,T3,F4]
			\itemch $1$ gam chất béo cung cấp khoảng $9$ kcal, trong khi $1$ gam carbohydrate chỉ cung cấp khoảng $4$ kcal. Phát biểu đúng.
			\itemch Dầu thực vật (dầu đậu nành, dầu olive) chứa nhiều acid béo không no như oleic, linoleic. Phát biểu đúng.
			\itemch Mỡ lợn, mỡ bò chứa nhiều acid béo no như palmitic, stearic. Phát biểu đúng.
			\itemch Chất béo không no có nhiệt độ nóng chảy thấp hơn chất béo no do có liên kết đôi $C=C$ làm giảm lực Van der Waals giữa các phân tử. Phát biểu sai.
		\end{itemchoice}
	}
\end{ex}

%%%=============TF_4=============%%%
\begin{ex}%[9K6T1-4]
	Vitamin và khoáng chất là những vi chất dinh dưỡng cần thiết cho cơ thể. Đánh giá tính đúng/sai của các phát biểu sau:
	\choiceTF
	{\True Vitamin C có nhiều trong các loại trái cây họ cam, quýt}
	{Vitamin A tan trong nước nên dễ bị mất khi nấu chín thực phẩm}
	{\True Rau xanh là nguồn cung cấp chất xơ và vitamin cho cơ thể}
	{\True Sữa là thực phẩm giàu calcium giúp xương chắc khỏe}
	\loigiai{
		\begin{itemchoice}[T1,F2,T3,T4]
			\itemch Cam, quýt, bưởi, chanh chứa hàm lượng vitamin C cao ($30-50$ mg/$100$ g). Phát biểu đúng.
			\itemch Vitamin A là vitamin tan trong dầu/chất béo, không tan trong nước. Vitamin tan trong nước là vitamin B, C. Phát biểu sai.
			\itemch Rau xanh chứa nhiều chất xơ (cellulose) và các vitamin như A, C, K, acid folic. Phát biểu đúng.
			\itemch Sữa chứa khoảng $120$ mg calcium/$100$ ml, là nguồn calcium quan trọng giúp phát triển xương và răng. Phát biểu đúng.
		\end{itemchoice}
	}
\end{ex}

%%%=============TF_5=============%%%
\begin{ex}%[9K6T1-5]
	Bảo quản thực phẩm đúng cách giúp giữ được chất dinh dưỡng và đảm bảo an toàn vệ sinh thực phẩm. Đánh giá tính đúng/sai của các phát biểu sau:
	\choiceTF
	{\True Bảo quản lạnh giúp làm chậm sự phát triển của vi sinh vật gây hại}
	{\True Muối ăn có tác dụng bảo quản thực phẩm do làm mất nước của vi khuẩn}
	{Thực phẩm đóng hộp không cần bảo quản lạnh vì đã được tiệt trùng hoàn toàn}
	{\True Phơi khô là phương pháp bảo quản thực phẩm truyền thống hiệu quả}
	\loigiai{
		\begin{itemchoice}[T1,T2,F3,T4]
			\itemch Ở nhiệt độ thấp ($0-4^\circ$C), các phản ứng sinh hóa và sự phát triển của vi sinh vật bị ức chế, giúp bảo quản thực phẩm lâu hơn. Phát biểu đúng.
			\itemch Muối ($NaCl$) có tính thẩm thấu cao, hút nước từ tế bào vi khuẩn ra ngoài (hiện tượng co nguyên sinh), làm vi khuẩn chết hoặc không phát triển được. Phát biểu đúng.
			\itemch Thực phẩm đóng hộp sau khi mở nắp vẫn cần bảo quản lạnh và sử dụng trong thời gian ngắn vì vi sinh vật có thể xâm nhập. Phát biểu sai.
			\itemch Phơi khô làm giảm độ ẩm của thực phẩm xuống dưới $15\%$, ngăn vi sinh vật phát triển. Đây là phương pháp bảo quản truyền thống hiệu quả. Phát biểu đúng.
		\end{itemchoice}
	}
\end{ex}
