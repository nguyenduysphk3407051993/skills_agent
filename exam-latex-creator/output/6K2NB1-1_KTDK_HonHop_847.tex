\documentclass[FileMain.tex]{subfiles}
\gdef\sophong{Sở GD \& ĐT Gia Lai}
\gdef\truong{Trường THCS \& THPT}
\gdef\monhoc{Khoa học tự nhiên 6}
\gdef\ngaykt{04/02/2026}
\gdef\nh{2025 - 2026}
\gdef\thoigian{45}
\gdef\made{847}
\setcounter{section}{0}
%\tatloigiai
%\hienthiloigiai
%\dongkeloigiai
\begin{document}
\section[Kiểm tra định kì - KHTN 6 - Mã đề \made]{Kiểm tra định kì}

%%%==============Phần trắc nghiệm nhiều lựa chọn==============%%%
\subsection{Bài tập trắc nghiệm nhiều lựa chọn}\textit{\large Thí sinh trả lời từ câu 1 đến câu 18. Mỗi câu thí sinh chỉ chọn một phương án}
\Opensolutionfile{ansex}[Ans/LGEX-6K2_HonHop_MADE847]
\Opensolutionfile{ans}[Ans/Ans-6K2_HonHop_MADE847]

%%%%%============EX_1================%%%%%%
\begin{ex}%[6K2NB1-1]
	Hỗn hợp là gì?
	\choice
	{Chất được tạo nên từ một nguyên tố hóa học}
	{\True Hai hay nhiều chất trộn lẫn với nhau}
	{Chất có thành phần và tính chất xác định}
	{Chất được tạo nên từ một loại phân tử}
	\loigiai{
		Hỗn hợp là hai hay nhiều chất trộn lẫn với nhau. Trong hỗn hợp, mỗi chất vẫn giữ nguyên tính chất của nó.
	}
\end{ex}

%%%%%============EX_2================%%%%%%
\begin{ex}%[6K2NB1-1]
	Chất tinh khiết là chất như thế nào?
	\choice
	{Chất được tạo nên từ nhiều loại phân tử}
	{Chất có nhiều tạp chất}
	{\True Chất không lẫn chất khác, có tính chất nhất định}
	{Chất được trộn lẫn từ nhiều chất}
	\loigiai{
		Chất tinh khiết là chất không lẫn chất khác, có thành phần và tính chất xác định. Ví dụ: nước cất, muối ăn tinh khiết, đường kính.
	}
\end{ex}

%%%%%============EX_3================%%%%%%
\begin{ex}%[6K2NB1-2]
	Trong các trường hợp sau, trường hợp nào là hỗn hợp?
	\choice
	{Nước cất}
	{Khí oxygen nguyên chất}
	{\True Không khí}
	{Vàng nguyên chất}
	\loigiai{
		Không khí là hỗn hợp gồm nhiều chất khí như: nitrogen (khoảng $78\%$), oxygen (khoảng $21\%$), carbon dioxide, hơi nước và các khí khác.
	}
\end{ex}

%%%%%============EX_4================%%%%%%
\begin{ex}%[6K2TH1-2]
	Hỗn hợp đồng nhất là hỗn hợp có đặc điểm nào sau đây?
	\choice
	{Nhìn thấy rõ ranh giới giữa các thành phần}
	{\True Không nhìn thấy ranh giới giữa các thành phần}
	{Các chất không tan vào nhau}
	{Các hạt chất lơ lửng trong hỗn hợp}
	\loigiai{
		Hỗn hợp đồng nhất là hỗn hợp không nhìn thấy ranh giới giữa các thành phần, các chất phân bố đều trong toàn bộ hỗn hợp. Ví dụ: nước muối, nước đường.
	}
\end{ex}

%%%%%============EX_5================%%%%%%
\begin{ex}%[6K2TH1-2]
	Hỗn hợp nào sau đây là hỗn hợp không đồng nhất?
	\choice
	{Nước muối}
	{Nước đường}
	{Giấm ăn}
	{\True Nước cam có tép}
	\loigiai{
		Nước cam có tép là hỗn hợp không đồng nhất vì có thể nhìn thấy rõ các tép cam (thành phần rắn) trong phần nước cam (thành phần lỏng).
	}
\end{ex}

%%%%%============EX_6================%%%%%%
\begin{ex}%[6K2TH1-2]
	Dung dịch là loại hỗn hợp nào?
	\choice
	{\True Hỗn hợp đồng nhất}
	{Hỗn hợp không đồng nhất}
	{Huyền phù}
	{Nhũ tương}
	\loigiai{
		Dung dịch là hỗn hợp đồng nhất của chất tan và dung môi. Trong dung dịch, các phân tử chất tan phân bố đều trong dung môi và không nhìn thấy ranh giới phân chia.
	}
\end{ex}

%%%%%============EX_7================%%%%%%
\begin{ex}%[6K2NB2-1]
	Huyền phù là hỗn hợp gồm các thành phần nào?
	\choice
	{Chất lỏng tan trong chất lỏng}
	{Chất khí tan trong chất lỏng}
	{\True Chất rắn lơ lửng trong chất lỏng}
	{Chất lỏng tan trong chất rắn}
	\loigiai{
		Huyền phù là hỗn hợp không đồng nhất gồm các hạt chất rắn lơ lửng trong chất lỏng. Ví dụ: nước phù sa, nước bột sắn khuấy đều.
	}
\end{ex}

%%%%%============EX_8================%%%%%%
\begin{ex}%[6K2NB2-1]
	Nhũ tương là hỗn hợp gồm các thành phần nào?
	\choice
	{Chất rắn phân tán trong chất lỏng}
	{\True Chất lỏng phân tán trong chất lỏng khác}
	{Chất khí phân tán trong chất lỏng}
	{Chất rắn phân tán trong chất rắn}
	\loigiai{
		Nhũ tương là hỗn hợp không đồng nhất gồm các giọt chất lỏng phân tán trong chất lỏng khác (hai chất lỏng không tan vào nhau). Ví dụ: sữa, dầu giấm, mayonnaise.
	}
\end{ex}

%%%%%============EX_9================%%%%%%
\begin{ex}%[6K2TH2-2]
	Trong các hỗn hợp sau, đâu là huyền phù?
	\choice
	{Nước muối}
	{Sữa tươi}
	{\True Nước bùn}
	{Nước chanh đường}
	\loigiai{
		Nước bùn là huyền phù vì gồm các hạt bùn (chất rắn) lơ lửng trong nước (chất lỏng). Nước muối và nước chanh đường là dung dịch, sữa tươi là nhũ tương.
	}
\end{ex}

%%%%%============EX_10================%%%%%%
\begin{ex}%[6K2TH2-2]
	Sữa tươi thuộc loại hỗn hợp nào?
	\choice
	{Dung dịch}
	{Huyền phù}
	{\True Nhũ tương}
	{Chất tinh khiết}
	\loigiai{
		Sữa tươi là nhũ tương vì gồm các giọt chất béo (chất lỏng) phân tán trong nước. Đây là hỗn hợp không đồng nhất của hai chất lỏng không tan vào nhau.
	}
\end{ex}

%%%%%============EX_11================%%%%%%
\begin{ex}%[6K2NB3-1]
	Phương pháp tách chất nào sau đây dựa trên sự khác nhau về kích thước hạt?
	\choice
	{Cô cạn}
	{Chiết}
	{\True Lọc}
	{Chưng cất}
	\loigiai{
		Phương pháp lọc dựa trên sự khác nhau về kích thước hạt. Chất có kích thước hạt lớn hơn lỗ lọc sẽ bị giữ lại trên giấy lọc, chất có kích thước hạt nhỏ hơn sẽ đi qua.
	}
\end{ex}

%%%%%============EX_12================%%%%%%
\begin{ex}%[6K2NB3-1]
	Phương pháp cô cạn dùng để tách chất nào từ dung dịch?
	\choice
	{Chất lỏng dễ bay hơi}
	{\True Chất rắn tan trong nước}
	{Chất khí tan trong nước}
	{Chất lỏng không tan trong nước}
	\loigiai{
		Phương pháp cô cạn dùng để tách chất rắn tan (không bay hơi) ra khỏi dung dịch bằng cách đun nóng để dung môi bay hơi, chất rắn còn lại.
	}
\end{ex}

%%%%%============EX_13================%%%%%%
\begin{ex}%[6K2TH3-2]
	Để tách muối ăn ra khỏi nước biển, người ta sử dụng phương pháp nào?
	\choice
	{Lọc}
	{Chiết}
	{\True Cô cạn}
	{Chưng cất}
	\loigiai{
		Để tách muối ăn ra khỏi nước biển, người ta sử dụng phương pháp cô cạn. Khi đun nóng (hoặc phơi nắng), nước bay hơi và muối ăn kết tinh lại.
	}
\end{ex}

%%%%%============EX_14================%%%%%%
\begin{ex}%[6K2TH3-2]
	Phương pháp chưng cất dựa trên sự khác nhau về tính chất nào của các chất?
	\choice
	{Kích thước hạt}
	{Khối lượng riêng}
	{\True Nhiệt độ sôi}
	{Độ tan trong nước}
	\loigiai{
		Phương pháp chưng cất dựa trên sự khác nhau về nhiệt độ sôi của các chất. Khi đun nóng, chất có nhiệt độ sôi thấp hơn sẽ bay hơi trước, sau đó ngưng tụ lại thành chất lỏng tinh khiết.
	}
\end{ex}

%%%%%============EX_15================%%%%%%
\begin{ex}%[6K2TH3-3]
	Để tách dầu ăn ra khỏi nước, ta dùng phương pháp nào?
	\choice
	{Lọc}
	{\True Chiết}
	{Cô cạn}
	{Chưng cất}
	\loigiai{
		Để tách dầu ăn ra khỏi nước, ta dùng phương pháp chiết vì dầu ăn không tan trong nước và nhẹ hơn nước nên nổi lên trên. Dùng phễu chiết có thể tách riêng hai lớp chất lỏng này.
	}
\end{ex}

%%%%%============EX_16================%%%%%%
\begin{ex}%[6K2VD3-3]
	Để tách riêng hỗn hợp cát và muối ăn, ta cần thực hiện các bước theo thứ tự nào?
	\choice
	{Lọc $\rightarrow$ Hòa tan $\rightarrow$ Cô cạn}
	{Cô cạn $\rightarrow$ Hòa tan $\rightarrow$ Lọc}
	{\True Hòa tan $\rightarrow$ Lọc $\rightarrow$ Cô cạn}
	{Lọc $\rightarrow$ Cô cạn $\rightarrow$ Hòa tan}
	\loigiai{
		Để tách riêng hỗn hợp cát và muối ăn:
		\begin{enumerate}
		\item Hòa tan hỗn hợp vào nước: muối tan, cát không tan
		\item Lọc: tách cát ra khỏi dung dịch nước muối
		\item Cô cạn: làm bay hơi nước, thu được muối ăn
		\end{enumerate}
	}
\end{ex}

%%%%%============EX_17================%%%%%%
\begin{ex}%[6K2VD3-3]
	Để thu được nước cất từ nước máy, người ta sử dụng phương pháp nào?
	\choice
	{Lọc}
	{Cô cạn}
	{Chiết}
	{\True Chưng cất}
	\loigiai{
		Để thu được nước cất từ nước máy, người ta sử dụng phương pháp chưng cất. Khi đun sôi nước máy, hơi nước bay lên được làm lạnh và ngưng tụ thành nước cất tinh khiết, các chất tan và tạp chất còn lại trong bình đun.
	}
\end{ex}

%%%%%============EX_18================%%%%%%
\begin{ex}%[6K2VC3-4]
	Một mẫu nước giếng có lẫn đất cát và muối hòa tan. Để thu được nước tinh khiết từ mẫu nước này, ta cần thực hiện theo thứ tự nào?
	\choice
	{Cô cạn rồi lọc}
	{Chưng cất rồi lọc}
	{\True Lọc rồi chưng cất}
	{Chiết rồi cô cạn}
	\loigiai{
		Để thu được nước tinh khiết từ mẫu nước giếng có lẫn đất cát và muối:
		\begin{enumerate}
		\item Lọc: tách đất cát (chất rắn không tan) ra khỏi nước
		\item Chưng cất: tách nước tinh khiết ra khỏi muối hòa tan
		\end{enumerate}
		Nước thu được sau chưng cất là nước tinh khiết (nước cất).
	}
\end{ex}

\Closesolutionfile{ans}
\Closesolutionfile{ansex}

%%%==============Phần bài tập tự luận==============%%%
\subsection{Bài tập tự luận}\textit{\large Thí sinh trả lời từ bài 1 đến bài 4}
\Opensolutionfile{ansbth}[Ans/LGBT-6K2_HonHop_MADE847]
\Opensolutionfile{ansbt}[Ans/AnsBT-6K2_HonHop_MADE847]

%%%%%============BT_1================%%%%%%
\begin{bt}%[6K2TH1-2]
	Phân biệt hỗn hợp đồng nhất và hỗn hợp không đồng nhất. Cho ví dụ minh họa mỗi loại.
	\loigiai{
		\textbf{Hỗn hợp đồng nhất:}
		\begin{itemize}
		\item Là hỗn hợp không nhìn thấy ranh giới giữa các thành phần
		\item Các chất phân bố đều trong toàn bộ hỗn hợp
		\item Ví dụ: nước muối, nước đường, giấm ăn, không khí, rượu pha nước
		\end{itemize}

		\textbf{Hỗn hợp không đồng nhất:}
		\begin{itemize}
		\item Là hỗn hợp nhìn thấy rõ ranh giới giữa các thành phần
		\item Các chất không phân bố đều trong hỗn hợp
		\item Ví dụ: nước cam có tép, nước bùn (huyền phù), dầu giấm (nhũ tương), cát trộn với sỏi
		\end{itemize}
	}
\end{bt}

%%%%%============BT_2================%%%%%%
\begin{bt}%[6K2VD2-2]
	Hãy phân loại các hỗn hợp sau thành dung dịch, huyền phù hoặc nhũ tương và giải thích:
	\begin{enumerate}
	\item Nước chanh pha đường
	\item Nước bột sắn khuấy đều
	\item Sữa tươi
	\item Nước mắm
	\end{enumerate}
	\loigiai{
		\begin{enumerate}
		\item \textbf{Nước chanh pha đường:} Đây là \textbf{dung dịch} vì đường và axit citric trong chanh tan hoàn toàn trong nước, tạo thành hỗn hợp đồng nhất, không nhìn thấy ranh giới giữa các thành phần.

		\item \textbf{Nước bột sắn khuấy đều:} Đây là \textbf{huyền phù} vì các hạt bột sắn (chất rắn) lơ lửng trong nước (chất lỏng). Để lâu, bột sắn sẽ lắng xuống đáy.

		\item \textbf{Sữa tươi:} Đây là \textbf{nhũ tương} vì gồm các giọt chất béo (chất lỏng) phân tán trong nước. Hai chất lỏng này không tan vào nhau.

		\item \textbf{Nước mắm:} Đây là \textbf{dung dịch} vì muối và các chất hữu cơ tan hoàn toàn trong nước, tạo thành hỗn hợp đồng nhất.
		\end{enumerate}
	}
\end{bt}

%%%%%============BT_3================%%%%%%
\begin{bt}%[6K2VD3-3]
	Trình bày phương pháp tách riêng các chất trong hỗn hợp sau:
	\begin{enumerate}
	\item Hỗn hợp bột sắt và bột lưu huỳnh
	\item Hỗn hợp nước và rượu
	\item Hỗn hợp cát, muối ăn và nước
	\end{enumerate}
	\loigiai{
		\begin{enumerate}
		\item \textbf{Hỗn hợp bột sắt và bột lưu huỳnh:}
		\\
		Sử dụng \textbf{nam châm} để tách. Sắt có tính chất bị nam châm hút, còn lưu huỳnh không bị nam châm hút.
		\\
		Cách làm: Đưa nam châm lại gần hỗn hợp, bột sắt sẽ bị hút dính vào nam châm, bột lưu huỳnh còn lại.

		\item \textbf{Hỗn hợp nước và rượu:}
		\\
		Sử dụng phương pháp \textbf{chưng cất} vì nước và rượu có nhiệt độ sôi khác nhau.
		\\
		Rượu có nhiệt độ sôi khoảng $78{,}3^\circ\text{C}$, nước có nhiệt độ sôi $100^\circ\text{C}$.
		\\
		Cách làm: Đun nóng hỗn hợp, rượu bay hơi trước (ở nhiệt độ thấp hơn), hơi rượu được làm lạnh và ngưng tụ thành rượu tinh khiết.

		\item \textbf{Hỗn hợp cát, muối ăn và nước:}
		\\
		Sử dụng kết hợp phương pháp \textbf{lọc} và \textbf{cô cạn}.
		\\
		Cách làm:
		\begin{itemize}
		\item Bước 1: Khuấy đều hỗn hợp để muối tan hết trong nước
		\item Bước 2: Lọc hỗn hợp qua giấy lọc. Cát (không tan) được giữ lại trên giấy lọc, dung dịch nước muối chảy qua
		\item Bước 3: Cô cạn dung dịch nước muối. Nước bay hơi, muối ăn kết tinh lại
		\end{itemize}
		\end{enumerate}
	}
\end{bt}

%%%%%============BT_4================%%%%%%
\begin{bt}%[6K2VC3-4]
	Một học sinh có $200$ ml nước ao hồ bị vẩn đục (có lẫn bùn đất) và có lẫn một ít dầu ăn nổi trên mặt. Hãy trình bày cách làm để thu được nước sạch nhất có thể từ mẫu nước này. Giải thích tại sao lại chọn các phương pháp đó.
	\loigiai{
		\textbf{Các bước thực hiện:}

		\textbf{Bước 1: Dùng phương pháp chiết để tách dầu ăn}
		\begin{itemize}
		\item Đổ hỗn hợp vào phễu chiết
		\item Để yên cho dầu ăn nổi lên trên (vì dầu nhẹ hơn nước và không tan trong nước)
		\item Mở khóa phễu chiết, cho nước (lớp dưới) chảy ra trước, đóng khóa khi dầu sắp chảy xuống
		\item Thu được: nước vẩn đục (không còn dầu) và dầu ăn riêng
		\end{itemize}

		\textbf{Bước 2: Dùng phương pháp lọc để tách bùn đất}
		\begin{itemize}
		\item Đặt giấy lọc vào phễu lọc
		\item Đổ nước vẩn đục qua phễu lọc
		\item Bùn đất (hạt rắn có kích thước lớn) bị giữ lại trên giấy lọc
		\item Thu được: nước trong hơn (đã tách bùn đất)
		\end{itemize}

		\textbf{Bước 3: Dùng phương pháp chưng cất để thu nước tinh khiết}
		\begin{itemize}
		\item Đun sôi nước trong bình chưng cất
		\item Hơi nước bay lên, được làm lạnh và ngưng tụ thành nước cất
		\item Các chất tan còn lại trong bình đun
		\item Thu được: nước cất (nước sạch nhất)
		\end{itemize}

		\textbf{Giải thích lý do chọn các phương pháp:}
		\begin{itemize}
		\item \textbf{Chiết:} Vì dầu ăn không tan trong nước và có khối lượng riêng nhỏ hơn nước nên nổi lên trên, dễ dàng tách riêng bằng phễu chiết.
		\item \textbf{Lọc:} Vì bùn đất là chất rắn không tan, có kích thước hạt lớn hơn lỗ giấy lọc nên bị giữ lại.
		\item \textbf{Chưng cất:} Vì nước và các chất tan có nhiệt độ sôi khác nhau. Nước bay hơi ở $100^\circ\text{C}$, khi ngưng tụ thu được nước tinh khiết, các chất tan (muối, chất hữu cơ) còn lại trong bình.
		\end{itemize}
	}
\end{bt}

\Closesolutionfile{ansbt}
\Closesolutionfile{ansbth}

\begin{center}
 \rule[4pt]{2cm}{1pt}\,\large\bfseries Hết\,\rule[4pt]{2cm}{1pt}
\end{center}
\label{x}
\end{document}
