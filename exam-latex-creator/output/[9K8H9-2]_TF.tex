%%%=============TF_1=============%%%
\begin{ex}%[9K8H9-2]
	Ethylic alcohol ($C_2H_5OH$) và Dimethyl ether ($CH_3OCH_3$) là hai đồng phân của nhau. Khi xét về cấu tạo và tính chất của hai chất này, các phát biểu sau đây là đúng hay sai?
	\choiceTF
	{\True Cả hai chất đều có cùng công thức phân tử là $C_2H_6O$.}
	{Cả hai chất đều phản ứng được với kim loại $Na$ giải phóng khí $H_2$.}
	{\True Ethylic alcohol có nhiệt độ sôi cao hơn Dimethyl ether.}
	{Trong phân tử Dimethyl ether có nhóm $-OH$ linh động.}
	\loigiai{
		\begin{itemchoice}[T1,F2,T3,F4]
			\itemch Cả $C_2H_5OH$ và $CH_3OCH_3$ đều có công thức phân tử $C_2H_6O$. Phát biểu đúng.
			\itemch Chỉ có ethylic alcohol có nhóm $-OH$ mới phản ứng với $Na$, dimethyl ether không phản ứng. Phát biểu sai.
			\itemch Ethylic alcohol có liên kết hydrogen liên phân tử nên nhiệt độ sôi cao hơn nhiều so với dimethyl ether (không có liên kết hydrogen). Phát biểu đúng.
			\itemch Dimethyl ether có công thức cấu tạo $CH_3-O-CH_3$, không có nhóm $-OH$. Phát biểu sai.
		\end{itemchoice}
	}
\end{ex}

%%%=============TF_2=============%%%
\begin{ex}%[9K8H9-2]
	Dung dịch ethylic alcohol $45^\circ$ có nghĩa là trong 100 ml dung dịch có 45 ml ethylic alcohol nguyên chất. Xét các phát biểu sau về độ cồn và tính chất của dung dịch rượu:
	\choiceTF
	{\True Độ cồn là số mililit ethylic alcohol nguyên chất có trong 100 ml dung dịch ở $20^\circ C$.}
	{Rượu $45^\circ$ nhẹ hơn nước nguyên chất.}
	{\True Khi pha loãng rượu nguyên chất với nước, thể tích dung dịch thu được bằng tổng thể tích rượu và nước ban đầu.}
	{Có thể dùng thước đo độ cồn (cồn kế) để xác định độ cồn của dung dịch.}
	\loigiai{
		\begin{itemchoice}[T1,T2,F3,T4]
			\itemch Đây là định nghĩa độ cồn. Phát biểu đúng.
			\itemch Ethylic alcohol ($D \approx 0,8 g/ml$) nhẹ hơn nước ($D \approx 1 g/ml$) nên dung dịch rượu nhẹ hơn nước. Phát biểu đúng.
			\itemch Khi pha trộn rượu và nước, có sự co thể tích (do liên kết hydrogen và sự sắp xếp phân tử) nên thể tích dung dịch thu được nhỏ hơn tổng thể tích thành phần. Phát biểu sai.
			\itemch Cồn kế hoạt động dựa trên lự đẩy Archimedes, dùng để đo tỉ trọng và suy ra độ cồn. Phát biểu đúng.
		\end{itemchoice}
	}
\end{ex}

%%%=============EX_3=============%%%
\begin{ex}%[9K8H9-2]
	Hiện tượng nào xảy ra khi đốt cháy hoàn toàn ethylic alcohol trong không khí?
	\choice
	{Ngọn lửa màu vàng, có khói đen}
	{\True Ngọn lửa màu xanh mờ, tỏa nhiều nhiệt}
	{Có kết tủa trắng tạo thành}
	{Không cháy, chỉ bay hơi}
	\loigiai{
		Ethylic alcohol cháy trong không khí với ngọn lửa màu xanh nhạt (xanh mờ), tỏa nhiều nhiệt và không sinh ra khói đen (khác với hydrocarbon như acetylene).
		$C_2H_5OH + 3O_2 \xrightarrow{t^\circ} 2CO_2 + 3H_2O$.
	}
\end{ex}
