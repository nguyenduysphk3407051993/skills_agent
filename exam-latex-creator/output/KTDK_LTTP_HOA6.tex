\documentclass[Main.tex]{subfiles}
\gdef\sophong{Sở GD \& ĐT Gia Lai}
\gdef\truong{Trường THPT Chi Lăng}
\gdef\truongh{Trường Mầm non, THCS, THPT Sao Việt}
\gdef\monhoc{Khoa học tự nhiên 6}
\gdef\ngaykt{04/02/2026}
\gdef\nh{2025 - 2026}
\gdef\thoigian{45}
\gdef\made{168}
\setcounter{section}{0}
%\tatloigiai
%\hienthiloigiai
%\dongkeloigiai
\begin{document}
%\section[Kiểm tra định kỳ KHTN 6 - Mã đề \made]{Kiểm tra định kỳ}
\Tieudegiua{Kiểm tra định kỳ KHTN 6 - Mã đề \made}

%%%==============Phần trắc nghiệm nhiều lựa chọn==============%%%
\subsection{Bài tập trắc nghiệm nhiều lựa chọn}\textit{\large Thí sinh trả lời từ câu 1 đến câu 12. Mỗi câu thí sinh chỉ chọn một phương án}
\Opensolutionfile{ansex}[Ans/LGEX-KTDK_LTTP_HOA6_MADE168]
\Opensolutionfile{ans}[Ans/Ans-KTDK_LTTP_HOA6_MADE168]

%%%=============EX_1=============%%%
\begin{ex}%[6K6N1-1]
	Thành phần chính của gạo, ngô, khoai, sắn là chất nào sau đây?
	\choice
	{Protein}
	{Chất béo}
	{\True Tinh bột}
	{Vitamin}
	\loigiai{
		Gạo, ngô, khoai, sắn là các loại lương thực có thành phần chính là tinh bột (carbohydrate), chiếm khoảng $70-80\%$ khối lượng.
	}
\end{ex}

%%%=============EX_2=============%%%
\begin{ex}%[6K6N1-2]
	Nhóm thực phẩm nào sau đây giàu protein?
	\choice
	{Gạo, ngô, khoai}
	{\True Thịt, cá, trứng, sữa}
	{Rau xanh, trái cây}
	{Dầu ăn, mỡ động vật}
	\loigiai{
		Thịt, cá, trứng, sữa là nhóm thực phẩm giàu protein. Protein cần thiết cho sự phát triển và tái tạo tế bào cơ thể.
	}
\end{ex}

%%%=============EX_3=============%%%
\begin{ex}%[6K6N1-3]
	Chất béo có nhiều trong loại thực phẩm nào sau đây?
	\choice
	{Rau cải, rau muống}
	{Cam, quýt, bưởi}
	{Gạo, bánh mì}
	{\True Dầu thực vật, mỡ lợn}
	\loigiai{
		Dầu thực vật và mỡ lợn là những thực phẩm giàu chất béo (lipid). Chất béo cung cấp năng lượng và giúp hấp thu các vitamin tan trong dầu.
	}
\end{ex}

%%%=============EX_4=============%%%
\begin{ex}%[6K6N1-4]
	Vitamin C có nhiều trong loại thực phẩm nào sau đây?
	\choice
	{Thịt bò, thịt lợn}
	{Gạo, ngô}
	{\True Cam, chanh, bưởi}
	{Trứng, sữa}
	\loigiai{
		Vitamin C có nhiều trong các loại trái cây họ cam quýt như cam, chanh, bưởi. Vitamin C giúp tăng cường hệ miễn dịch và chống oxy hóa.
	}
\end{ex}

%%%=============EX_5=============%%%
\begin{ex}%[6K6T1-5]
	Lương thực khác với thực phẩm ở điểm nào sau đây?
	\choice
	{Lương thực cung cấp vitamin, thực phẩm cung cấp năng lượng}
	{Lương thực có nguồn gốc động vật, thực phẩm có nguồn gốc thực vật}
	{\True Lương thực là nguồn cung cấp tinh bột chính, thực phẩm là khái niệm rộng hơn bao gồm cả lương thực}
	{Lương thực và thực phẩm là hai khái niệm hoàn toàn giống nhau}
	\loigiai{
		Lương thực là nhóm thực phẩm cung cấp tinh bột chính cho cơ thể (gạo, ngô, khoai, sắn). Thực phẩm là khái niệm rộng hơn, bao gồm tất cả những gì con người ăn uống được, trong đó có lương thực.
	}
\end{ex}

%%%=============EX_6=============%%%
\begin{ex}%[6K6T1-6]
	Khi tinh bột trong thực phẩm vào cơ thể sẽ được chuyển hóa thành chất nào để cung cấp năng lượng?
	\choice
	{Protein}
	{Chất béo}
	{\True Glucose}
	{Cellulose}
	\loigiai{
		Tinh bột khi vào cơ thể sẽ được enzyme amylase thủy phân thành glucose. Glucose sau đó được hô hấp tế bào để tạo năng lượng ATP cho các hoạt động sống.
	}
\end{ex}

%%%=============EX_7=============%%%
\begin{ex}%[6K6T1-7]
	Phương pháp bảo quản thực phẩm nào sau đây dựa trên nguyên tắc làm giảm độ ẩm?
	\choice
	{Bảo quản lạnh}
	{\True Phơi khô, sấy khô}
	{Muối chua}
	{Đóng hộp}
	\loigiai{
		Phơi khô, sấy khô là phương pháp bảo quản dựa trên nguyên tắc làm giảm độ ẩm của thực phẩm xuống dưới $15\%$, ngăn vi sinh vật phát triển.
	}
\end{ex}

%%%=============EX_8=============%%%
\begin{ex}%[6K6N1-8]
	Calcium là khoáng chất cần thiết cho xương và răng. Thực phẩm nào sau đây giàu calcium nhất?
	\choice
	{Thịt lợn}
	{Gạo trắng}
	{\True Sữa và các sản phẩm từ sữa}
	{Dầu ăn}
	\loigiai{
		Sữa và các sản phẩm từ sữa (phô mai, sữa chua) là nguồn calcium dồi dào nhất, giúp xương và răng chắc khỏe.
	}
\end{ex}

%%%=============EX_9=============%%%
\begin{ex}%[6K6T1-9]
	Vì sao cần ăn đa dạng các loại thực phẩm?
	\choice
	{Để thức ăn ngon miệng hơn}
	{Để tiết kiệm chi phí mua thực phẩm}
	{\True Để cơ thể được cung cấp đầy đủ các chất dinh dưỡng cần thiết}
	{Để thực phẩm bảo quản được lâu hơn}
	\loigiai{
		Mỗi loại thực phẩm cung cấp các chất dinh dưỡng khác nhau. Ăn đa dạng giúp cơ thể nhận được đầy đủ protein, carbohydrate, lipid, vitamin và khoáng chất cần thiết cho sức khỏe.
	}
\end{ex}

%%%=============EX_10=============%%%
\begin{ex}%[6K6V1-10]
	Một học sinh ăn sáng với $100$ g bánh mì (chứa $50\%$ tinh bột). Biết $1$ g tinh bột cung cấp $4$ kcal. Năng lượng từ tinh bột mà học sinh đó nhận được là bao nhiêu?
	\choice
	{$100$ kcal}
	{$150$ kcal}
	{\True $200$ kcal}
	{$400$ kcal}
	\loigiai{
		Khối lượng tinh bột trong $100$ g bánh mì: $m = 100 \times 50\% = 50$ g.\\
		Năng lượng nhận được: $E = 50 \times 4 = 200$ kcal.
	}
\end{ex}

%%%=============EX_11=============%%%
\begin{ex}%[6K6T1-11]
	Thực phẩm bị ôi thiu là do nguyên nhân chính nào sau đây?
	\choice
	{Do ánh sáng mặt trời}
	{Do nhiệt độ quá thấp}
	{\True Do vi sinh vật phân hủy}
	{Do thiếu nước}
	\loigiai{
		Thực phẩm bị ôi thiu chủ yếu do vi sinh vật (vi khuẩn, nấm mốc) phân hủy các chất hữu cơ trong thực phẩm, tạo ra các chất có mùi khó chịu và độc hại.
	}
\end{ex}

%%%=============EX_12=============%%%
\begin{ex}%[6K6N1-12]
	Chất xơ (cellulose) có nhiều trong nhóm thực phẩm nào sau đây?
	\choice
	{Thịt, cá, trứng}
	{Dầu ăn, bơ}
	{Sữa, phô mai}
	{\True Rau xanh, trái cây}
	\loigiai{
		Chất xơ (cellulose) có nhiều trong rau xanh và trái cây. Chất xơ giúp hệ tiêu hóa hoạt động tốt, ngăn ngừa táo bón.
	}
\end{ex}

\Closesolutionfile{ans}
\Closesolutionfile{ansex}
%\bangdapan{Ans-KTDK_LTTP_HOA6_MADE168}

%%%==============Phần trắc nghiệm đúng sai==============%%%
\subsection{Trắc nghiệm đúng sai}\textit{\large Thí sinh trả lời từ câu 1 đến câu 4. Trong mỗi ý a), b), c), d) ở mỗi câu thí sinh chọn đúng hoặc sai}
\Opensolutionfile{ansex}[Ans/LGTF-KTDK_LTTP_HOA6_MADE168]
\Opensolutionfile{ansbook}[Ansbook/AnsTF-KTDK_LTTP_HOA6_MADE168]
\Opensolutionfile{ans}[Ans/Tempt-KTDK_LTTP_HOA6_MADE168]
\setcounter{ex}{0}

%%%=============TF_1=============%%%
\begin{ex}%[6K6T1-1]
	Gạo là lương thực chính của người Việt Nam, cung cấp năng lượng cho hoạt động sống hàng ngày. Đánh giá tính đúng/sai của các phát biểu sau:
	\choiceTF
	{\True Thành phần chính của gạo là tinh bột}
	{Gạo chứa nhiều vitamin C hơn cam}
	{\True Tinh bột trong gạo khi vào cơ thể được chuyển hóa thành glucose}
	{\True Gạo thuộc nhóm lương thực}
	\loigiai{
		\begin{itemchoice}[T1,F2,T3,T4]
			\itemch Gạo chứa khoảng $75-80\%$ tinh bột, đây là thành phần chính. Phát biểu đúng.
			\itemch Gạo chứa rất ít vitamin C, trong khi cam rất giàu vitamin C ($30-50$ mg/$100$ g). Phát biểu sai.
			\itemch Tinh bột được enzyme amylase thủy phân thành glucose để cung cấp năng lượng. Phát biểu đúng.
			\itemch Gạo là lương thực, cùng nhóm với ngô, khoai, sắn. Phát biểu đúng.
		\end{itemchoice}
	}
\end{ex}

%%%=============TF_2=============%%%
\begin{ex}%[6K6T1-2]
	Protein là chất dinh dưỡng quan trọng cho sự phát triển cơ thể. Đánh giá tính đúng/sai của các phát biểu sau:
	\choiceTF
	{\True Thịt, cá, trứng là thực phẩm giàu protein}
	{Protein chỉ có trong thực phẩm có nguồn gốc động vật}
	{\True Protein giúp xây dựng và tái tạo tế bào cơ thể}
	{Gạo không chứa protein}
	\loigiai{
		\begin{itemchoice}[T1,F2,T3,F4]
			\itemch Thịt chứa $18-22\%$ protein, cá $15-20\%$, trứng $12-13\%$. Đây đều là thực phẩm giàu protein. Phát biểu đúng.
			\itemch Protein có cả trong thực phẩm thực vật như đậu nành, đậu xanh. Phát biểu sai.
			\itemch Protein là nguyên liệu chính để xây dựng tế bào, cơ, da, tóc và các mô trong cơ thể. Phát biểu đúng.
			\itemch Gạo có chứa protein (khoảng $7-8\%$), tuy không nhiều như thịt cá. Phát biểu sai.
		\end{itemchoice}
	}
\end{ex}

%%%=============TF_3=============%%%
\begin{ex}%[6K6T1-3]
	Bảo quản thực phẩm đúng cách giúp giữ được chất dinh dưỡng và đảm bảo an toàn. Đánh giá tính đúng/sai của các phát biểu sau:
	\choiceTF
	{\True Bảo quản lạnh giúp làm chậm sự phát triển của vi sinh vật}
	{\True Phơi khô là phương pháp bảo quản bằng cách giảm độ ẩm}
	{Thực phẩm đã nấu chín không bao giờ bị hỏng}
	{\True Muối ăn có tác dụng bảo quản thực phẩm}
	\loigiai{
		\begin{itemchoice}[T1,T2,F3,T4]
			\itemch Ở nhiệt độ thấp, vi sinh vật phát triển chậm, giúp bảo quản thực phẩm lâu hơn. Phát biểu đúng.
			\itemch Phơi khô làm giảm độ ẩm xuống dưới $15\%$, ngăn vi sinh vật phát triển. Phát biểu đúng.
			\itemch Thực phẩm đã nấu chín vẫn có thể bị hỏng nếu bảo quản không đúng cách. Phát biểu sai.
			\itemch Muối hút nước từ tế bào vi khuẩn, ức chế sự phát triển của vi sinh vật. Phát biểu đúng.
		\end{itemchoice}
	}
\end{ex}

%%%=============TF_4=============%%%
\begin{ex}%[6K6T1-4]
	Vitamin và khoáng chất là các vi chất dinh dưỡng cần thiết cho cơ thể. Đánh giá tính đúng/sai của các phát biểu sau:
	\choiceTF
	{\True Vitamin C có nhiều trong cam, chanh, bưởi}
	{\True Sữa là thực phẩm giàu calcium}
	{Vitamin A tan trong nước}
	{Rau xanh không cung cấp chất xơ cho cơ thể}
	\loigiai{
		\begin{itemchoice}[T1,T2,F3,F4]
			\itemch Cam, chanh, bưởi thuộc họ cam quýt, rất giàu vitamin C. Phát biểu đúng.
			\itemch Sữa chứa khoảng $120$ mg calcium/$100$ ml, là nguồn calcium quan trọng. Phát biểu đúng.
			\itemch Vitamin A tan trong dầu/chất béo, không tan trong nước. Vitamin tan trong nước là B, C. Phát biểu sai.
			\itemch Rau xanh chứa nhiều chất xơ (cellulose), giúp hệ tiêu hóa hoạt động tốt. Phát biểu sai.
		\end{itemchoice}
	}
\end{ex}

\Closesolutionfile{ans}
\Closesolutionfile{ansbook}
\Closesolutionfile{ansex}
%\bangdapanTF{AnsTF-KTDK_LTTP_HOA6_MADE168}

%%==============Phần bài tập trả lời ngắn==============%%%
\subsection{Bài tập trả lời ngắn}\textit{\large Thí sinh trả lời từ câu 1 đến câu 4}
\Opensolutionfile{ansex}[Ans/LGSA-KTDK_LTTP_HOA6_MADE168]
\Opensolutionfile{ansexh}[Ans/AnsSA-KTDK_LTTP_HOA6_MADE168]
\setcounter{ex}{0}

%%%=============SA_1=============%%%
\begin{ex}%[6K6V1-1]
	Một quả trứng gà có khối lượng $50$ g, trong đó protein chiếm $12\%$. Tính khối lượng protein (đơn vị gam) có trong quả trứng đó.
	\shortans{$6$}
	\loigiai{
		Khối lượng protein trong quả trứng:\\
		$m_{protein} = 50 \times 12\% = 50 \times 0{,}12 = 6$ g.
	}
\end{ex}

%%%=============SA_2=============%%%
\begin{ex}%[6K6V1-2]
	Biết $1$ g chất béo cung cấp $9$ kcal năng lượng. Nếu một người ăn $30$ g dầu ăn, năng lượng từ chất béo mà người đó nhận được là bao nhiêu kcal?
	\shortans{$270$}
	\loigiai{
		Năng lượng từ chất béo:\\
		$E = 30 \times 9 = 270$ kcal.
	}
\end{ex}

%%%=============SA_3=============%%%
\begin{ex}%[6K6V1-3]
	Một bát cơm chứa $150$ g gạo đã nấu chín, trong đó tinh bột chiếm $30\%$. Biết $1$ g tinh bột cung cấp $4$ kcal. Tính năng lượng (đơn vị kcal) từ tinh bột mà bát cơm cung cấp.
	\shortans{$180$}
	\loigiai{
		Khối lượng tinh bột trong bát cơm: $m = 150 \times 30\% = 45$ g.\\
		Năng lượng từ tinh bột: $E = 45 \times 4 = 180$ kcal.
	}
\end{ex}

%%%=============SA_4=============%%%
\begin{ex}%[6K6V1-4]
	Một hộp sữa có thể tích $200$ ml, chứa $240$ mg calcium. Một học sinh cần bổ sung $720$ mg calcium mỗi ngày từ sữa. Hỏi học sinh đó cần uống bao nhiêu hộp sữa mỗi ngày?
	\shortans{$3$}
	\loigiai{
		Số hộp sữa cần uống mỗi ngày:\\
		$n = \dfrac{720}{240} = 3$ hộp.
	}
\end{ex}

\Closesolutionfile{ansexh}
\Closesolutionfile{ansex}
%\bangdapanSA{AnsSA-KTDK_LTTP_HOA6_MADE168}



%%%==============Phần bài tập tự luận==============%%%
\subsection{Bài tập tự luận}\textit{\large Thí sinh trả lời từ bài 1 đến bài 3}
\Opensolutionfile{ansbth}[Ans/LGBT-KTDK_LTTP_HOA6_MADE168]
\Opensolutionfile{ansbt}[Ans/AnsBT-KTDK_LTTP_HOA6_MADE168]

%%%=============BT_1=============%%%
\begin{bt}%[6K6T1-1]
	Hãy phân biệt lương thực và thực phẩm. Cho ví dụ minh họa.
	\loigiai{
		\textbf{Phân biệt lương thực và thực phẩm:}
		\begin{enumerate}
			\item \textbf{Lương thực:}
			\begin{itemize}
				\item Là nhóm thực phẩm cung cấp nguồn năng lượng chính cho cơ thể.
				\item Thành phần chủ yếu là tinh bột (carbohydrate).
				\item Ví dụ: gạo, ngô, khoai, sắn, lúa mì.
			\end{itemize}
			\item \textbf{Thực phẩm:}
			\begin{itemize}
				\item Là khái niệm rộng hơn, bao gồm tất cả những gì con người ăn uống được.
				\item Cung cấp đa dạng các chất dinh dưỡng: protein, lipid, vitamin, khoáng chất, carbohydrate.
				\item Ví dụ: thịt, cá, trứng, sữa, rau, trái cây, lương thực,...
			\end{itemize}
			\item \textbf{Mối quan hệ:} Lương thực là một phần của thực phẩm. Tất cả lương thực đều là thực phẩm, nhưng không phải thực phẩm nào cũng là lương thực.
		\end{enumerate}
	}
\end{bt}

%%%=============BT_2=============%%%
\begin{bt}%[6K6V1-2]
	Một gia đình có $4$ người, mỗi người trung bình cần $2000$ kcal năng lượng mỗi ngày. Biết rằng $60\%$ năng lượng được cung cấp từ tinh bột và $1$ g tinh bột cung cấp $4$ kcal.
	\begin{enumerate}
		\item Tính tổng năng lượng cần thiết cho gia đình trong một ngày.
		\item Tính khối lượng tinh bột (đơn vị kg) cần thiết cho gia đình trong một ngày.
	\end{enumerate}
	\loigiai{
		\begin{enumerate}
			\item Tổng năng lượng cần thiết cho gia đình trong một ngày:
			\[ E_{tổng} = 4 \times 2000 = 8000 \text{ kcal} \]

			\item Năng lượng từ tinh bột:
			\[ E_{tinh bột} = 8000 \times 60\% = 4800 \text{ kcal} \]

			Khối lượng tinh bột cần thiết:
			\[ m_{tinh bột} = \dfrac{4800}{4} = 1200 \text{ g} = 1{,}2 \text{ kg} \]
		\end{enumerate}
	}
\end{bt}

%%%=============BT_3=============%%%
\begin{bt}%[6K6T1-3]
	Giải thích vì sao cần bảo quản thực phẩm đúng cách. Nêu $3$ phương pháp bảo quản thực phẩm thường dùng trong gia đình và giải thích nguyên tắc của từng phương pháp.
	\loigiai{
		\textbf{Lý do cần bảo quản thực phẩm đúng cách:}
		\begin{itemize}
			\item Ngăn ngừa sự phát triển của vi sinh vật gây hại (vi khuẩn, nấm mốc).
			\item Giữ được chất dinh dưỡng trong thực phẩm.
			\item Đảm bảo an toàn vệ sinh thực phẩm, tránh ngộ độc.
			\item Kéo dài thời gian sử dụng thực phẩm.
		\end{itemize}

		\textbf{Ba phương pháp bảo quản thực phẩm thường dùng:}
		\begin{enumerate}
			\item \textbf{Bảo quản lạnh (tủ lạnh, tủ đông):}
			\begin{itemize}
				\item Nguyên tắc: Ở nhiệt độ thấp ($0-4^\circ$C hoặc dưới $0^\circ$C), các phản ứng sinh hóa và sự phát triển của vi sinh vật bị ức chế.
				\item Áp dụng: Thịt, cá, rau, trái cây, sữa,...
			\end{itemize}

			\item \textbf{Phơi khô/sấy khô:}
			\begin{itemize}
				\item Nguyên tắc: Làm giảm độ ẩm của thực phẩm xuống dưới $15\%$, vi sinh vật không thể phát triển được.
				\item Áp dụng: Cá khô, mực khô, rau củ sấy, ngũ cốc,...
			\end{itemize}

			\item \textbf{Ướp muối:}
			\begin{itemize}
				\item Nguyên tắc: Muối có tính thẩm thấu cao, hút nước từ tế bào vi sinh vật ra ngoài (hiện tượng co nguyên sinh), làm vi sinh vật chết hoặc không phát triển được.
				\item Áp dụng: Cá muối, thịt muối, dưa muối,...
			\end{itemize}
		\end{enumerate}
	}
\end{bt}

\Closesolutionfile{ansbt}
\Closesolutionfile{ansbth}

\begin{center}
 \rule[4pt]{2cm}{1pt}\,\large\bfseries Hết\,\rule[4pt]{2cm}{1pt}
\end{center}
\label{x}
\end{document}
