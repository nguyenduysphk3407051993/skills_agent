\documentclass[FileMain.tex]{subfiles}
\gdef\made{201}
\begin{document}
\Tieudegiua{Kiểm tra định kỳ KHTN 6 - Mã đề \made}
%%%=============PHẦN 1: CÂU HỎI TRẮC NGHIỆM NHIỀU LỰA CHỌN=============%%%

%%%=============EX_1=============%%%
\begin{ex}%[6K6N1-1]
	Lương thực là gì?
	\choice
	{Là các loại rau, củ, quả}
	{\True Là các loại hạt, củ chứa nhiều tinh bột dùng làm nguồn cung cấp năng lượng chính}
	{Là các loại thịt, cá, trứng}
	{Là các loại đồ uống có ga}
	\loigiai{
		Lương thực là các loại hạt hoặc củ chứa nhiều tinh bột, được con người sử dụng làm nguồn cung cấp năng lượng chính trong bữa ăn hàng ngày như gạo, ngô, khoai, sắn.
	}
\end{ex}

%%%=============EX_2=============%%%
\begin{ex}%[6K6N1-1]
	Loại lương thực nào sau đây là lương thực chính của người Việt Nam?
	\choice
	{Ngô}
	{Khoai lang}
	{\True Gạo}
	{Sắn}
	\loigiai{
		Gạo là lương thực chính của người Việt Nam, cung cấp nguồn năng lượng chủ yếu cho bữa ăn hàng ngày của đại đa số người dân.
	}
\end{ex}

%%%=============EX_3=============%%%
\begin{ex}%[6K6N1-1]
	Chất dinh dưỡng chính có trong lương thực là gì?
	\choice
	{Protein}
	{Chất béo}
	{\True Tinh bột (carbohydrate)}
	{Vitamin}
	\loigiai{
		Lương thực chứa chủ yếu tinh bột (carbohydrate), là nguồn cung cấp năng lượng chính cho cơ thể. Khi tiêu hóa, tinh bột được chuyển hóa thành glucose để cung cấp năng lượng.
	}
\end{ex}

%%%=============EX_4=============%%%
\begin{ex}%[6K6N1-1]
	Thực phẩm nào sau đây thuộc nhóm thực phẩm giàu protein?
	\choice
	{Cơm trắng}
	{Bánh mì}
	{\True Thịt gà}
	{Khoai tây}
	\loigiai{
		Thịt gà là thực phẩm giàu protein (chất đạm). Protein có vai trò quan trọng trong việc xây dựng và sửa chữa các mô trong cơ thể.
	}
\end{ex}

%%%=============EX_5=============%%%
\begin{ex}%[6K6N1-1]
	Năng lượng mà $1$ gam tinh bột cung cấp cho cơ thể là bao nhiêu?
	\choice
	{$2$ kcal}
	{$3$ kcal}
	{\True $4$ kcal}
	{$9$ kcal}
	\loigiai{
		Mỗi gam tinh bột (carbohydrate) khi được cơ thể chuyển hóa hoàn toàn sẽ cung cấp khoảng $4$ kcal năng lượng.
	}
\end{ex}

%%%=============EX_6=============%%%
\begin{ex}%[6K6N1-1]
	Vitamin C có nhiều trong loại thực phẩm nào sau đây?
	\choice
	{Thịt bò}
	{Gạo}
	{\True Cam, chanh}
	{Dầu ăn}
	\loigiai{
		Vitamin C có nhiều trong các loại trái cây họ cam quýt như cam, chanh, bưởi. Vitamin C giúp tăng cường hệ miễn dịch và chống oxy hóa.
	}
\end{ex}

%%%=============EX_7=============%%%
\begin{ex}%[6K6N1-1]
	Loại thực phẩm nào sau đây cung cấp nhiều canxi nhất?
	\choice
	{Thịt heo}
	{\True Sữa và các sản phẩm từ sữa}
	{Cơm}
	{Rau muống}
	\loigiai{
		Sữa và các sản phẩm từ sữa (sữa chua, phô mai) là nguồn cung cấp canxi dồi dào nhất. Canxi cần thiết cho sự phát triển và duy trì xương, răng chắc khỏe.
	}
\end{ex}

%%%=============EX_8=============%%%
\begin{ex}%[6K6N1-1]
	Chất béo cung cấp bao nhiêu năng lượng cho mỗi gam?
	\choice
	{$4$ kcal}
	{$5$ kcal}
	{$7$ kcal}
	{\True $9$ kcal}
	\loigiai{
		Mỗi gam chất béo (lipid) khi được cơ thể chuyển hóa hoàn toàn sẽ cung cấp khoảng $9$ kcal năng lượng, cao hơn so với tinh bột và protein.
	}
\end{ex}

%%%=============EX_9=============%%%
\begin{ex}%[6K6N1-1]
	Thực phẩm nào sau đây KHÔNG thuộc nhóm lương thực?
	\choice
	{Gạo}
	{Ngô}
	{\True Cá hồi}
	{Khoai mì}
	\loigiai{
		Cá hồi là thực phẩm giàu protein và chất béo, không thuộc nhóm lương thực. Lương thực bao gồm các loại hạt, củ giàu tinh bột như gạo, ngô, khoai mì.
	}
\end{ex}

%%%=============EX_10=============%%%
\begin{ex}%[6K6N1-1]
	Để bảo quản thực phẩm tươi sống, phương pháp nào sau đây thường được sử dụng?
	\choice
	{Phơi nắng}
	{\True Bảo quản lạnh}
	{Ngâm trong nước}
	{Để ngoài trời}
	\loigiai{
		Bảo quản lạnh (trong tủ lạnh hoặc tủ đông) là phương pháp phổ biến để giữ thực phẩm tươi sống được lâu hơn vì nhiệt độ thấp làm chậm sự phát triển của vi khuẩn.
	}
\end{ex}

%%%=============EX_11=============%%%
\begin{ex}%[6K6N1-1]
	Protein có vai trò chính nào sau đây trong cơ thể?
	\choice
	{Cung cấp năng lượng chính cho hoạt động}
	{Hòa tan các vitamin}
	{\True Xây dựng và tái tạo các mô, tế bào}
	{Tạo lớp mỡ dự trữ dưới da}
	\loigiai{
		Protein (chất đạm) có vai trò chính là xây dựng và tái tạo các mô, tế bào trong cơ thể. Protein là thành phần cấu tạo chính của cơ bắp, da, tóc, móng và các cơ quan nội tạng.
	}
\end{ex}

%%%=============EX_12=============%%%
\begin{ex}%[6K6N1-1]
	Thiếu vitamin A có thể gây ra bệnh gì?
	\choice
	{Bệnh còi xương}
	{Bệnh thiếu máu}
	{\True Bệnh quáng gà}
	{Bệnh bướu cổ}
	\loigiai{
		Thiếu vitamin A có thể gây ra bệnh quáng gà (khó nhìn trong điều kiện ánh sáng yếu) và các vấn đề về mắt. Vitamin A có nhiều trong gan, trứng, cà rốt, rau xanh đậm.
	}
\end{ex}

%%%=============PHẦN 2: CÂU HỎI TRẮC NGHIỆM ĐÚNG SAI=============%%%

%%%=============TF_1=============%%%
\begin{ex}%[6K6N1-2]
	Cho các phát biểu sau về lương thực và thực phẩm:
	\choiceTF
	{\True Gạo là lương thực chính của người Việt Nam}
	{Protein cung cấp $9$ kcal năng lượng cho mỗi gam}
	{\True Vitamin C giúp tăng cường hệ miễn dịch}
	{Chất béo không cần thiết cho cơ thể}
	\loigiai{
		\begin{itemchoice}[T1,F2,T3,F4]
			\itemch Đúng. Gạo là lương thực chính của người Việt Nam, cung cấp nguồn năng lượng chủ yếu cho bữa ăn hàng ngày.
			\itemch Sai. Protein cung cấp $4$ kcal năng lượng cho mỗi gam, không phải $9$ kcal. Chất béo mới cung cấp $9$ kcal/gam.
			\itemch Đúng. Vitamin C giúp tăng cường hệ miễn dịch, chống oxy hóa và giúp cơ thể hấp thu sắt tốt hơn.
			\itemch Sai. Chất béo rất cần thiết cho cơ thể vì giúp hấp thu vitamin tan trong dầu (A, D, E, K), cung cấp năng lượng dự trữ và bảo vệ các cơ quan nội tạng.
		\end{itemchoice}
	}
\end{ex}

%%%=============TF_2=============%%%
\begin{ex}%[6K6N1-2]
	Cho các phát biểu sau về các nhóm chất dinh dưỡng:
	\choiceTF
	{Tinh bột cung cấp $9$ kcal năng lượng cho mỗi gam}
	{\True Canxi có nhiều trong sữa và các sản phẩm từ sữa}
	{\True Thiếu sắt có thể gây bệnh thiếu máu}
	{\True Rau xanh là nguồn cung cấp chất xơ tốt}
	\loigiai{
		\begin{itemchoice}[F1,T2,T3,T4]
			\itemch Sai. Tinh bột cung cấp $4$ kcal năng lượng cho mỗi gam, không phải $9$ kcal.
			\itemch Đúng. Sữa và các sản phẩm từ sữa như phô mai, sữa chua là nguồn cung cấp canxi dồi dào nhất.
			\itemch Đúng. Sắt là thành phần quan trọng của hemoglobin trong hồng cầu, thiếu sắt sẽ dẫn đến thiếu máu.
			\itemch Đúng. Rau xanh chứa nhiều chất xơ giúp hệ tiêu hóa hoạt động tốt và ngăn ngừa táo bón.
		\end{itemchoice}
	}
\end{ex}

%%%=============TF_3=============%%%
\begin{ex}%[6K6N1-2]
	Cho các phát biểu sau về bảo quản thực phẩm:
	\choiceTF
	{\True Bảo quản lạnh giúp làm chậm sự phát triển của vi khuẩn}
	{\True Phơi khô là phương pháp bảo quản thực phẩm truyền thống}
	{Thực phẩm đóng hộp không cần kiểm tra hạn sử dụng}
	{\True Muối có tác dụng bảo quản thực phẩm}
	\loigiai{
		\begin{itemchoice}[T1,T2,F3,T4]
			\itemch Đúng. Nhiệt độ thấp làm chậm sự sinh trưởng và phát triển của vi khuẩn gây hại.
			\itemch Đúng. Phơi khô làm giảm độ ẩm trong thực phẩm, ngăn cản vi khuẩn phát triển.
			\itemch Sai. Thực phẩm đóng hộp vẫn có hạn sử dụng và cần kiểm tra trước khi sử dụng để đảm bảo an toàn.
			\itemch Đúng. Muối có tác dụng khử nước và ức chế vi khuẩn, được dùng để bảo quản thịt, cá từ xa xưa.
		\end{itemchoice}
	}
\end{ex}

%%%=============TF_4=============%%%
\begin{ex}%[6K6N1-2]
	Cho các phát biểu sau về vai trò của các chất dinh dưỡng:
	\choiceTF
	{Vitamin D có nhiều trong các loại trái cây họ cam quýt}
	{\True Protein cần thiết cho sự phát triển cơ bắp}
	{\True Chất xơ giúp hệ tiêu hóa hoạt động tốt}
	{Thiếu iốt gây bệnh quáng gà}
	\loigiai{
		\begin{itemchoice}[F1,T2,T3,F4]
			\itemch Sai. Vitamin D có nhiều trong cá, trứng, sữa và được tổng hợp khi da tiếp xúc với ánh nắng mặt trời. Trái cây họ cam quýt giàu vitamin C.
			\itemch Đúng. Protein là thành phần cấu tạo chính của cơ bắp, cần thiết cho sự phát triển và sửa chữa các mô cơ.
			\itemch Đúng. Chất xơ giúp tăng nhu động ruột, ngăn ngừa táo bón và duy trì hệ tiêu hóa khỏe mạnh.
			\itemch Sai. Thiếu iốt gây bệnh bướu cổ, không phải bệnh quáng gà. Bệnh quáng gà do thiếu vitamin A.
		\end{itemchoice}
	}
\end{ex}

%%%=============PHẦN 3: CÂU HỎI TRẢ LỜI NGẮN=============%%%

%%%=============SA_1=============%%%
\begin{ex}%[6K6N1-3]
	Một người cần $2000$ kcal năng lượng mỗi ngày. Nếu $60\%$ năng lượng được cung cấp từ tinh bột, biết $1$ gam tinh bột cung cấp $4$ kcal. Hỏi người đó cần ăn bao nhiêu gam tinh bột mỗi ngày?
	\shortans{$300$}
	\loigiai{
		Năng lượng cần từ tinh bột:
		\[ E_{\text{tinh bột}} = 2000 \times 60\% = 2000 \times 0{,}6 = 1200 \text{ kcal} \]
		Khối lượng tinh bột cần thiết:
		\[ m_{\text{tinh bột}} = \dfrac{1200}{4} = 300 \text{ g} \]
	}
\end{ex}

%%%=============SA_2=============%%%
\begin{ex}%[6K6N1-3]
	Một học sinh ăn bữa sáng gồm $100$ g bánh mì (chứa $50\%$ tinh bột) và uống $200$ ml sữa (cung cấp $130$ kcal). Biết $1$ gam tinh bột cung cấp $4$ kcal. Tính tổng năng lượng bữa sáng cung cấp (đơn vị kcal).
	\shortans{$330$}
	\loigiai{
		Khối lượng tinh bột trong bánh mì:
		\[ m_{\text{tinh bột}} = 100 \times 50\% = 50 \text{ g} \]
		Năng lượng từ bánh mì:
		\[ E_{\text{bánh mì}} = 50 \times 4 = 200 \text{ kcal} \]
		Tổng năng lượng bữa sáng:
		\[ E_{\text{tổng}} = 200 + 130 = 330 \text{ kcal} \]
	}
\end{ex}

%%%=============SA_3=============%%%
\begin{ex}%[6K6N1-3]
	Một gia đình $4$ người, mỗi người cần trung bình $2200$ kcal/ngày. Nếu $55\%$ năng lượng được lấy từ gạo và $1$ kg gạo cung cấp $3500$ kcal. Hỏi gia đình đó cần bao nhiêu kg gạo cho $1$ tuần (làm tròn đến một chữ số thập phân)?
	\shortans{$9{,}7$}
	\loigiai{
		Tổng năng lượng cần cho gia đình mỗi ngày:
		\[ E_{\text{ngày}} = 4 \times 2200 = 8800 \text{ kcal} \]
		Năng lượng cần từ gạo mỗi ngày:
		\[ E_{\text{gạo/ngày}} = 8800 \times 55\% = 4840 \text{ kcal} \]
		Năng lượng cần từ gạo trong 1 tuần:
		\[ E_{\text{gạo/tuần}} = 4840 \times 7 = 33880 \text{ kcal} \]
		Khối lượng gạo cần cho 1 tuần:
		\[ m_{\text{gạo}} = \dfrac{33880}{3500} \approx 9{,}7 \text{ kg} \]
	}
\end{ex}

%%%=============SA_4=============%%%
\begin{ex}%[6K6N1-3]
	Một bữa ăn cung cấp $30$ g protein, $20$ g chất béo và $80$ g tinh bột. Biết $1$ g protein cung cấp $4$ kcal, $1$ g chất béo cung cấp $9$ kcal, $1$ g tinh bột cung cấp $4$ kcal. Tính tổng năng lượng của bữa ăn (đơn vị kcal).
	\shortans{$620$}
	\loigiai{
		Năng lượng từ protein:
		\[ E_{\text{protein}} = 30 \times 4 = 120 \text{ kcal} \]
		Năng lượng từ chất béo:
		\[ E_{\text{béo}} = 20 \times 9 = 180 \text{ kcal} \]
		Năng lượng từ tinh bột:
		\[ E_{\text{tinh bột}} = 80 \times 4 = 320 \text{ kcal} \]
		Tổng năng lượng:
		\[ E_{\text{tổng}} = 120 + 180 + 320 = 620 \text{ kcal} \]
	}
\end{ex}

%%%=============PHẦN 4: CÂU HỎI TỰ LUẬN=============%%%

%%%=============BT_1=============%%%
\begin{bt}%[6K6N1-4]
	Một học sinh lớp 6 cần $1800$ kcal năng lượng mỗi ngày. Trong đó, năng lượng từ các chất dinh dưỡng được phân bổ như sau: tinh bột chiếm $60\%$, protein chiếm $15\%$, chất béo chiếm $25\%$.
	\begin{enumerate}
		\item Tính năng lượng cần thiết từ mỗi loại chất dinh dưỡng.
		\item Tính khối lượng mỗi loại chất dinh dưỡng cần thiết, biết $1$ g tinh bột và protein đều cung cấp $4$ kcal, $1$ g chất béo cung cấp $9$ kcal.
	\end{enumerate}
	\loigiai{
		\textbf{1. Năng lượng cần thiết từ mỗi loại chất dinh dưỡng:}
		\begin{itemize}
			\item Năng lượng từ tinh bột: $E_{\text{tinh bột}} = 1800 \times 60\% = 1080$ kcal
			\item Năng lượng từ protein: $E_{\text{protein}} = 1800 \times 15\% = 270$ kcal
			\item Năng lượng từ chất béo: $E_{\text{béo}} = 1800 \times 25\% = 450$ kcal
		\end{itemize}

		\textbf{2. Khối lượng mỗi loại chất dinh dưỡng:}
		\begin{itemize}
			\item Khối lượng tinh bột: $m_{\text{tinh bột}} = \dfrac{1080}{4} = 270$ g
			\item Khối lượng protein: $m_{\text{protein}} = \dfrac{270}{4} = 67{,}5$ g
			\item Khối lượng chất béo: $m_{\text{béo}} = \dfrac{450}{9} = 50$ g
		\end{itemize}
	}
\end{bt}

%%%=============BT_2=============%%%
\begin{bt}%[6K6N1-4]
	Bảng sau cho biết thành phần dinh dưỡng trong $100$ g một số loại thực phẩm:
	\begin{center}
		\begin{tabular}{|c|c|c|c|}
			\hline
			\textbf{Thực phẩm} & \textbf{Tinh bột (g)} & \textbf{Protein (g)} & \textbf{Chất béo (g)} \\
			\hline
			Gạo & $78$ & $7$ & $1$ \\
			\hline
			Thịt gà & $0$ & $23$ & $4$ \\
			\hline
			Trứng gà & $1$ & $13$ & $11$ \\
			\hline
		\end{tabular}
	\end{center}
	Một bữa ăn gồm $200$ g cơm (từ gạo), $100$ g thịt gà và $1$ quả trứng gà ($50$ g).
	\begin{enumerate}
		\item Tính tổng khối lượng mỗi loại chất dinh dưỡng trong bữa ăn.
		\item Tính tổng năng lượng của bữa ăn.
	\end{enumerate}
	\loigiai{
		\textbf{1. Tổng khối lượng mỗi loại chất dinh dưỡng:}

		Từ $200$ g gạo:
		\begin{itemize}
			\item Tinh bột: $\dfrac{200 \times 78}{100} = 156$ g
			\item Protein: $\dfrac{200 \times 7}{100} = 14$ g
			\item Chất béo: $\dfrac{200 \times 1}{100} = 2$ g
		\end{itemize}

		Từ $100$ g thịt gà:
		\begin{itemize}
			\item Tinh bột: $0$ g
			\item Protein: $23$ g
			\item Chất béo: $4$ g
		\end{itemize}

		Từ $50$ g trứng:
		\begin{itemize}
			\item Tinh bột: $\dfrac{50 \times 1}{100} = 0{,}5$ g
			\item Protein: $\dfrac{50 \times 13}{100} = 6{,}5$ g
			\item Chất béo: $\dfrac{50 \times 11}{100} = 5{,}5$ g
		\end{itemize}

		Tổng cộng:
		\begin{itemize}
			\item Tinh bột: $156 + 0 + 0{,}5 = 156{,}5$ g
			\item Protein: $14 + 23 + 6{,}5 = 43{,}5$ g
			\item Chất béo: $2 + 4 + 5{,}5 = 11{,}5$ g
		\end{itemize}

		\textbf{2. Tổng năng lượng của bữa ăn:}
		\[ E = 156{,}5 \times 4 + 43{,}5 \times 4 + 11{,}5 \times 9 = 626 + 174 + 103{,}5 = 903{,}5 \text{ kcal} \]
	}
\end{bt}

%%%=============BT_3=============%%%
\begin{bt}%[6K6N1-4]
	Một gia đình có $5$ người chuẩn bị lương thực cho $1$ tháng ($30$ ngày). Biết:
	\begin{itemize}
		\item Mỗi người cần trung bình $300$ g gạo mỗi ngày
		\item Giá gạo là $15000$ đồng/kg
		\item Gia đình muốn dự trữ thêm $10\%$ lượng gạo phòng trường hợp khách đến chơi
	\end{itemize}
	\begin{enumerate}
		\item Tính lượng gạo cần mua cho gia đình trong $1$ tháng (đã bao gồm phần dự trữ).
		\item Tính số tiền cần chi để mua gạo.
		\item Nếu $1$ kg gạo cung cấp $3500$ kcal, tính tổng năng lượng mà lượng gạo trên cung cấp.
	\end{enumerate}
	\loigiai{
		\textbf{1. Lượng gạo cần mua:}

		Lượng gạo cần dùng trong 1 tháng:
		\[ m_{\text{cần}} = 5 \times 300 \times 30 = 45000 \text{ g} = 45 \text{ kg} \]

		Lượng gạo cần mua (bao gồm dự trữ $10\%$):
		\[ m_{\text{mua}} = 45 \times (1 + 10\%) = 45 \times 1{,}1 = 49{,}5 \text{ kg} \]

		\textbf{2. Số tiền cần chi:}
		\[ T = 49{,}5 \times 15000 = 742500 \text{ đồng} \]

		\textbf{3. Tổng năng lượng:}
		\[ E = 49{,}5 \times 3500 = 173250 \text{ kcal} \]
	}
\end{bt}

\end{document}
