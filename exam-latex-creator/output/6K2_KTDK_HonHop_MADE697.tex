\documentclass[FileMain.tex]{subfiles}
\gdef\sophong{Sở GD \& ĐT Gia Lai}
\gdef\truong{Trường THCS \& THPT}
\gdef\monhoc{Khoa học tự nhiên 6}
\gdef\ngaykt{04/02/2026}
\gdef\nh{2025 - 2026}
\gdef\thoigian{45}
\gdef\made{697}
\setcounter{section}{0}
\begin{document}
\section[Kiểm tra định kì - KHTN 6 - Mã đề \made]{Kiểm tra định kì}

%%%==============Phần trắc nghiệm nhiều lựa chọn==============%%%
\subsection{Bài tập trắc nghiệm nhiều lựa chọn}\textit{\large Thí sinh trả lời từ câu 1 đến câu 12. Mỗi câu thí sinh chỉ chọn một phương án}
\Opensolutionfile{ansex}[Ans/LGEX-6K2_HonHop_MADE697]
\Opensolutionfile{ans}[Ans/Ans-6K2_HonHop_MADE697]

%%%%%============EX_1================%%%%%%
\begin{ex}%[6K2NB1-1]
	Hỗn hợp gồm nhiều chất trộn lẫn với nhau. Trong hỗn hợp, mỗi chất
	\choice
	{mất đi tính chất ban đầu}
	{có tính chất mới hoàn toàn}
	{\True vẫn giữ nguyên tính chất của nó}
	{chỉ giữ một phần tính chất}
	\loigiai{
		Trong hỗn hợp, các chất chỉ trộn lẫn với nhau mà không xảy ra phản ứng hóa học, do đó mỗi chất vẫn giữ nguyên tính chất của nó.
	}
\end{ex}

%%%%%============EX_2================%%%%%%
\begin{ex}%[6K2NB1-1]
	Chất nào sau đây KHÔNG phải là chất tinh khiết?
	\choice
	{Nước cất}
	{Muối ăn tinh khiết}
	{\True Nước biển}
	{Đường kính}
	\loigiai{
		Nước biển là hỗn hợp vì chứa nước và các muối khoáng hòa tan. Các chất còn lại là chất tinh khiết.
	}
\end{ex}

%%%%%============EX_3================%%%%%%
\begin{ex}%[6K2TH1-2]
	Đặc điểm nào sau đây là của hỗn hợp đồng nhất?
	\choice
	{Nhìn thấy rõ ranh giới giữa các thành phần}
	{Các chất rắn lơ lửng trong chất lỏng}
	{Các chất lỏng tách thành từng lớp}
	{\True Các chất phân bố đều, không phân biệt được ranh giới}
	\loigiai{
		Hỗn hợp đồng nhất là hỗn hợp mà các chất phân bố đều trong toàn bộ hỗn hợp, không nhìn thấy ranh giới giữa các thành phần.
	}
\end{ex}

%%%%%============EX_4================%%%%%%
\begin{ex}%[6K2TH1-2]
	Hỗn hợp nào sau đây là hỗn hợp không đồng nhất?
	\choice
	{Giấm ăn}
	{Nước đường}
	{\True Dầu ăn lẫn nước}
	{Rượu pha nước}
	\loigiai{
		Dầu ăn không tan trong nước và có khối lượng riêng nhỏ hơn nước nên nổi lên trên, tạo thành hỗn hợp không đồng nhất (nhìn thấy rõ hai lớp).
	}
\end{ex}

%%%%%============EX_5================%%%%%%
\begin{ex}%[6K2NB2-1]
	Hỗn hợp nào sau đây thuộc loại huyền phù?
	\choice
	{Nước muối}
	{\True Nước sông có phù sa}
	{Dầu giấm}
	{Nước đường}
	\loigiai{
		Nước sông có phù sa là huyền phù vì chứa các hạt phù sa (chất rắn) lơ lửng trong nước.
	}
\end{ex}

%%%%%============EX_6================%%%%%%
\begin{ex}%[6K2TH2-2]
	Trong các hỗn hợp sau, hỗn hợp nào thuộc loại nhũ tương?
	\choice
	{Nước bột gạo}
	{Nước chanh đường}
	{\True Kem dưỡng da}
	{Nước mắm}
	\loigiai{
		Kem dưỡng da là nhũ tương vì gồm các giọt dầu/chất béo phân tán trong nước hoặc ngược lại, hai pha lỏng không tan vào nhau.
	}
\end{ex}

%%%%%============EX_7================%%%%%%
\begin{ex}%[6K2TH2-2]
	Điểm khác nhau cơ bản giữa huyền phù và nhũ tương là
	\choice
	{huyền phù là hỗn hợp đồng nhất, nhũ tương không đồng nhất}
	{huyền phù bền, nhũ tương không bền}
	{\True huyền phù gồm chất rắn trong chất lỏng, nhũ tương gồm chất lỏng trong chất lỏng}
	{huyền phù có màu, nhũ tương không có màu}
	\loigiai{
		Huyền phù gồm các hạt chất rắn lơ lửng trong chất lỏng, còn nhũ tương gồm các giọt chất lỏng phân tán trong chất lỏng khác.
	}
\end{ex}

%%%%%============EX_8================%%%%%%
\begin{ex}%[6K2NB3-1]
	Phương pháp nào sau đây dùng để tách chất rắn không tan ra khỏi chất lỏng?
	\choice
	{Cô cạn}
	{Chiết}
	{\True Lọc}
	{Chưng cất}
	\loigiai{
		Phương pháp lọc dùng để tách chất rắn không tan ra khỏi chất lỏng, dựa vào kích thước hạt.
	}
\end{ex}

%%%%%============EX_9================%%%%%%
\begin{ex}%[6K2NB3-1]
	Phương pháp nào sau đây dựa trên sự khác nhau về khối lượng riêng của hai chất lỏng không tan vào nhau?
	\choice
	{Lọc}
	{Cô cạn}
	{\True Chiết}
	{Chưng cất}
	\loigiai{
		Phương pháp chiết dùng để tách hai chất lỏng không tan vào nhau, dựa vào sự khác nhau về khối lượng riêng (chất nhẹ nổi lên trên).
	}
\end{ex}

%%%%%============EX_10================%%%%%%
\begin{ex}%[6K2TH3-2]
	Để tách rượu ra khỏi hỗn hợp rượu và nước, người ta dùng phương pháp
	\choice
	{lọc}
	{cô cạn}
	{chiết}
	{\True chưng cất}
	\loigiai{
		Rượu (ethanol) có nhiệt độ sôi $78{,}3^\circ$C, nước có nhiệt độ sôi $100^\circ$C. Dùng chưng cất để tách dựa vào sự khác nhau về nhiệt độ sôi.
	}
\end{ex}

%%%%%============EX_11================%%%%%%
\begin{ex}%[6K2VD3-3]
	Để tách hỗn hợp gồm mạt sắt và bột lưu huỳnh, phương pháp đơn giản nhất là dùng
	\choice
	{phương pháp lọc}
	{\True nam châm}
	{phương pháp chiết}
	{phương pháp cô cạn}
	\loigiai{
		Sắt có tính chất bị nam châm hút, lưu huỳnh không bị nam châm hút. Dùng nam châm để tách sắt ra khỏi lưu huỳnh là phương pháp đơn giản nhất.
	}
\end{ex}

%%%%%============EX_12================%%%%%%
\begin{ex}%[6K2VD3-4]
	Hỗn hợp gồm nước, cát và dầu ăn. Để tách riêng từng chất, ta cần thực hiện theo thứ tự
	\choice
	{Chiết $\rightarrow$ Cô cạn $\rightarrow$ Lọc}
	{Lọc $\rightarrow$ Cô cạn $\rightarrow$ Chiết}
	{Cô cạn $\rightarrow$ Lọc $\rightarrow$ Chiết}
	{\True Lọc $\rightarrow$ Chiết}
	\loigiai{
		Lọc để tách cát (chất rắn không tan), sau đó chiết để tách dầu ăn và nước (hai chất lỏng không tan vào nhau).
	}
\end{ex}

\Closesolutionfile{ans}
\Closesolutionfile{ansex}

%%%==============Phần trắc nghiệm đúng sai==============%%%
\subsection{Trắc nghiệm đúng sai}\textit{\large Thí sinh trả lời từ câu 1 đến câu 4. Trong mỗi ý a), b), c), d) ở mỗi câu thí sinh chọn đúng hoặc sai}
\Opensolutionfile{ansex}[Ans/LGTF-6K2_HonHop_MADE697]
\Opensolutionfile{ansbook}[Ansbook/AnsTF-6K2_HonHop_MADE697]
\Opensolutionfile{ans}[Ans/Tempt-6K2_HonHop_MADE697]
\setcounter{ex}{0}

%%%%%============TF_1================%%%%%%
\begin{ex}%[6K2TH1-2]
	Cho các phát biểu sau về chất tinh khiết và hỗn hợp:
	\choiceTF
	{\True Chất tinh khiết có nhiệt độ nóng chảy và nhiệt độ sôi xác định}
	{\True Hỗn hợp có thể gồm hai hay nhiều chất trộn lẫn với nhau}
	{Tất cả hỗn hợp đều có thể nhìn thấy ranh giới giữa các thành phần}
	{\True Nước cất là chất tinh khiết, nước khoáng là hỗn hợp}
	\loigiai{
		\begin{itemchoice}[T1,T2,F3,T4]
			\itemch Chất tinh khiết có thành phần xác định nên có nhiệt độ nóng chảy và nhiệt độ sôi xác định
			\itemch Định nghĩa của hỗn hợp là hai hay nhiều chất trộn lẫn với nhau
			\itemch Hỗn hợp đồng nhất không nhìn thấy ranh giới giữa các thành phần (như nước muối)
			\itemch Nước cất chỉ chứa $H_2O$ là chất tinh khiết, nước khoáng chứa các muối hòa tan là hỗn hợp
		\end{itemchoice}
	}
\end{ex}

%%%%%============TF_2================%%%%%%
\begin{ex}%[6K2TH2-3]
	Cho các phát biểu sau về dung dịch, huyền phù và nhũ tương:
	\choiceTF
	{\True Dung dịch là hỗn hợp đồng nhất của chất tan và dung môi}
	{Để phân biệt huyền phù và dung dịch, ta không thể dùng mắt thường}
	{\True Sữa tươi và mayonnaise đều là nhũ tương}
	{Nước bột sắn sau khi khuấy đều là dung dịch vì không nhìn thấy hạt bột}
	\loigiai{
		\begin{itemchoice}[T1,F2,T3,F4]
			\itemch Dung dịch gồm chất tan phân bố đều trong dung môi, tạo hỗn hợp đồng nhất
			\itemch Huyền phù có các hạt rắn lơ lửng, có thể nhìn thấy bằng mắt thường; dung dịch trong suốt
			\itemch Sữa tươi (giọt chất béo trong nước) và mayonnaise (giọt dầu trong giấm và trứng) đều là nhũ tương
			\itemch Nước bột sắn là huyền phù vì bột sắn không tan trong nước, để lâu sẽ lắng xuống
		\end{itemchoice}
	}
\end{ex}

%%%%%============TF_3================%%%%%%
\begin{ex}%[6K2VD3-3]
	Cho các phát biểu sau về phương pháp tách chất:
	\choiceTF
	{Phương pháp lọc có thể tách đường ra khỏi nước đường}
	{\True Phương pháp cô cạn dùng để tách chất rắn tan không bay hơi ra khỏi dung dịch}
	{\True Phương pháp chưng cất có thể tách được nước tinh khiết từ nước muối}
	{\True Phương pháp chiết dùng phễu chiết để tách hai chất lỏng không tan vào nhau}
	\loigiai{
		\begin{itemchoice}[F1,T2,T3,T4]
			\itemch Đường tan trong nước nên không thể dùng phương pháp lọc, phải dùng cô cạn
			\itemch Cô cạn làm bay hơi dung môi, chất rắn tan (không bay hơi) kết tinh lại
			\itemch Chưng cất nước muối: nước bay hơi rồi ngưng tụ thành nước cất, muối còn lại trong bình
			\itemch Phễu chiết có van để tách riêng hai lớp chất lỏng không tan vào nhau
		\end{itemchoice}
	}
\end{ex}

%%%%%============TF_4================%%%%%%
\begin{ex}%[6K2VD3-4]
	Cho các phát biểu sau về ứng dụng của phương pháp tách chất:
	\choiceTF
	{\True Máy lọc nước RO dùng để tách các chất hòa tan và vi khuẩn ra khỏi nước}
	{\True Xăng, dầu diesel, dầu hỏa được tách từ dầu mỏ bằng phương pháp chưng cất}
	{Sàng bột mì là ứng dụng của phương pháp chưng cất}
	{\True Vớt dầu tràn trên biển là ứng dụng của phương pháp chiết}
	\loigiai{
		\begin{itemchoice}[T1,T2,F3,T4]
			\itemch Máy lọc nước RO dùng màng lọc siêu nhỏ để tách các chất hòa tan và vi khuẩn
			\itemch Dầu mỏ được chưng cất ở các nhiệt độ khác nhau để thu các sản phẩm khác nhau
			\itemch Sàng bột mì là ứng dụng dựa vào kích thước hạt (tương tự phương pháp lọc), không phải chưng cất
			\itemch Dầu nổi trên mặt nước biển, vớt dầu tràn là ứng dụng của phương pháp chiết
		\end{itemchoice}
	}
\end{ex}

\Closesolutionfile{ans}
\Closesolutionfile{ansbook}
\Closesolutionfile{ansex}

%%%==============Phần bài tập trả lời ngắn==============%%%
\subsection{Bài tập trả lời ngắn}\textit{\large Thí sinh trả lời từ câu 1 đến câu 4}
\Opensolutionfile{ansex}[Ans/LGSA-6K2_HonHop_MADE697]
\Opensolutionfile{ansexh}[Ans/AnsSA-6K2_HonHop_MADE697]
\setcounter{ex}{0}

%%%%%============SA_1================%%%%%%
\begin{ex}%[6K2TH2-2]
	Hòa tan hoàn toàn $35$ g muối ăn vào $165$ g nước. Tính khối lượng (theo đơn vị gam) của dung dịch nước muối thu được.
	\shortans{$200$}
	\loigiai{
		Khối lượng dung dịch:
		\[ m_{\text{dd}} = m_{\text{muối}} + m_{\text{nước}} = 35 + 165 = 200 \text{ g} \]
	}
\end{ex}

%%%%%============SA_2================%%%%%%
\begin{ex}%[6K2VD2-3]
	Một dung dịch nước đường có khối lượng $250$ g và nồng độ $12\%$. Tính khối lượng đường (theo đơn vị gam) có trong dung dịch.
	\shortans{$30$}
	\loigiai{
		Khối lượng đường:
		\[ m_{\text{đường}} = \dfrac{C\% \times m_{\text{dd}}}{100\%} = \dfrac{12 \times 250}{100} = 30 \text{ g} \]
	}
\end{ex}

%%%%%============SA_3================%%%%%%
\begin{ex}%[6K2VD3-3]
	Hòa tan $40$ g đường vào nước được dung dịch có nồng độ $20\%$. Tính khối lượng (theo đơn vị gam) của dung dịch nước đường thu được.
	\shortans{$200$}
	\loigiai{
		Từ công thức nồng độ phần trăm:
		\[ C\% = \dfrac{m_{\text{đường}}}{m_{\text{dd}}} \times 100\% \]
		Suy ra:
		\[ m_{\text{dd}} = \dfrac{m_{\text{đường}} \times 100\%}{C\%} = \dfrac{40 \times 100}{20} = 200 \text{ g} \]
	}
\end{ex}

%%%%%============SA_4================%%%%%%
\begin{ex}%[6K2VD3-4]
	Trong không khí, carbon dioxide ($CO_2$) chiếm khoảng $0{,}04\%$ về thể tích. Tính thể tích khí $CO_2$ (theo đơn vị lít) có trong $10000$ lít không khí.
	\shortans{$4$}
	\loigiai{
		Thể tích khí $CO_2$:
		\[ V_{CO_2} = \dfrac{0{,}04\% \times 10000}{100\%} = \dfrac{0{,}04 \times 10000}{100} = 4 \text{ lít} \]
	}
\end{ex}

\Closesolutionfile{ansexh}
\Closesolutionfile{ansex}

%%%==============Phần bài tập tự luận==============%%%
\subsection{Bài tập tự luận}\textit{\large Thí sinh trả lời từ bài 1 đến bài 3}
\Opensolutionfile{ansbth}[Ans/LGBT-6K2_HonHop_MADE697]
\Opensolutionfile{ansbt}[Ans/AnsBT-6K2_HonHop_MADE697]

%%%%%============BT_1================%%%%%%
\begin{bt}%[6K2TH1-2]
	Cho các hỗn hợp sau: nước muối, nước bùn, sữa tươi, không khí, nước cam ép có tép.
	\begin{enumerate}
	\item Phân loại các hỗn hợp trên thành hỗn hợp đồng nhất và hỗn hợp không đồng nhất.
	\item Trong các hỗn hợp không đồng nhất, hãy chỉ ra đâu là huyền phù, đâu là nhũ tương.
	\end{enumerate}
	\loigiai{
		\begin{enumerate}
		\item \textbf{Phân loại hỗn hợp:}
		\begin{itemize}
		\item \textbf{Hỗn hợp đồng nhất:} Nước muối, không khí
		\item \textbf{Hỗn hợp không đồng nhất:} Nước bùn, sữa tươi, nước cam ép có tép
		\end{itemize}

		\item \textbf{Phân loại hỗn hợp không đồng nhất:}
		\begin{itemize}
		\item \textbf{Huyền phù:} Nước bùn (hạt bùn lơ lửng trong nước), nước cam ép có tép (tép cam lơ lửng trong nước cam)
		\item \textbf{Nhũ tương:} Sữa tươi (giọt chất béo phân tán trong nước)
		\end{itemize}
		\end{enumerate}
	}
\end{bt}

%%%%%============BT_2================%%%%%%
\begin{bt}%[6K2VD3-3]
	Nêu nguyên tắc và trình bày cách thực hiện phương pháp lọc. Cho hai ví dụ ứng dụng của phương pháp này trong đời sống hàng ngày.
	\loigiai{
		\textbf{Nguyên tắc:}
		\\
		Phương pháp lọc dựa trên sự khác nhau về kích thước hạt của các chất trong hỗn hợp. Chất có kích thước hạt lớn hơn lỗ lọc sẽ bị giữ lại trên màng lọc (hoặc giấy lọc), chất có kích thước hạt nhỏ hơn sẽ đi qua.

		\textbf{Cách thực hiện:}
		\begin{enumerate}
		\item Chuẩn bị phễu lọc, giấy lọc (hoặc vải lọc), bình hứng
		\item Gấp giấy lọc thành hình nón, đặt vào phễu lọc
		\item Làm ướt giấy lọc bằng nước để giấy dính chặt vào phễu
		\item Đổ hỗn hợp cần lọc vào phễu (đổ từ từ theo đũa thủy tinh)
		\item Chất rắn không tan được giữ lại trên giấy lọc, chất lỏng chảy qua vào bình hứng
		\end{enumerate}

		\textbf{Ví dụ ứng dụng:}
		\begin{enumerate}
		\item Lọc nước sinh hoạt bằng bình lọc gia đình để loại bỏ cặn bẩn
		\item Lọc cà phê bằng phin để tách bã cà phê ra khỏi nước cà phê
		\end{enumerate}
	}
\end{bt}

%%%%%============BT_3================%%%%%%
\begin{bt}%[6K2VC3-4]
	Một học sinh tiến hành thí nghiệm pha chế dung dịch nước đường như sau: Hòa tan $50$ g đường vào $200$ g nước, khuấy đều cho đường tan hết.
	\begin{enumerate}
	\item Tính nồng độ phần trăm của dung dịch nước đường thu được.
	\item Nếu muốn thu được $100$ g đường từ dung dịch trên, cần cô cạn bao nhiêu gam dung dịch?
	\item Giải thích tại sao khi cô cạn dung dịch nước đường ta thu được đường ở dạng rắn.
	\end{enumerate}
	\loigiai{
		\begin{enumerate}
		\item \textbf{Tính nồng độ phần trăm:}
		\\
		Khối lượng dung dịch:
		\[ m_{\text{dd}} = m_{\text{đường}} + m_{\text{nước}} = 50 + 200 = 250 \text{ g} \]
		Nồng độ phần trăm:
		\[ C\% = \dfrac{m_{\text{đường}}}{m_{\text{dd}}} \times 100\% = \dfrac{50}{250} \times 100\% = 20\% \]

		\item \textbf{Tính khối lượng dung dịch cần cô cạn:}
		\\
		Để thu được $100$ g đường, cần khối lượng dung dịch:
		\[ m_{\text{dd cần}} = \dfrac{m_{\text{đường}} \times 100\%}{C\%} = \dfrac{100 \times 100}{20} = 500 \text{ g} \]
		Tuy nhiên, dung dịch chỉ có $250$ g (chứa $50$ g đường), nên chỉ thu được tối đa $50$ g đường.
		\\
		\textit{Hoặc:} Nếu đề bài hỏi cần cô cạn bao nhiêu gam dung dịch $20\%$ để thu được $100$ g đường, thì cần $500$ g dung dịch.

		\item \textbf{Giải thích:}
		\\
		Khi cô cạn dung dịch nước đường, nước bay hơi (vì nước có nhiệt độ sôi $100^\circ$C), còn đường không bay hơi (đường có nhiệt độ nóng chảy cao và không bay hơi ở nhiệt độ thường). Do đó, sau khi nước bay hơi hết, đường kết tinh lại ở dạng rắn.
		\end{enumerate}
	}
\end{bt}

\Closesolutionfile{ansbt}
\Closesolutionfile{ansbth}

\begin{center}
 \rule[4pt]{2cm}{1pt}\,\large\bfseries Hết\,\rule[4pt]{2cm}{1pt}
\end{center}
\label{x}
\end{document}
