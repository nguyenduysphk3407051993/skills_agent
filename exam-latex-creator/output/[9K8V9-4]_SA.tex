%%%=============SA_1=============%%%
\begin{ex}%[9K8V9-4]
	Cho $4{,}6$ gam ethylic alcohol tác dụng hết với kim loại Sodium ($Na$) dư. Tính thể tích khí Hydrogen ($H_2$) thu được ở điều kiện chuẩn ($25^\circ C$, $1$ bar). Biết $M_{C_2H_5OH} = 46$ g/mol. (Kết quả làm tròn đến 2 chữ số thập phân).
	\shortans{$1{,}24$}
	\loigiai{
		$n_{C_2H_5OH} = \dfrac{4{,}6}{46} = 0{,}1$ (mol).
		\\
		Phương trình hóa học: $2C_2H_5OH + 2Na \xrightarrow{} 2C_2H_5ONa + H_2$.
		\\
		Theo PTHH: $n_{H_2} = \dfrac{1}{2} n_{C_2H_5OH} = \dfrac{1}{2} \times 0{,}1 = 0{,}05$ (mol).
		\\
		Thể tích khí $H_2$ ở điều kiện chuẩn:
		$V_{H_2} = 0{,}05 \times 24{,}79 = 1{,}2395 \approx 1{,}24$ (lít).
	}
\end{ex}

%%%=============SA_2=============%%%
\begin{ex}%[9K8V9-4]
	Một chai rượu vang có dung tích $750$ ml và độ cồn là $14^\circ$. Tính thể tích ethylic alcohol nguyên chất có trong chai rượu đó (đơn vị ml).
	\shortans{$105$}
	\loigiai{
		Độ cồn $14^\circ$ nghĩa là trong $100$ ml dung dịch có $14$ ml $C_2H_5OH$.
		\\
		Vậy trong $750$ ml dung dịch có lượng $C_2H_5OH$ là:
		$V_{C_2H_5OH} = \dfrac{750 \times 14}{100} = 105$ (ml).
	}
\end{ex}

%%%=============SA_3=============%%%
\begin{ex}%[9K8V9-4]
	Đốt cháy hoàn toàn $1$ mol ethylic alcohol cần dùng vừa đủ bao nhiêu lít không khí (đkc)? Biết khí oxygen chiếm $20\%$ thể tích không khí. (Lấy thể tích mol khí ở đkc là $24{,}79$ lít, kết quả làm tròn đến hàng đơn vị).
	\shortans{$372$}
	\loigiai{
		Phương trình cháy: $C_2H_5OH + 3O_2 \xrightarrow{t^\circ} 2CO_2 + 3H_2O$.
		\\
		Theo PTHH: $n_{O_2} = 3 \times n_{C_2H_5OH} = 3 \times 1 = 3$ (mol).
		\\
		Thể tích $O_2$ (đkc): $V_{O_2} = 3 \times 24{,}79 = 74{,}37$ (lít).
		\\
		Thể tích không khí: $V_{KK} = 5 \times V_{O_2} = 5 \times 74{,}37 = 371{,}85 \approx 372$ (lít).
	}
\end{ex}
