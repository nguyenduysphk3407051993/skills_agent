\documentclass[FileMain.tex]{subfiles}
\gdef\sophong{Sở GD \& ĐT Gia Lai} 
\gdef\truong{Trường THPT Chi Lăng} 
\gdef\truongh{Trường Mầm non, THCS, THPT Sao Việt} 
\gdef\monhoc{Khoa học tự nhiên 6} 
\gdef\ngaykt{04/02/2026} 
\gdef\nh{2025 - 2026} 
\gdef\thoigian{45 phút}
\gdef\made{825} 
\setcounter{section}{0}
%\tatloigiai
%\hienthiloigiai
%\dongkeloigiai
\begin{document}
\section[Truy bài định kì - Mã đề \made]{Truy bài định kì} 
%\Tieudegiua{Kiểm tra chủ đề Lương thực - Thực phẩm - Mã đề \made}

%%%==============Phần trắc nghiệm nhiều lựa chọn==============%%% 
\subsection{Bài tập trắc nghiệm nhiều lựa chọn}\textit{\large Thí sinh trả lời từ câu 1 đến 12. Mỗi câu thí sinh chỉ chọn một phương án}
\Opensolutionfile{ansex}[Ans/LGEX-LTTP_KHTN6_825_MADE825]
\Opensolutionfile{ans}[Ans/Ans-LTTP_KHTN6_825_MADE825]

%%%============EX_1=============%%%
\begin{ex}%[6K3N4-1]
	Lương thực là các loại thực phẩm chứa hàm lượng lớn chất nào sau đây?
	\choice
	{Protein}
	{\True Tinh bột}
	{Lipid}
	{Vitamin}
	\loigiai{
		Lương thực (gạo, ngô, khoai, sắn\dots) chứa hàm lượng lớn tinh bột, cung cấp năng lượng chính cho cơ thể.
	}
\end{ex}

%%%============EX_2=============%%%
\begin{ex}%[6K3N4-2]
	Loại thực phẩm nào sau đây chứa nhiều chất đạm (protein)?
	\choice
	{{\True Thịt bò}}
	{Rau muống}
	{Mỡ lợn}
	{Gạo tẻ}
	\loigiai{
		Thịt bò chứa nhiều chất đạm (protein). Rau muống chứa nhiều chất xơ, vitamin. Mỡ lợn chứa nhiều chất béo (lipid). Gạo tẻ chứa nhiều tinh bột.
	}
\end{ex}

%%%============EX_3=============%%%
\begin{ex}%[6K3H4-3]
	Dấu hiệu nào sau đây cho thấy thực phẩm đã bị hỏng?
	\choice
	{Thịt tươi có độ đàn hồi cao}
	{Cá tươi có mang đỏ, mắt trong}
	{Rau xanh tươi, không bị héo}
	{{\True Bánh mì xuất hiện các đốm xanh, mốc}}
	\loigiai{
		Bánh mì xuất hiện các đốm xanh, mốc là dấu hiệu của nấm mốc phát triển, cho thấy thực phẩm đã bị hỏng.
	}
\end{ex}

%%%============EX_4=============%%%
\begin{ex}%[6K3V4-4]
	Để bảo quản gạo được lâu, người ta thường dùng phương pháp nào sau đây?
	\choice
	{Đông lạnh}
	{{\True Sấy khô, để nơi khô ráo}}
	{Ngâm nước}
	{Muối chua}
	\loigiai{
		Gạo là lương thực khô, cần bảo quản bằng cách sấy khô và để nơi thoáng mát, tránh ẩm mốc để vi khuẩn và nấm không phát triển.
	}
\end{ex}

%%%============EX_5=============%%%
\begin{ex}%[6K3N4-1]
	Nhóm thực phẩm nào sau đây thuộc nhóm giàu chất béo?
	\choice
	{Gạo, ngô, khoai}
	{Thịt, cá, trứng}
	{{\True Mỡ động vật, dầu thực vật}}
	{Rau xanh, hoa quả}
	\loigiai{
		Mỡ động vật và dầu thực vật là nguồn cung cấp chất béo (lipid) dồi dào.
	}
\end{ex}

%%%============EX_6=============%%%
\begin{ex}%[6K3H4-2]
	Tại sao không nên ăn thực phẩm đã bị ôi thiu?
	\choice
	{Vì nó có mùi thơm quá mức}
	{Vì nó cung cấp nhiều năng lượng hơn}
	{{\True Vì nó chứa vi khuẩn và độc tố gây hại cho sức khỏe}}
	{Vì nó giúp tiêu hóa tốt hơn}
	\loigiai{
		Thực phẩm ôi thiu chứa nhiều vi sinh vật gây bệnh và các độc tố do chúng tiết ra, ăn vào dễ gây ngộ độc thực phẩm, đau bụng, tiêu chảy\dots
	}
\end{ex}

%%%============EX_7=============%%%
\begin{ex}%[6K3V4-3]
	Cây lúa gạo là nguồn lương thực chính của người dân khu vực nào?
	\choice
	{Châu Âu}
	{{\True Châu Á}}
	{Châu Phi}
	{Châu Mỹ}
	\loigiai{
		Lúa gạo là cây lương thực chính, quan trọng nhất của người dân khu vực Châu Á, đặc biệt là Đông Nam Á.
	}
\end{ex}

%%%============EX_8=============%%%
\begin{ex}%[6K3N4-4]
	Trong các loại củ sau, loại nào **không** được xem là lương thực chính?
	\choice
	{Khoai lang}
	{Sắn (mì)}
	{Khoai tây}
	{{\True Cà rốt}}
	\loigiai{
		Khoai lang, sắn, khoai tây chứa nhiều tinh bột, thường dùng làm lương thực. Cà rốt chủ yếu cung cấp vitamin và chất xơ, được xem là rau củ (thực phẩm), không phải lương thực chính.
	}
\end{ex}

%%%============EX_9=============%%%
\begin{ex}%[6K3H4-1]
	Chất bột đường (Carbohydrate) có vai trò chính là gì đối với cơ thể?
	\choice
	{{\True Cung cấp năng lượng cho các hoạt động sống}}
	{Xây dựng tế bào và cơ thể}
	{Hòa tan các vitamin}
	{Giúp cơ thể chống lại bệnh tật}
	\loigiai{
		Chất bột đường (tinh bột, đường) có vai trò chính là cung cấp năng lượng cho mọi hoạt động của cơ thể.
	}
\end{ex}

%%%============EX_10=============%%%
\begin{ex}%[6K3V4-2]
	Khi mua thực phẩm đóng hộp, điều quan trọng nhất cần kiểm tra trên bao bì là gì?
	\choice
	{Hình ảnh quảng cáo}
	{Màu sắc bao bì}
	{{\True Hạn sử dụng}}
	{Tên người bán}
	\loigiai{
		Hạn sử dụng (HSD) là thông tin quan trọng nhất để biết thực phẩm còn an toàn để sử dụng hay không.
	}
\end{ex}

%%%============EX_11=============%%%
\begin{ex}%[6K3N4-3]
	Loại thực phẩm nào sau đây cần bảo quản ở ngăn đá tủ lạnh để giữ được lâu nhất?
	\choice
	{Rau xà lách}
	{Trứng gà}
	{{\True Thịt tươi sống}}
	{Sữa tươi tiệt trùng chưa mở nắp}
	\loigiai{
		Thịt tươi sống dễ bị vi khuẩn phân hủy, cần bảo quản đông lạnh (ngăn đá) để ức chế vi khuẩn và giữ được lâu. Rau, trứng, sữa thường để ngăn mát hoặc nhiệt độ thường (sữa chưa mở).
	}
\end{ex}

%%%============EX_12=============%%%
\begin{ex}%[6K3H4-4]
	Thức ăn nào sau đây chứa nhiều Vitamin C nhất?
	\choice
	{Thịt gà}
	{Cơm trắng}
	{{\True Quả cam, quả ổi}}
	{Dầu ăn}
	\loigiai{
		Các loại trái cây có múi (cam, bưởi) và ổi chứa hàm lượng Vitamin C rất cao.
	}
\end{ex}

\Closesolutionfile{ans}
\Closesolutionfile{ansex}
%\bangdapan{Ans-LTTP_KHTN6_825_MADE825}

%%%==============Phần trắc nghiệm đúng sai==============%%% 
\subsection{Trắc nghiệm đúng sai}\textit{\large Thí sinh trả lời từ câu 1 đến câu 4. Trong mỗi ý a), b), c), d) ở mỗi câu thí sinh chọn đúng hoặc sai}
\Opensolutionfile{ansex}[Ans/LGTF-LTTP_KHTN6_825_MADE825]
\Opensolutionfile{ansbook}[Ansbook/AnsTF-LTTP_KHTN6_825_MADE825]
\Opensolutionfile{ans}[Ans/Tempt-LTTP_KHTN6_825_MADE825]
\setcounter{ex}{0}

%%%=============TF_1=============%%%
\begin{ex}%[6K3H4-1]
	Nhận định về vai trò của các nhóm chất dinh dưỡng trong thực phẩm:
	\choiceTF
	{\True Chất đạm (Protein) giúp xây dựng, tạo ra các tế bào mới để cơ thể lớn lên.}
	{Chất béo (Lipid) không cung cấp năng lượng mà chỉ giúp hòa tan vitamin.}
	{\True Vitamin và chất khoáng giúp tăng cường sức đề kháng, bảo vệ cơ thể khỏi bệnh tật.}
	{Chất bột đường (Carbohydrate) chủ yếu giúp cơ thể phát triển chiều cao.}
	\loigiai{
		\begin{itemchoice}[T1,F2,T3,F4]
			\itemch Protein là thành phần chính cấu tạo nên tế bào và cơ thể.
			\itemch Chất béo cung cấp năng lượng rất lớn (gấp đôi chất bột đường) và giúp hòa tan vitamin tan trong dầu (A, D, E, K).
			\itemch Vitamin và khoáng chất dù cần lượng nhỏ nhưng rất quan trọng cho hệ miễn dịch và chuyển hóa.
			\itemch Chất bột đường chủ yếu cung cấp năng lượng, còn phát triển chiều cao chủ yếu nhờ Canxi (khoáng chất) và Protein.
		\end{itemchoice}
	}
\end{ex}

%%%=============TF_2=============%%%
\begin{ex}%[6K3V4-2]
	Về cách bảo quản lương thực, thực phẩm:
	\choiceTF
	{\True Phơi khô lúa, ngô sau khi thu hoạch để giảm độ ẩm, tránh nấm mốc.}
	{Có thể dùng hóa chất không rõ nguồn gốc để giữ trái cây tươi lâu hơn.}
	{\True Hút chân không là phương pháp loại bỏ không khí để hạn chế vi khuẩn phát triển.}
	{Thực phẩm đã nấu chín có thể để ở nhiệt độ phòng qua đêm mà không cần đậy đệm.}
	\loigiai{
		\begin{itemchoice}[T1,F2,T3,F4]
			\itemch Vi khuẩn và nấm mốc cần độ ẩm để phát triển, làm khô giúp hạn chế chúng.
			\itemch Không được dùng hóa chất cấm, không rõ nguồn gốc vì gây hại cho sức khỏe.
			\itemch Loại bỏ oxy giúp ngăn chặn quá trình oxy hóa và sự phát triển của vi sinh vật hiếu khí.
			\itemch Thực phẩm chín để bên ngoài dễ bị ôi thiu và ruồi nhặng đậu vào, cần bảo quản lạnh hoặc đậy kín.
		\end{itemchoice}
	}
\end{ex}

%%%=============TF_3=============%%%
\begin{ex}%[6K3N4-3]
	Các phát biểu về ngộ độc thực phẩm:
	\choiceTF
	{\True Nôn mửa, đau bụng, tiêu chảy là các triệu chứng thường gặp của ngộ độc thực phẩm.}
	{Chỉ có thực phẩm ôi thiu mới gây ngộ độc, thực phẩm chứa hóa chất thì không.}
	{\True Ăn khoai tây mọc mầm có thể gây ngộ độc do chứa độc tố solanine.}
	{Cá nóc là loại thực phẩm an toàn, không chứa độc tố tự nhiên.}
	\loigiai{
		\begin{itemchoice}[T1,F2,T3,F4]
			\itemch Đây là các phản ứng của cơ thể để đào thải độc tố.
			\itemch Thực phẩm chứa hóa chất độc hại (thuốc trừ sâu, kim loại nặng) cũng gây ngộ độc.
			\itemch Mầm khoai tây chứa solanine rất độc.
			\itemch Cá nóc chứa độc tố tetrodotoxin cực mạnh, có thể gây tử vong nếu không chế biến đúng cách.
		\end{itemchoice}
	}
\end{ex}

%%%=============TF_4=============%%%
\begin{ex}%[6K3H4-4]
	Về nguồn gốc của lương thực, thực phẩm:
	\choiceTF
	{Lương thực như gạo, ngô, khoai đều có nguồn gốc từ động vật.}
	{\True Thực phẩm có thể có nguồn gốc từ thực vật hoặc động vật.}
	{\True Sữa, trứng, mật ong là thực phẩm có nguồn gốc động vật.}
	{Nước giải khát có ga được xem là thực phẩm tự nhiên tốt cho sức khỏe.}
	\loigiai{
		\begin{itemchoice}[F1,T2,T3,F4]
			\itemch Lương thực (gạo, ngô\dots) có nguồn gốc từ thực vật.
			\itemch Ví dụ rau (thực vật), thịt (động vật).
			\itemch Chúng được lấy từ bò, gà, ong.
			\itemch Nước ngọt có ga là thực phẩm chế biến công nghiệp, chứa nhiều đường và phẩm màu, không tốt nếu uống nhiều.
		\end{itemchoice}
	}
\end{ex}

\Closesolutionfile{ans}
\Closesolutionfile{ansbook}
\Closesolutionfile{ansex}
%\bangdapanTF{AnsTF-LTTP_KHTN6_825_MADE825}

%%==============Phần bài tập trả lời ngắn==============%%% 
\subsection{Bài tập trả lời ngắn}\textit{\large Thí sinh trả lời từ câu 1 đến câu 4}
\Opensolutionfile{ansex}[Ans/LGSA-LTTP_KHTN6_825_MADE825]
\Opensolutionfile{ansexh}[Ans/AnsSA-LTTP_KHTN6_825_MADE825]
\setcounter{ex}{0}

%%%=============SA_1=============%%%
\begin{ex}%[6K3V4-1]
	Nhiệt độ (độ C) tối ưu để bảo quản thực phẩm trong ngăn mát tủ lạnh thường là bao nhiêu (viết một con số cụ thể trung bình)?
	\shortans{$4$}
	\loigiai{
		Nhiệt độ ngăn mát tủ lạnh thường duy trì ở mức 1 đến 5 độ C, lý tưởng nhất là khoảng 4 độ C để làm chậm sự phát triển của vi khuẩn.
	}
\end{ex}

%%%=============SA_2=============%%%
\begin{ex}%[6K3V4-2]
	Một gam chất béo (Lipid) khi oxy hóa hoàn toàn trong cơ thể cung cấp khoảng 9,3 kcal năng lượng. Một gam chất bột đường (Carbohydrate) cung cấp khoảng 4,1 kcal. Một người ăn 20 gam chất béo thì năng lượng nhận được tương đương với ăn bao nhiêu gam chất bột đường? (Làm tròn kết quả đến hàng đơn vị).
	\shortans{$45$}
	\loigiai{
		Năng lượng từ 20g chất béo: $20 \times 9{,}3 = 186$ kcal.
		Lượng chất bột đường cần thiết: $186 : 4{,}1 \approx 45{,}36$ gam.
		Làm tròn: 45 gam.
	}
\end{ex}

%%%=============SA_3=============%%%
\begin{ex}%[6K3H4-3]
	Có bao nhiêu nhóm chất dinh dưỡng chính cần thiết cho cơ thể con người?
	\shortans{$4$}
	\loigiai{
		Có 4 nhóm chất dinh dưỡng chính: Chất bột đường (Carbohydrate), Chất đạm (Protein), Chất béo (Lipid), Vitamin và khoáng chất.
	}
\end{ex}

%%%=============SA_4=============%%%
\begin{ex}%[6K3N4-4]
	Trong tháp dinh dưỡng cân đối dành cho người trưởng thành, nhóm thực phẩm nào cần ăn hạn chế nhất (ở đỉnh tháp)? (Nhập 1: Lương thực, 2: Rau quả, 3: Thịt cá, 4: Muối).
	\shortans{$4$}
	\loigiai{
		Ở đỉnh tháp dinh dưỡng là nhóm Muối, cần ăn hạn chế (dưới 5g/ngày).
	}
\end{ex}

\Closesolutionfile{ansexh}
\Closesolutionfile{ansex}
%\bangdapanSA{AnsSA-LTTP_KHTN6_825_MADE825}

%%%==============Phần bài tập tự luận==============%%% 
\subsection{Bài tập tự luận}\textit{\large Thí sinh trả lời từ bài 1 đến bài 3}
\Opensolutionfile{ansbth}[Ans/LGBT-LTTP_KHTN6_825_MADE825]
\Opensolutionfile{ansbt}[Ans/AnsBT-LTTP_KHTN6_825_MADE825]

%%%=============BT_1=============%%%
\begin{bt}%[6K3H4-1]
	Tại sao ta cần phải ăn nhiều loại thức ăn khác nhau trong các bữa ăn hàng ngày mà không chỉ ăn một loại yêu thích?
	\loigiai{
		Chúng ta cần ăn đa dạng các loại thức ăn vì:
		\begin{enumerate}
			\item Mọi loại thức ăn đều không chứa đầy đủ tất cả các chất dinh dưỡng cần thiết cho cơ thể.
			\item Ăn đa dạng giúp bổ sung đầy đủ 4 nhóm chất: bột đường, đạm, béo, vitamin và khoáng chất.
			\item Giúp thay đổi khẩu vị, ăn ngon miệng hơn và hệ tiêu hóa hoạt động hiệu quả hơn.
			\item Tránh được tình trạng thừa chất này nhưng thiếu chất khác, gây mất cân bằng dinh dưỡng.
		\end{enumerate}
	}
\end{bt}

%%%=============BT_2=============%%%
\begin{bt}%[6K3V4-2]
	Em hãy đề xuất 3 biện pháp cụ thể để bảo quản thực phẩm an toàn tại gia đình em và giải thích ngắn gọn vì sao biện pháp đó hiệu quả.
	\loigiai{
		3 biện pháp bảo quản thực phẩm:
		\begin{enumerate}
			\item **Bảo quản lạnh (Tủ lạnh):** Nhiệt độ thấp ức chế sự sinh trưởng của vi khuẩn và nấm mốc. Dùng cho thịt, cá, rau, sữa.
			\item **Phơi khô hoặc sấy khô:** Loại bỏ nước trong thực phẩm, làm cho vi sinh vật không có điều kiện phát triển. Dùng cho lúa, ngô, cá khô, tôm khô.
			\item **Muối chua (Lên men):** Tạo môi trường axit (pH thấp) ức chế vi khuẩn gây thối. Dùng cho dưa cải, cà pháo.
		\end{enumerate}
	}
\end{bt}

%%%=============BT_3=============%%%
\begin{bt}%[6K3C4-3]
	Một học sinh có thói quen không ăn rau xanh và chỉ thích ăn thịt chiên rán. Hãy phân tích tác hại của thói quen này đối với sức khỏe và đưa ra lời khuyên cho bạn học sinh đó.
	\loigiai{
		**Phân tích tác hại:**
		\begin{enumerate}
			\item Không ăn rau xanh: Cơ thể sẽ thiếu Vitamin, khoáng chất và chất xơ. Thiếu chất xơ dễ gây táo bón, khó tiêu. Thiếu vitamin làm giảm sức đề kháng.
			\item Chỉ ăn thịt chiên rán: Nạp quá nhiều chất béo và protein nhưng thiếu cân bằng. Dễ dẫn đến béo phì, tim mạch, khó tiêu hóa. Chiên rán ở nhiệt độ cao có thể sinh ra chất độc hại.
		\end{enumerate}
		**Lời khuyên:**
		\begin{enumerate}
			\item Cần tập ăn rau xanh và trái cây mỗi ngày.
			\item Giảm bớt đồ chiên rán, thay bằng luộc, hấp.
			\item Ăn uống cân đối 4 nhóm chất để cơ thể khỏe mạnh và phát triển chiều cao, trí tuệ tốt nhất.
		\end{enumerate}
	}
\end{bt}

\Closesolutionfile{ansbt}
\Closesolutionfile{ansbth}

\begin{center}
 \rule[4pt]{2cm}{1pt}\,\large\bfseries Hết\,\rule[4pt]{2cm}{1pt}
\end{center}
\label{x}
\end{document}
