\documentclass[border=3pt,tikz]{standalone}
\usepackage[utf8]{vietnam}
\usetikzlibrary{calc,angles,intersections,shapes.geometric,arrows,decorations.markings,arrows.meta,patterns.meta,patterns}
\usepackage{tikz-3dplot,pgfplots}
\pgfplotsset{compat=1.18}
\usepgfplotslibrary{polar}
\usepackage{amsmath}
\usepackage{etoolbox}
\usepackage{xcolor} 
\definecolor{dnvang}{HTML}{994D1C}
\definecolor{dnxanh}{HTML}{0766AD}
\definecolor{dnxanhdam}{HTML}{19376D}
\definecolor{dndo}{HTML}{BB2649}
\def\mycolor{dnvang}
\def\mauphu{dnxanh}
\def\maudam{dnxanhdam}
\def\maunhan{dndo}
\begin{document}
	\begin{tikzpicture}[line join=round,line cap=round,declare function={d=0.2cm;}]
		\pgfmathsetmacro{\hsa}{1.5}
		\pgfmathsetmacro{\hsb}{4}
		%%% Vẽ 2 truc tọa độ
		\draw [thick,-latex,\maunhan] (0,0)coordinate (xw)--(11,0)coordinate (xe) node 	[below]{(m/z)};
		\draw [thick,-latex,\maunhan] (0,0)coordinate (ys)--(0,5.5)coordinate (yn) 	node[right]{(\%)};
		
		%%% Các giá trị truc y
		\foreach \y[evaluate =\y as \yt using int(\y*20)] in {1,...,5}{
			\draw[\maunhan!80!black] (0.1,\y)--+(180:0.2) node 		[left,font=\small\bfseries]{\yt};
		}
		%%% Các giá trị truc x
		\foreach \x [evaluate =\x as \xt using int(\x+\hsb)] in {2,3}{
			\draw[\maunhan!80!black] (\hsa*\x,0.1)--+(-90:.2) node 		[below,font=\small\bfseries]{\xt};
		}
		\foreach \x [evaluate =\x as \xt using int(\x+\hsb)] in {1,4,5,6}{
			\draw[\maunhan!80!black] (\hsa*\x,0.1)--+(-90:.2) node 		[below,font=\small\bfseries]{};
		}
		%%% Vẽ cột 
		\foreach \x/\y[evaluate =\y as \yt using \y*20] in {
			2/0.375,3/4.625
		}{
			\path[fill=\maunhan!80] ([xshift=-d]\hsa*\x,0) rectangle +({2*d},\y) node 	[above,xshift=-d,font=\bfseries\color{\maunhan!60!black}] {\pgfmathprintnumber[fixed, precision=2,use comma]{\yt}\%};
		}
		%%% Hiển thị thông tin trục y
		\path ([xshift=-1.2cm]ys)--([xshift=-1.2cm]yn) node 	[pos=0.5,midway,sloped,font =\scriptsize \color{\maunhan!50!black}\bfseries\sffamily]{Phần trăm số nguyên tử đồng vị};
		%%% Hiển thị thông tin trục x
		\path ([yshift=-0.9cm]xw)--([yshift=-0.9cm]xe) node 	[pos=0.5,midway,sloped,font 	=\scriptsize\color{\maunhan!50!black}\bfseries\sffamily]{Tỉ số nguyên tử khối điện tích};
	\end{tikzpicture}
\end{document}