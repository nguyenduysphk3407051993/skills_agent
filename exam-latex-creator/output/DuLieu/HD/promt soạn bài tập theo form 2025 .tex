## =========  promt trắc nghiệm nhiều lựa chọn==============##
Với vai trò là giáo viên bộ môm hóa học hãy giúp tôi tạo ra 10 bài tập trắc nghiệm 4 phương án trong đó có 1 phương án đúng với nội dung tương tự như form latex tôi đưa ra bên dưới nhưng nội dung phải khác nhau , và nhớ kèm theo lời giải chi tiết trong phần lệnh \loigiai. 
Chú ý :
- Nội dung  phải soạn theo cú pháp  Latex. 
- Phương án đúng phải đặt lệnh \True như ví dụ này .
\begin{ex}
	Nội dung câu hỏi trắc nghiệm 	
	\choice
	{Nội dung phương án sai}
	{\True Nội dung Phuong án đúng}
	{Nội dung phương án sai}
	{Nội dung Phuong án sai}
	\loigiai{Nội dung lời giải chi tiết}
\end{ex}
- Các công thức phải đặt trong cặp $$
- Phương án đúng phải xáo trộn vị trí
- Phải theo đúng ý tưởng mà tôi cung cấp sau đây:
Ý tưởng 1:
Nội dung câu hỏi: Cho kí hiệu nguyên tố ví dụ $_{17}^{35}Cl ở lấy các nguyên tố phổ biến nhóm A như Cl, F, O, Si, P, S và các nguyên tố nhóm B như Cu, Ag, Fe, Mn, Cr, Zn
Các phương án đúng sai:
\choiceTF[t]
	{Nội dung phương án sai}
	{\True Nội dung Phuong án đúng}
	{Nội dung phương án sai}
	{\True Nội dung Phuong án đúng}
	Phương án 1: đưa ra kiến thức về số electron trên 1 phân lớp s hay p hay d
	Phương án 2: đưa ra thông tin về cấu hình electron ví dụ $1s^22s^22p^63s^23p^5$ gây nhầm lẫn cho học sinh
	Phương án 3: đưa ra thông tin về tính chất của nguyên tố đó phi kim hay kim loại
	Phương án 4: đưa ra thông tin về khối nguyên tố s,p,d,f
Lời giải theo form này
\loigiai{
\begin{enumerate}
 	\item Nội dung lời giải phương án 1
	\item Nội dung lời giải phương án 2
	\item Nội dung lời giải phương án 3
	\item Nội dung lời giải phương án 4
\end{enumerate}
}










## =========  promt trắc nghiệm đúng sai==============##
Với vai trò là giáo viên bộ môm hóa học hãy giúp tôi tạo ra 10 bài tập trắc nghiệm dạng đúng sai với nội dung tương tự như form latex tôi đưa ra bên dưới nhưng nội dung phải khác nhau , và nhớ kèm theo lời giải chi tiết trong phần lệnh \loigiai. 
Chú ý :
- Nội dung  phải soạn theo cú pháp  Latex. 
- Phương án đúng phải đặt lệnh \True như ví dụ này .
\begin{ex}
	Nội dung câu hỏi trắc nghiệm đúng sai
	\choiceTF[t]
	{Nội dung phương án sai}
	{\True Nội dung Phuong án đúng}
	{Nội dung phương án sai}
	{\True Nội dung Phuong án đúng}
	\loigiai{Nội dung lời giải chi tiết}
\end{ex}
- Các công thức phải đặt trong cặp $$
- Số câu đúng trong một câu hởi không được bằng 1, có thể 2,3,4 
- Phải theo đúng ý tưởng mà tôi cung cấp sau đây:
Ý tưởng 1:
Nội dung câu hỏi: Cho kí hiệu nguyên tố ví dụ $_{17}^{35}Cl ở lấy các nguyên tố phổ biến nhóm A như Cl, F, O, Si, P, S và các nguyên tố nhóm B như Cu, Ag, Fe, Mn, Cr, Zn
Các phương án đúng sai:
\choiceTF[t]
	{Nội dung phương án sai}
	{\True Nội dung Phuong án đúng}
	{Nội dung phương án sai}
	{\True Nội dung Phuong án đúng}
	Phương án 1: đưa ra kiến thức về số electron trên 1 phân lớp s hay p hay d
	Phương án 2: đưa ra thông tin về cấu hình electron ví dụ $1s^22s^22p^63s^23p^5$ gây nhầm lẫn cho học sinh
	Phương án 3: đưa ra thông tin về tính chất của nguyên tố đó phi kim hay kim loại
	Phương án 4: đưa ra thông tin về khối nguyên tố s,p,d,f
Lời giải theo form này
\loigiai{
\begin{itemchoice}[T1,F2,F3,T4]%nếu phương án 1, 4 đúng, Phương án 2,3 sai
 	\itemch Nội dung lời giải phương án 1
	\itemch Nội dung lời giải phương án 2
	\itemch Nội dung lời giải phương án 3
	\itemch Nội dung lời giải phương án 4
\end{itemchoice}
}



- Ý tưởng 2 :
Nội dung câu hỏi: Cho kí hiệu nguyên tố ví dụ $_{17}^{35}Cl
Hãy chọn các phát biểu đúng , sai trong các phát biểu sau?
\choiceTF[t]
	{Nội dung phương án sai}
	{\True Nội dung Phuong án đúng}
	{Nội dung phương án sai}
	{\True Nội dung Phuong án đúng}
	Phương án 1: đưa ra kiến thức tính chat của nguyên tố kim loại hay phi kim
	Phương án 2: đưa ra thông tin về số electron có trong lớp K hoặc L hoặc M	
	Phương án 3: đưa ra thông tin về cấu hình e
	Phương án 4: đưa ra thông tin về eletron có mức năng lượng cao nhất điền vào phân lớp s

- Ý tưởng 3 :
Nội dung câu hỏi: Cho kí hiệu nguyên tố ví dụ $_{17}^{35}Cl
Hãy chọn các phát biểu đúng , sai trong các phát biểu sau?
\choiceTF[t]
	{Nội dung phương án sai}
	{\True Nội dung Phương án đúng}
	{Nội dung phương án sai}
	{\True Nội dung Phương án đúng}
	Phương án 1: đưa ra kiến thức về xu hướng nhường nhận bao nhiêu electron?
	Phương án 2: đưa ra thông tin về số thứ tự nhóm A hay B (chú ý học sinh hay nhầm lẫn xác định nhóm A hay nhóm B)	
	Phương án 3: đưa ra thông tin về cấu hình e
	Phương án 4: đưa ra thông tin về eletron có mức năng lượng cao nhất điền vào phân lớp s

#####========Promt viết lại nội dung từ ChatGPT theo form Latex===========####

Cố gắng viết lại nội dung theo code latex có form theo yêu cầu như ví dụ sau:
%%%% ============BT_01==================%%%
\begin{bt}
	Nội dung câu hỏi
	%%Nếu có nhiều câu hỏi thì dung môi trường enumerate
	\begin{enumerate}
		\item Ý hỏi a
		\item Ý hỏi b
		\item Ý hỏi n
	\end{enumerate}
	%%%Đưa nội dung sau hướng dẫn giải vào trong lệnh
	\loigiai{Nội dung sau Hướng dẫn giải}

\end{bt}
-Chú ý các công thức phải đưa vào cặp dấu $$ chẳng hạn 1s² 2s² 2p⁶ 3s¹ phải viết lại $1s^22s^22p^63s^1$
Và đây là nội dung cần viết lại:

Bài 1. Natri (Sodium) là tên một nguyên tố hóa học trị một trong bảng tuần hoàn nguyên tử có ký hiệu Na và số hiệu nguyên tử bằng 11. Nhiều hợp chất của natri được sử dụng rộng rãi như Sodium hydroxide để làm xà phòng, và sodium chloride dùng làm chất tạo hương và là một chất dinh dưỡng (muối ăn). Natri là một nguyên tố thiết yếu cho tất cả động vật và một số thực vật.
a) Viết cấu hình electron của nguyên tử Sodium ?
b) Sodium là kim loại hay phi kim?

Hướng dẫn giải

a) Viết cấu hình electron của nguyên tử Sodium:
+ Bước 1: số electron: 11
+ Bước 2: phân bố electron theo thứ tự tăng dần các mức năng lượng AO:
1s² 2s² 2p⁶ 3s¹
⇒ Cấu hình electron: 1s² 2s² 2p⁶ 3s¹

b) Sodium là kim loại (vì có 1 electron lớp ngoài cùng 3s¹)

Bài 2. Cho các nguyên tử sau: H(Z=1), He(Z=2), Li(Z=3), Be(Z=4), B(Z=5), C(Z=6), N(Z=7), O(Z=8), F(Z=9), Ne(Z=10). Viết cấu hình electron của các nguyên tử trên, phân bố electron vào các Orbital, cho biết chúng là kim loại, phi kim hay khí hiếm.
Hướng dẫn giải
Nhận xét: các nguyên tử trên đều có Z < 20 ⇒ Ta chỉ cần thực hiện 2 bước để có được cấu hình electron đúng.
Chẳng hạn: C (Z=6):
+ Bước 1: số electron: 6
+ Bước 2: phân bố electron theo thứ tự tăng dần các mức năng lượng AO:
1s² 2s² 2p²
Cấu hình electron: 1s² 2s² 2p²




