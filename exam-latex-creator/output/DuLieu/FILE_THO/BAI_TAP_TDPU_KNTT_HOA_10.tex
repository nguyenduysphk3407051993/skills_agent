%%%=============EX_1=============%%%
\begin{ex}
	Cho phản ứng xảy ra trong pha khí sau:
	\[\text{H}_2 + \text{Cl}_2 \xrightarrow 2\text{H}\text{Cl}\]
	Biểu thức tốc độ trung bình của phản ứng là
	\choice
	{$v = \dfrac{\Delta[H_2]}{\Delta t} = \dfrac{\Delta[Cl_2]}{\Delta t} = \dfrac{\Delta[HCl]}{\Delta t}$}
	{$v = \dfrac{\Delta[H_2]}{\Delta t} = \dfrac{\Delta[Cl_2]}{\Delta t} = -\dfrac{\Delta[HCl]}{\Delta t}$}
	{$v = -\dfrac{\Delta[H_2]}{\Delta t} = -\dfrac{\Delta[Cl_2]}{\Delta t} = \dfrac{\Delta[HCl]}{\Delta t}$}
	{\True $v = -\dfrac{\Delta[H_2]}{\Delta t} = -\dfrac{\Delta[Cl_2]}{\Delta t} = \dfrac{\Delta[HCl]}{2\Delta t}$}
	\loigiai{
		Tốc độ trung bình của phản ứng được tính bằng độ biến thiên nồng độ của các chất chia cho thời gian, có dấu âm cho chất phản ứng và chia cho hệ số tỉ lượng nếu hệ số khác 1.
	}
\end{ex}
%%%=============EX_2================%%%%
\begin{ex}
	Trong dung dịch phản ứng thuỷ phân ethyl acetate ($CH_3COOC_2H_5$) có xúc tác acid vô cơ xảy ra như sau:
	\[\text{C}\text{H}_3\text{C}\text{O}\text{O}\text{C}_2\text{H}_5 + \text{H}_2\text{O} \xrightarrow[$1$] \text{C}\text{H}_3\text{C}\text{O}\text{O}\text{H} + \text{C}_2\text{H}_5\text{O}\text{H}\]
	Phát biểu nào sau đây đúng?
	\choice
	{Nồng độ acid tăng dần theo thời gian}
	{Thời điểm ban đầu, nồng độ acid trong bình phản ứng bằng 0}
	{Tỉ lệ mol giữa chất đầu và chất sản phẩm luôn bằng 1}
	{HCl chuyển hoá dần thành $CH_3COOH$ nên nồng độ HCl giảm dần theo thời gian}
	\loigiai{
		HCl là chất xúc tác, không bị biến đổi trong phản ứng. Nồng độ acid (HCl) không đổi.
	}
\end{ex}

%%%=============BT_3=============%%%
\begin{bt}
	Sục khí $CO_2$ vào bình chứa dung dịch $Na_2CO_3$.
	\begin{enumerate}[a)]
		\item Tốc độ hấp thụ khí $CO_2$ sẽ thay đổi như thế nào nếu thêm các chất sau đây vào dung dịch:
		\begin{enumerate}[(i)]
			\item $HCl$
			\item $NaCl$
			\item $H_2O$
			\item $K_2CO_3$
		\end{enumerate}
		\item Nếu tăng áp suất, tốc độ phản ứng thay đổi như thế nào?
	\end{enumerate}
	\loigiai{
		\begin{enumerate}[a)]
			\item 
			$CO_2 + Na_2CO_3 + H_2O \rightleftharpoons 2NaHCO_3$
			\begin{enumerate}[(i)]
				\item Thêm $HCl$: $HCl + Na_2CO_3 \xrightarrow NaCl + NaHCO_3$. Làm giảm $Na_2CO_3$ nên tốc độ hấp thụ $CO_2$ giảm.
				\item Thêm $NaCl$: Không ảnh hưởng đáng kể đến tốc độ hấp thụ $CO_2$.
				\item Thêm $H_2O$: Pha loãng dung dịch, làm giảm nồng độ $Na_2CO_3$ nên tốc độ hấp thụ $CO_2$ giảm.
				\item Thêm $K_2CO_3$: Làm tăng nồng độ $CO_3^{2-}$ nên tốc độ hấp thụ $CO_2$ tăng.
			\end{enumerate}
			\item Nếu tăng áp suất, tốc độ phản ứng tăng do làm tăng nồng độ $CO_2$ trong dung dịch.
		\end{enumerate}
	}
\end{bt}
%%%=============BT_4=============%%%
\begin{bt}
	Cho các phản ứng hoá học sau:
	\begin{enumerate}[a)]
		\item $Fe_3O_4(s) + 4CO(g) \xrightarrow 3Fe(s) + 4CO_2(g)$
		\item $2NO_2(g) \xrightarrow N_2O_4(g)$
		\item $H_2(g) + Cl_2(g) \xrightarrow 2HCl(g)$
		\item $CaO(s) + SiO_2(s) \xrightarrow CaSiO_3(s)$
		\item $CaO(s) + CO_2(g) \xrightarrow CaCO_3(s)$
		\item $2KI(aq) + H_2O_2(aq) \xrightarrow I_2(s) + 2KOH(aq)$
	\end{enumerate}
	Tốc độ những phản ứng nào ở trên thay đổi khi áp suất thay đổi?
	\loigiai{
		Tốc độ phản ứng thay đổi khi áp suất thay đổi đối với các phản ứng có chất khí.
		\begin{itemize}
			\item Phản ứng a) có chất khí tham gia và tạo thành.
			\item Phản ứng b) có chất khí tham gia và tạo thành.
			\item Phản ứng c) có chất khí tham gia và tạo thành.
			\item Phản ứng d) không có chất khí.
			\item Phản ứng e) có chất khí tham gia.
			\item Phản ứng g) không có chất khí.
		\end{itemize}
		Vậy các phản ứng a, b, c, e có tốc độ thay đổi khi áp suất thay đổi.
	}
\end{bt}
%%%=============EX_5================%%%%
\begin{ex}
	Cho bột Fe vào dung dịch HCl loãng. Sau đó đun nóng hỗn hợp này.
	Phát biểu nào sau đây không đúng?
	\choice
	{Khí $H_2$ thoát ra nhanh hơn}
	{Bột Fe tan nhanh hơn}
	{Lượng muối thu được nhiều hơn}
	{\True Nồng độ HCl giảm nhanh hơn}
	\loigiai{
		Khi đun nóng, tốc độ phản ứng tăng, Fe tan nhanh hơn, khí $H_2$ thoát ra nhanh hơn. Lượng muối thu được không đổi, chỉ có tốc độ phản ứng tăng.
	}
\end{ex}

%%%=============EX_6================%%%%
\begin{ex}
	Cho phản ứng hoá học xảy ra trong pha khí sau:
	\[\text{N}_2 + 3\text{H}_2 \xrightarrow 2\text{N}\text{H}_3\]
	Khi nhiệt độ phản ứng tăng lên, phát biểu nào sau đây không đúng?
	\choice
	{Tốc độ chuyển động của phân tử chất đầu ($N_2$, $H_2$) tăng lên}
	{Tốc độ va chạm giữa phân tử $N_2$ và $H_2$ tăng lên}
	{Số va chạm hiệu quả tăng lên}
	{\True Tốc độ chuyển động của phân tử chất sản phẩm ($NH_3$) giảm}
	\loigiai{
		Khi nhiệt độ tăng, tốc độ chuyển động của các phân tử tăng, số va chạm tăng và số va chạm hiệu quả cũng tăng. Tốc độ chuyển động của phân tử sản phẩm cũng tăng.
	}
\end{ex}
%%%=============BT_7=============%%%
\begin{bt}
	Cho bột magnesium vào nước, phản ứng xảy ra rất chậm. Hãy nêu cách làm tăng tốc độ phản ứng trên.
	\loigiai{
		Để tăng tốc độ phản ứng giữa bột magnesium và nước, ta có thể áp dụng các biện pháp sau:
		\begin{itemize}
			\item \textbf{Tăng nhiệt độ}: Đun nóng hỗn hợp phản ứng.
			\item \textbf{Tăng diện tích bề mặt tiếp xúc}: Sử dụng bột magnesium mịn hơn.
			\item \textbf{Sử dụng chất xúc tác}: Thêm một lượng nhỏ acid vào nước.
			\item \textbf{Khuấy trộn}: Khuấy đều hỗn hợp phản ứng để tăng sự tiếp xúc giữa magnesium và nước.
		\end{itemize}
	}
\end{bt}
%%%=============EX_8================%%%%
\begin{ex}
	Cho phản ứng hoá học sau:
	\[\text{Zn}(s) + \text{H}_2\text{S}\text{O}_4(aq) \xrightarrow \text{Zn}\text{S}\text{O}_4(aq) + \text{H}_2(g)\]
	Yếu tố nào sau đây không ảnh hưởng đến tốc độ phản ứng?
	\choice
	{Diện tích bề mặt zinc}
	{Nồng độ dung dịch sulfuric acid}
	{\True Thể tích dung dịch sulfuric acid}
	{Nhiệt độ của dung dịch sulfuric acid}
	\loigiai{
		Các yếu tố ảnh hưởng đến tốc độ phản ứng là: diện tích bề mặt chất rắn, nồng độ chất tan, nhiệt độ. Thể tích dung dịch không ảnh hưởng trực tiếp đến tốc độ phản ứng.
	}
\end{ex}

%%%=============EX_9================%%%%
\begin{ex}
	Phát biểu nào sau đây là đúng về xúc tác?
	\choice
	{Xúc tác giúp làm tăng năng lượng hoạt hoá của phản ứng}
	{\True Khối lượng xúc tác không thay đổi sau phản ứng}
	{Xúc tác không tương tác với các chất trong quá trình phản ứng}
	{Xúc tác kết hợp với sản phẩm phản ứng tạo thành hợp chất bền}
	\loigiai{
		Xúc tác làm giảm năng lượng hoạt hóa của phản ứng, khối lượng không đổi sau phản ứng, có tương tác với chất phản ứng để tạo phức chất trung gian.
	}
\end{ex}

%%%=============EX_10================%%%%
\begin{ex}
	Cho phản ứng thuỷ phân tinh bột có xúc tác là HCl.
	Phát biểu nào sau đây không đúng?
	\choice
	{HCl không tác dụng với tinh bột trong quá trình phản ứng}
	{Nếu nồng độ HCl tăng, tốc độ phản ứng tăng}
	{\True Khi không có HCl, phản ứng thuỷ phân tinh bột vẫn xảy ra nhưng với tốc độ chậm}
	{Nồng độ HCl không đổi sau phản ứng}
	\loigiai{
		HCl là chất xúc tác, không bị biến đổi trong phản ứng. Phản ứng thuỷ phân tinh bột cần có xúc tác acid hoặc enzyme.
	}
\end{ex}
%%%=============BT_11=============%%%
\begin{bt}
	Cho các phản ứng hoá học sau:
	\begin{enumerate}
		\item $FeCl_3 + 3NaOH \xrightarrow Fe(OH)_3 + 3NaCl$
		\item $3Fe + 2O_2 \xrightarrow Fe_3O_4$
		\item $4K + O_2 \xrightarrow 2K_2O$
		\item $CH_3COOH + C_2H_5OH \xrightarrow CH_3COOC_2H_5 + H_2O$
	\end{enumerate}
	Ở điều kiện thường, phản ứng nào xảy ra nhanh, phản ứng nào xảy ra chậm?
	\loigiai{
		\begin{itemize}
			\item Phản ứng (1) xảy ra nhanh do là phản ứng trao đổi ion trong dung dịch.
			\item Phản ứng (2) xảy ra chậm do là phản ứng giữa chất rắn và khí ở nhiệt độ thường.
			\item Phản ứng (3) xảy ra nhanh do Kali là kim loại kiềm có tính khử mạnh, phản ứng với oxi xảy ra dễ dàng.
			\item Phản ứng (4) xảy ra chậm do là phản ứng este hoá cần xúc tác và nhiệt độ.
		\end{itemize}
	}
\end{bt}

%%%=============SA_12=============%%%
\begin{bt}
	Thả 1 mảnh magnesium có khối lượng $0{,}1 \text{ g}$ vào dung dịch HCl loãng. Sau 5 giây thấy mảnh magnesium tan hết. Hãy tính tốc độ trung bình của phản ứng hoà tan magnesium.
	\shortans{$0{,}02$}
	\loigiai{
		Tốc độ trung bình của phản ứng hoà tan magnesium được tính bằng lượng magnesium tan hết trong một đơn vị thời gian.
		$v = \dfrac{0{,}1 \text{ g}}{5 \text{ s}} = 0{,}02 \text{ g/s}$
	}
\end{bt}
%%%=============BT_13=============%%%
\begin{bt}
	Trong một thí nghiệm, người ta đo được tốc độ trung bình của phản ứng của zinc (dạng bột) với dung dịch $H_2SO_4$ loãng là $0{,}005 \text{ mol/s}$.
	Nếu ban đầu cho $0{,}4 \text{ mol}$ zinc (dạng bột) vào dung dịch $H_2SO_4$ ở trên thì sau bao lâu còn lại $0{,}05 \text{ mol}$ zinc.
	\loigiai{
		Lượng zinc đã phản ứng là $0{,}4 - 0{,}05 = 0{,}35 \text{ mol}$.
		Thời gian phản ứng là $t = \dfrac{0{,}35 \text{ mol}}{0{,}005 \text{ mol/s}} = 70 \text{ s}$.
	}
\end{bt}

%%%=============BT_14=============%%%
\begin{bt}
	Xét phản ứng: $3O_2 \xrightarrow 2O_3$.
	Nồng độ ban đầu của oxygen là $0{,}024 \text{ M}$. Sau 5 giây nồng độ của oxygen còn lại là $0{,}02 \text{ M}$. Tính tốc độ trung bình của phản ứng trong khoảng thời gian trên.
	\loigiai{
		Tốc độ trung bình của phản ứng được tính theo công thức:
		\[v = -\dfrac{1}{3} \dfrac{\\text{Delta} [\text{O}_2]}{\\text{Delta} t} = -\dfrac{1}{3} \dfrac{[\text{O}_2]_t - [\text{O}_2]_0}{\\text{Delta} t}\]
		Thay số:
		\[v = -\dfrac{1}{3} \dfrac{0{,}02 - 0{,}024}{5} = -\dfrac{1}{3} \dfrac{-0{,}004}{5} = \dfrac{0{,}004}{15} \approx 0{,}000267 \text{ \text{M}/s}\]
	}
\end{bt}

%%%=============BT_15=============%%%
\begin{bt}
	Cho các phản ứng hoá học sau:
	\begin{enumerate}[a)]
		\item $CH_3COOC_2H_5(l) + H_2O(l) \xrightarrow CH_3COOH(l) + C_2H_5OH(l)$
		\item $Zn(s) + H_2SO_4(aq) \xrightarrow ZnSO_4(aq) + H_2(g)$
		\item $H_2C_2O_4(aq) + 2KMnO_4(aq) + 3H_2SO_4(aq) \xrightarrow 10CO_2(g) + 2MnSO_4(aq) + 8H_2O(l)$
	\end{enumerate}
	Tốc độ các phản ứng trên sẽ thay đổi thế nào nếu ta thêm nước vào bình phản ứng?
	\loigiai{
		\begin{enumerate}[a)]
			\item Thêm nước vào, nồng độ các chất giảm, tốc độ phản ứng giảm.
			\item Thêm nước vào, nồng độ $H_2SO_4$ giảm, tốc độ phản ứng giảm.
			\item Thêm nước vào, nồng độ các chất giảm, tốc độ phản ứng giảm.
		\end{enumerate}
	}
\end{bt}
%%%=============BT_16=============%%%
\begin{bt}
	Thực hiện hai thí nghiệm của cùng một lượng $CaCO_3$ với dung dịch $HCl$ (dư) có nồng độ khác nhau. Thể tích khí $CO_2$ thoát ra theo thời gian được ghi lại trên đồ thị sau: 
	Phản ứng nào đã dùng $HCl$ với nồng độ cao hơn?
	\loigiai{
		Đường cong (1) cho thấy phản ứng xảy ra nhanh hơn (thể tích $CO_2$ thoát ra nhanh hơn trong cùng một khoảng thời gian). Do đó, phản ứng (1) đã dùng $HCl$ với nồng độ cao hơn.
	}
\end{bt}
%%%=============BT_17=============%%%
\begin{bt}
	Cho phản ứng hoá học sau:
	\[\text{H}_2\text{O}_2 \xrightarrow \text{H}_2\text{O} + \dfrac{1}{2} \text{O}_2\]
	Biết rằng tốc độ của phản ứng này tuân theo biểu thức của định luật tác dụng khối lượng.
	\begin{enumerate}[a)]
		\item Hãy viết biểu thức tốc độ phản ứng.
		\item Tốc độ phản ứng tức thời tăng dần hay giảm dần theo thời gian?
	\end{enumerate}
	\loigiai{
		\begin{enumerate}[a)]
			\item Biểu thức tốc độ phản ứng:
			\[v = k[\text{H}_2\text{O}_2]\]
			Trong đó:
			\begin{itemize}
				\item $v$ là tốc độ phản ứng
				\item $k$ là hằng số tốc độ
				\item $[H_2O_2]$ là nồng độ của $H_2O_2$
			\end{itemize}
			\item Tốc độ phản ứng tức thời giảm dần theo thời gian vì nồng độ $H_2O_2$ giảm dần theo thời gian.
		\end{enumerate}
	}
\end{bt}
%%%=============EX_18================%%%%
\begin{ex}
	Cách nào sau đây sẽ làm củ khoai tây chín nhanh nhất?
	\choice
	{Luộc trong nước sôi}
	{Hấp cách thuỷ trong nồi cơm}
	{\True Nướng ở $180^\circ C$}
	{Hấp trên nồi hơi}
	\loigiai{
		Nướng ở $180^\circ C$ nhanh chín hơn vì nhiệt độ cao hơn.
	}
\end{ex}
%%%=============BT_19=============%%%
\begin{bt}
	Các nhà khảo cổ thường tìm được xác các loài động thực vật thời tiền sử nguyên vẹn trong băng. Hãy giải thích tại sao băng lại giúp bảo quản xác động thực vật.
	\loigiai{
		Băng giúp bảo quản xác động thực vật vì:
		\begin{itemize}
			\item \textbf{Nhiệt độ thấp}: Nhiệt độ thấp làm chậm hoặc ngừng các quá trình phân hủy sinh học do vi sinh vật gây ra.
			\item \textbf{Ngăn chặn oxy}: Băng ngăn chặn sự tiếp xúc của xác động thực vật với oxy, làm chậm quá trình oxy hóa và phân hủy.
			\item \textbf{Ức chế hoạt động enzyme}: Nhiệt độ thấp ức chế hoạt động của các enzyme tự phân hủy trong cơ thể động thực vật.
		\end{itemize}
	}
\end{bt}

%%%=============BT_20=============%%%
\begin{bt}
	$NOCl$ là chất khí độc, sinh ra do sự phân huỷ nước cường toan (hỗn hợp $HNO_3$ và $HCl$ có tỉ lệ mol 1:3). $NOCl$ có tính oxi hoá mạnh, ở nhiệt độ cao bị phân huỷ theo phản ứng hoá học sau:
	\[2\text{N}\text{O}\text{Cl} \xrightarrow 2\text{N}\text{O} + \text{Cl}_2\]
	Tốc độ phản ứng ở $70^\circ C$ là $2 \cdot 10^{-7} \text{ mol/(L.s)}$ và ở $80^\circ C$ là $4{,}5 \cdot 10^{-7} \text{ mol/(L.s)}$.
	\begin{enumerate}[a)]
		\item Tính hệ số nhiệt độ $\gamma$ của phản ứng.
		\item Dự đoán tốc độ phản ứng ở $60^\circ C$.
	\end{enumerate}
	\loigiai{
		\begin{enumerate}[a)]
			\item Hệ số nhiệt độ $\gamma$ được tính theo công thức:
			\[\gamma = \left( \dfrac{v_{t+10}}{v_t} \right)\]
			Trong đó:
			\begin{itemize}
				\item $v_t$ là tốc độ phản ứng ở nhiệt độ $t$
				\item $v_{t+10}$ là tốc độ phản ứng ở nhiệt độ $t+10$
			\end{itemize}
			Thay số:
			\[\gamma = \dfrac{4{,}5 \cdot 10^{-7}}{2 \cdot 10^{-7}} = 2{,}25\]
			\item Dự đoán tốc độ phản ứng ở $60^\circ C$:
			\[v_{70} = v_{60} \cdot \gamma\]
			\[v_{60} = \dfrac{v_{70}}{\gamma} = \dfrac{2 \cdot 10^{-7}}{2{,}25} \approx 0{,}89 \cdot 10^{-7} \text{ mol/(\text{L}.s)}\]
		\end{enumerate}
	}
\end{bt}
%%%=============BT_21=============%%%
\begin{bt}
	Khi thắng đường để làm caramen hoặc nước hàng, ta thường dùng đường kính chứ không dùng đường phèn. Giải thích.
	\loigiai{
		Đường kính có kích thước hạt nhỏ hơn đường phèn, do đó diện tích bề mặt tiếp xúc với nhiệt lớn hơn, giúp đường tan chảy và chuyển thành caramen nhanh hơn. Đường phèn có kích thước hạt lớn, cần nhiều thời gian và nhiệt lượng hơn để tan chảy hoàn toàn, dễ dẫn đến cháy khét trước khi đạt được màu và hương vị mong muốn.
	}
\end{bt}

%%%=============BT_22=============%%%
\begin{bt}
	Khi dùng $MnO_2$ làm xúc tác trong phản ứng phân huỷ $H_2O_2$, tại sao ta cần dùng $MnO_2$ ở dạng bột chứ không dùng ở dạng viên.
	\loigiai{
		$MnO_2$ ở dạng bột có diện tích bề mặt tiếp xúc với $H_2O_2$ lớn hơn so với dạng viên. Diện tích bề mặt tiếp xúc càng lớn, tốc độ phản ứng càng nhanh, giúp quá trình phân huỷ $H_2O_2$ diễn ra hiệu quả hơn.
	}
\end{bt}

%%%=============BT_23=============%%%
\begin{bt}
	Trong công nghiệp, vôi sống được sản xuất bằng cách nung đá vôi. Phản ứng hoá học xảy ra như sau:
	\[\text{Ca}\text{C}\text{O}_3 \xrightarrow \text{Ca}\text{O} + \text{C}\text{O}_2\]
	Khi nung, đá vôi cần phải được đập nhỏ nhưng không nên nghiền mịn đá vôi thành bột. Giải thích.
	\loigiai{
		\begin{itemize}
			\item \textbf{Đập nhỏ đá vôi}: Tăng diện tích bề mặt tiếp xúc, giúp quá trình nung diễn ra nhanh hơn và đều hơn.
			\item \textbf{Không nghiền mịn thành bột}:
			\begin{itemize}
				\item Bột đá vôi có thể gây khó khăn cho việc thông khí, làm chậm quá trình thoát khí $CO_2$, gây cản trở phản ứng.
				\item Bột đá vôi có thể bị vón cục, làm giảm diện tích bề mặt tiếp xúc thực tế.
				\item Bột đá vôi dễ bị cuốn theo khí thải, gây thất thoát nguyên liệu.
			\end{itemize}
		\end{itemize}
	}
\end{bt}
%%%=============BT_24=============%%%
\begin{bt}
	Trong quá trình tổng hợp nitric acid, có giai đoạn đốt cháy $NH_3$ bằng $O_2$ có xúc tác. Phản ứng xảy ra trong pha khí như sau:
	\[4\text{N}\text{H}_3 + 5\text{O}_2 \xrightarrow 4\text{N}\text{O} + 6\text{H}_2\text{O}\]
	Trong một thí nghiệm, cho vào bình phản ứng (bình kín) $560 \text{ mL}$ khí $NH_3$ và $672 \text{ mL}$ khí $O_2$ (có xúc tác, các thể tích khí đo ở đktc). Sau khi thực hiện phản ứng $2{,}5$ giờ, thấy có $0{,}432 \text{ g}$ nước tạo thành.
	\begin{enumerate}[a)]
		\item Viết biểu thức tính tốc độ trung bình của phản ứng theo các chất tham gia và chất tạo thành trong phản ứng.
		\item Tính tốc độ trung bình của phản ứng theo đơn vị $\text{mol/h}$.
		\item Tính số mol $NH_3$ và $O_2$ sau $2{,}5$ giờ.
	\end{enumerate}
	\loigiai{
		\begin{enumerate}[a)]
			\item Biểu thức tính tốc độ trung bình của phản ứng:
			\[v = -\dfrac{1}{4} \dfrac{\\text{Delta} [\text{N}\text{H}_3]}{\\text{Delta} t} = -\dfrac{1}{5} \dfrac{\\text{Delta} [\text{O}_2]}{\\text{Delta} t} = \dfrac{1}{4} \dfrac{\\text{Delta} [\text{N}\text{O}]}{\\text{Delta} t} = \dfrac{1}{6} \dfrac{\\text{Delta} [\text{H}_2\text{O}]}{\\text{Delta} t}\]
			\item Số mol $H_2O$ tạo thành:
			\[n_{\text{H}_2\text{O}} = \dfrac{0{,}432}{18} = 0{,}024 \text{ mol}\]
			Tốc độ trung bình của phản ứng tính theo $H_2O$:
			\[v = \dfrac{1}{6} \dfrac{\\text{Delta} n_{\text{H}_2\text{O}}}{\\text{Delta} t} = \dfrac{1}{6} \dfrac{0{,}024}{2{,}5} = 0{,}0016 \text{ mol/h}\]
			\item Số mol $NH_3$ ban đầu:
			\[n_{\text{N}\text{H}_3} = \dfrac{0{,}56}{22{,}4} = 0{,}025 \text{ mol}\]
			Số mol $O_2$ ban đầu:
			\[n_{\text{O}_2} = \dfrac{0{,}672}{22{,}4} = 0{,}03 \text{ mol}\]
			Theo phương trình phản ứng:
			\[4\text{N}\text{H}_3 + 5\text{O}_2 \xrightarrow 4\text{N}\text{O} + 6\text{H}_2\text{O}\]
			$0{,}016 \leftarrow 0{,}02 \leftarrow \hphantom{000000} 0{,}024$
			Số mol $NH_3$ phản ứng:
			\[n_{\text{N}\text{H}_3(pu)} = \dfrac{4}{6} n_{\text{H}_2\text{O}} = \dfrac{4}{6} \cdot 0{,}024 = 0{,}016 \text{ mol}\]
			Số mol $O_2$ phản ứng:
			\[n_{\text{O}_2(pu)} = \dfrac{5}{6} n_{\text{H}_2\text{O}} = \dfrac{5}{6} \cdot 0{,}024 = 0{,}02 \text{ mol}\]
			Số mol $NH_3$ còn lại sau $2{,}5$ giờ:
			\[n_{\text{N}\text{H}_3(conlai)} = 0{,}025 - 0{,}016 = 0{,}009 \text{ mol}\]
			Số mol $O_2$ còn lại sau $2{,}5$ giờ:
			\[n_{\text{O}_2(conlai)} = 0{,}03 - 0{,}02 = 0{,}01 \text{ mol}\]
		\end{enumerate}
	}
\end{bt}
%%%=============EX_25================%%%%
\begin{ex}
	Thực hiện phản ứng sau:
	\[\text{Ca}\text{C}\text{O}_3 + 2\text{H}\text{Cl} \xrightarrow \text{Ca}\text{Cl}_2 + \text{C}\text{O}_2\uparrow + \text{H}_2\text{O}\]
	Theo dõi thể tích $CO_2$ thoát ra theo thời gian, thu được đồ thị như sau: {Đồ thị thể hiện sự tăng dần của thể tích CO2 theo thời gian, đạt trạng thái ổn định sau 90s}
	Trong các phát biểu sau, phát biểu nào không đúng?
	\choice
	{Ở thời điểm 90 giây, tốc độ phản ứng bằng 0}
	{Tốc độ phản ứng giảm dần theo thời gian}
	{Tốc độ trung bình của phản ứng trong khoảng thời gian từ thời điểm đầu đến 75 giây là $0{,}33 \text{ mL/s}$}
	{\True Tốc độ trung bình của phản ứng trong các khoảng thời gian 15 giây là như nhau}
	\loigiai{
		\begin{itemize}
			\item A. Đúng, ở thời điểm 90 giây, phản ứng đã kết thúc, thể tích $CO_2$ không đổi, tốc độ phản ứng bằng 0.
			\item B. Đúng, tốc độ phản ứng giảm dần theo thời gian do nồng độ các chất phản ứng giảm.
			\item C. Đúng, thể tích $CO_2$ ở thời điểm 75 giây khoảng $25 \text{ mL}$, tốc độ trung bình là $\dfrac{25}{75} \approx 0{,}33 \text{ mL/s}$.
			\item D. Sai, tốc độ phản ứng giảm dần theo thời gian, nên tốc độ trung bình trong các khoảng thời gian 15 giây không bằng nhau.
		\end{itemize}
	}
\end{ex}
%%%=============BT_26=============%%%
\begin{bt}
	Thực hiện phản ứng sau:
	\[\text{H}_2\text{S}\text{O}_4 + \text{Na}_2\text{S}_2\text{O}_3 \xrightarrow \text{Na}_2\text{S}\text{O}_4 + \text{S}\text{O}_2 + \text{S} + \text{H}_2\text{O}\]
	Theo dõi thể tích $SO_2$ thoát ra theo thời gian, ta có bảng sau (thể tích khí được đo ở áp suất khí quyển và nhiệt độ phòng).
	\begin{center}
		\begin{tabular}{|c|c|c|c|c|c|c|c|c|}
			\hline
			Thời gian (s) & 0 & 10 & 20 & 30 & 40 & 50 & 60 & 70 \\
			\hline
			Thể tích $SO_2$ (mL) & 0{,}0 & 12{,}5 & 20{,}0 & 26{,}5 & 31{,}0 & 32{,}5 & 33 & 33 \\
			\hline
		\end{tabular}
	\end{center}
	\begin{enumerate}[a)]
		\item Vẽ đồ thị biểu diễn sự phụ thuộc thể tích khí $SO_2$ vào thời gian phản ứng. {Đồ thị thể hiện sự tăng dần của thể tích SO2 theo thời gian, đạt trạng thái ổn định sau 60s}
		\item Thời điểm đầu, tốc độ phản ứng nhanh hay chậm?
		\item Thời điểm kết thúc phản ứng, đồ thị có hình dạng như thế nào?
		\item Tính tốc độ trung bình của phản ứng trong khoảng: từ 0 - 10 giây; từ 10 – 20 giây; từ 20 - 40 giây.
	\end{enumerate}
	\loigiai{
		\begin{enumerate}[a)]
			\item Đồ thị biểu diễn sự phụ thuộc thể tích khí $SO_2$ vào thời gian phản ứng.
			\item Thời điểm đầu, tốc độ phản ứng nhanh.
			\item Thời điểm kết thúc phản ứng, đồ thị có dạng đường thẳng nằm ngang (song song với trục thời gian).
			\item Tốc độ trung bình của phản ứng:
			\begin{itemize}
				\item Từ 0 - 10 giây: $v_{tb} = \dfrac{12{,}5 - 0}{10 - 0} = 1{,}25 \text{ mL/s}$
				\item Từ 10 - 20 giây: $v_{tb} = \dfrac{20{,}0 - 12{,}5}{20 - 10} = 0{,}75 \text{ mL/s}$
				\item Từ 20 - 40 giây: $v_{tb} = \dfrac{31{,}0 - 20{,}0}{40 - 20} = 0{,}55 \text{ mL/s}$
			\end{itemize}
		\end{enumerate}
	}
\end{bt}
%%%=============BT_27=============%%%
\begin{bt}
	Xét phản ứng sau:
	\[2\text{Cl}\text{O}_2 + 2\text{Na}\text{O}\text{H} \xrightarrow \text{Na}\text{Cl}\text{O}_3 + \text{Na}\text{Cl}\text{O}_2 + \text{H}_2\text{O}\]
	Tốc độ phản ứng được viết như sau: $v = k \cdot [ClO_2]^x \cdot [NaOH]^y$
	Thực hiện phản ứng với những nồng độ chất đầu khác nhau và đo tốc độ phản ứng tương ứng thu được kết quả trong bảng sau:
	\begin{center}
		\begin{tabular}{|c|c|c|c|}
			\hline
			STT & Nồng độ $ClO_2$ (M) & Nồng độ $NaOH$ (M) & Tốc độ phản ứng (mol/(L.s)) \\
			\hline
			1 & 0{,}01 & 0{,}01 & $2 \cdot 10^{-4}$ \\
			\hline
			2 & 0{,}02 & 0{,}01 & $8 \cdot 10^{-4}$ \\
			\hline
			3 & 0{,}01 & 0{,}02 & $4 \cdot 10^{-4}$ \\
			\hline
		\end{tabular}
	\end{center}
	Hãy tính $x$ và $y$ trong biểu thức tốc độ phản ứng.
	\loigiai{
		Từ thí nghiệm 1 và 2:
		\[\dfrac{v_2}{v_1} = \dfrac{k \cdot [\text{Cl}\text{O}_2]_2^x \cdot [\text{Na}\text{O}\text{H}]_2^y}{k \cdot [\text{Cl}\text{O}_2]_1^x \cdot [\text{Na}\text{O}\text{H}]_1^y} = \dfrac{8 \cdot 10^{-4}}{2 \cdot 10^{-4}} = 4\]
		\[\dfrac{(0{,}02)^x}{(0{,}01)^x} = 4 \\text{Rightarrow} 2^x = 4 \\text{Rightarrow} x = 2\]
		Từ thí nghiệm 1 và 3:
		\[\dfrac{v_3}{v_1} = \dfrac{k \cdot [\text{Cl}\text{O}_2]_3^x \cdot [\text{Na}\text{O}\text{H}]_3^y}{k \cdot [\text{Cl}\text{O}_2]_1^x \cdot [\text{Na}\text{O}\text{H}]_1^y} = \dfrac{4 \cdot 10^{-4}}{2 \cdot 10^{-4}} = 2\]
		\[\dfrac{(0{,}02)^y}{(0{,}01)^y} = 2 \\text{Rightarrow} 2^y = 2 \\text{Rightarrow} y = 1\]
		Vậy $x = 2$ và $y = 1$.
	}
\end{bt}
%%%=============BT_28=============%%%
\begin{bt}
	Hãy đề xuất một phương pháp thực nghiệm để nghiên cứu tốc độ các phản ứng sau đây. Trong đó chỉ rõ: đại lượng nào em sẽ đo; đồ thị theo dõi sự thay đổi của đại lượng đó theo thời gian có dạng thế nào.
	\begin{enumerate}[a)]
		\item Phản ứng xảy ra trong dung dịch:
		\[\text{C}\text{H}_3\text{C}\text{H}_2\text{Br} + \text{H}_2\text{O} \xrightarrow \text{C}\text{H}_3\text{C}\text{H}_2\text{O}\text{H} + \text{H}\text{Br}\]
		\item Phản ứng xảy ra trong pha khí:
		\[2\text{N}\text{O} + \text{Cl}_2 \xrightarrow 2\text{N}\text{O}\text{Cl}\]
	\end{enumerate}
	\loigiai{
		\begin{enumerate}[a)]
			\item Phản ứng xảy ra trong dung dịch:
			\begin{itemize}
				\item \textbf{Đại lượng đo}: Nồng độ của $HBr$ hoặc độ dẫn điện của dung dịch.
				\item \textbf{Phương pháp}:
				\begin{itemize}
					\item Đo nồng độ $HBr$ bằng phương pháp chuẩn độ acid-base theo thời gian.
					\item Đo độ dẫn điện của dung dịch theo thời gian (do $HBr$ là chất điện ly mạnh).
				\end{itemize}
				\item \textbf{Dạng đồ thị}: Đồ thị biểu diễn sự tăng dần của nồng độ $HBr$ (hoặc độ dẫn điện) theo thời gian, sau đó đạt đến trạng thái ổn định.
			\end{itemize}
			\item Phản ứng xảy ra trong pha khí:
			\begin{itemize}
				\item \textbf{Đại lượng đo}: Áp suất của hệ hoặc nồng độ của một trong các chất khí.
				\item \textbf{Phương pháp}:
				\begin{itemize}
					\item Đo áp suất của hệ theo thời gian (do số mol khí giảm theo phản ứng).
					\item Đo nồng độ của $NO$, $Cl_2$ hoặc $NOCl$ bằng phương pháp quang phổ theo thời gian.
				\end{itemize}
				\item \textbf{Dạng đồ thị}:
				\begin{itemize}
					\item Nếu đo áp suất: Đồ thị biểu diễn sự giảm dần của áp suất theo thời gian, sau đó đạt đến trạng thái ổn định.
					\item Nếu đo nồng độ: Đồ thị biểu diễn sự giảm dần của nồng độ chất phản ứng (NO, $Cl_2$) hoặc sự tăng dần của nồng độ sản phẩm ($NOCl$) theo thời gian, sau đó đạt đến trạng thái ổn định.
				\end{itemize}
			\end{itemize}
		\end{enumerate}
	}
\end{bt}
%%%=============BT_29=============%%%
\begin{bt}
	Thực hiện phản ứng:
	\[2\text{I}\text{Cl} + \text{H}_2 \xrightarrow \text{I}_2 + 2\text{H}\text{Cl}\]
	Nồng độ đầu của $ICl$ và $H_2$ được lấy đúng theo tỉ lệ hợp thức. Nghiên cứu sự thay đổi nồng độ các chất tham gia và chất tạo thành trong phản ứng theo thời gian, thu được đồ thị sau: {Đồ thị biểu diễn sự thay đổi nồng độ của các chất theo thời gian, trong đó (a) và (b) giảm dần, (c) và (d) tăng dần}
	Cho biết các đường (a), (b), (c), (d) tương ứng với sự biến đổi nồng độ các chất nào trong phương trình phản ứng trên. Giải thích.
	\loigiai{
		\begin{itemize}
			\item Đường (a) và (b) biểu diễn sự giảm nồng độ của các chất phản ứng ($ICl$ và $H_2$). Do nồng độ đầu của $ICl$ và $H_2$ được lấy đúng theo tỉ lệ hợp thức (2:1), và hệ số của $ICl$ lớn hơn, nên $ICl$ sẽ giảm nhanh hơn $H_2$. Vậy:
			\begin{itemize}
				\item Đường (a) tương ứng với sự biến đổi nồng độ của $ICl$.
				\item Đường (b) tương ứng với sự biến đổi nồng độ của $H_2$.
			\end{itemize}
			\item Đường (c) và (d) biểu diễn sự tăng nồng độ của các chất sản phẩm ($I_2$ và $HCl$). Do hệ số của $HCl$ lớn hơn, nên $HCl$ sẽ tăng nhanh hơn $I_2$. Vậy:
			\begin{itemize}
				\item Đường (c) tương ứng với sự biến đổi nồng độ của $I_2$.
				\item Đường (d) tương ứng với sự biến đổi nồng độ của $HCl$.
			\end{itemize}
		\end{itemize}
	}
\end{bt}
%%%=============BT_30=============%%%
\begin{bt}
	Phosgen ($COCl_2$) là một chất độc hoá học được sử dụng trong chiến tranh thế giới thứ nhất.
	Phản ứng tổng hợp phosgen như sau:
	\[\text{C}\text{O} + \text{Cl}_2 \xrightarrow \text{C}\text{O}\text{Cl}_2\]
	Biểu thức tốc độ phản ứng có dạng: $v = k \cdot [CO] \cdot [Cl_2]$
	Tốc độ phản ứng thay đổi như nào nếu:
	\begin{enumerate}[a)]
		\item Tăng nồng độ $CO$ lên 2 lần.
		\item Giảm nồng độ $Cl_2$ xuống 4 lần.
	\end{enumerate}
	\loigiai{
		\begin{enumerate}[a)]
			\item Nếu tăng nồng độ $CO$ lên 2 lần:
			\[v' = k \cdot (2[\text{C}\text{O}]) \cdot [\text{Cl}_2] = 2 \cdot k \cdot [\text{C}\text{O}] \cdot [\text{Cl}_2] = 2v\]
			Vậy tốc độ phản ứng tăng lên 2 lần.
			\item Nếu giảm nồng độ $Cl_2$ xuống 4 lần:
			\[v' = k \cdot [\text{C}\text{O}] \cdot \dfrac{[\text{Cl}_2]}{4} = \dfrac{1}{4} \cdot k \cdot [\text{C}\text{O}] \cdot [\text{Cl}_2] = \dfrac{1}{4}v\]
			Vậy tốc độ phản ứng giảm xuống 4 lần.
		\end{enumerate}
	}
\end{bt}

%%%=============BT_31=============%%%
\begin{bt}
	Cho phản ứng hoá học sau:
	\[\text{Zn}(s) + \text{H}_2\text{S}\text{O}_4(aq) \xrightarrow \text{Zn}\text{S}\text{O}_4(aq) + \text{H}_2(g)\]
	\begin{enumerate}[a)]
		\item Ở nhiệt độ phòng, đo được sau 1 phút có $7{,}5 \text{ mL}$ khí hydrogen thoát ra. Tính tốc độ trung bình của phản ứng theo hydrogen.
		\item Ở nhiệt độ thấp, tốc độ phản ứng là $3 \text{ mL/min}$. Hãy tính xem sau bao lâu thì thu được $7{,}5 \text{ mL}$ khí hydrogen.
	\end{enumerate}
	\loigiai{
		\begin{enumerate}[a)]
			\item Tốc độ trung bình của phản ứng theo hydrogen:
			\[v = \dfrac{\\text{Delta} \text{V}_{\text{H}_2}}{\\text{Delta} t} = \dfrac{7{,}5 \text{ m\text{L}}}{1 \text{ min}} = 7{,}5 \text{ m\text{L}/min}\]
			\item Thời gian thu được $7{,}5 \text{ mL}$ khí hydrogen ở nhiệt độ thấp:
			\[t = \dfrac{\text{V}_{\text{H}_2}}{v} = \dfrac{7{,}5 \text{ m\text{L}}}{3 \text{ m\text{L}/min}} = 2{,}5 \text{ min}\]
		\end{enumerate}
	}
\end{bt}

%%%=============BT_32=============%%%
\begin{bt}
	Khi nhiệt độ phòng là $25^\circ C$, cho $10 \text{ g}$ đá vôi (dạng viên) vào cốc đựng $100 \text{ g}$ dung dịch $HCl$ loãng và nhanh chóng cho lên một cân điện tử. Đọc giá trị khối lượng cốc tại thời điểm ban đầu và sau 1 phút.
	Lặp lại thí nghiệm khi nhiệt độ phòng là $35^\circ C$. Kết quả thí nghiệm được ghi trong bảng sau:
	\begin{center}
		\begin{tabular}{|c|c|c|c|c|}
			\hline
			STT & Nhiệt độ ($^\circ C$) & Khối lượng cốc (g) & Thời điểm đầu & Sau 1 phút \\
			\hline
			1 & 25 & 235{,}40 & 235{,}13 \\
			\hline
			2 & 35 & 235{,}78 & 235{,}21 \\
			\hline
		\end{tabular}
	\end{center}
	\begin{enumerate}[a)]
		\item Tính hệ số nhiệt độ của phản ứng.
		\item Giả sử ban đầu cốc chứa dung dịch $HCl$ và đá vôi có khối lượng $235{,}40 \text{ g}$. Thực hiện thí nghiệm ở $45^\circ C$. Hỏi sau 1 phút, khối lượng cốc là bao nhiêu? (Bỏ qua khối lượng nước bay hơi)
	\end{enumerate}
	\loigiai{
		\begin{enumerate}[a)]
			\item Khối lượng $CO_2$ thoát ra ở $25^\circ C$ sau 1 phút:
			\[m_{\text{C}\text{O}_2(25)} = 235{,}40 - 235{,}13 = 0{,}27 \text{ g}\]
			Khối lượng $CO_2$ thoát ra ở $35^\circ C$ sau 1 phút:
			\[m_{\text{C}\text{O}_2(35)} = 235{,}78 - 235{,}21 = 0{,}57 \text{ g}\]
			Tốc độ phản ứng tỉ lệ với khối lượng $CO_2$ thoát ra. Hệ số nhiệt độ $\gamma$:
			\[\gamma = \dfrac{v_{35}}{v_{25}} = \dfrac{m_{\text{C}\text{O}_2(35)}}{m_{\text{C}\text{O}_2(25)}} = \dfrac{0{,}57}{0{,}27} \approx 2{,}11\]
			\item Ở $45^\circ C$, tốc độ phản ứng tăng so với $35^\circ C$ là:
			\[v_{45} = v_{35} \cdot \gamma = 0{,}57 \cdot 2{,}11 \approx 1{,}20 \text{ g/min}\]
			Vậy sau 1 phút, khối lượng $CO_2$ thoát ra là $1{,}20 \text{ g}$.
			Khối lượng cốc sau 1 phút ở $45^\circ C$:
			\[m_{coc(45)} = 235{,}40 - 1{,}20 = 234{,}20 \text{ g}\]
		\end{enumerate}
	}
\end{bt}

%%%=============BT_33=============%%%
\begin{bt}
	Có hai miếng iron có kích thước giống hệt nhau, một miếng là khối iron đặc (A), một miếng có nhiều lỗ nhỏ li ti bên trong và trên bề mặt (B). Thả hai miếng iron vào hai cốc đựng dung dịch $HCl$ cùng thể tích và nồng độ, theo dõi thể tích khí hydrogen thoát ra theo thời gian. Vẽ đồ thị thể tích khí theo thời gian, thu được hai đồ thị sau: {Đồ thị thể hiện đường (1) có tốc độ phản ứng nhanh hơn đường (2)}
	Cho biết đồ thị nào mô tả tốc độ thoát khí từ miếng sắt A, miếng sắt B. Giải thích.
	\loigiai{
		\begin{itemize}
			\item Miếng sắt B có nhiều lỗ nhỏ li ti bên trong và trên bề mặt, do đó diện tích bề mặt tiếp xúc với dung dịch $HCl$ lớn hơn so với miếng sắt A (khối iron đặc).
			\item Diện tích bề mặt tiếp xúc càng lớn, tốc độ phản ứng càng nhanh.
			\item Vậy:
			\begin{itemize}
				\item Đồ thị (1) mô tả tốc độ thoát khí từ miếng sắt B.
				\item Đồ thị (2) mô tả tốc độ thoát khí từ miếng sắt A.
			\end{itemize}
		\end{itemize}
	}
\end{bt}

%%%=============BT_34=============%%%
\begin{bt}
	Xúc tác có hiệu quả cao là xúc tác làm tăng nhanh tốc độ phản ứng. Hai chất $MnO_2$ và $Fe_2O_3$ đều có khả năng xúc tác cho phản ứng phân huỷ $H_2O_2$.
	Đo nồng độ $H_2O_2$ theo thời gian, thu được đồ thị sau: {Đồ thị thể hiện nồng độ H2O2 giảm nhanh hơn khi có xúc tác MnO2 so với xúc tác Fe2O3}
	Cho biết xúc tác nào có hiệu quả hơn. Giải thích.
	\loigiai{
		Từ đồ thị, ta thấy nồng độ $H_2O_2$ giảm nhanh hơn khi có xúc tác $MnO_2$ so với xúc tác $Fe_2O_3$. Điều này chứng tỏ $MnO_2$ làm tăng tốc độ phản ứng phân huỷ $H_2O_2$ nhanh hơn $Fe_2O_3$. Vậy $MnO_2$ có hiệu quả hơn.
	}
\end{bt}

%%%=============BT_35=============%%%
\begin{bt}
	Khí oxygen và hydrogen có thể cùng tồn tại trong một bình kín ở điều kiện bình thường mà không nguy hiểm. Nhưng khi có tia lửa điện hoặc một ít bột kim loại được thêm vào bình thì lập tức có phản ứng mãnh liệt xảy ra và có thể gây nổ.
	\begin{enumerate}[a)]
		\item Tia lửa điện có phải chất xúc tác không? Giải thích.
		\item Bột kim loại có phải chất xúc tác không? Giải thích.
	\end{enumerate}
	\loigiai{
		\begin{enumerate}[a)]
			\item Tia lửa điện không phải chất xúc tác. Tia lửa điện cung cấp năng lượng hoạt hóa cho phản ứng giữa oxygen và hydrogen, giúp phản ứng xảy ra. Chất xúc tác làm giảm năng lượng hoạt hóa, nhưng tia lửa điện cung cấp năng lượng để vượt qua rào cản năng lượng.
			\item Bột kim loại có thể là chất xúc tác. Một số kim loại (ví dụ: platinum, palladium) có khả năng hấp phụ oxygen và hydrogen trên bề mặt, làm tăng nồng độ các chất phản ứng và tạo điều kiện cho phản ứng xảy ra dễ dàng hơn. Bột kim loại không bị tiêu thụ trong phản ứng.
		\end{enumerate}
	}
\end{bt}