%%%=============EX_1=============%%%
\begin{ex}
	Cho phản ứng \(2A + B \xrightarrow C\), có phương trình tốc độ \(v = k[A]^2[B]\). Nếu nồng độ của A tăng gấp đôi và nồng độ của B không đổi, tốc độ phản ứng sẽ:
	\choice
		{Không đổi}
		{Tăng gấp đôi}
		{\True Tăng gấp bốn}
		{Giảm một nửa}
	\loigiai{
		Khi \([A]\) tăng gấp đôi, \([A]^2\) tăng gấp \(2^2 = 4\) lần. Vì \(v \propto [A]^2\), tốc độ phản ứng sẽ tăng gấp bốn.
		}
\end{ex}
%%%=============EX_2=============%%%
\begin{ex}
	Phản ứng \(A \xrightarrow B\) có phương trình tốc độ \(v = k[A]\). Nếu tốc độ phản ứng là \(0.02 \, \text{mol/L·s}\) khi \([A] = 0.4 \, \text{mol/L}\), hằng số tốc độ \(k\) là:
	\choice
		{0.02}
		{0.05}
		{\True 0.05 \, \text{s}^{-1}}
		{0.08}
	\loigiai{
		Áp dụng công thức \(k = \dfrac{v}{[A]} = \dfrac{0.02}{0.4} = 0.05 \, \text{s}^{-1}\).
		}
\end{ex}
%%%=============EX_3=============%%%
\begin{ex}
	Cho phản ứng \(A \xrightarrow B\) là phản ứng bậc nhất với \(k = 0.2 \, \text{s}^{-1}\). Thời gian bán hủy của phản ứng này là:
	\choice
		{2.31 \, \text{s}}
		{6.93 \, \text{s}}
		{0.34 \, \text{s}}
		{\True 3.47 \, \text{s}}
	\loigiai{
		Thời gian bán hủy của phản ứng bậc nhất là \(t_{1/2} = \dfrac{\ln(2)}{k} = \dfrac{0.693}{0.2} \approx 3.47 \, \text{s}\).
		}
\end{ex}
%%%=============EX_4=============%%%
\begin{ex}
	Phản ứng \(A + B \xrightarrow C\) có phương trình tốc độ \(v = k[A][B]^2\). Nếu \([A]\) tăng gấp ba và \([B]\) giảm một nửa, tốc độ phản ứng thay đổi như thế nào?
	\choice
		{Tăng gấp 3/4}
		{\True Giảm 3/4}
		{Tăng gấp 4/3}
		{Giảm 4/3}
	\loigiai{
		Nếu \([A]\) tăng gấp ba và \([B]\) giảm một nửa, tốc độ phản ứng sẽ thay đổi \(v_2 = k(3[A])(\dfrac{1}{2}[B])^2 = k \cdot 3[A] \cdot \dfrac{1}{4}[B]^2 = \dfrac{3}{4}k[A][B]^2\). Vậy tốc độ phản ứng giảm 3/4.
		}
\end{ex}
%%%=============EX_5=============%%%
\begin{ex}
	Cho phản ứng \(A \xrightarrow B\) có năng lượng hoạt hóa \(E_a = 50 \, \text{kJ/mol}\) và hằng số khí \(R = 8.314 \, \text{J/mol·K}\). Nếu nhiệt độ tăng từ 300 K lên 310 K, hằng số tốc độ phản ứng tăng lên bao nhiêu lần?
	\choice
		{Khoảng 1.2 lần}
		{Khoảng 1.5 lần}
		{\True Khoảng 2.0 lần}
		{Khoảng 2.5 lần}
	\loigiai{
		Sử dụng phương trình Arrhenius: \(\ln\left(\dfrac{k_2}{k_1}\right) = \dfrac{E_a}{R} \left(\dfrac{1}{T_1} - \dfrac{1}{T_2}\right)\). Thay số và tính toán, ta được \(\dfrac{k_2}{k_1} \approx 2.0\).
		}
\end{ex}
%%%=============EX_6=============%%%
\begin{ex}
	Phản ứng \(2A \xrightarrow B\) là phản ứng bậc hai với \(k = 0.01 \, \text{L/mol·s}\). Nếu \([A]_0 = 1 \, \text{mol/L}\), thời gian để \([A]\) giảm xuống còn 0.5 mol/L là:
	\choice
		{50 \, \text{s}}
		{100 \, \text{s}}
		{\True 100 \, \text{s}}
		{200 \, \text{s}}
	\loigiai{
		Áp dụng phương trình bậc hai: \(\dfrac{1}{[A]_t} - \dfrac{1}{[A]_0} = kt\). Thay số và giải phương trình, ta được \(t = 100 \, \text{s}\).
		}
\end{ex}
%%%=============EX_7=============%%%
\begin{ex}
	Phản ứng \(A \xrightarrow B\) có phương trình tốc độ \(v = k[A]^2\). Nếu \([A]\) tăng gấp đôi, tốc độ phản ứng sẽ:
	\choice
		{Tăng gấp đôi}
		{Không đổi}
		{Giảm một nửa}
		{\True Tăng gấp bốn}
	\loigiai{
		Vì \(v \propto [A]^2\), khi \([A]\) tăng gấp đôi, tốc độ phản ứng sẽ tăng gấp \(2^2 = 4\) lần.
		}
\end{ex}
%%%=============EX_8=============%%%
\begin{ex}
	Cho phản ứng \(A + B \xrightarrow C\) có bậc riêng phần với A là 1 và bậc riêng phần với B là 0. Phương trình tốc độ của phản ứng là:
	\choice
		{\(v = k[A]^0[B]\)}
		{\(v = k[A][B]\)}
		{\(v = k[A]^2[B]\)}
		{\True \(v = k[A]\)}
	\loigiai{
		Vì bậc riêng phần với A là 1 và bậc riêng phần với B là 0, phương trình tốc độ là \(v = k[A]^1[B]^0 = k[A]\).
		}
\end{ex}
%%%=============EX_9=============%%%
\begin{ex}
	Phản ứng \(A \xrightarrow B\) có hằng số tốc độ \(k = 0.05 \, \text{s}^{-1}\). Nếu \([A]_0 = 0.8 \, \text{mol/L}\), tốc độ phản ứng ban đầu là:
	\choice
		{0.02 \, \text{mol/L·s}}
		{\True 0.04 \, \text{mol/L·s}}
		{0.06 \, \text{mol/L·s}}
		{0.08 \, \text{mol/L·s}}
	\loigiai{
		Tốc độ phản ứng ban đầu \(v = k[A]_0 = 0.05 \cdot 0.8 = 0.04 \, \text{mol/L·s}\).
		}
\end{ex}
%%%=============EX_10=============%%%
\begin{ex}
	Phản ứng \(A \xrightarrow B\) có thời gian bán hủy là 10 giây. Nếu \([A]_0 = 1 \, \text{mol/L}\), sau 20 giây, nồng độ \(A\) còn lại là: (Giả sử phản ứng bậc nhất)
	\choice
		{0.125 \, \text{mol/L}}
		{\True 0.25 \, \text{mol/L}}
		{0.5 \, \text{mol/L}}
		{0 \, \text{mol/L}}
	\loigiai{
		Sau một thời gian bán hủy (10 giây), nồng độ \(A\) giảm còn một nửa. Sau hai thời gian bán hủy (20 giây), nồng độ \(A\) giảm còn một nửa của một nửa, tức là \(1/4\). Vậy \([A] = 0.25 \, \text{mol/L}\).
		}
\end{ex}
%%%=============EX_11=============%%%
\begin{ex}
	Cho phản ứng \(A + B \xrightarrow C\) có phương trình tốc độ \(v = k[A]^m[B]^n\). Biết khi tăng nồng độ A gấp đôi, tốc độ phản ứng tăng gấp 8 lần, và khi tăng nồng độ B gấp đôi, tốc độ phản ứng tăng gấp đôi. Giá trị của m và n lần lượt là:
	\choice
		{\(m = 1, n = 1\)}
		{\(m = 2, n = 1\)}
		{\(m = 1, n = 2\)}
		{\True \(m = 3, n = 1\)}
	\loigiai{
		Khi \([A]\) tăng gấp đôi, \(v \propto [A]^m\), tốc độ tăng gấp 8 lần, suy ra \(m = 3\). Khi \([B]\) tăng gấp đôi, \(v \propto [B]^n\), tốc độ tăng gấp đôi, suy ra \(n = 1\).
		}
\end{ex}
%%%=============EX_12=============%%%
\begin{ex}
	Cho phản ứng \(A \xrightarrow B\) có \(k = 0.02 \, \text{s}^{-1}\) ở 25^\circ C và \(k = 0.08 \, \text{s}^{-1}\) ở 45^\circ C. Phản ứng này có năng lượng hoạt hóa là bao nhiêu? (R = 8.314 J/mol.K)
	\choice
		{57.6 kJ/mol}
		{\True 57.6 J/mol}
		{28.8 kJ/mol}
		{28.8 J/mol}
	\loigiai{
		Sử dụng phương trình Arrhenius: \(\ln\left(\dfrac{k_2}{k_1}\right) = \dfrac{E_a}{R} \left(\dfrac{1}{T_1} - \dfrac{1}{T_2}\right)\). Thay số và tính toán, ta được \(E_a = 57.6 J/mol\).
		}
\end{ex}
%%%=============EX_13=============%%%
\begin{ex}
	Phản ứng \(A \xrightarrow B + C\) là phản ứng bậc nhất. Sau thời gian t, nồng độ chất A giảm đi 75% so với nồng độ ban đầu. Biểu thức nào sau đây đúng với thời gian t?
	\choice
		{\(t = \dfrac{\ln(4)}{k}\)}
		{\(t = \dfrac{\ln(0.25)}{k}\)}
		{\(t = \dfrac{\ln(2)}{k}\)}
		{\True \(t = 2 \cdot t_{1/2}\)}
	\loigiai{
		Sau một thời gian bán hủy, nồng độ A giảm 50%. Sau hai thời gian bán hủy, nồng độ A giảm 75%. Vậy \(t = 2 \cdot t_{1/2}\).
		}
\end{ex}
%%%=============EX_14=============%%%
\begin{ex}
	Phản ứng \(A + B \xrightarrow C\) có phương trình tốc độ \(v = k[A]^2[B]\). Nếu tăng nồng độ A lên 4 lần và giảm nồng độ B xuống 2 lần thì tốc độ phản ứng sẽ thay đổi như thế nào?
	\choice
		{Tăng 2 lần}
		{Tăng 4 lần}
		{\True Tăng 8 lần}
		{Tăng 16 lần}
	\loigiai{
		Tốc độ phản ứng tỉ lệ với \([A]^2\), nên khi tăng A lên 4 lần, tốc độ tăng \(4^2 = 16\) lần. Tốc độ phản ứng tỉ lệ với [B], nên khi giảm B xuống 2 lần, tốc độ giảm 2 lần. Vậy tốc độ phản ứng tăng \(16/2 = 8\) lần.
		}
\end{ex}
%%%=============EX_15=============%%%
\begin{ex}
	Cho phản ứng \(A \xrightarrow B\) có \(k = 0.693 \times 10^{-2} \, \text{s}^{-1}\) ở 27^\circ C. Thời gian bán hủy của phản ứng là:
	\choice
		{10 s}
		{50 s}
		{\True 100 s}
		{200 s}
	\loigiai{
		Thời gian bán hủy \(t_{1/2} = \dfrac{0.693}{k} = \dfrac{0.693}{0.693 \times 10^{-2}} = 100 \, \text{s}\).
		}
\end{ex}
%%%=============EX_16=============%%%
\begin{ex}
	Phản ứng \(A \xrightarrow B\) có phương trình tốc độ \(v = k[A]^2\). Nếu nồng độ ban đầu của A là 1M, sau 100 giây nồng độ A còn lại 0.5M. Hằng số tốc độ k của phản ứng là:
	\choice
		{0.01}
		{\True 0.02}
		{0.03}
		{0.04}
	\loigiai{
		Áp dụng phương trình động học bậc 2: \(\dfrac{1}{[A]_t} - \dfrac{1}{[A]_0} = kt\). Thay số vào, ta được \(k = 0.02\).
		}
\end{ex}
%%%=============EX_17=============%%%
\begin{ex}
	Cho phản ứng \(A \xrightarrow B\) có thời gian bán hủy không phụ thuộc vào nồng độ ban đầu. Phản ứng này thuộc bậc nào?
	\choice
		{Bậc 0}
		{\True Bậc 1}
		{Bậc 2}
		{Bậc 3}
	\loigiai{
		Thời gian bán hủy không phụ thuộc vào nồng độ ban đầu là đặc trưng của phản ứng bậc 1.
		}
\end{ex}
%%%=============EX_18=============%%%
\begin{ex}
	Phản ứng \(A + B \xrightarrow C\) có phương trình tốc độ \(v = k[A][B]\). Nếu tăng nồng độ A lên 2 lần và giảm nồng độ B xuống 2 lần, tốc độ phản ứng sẽ:
	\choice
		{\True Không đổi}
		{Tăng 2 lần}
		{Giảm 2 lần}
		{Tăng 4 lần}
	\loigiai{
		Khi tăng nồng độ A lên 2 lần và giảm nồng độ B xuống 2 lần, tốc độ phản ứng sẽ không đổi vì \(v = k(2[A])(\dfrac{1}{2}[B]) = k[A][B]\).
		}
\end{ex}
%%%=============EX_19=============%%%
\begin{ex}
	Phản ứng \(A \xrightarrow B\) có năng lượng hoạt hóa là 100 kJ/mol. Nếu nhiệt độ tăng từ 27^\circ C lên 37^\circ C, tốc độ phản ứng tăng lên bao nhiêu lần?
	\choice
		{Khoảng 1.5 lần}
		{\True Khoảng 2 lần}
		{Khoảng 2.5 lần}
		{Khoảng 3 lần}
	\loigiai{
		Áp dụng phương trình Arrhenius. Với sự thay đổi nhiệt độ nhỏ, tốc độ phản ứng tăng lên khoảng 2 lần.
		}
\end{ex}
%%%=============EX_20=============%%%
\begin{ex}
	Cho phản ứng \(A \xrightarrow B\) có \(k = 10^{-3} s^{-1}\). Sau thời gian bao lâu thì nồng độ chất A giảm đi một nửa? (ln2 = 0,693)
	\choice
		{346,5 s}
		{\True 693 s}
		{1386 s}
		{2079 s}
	\loigiai{
		Thời gian bán hủy \(t_{1/2} = \dfrac{0.693}{k} = \dfrac{0.693}{10^{-3}} = 693 \, \text{s}\).
		}
\end{ex}
%%%=============EX_21=============%%%
\begin{ex}
	Cho phản ứng: \(A + B \xrightarrow C\). Nếu nồng độ của A tăng gấp đôi và nồng độ của B không đổi, tốc độ phản ứng thay đổi như thế nào?
	\choice
		{Tốc độ phản ứng giảm đi một nửa}
		{Tốc độ phản ứng không đổi}
		{\True Tốc độ phản ứng tăng gấp đôi}
		{Tốc độ phản ứng tăng gấp bốn}
	\loigiai{Tốc độ phản ứng tỉ lệ thuận với nồng độ chất phản ứng. Do đó, khi nồng độ A tăng gấp đôi, tốc độ phản ứng cũng tăng gấp đôi.}
\end{ex}
%%%=============EX_22=============%%%
\begin{ex}
	Phản ứng \(X + Y \xrightarrow Z\) có biểu thức tốc độ \(v = k[X][Y]^2\). Nếu nồng độ X giảm một nửa và nồng độ Y tăng gấp đôi, tốc độ phản ứng thay đổi như thế nào?
	\choice
		{Tốc độ phản ứng giảm đi một nửa}
		{Tốc độ phản ứng không đổi}
		{\True Tốc độ phản ứng tăng gấp đôi}
		{Tốc độ phản ứng tăng gấp bốn}
	\loigiai{
		\[
		v_1 = k[X][Y]^2
		\]
		\[
		v_2 = k\left(\dfrac{[X]}{2}\right)(2[Y])^2 = k\left(\dfrac{[X]}{2}\right)(4[Y]^2) = 2k[X][Y]^2 = 2v_1.
		\]
		Vậy tốc độ phản ứng tăng gấp 2 lần.
		}
\end{ex}
%%%=============EX_23=============%%%
\begin{ex}
	Trong phản ứng phân hủy \( 2H_2O_2 \xrightarrow 2H_2O + O_2 \), tại sao khi thêm \( MnO_2 \), tốc độ phản ứng tăng lên?
	\choice
		{ \( MnO_2 \) làm giảm nồng độ các chất phản ứng}
		{ \( MnO_2 \) làm tăng năng lượng hoạt hóa của phản ứng}
		{\True \( MnO_2 \) cung cấp một cơ chế phản ứng khác với năng lượng hoạt hóa thấp hơn}
		{ \( MnO_2 \) làm giảm nhiệt độ của hệ phản ứng}
	\loigiai{\( MnO_2 \) là chất xúc tác, nó cung cấp một cơ chế phản ứng khác với năng lượng hoạt hóa thấp hơn, tăng số lượng va chạm hiệu quả giữa các phân tử \( H_2O_2 \), từ đó làm tăng tốc độ phản ứng.}
\end{ex}
