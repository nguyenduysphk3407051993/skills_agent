9.1. Điều nào dưới đây đúng khi nói về ion $S^{2-}$?
A. Có chứa 18 proton.
B. Có chứa 18 electron.
C. Trung hoà về điện.
D. Được tạo thành khi nguyên tử sulfur (S) nhận vào 2 proton.

9.2. Điều nào dưới đây **không** đúng khi nói về hợp chất sodium oxide ($Na_2O$)?
A. Trong phân tử $Na_2O$, các ion sodium $Na^+$ và ion oxide $O^{2-}$ đều đạt cấu hình electron bền vững của khí hiếm neon.
B. Phân tử $Na_2O$ tạo bởi lực hút tĩnh điện giữa hai ion $Na^+$ và một ion $O^{2-}$.
C. Là chất rắn trong điều kiện thường.
D. Không tan trong nước, chỉ tan trong dung môi không phân cực như benzene, carbon tetrachloride,...

9.3. Tính chất nào dưới đây đúng khi nói về hợp chất ion?
A. Hợp chất ion có nhiệt độ nóng chảy thấp.
B. Hợp chất ion tan tốt trong dung môi không phân cực.
C. Hợp chất ion có cấu trúc tinh thể.
D. Hợp chất ion dẫn điện ở trạng thái rắn.

9.4. Hợp chất A có các tính chất sau: Ở thể rắn trong điều kiện thường, dễ tan trong nước tạo dung dịch dẫn điện được. Hợp chất A là
A. sodium chloride.
B. glucose.
C. sucrose.
D. fructose.
9.5. Tính chất nào sau đây không phải của magnesium oxide $(MgO)$?
A. Có nhiệt độ nóng chảy cao hơn so với $NaCl$.
B. Chất khí ở điều kiện thường.
C. Có cấu trúc tinh thể.
D. Phân tử tạo bởi lực hút tĩnh điện giữa ion $Mg^{2+}$ và $O^{2-}$.

9.6. Sodium sulfide $(Na_2S)$ là một hợp chất hoá học được sử dụng trong ngành công nghiệp giấy và bột giấy, xử lí nước, công nghiệp dệt may và các quy trình sản xuất hoá chất khác như sản xuất cao su, thuốc nhuộm lưu huỳnh và thu hồi dầu,... Điều thú vị là sodium sulfide đã được chứng minh là có vai trò trong bảo vệ tim mạch, chống lại chứng thiếu máu cục bộ ở tim và giúp bảo vệ phổi, chống lại tổn thương phổi do máy thở. Trình bày sự tạo thành sodium sulfide khi cho sodium phản ứng với sulfur.
9.7. Chỉ ra cấu trúc đúng của ô mạng tinh thể sodium chloride:
A. {$Hình\ vẽ\ A$}
$= Cl^-$ $= Na^+$
B. {$Hình\ vẽ\ B$}
$= Cl^-$ $= Na^+$
C. {$Hình\ vẽ\ C$}
$= Cl^-$ $= Na^+$
D. {$Hình\ vẽ\ D$}
$= Cl^-$ $= Na^+$
9.8. Magnesium chloride là một chất xúc tác phổ biến trong hoá học hữu cơ. Trình bày sự hình thành phân tử $MgCl_2$ khi cho magnesium tác dụng với chlorine.
9.9. Trong đời sống, muối ăn ($NaCl$) và các gia vị, phụ gia ($C_2H_5NO_2Na$: bột ngọt; $C_2H_3O_2Na$: chất bảo quản thực phẩm) đều có chứa ion sodium. Hiệp hội Tim mạch Hoa Kỳ khuyến cáo các cá nhân nên hạn chế lượng sodium xuống dưới $2,300$ mg mỗi ngày vì nếu tiêu thụ nhiều hơn sẽ ảnh hưởng đến tim mạch và thận. Nếu trung bình mỗi ngày, một người dùng tổng cộng $5,0$ gam muối ăn; $0,5$ gam bột ngọt và $0,05$ gam chất bảo quản thì lượng sodium tiêu thụ có vượt mức giới hạn cho phép nói trên không?
9.10. Trình bày cách vẽ một ô mạng tinh thể $NaCl$.
9.11*. Biểu đồ dưới đây cho biết mối quan hệ giữa năng lượng của hệ các ion trái dấu so với khoảng cách giữa chúng:
Biểu đồ cho thấy khoảng cách giữa các ion càng gần càng thuận lợi để hệ đạt được trạng thái năng lượng tối thiểu (bền vững). Tuy nhiên, ở khoảng cách nhỏ quá, các ion lại đẩy nhau do hạt nhân của các ion đều mang điện tích dương. Năng lượng tối thiểu đại diện cho độ bền liên kết và khoảng cách $r_o$ tại mức năng lượng tối thiểu gọi là độ dài liên kết. Bằng cách thực hiện một loạt các phép tính, người ta thấy rằng các hợp chất ion được hình thành bởi các ion có điện tích lớn hơn sẽ tạo ra liên kết mạnh hơn và các hợp chất ion có độ dài liên kết ngắn hơn sẽ hình thành liên kết mạnh hơn.
Sử dụng nhận định trên để dự đoán và giải thích độ bền liên kết giữa các hợp chất ion sau:
a) $NaCl$ và $Na_2O$.
b) $NaCl$ và $NaF$.
9.12*. $X$, $Y$, $Z$ là các hợp chất ion thuộc trong số các chất sau: $NaF$, $MgO$ và $MgCl_2$. Nhiệt độ nóng chảy của các hợp chất $X$, $Y$, $Z$ được thể hiện qua biểu đồ:
Trình bày cách xác định các chất $X$, $Y$, $Z$.

9.13. Cho biết lực hút tĩnh điện được tính theo công thức sau:
\[F = k \dfrac{|q_1||q_2|}{r^2}\]
($q_1$, $q_2$ là giá trị điện tích của hai điện tích điểm, đơn vị là $C$ (coulomb); $r$ là khoảng cách giữa hai điện tích điểm, đơn vị là $m$ (meter); $k$ là hằng số coulomb). Dựa vào công thức trên, hãy so sánh gần đúng lực hút tĩnh điện giữa các ion trái dấu trong phân tử NaCl và phân tử MgO. Từ đó, cho biết nhiệt độ nóng chảy và nhiệt độ sôi của hợp chất nào cao hơn.

9.14. Hình dạng và cấu trúc tinh thể của mọi hợp chất ion có giống nhau không? Giải thích.

9.15. Vì sao các hợp chất ion thường tồn tại ở trạng thái rắn và cứng trong điều kiện thường, nhưng lại giòn, dễ vỡ?

9.16. Vì sao nói sodium chloride có cấu trúc mạng tinh thể kiểu lập phương tâm diện?




