\def\myytb{}
\def\myqrcodeytb{}
\def\myqrcodezalo{}
\def\quetmaqr{}
\def\thamgianhomhoctap{}
\newpage
\setcounter{ex}{0}
%%%======================%%%
\Tieudegiua[\maunhan]{Đề cương ôn tập môn tiếng việt lớp 5}

\begin{center}
    {\LARGE\sffamily \textbf{Biển có hai màu}}
\end{center}

Rời thềm lục địa Vũng Tàu với màu biển thoáng xanh, theo tàu thẳng tiến khơi xa ra quần đảo Trường Sa nước mình, bạn sẽ được mục kích vùng biển tổ quốc bao la hai màu. Chúng đan vào nhau như tấm thảm đại dương bát ngát, xanh lá cây trong vắt gương soi từng mảng san hô nhấp nhô khổng lồ và xanh dương sẫm màu là hai vạn dặm dưới đáy biển dạt dào tôm cá. Hoàng hôn nhuộm ráng cam đỏ rực biển chiều, lấp lánh hàng đàn cá chuồn búng mình ngoạn mục trên mặt sóng lô xô…

Đi với hai màu biển là hàng trăm đảo nổi đảo chìm của quần đảo Trường Sa trải dài trên vùng biển rộng gần 200.000km2. Thiên nhiên đảo nổi thật bình yên với bạt ngàn rừng cây xanh chắn gió, ngan ngát muôn loài hoa dại và những chú chim én bay là là mặt đất rất dạn hơi người. “Trùm” cây xanh ở Trường Sa là họ hàng nhà phong ba bão táp có tán rộng; lá to, dày; hoa trắng li ti ken dày quanh cuống lá. Cây phong ba mọc khắp nơi, làm “người hùng” trên bãi chắn sóng, toả bóng mát nơi thao trường, là dáng xanh duyên dáng trong doanh trại bộ đội, xoè tán chở che cho trẻ con trên đảo chơi lò cò, bắn bi, đuổi bắt…

Đảo nổi Sơn Ca ngày đầu hè vàng ruôm ánh nắng bên giàn mướp trĩu trái có đàn bướm trắng chấp chới bay, người lính trẻ đâu đấy vừa hát nghêu ngao vừa lau súng. Doanh trại kiên cố hoặc nguyên sơ trên đảo chìm đều có nhiều lính trẻ đang sống và làm việc. Những chàng trai trẻ mắt sáng môi tươi vào đời phơi phới một tình yêu.

Trường Sa xa ngái nhưng cũng thật gần trong những ai đã một lần đặt chân đến đảo nổi đảo chìm lô xô sóng bạc.

Ở nơi xa ấy, giữa tiếng gà gáy trưa bên triền cát tím màu hoa muống biển, có những người lính dãi dầu nắng mưa căng mình giữ đảo.Ở nơi xa ấy, nơi bọn trẻ hồn nhiên rượt đuổi nhau quanh cột mốc chủ quyền biển đảo, là sự sống nghiêng mình kính cẩn trước một tình yêu bất biến.Ở nơi xa ấy, nơi chót vót những ngọn đèn biển chong mình thao thức, có dải mây trời vắt ngang để minh chứng rằng sông núi nước Nam…

{\vphantom{x} \hfill \textbf{\textit{NGUYỄN THU TRÂN}}}

{\textit{Em hãy khoanh tròn chữ cái trước ý trả lời đúng nhất cho từng câu dưới đây :}}
\tatloigiaiex
%%%============EX_1==============%%%
\begin{ex}
	Hai màu nước biển ở Trường Sa là màu:
	\choice
	{Xanh dương, đỏ rực}
	{Xanh dương,xanh lá cây}
	{Đỏ rực, xanh lá cây}
	{Cam,đỏ rực}
	\loigiai{}
\end{ex}
%%%============EX_2==============%%%
\begin{ex}
	Cây phong ba ở Trường Sa có tác dụng gì?
	\choice
	{Chắn gió, chắn sóng biển và cho bóng mát}
	{Chắn sóng biển, làm đẹp nơi doanh trại}
	{Chắn sóng, toả bóng mát nơi thao trường}
	{làm “người hùng” trên bãi chắn sóng}
	\loigiai{}
\end{ex}
%%%============EX_3==============%%%
\begin{ex}
	Hình ảnh ở Trường Sa gần gũi, quen thuộc với cuộc sống người dân ở đất liền là:
	\choice
	{Những chú chim én bay là là mặt đất}
	{Từng mảng san hô nhấp nhô khổng lồ}
	{Giàn mướp trĩu trái có đàn bướm trắng chấp chới bay}
	{Hàng đàn cá chuồn búng mình ngoạn mục trên mặt sóng lô xô}
	\loigiai{}
\end{ex}
%%%============EX_4==============%%%
\begin{ex}
	Ở đoạn cuối bài, tác giả đã cố ý lặp từ ngữ “Ở nơi xa ấy” vào mỗi đầu câu với mục đích muốn nhấn mạnh những nội dung nào trong câu?
	\choice
	{Quần đảo Trường Sa thuộc chủ quyền của Việt Nam luôn được bảo vệ bởi những người lính kiên cường với lòng yêu nước mãnh liệt}
	{Trường Sa tuy rất xa nhưng cuộc sống nới ấy thật yên bình, cảnh vật và con người luôn hòa quyện với nhau.Quân dân đoàn kết một lòng giữ đảo}
	{Ca ngợi tinh thần kiên cường,dũng cảm của những người lính ngày đêm bảo vệ vùng biển quê hương}
	{Dù Trường Sa xa xôi nhưng toàn dân ta luôn hướng về nơi đóđể cùng đoàn kết, một lòng yêu nước bảo vệ vùng biển than yêu của Tổ quốc}
	\loigiai{}
\end{ex}
%%%============EX_5==============%%%
\begin{ex}
	Em hiểu từ bất biến trong cụm từ “là sự sống nghiêng mình kính cẩn trước một tình yêu bất biến. ”là gì?
	\choice
	{Không tan biến}
	{Không biến mất}
	{Không thay đổi}
	{Không chuyển biến}
	\loigiai{}
\end{ex}
%%%============EX_6==============%%%
\begin{ex}
	Câu “Cây phong ba có tán rộng; lá to, dày.”có mấy tính từ?Đó là những từ nào?
	\choice
	{1 từ. Đó là từ: rộng}
	{2 từ. Đó là từ: rộng, dày}
	{3 từ. Đó là từ: rộng, to,dày}
	{4 từ. Đó là từ: có, rộng, to,dày}
	\loigiai{}
\end{ex}
%%%============EX_7==============%%%
\begin{ex}
	Từ in đậm trong dòng nào dưới đây được dung theo nghĩa gốc?
	\choice
	{Những chú chim én bay là là mặt đất rất dạn hơi người.}
	{Hoàng hôn nhuộm ráng cam đỏ rực biển chiều}
	{Những ngọn đèn biển chong mình thao thức}
	{Có dải mây trời vắt ngang để minh chứng rằng sông núi nước Nam}
	\loigiai{}
\end{ex}
%%%============EX_8==============%%%
\begin{ex}
	Trong hai câu: “Doanh trại kiên cố hoặc nguyên sơ trên đảo chìm đều có nhiều lính trẻ đang sống và làm việc. Những chàng trai trẻ mắt sáng môi tươi vào đời phơi phới một tình yêu.”, câu in đậm đã liên kết với câu đứng trước nó bằng cách nào?ùng từ ngữ nối và lặp từ ngữ.
	\choice
	{Dùng từ ngữ nối và thay thế từ ngữ}
	{Thay thế từ ngữ và lặp từ ngữ}
	{Lặp từ ngữ}
	{}
	\loigiai{}
\end{ex}
%%%============EX_9==============%%%
\begin{ex}
	Những từ nào trong câu: “Thiên nhiên đảo nổi thật bình yên với bạt ngàn rừng cây xanh chắn gió, ngan ngát muôn loài hoa dại và những chú chim én bay là là mặt đất.” là quan hệ từ?
	\choice
	{với}
	{với, và, muôn}
	{với, và}
	{với, và, thật}
	\loigiai{}
\end{ex}

\begin{ex}
	Đặt một câu có sử dụng dấu phẩy ngăn cách các bộ phận cùng chức vụ trong câu 
	\Pointilles{4}
	\loigiai{}
\end{ex}