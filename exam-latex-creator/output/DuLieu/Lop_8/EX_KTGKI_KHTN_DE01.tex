%%%==============Cau_EX1==============%%%
\begin{ex}
	Trong các quá trình sau, quá trình nào xảy ra biển đổi vật lí
	\choice
	{Nước hồ bị bốc hơi khi trời nắng}
	{Diệm bị chảy khi quẹt vào vỏ hộp diêm}
	{Thịt bị cháy khi nướng}
	{Pháo hoa có nhiều màu sắc sặc sỡ}
	\loigiai{}
\end{ex}
%%%==============HetCau_EX1==============%%%

%%%==============Cau_EX2==============%%%
\begin{ex}
	Trong các quá trình sau, quá trình nào xảy ra biến đổi hóa học?
	\choice
	{Bóng đèn phát sáng, kèm theo tỏa nhiệt}
	{Hòa tan đường vào nước để được nước đường}
	{Đun nóng đường, đường chảy rồi chuyển thành màu đen, có mùi hắc}
	{Trời nắng, nước bốc hơi thành hình mây}
	\loigiai{}
\end{ex}
%%%==============HetCau_EX2==============%%%

%%%==============Cau_EX3==============%%%
\begin{ex}
	Cho các quá trình sau:
	\begin{enumerate}[(1)]
		\item  Sắt (iron) được cắt nhỏ và tán thành định.
		\item  Vành xe đạp bằng sắt bị phủ một lớp gỉ là chất màu nâu đỏ.
		\item  Rượu để lâu trong không khí thường bị chua.
		\item  Đèn tín hiệu chuyển từ màu xanh sang màu đỏ.
		\item  Dây tóc trong bóng đèn điện nóng và sáng lên khi dòng điện đi qua.
	\end{enumerate}
	Số quá trình xảy ra biển đổi hóa học là
	\choice
	{$4$}
	{$3$}
	{$2$}
	{$1$}
	\loigiai{}
\end{ex}
%%%==============HetCau_EX3==============%%%

%%%==============Cau_EX4==============%%%
\begin{ex}
	Trong các quá trình sau, quá trình nào xảy ra biến đổi hóa học?
	\choice
	{Hiện tượng băng tan}
	{Hòa tan vôi sống vào nước thu được vôi tôi}
	{Nhỏ vài giọt mực vào cốc nước và khuấy đều thấy mực loang ra cả cốc nước}
	{Mặt trời mọc lên, dưới ánh nắng mặt trời làm cho các giọt sương tan dần}
	\loigiai{}
\end{ex}
%%%==============HetCau_EX4==============%%%

%%%==============Cau_EX5==============%%%
\begin{ex}
	Sản phẩm của phản ứng: Sodium+Oxygen $\to$ Sodium oxide là
	\choice
	{Sodium}
	{Oxygen}
	{Sodium oxide}
	{Sodium oxide và oxygen}
	\loigiai{}
\end{ex}
%%%==============HetCau_EX5==============%%%

%%%==============Cau_EX6==============%%%
\begin{ex}
	Phản ứng hóa học là
	\choice
	{Quá trình kết hợp các đơn chất thành hợp chất}
	{Quá trình biến đổi chất này thành chất khác}
	{Sự trao đổi của 2 hay nhiều chất ban đầu để tạo chất mới}
	{Là quá trình phân hủy chất ban đầu thành nhiều chất}
	\loigiai{}
\end{ex}
%%%==============HetCau_EX6==============%%%

%%%==============Cau_EX7==============%%%
\begin{ex}
	Các câu sau, câu nào sai?
	\choice
	{Trong phản ứng hoá học các nguyên tử được bảo toàn}
	{Trong phản ứng hoá học, các nguyên tử bị phân chia}
	{Trong phản ứng hoá học, các phân tử bị phân chia}
	{Trong phản ứng hoá học, các phân tử bị phá vỡ}
	\loigiai{}
\end{ex}
%%%==============HetCau_EX7==============%%%

%%%==============Cau_EX8==============%%%
\begin{ex}
	Phản ứng nào sau đây là phản ứng toả nhiệt"
	\choice
	{Phản ứng nung đá vôi $\mathrm{CaCO}_3$}
	{Phản ứng đốt cháy khi gas}
	{Phản ứng hòa tan viên C sủi vào nước}
	{Phản ứng phân hủy đường}
	\loigiai{}
\end{ex}
%%%==============HetCau_EX8==============%%%

%%%==============Cau_EX9==============%%%
\begin{ex}
	Điền vào chỗ trống: \lq\lq Trong một phản ứng hóa học, tổng khối lượng của các chất sản phẩm\ldots tổng khối lượng của các chất phản ứng\rq\rq\
	\choice
	{Lớn hơn}
	{Nhỏ hơn}
	{Bằng}
	{Nhỏ hơn hoặc bằng}
	\loigiai{}
\end{ex}
%%%==============HetCau_EX9==============%%%

%%%==============Cau_EX10==============%%%
\begin{ex}
	Trong phản ứng hóa học, yếu tố nào sau đây không thay đổi:
	\choice
	{Số phân tử trước và sau phản ứng}
	{Liên kết giữa các nguyên tử trước và sau phản ứng}
	{Số nguyên tử mỗi nguyên tố trước và sau phản ứng}
	{Trạng thái chất trước và sau phản ứng}
	\loigiai{}
\end{ex}
%%%==============HetCau_EX10==============%%%

%%%==============Cau_EX11==============%%%
\begin{ex}
	Cho biết tỉ lệ giữa các chất tham gia phản ứng trong phương trình hóa học sau:
	$$ 
	\mathrm{Ba}(OH)_2+\mathrm{CuSO}_4 \to \mathrm{Cu}(OH)_2+\mathrm{BaSO}_4
	$$ 
	\choice
	{$1:1$}
	{$1:2$}
	{$2:1$}
	{$2:3$}
	\loigiai{}
\end{ex}
%%%==============HetCau_EX11==============%%%

%%%==============Cau_EX12==============%%%
\begin{ex}
	Một vật thể bằng Iron $(\mathrm{Fe})$ để ngoài trời, sau một thời gian bị gỉ sét. Hỏi khối lượng của vật thay đổi thế nào so với khối lượng của vật trước khi gỉ sét?
	\choice
	{Tăng}
	{Giảm}
	{Không thay đổi}
	{Không thể biết}
	\loigiai{}
\end{ex}
%%%==============HetCau_EX12==============%%%

%%%%==============Cau_EX13==============%%%
\begin{ex}
	Số Avogadro có giá trị là?
	\choice
	{$6,022\cdot 10^{23}$}
	{$6,023.10^{23}$}
	{$6,021.10^{21}$}
	{$6.10^{22}$}
	\loigiai{}
\end{ex}
%%%%==============HetCau_EX13==============%%%

%%%==============Cau_EX14==============%%%
\begin{ex}
	1 mol chất khí ở điều kiện chuẩn (nhiệt độ 25 C, áp suất 1 bar) có thể tích là:
	\choice
	{$2{,}24$ lit}
	{$24{,}79$ lit}
	{$22{,}4$ lit}
	{$24,79\mathrm{ml}$}
	\loigiai{}
\end{ex}
%%%==============HetCau_EX14==============%%%

%%%==============Cau_EX15==============%%%
\begin{ex}
	Khi nào nhẹ nhất trong tất cả các khí sau $(C=12, O=16, H=1, \mathrm{He}=4)$ 
	\choice
	{Khí methane $\left(CH_4\right)$}
	{Khí carbon dioxide $\left(CO_2\right)$}
	{Khí helium (He)}
	{Khí hydrogen $\left(H_2\right)$}
	\loigiai{}
\end{ex}
%%%==============HetCau_EX15==============%%%

%%%==============Cau_EX16==============%%%
\begin{ex}
	Mưa acid được phát hiện ra đầu tiên năm 1948 tại Thụy Điển. Nguyên nhân của tình trạng mưa acid bắt nguồn từ việc con người tiêu thụ nhiều nguyên liệu tự nhiên như than đỏ, dầu mở cho quá trình sông, phát triển sản xuất. Một trong những tác nhân gây ra hiện tượng mưa acid kể trên là chất khi A có công thức phân tử dụng RO. Biết tỉ khối khí A so với Hydrogen là 32. Công thức phân tử của khí A là?
	\choice
	{$SO_2$}
	{$CO_2$}
	{$NO$}
	{$H_2\mathrm{~S}$}
	\loigiai{}
\end{ex}
%%%==============HetCau_EX16==============%%%

%%%==============Cau_EX17==============%%%
\begin{ex}
	Khi đốt củi, để tăng tốc độ cháy, người ta sử dụng biện pháp nào sau đây?
	\choice
	{Đốt trong lò kín}
	{Xếp củi chặt khít}
	{Thổi không khí khô}
	{Thổi hởi nước}
	\loigiai{}
\end{ex}
%%%==============HetCau_EX17==============%%%

%%%==============Cau_EX18==============%%%
\begin{ex}
	Chất xúc tác là chất
	\choice
	{Làm tăng tốc độ của phản ứng}
	{Làm tăng tốc độ của phản ứng nhưng không bị thay đổi sau phản ứng}
	{Làm tăng tốc độ của phản ứng và bị thay đổi sau phản ứng}
	{Làm giảm tốc độ của phản ứng và bị thay đổi sau phản ứng}
	\loigiai{}
\end{ex}
%%%==============HetCau_EX18==============%%%

%%%==============Cau_EX19==============%%%
\begin{ex}
	Những yếu tố nào sau đây ảnh hưởng đến tốc độ của phản ứng hóa học? (1) Diện tích bề mặt tiếp xúc (2) Nhiệt độ (3) Nồng độ (4) Chất xúc tác
	\choice
	{$(1)(2)(3)$}
	{$(1)(3)$ (4)}
	{$(2)(3)$ (4)}
	{(1) $(2)(3)(4)$}
	\loigiai{}
\end{ex}
%%%==============HetCau_EX19==============%%%

%%%==============Cau_EX20==============%%%
\begin{ex}
	Hãy sắp xếp các phản ứng sau theo chiều tăng dần tốc độ của phản ứng
	(1) Phản ứng than cháy trong không khí
	(2) Phản ứng gỉ sắt (iron)
	(3) Phản ứng nổ của khi bình ga
	\choice
	{$(1)(2)(3)$}
	{$(3)(2)(1)$}
	{$(2)(1)(3)$}
	{$(2)(3)(1)$}
	\loigiai{}
\end{ex}
%%%==============HetCau_EX20==============%%%

%%%==============Cau_EX21==============%%%
\begin{ex}
	Số nguyên tử hydrogen trong $0,05\mathrm{~mol}$ khí hydrogen là
	\choice
	{$0\cdot 301\cdot 10^{22}$}
	{$3,01\cdot 10^{23}$}
	{$6,022\cdot 10^{22}$}
	{$0\cdot 6022\cdot 10^{22}$} 
	\loigiai{}
\end{ex}
%%%==============HetCau_EX21==============%%%

%%%==============Cau_EX22==============%%%
\begin{ex}
	Để tính khối lượng và số mol của chất phản ứng và sản phẩm trong một phản ứng hóa học, ta thực hiện theo mấy bước
	\choice
	{1 bước}
	{2 bước}
	{3 bước}
	{4 bước}
	\loigiai{}
\end{ex}
%%%==============HetCau_EX22==============%%%

%%%==============Cau_EX23==============%%%
\begin{ex}
	Cho phương trình hóa học: $\mathrm{Zn}+2\mathrm{HCl} \to \mathrm{ZnCl}_2+H_2$. Để thu được 2.479 lít khí $H_2$ ở (dkc) thì cần bao nhiêu mol Zn?
	\choice
	{$0,3\mathrm{~mol}$}
	{$0,1\mathrm{~mol}$}
	{$0,2\mathrm{~mol}$}
	{$0,5\mathrm{~mol}$}
	\loigiai{}
\end{ex}
%%%==============HetCau_EX23==============%%%

%%%==============Cau_EX24==============%%%
\begin{ex}
	Trong phòng thí nghiệm, khí Oxygen, được điều chế bằng cách nhiệt phân $\mathrm{KMnO}_4$. Phương trình: $2\mathrm{KMnO}_4\to \mathrm{KMnO}_2+\mathrm{MnO}_2+O_2$ 
	Khi đem nhiệt phân hoàn toàn 7,9 gam $\mathrm{KMnO}_4$, thì khối lượng khí $O_2$ thu được là
	\choice
	{$0{,}8$ gam}
	{$1{,}6$ gam}
	{$0{,}4$ gam}
	{$0{,}2$ gam}
	\loigiai{}
\end{ex}
%%%==============HetCau_EX24==============%%%

%%%==============Cau_EX25==============%%%
\begin{ex}
	Để điều chế được 12,8 gam Cu theo phương trình: $\mathrm{CuO}+H_2\to \mathrm{Cu}+H_2O$ thì cần dùng bao nhiêu lít $H_2$ ở đkc?
	\choice
	{$6{,}198$ lít}
	{$3{,}719$ lít}
	{$4{,}958$ lít}
	{$2{,}479$ lít}
	\loigiai{}
\end{ex}
%%%==============HetCau_EX25==============%%%

%%%==============Cau_EX26==============%%%
\begin{ex}
	Dung dịch là:
	\choice
	{Hỗn hợp gồm dung môi và chất tan}
	{Hỗn hợp gồm dung môi và nước}
	{Hỗn hợp gồm nước và chất tan}
	{Hỗn hợp đồng nhất gồm dung môi và chất tan}
	\loigiai{}
\end{ex}
%%%==============HetCau_EX26==============%%%

%%%==============Cau_EX27==============%%%
\begin{ex}
	Hai chất không thể hòa tan với nhau tạo thành dung dịch là:
	\choice
	{Nước và đường}
	{Dầu ăn và cắt}
	{Rượu và nước}
	{Dầu ăn và xăng}
	\loigiai{}
\end{ex}
%%%==============HetCau_EX27==============%%%

%%%==============Cau_EX28==============%%%
\begin{ex}
	Khi cho đường vào nước rồi đun lên, độ tan của đường trong nước sẽ thay đổi như thế nào?
	\choice
	{Tăng lên}
	{Giảm đi}
	{Không đổi}
	{Không xác định}
	\loigiai{}
\end{ex}
%%%==============HetCau_EX28==============%%%

%%%==============Cau_EX29==============%%%
\begin{ex}
	Số $\mathrm{mol} \mathrm{CuSO}_4$, có trong 100 ml dung dịch $\mathrm{CuSO}_40,5M$ là:
	\choice
	{$0,5\mathrm{~mol}$}
	{$0,4\mathrm{~mol}$}
	{$0,05\mathrm{~mol}$}
	{$1$ mol}
	\loigiai{}
\end{ex}
%%%==============HetCau_EX29==============%%%

%%%==============Cau_EX30==============%%%
\begin{ex}
	Thể tích của $0,25\mathrm{~mol}$ khí hydrogen ở đkc là
	\choice
	{$61{,}975$ lít}
	{$0{,}01$ lít}
	{$5{,}6$ lít}
	{$6{,}1975$ lít}
	\loigiai{}
\end{ex}
%%%==============HetCau_EX30==============%%%