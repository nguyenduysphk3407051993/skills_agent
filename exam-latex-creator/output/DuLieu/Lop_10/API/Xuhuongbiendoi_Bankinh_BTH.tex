%%%%=================EX_01====================%%%
\begin{ex}
    Trong một chu kỳ của bảng tuần hoàn, từ trái sang phải, bán kính nguyên tử có xu hướng:
    \choice
    {Tăng dần}
    {Không thay đổi}
    {\True Giảm dần}
    {Tăng rồi giảm}
    \loigiai{Trong một chu kỳ, từ trái sang phải, số proton trong hạt nhân tăng dần trong khi số lớp electron không đổi. Điều này làm tăng lực hút giữa hạt nhân và electron, kéo các electron lại gần hạt nhân hơn, dẫn đến bán kính nguyên tử giảm dần.}
\end{ex}

%%%%=================EX_02====================%%%
\begin{ex}
    Trong một nhóm của bảng tuần hoàn, từ trên xuống dưới, bán kính nguyên tử có xu hướng:
    \choice
    {\True Tăng dần}
    {Giảm dần}
    {Không thay đổi}
    {Giảm rồi tăng}
    \loigiai{Trong một nhóm, từ trên xuống dưới, số lớp electron tăng dần. Điều này làm tăng khoảng cách từ hạt nhân đến lớp electron ngoài cùng, dẫn đến bán kính nguyên tử tăng dần.}
\end{ex}

%%%%=================EX_03====================%%%
\begin{ex}
    Yếu tố nào sau đây không ảnh hưởng trực tiếp đến bán kính nguyên tử?
    \choice
    {Số proton trong hạt nhân}
    {Số lớp electron}
    {Hiệu ứng che chắn}
    {\True Số neutron trong hạt nhân}
    \loigiai{Số neutron trong hạt nhân không ảnh hưởng trực tiếp đến bán kính nguyên tử. Các yếu tố ảnh hưởng chính bao gồm số proton (quyết định lực hút của hạt nhân), số lớp electron, và hiệu ứng che chắn của các electron nội.}
\end{ex}

%%%%=================EX_04====================%%%
\begin{ex}
    So với nguyên tử trung hòa tương ứng, bán kính của ion dương (cation) luôn:
    \choice
    {Lớn hơn}
    {\True Nhỏ hơn}
    {Bằng nhau}
    {Không thể xác định}
    \loigiai{Ion dương (cation) luôn có bán kính nhỏ hơn nguyên tử trung hòa tương ứng. Điều này là do cation mất electron, làm giảm hiệu ứng che chắn và tăng lực hút giữa hạt nhân và các electron còn lại, kéo chúng lại gần hạt nhân hơn.}
\end{ex}

%%%%=================EX_05====================%%%
\begin{ex}
    So với nguyên tử trung hòa tương ứng, bán kính của ion âm (anion) luôn:
    \choice
    {\True Lớn hơn}
    {Nhỏ hơn}
    {Bằng nhau}
    {Không thể xác định}
    \loigiai{Ion âm (anion) luôn có bán kính lớn hơn nguyên tử trung hòa tương ứng. Điều này là do anion nhận thêm electron, làm tăng hiệu ứng che chắn và tăng lực đẩy giữa các electron, đẩy chúng ra xa hạt nhân hơn.}
\end{ex}

%%%%=================EX_06====================%%%
\begin{ex}
    Trong các ion đẳng electron (có cùng số electron), khi điện tích hạt nhân tăng, bán kính ion sẽ:
    \choice
    {Tăng}
    {\True Giảm}
    {Không đổi}
    {Tăng rồi giảm}
    \loigiai{Đối với các ion đẳng electron, khi điện tích hạt nhân tăng (số proton tăng) mà số electron không đổi, lực hút giữa hạt nhân và electron mạnh hơn, kéo các electron lại gần hơn, làm giảm bán kính ion.}
\end{ex}

%%%%=================EX_07====================%%%
\begin{ex}
    Nguyên nhân chính làm cho bán kính nguyên tử giảm dần trong một chu kỳ là:
    \choice
    {Số lớp electron tăng}
    {Hiệu ứng che chắn tăng}
    {\True Điện tích hạt nhân hiệu dụng tăng}
    {Số electron giảm}
    \loigiai{Trong một chu kỳ, từ trái sang phải, điện tích hạt nhân hiệu dụng tăng do số proton tăng mà số lớp electron không đổi. Điều này làm tăng lực hút giữa hạt nhân và electron, dẫn đến bán kính nguyên tử giảm dần.}
\end{ex}

%%%%=================EX_08====================%%%
\begin{ex}
    Nguyên tố nào sau đây có bán kính nguyên tử lớn nhất?
    \choice
    {Na}
    {K}
    {\True Cs}
    {Li}
    \loigiai{Trong nhóm kim loại kiềm (IA), bán kính nguyên tử tăng dần từ trên xuống dưới. Cs nằm ở dưới cùng trong nhóm này, nên có bán kính nguyên tử lớn nhất trong số các nguyên tố đã cho.}
\end{ex}

%%%%=================EX_09====================%%%
\begin{ex}
    Nguyên tố nào sau đây có bán kính nguyên tử nhỏ nhất?
    \choice
    {Be}
    {\True F}
    {N}
    {O}
    \loigiai{Trong chu kỳ 2, bán kính nguyên tử giảm dần từ trái sang phải. F là nguyên tố ở gần cuối chu kỳ 2 (trừ khí hiếm), nên có bán kính nguyên tử nhỏ nhất trong số các nguyên tố đã cho.}
\end{ex}

%%%%=================EX_10====================%%%
\begin{ex}
    So sánh bán kính của các ion $Na^{+}$, $Mg^{2+}$, $Al^{3+}$:
    \choice
    {$Na^{+} > Mg^{2+} > Al^{3+}$}
    {\True $Na^{+} < Mg^{2+} < Al^{3+}$}
    {$Na^{+} = Mg^{2+} = Al^{3+}$}
    {$Al^{3+} > Mg^{2+} > Na^{+}$}
    \loigiai{Các ion $Na^{+}$, $Mg^{2+}$, $Al^{3+}$ là các ion đẳng electron (cấu hình electron giống $Ne$). Theo quy luật 2, khi điện tích hạt nhân tăng (từ $Na^{+}$ đến $Al^{3+}$), bán kính ion giảm. Vậy thứ tự bán kính là: $Na^{+} > Mg^{2+} > Al^{3+}$.}
\end{ex}

%%%%=================EX_11====================%%%
\begin{ex}
    Trong các ion sau, ion nào có bán kính lớn nhất?
    \choice
    {$Na^{+}$}
    {$F^{-}$}
    {\True $Cl^{-}$}
    {$K^{+}$}
    \loigiai{$Cl^{-}$ có bán kính lớn nhất vì: (1) Là anion nên có bán kính lớn hơn nguyên tử trung hòa, (2) Nằm ở chu kỳ 3, có nhiều lớp electron hơn so với $F^{-}$ ở chu kỳ 2, (3) Các cation ($Na^{+}$, $K^{+}$) luôn có bán kính nhỏ hơn nguyên tử trung hòa tương ứng.}
\end{ex}

%%%%=================EX_12====================%%%
\begin{ex}
    Hiệu ứng che chắn ảnh hưởng như thế nào đến bán kính nguyên tử?
    \choice
    {Làm giảm bán kính nguyên tử}
    {\True Làm tăng bán kính nguyên tử}
    {Không ảnh hưởng đến bán kính nguyên tử}
    {Làm bán kính nguyên tử tăng rồi giảm}
    \loigiai{Hiệu ứng che chắn làm giảm lực hút giữa hạt nhân và các electron ở lớp ngoài cùng, do các electron ở lớp trong che chắn một phần lực hút của hạt nhân. Điều này làm tăng bán kính nguyên tử.}
\end{ex}

%%%%=================EX_13====================%%%
\begin{ex}
    Trong các cặp ion sau, cặp nào có bán kính gần bằng nhau nhất?
    \choice
    {$Na^{+}$ và $F^{-}$}
    {$K^{+}$ và $Cl^{-}$}
    {\True $K^{+}$ và $Cl^{-}$}
    {$Mg^{2+}$ và $O^{2-}$}
    \loigiai{$K^{+}$ và $Cl^{-}$ có bán kính gần bằng nhau nhất vì chúng là cặp ion đẳng electron (cùng cấu hình electron với $Ar$) và nằm gần nhau trong bảng tuần hoàn (chu kỳ 4 và 3). Mặc dù $K^{+}$ nhỏ hơn $K$ và $Cl^{-}$ lớn hơn $Cl$, nhưng sự chênh lệch này gần như cân bằng nhau.}
\end{ex}

%%%%=================EX_14====================%%%
\begin{ex}
    Nguyên tố nào sau đây có xu hướng tạo ra ion dương (cation) có bán kính nhỏ nhất?
    \choice
    {Na}
    {Mg}
    {\True Al}
    {K}
    \loigiai{Trong số các nguyên tố này, Al có xu hướng tạo ra ion $Al^{3+}$, có điện tích dương lớn nhất. Ion có điện tích dương càng lớn thì bán kính càng nhỏ do lực hút mạnh hơn giữa hạt nhân và các electron còn lại. Vì vậy, $Al^{3+}$ sẽ có bán kính nhỏ nhất.}
\end{ex}

%%%%=================EX_15====================%%%
\begin{ex}
    Trong một nhóm của bảng tuần hoàn, yếu tố nào quyết định sự tăng dần của bán kính nguyên tử từ trên xuống dưới?
    \choice
    {Số proton tăng}
    {\True Số lớp electron tăng}
    {Hiệu ứng che chắn giảm}
    {Điện tích hạt nhân hiệu dụng tăng}
    \loigiai{Trong một nhóm, từ trên xuống dưới, số lớp electron tăng dần. Đây là yếu tố quyết định làm tăng khoảng cách từ hạt nhân đến lớp electron ngoài cùng, dẫn đến bán kính nguyên tử tăng dần.}
\end{ex}

%%%%=================EX_16====================%%%
\begin{ex}
    Nguyên tố nào sau đây có xu hướng tạo ra anion có bán kính lớn nhất?
    \choice
    {O}
    {F}
    {\True I}
    {Cl}
    \loigiai{Trong nhóm halogen (VIIA), bán kính nguyên tử tăng dần từ trên xuống dưới. I nằm ở dưới cùng trong nhóm này, nên khi tạo anion $I^{-}$, nó sẽ có bán kính lớn nhất trong số các anion của các nguyên tố đã cho.}
\end{ex}

%%%%=================EX_17====================%%%
\begin{ex}
    So sánh bán kính của các nguyên tử $Ne$, $F$, và $Na$:
    \choice
    {$Ne > F > Na$}
    {$F > Ne > Na$}
    {\True $Na > Ne > F$}
    {$Na < F < Ne$}
    \loigiai{$Na$, $Ne$, và $F$ nằm cùng chu kỳ 2 trong bảng tuần hoàn. Bán kính nguyên tử giảm dần từ trái sang phải trong một chu kỳ. Do đó, $Na$ (ở đầu chu kỳ) có bán kính lớn nhất, tiếp theo là $Ne$, và cuối cùng là $F$ (gần cuối chu kỳ).}
\end{ex}

%%%%=================EX_18====================%%%
\begin{ex}
    Trong các nguyên tố sau, nguyên tố nào có năng lượng ion hóa thứ nhất lớn nhất?
    \choice
    {Li}
    {Be}
    {B}
    {\True F}
    \loigiai{Năng lượng ion hóa thứ nhất có xu hướng tăng khi bán kính nguyên tử giảm. Trong các nguyên tố đã cho, F có bán kính nguyên tử nhỏ nhất, nên có năng lượng ion hóa thứ nhất lớn nhất.}
\end{ex}

%%%%=================EX_19====================%%%
\begin{ex}
    So sánh bán kính của các ion $O^{2-}$, $F^{-}$, $Na^{+}$, và $Mg^{2+}$:
    \choice
    {$O^{2-} < F^{-} < Na^{+} < Mg^{2+}$}
    {\True $Mg^{2+} < Na^{+} < F^{-} < O^{2-}$}
    {$Na^{+} < Mg^{2+} < F^{-} < O^{2-}$}
    {$Mg^{2+} < Na^{+} < O^{2-} < F^{-}$}
    \loigiai{Các ion này đều là ion đẳng electron (10 electron). Khi điện tích hạt nhân tăng, lực hút giữa hạt nhân và electron tăng, làm bán kính ion giảm. Do đó, $Mg^{2+}$ (điện tích hạt nhân lớn nhất) có bán kính nhỏ nhất, tiếp theo là $Na^{+}$, $F^{-}$ và cuối cùng là $O^{2-}$ (điện tích hạt nhân nhỏ nhất).}
\end{ex}

%%%%=================EX_20====================%%%
\begin{ex}
    Nguyên tử của nguyên tố nào sau đây dễ nhận electron nhất?
    \choice
    {Na}
    {Mg}
    {Cl}
    {\True F}
    \loigiai{Nguyên tử có bán kính nhỏ và điện tích hạt nhân lớn có xu hướng dễ nhận electron nhất. Trong số các nguyên tố đã cho, F có bán kính nguyên tử nhỏ nhất và điện tích hạt nhân lớn nhất, nên có khả năng thu hút electron mạnh nhất.}
\end{ex} 
