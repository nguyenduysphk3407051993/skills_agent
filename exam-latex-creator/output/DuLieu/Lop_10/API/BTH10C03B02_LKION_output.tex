```latex
%%%%=================EX_01====================%%%
\begin{ex}
Ion nào sau đây có cấu hình electron của khí hiếm neon?
    \choice
    {$Mg^{3+}$}
    {\True $O^{2-}$}
    {$Na^+$}
    {$Cl^-$}
    \loigiai{Ion $O^{2-}$ có 10 electron, cấu hình electron $1s^22s^22p^6$, giống với cấu hình electron của khí hiếm neon.}
\end{ex}

%%%%=================EX_02====================%%%
\begin{ex}
Liên kết ion được hình thành do:
    \choice
    {Lực hút tĩnh điện giữa các ion mang điện tích cùng dấu.}
    {\True Lực hút tĩnh điện giữa các ion mang điện tích trái dấu.}
    {Sự dùng chung electron giữa các nguyên tử.}
    {Sự cho nhận proton giữa các nguyên tử.}
    \loigiai{Liên kết ion là liên kết được hình thành bởi lực hút tĩnh điện giữa các ion mang điện tích trái dấu.}
\end{ex}

%%%%=================EX_03====================%%%
\begin{ex}
Nguyên tử nào sau đây có xu hướng nhường 2 electron để tạo thành ion dương?
    \choice
    {Cl}
    {\True Mg}
    {O}
    {Na}
    \loigiai{Mg là kim loại điển hình, có 2 electron ở lớp ngoài cùng, có xu hướng nhường 2 electron để đạt cấu hình electron bền vững của khí hiếm.}
\end{ex}

%%%%=================EX_04====================%%%
\begin{ex}
Nguyên tử nào sau đây có xu hướng nhận 1 electron để tạo thành ion âm?
    \choice
    {Na}
    {Mg}
    {O}
    {\True Cl}
    \loigiai{Cl là phi kim điển hình, có 7 electron ở lớp ngoài cùng, có xu hướng nhận thêm 1 electron để đạt cấu hình electron bền vững của khí hiếm.}
\end{ex}

%%%%=================EX_05====================%%%
\begin{ex}
Tinh thể ion nào sau đây có cấu trúc hình lập phương?
    \choice
    {$MgO$}
    {$KCl$}
    {\True $NaCl$}
    {$CaO$}
    \loigiai{Tinh thể $NaCl$ có cấu trúc hình lập phương, với các ion $Na^+$ và $Cl^-$ sắp xếp luân phiên nhau.}
\end{ex}

%%%%=================EX_06====================%%%
\begin{ex}
Trong tinh thể $NaCl$, xung quanh mỗi ion $Na^+$ có bao nhiêu ion $Cl^-$ gần nhất?
    \choice
    {4}
    {\True 6}
    {8}
    {12}
    \loigiai{Trong tinh thể $NaCl$, xung quanh mỗi ion $Na^+$ có 6 ion $Cl^-$ gần nhất và ngược lại.}
\end{ex}

%%%%=================EX_07====================%%%
\begin{ex}
Hợp chất ion nào sau đây được tạo thành từ kim loại điển hình và phi kim điển hình?
    \choice
    {$H_2O$}
    {$CO_2$}
    {\True $NaCl$}
    {$NH_3$}
    \loigiai{$NaCl$ được tạo thành từ kim loại điển hình Na và phi kim điển hình Cl.}
\end{ex}

%%%%=================EX_08====================%%%
\begin{ex}
Ion $K^+$ có cấu hình electron của khí hiếm nào?
    \choice
    {Ne}
    {He}
    {\True Ar}
    {Kr}
    \loigiai{Ion $K^+$ có cấu hình electron $1s^22s^22p^63s^23p^6$, giống với cấu hình electron của khí hiếm Ar.}
\end{ex}


%%%%=================EX_09====================%%%
\begin{ex}
Công thức hóa học của hợp chất tạo bởi ion $Mg^{2+}$ và ion $O^{2-}$ là:
    \choice
    {$Mg_2O$}
    {$MgO_2$}
    {\True $MgO$}
    {$Mg_2O_2$}
    \loigiai{Do điện tích của $Mg^{2+}$ là 2+ và $O^{2-}$ là 2-, nên công thức hóa học là $MgO$.}
\end{ex}

%%%%=================EX_10====================%%%
\begin{ex}
Điện tích của ion được hình thành khi nguyên tử Na nhường 1 electron là:
    \choice
    {2+}
    {\True 1+}
    {1-}
    {2-}
    \loigiai{Khi nguyên tử Na nhường 1 electron, nó sẽ trở thành ion $Na^+$ mang điện tích 1+.}
\end{ex}


%%%%=================TF_01====================%%%
\begin{ex}
	Các ion cùng dấu thì hút nhau, các ion trái dấu thì đẩy nhau.
	\choiceTF[t]
	{\True Các ion cùng dấu thì đẩy nhau, các ion trái dấu thì hút nhau.}
	{Các ion dương và ion âm hút nhau tạo thành liên kết ion.}
    {Kim loại có xu hướng nhường electron, phi kim có xu hướng nhận electron.}
    {\True Các hợp chất ion thường là chất rắn ở điều kiện thường.}
	\loigiai{Các ion cùng dấu sẽ đẩy nhau, các ion trái dấu sẽ hút nhau. Liên kết ion được hình thành do lực hút tĩnh điện giữa các ion trái dấu. Kim loại có xu hướng nhường electron để tạo thành cation, phi kim có xu hướng nhận electron để tạo thành anion. Các hợp chất ion thường là chất rắn ở điều kiện thường do lực hút tĩnh điện mạnh giữa các ion.}
\end{ex}

%%%%=================TF_02====================%%%
\begin{ex}
	Ion $Na^+$ có cấu hình electron giống với khí hiếm He.
	\choiceTF[t]
	{\True Ion $Na^+$ có cấu hình electron giống với khí hiếm Ne.}
	{Liên kết ion chỉ tồn tại giữa kim loại và phi kim.}
    {\True Tinh thể $NaCl$ có cấu trúc hình lập phương.}
    {Độ âm điện của kim loại lớn hơn độ âm điện của phi kim.}
	\loigiai{Ion $Na^+$ có cấu hình electron là $1s^22s^22p^6$, giống với cấu hình electron của khí hiếm Ne. Liên kết ion thường tồn tại giữa kim loại và phi kim. Tinh thể $NaCl$ có cấu trúc hình lập phương. Độ âm điện của phi kim lớn hơn độ âm điện của kim loại.}
\end{ex}


%%%%=================TF_03====================%%%
\begin{ex}
	Nguyên tử Cl có xu hướng nhường 1 electron để tạo thành ion $Cl^-$.
	\choiceTF[t]
	{\True Nguyên tử Cl có xu hướng nhận 1 electron để tạo thành ion $Cl^-$.}
	{Ion $Mg^{2+}$ có cấu hình electron giống với khí hiếm Ne.}
    {\True Hợp chất $NaCl$ có liên kết ion.}
    {Các hợp chất ion không dẫn điện khi nóng chảy.}
	\loigiai{Nguyên tử Cl có 7 electron lớp ngoài cùng, có xu hướng nhận thêm 1 electron để tạo thành ion $Cl^-$. Ion $Mg^{2+}$ có cấu hình electron giống với khí hiếm Ne. Hợp chất $NaCl$ có liên kết ion. Các hợp chất ion dẫn điện khi nóng chảy hoặc khi hòa tan trong nước do các ion được giải phóng và có thể di chuyển tự do.}
\end{ex}


%%%%=================TF_04====================%%%
\begin{ex}
    Nguyên tử oxi có xu hướng nhường 2 electron để tạo thành ion $O^{2-}$.
    \choiceTF[t]
    {\True Nguyên tử oxi có xu hướng nhận 2 electron để tạo thành ion $O^{2-}$.}
    {Các hợp chất ion có nhiệt độ nóng chảy và nhiệt độ sôi thấp.}
    {\True $Na_2O$ là hợp chất ion.}
    {Ion $Ca^{2+}$ có cấu hình electron giống khí hiếm He.}
    \loigiai{Nguyên tử oxi có 6 electron lớp ngoài cùng, có xu hướng nhận thêm 2 electron để tạo thành ion $O^{2-}$. Các hợp chất ion có nhiệt độ nóng chảy và nhiệt độ sôi cao do lực hút tĩnh điện mạnh giữa các ion. $Na_2O$ là hợp chất ion được tạo thành từ ion $Na^+$ và $O^{2-}$. Ion $Ca^{2+}$ có cấu hình electron giống khí hiếm Ar.}
\end{ex}


%%%%=================TF_05====================%%%
\begin{ex}
    Liên kết ion là liên kết được hình thành do sự dùng chung electron giữa các nguyên tử.
    \choiceTF[t]
    {\True Liên kết ion là liên kết được hình thành do lực hút tĩnh điện giữa các ion trái dấu.}
    {Các hợp chất ion thường tồn tại ở trạng thái khí ở điều kiện thường.}
    {\True Trong tinh thể $NaCl$, các ion $Na^+$ và $Cl^-$ sắp xếp luân phiên nhau.}
    {Kim loại có độ âm điện lớn hơn phi kim.}
    \loigiai{Liên kết ion là liên kết được hình thành do lực hút tĩnh điện giữa các ion trái dấu. Các hợp chất ion thường tồn tại ở trạng thái rắn ở điều kiện thường. Trong tinh thể $NaCl$, các ion $Na^+$ và $Cl^-$ sắp xếp luân phiên nhau. Phi kim có độ âm điện lớn hơn kim loại.}
\end{ex}



%%%%=================TF_06====================%%%
\begin{ex}
    Ion $Cl^-$ có cấu hình electron giống với khí hiếm Ar.
    \choiceTF[t]
    {\True Ion $Cl^-$ có cấu hình electron giống với khí hiếm Ar.}
    {Mg có xu hướng nhận 2 electron để tạo thành ion $Mg^{2+}$.}
    {\True K có xu hướng nhường 1 electron để tạo thành ion $K^+$.}
    {$K_2O$ là hợp chất cộng hóa trị.}
    \loigiai{Ion $Cl^-$ có cấu hình electron $1s^22s^22p^63s^23p^6$, giống với cấu hình electron của khí hiếm Ar. Mg có xu hướng nhường 2 electron để tạo thành ion $Mg^{2+}$. K có xu hướng nhường 1 electron để tạo thành ion $K^+$. $K_2O$ là hợp chất ion.}
\end{ex}

%%%%=================TF_07====================%%%
\begin{ex}
    $MgO$ có nhiệt độ nóng chảy thấp.
    \choiceTF[t]
    {\True $MgO$ có nhiệt độ nóng chảy cao.}
    {Ion $O^{2-}$ có cấu hình electron giống khí hiếm He.}
    {\True Ion $Mg^{2+}$ có cấu hình electron giống khí hiếm Ne.}
    {Liên kết trong phân tử $H_2O$ là liên kết ion.}
    \loigiai{$MgO$ có nhiệt độ nóng chảy cao do lực hút tĩnh điện mạnh giữa các ion $Mg^{2+}$ và $O^{2-}$. Ion $O^{2-}$ có cấu hình electron giống khí hiếm Ne. Ion $Mg^{2+}$ có cấu hình electron giống khí hiếm Ne. Liên kết trong phân tử $H_2O$ là liên kết cộng hóa trị.}
\end{ex}


%%%%=================TF_08====================%%%
\begin{ex}
    Nguyên tử Na có xu hướng nhận 1 electron để tạo thành ion $Na^+$.
    \choiceTF[t]
    {\True Nguyên tử Na có xu hướng nhường 1 electron để tạo thành ion $Na^+$.}
    {Các hợp chất ion không dẫn điện khi hòa tan trong nước.}
    {\True $NaCl$ là chất rắn ở điều kiện thường.}
    {Độ âm điện của Na lớn hơn độ âm điện của Cl.}
    \loigiai{Nguyên tử Na có xu hướng nhường 1 electron để tạo thành ion $Na^+$. Các hợp chất ion dẫn điện khi hòa tan trong nước. $NaCl$ là chất rắn ở điều kiện thường. Độ âm điện của Cl lớn hơn độ âm điện của Na.}
\end{ex}


%%%%=================TF_09====================%%%
\begin{ex}
    $CaF_2$ là hợp chất cộng hóa trị.
    \choiceTF[t]
    {\True $CaF_2$ là hợp chất ion.}
    {Ion $Ca^{2+}$ có cấu hình electron giống khí hiếm Ne.}
    {\True Ion $F^-$ có cấu hình electron giống khí hiếm Ne.}
    {Nguyên tử F có xu hướng nhường 1 electron để tạo thành ion $F^-$.}
    \loigiai{$CaF_2$ là hợp chất ion. Ion $Ca^{2+}$ có cấu hình electron giống khí hiếm Ar. Ion $F^-$ có cấu hình electron giống khí hiếm Ne. Nguyên tử F có xu hướng nhận 1 electron để tạo thành ion $F^-$.}
\end{ex}


%%%%=================TF_10====================%%%
\begin{ex}
    $Na_2O$ có nhiệt độ nóng chảy thấp.
    \choiceTF[t]
    {\True $Na_2O$ có nhiệt độ nóng chảy cao.}
    {$NaCl$ dẫn điện ở trạng thái rắn.}
    {\True $MgO$ có cấu trúc tinh thể.}
    {Ion $Na^+$ có cấu hình electron giống khí hiếm He.}
    \loigiai{$Na_2O$ có nhiệt độ nóng chảy cao. $NaCl$ không dẫn điện ở trạng thái rắn. $MgO$ có cấu trúc tinh thể. Ion $Na^+$ có cấu hình electron giống khí hiếm Ne.}
\end{ex}




%%%%=================BT_01====================%%%
\begin{bt}
Trình bày sự hình thành liên kết ion trong phân tử $CaCl_2$.
\loigiai{Nguyên tử Ca có 2 electron ở lớp ngoài cùng, có xu hướng nhường 2 electron để đạt cấu hình electron bền vững của khí hiếm Ar. Nguyên tử Cl có 7 electron ở lớp ngoài cùng, có xu hướng nhận 1 electron để đạt cấu hình electron bền vững của khí hiếm Ar. Vậy, nguyên tử Ca nhường 2 electron cho 2 nguyên tử Cl, tạo thành ion $Ca^{2+}$ và 2 ion $Cl^-$.  Các ion trái dấu $Ca^{2+}$ và $Cl^-$ hút nhau bằng lực hút tĩnh điện, tạo thành phân tử $CaCl_2$.}
\end{bt}


%%%%=================BT_02====================%%%
\begin{bt}
Giải thích tại sao các hợp chất ion thường là chất rắn ở điều kiện thường và có nhiệt độ nóng chảy cao.
\loigiai{Các hợp chất ion là chất rắn ở điều kiện thường và có nhiệt độ nóng chảy cao do lực hút tĩnh điện mạnh giữa các ion dương và ion âm trong mạng tinh thể ion. Để phá vỡ mạng tinh thể này, cần cung cấp một năng lượng lớn, dẫn đến nhiệt độ nóng chảy và nhiệt độ sôi cao.}
\end{bt}


%%%%=================BT_03====================%%%
\begin{bt}
So sánh độ âm điện của kim loại và phi kim. Giải thích.
\loigiai{Kim loại có độ âm điện nhỏ hơn phi kim. Kim loại có xu hướng nhường electron để tạo thành ion dương, trong khi phi kim có xu hướng nhận electron để tạo thành ion âm. Độ âm điện là khả năng hút electron của nguyên tử khi hình thành liên kết hóa học. Do đó, phi kim có độ âm điện lớn hơn kim loại.}
\end{bt}


%%%%=================BT_04====================%%%
\begin{bt}
Viết cấu hình electron của ion $Al^{3+}$ và cho biết ion này có cấu hình electron của khí hiếm nào.
\loigiai{Cấu hình electron của nguyên tử Al là $1s^22s^22p^63s^23p^1$. Khi mất 3 electron, Al tạo thành ion $Al^{3+}$ có cấu hình electron là $1s^22s^22p^6$. Đây là cấu hình electron của khí hiếm Ne.}
\end{bt}


%%%%=================BT_05====================%%%
\begin{bt}
Một người ăn 10g muối ăn mỗi ngày. Tính khối lượng ion $Na^+$ mà người đó nạp vào cơ thể. Biết muối ăn có công thức là $NaCl$.
\loigiai{Khối lượng mol của NaCl là 58.5 g/mol.
Khối lượng mol của $Na^+$ là 23 g/mol.

Trong 58.5g NaCl có 23g $Na^+$.
Vậy trong 10g NaCl có x g $Na^+$.

$x = \dfrac{10 \times 23}{58.5} \approx 3.93g$

Vậy lượng ion $Na^+$ mà người đó nạp vào cơ thể là khoảng 3.93g.}
\end{bt}


%%%%=================BT_06====================%%%
\begin{bt}
Trình bày sự hình thành liên kết ion trong phân tử $K_2S$.
\loigiai{Nguyên tử K có 1 electron ở lớp ngoài cùng, có xu hướng nhường 1 electron để đạt cấu hình electron bền vững của khí hiếm Ar. Nguyên tử S có 6 electron ở lớp ngoài cùng, có xu hướng nhận 2 electron để đạt cấu hình electron bền vững của khí hiếm Ar. Vậy, 2 nguyên tử K nhường 2 electron cho 1 nguyên tử S, tạo thành 2 ion $K^+$ và 1 ion $S^{2-}$. Các ion trái dấu $K^+$ và $S^{2-}$ hút nhau bằng lực hút tĩnh điện, tạo thành phân tử $K_2S$.}
\end{bt}

```