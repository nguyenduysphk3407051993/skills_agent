%%%Mức độ 1: Nhận biết (4 câu)
%%%%=================EX01====================%%%
\begin{ex}
	Trong một chu kì, theo chiều tăng dần của điện tích hạt nhân, tính acid của oxide và hydroxide có xu hướng:
	\choice
	{Giảm dần}
	{\True Tăng dần}
	{Không thay đổi}
	{Vừa tăng vừa giảm}
	\loigiai{Trong một chu kì, theo chiều tăng dần của điện tích hạt nhân, bán kính nguyên tử giảm dần, khả năng nhường electron của nguyên tử giảm, khả năng hút electron của nguyên tử tăng, do đó tính kim loại giảm dần, tính phi kim tăng dần. Oxide và hydroxide có tính acid tăng dần.}
\end{ex}
%%%%=================EX_02====================%%%
\begin{ex}
	Oxide nào sau đây có tính base mạnh nhất?
	\choice
	{$Na_2O$}
	{$MgO$}
	{$Al_2O_3$}
	{$SO_3$}
	\loigiai{$Na_2O$ là oxide của kim loại kiềm, có tính base mạnh nhất trong các oxide đã cho.}
\end{ex}
%%%%=================EX03====================%%%
\begin{ex}
	Hidroxide nào sau đây có tính acid mạnh nhất?
	\choice
	{$NaOH$}
	{$Mg(OH)_2$}
	{$Al(OH)_3$}
	{$HClO_4$}
	\loigiai{$HClO_4$ là axit rất mạnh, có tính acid mạnh nhất trong các hydroxide đã cho.}
\end{ex}
%%%%=================EX_04====================%%%
\begin{ex}
	Phản ứng nào sau đây không xảy ra?
	\choice
	{$CO_2 + H_2O \longrightarrow H_2CO_3$}
	{$CaO + H_2O \longrightarrow Ca(OH)_2$}
	{$SO_3 + H_2O \longrightarrow H_2SO_4$}
	{\True $SiO_2 + H_2O \longrightarrow H_2SiO_3$}
	\loigiai{$SiO_2$ là oxide acid rất yếu, không phản ứng với nước.}
\end{ex}

%%%Mức độ 2: Thông hiểu (4 câu)

%%%%=================EX05====================%%%
\begin{ex}
	Sắp xếp các oxide sau theo chiều tăng dần tính acid: $Na_2O$, $Al_2O_3$, $SO_3$, $SiO_2$.
	\choice
	{$Na_2O$, $Al_2O_3$, $SiO_2$, $SO_3$}
	{$SO_3$, $SiO_2$, $Al_2O_3$, $Na_2O$}
	{$Al_2O_3$, $Na_2O$, $SO_3$, $SiO_2$}
	{\True $Na_2O$, $Al_2O_3$, $SO_3$, $SiO_2$}
	\loigiai{Trong một chu kì, theo chiều tăng dần của điện tích hạt nhân, tính acid của oxide tăng dần. $Na_2O$ là oxide base, $Al_2O_3$ là oxide lưỡng tính, $SiO_2$ và $SO_3$ là oxide acid. Trong đó, $SO_3$ có tính acid mạnh hơn $SiO_2$ do lưu huỳnh ở dưới oxi trong cùng một nhóm A.}
\end{ex}
%%%%=================EX_06====================%%%
\begin{ex}
	Cho các chất sau: $HCl$, $NaOH$, $Al(OH)_3$, $H_2SO_4$. Chất nào có thể phản ứng được với cả dung dịch $HCl$ và dung dịch $NaOH$?
	\choice
	{$HCl$}
	{$NaOH$}
	{\True $Al(OH)_3$}
	{$H_2SO_4$}
	\loigiai{$Al(OH)_3$ là hydroxide lưỡng tính, có thể phản ứng được với cả dung dịch acid ($HCl$) và dung dịch base ($NaOH$).}
\end{ex}
%%%%=================EX_07====================%%%
\begin{ex}
	Trong các phản ứng sau, phản ứng nào oxide đóng vai trò là oxide acid?
	\choice
	{$CaO + H_2O \longrightarrow Ca(OH)_2$}
	{\True $CO_2 + 2NaOH \longrightarrow Na_2CO_3 + H_2O$}
	{$Na_2O + SO_2 \longrightarrow Na_2SO_3$}
	{$Al_2O_3 + 6HCl \longrightarrow 2AlCl_3 + 3H_2O$}
	\loigiai{Trong phản ứng $CO_2 + 2NaOH \longrightarrow Na_2CO_3 + H_2O$, $CO_2$ phản ứng với base ($NaOH$) tạo thành muối ($Na_2CO_3$) và nước. Đây là phản ứng thể hiện tính acid của oxide.}
\end{ex}
%%%%=================EX_08====================%%%
\begin{ex}
	Nguyên tố X thuộc chu kì 3, nhóm VIIA. Oxide cao nhất của X có tính chất nào sau đây?
	\choice
	{Tác dụng với nước tạo dung dịch base}
	{\True Tác dụng với nước tạo dung dịch acid}
	{Không tác dụng với nước}
	{Vừa tác dụng với dung dịch acid, vừa tác dụng với dung dịch base}
	\loigiai{Nguyên tố X thuộc chu kì 3, nhóm VIIA là nguyên tố phi kim. Oxide cao nhất của X là oxide acid, tác dụng với nước tạo dung dịch acid.}
\end{ex}

%%% Mức độ 3: Vận dụng (4 câu)

%%%%=================EX_09====================%%%
\begin{ex}
	Cho dung dịch $Ba(OH)_2$ dư vào dung dịch chứa hỗn hợp $FeCl_3$ và $AlCl_3$, thu được kết tủa A. Nung A trong không khí đến khối lượng không đổi, thu được chất rắn B. Xác định thành phần của B.
	\choice
	{$Fe_2O_3$}
	{$Al_2O_3$}
	{\True $Fe_2O_3$ và $BaAl_2O_4$}
	{$Fe_2O_3$, $Al_2O_3$ và $BaAl_2O_4$}
	\loigiai{Kết tủa A gồm $Fe(OH)_3$ và $Al(OH)_3$. Khi nung A trong không khí đến khối lượng không đổi, $Fe(OH)_3$ bị nhiệt phân thành $Fe_2O_3$, còn $Al(OH)_3$ bị nhiệt phân thành $Al_2O_3$. $Al_2O_3$ tác dụng với $BaO$ (từ $Ba(OH)_2$ bị nhiệt phân) tạo thành $BaAl_2O_4$.}
\end{ex}
%%%%=================EX_10====================%%%
\begin{ex}
	Hòa tan hoàn toàn m gam hỗn hợp $Na_2O$ và $Al_2O_3$ (tỉ lệ mol 1:1) vào nước, thu được dung dịch X. Cho dung dịch $HCl$ 1M vào X đến khi bắt đầu xuất hiện kết tủa thì dừng lại, thu được dung dịch Y. Cho tiếp dung dịch $HCl$ 1M vào Y đến dư, thấy lượng $HCl$ dùng hết ở phản ứng thứ hai là 100 ml. Giá trị của m là:
	\choice
	{10,2 gam}
	{\True 5,1 gam}
	{2,55 gam}
	{1,275 gam}
	\loigiai{Phản ứng thứ nhất: $NaAlO_2 + HCl + H_2O \longrightarrow Al(OH)_3 + NaCl$. Phản ứng thứ hai: $Al(OH)_3 + 3HCl \longrightarrow AlCl_3 + 3H_2O$. Từ phương trình, ta tính được $n_{Al_2O_3} = n_{Na_2O} = \dfrac{1}{4}n_{HCl} = 0,025$ mol. Vậy $m = 0,025 \times (46 + 102) = 5,1$ gam.}
\end{ex}
%%%%=================EX_11====================%%%
\begin{ex}
	Cho các chất sau: $CO_2$, $SO_2$, $Na_2O$, $CaO$. Có thể dùng chất nào sau đây để nhận biết các chất trên?
	\choice
	{Quỳ tím ẩm}
	{\True Dung dịch $Br_2$}
	{Nước}
	{Dung dịch $Ca(OH)_2$}
	\loigiai{Dùng dung dịch $Br_2$ có thể phân biệt $SO_2$ ($Br_2$ mất màu) với $CO_2$ ($Br_2$ không mất màu). Sau đó, dùng quỳ tím ẩm để phân biệt $Na_2O$ (quỳ tím chuyển xanh) và $CaO$ (quỳ tím chuyển xanh).}
\end{ex}
%%%%=================EX_12====================%%%
\begin{ex}
	Cho dung dịch $Na_2CO_3$ vào dung dịch $AlCl_3$ đến dư, hiện tượng quan sát được là:
	\choice
	{Xuất hiện kết tủa keo trắng}
	{Xuất hiện kết tủa keo trắng và khí không màu}
	{Không có hiện tượng gì}
	{\True Xuất hiện kết tủa keo trắng, sau đó kết tủa tan một phần}
	\loigiai{Ban đầu, xảy ra phản ứng: $3Na_2CO_3 + 2AlCl_3 + 3H_2O \longrightarrow 2Al(OH)_3 + 6NaCl + 3CO_2$. Sau đó, $Al(OH)_3$ tan một phần trong dung dịch $Na_2CO_3$ dư: $Al(OH)_3 + Na_2CO_3 \longrightarrow NaAlO_2 + NaHCO_3 + H_2O$.}
\end{ex}

%%% Mức độ 4: Vận dụng cao (3 câu)

%%%%=================EX_13====================%%%
\begin{ex}
	Cho hỗn hợp khí X gồm $CO_2$ và $SO_2$ (tỉ khối hơi của X so với $H_2$ là 27). Hấp thụ hoàn toàn X vào 200 gam dung dịch $NaOH$ 8%, thu được dung dịch Y. Cô cạn Y, thu được m gam chất rắn khan. Giá trị của m là:
	\choice
	{25,4 gam}
	{\True 26,6 gam}
	{27,8 gam}
	{29,0 gam}
	\loigiai{Gọi số mol $CO_2$ và $SO_2$ trong X lần lượt là a và b mol. Ta có: $\dfrac{44a+64b}{a+b}=27\times2=54$. Giải hệ, ta được: a = b = 0,2 mol. $n_{NaOH} = 0,4$ mol. Ta có: $1 < \dfrac{n_{NaOH}}{n_{CO_2}+n_{SO_2}} = 1 < 2$. Vậy, phản ứng tạo ra 2 muối là: $NaHCO_3$ (0,2 mol) và $NaHSO_3$ (0,2 mol). $m = 0,2 \times (84 + 114) = 26,6$ gam.}
\end{ex}
%%%%=================EX_14====================%%%
\begin{ex}
	Hòa tan hoàn toàn 10 gam hỗn hợp gồm $Fe$ và $Fe_xO_y$ bằng dung dịch $HNO_3$ loãng (dư), thu được 1,12 lít khí $NO$ (sản phẩm khử duy nhất, ở đktc) và dung dịch chứa 27,5 gam muối. Công thức của oxide là:
	\choice
	{$FeO$}
	{$Fe_2O_3$}
	{\True $Fe_3O_4$}
	{$FeO$ hoặc $Fe_3O_4$}
	\loigiai{Gọi số mol $Fe$ và $Fe_xO_y$ lần lượt là a và b mol. Ta có hệ phương trình: $56a+56xb+16yb=10$ (Bảo toàn khối lượng) và $3a+3xb-2yb=0,15$ (Bảo toàn electron). Khối lượng muối gồm: $Fe(NO_3)_3$: (a + xb) mol. Áp dụng định luật bảo toàn khối lượng: $242(a+xb)=27,5+10+63\times0,15$. Giải hệ, ta được: a = 0,1 mol, xb = 0,05 mol, yb = 0,1 mol. Suy ra: x/y = 1/2. Vậy, công thức của oxide là $Fe_3O_4$.}
\end{ex}
%%%%=================EX_15====================%%%
\begin{ex}
	Cho m gam hỗn hợp X gồm $Al$, $Fe$ và $Cu$ tác dụng với dung dịch $H_2SO_4$ loãng, dư thu được 5,6 lít $H_2$ (ở đktc), dung dịch Y và 6,4 gam chất rắn Z. Cho dung dịch $NaOH$ dư vào Y, thu được kết tủa T. Nung T trong không khí đến khối lượng không đổi, thu được 16 gam chất rắn. Giá trị của m là:
	\choice
	{20,4 gam}
	{21,6 gam}
	{22,8 gam}
	{\True 24,0 gam}
	\loigiai{Z là $Cu$. $n_{H_2} = 0,25$ mol. Chất rắn thu được sau khi nung T gồm $Fe_2O_3$ (0,1 mol) và $Al_2O_3$ (0,05 mol). Bảo toàn nguyên tố, ta có: $n_{Fe} = 0,2$ mol, $n_{Al} = 0,1$ mol. Vậy $m = 0,1 \times 27 + 0,2 \times 56 + 6,4 = 24$ gam.}
\end{ex}
