%%%=================EX_01====================%%%
\begin{ex}
Về xu hướng biến đổi bán kính nguyên tử trong bảng tuần hoàn, chọn phát biểu đúng:
\choiceTF[t]
{\True Trong một chu kỳ, bán kính nguyên tử giảm dần từ trái sang phải}
{\True Trong một nhóm A, bán kính nguyên tử tăng dần từ trên xuống dưới}
{Nguyên tử của các nguyên tố họ s có bán kính nhỏ hơn nguyên tử của nguyên tố họ p cùng chu kỳ}
{Li có bán kính nguyên tử lớn hơn Na}
\loigiai{
\begin{itemchoice}[T1,T2,F3,F4]
\itemch Trong cùng chu kỳ, do điện tích hạt nhân tăng dần, lực hút electron tăng nên bán kính giảm dần từ trái sang phải
\itemch Trong nhóm A, khi xuống dưới thêm lớp electron mới nên bán kính tăng
\itemch Nguyên tử họ s có bán kính lớn hơn nguyên tử họ p cùng chu kỳ do ít electron hơn
\itemch Na ở chu kỳ 3 có bán kính lớn hơn Li ở chu kỳ 2 do có thêm lớp electron
\end{itemchoice}
}
\end{ex}

%%%=================EX_02====================%%%
\begin{ex}
Về độ âm điện của các nguyên tố, chọn phát biểu đúng:
\choiceTF[t]
{\True Trong một chu kỳ, độ âm điện tăng dần từ trái sang phải}
{\True F là nguyên tố có độ âm điện lớn nhất}
{\True Trong một nhóm A, độ âm điện giảm dần từ trên xuống dưới}
{Cs có độ âm điện lớn hơn Cl}
\loigiai{
\begin{itemchoice}[T1,T2,T3,F4]
\itemch Độ âm điện tăng theo chiều tăng của tính phi kim, từ trái sang phải trong chu kỳ
\itemch F có độ âm điện lớn nhất (4,0 theo thang Pauling)
\itemch Khi xuống dưới, electron hóa trị xa hạt nhân hơn nên độ âm điện giảm
\itemch Cs là kim loại kiềm có độ âm điện rất thấp (0,7), thấp hơn nhiều so với Cl (3,0)
\end{itemchoice}
}
\end{ex}

%%%=================EX_03====================%%%
\begin{ex}
Về tính kim loại của các nguyên tố, các phát biểu nào sau đây đúng?
\choiceTF[t]
{\True Tính kim loại giảm dần từ trái sang phải trong một chu kỳ}
{\True Các nguyên tố nhóm IA có tính kim loại mạnh nhất trong cùng chu kỳ}
{\True Tính kim loại tăng dần từ trên xuống dưới trong một nhóm A}
{Be có tính kim loại mạnh hơn Mg}
\loigiai{
\begin{itemchoice}[T1,T2,T3,F4]
\itemch Từ trái sang phải, tính phi kim tăng nên tính kim loại giảm
\itemch Các kim loại kiềm nhóm IA có tính kim loại mạnh nhất do dễ nhường electron nhất
\itemch Khi xuống dưới, electron hóa trị ở xa hạt nhân hơn nên dễ nhường electron, tính kim loại tăng
\itemch Mg ở chu kỳ 3 có tính kim loại mạnh hơn Be ở chu kỳ 2 do electron hóa trị ở xa hạt nhân hơn
\end{itemchoice}
}
\end{ex}

%%%=================EX_04====================%%%
\begin{ex}
Về cấu hình electron, chọn phát biểu đúng:
\choiceTF[t]
{\True Cấu hình electron của Na là $1s^22s^22p^63s^1$}
{Orbital 4s được điền electron sau orbital 3d}
{\True Nguyên tử các nguyên tố họ s có 1 hoặc 2 electron ở lớp ngoài cùng}
{\True Nguyên tử các nguyên tố nhóm VIIIA có 8 electron ở lớp ngoài cùng (trừ He)}
\loigiai{
\begin{itemchoice}[T1,F2,T3,T4]
\itemch Na có Z=11, cấu hình electron đúng là $1s^22s^22p^63s^1$
\itemch Theo nguyên lý aufbau, orbital 4s được điền trước 3d
\itemch Các nguyên tố họ s có cấu hình electron ngoài cùng là ns¹ hoặc ns²
\itemch Các khí hiếm nhóm VIIIA (trừ He) có cấu hình electron lớp ngoài là ns²np⁶
\end{itemchoice}
}
\end{ex}

%%%=================EX_05====================%%%
\begin{ex}
Về tính phi kim của các nguyên tố, chọn phát biểu đúng:
\choiceTF[t]
{\True Tính phi kim tăng dần từ trái sang phải trong một chu kỳ}
{Tính phi kim tăng dần từ trên xuống dưới trong một nhóm A}
{\True Các nguyên tố nhóm VIIA có tính phi kim mạnh thứ hai sau nhóm VIIIA}
{\True Clo có tính phi kim mạnh hơn lưu huỳnh}
\loigiai{
\begin{itemchoice}[T1,F2,T3,T4]
\itemch Trong chu kỳ, từ trái sang phải, độ âm điện tăng nên tính phi kim tăng
\itemch Trong nhóm A, từ trên xuống dưới, electron hóa trị xa hạt nhân hơn nên tính phi kim giảm
\itemch Các halogen nhóm VIIA có tính phi kim mạnh chỉ sau các khí hiếm
\itemch Cl ở nhóm VIIA có tính phi kim mạnh hơn S ở nhóm VIA cùng chu kỳ
\end{itemchoice}
}
\end{ex}
%%%=================EX_06====================%%%
\begin{ex}
Về xu hướng biến đổi tính chất trong bảng tuần hoàn, xét các phát biểu:
\choiceTF[t]
{\True Trong các nguyên tố $_{11}$Na, $_{19}$K, $_{37}$Rb, $_{55}$Cs, nguyên tử Cs có tính khử mạnh nhất}
{\True Năng lượng ion hóa thứ nhất của Mg lớn hơn của Na nhưng nhỏ hơn của Ne}
{Trong dãy O, S, Se, Te, độ âm điện tăng dần do bán kính nguyên tử tăng dần}
{\True Trong các ion $\text{Na}^+$, $\text{Mg}^{2+}$, $\text{Al}^{3+}$, ion $\text{Al}^{3+}$ có bán kính nhỏ nhất}
\loigiai{
\begin{itemchoice}[T1,T2,F3,T4]
\itemch Xuống dưới trong nhóm IA, bán kính tăng, electron hóa trị xa hạt nhân hơn nên dễ nhường electron, tính khử tăng. Cs có tính khử mạnh nhất
\itemch Trong chu kỳ 3, năng lượng ion hóa tăng dần: Na < Mg < ... < Ne do lực hút electron của hạt nhân tăng
\itemch Xuống dưới nhóm VIA, độ âm điện giảm dần do electron hóa trị xa hạt nhân hơn, mặc dù bán kính tăng
\itemch Ba ion này có cùng cấu hình electron của Ne ($2s^22p^6$) nhưng điện tích hạt nhân tăng dần nên bán kính giảm dần: $\text{Na}^+ > \text{Mg}^{2+} > \text{Al}^{3+}$
\end{itemchoice}
}
\end{ex}

%%%=================EX_07====================%%%
\begin{ex}
Về sự biến đổi tính chất các nguyên tố chu kỳ 3, chọn phát biểu đúng:
\choiceTF[t]
{\True Trong dãy $\text{Na}^+ < \text{Mg}^{2+} < \text{Al}^{3+} < \text{Si}^{4+}$, bán kính ion giảm dần}
{Trong dãy Na, Mg, Al, Si, độ âm điện giảm dần do tính phi kim giảm dần}
{\True Trong các oxit $\text{Na}_2\text{O}$, MgO, $\text{Al}_2\text{O}_3$, $\text{SiO}_2$, $\text{P}_4\text{O}_{10}$, tính axit tăng dần từ trái sang phải}
{\True Trong các hiđrua NaH, $\text{MgH}_2$, $\text{AlH}_3$, $\text{SiH}_4$, $\text{PH}_3$, độ bền nhiệt giảm dần từ trái sang phải}
\loigiai{
\begin{itemchoice}[T1,F2,T3,T4]
\itemch Các ion này có cùng cấu hình electron của Ne nhưng điện tích hạt nhân và điện tích ion tăng dần nên bán kính giảm dần
\itemch Trong chu kỳ 3, từ Na đến Si, độ âm điện tăng dần do tính phi kim tăng dần
\itemch Từ trái sang phải, tính kim loại giảm, tính phi kim tăng nên tính bazơ của oxit giảm, tính axit tăng
\itemch Từ trái sang phải, liên kết M-H có tính ion giảm dần, tính cộng hóa trị tăng dần nên độ bền nhiệt giảm
\end{itemchoice}
}
\end{ex}

%%%=================EX_08====================%%%
\begin{ex}
Xét các phát biểu về xu hướng biến đổi tính chất của các đơn chất trong bảng tuần hoàn:
\choiceTF[t]
{\True Điểm nóng chảy của các kim loại kiềm giảm dần theo chiều Li > Na > K > Rb > Cs}
{Trong dãy halogen F$_2$, Cl$_2$, Br$_2$, I$_2$, tính oxi hóa giảm dần nhưng nhiệt độ sôi tăng dần}
{\True Trong một chu kỳ, kim loại có nhiệt độ nóng chảy cao nhất thường là các nguyên tố nhóm VIB (Cr), VIIB (Mn), VIIIB (Fe, Co, Ni)}
{\True Giữa hai phi kim liên tiếp trong cùng chu kỳ, phi kim có số thứ tự nguyên tử lớn hơn có điểm nóng chảy và điểm sôi cao hơn}
\loigiai{
\begin{itemchoice}[T1,F2,T3,T4]
\itemch Kim loại kiềm có liên kết kim loại yếu dần khi xuống dưới do bán kính tăng nên điểm nóng chảy giảm
\itemch Phần đầu phát biểu đúng (tính oxi hóa giảm) nhưng từ "nhưng" là thừa vì cả 2 đều cùng xu hướng giảm dần
\itemch Kim loại chuyển tiếp nhóm VIB, VIIB, VIIIB có liên kết kim loại bền nhất nên điểm nóng chảy cao nhất
\itemch Phi kim có số thứ tự nguyên tử lớn hơn có khối lượng phân tử và lực liên kết phân tử lớn hơn nên có nhiệt độ nóng chảy, sôi cao hơn
\end{itemchoice}
}
\end{ex}

%%%=================EX_09====================%%%
\begin{ex}
Về các đồng vị của nguyên tử hiđro, chọn phát biểu đúng:
\choiceTF[t]
{\True Độ âm điện của các đồng vị $\text{H}$ ($^1\text{H}$), $\text{D}$ ($^2\text{H}$), $\text{T}$ ($^3\text{H}$) là như nhau}
{Bán kính nguyên tử của T lớn hơn của D và của H do có nhiều nơtron hơn}
{\True Trong các phân tử $\text{H}_2$, $\text{D}_2$, $\text{T}_2$, phân tử $\text{T}_2$ có năng lượng liên kết lớn nhất}
{\True Trong các hợp chất RH, RD, RT (R là gốc hữu cơ), liên kết R-T bền vững nhất}
\loigiai{
\begin{itemchoice}[T1,F2,T3,T4]
\itemch Các đồng vị có cùng số electron hóa trị và cấu hình electron nên có cùng độ âm điện
\itemch Các đồng vị có cùng số proton và electron nên có cùng bán kính nguyên tử, số nơtron không ảnh hưởng
\itemch Do khối lượng tăng dần H < D < T nên năng lượng dao động phân tử giảm, dẫn đến năng lượng liên kết tăng
\itemch Tương tự, do khối lượng T lớn nhất nên liên kết R-T bền vững nhất (hiệu ứng đồng vị động học)
\end{itemchoice}
}
\end{ex}

%%%=================EX_10====================%%%
\begin{ex}
Về sự biến thiên tính chất các nguyên tố chu kỳ 4, chọn phát biểu đúng:
\choiceTF[t]
{\True Trong dãy K, Ca, Sc, Ti, V, Cr, Mn, Fe, Co, Ni, Cu, nhiệt độ nóng chảy đạt cực đại ở Cr và W}
{Trong các ion $\text{K}^+$, $\text{Ca}^{2+}$, $\text{Sc}^{3+}$, $\text{Ti}^{4+}$, ion $\text{Ti}^{4+}$ có tính oxi hóa mạnh nhất}
{\True Trong dãy KCl, CaCl$_2$, ScCl$_3$, TiCl$_4$, độ tan trong nước giảm dần do tính ion giảm dần}
{\True Zn có nhiệt độ nóng chảy thấp hơn Cu do các electron 3d đã được điền đầy, không tham gia liên kết kim loại}
\loigiai{
\begin{itemchoice}[T1,F2,T3,T4]
\itemch Cr có orbital d bán đầy (3d$^5$) và W có orbital d đầy (5d$^5$) nên liên kết kim loại bền nhất, nhiệt độ nóng chảy cao nhất
\itemch Các ion này có cùng cấu hình electron của Ar nên tính oxi hóa phụ thuộc vào điện tích, $\text{Ti}^{4+}$ có tính oxi hóa yếu nhất
\itemch Từ K đến Ti, tính ion của liên kết trong clorua giảm, tính cộng hóa trị tăng nên độ tan trong nước giảm
\itemch Zn có orbital 3d đầy (3d$^{10}$) nên các electron này không tham gia liên kết kim loại, làm cho liên kết yếu hơn Cu (3d$^9$4s$^1$)
\end{itemchoice}
}
\end{ex}
%%%=================EX_11====================%%%
\begin{ex}
Về các nguyên tố chu kỳ 4, xét các phát biểu sau:
\choiceTF[t]
{\True Các nguyên tố Cr ($Z=24$) và Cu ($Z=29$) có electron độc thân ở orbital 4s}
{Trong dãy Ga, Ge, As, Se, Br, tính axit của oxit cao nhất giảm dần}
{\True Độ dẫn điện của các kim loại trong chu kỳ 4 đạt cực đại ở Cu do có 1 electron tự do/nguyên tử và bán kính ion nhỏ}
{\True Trong các ion $\text{Mn}^{2+}$, $\text{Fe}^{2+}$, $\text{Co}^{2+}$, $\text{Ni}^{2+}$, $\text{Cu}^{2+}$, ion $\text{Mn}^{2+}$ có tính khử mạnh nhất}
\loigiai{
\begin{itemchoice}[T1,F2,T3,T4]
\itemch Cr có cấu hình 3d$^5$4s$^1$, Cu có cấu hình 3d$^{10}$4s$^1$, đều có 1 electron ở orbital 4s
\itemch Từ Ga đến Br, tính phi kim tăng dần nên tính axit của oxit cao nhất tăng dần
\itemch Cu có độ dẫn điện tốt nhất do có 1 electron tự do/nguyên tử (4s$^1$) và bán kính ion nhỏ, tạo điều kiện tốt cho sự dịch chuyển electron
\itemch Trong dãy ion này, $\text{Mn}^{2+}$ có cấu hình 3d$^5$ (bán đầy bền) nên khó mất electron nhất, có tính khử mạnh nhất
\end{itemchoice}
}
\end{ex}

%%%=================EX_12====================%%%
\begin{ex}
Về tính chất các nguyên tố họ d, xét các phát biểu:
\choiceTF[t]
{\True Trong các ion $\text{Sc}^{3+}$, $\text{Ti}^{3+}$, $\text{V}^{3+}$, $\text{Cr}^{3+}$, $\text{Mn}^{3+}$, ion $\text{Sc}^{3+}$ có tính oxi hóa yếu nhất}
{Trong một chu kỳ, bán kính nguyên tử của nguyên tố họ d luôn nhỏ hơn nguyên tố họ s cùng chu kỳ}
{\True Các nguyên tố họ d thường có nhiều số oxi hóa khác nhau do các electron d tham gia liên kết}
{\True Trong một chu kỳ, nhiệt độ nóng chảy của các kim loại họ d thường đạt cực đại ở nguyên tố có orbital d bán đầy hoặc gần bán đầy}
\loigiai{
\begin{itemchoice}[T1,F2,T3,T4]
\itemch $\text{Sc}^{3+}$ có cấu hình electron của Ar (3d$^0$), các ion còn lại có thêm electron d nên $\text{Sc}^{3+}$ có tính oxi hóa yếu nhất
\itemch Trong một chu kỳ, nguyên tố họ d có bán kính lớn hơn nguyên tố họ s do có thêm electron ở orbital d làm tăng lực đẩy electron
\itemch Electron d có năng lượng gần với electron hóa trị nên dễ tham gia liên kết, tạo nhiều số oxi hóa khác nhau
\itemch Orbital d bán đầy (d$^5$) hoặc gần bán đầy có nhiều electron không ghép cặp, tăng cường liên kết kim loại, làm tăng nhiệt độ nóng chảy
\end{itemchoice}
}
\end{ex}

%%%=================EX_13====================%%%
\begin{ex}
Về năng lượng ion hóa của các nguyên tố, chọn phát biểu đúng:
\choiceTF[t]
{\True Trong các nguyên tố $_{3}$Li, $_{11}$Na, $_{19}$K, $_{37}$Rb, $_{55}$Cs, năng lượng ion hóa thứ nhất giảm dần từ trên xuống dưới}
{\True Be ($1s^22s^2$) có năng lượng ion hóa thứ nhất lớn hơn B ($1s^22s^22p^1$) do orbital 2s đầy electron bền vững}
{Trong một chu kỳ, năng lượng ion hóa thứ hai luôn lớn hơn năng lượng ion hóa thứ nhất}
{\True Nguyên tố có năng lượng ion hóa thứ nhất lớn nhất trong bảng tuần hoàn là He}
\loigiai{
\begin{itemchoice}[T1,T2,F3,T4]
\itemch Xuống dưới trong nhóm IA, bán kính tăng, electron hóa trị xa hạt nhân hơn nên năng lượng ion hóa giảm
\itemch Orbital 2s đầy electron ở Be tạo cấu hình bền vững nên khó mất electron hơn B có 1 electron p
\itemch Năng lượng ion hóa thứ hai không phải luôn lớn hơn thứ nhất. Ví dụ: Na$^+$ → Na$^{2+}$ cần ít năng lượng hơn Na → Na$^+$
\itemch He có năng lượng ion hóa thứ nhất lớn nhất (2372 kJ/mol) do bán kính nhỏ nhất và orbital 1s đầy electron
\end{itemchoice}
}
\end{ex}

%%%=================EX_14====================%%%
\begin{ex}
Về sự biến thiên độ âm điện trong bảng tuần hoàn, xét các phát biểu:
\choiceTF[t]
{Độ âm điện của N lớn hơn của O do N có 3 electron độc thân bền vững}
{\True Trong các halogen (F, Cl, Br, I), F có độ âm điện lớn nhất nhưng có năng lượng liên kết F-F nhỏ nhất}
{\True Trong các nguyên tố nhóm IVA (C, Si, Ge, Sn, Pb), C có độ âm điện lớn nhất và có xu hướng tạo liên kết cộng hóa trị nhiều nhất}
{\True Be có độ âm điện lớn hơn Li và Na do hạt nhân Be có lực hút electron mạnh hơn}
\loigiai{
\begin{itemchoice}[F1,T2,T3,T4]
\itemch O có độ âm điện lớn hơn N do O có bán kính nhỏ hơn và lực hút electron của hạt nhân mạnh hơn
\itemch F có độ âm điện lớn nhất do bán kính nhỏ nhất, nhưng liên kết F-F yếu do lực đẩy giữa các cặp electron không liên kết
\itemch C có bán kính nhỏ nhất trong nhóm IVA nên có độ âm điện lớn nhất và xu hướng tạo liên kết cộng hóa trị mạnh nhất
\itemch Be (Z=4) có điện tích hạt nhân lớn hơn Li (Z=3) nên có lực hút electron mạnh hơn, dẫn đến độ âm điện lớn hơn
\end{itemchoice}
}
\end{ex}

%%%=================EX_15====================%%%
\begin{ex}
Về tính oxi hóa và tính khử của các nguyên tố, chọn phát biểu đúng:
\choiceTF[t]
{\True Trong nhóm VIIA, tính oxi hóa giảm dần theo thứ tự: F > Cl > Br > I nhưng tính khử tăng dần theo thứ tự ngược lại}
{Nguyên tử của nguyên tố có độ âm điện càng lớn thì tính oxi hóa càng mạnh và tính khử càng yếu}
{\True Các phi kim có tính oxi hóa mạnh nhưng anion của chúng lại có tính khử mạnh}
{\True H có thể thể hiện tính oxi hóa trong phản ứng với kim loại kiềm và tính khử trong phản ứng với phi kim}
\loigiai{
\begin{itemchoice}[T1,F2,T3,T4]
\itemch Xuống dưới nhóm VIIA, do bán kính tăng nên tính oxi hóa giảm và tính khử tăng
\itemch Phần đầu đúng (độ âm điện lớn, tính oxi hóa mạnh) nhưng không có mối liên hệ trực tiếp với tính khử
\itemch Phi kim dễ nhận electron (tính oxi hóa mạnh) nhưng anion của chúng đã có thêm electron nên dễ nhường electron (tính khử mạnh)
\itemch H có thể nhường electron cho phi kim (thể hiện tính khử) và nhận electron từ kim loại kiềm (thể hiện tính oxi hóa)
\end{itemchoice}
}
\end{ex}

%%%=================EX_16====================%%%
\begin{ex}
Về cấu hình electron và tính chất của các nguyên tố họ f, chọn phát biểu đúng:
\choiceTF[t]
{\True Trong dãy Lanthanoid, từ La đến Lu, bán kính nguyên tử và ion $\text{Ln}^{3+}$ giảm dần do hiệu ứng co lan-than}
{Các electron f không tham gia liên kết hóa học nên các nguyên tố họ f không có nhiều trạng thái oxi hóa khác nhau}
{\True Các ion $\text{Ln}^{3+}$ đều có màu (trừ La$^{3+}$ và Lu$^{3+}$) do các chuyển dời electron giữa các obital f}
{\True Độ từ của các nguyên tố họ f đạt cực đại ở các ion có orbital f bán đầy (Gd$^{3+}$) hoặc gần bán đầy (Dy$^{3+}$)}
\loigiai{
\begin{itemchoice}[T1,F2,T3,T4]
\itemch Hiệu ứng co lan-than do các electron f không che chắn tốt làm tăng hiệu quả lực hút của hạt nhân
\itemch Nhiều nguyên tố họ f có các electron f tham gia liên kết, tạo nhiều trạng thái oxi hóa như Ce$^{3+}$/Ce$^{4+}$, Eu$^{2+}$/Eu$^{3+}$
\itemch Các chuyển dời electron f-f tạo ra ánh sáng trong vùng khả kiến, gây ra màu sắc đặc trưng của các ion $\text{Ln}^{3+}$
\itemch Orbital f bán đầy (f$^7$) hoặc gần bán đầy có nhiều electron không ghép cặp, tạo moment từ lớn
\end{itemchoice}
}
\end{ex}

%%%=================EX_17====================%%%
\begin{ex}
Về sự biến thiên tính chất các phân tử trong bảng tuần hoàn, xét các phát biểu:
\choiceTF[t]
{\True Nhiệt độ sôi của các hiđrua $\text{NH}_3$, $\text{H}_2\text{O}$, HF cao bất thường do có liên kết hiđro}
{\True Bậc liên kết trong phân tử $\text{N}_2$ (3) lớn hơn trong $\text{O}_2$ (2) nên năng lượng liên kết của $\text{N}_2$ lớn hơn $\text{O}_2$}
{Góc liên kết H-E-H (E là nguyên tố nhóm VIA) tăng dần theo thứ tự: $\text{H}_2\text{O}$ < $\text{H}_2\text{S}$ < $\text{H}_2\text{Se}$ < $\text{H}_2\text{Te}$}
{\True Trong các phân tử $\text{CCl}_4$, $\text{SiCl}_4$, $\text{GeCl}_4$, $\text{SnCl}_4$, tính phân cực tăng dần từ trên xuống do bán kính nguyên tử trung tâm tăng}
\loigiai{
\begin{itemchoice}[T1,T2,F3,T4]
\itemch N, O, F có độ âm điện lớn và bán kính nhỏ nên tạo được liên kết hiđro mạnh, làm tăng nhiệt độ sôi
\itemch $\text{N}_2$ có 3 liên kết (1 σ và 2 π) còn $\text{O}_2$ có 2 liên kết (1 σ và 1 π) nên năng lượng liên kết của $\text{N}_2$ lớn hơn
\itemch Góc liên kết H-E-H giảm dần khi xuống dưới do các orbital p của E có kích thước lớn hơn, chồng lấn ít hơn
\itemch Khi xuống dưới, bán kính nguyên tử trung tâm tăng làm tăng độ phân cực của liên kết E-Cl và của phân tử
\end{itemchoice}
}
\end{ex}

%%%=================EX_18====================%%%
\begin{ex}
Về các đơn chất trong bảng tuần hoàn, chọn phát biểu đúng:
\choiceTF[t]
{\True Điểm nóng chảy của Si ($1414°$C) cao hơn của C (kim cương, $3550°$C) do liên kết Si-Si yếu hơn liên kết C-C}
{\True Tinh thể kim loại chuyển tiếp thường cứng và có nhiệt độ nóng chảy cao do có nhiều electron không ghép cặp tham gia liên kết}
{P trắng có cấu trúc phân tử $\text{P}_4$ còn S có cấu trúc phân tử $\text{S}_8$ do liên kết P-P bền vững hơn liên kết S-S}
{\True Các nguyên tố nhóm VIIIA tồn tại ở dạng đơn nguyên tử do các electron lớp ngoài cùng đã được điền đầy}
\loigiai{
\begin{itemchoice}[T1,T2,F3,T4]
\itemch C có bán kính nhỏ hơn Si nên tạo được liên kết cộng hóa trị mạnh hơn, dẫn đến nhiệt độ nóng chảy cao hơn
\itemch Electron d không ghép cặp của kim loại chuyển tiếp tham gia liên kết kim loại làm tăng độ cứng và nhiệt độ nóng chảy
\itemch P trắng có cấu trúc $\text{P}_4$ do góc liên kết phù hợp với cấu trúc tứ diện, không liên quan đến độ bền liên kết
\itemch Các khí hiếm có 8 electron lớp ngoài (trừ He có 2) tạo cấu hình electron bền vững nên tồn tại ở dạng đơn nguyên tử
\end{itemchoice}
}
\end{ex}

%%%=================EX_19====================%%%
\begin{ex}
Về độ âm điện và tính chất liên kết trong bảng tuần hoàn, xét các phát biểu:
\choiceTF[t]
{\True Liên kết Ba-F có tính ion nhiều hơn liên kết Li-F do Ba có độ âm điện nhỏ hơn Li}
{Liên kết trong phân tử $\text{H}_2$ là liên kết cộng hóa trị không cực do hai nguyên tử H có cùng độ âm điện}
{\True Trong phân tử $\text{BF}_3$, liên kết B-F có tính cộng hóa trị phân cực do chênh lệch độ âm điện giữa B và F}
{\True Các oxit của kim loại kiềm thổ có tính ion nhiều hơn các hiđrua tương ứng do O có độ âm điện lớn hơn H}
\loigiai{
\begin{itemchoice}[T1,F2,T3,T4]
\itemch Ba (0,89) có độ âm điện nhỏ hơn Li (0,98) nên chênh lệch độ âm điện Ba-F lớn hơn Li-F, tính ion nhiều hơn
\itemch $\text{H}_2$ có liên kết cộng hóa trị không cực, nhưng không phải do hai nguyên tử H có cùng độ âm điện mà do chúng là hai nguyên tử giống nhau
\itemch Độ âm điện của F (3,98) lớn hơn nhiều so với B (2,04) nên liên kết B-F có tính cộng hóa trị phân cực mạnh
\itemch O (3,44) có độ âm điện lớn hơn H (2,20) nên oxit của kim loại kiềm thổ có tính ion nhiều hơn hiđrua tương ứng
\end{itemchoice}
}
\end{ex}

%%%=================EX_20====================%%%
\begin{ex}
Về electron hoá trị và các trạng thái oxi hóa, chọn phát biểu đúng:
\choiceTF[t]
{\True Trong hợp chất $\text{KMnO}_4$, trạng thái oxi hóa cao nhất +7 của Mn tương ứng với việc tham gia cả electron 3d và 4s}
{Các nguyên tố họ s chỉ có một trạng thái oxi hóa duy nhất trong hợp chất do chỉ có electron s tham gia liên kết}
{\True Cu có thể có trạng thái oxi hóa +1 ($\text{Cu}_2\text{O}$) và +2 (CuO) do có thể mất 1 electron 4s hoặc thêm 1 electron 3d}
{\True Trong $\text{CrO}_4^{2-}$ và $\text{Cr}_2\text{O}_7^{2-}$, Cr đều có trạng thái oxi hóa +6 do mất hết 6 electron hóa trị (3d$^5$4s$^1$)}
\loigiai{
\begin{itemchoice}[T1,F2,T3,T4]
\itemch Mn ($3d^54s^2$) đạt trạng thái oxi hóa +7 trong $\text{KMnO}_4$ do mất 7 electron (5 electron 3d và 2 electron 4s)
\itemch Be có thể có trạng thái oxi hóa 0 và +2, Sn có thể có trạng thái oxi hóa +2 và +4
\itemch Cu ($3d^{10}4s^1$) có thể mất 1 electron 4s tạo Cu$^+$ hoặc thêm 1 electron từ 4s vào 3d và mất 2 electron 4s tạo Cu$^{2+}$
\itemch Cr ($3d^54s^1$) trong cả $\text{CrO}_4^{2-}$ và $\text{Cr}_2\text{O}_7^{2-}$ đều mất 6 electron hóa trị tạo trạng thái oxi hóa +6
\end{itemchoice}
}
\end{ex}

