```latex
%%%%=================EX_01====================%%%
\begin{ex}
    Nguyên tử của nguyên tố nhóm A nào sau đây có xu hướng tạo thành lớp vỏ ngoài cùng có 8 electron khi hình thành liên kết hóa học?
    \choice
    {He}
    {\True Cl}
    {H}
    {Fe}
    \loigiai{Nguyên tử của nguyên tố nhóm A, ngoại trừ H và He, có xu hướng tạo thành lớp vỏ ngoài cùng có 8 electron khi hình thành liên kết hóa học (quy tắc octet).}
\end{ex}

%%%%=================EX_02====================%%%
\begin{ex}
    Quy tắc octet áp dụng cho nguyên tử của nguyên tố nào sau đây?
    \choice
    {Nguyên tố nhóm B}
    {Nguyên tố khí hiếm}
    {\True Nguyên tố nhóm A}
    {Tất cả các nguyên tố}
    \loigiai{Quy tắc octet áp dụng cho nguyên tử của nguyên tố nhóm A khi hình thành liên kết hóa học.}
\end{ex}

%%%%=================EX_03====================%%%
\begin{ex}
    Khi hình thành liên kết hóa học, nguyên tử Li có xu hướng đạt cấu hình electron của nguyên tử nào?
    \choice
    {Be}
    {\True He}
    {Ne}
    {Ar}
    \loigiai{Nguyên tử Li có 3 electron. Khi hình thành liên kết hóa học, Li có xu hướng mất 1 electron để đạt cấu hình electron của He (2 electron).}
\end{ex}

%%%%=================EX_04====================%%%
\begin{ex}
    Ion $Na^+$ có bao nhiêu electron?
    \choice
    {11}
    {9}
    {\True 10}
    {12}
    \loigiai{Na (Z = 11) có 11 electron. $Na^+$ mất 1 electron nên có 10 electron.}
\end{ex}

%%%%=================EX_05====================%%%
\begin{ex}
    Ion $O^{2-}$ có bao nhiêu electron?
    \choice
    {6}
    {7}
    {9}
    {\True 10}
    \loigiai{O (Z = 8) có 8 electron. $O^{2-}$ nhận thêm 2 electron nên có 10 electron.}
\end{ex}

%%%%=================EX_06====================%%%
\begin{ex}
    Trong phân tử $N_2$, mỗi nguyên tử N có bao nhiêu electron ở lớp vỏ ngoài cùng?
    \choice
    {5}
    {7}
    {\True 8}
    {10}
    \loigiai{Trong phân tử $N_2$, hai nguyên tử N liên kết với nhau bằng liên kết ba. Mỗi nguyên tử N có 5 electron lớp ngoài cùng, khi hình thành liên kết, mỗi nguyên tử N có 8 electron ở lớp vỏ ngoài cùng.}
\end{ex}

%%%%=================EX_07====================%%%
\begin{ex}
    Nguyên tử của nguyên tố X có 7 electron lớp ngoài cùng. Khi hình thành liên kết hóa học, X có xu hướng:
    \choice
    {Nhường 7 electron}
    {Nhường 1 electron}
    {\True Nhận 1 electron}
    {Nhận 7 electron}
    \loigiai{Nguyên tử của nguyên tố X có 7 electron lớp ngoài cùng. Khi hình thành liên kết hóa học, X có xu hướng nhận 1 electron để đạt cấu hình 8 electron lớp ngoài cùng (quy tắc octet).}
\end{ex}

%%%%=================EX_08====================%%%
\begin{ex}
    Nguyên tử của nguyên tố Y thuộc nhóm IIA. Khi hình thành liên kết hóa học, Y có xu hướng:
    \choice
    {Nhận 6 electron}
    {\True Nhường 2 electron}
    {Nhận 2 electron}
    {Nhường 6 electron}
    \loigiai{Nguyên tử của nguyên tố Y thuộc nhóm IIA, có 2 electron lớp ngoài cùng. Khi hình thành liên kết hóa học, Y có xu hướng nhường 2 electron để đạt cấu hình 8 electron lớp ngoài cùng (quy tắc octet).}
\end{ex}


%%%%=================EX_09====================%%%
\begin{ex}
    Khí hiếm nào sau đây có 2 electron ở lớp vỏ ngoài cùng?
    \choice
    {Ne}
    {\True He}
    {Ar}
    {Kr}
    \loigiai{He là khí hiếm duy nhất có 2 electron ở lớp vỏ ngoài cùng.}
\end{ex}

%%%%=================EX_10====================%%%
\begin{ex}
    Nguyên tố nào sau đây KHÔNG tuân theo quy tắc octet khi hình thành liên kết hóa học?
    \choice
    {O}
    {Cl}
    {N}
    {\True H}
    \loigiai{H là một ngoại lệ của quy tắc octet. Khi hình thành liên kết hóa học, H có xu hướng đạt cấu hình electron của He (2 electron), chứ không phải 8 electron.}
\end{ex}


%%%%=================TF_01====================%%%
\begin{ex}
    Quy tắc octet áp dụng cho tất cả các nguyên tố trong bảng tuần hoàn.
    \choiceTF[t]
    {\True Quy tắc octet áp dụng cho tất cả các nguyên tố trong bảng tuần hoàn.}
    {Quy tắc octet áp dụng cho nguyên tử của nguyên tố nhóm A.}
    {\True Quy tắc octet không áp dụng cho nguyên tử H và He.}
    {Quy tắc octet phát biểu rằng khi hình thành liên kết hóa học, nguyên tử có xu hướng đạt cấu hình electron bền vững của khí hiếm gần nó nhất.}
    \loigiai{Quy tắc octet áp dụng cho nguyên tử của nguyên tố nhóm A, ngoại trừ H và He. }
\end{ex}


%%%%=================TF_02====================%%%
\begin{ex}
	Các phát biểu sau đúng hay sai?
	\choiceTF[t]
	{Nguyên tử Li có xu hướng nhận thêm 7 electron khi hình thành liên kết hóa học.}
	{\True Ion $Li^+$ có cấu hình electron giống He.}
	{\True Ion $O^{2-}$ có 10 electron.}
    {Trong phân tử $N_2$, mỗi nguyên tử N có 5 electron lớp ngoài cùng.}
	\loigiai{
        - Nguyên tử Li có xu hướng nhường 1 electron khi hình thành liên kết hóa học.
        - Ion $Li^+$ có 2 electron, giống cấu hình electron của He.
        - Ion $O^{2-}$ có 8 + 2 = 10 electron.
        - Trong phân tử $N_2$, mỗi nguyên tử N có 8 electron lớp ngoài cùng.
    }
\end{ex}

%%%%=================TF_03====================%%%
\begin{ex}
	Xét các phát biểu sau về quy tắc octet:
	\choiceTF[t]
	{\True Quy tắc octet còn được gọi là quy tắc bát tử.}
	{Quy tắc octet áp dụng cho tất cả các nguyên tố.}
	{\True Nguyên tử Na có xu hướng nhường 1 electron để đạt cấu hình octet.}
    {\True Ion $Cl^-$ có cấu hình electron giống Ar.}
	\loigiai{
        - Quy tắc octet (octet - bát tử) nói về xu hướng đạt 8 electron lớp ngoài cùng của nguyên tử khi hình thành liên kết hóa học.
        - Quy tắc octet không áp dụng cho tất cả các nguyên tố, mà chỉ áp dụng cho nguyên tử của nguyên tố nhóm A (trừ H và He).
        - Nguyên tử Na (Z = 11) có 1 electron lớp ngoài cùng, có xu hướng nhường 1 electron để đạt cấu hình octet của khí hiếm Ne (Z = 10).
        - Ion $Cl^-$ (17 + 1 = 18 electron) có cấu hình electron giống Ar (Z = 18).
    }
\end{ex}


%%%%=================TF_04====================%%%
\begin{ex}
	Cho các phát biểu sau:
	\choiceTF[t]
	{\True Nguyên tử Cl có xu hướng nhận 1 electron khi tham gia liên kết hóa học.}
	{Ion $Mg^{2+}$ có 10 electron.}
	{\True Trong phân tử $F_2$, mỗi nguyên tử F có 8 electron lớp ngoài cùng.}
    {Nguyên tử H luôn tuân theo quy tắc octet.}
	\loigiai{
        - Nguyên tử Cl (nhóm VIIA) có 7 electron lớp ngoài cùng, có xu hướng nhận 1 electron để đạt cấu hình octet.
        - Ion $Mg^{2+}$ (12 - 2 = 10 electron).
        - Trong phân tử $F_2$, mỗi nguyên tử F có 7 electron lớp ngoài cùng. Khi hình thành liên kết, mỗi nguyên tử F góp chung 1 electron để tạo thành liên kết cộng hóa trị, lúc này mỗi nguyên tử F có 8 electron lớp ngoài cùng.
        - Nguyên tử H là ngoại lệ của quy tắc octet, H có xu hướng đạt 2 electron lớp ngoài cùng khi tham gia liên kết hóa học.
    }
\end{ex}


%%%%=================TF_05====================%%%
\begin{ex}
	Các phát biểu sau đúng hay sai?
	\choiceTF[t]
	{Nguyên tử O có xu hướng nhận 6 electron khi tham gia liên kết hóa học.}
	{\True Ion $S^{2-}$ có cấu hình electron giống Ar.}
	{Trong phân tử $O_2$, mỗi nguyên tử O có 6 electron lớp ngoài cùng.}
    {\True Nguyên tử He có 2 electron lớp ngoài cùng.}
	\loigiai{
        - Nguyên tử O (nhóm VIA) có 6 electron lớp ngoài cùng, có xu hướng nhận 2 electron để đạt cấu hình octet.
        - Ion $S^{2-}$ (16 + 2 = 18 electron) có cấu hình electron giống Ar (Z = 18).
        - Trong phân tử $O_2$, mỗi nguyên tử O có 8 electron lớp ngoài cùng.
        - Nguyên tử He có 2 electron và nằm trong nhóm VIIIA (khí hiếm).
    }
\end{ex}

%%%%=================TF_06====================%%%
\begin{ex}
    Phát biểu nào sau đây đúng về quy tắc octet?
    \choiceTF[t]
    {\True  Nguyên tử của các nguyên tố nhóm A có xu hướng đạt 8 electron lớp ngoài cùng khi hình thành liên kết hóa học.}
    {Quy tắc octet áp dụng cho tất cả các nguyên tố.}
    {\True  Ion $Na^+$ có cấu hình octet.}
    {Nguyên tử H có xu hướng đạt 8 electron lớp ngoài cùng khi hình thành liên kết.}
    \loigiai{
        - Quy tắc octet áp dụng cho các nguyên tố nhóm A, trừ H và He. 
        - Ion $Na^+$ (11 - 1 = 10 electron), có cấu hình giống Ne, thỏa mãn quy tắc octet.
        - Nguyên tử H là ngoại lệ của quy tắc octet.
    }
\end{ex}

%%%%=================TF_07====================%%%
\begin{ex}
    Cho các phát biểu sau:
    \choiceTF[t]
    {\True  Nguyên tử F có xu hướng nhận thêm 1 electron.}
    {Ion $Ca^{2+}$ có 20 electron.}
    {Trong phân tử $Cl_2$, mỗi nguyên tử Cl có 7 electron lớp ngoài cùng.}
    {\True Nguyên tử He là một ngoại lệ của quy tắc octet.}
    \loigiai{
        - F thuộc nhóm VIIA, có 7 electron lớp ngoài cùng, có xu hướng nhận 1 electron để đạt octet.
        - $Ca^{2+}$ có 20 - 2 = 18 electron.
        - Trong phân tử $Cl_2$, mỗi nguyên tử Cl có 8 electron lớp ngoài cùng.
        - He có 2 electron lớp ngoài cùng, là một ngoại lệ của quy tắc octet.
    }
\end{ex}


%%%%=================TF_08====================%%%
\begin{ex}
    Xét các phát biểu sau:
    \choiceTF[t]
    {Nguyên tử N có xu hướng nhận thêm 5 electron khi hình thành liên kết.}
    {\True Ion $N^{3-}$ có 10 electron.}
    {\True Trong phân tử $H_2O$, nguyên tử O có 8 electron lớp ngoài cùng.}
    {Nguyên tử Li tuân theo quy tắc octet.}
    \loigiai{
        - N có 5 electron lớp ngoài cùng, có xu hướng nhận thêm 3 electron hoặc hình thành liên kết cộng hóa trị để đạt octet.
        - $N^{3-}$ có 7 + 3 = 10 electron.
        - Trong $H_2O$, O có 8 electron lớp ngoài cùng (6 electron của O và 2 electron góp chung từ 2 nguyên tử H).
        - Li là ngoại lệ của quy tắc octet, có xu hướng đạt 2 electron lớp ngoài cùng.
    }
\end{ex}

%%%%=================TF_09====================%%%
\begin{ex}
    Cho các phát biểu sau:
    \choiceTF[t]
    {\True Nguyên tử Mg có xu hướng nhường đi 2 electron.}
    {Ion $Al^{3+}$ có 10 electron.}
    {Trong phân tử $CH_4$, nguyên tử C có 4 electron lớp ngoài cùng.}
    {\True Khí hiếm Ne tuân theo quy tắc octet.}
    \loigiai{
        - Mg thuộc nhóm IIA, có 2 electron lớp ngoài cùng, có xu hướng nhường 2 electron.
        - $Al^{3+}$ có 13 - 3 = 10 electron.
        - Trong $CH_4$, nguyên tử C có 8 electron lớp ngoài cùng.
        - Ne có 8 electron lớp ngoài cùng (trừ He), nên tuân theo quy tắc octet.
    }
\end{ex}

%%%%=================TF_10====================%%%
\begin{ex}
    Các phát biểu sau đúng hay sai?
    \choiceTF[t]
    {Nguyên tử S có xu hướng nhận thêm 4 electron khi hình thành liên kết.}
    {\True Ion $P^{3-}$ có 18 electron.}
    {Trong phân tử $CO_2$, nguyên tử C có 6 electron lớp ngoài cùng.}
    {\True Nguyên tử Be là một ngoại lệ của quy tắc octet.}
    \loigiai{
        - S thuộc nhóm VIA, có xu hướng nhận 2 electron.
        - $P^{3-}$ có 15 + 3 = 18 electron.
        - Trong $CO_2$, C có 8 electron lớp ngoài cùng.
        - Be có xu hướng đạt 4 electron lớp ngoài cùng, là một ngoại lệ của quy tắc octet.
    }
\end{ex}


%%%%=================BT_01====================%%%
\begin{bt}
    Hãy giải thích tại sao nguyên tử Na có xu hướng tạo thành ion $Na^+$, còn nguyên tử Cl có xu hướng tạo thành ion $Cl^-$.
    \loigiai{
        - Na (Z = 11) có cấu hình electron là $1s^22s^22p^63s^1$. Na có 1 electron ở lớp ngoài cùng. Để đạt cấu hình electron bền vững của khí hiếm gần nhất (Ne), Na có xu hướng nhường 1 electron để tạo thành ion $Na^+$.
        - Cl (Z = 17) có cấu hình electron là $1s^22s^22p^63s^23p^5$. Cl có 7 electron ở lớp ngoài cùng. Để đạt cấu hình electron bền vững của khí hiếm gần nhất (Ar), Cl có xu hướng nhận 1 electron để tạo thành ion $Cl^-$.
    }
\end{bt}


%%%%=================BT_02====================%%%
\begin{bt}
    Vẽ sơ đồ Lewis của phân tử $N_2$. Giải thích tại sao phân tử $N_2$ lại bền vững.
    \loigiai{
        Sơ đồ Lewis của phân tử $N_2$: $:N \equiv N:$

        Phân tử $N_2$ bền vững vì mỗi nguyên tử N đều đạt được cấu hình octet (8 electron lớp ngoài cùng) nhờ liên kết ba giữa hai nguyên tử N. Liên kết ba rất bền vững, do đó phân tử $N_2$ cũng rất bền vững.
    }
\end{bt}


%%%%=================BT_03====================%%%
\begin{bt}
    Giải thích sự hình thành liên kết ion trong phân tử NaCl.
    \loigiai{
        - Na (Z = 11) có 1 electron ở lớp ngoài cùng, có xu hướng nhường 1e để tạo thành ion $Na^+$.
        - Cl (Z = 17) có 7 electron ở lớp ngoài cùng, có xu hướng nhận 1e để tạo thành ion $Cl^-$.
        - Ion $Na^+$ và $Cl^-$ hút nhau bằng lực hút tĩnh điện, tạo thành liên kết ion trong phân tử NaCl.
    }
\end{bt}



%%%%=================BT_04====================%%%
\begin{bt}
    Hãy viết cấu hình electron của ion $Mg^{2+}$ và $O^{2-}$.
    \loigiai{
        - Mg (Z = 12): $1s^22s^22p^63s^2$. $Mg^{2+}$ nhường 2 electron lớp ngoài cùng, cấu hình e: $1s^22s^22p^6$ (giống Ne).
        - O (Z = 8): $1s^22s^22p^4$. $O^{2-}$ nhận thêm 2 electron, cấu hình e: $1s^22s^22p^6$ (giống Ne).
    }
\end{bt}

%%%%=================BT_05====================%%%
\begin{bt}
    So sánh số electron của nguyên tử F và ion $F^-$.
    \loigiai{
        - Nguyên tử F (Z = 9) có 9 electron.
        - Ion $F^-$ nhận thêm 1 electron, có 9 + 1 = 10 electron.
    }
\end{bt}

%%%%=================BT_06====================%%%
\begin{bt}
    Tại sao nguyên tử H là một ngoại lệ của quy tắc octet?
    \loigiai{
        Nguyên tử H (Z = 1) chỉ có 1 electron. Khi tham gia liên kết hóa học, H có xu hướng đạt cấu hình electron của He (2 electron), chứ không phải 8 electron như quy tắc octet. Do đó, H là một ngoại lệ của quy tắc octet.
    }
\end{bt}
```