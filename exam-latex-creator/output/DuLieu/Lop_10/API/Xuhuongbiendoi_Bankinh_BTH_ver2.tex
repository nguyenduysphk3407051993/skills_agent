
%%%%=================EX01====================%%%
\begin{ex}
	Nguyên tố nào sau đây có bán kính nguyên tử lớn nhất?
	\choice
	{Li}
	{Na}
	{K}
    {\True Rb}
	\loigiai{Bán kính nguyên tử tăng dần khi đi xuống một nhóm A trong bảng tuần hoàn do số lớp electron tăng. Rb nằm ở vị trí thấp nhất trong nhóm IA nên có bán kính lớn nhất.}
\end{ex}
%%%%=================EX_02====================%%%
\begin{ex}
	Nguyên tố nào sau đây có bán kính nguyên tử nhỏ nhất?
	\choice
	{F}
	{\True Ne}
	{Na}
	{O}
	\loigiai{Bán kính nguyên tử giảm dần từ trái sang phải trong một chu kì do điện tích hạt nhân tăng. Ne và F nằm trong cùng chu kì 2, nhưng Ne có điện tích hạt nhân lớn hơn, nên bán kính nguyên tử nhỏ hơn.}
\end{ex}
%%%%=================EX03====================%%%
\begin{ex}
	So sánh bán kính của các ion sau: $O^{2-}$, $F^-$, $Na^+$, $Mg^{2+}$
	\choice
	{$Mg^{2+}$ > $Na^+$ > $F^-$ > $O^{2-}$}
	{$O^{2-}$ > $F^-$ > $Na^+$ > $Mg^{2+}$}
	{$Na^+$ > $Mg^{2+}$ > $F^-$ > $O^{2-}$}
    {\True $O^{2-}$ > $F^-$ > $Mg^{2+}$ > $Na^+$}
	\loigiai{Các ion có cùng cấu hình electron (isoelectronic) thì ion nào có số proton ít hơn sẽ có bán kính lớn hơn. $O^{2-}$, $F^-$, $Na^+$, $Mg^{2+}$ đều có cấu hình electron của Ne. Số proton tăng dần từ $O^{2-}$ đến $Mg^{2+}$, nên bán kính giảm dần.}
\end{ex}
%%%%=================EX_04====================%%%
\begin{ex}
	Nguyên tố X có cấu hình electron là [Ne]$3s^23p^4$. Ion $X^{2-}$ có bán kính như thế nào so với nguyên tử trung tính X?
	\choice
	{Nhỏ hơn}
	{\True Lớn hơn}
	{Bằng nhau}
	{Không xác định được}
	\loigiai{Khi nguyên tử X nhận thêm 2 electron để trở thành ion $X^{2-}$, lực hút giữa hạt nhân và lớp vỏ electron ngoài cùng giảm đi, dẫn đến bán kính ion lớn hơn bán kính nguyên tử.}
\end{ex}
%%%%=================EX05====================%%%
\begin{ex}
	Ion nào sau đây có bán kính lớn nhất?
	\choice
	{$K^+$}
	{$Ca^{2+}$}
	{\True $S^{2-}$}
	{$Cl^-$}
	\loigiai{$S^{2-}$ và $Cl^-$ nằm cùng chu kì 3, $S^{2-}$ có điện tích hạt nhân nhỏ hơn nên có bán kính lớn hơn. $K^+$ và $Ca^{2+}$ có số lớp electron ít hơn $S^{2-}$ nên bán kính nhỏ hơn.}
\end{ex}
%%%%=================EX_06====================%%%
\begin{ex}
	Sắp xếp các nguyên tố sau theo chiều tăng dần bán kính nguyên tử: Be, Mg, Ca, Sr
	\choice
	{Sr < Ca < Mg < Be}
	{Be < Mg < Sr < Ca}
	{Ca < Sr < Be < Mg}
    {\True Be < Mg < Ca < Sr}
	\loigiai{Be, Mg, Ca, Sr cùng thuộc nhóm IIA, bán kính nguyên tử tăng dần khi đi xuống một nhóm A trong bảng tuần hoàn do số lớp electron tăng.}
\end{ex}
%%%%=================EX07====================%%%
\begin{ex}
	So sánh bán kính của nguyên tử Na và ion $Na^+$
	\choice
	{Bằng nhau}
	{\True Bán kính Na > Bán kính $Na^+$}
	{Bán kính Na < Bán kính $Na^+$}
	{Không xác định được}
	\loigiai{Ion $Na^+$ được tạo thành khi nguyên tử Na mất đi 1 electron lớp ngoài cùng.  Khi đó, lực hút giữa hạt nhân và lớp vỏ electron ngoài cùng tăng lên, dẫn đến bán kính ion nhỏ hơn bán kính nguyên tử.}
\end{ex}
%%%%=================EX_08====================%%%
\begin{ex}
	Nguyên tố nào sau đây có năng lượng ion hóa thứ nhất lớn nhất?
	\choice
	{Na}
	{Mg}
	{\True Cl}
	{S}
	\loigiai{Năng lượng ion hóa thứ nhất tăng dần từ trái sang phải trong một chu kì. Cl nằm ở vị trí bên phải so với Na, Mg, S trong bảng tuần hoàn nên có năng lượng ion hóa thứ nhất lớn nhất.}
\end{ex}
%%%%=================EX09====================%%%
\begin{ex}
	Trong chu kì 3, nguyên tử của nguyên tố nào sau đây có bán kính nhỏ nhất?
	\choice
	{Na}
	{Al}
	{S}
    {\True Ar}
	\loigiai{Bán kính nguyên tử giảm dần từ trái sang phải trong một chu kì. Ar nằm ở vị trí cuối cùng bên phải của chu kì 3 nên có bán kính nhỏ nhất.}
\end{ex}
%%%%=================EX_10====================%%%
\begin{ex}
	Trong nhóm halogen (nhóm VIIA), bán kính nguyên tử biến đổi như thế nào từ flo (F) đến iot (I)?
	\choice
	{Giảm dần}
	{\True Tăng dần}
	{Không thay đổi}
	{Tăng rồi giảm}
	\loigiai{Bán kính nguyên tử tăng dần khi đi xuống một nhóm A trong bảng tuần hoàn do số lớp electron tăng. Do đó, bán kính nguyên tử tăng dần từ F đến I trong nhóm halogen.}
\end{ex}
%%%%=================EX11====================%%%
\begin{ex}
	So sánh bán kính của cặp ion sau:  $Fe^{2+}$ và $Fe^{3+}$
	\choice
	{Bán kính $Fe^{2+}$ < Bán kính $Fe^{3+}$}
	{\True Bán kính $Fe^{2+}$ > Bán kính $Fe^{3+}$}
	{Bán kính $Fe^{2+}$ = Bán kính $Fe^{3+}$}
	{Không xác định được}
	\loigiai{$Fe^{3+}$ có số electron ít hơn $Fe^{2+}$ nhưng cùng số proton, dẫn đến lực hút giữa hạt nhân và lớp vỏ electron ngoài cùng lớn hơn, nên bán kính $Fe^{3+}$ nhỏ hơn $Fe^{2+}$.}
\end{ex}
%%%%=================EX_12====================%%%
\begin{ex}
	Nguyên tố nào sau đây có ái lực electron lớn nhất (theo giá trị tuyệt đối)?
	\choice
	{Na}
	{O}
	{\True F}
	{S}
	\loigiai{Ái lực electron thường tăng dần từ trái sang phải trong một chu kì. F là nguyên tố có độ âm điện lớn nhất trong bảng tuần hoàn, nên có ái lực electron lớn nhất.}
\end{ex}
%%%%=================EX13====================%%%
\begin{ex}
	Sắp xếp các ion sau theo chiều giảm dần bán kính ion:  $N^{3-}$, $O^{2-}$, $F^-$, $Na^+$, $Mg^{2+}$
	\choice
	{$Mg^{2+}$ > $Na^+$ > $F^-$ > $O^{2-}$ > $N^{3-}$}
	{$N^{3-}$ > $O^{2-}$ > $F^-$ > $Na^+$ > $Mg^{2+}$}
	{$Na^+$ > $Mg^{2+}$ > $F^-$ > $O^{2-}$ > $N^{3-}$}
    {\True $N^{3-}$ > $O^{2-}$ > $F^-$ > $Mg^{2+}$ > $Na^+$}
	\loigiai{Các ion có cùng cấu hình electron (isoelectronic) thì ion nào có số proton ít hơn sẽ có bán kính lớn hơn. $N^{3-}$, $O^{2-}$, $F^-$, $Na^+$, $Mg^{2+}$ đều có cấu hình electron của Ne. Số proton tăng dần từ $N^{3-}$ đến $Mg^{2+}$, nên bán kính giảm dần.}
\end{ex}
%%%%=================EX_14====================%%%
\begin{ex}
	Nguyên tố X thuộc chu kì 3, nhóm VA. So sánh bán kính của nguyên tử X với bán kính của nguyên tử P (cùng nhóm với X)?
	\choice
	{Lớn hơn}
	{\True Nhỏ hơn}
	{Bằng nhau}
	{Không xác định được}
	\loigiai{X và P cùng thuộc nhóm VA, nhưng X nằm ở chu kì 3, còn P nằm ở chu kì 2. Bán kính nguyên tử tăng dần khi đi xuống một nhóm A trong bảng tuần hoàn, nên bán kính nguyên tử X lớn hơn bán kính nguyên tử P.}
\end{ex}
%%%%=================EX15====================%%%
\begin{ex}
	Nguyên tố nào sau đây có độ âm điện lớn nhất?
	\choice
	{K}
	{Al}
	{S}
    {\True Cl}
	\loigiai{Độ âm điện thường tăng dần từ trái sang phải trong một chu kì và giảm dần từ trên xuống dưới trong một nhóm A. Cl nằm ở vị trí trên cùng bên phải trong số các nguyên tố đã cho nên có độ âm điện lớn nhất.}
\end{ex} 
