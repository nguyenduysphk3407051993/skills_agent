\subsubsection{Sự tạo thành ion}
	\begin{figure}[thb]
		\begin{center}
			\subcaptionbox[0.4\linewidth]{Sự tạo thành ion $Na^+$\label{subfig:ionNa}}{\includegraphics[height=3cm]{Images/Tikz/suhinhthanhionNa.pdf}}
			\hspace{2cm}
			\subcaptionbox[0.4\linewidth]{Sự tạo thành ion $O^{2-}$\label{subfig:ionO}}{\includegraphics[height=3cm]{Images/Tikz/suhinhthanhionO.pdf}}
		\end{center}
		\caption{Minh họa quá trình hình thành ion\label{fig:taothanhion}}
	\end{figure}
	\begin{tomtat}
		\begin{itemize}
			\item Khi \indam[\maunhan]{cho electron}, nguyên tử trở thành \indam[\maunhan]{ion dương} (cation).
			\item Khi \indam[\maunhan]{nhận electron}, nguyên tử trở thành \indam[\maunhan]{ion âm} (anion).
			\item Giá trị điện tích trên cation hoặc anion bằng số electron mà nguyên tử đã nhường hoặc nhận.
		\end{itemize}
	\end{tomtat}
	\begin{hoivadap}
		\begin{cauhoi}
			Quan sát Hình \ref{fig:taothanhion}, nhận xét số electron trên lớp vỏ với số proton trong hạt nhân của mỗi ion tạo thành.
		\end{cauhoi}
		\begin{cauhoi}
			Trình bày cách tính điện tích của các ion thu được khi nguyên tử nhường hoặc nhận thêm electron trong Hình \ref{fig:taothanhion}.
		\end{cauhoi}
		\begin{cauhoi}
			Ion $\mathrm{Na}^{+}$và ion $\mathrm{O}^{2-}$ thu được có bền vững vế mặt hoá học không? Chúng có cấu hình electron nguyên tử của nguyên tố nào?
		\end{cauhoi}
		\loigiai{}
	\end{hoivadap}
\subsubsection{Sự tạo thành liên kết ion}
	\begin{figure}[thb]
		\begin{center}
			\includegraphics[height=6cm]{Images/Tikz/suhinhthanhlienketion.pdf}
		\end{center}
		\caption{Minh họa quá trình hình thành liên kết ion trong phân tử NaCl\label{fig:lkionNaCl}}
	\end{figure}
\begin{tomtat}
	\begin{itemize}
		\item Liên kết ion là liên kết được hình thành bởi lực hút tĩnh điện giửa các ion mang điện tích trái dấu.
		\item Liên kết ion thường được hình thành khi kim loại điển hình tác dụng với phi kim điển hình.
	\end{itemize}
\end{tomtat}
	\begin{hoivadap}
		\begin{cauhoi}
			Cho các ion: $\mathrm{Na}^{+}, \mathrm{Mg}^{2+}, \mathrm{O}^{2-}, \mathrm{Cl}^{-}$. Những ion nào có thể kết hợp với nhau tạo thành liên kết ion?
		\end{cauhoi}
		\begin{cauhoi}
			Mô tả sự tạo thành liên kết ion trong:
				\begin{enumerate}[a)]
				\item Calcium oxide.
				\item Magnesium chloride.
				\end{enumerate}
		\end{cauhoi}
		\loigiai{}
	\end{hoivadap}
\subsubsection{Tinh thể ion}
	\Noibat[\maunhan][][\faStar][]{Cấu trúc tinh thể ion}
		\begin{hopdongian}
			Các ion được sắp xếp theo một trật tự nhất định trong không gian theo kiểu mạng lưới, trong đó ở các nút của mạng lưới là những ion dương và ion âm được sắp xếp luân phiên, liên kết chặt chẽ với nhau do sự cân bằng giữa lực hút (các ion trái dấu hút nhau) và lực đẩy (các ion cùng dấu đẩy nhau), tạo thành mạng tinh thể ion.
		\end{hopdongian}
		\begin{figure}[thb]
			\begin{center}
				\subcaptionbox[0.4\linewidth]{Tinh thể NaCl thực tế\label{subfig:img_crytalNaCl}}{\includegraphics[height=5cm]{Images/anhhoahoc10/anhminhoa/Sodium_chloride_crystals.jpg}}
				\hspace{2cm}
				\subcaptionbox[0.4\linewidth]{Ô mạng tinh thể NaCl \label{subfig:crytalNaCl}}{\includegraphics[height=5cm]{Images/Tikz/crytalNaCl.pdf}}
			\end{center}
			\caption{Tinh thể NaCl thực tế và mô hình ô mạng tinh thể NaCl}
		\end{figure}
	\Noibat[\maunhan][][\faStar][]{Độ bền và tính chất hợp chất ion}
	\vspace{0.25cm}
	\begin{tomtat}
		\begin{itemize}
			\item Trong tinh thể ion, giữa các ion có lực hút tĩnh điện rất mạnh nên các hợp chất ion thường là chất rắn, khó nóng chảy, khó bay hơi ở điều kiện thường.
			\item Do lực hút tĩnh điện rất mạnh giữa các ion nên các tinh thể ion khá rắn chắc, nhưng khá giòn.
		\end{itemize}
	\end{tomtat}
	\begin{hoivadap}
		\begin{cauhoi}
			Hãy trả lời các câu hỏi sau:
			\begin{enumerate}[a)]
				\item Vì sao muối ăn có nhiệt độ nóng chảy cao $\left(801^{\circ} \mathrm{C}\right)$ ?
				\item Hợp chất ion dẫn điện trong trường hợp nào? Vì sao?
			\end{enumerate}
		\end{cauhoi}
		\loigiai{}
	\end{hoivadap}
	