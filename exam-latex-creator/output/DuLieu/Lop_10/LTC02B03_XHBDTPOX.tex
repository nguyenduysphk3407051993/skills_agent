\subsubsection{Thành phần của các oxide và hydroxide}
\vspace{0.25cm}
\begin{tomtat}
	\begin{tabular}{*{8}{|c}|}
		\hline\rowcolor{\maunhan!20}
		\textbf{Nhóm}&IA&IIA&IIIA&IVA&VA&VIA&VIIA\\
		\hline \rowcolor{\maudam!10}
		\textbf{Oxit cao nhat}&$R_2O$&$RO$&$R_2O_3$&$RO_2$&$R_2O_5$&$RO_3$&$R_2O_7$\\
		\hline \rowcolor{\mauphu!20}
		\textbf{Hợp chất hiđroxit}&$ROH$&$R(OH)_2$&$R(OH)_3$&$H_2RO_3$&\makecell[c]{$H_3RO_4$\\(ngoại lệ $HNO_3$)}&$H_2RO_4$&$HRO_4$\\
		\hline
	\end{tabular}
\end{tomtat}
\begin{hoivadap}
	\begin{cauhoi}
		Nguyên tố Aluminium thuộc nhóm IIIA và nguyên tố sulfur thuộc nhóm VIA của bảng tuần hoàn. Viết công thức hoá học của oxide, hydroxide (ứng với hoá trị cao nhất) của hai nguyên tố trên.
	\end{cauhoi}
\end{hoivadap}
\subsubsection{Tính chất của oxide và hydroxide}
\Noibat[][][]{Tác dụng với nước}\\
Các oxide khi tác dụng với nước tạo thành hydroxide có tính base hoặc acid.
\begin{vidu}
	\begin{itemize}
		\item $Na_2O + H_2O \xrightarrow 2NaOH$
		\item $P_2O_5 + 3H_2O \xrightarrow 2H_3PO_4$
	\end{itemize}
\end{vidu}
\Noibat[][][]{Phản ứng của muối với dung dịch axit}\\
Các dung dịch axit mạnh tác dụng với dung dịch muối của axit yếu hơn
\begin{vidu}
	\begin{itemize}
		\item $Na_2CO_3 + 2HNO_3 \xrightarrow 2NaNO_3 + CO_2 + H_2O$
	\end{itemize}
\end{vidu}
\subsubsection{Xu hướng biến đổi}
\vspace{0.25 cm}
\begin{tomtat}
	Trong một chu kì, theo chiều tăng dần của điện tích hạt nhân, tính base của oxide và hydroxide tương ứng giảm dần, đồng thời tính acid của chúng tăng dần.
\end{tomtat}