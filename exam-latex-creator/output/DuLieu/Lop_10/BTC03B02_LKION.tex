\phan{Trắc nghiệm nhiều lựa chọn}
%%%=============SOẠN EX===============%%%
\Opensolutionfile{ansex}[Ans/LGEX-C03_B02_LKION.tex]
\Opensolutionfile{ans}[Ans/Ans-C03_B02_LKION.tex]
%%%==============Cau_1==============%%%
\begin{ex}
	Liên kết ion được tạo thành giữa hai nguyên tử bằng
	\choice
	{một hay nhiều cặp electron dùng chung}
	{một hay nhiều cặp electron dùng chung chỉ do một nguyên tử đóng góp}
	{\True lực hút tĩnh điện giữa các ion mang điện tích trái dấu}
	{một hay nhiều cặp electron dùng chung và các cặp electron này lệch về nguyên tử có độ âm điện lớn hơn}
	\loigiai{}
\end{ex}
%%%==============HetCau_1==============%%%

%%%==============Cau_2==============%%%
\begin{ex}
	Liên kết ion là loại liên kết hoá học được hình thành nhờ lực hút tĩnh điện giữa các phần tử nào sau đây?
	\choice
	{\True cation và anion}
	{các anion}
	{cation và electron tự do}
	{electron và hạt nhân nguyên tử}
	\loigiai{}
\end{ex}
%%%==============HetCau_2==============%%%

%%%==============Cau_3==============%%%
\begin{ex}
	Biểu diễn sự tạo thành ion nào sau đây đúng?
	\choice
	{$\mathrm{Na}+\mathrm{le} \to \mathrm{Na}^{+}$}
	{$\mathrm{Cl}_2\to 2\mathrm{Cl}^{-}+2e$}
	{$O_2+2e \to 2O^{2-}$}
	{\True $\mathrm{Al} \to \mathrm{Al}^{3+}+3e$}
	\loigiai{}
\end{ex}
%%%==============HetCau_3==============%%%

%%%==============Cau_4==============%%%
\begin{ex}
	Số electron và số proton trong ion $NH_4^{+}$ là
	\choice
	{11 electron và 11 proton}
	{\True 10 electron và 11 proton}
	{11 electron và 10 proton}
	{11 electron và 12 proton}
	\loigiai{}
\end{ex}
%%%==============HetCau_4==============%%%

%%%==============Cau_5==============%%%
\begin{ex}
	Cặp nguyên tử nào sau đây không tạo hợp chất dạng $X_2^{+} Y^{2-}$ hoặc $X^{2+} Y_2^{-}$?
	\choice
	{Na và O}
	{K và S}
	{\True Ca và O}
	{Ca và Cl}
	\loigiai{}
\end{ex}
%%%==============HetCau_5==============%%%

%%%==============Cau_6==============%%%
\begin{ex}
	Tính chất nào sau đây là tính chất của hợp chất ion?
	\choice
	{Hợp chất ion có nhiệt độ nóng chảy thấp}
	{\True Hợp chất ion có nhiệt độ nóng chảy cao}
	{Hợp chất ion dễ hoá lỏng}
	{Hợp chất ion có nhiệt độ sôi không xác định.}
	\loigiai{}
	\end{ex}
	
	%%%==============Cau_7==============%%%
	\begin{ex}
		Cho các phân tử sau: $\mathrm{HCl}, \mathrm{NaCl}, \mathrm{CaCl}_2, \mathrm{AlCl}_3$. Phân tử có liên kết mang nhiều tính chất ion nhất là
		\choice
		{HCl}
		{\True NaCl}
		{$\mathrm{CaCl}_2$}
		{$\mathrm{AlCl}_3$}
		\loigiai{}
	\end{ex}
	
	%%%==============Cau_8==============%%%
	\begin{ex}
		Dãy gồm các phân tử đều có liên kết ion là
		\choice
		{$\mathrm{Cl}_2, \mathrm{Br}_2, I_2, \mathrm{HCl}$}
		{$\mathrm{HCl}, H_2\mathrm{~S}, \mathrm{NaCl}, N_2O$}
		{\True $\mathrm{Na}_2O, \mathrm{KCl}, \mathrm{BaCl}_2, \mathrm{Al}_2O_3$}
		{$\mathrm{MgO}, H_2SO_4, H_3PO_4, \mathrm{HCl}$}
		\loigiai{}
	\end{ex}

	%%%=============EX_9=============%%%
	\begin{ex}
		Liên kết ion là liên kết được tạo thành bằng
		\choice
		{lực hút tĩnh điện giữa các electron tự do với ion dương kim loại}
		{cặp electron chung giữa hai nguyên tử}
		{\True lực hút tĩnh điện giữa các ion mang điện trái dấu}
		{cặp electron chung chỉ do một nguyên tử đóng góp}
		\loigiai{}
	\end{ex}
	
	%%%=============EX_10=============%%%
	\begin{ex}
		Ion dương được hình thành khi nguyên tử
		\choice
		{\True nhường electron}
		{nhận electron}
		{nhường proton}
		{nhận proton}
		\loigiai{}
	\end{ex}
	
	%%%=============EX_11=============%%%
	\begin{ex}
		Khi hình thành liên kết hóa học, nguyên tử $\mathrm{Na}(Z=11)$ có xu hướng nhường electron tạo thành ion
		\choice
		{\True $\mathrm{Na}^{+}$}
		{$\mathrm{Na}^{2+}$}
		{$\mathrm{Na}^{-}$}
		{$\mathrm{Na}^{2-}$}
		\loigiai{}
	\end{ex}
	
	%%%=============EX_12=============%%%
	\begin{ex}
		Phân tử nào sau đây có liên kết ion?
		\choice
		{$NH_3$}
		{$H_2\mathrm{~S}$}
		{HCl}
		{\True NaBr}
		\loigiai{}
	\end{ex}
	
	%%%=============EX_13=============%%%
	\begin{ex}
		Liên kết trong phân tử nào sau đây là liên kết ion?
		\choice
		{\True CaO}
		{$NH_3$}
		{$\mathrm{Cl}_2O_5$}
		{$\mathrm{Br}_2O_7$}
		\loigiai{}
	\end{ex}
	
	%%%=============EX_14=============%%%
	\begin{ex}
		Biết potasium có $Z=19$, trong phân tử $K_2O$, mỗi ion $K^{+}$ có số electron là
		\choice
		{\True $18$}
		{$19$}
		{$20$}
		{$10$}
		\loigiai{}
	\end{ex}
	
	%%%=============EX_15=============%%%
	\begin{ex}
		Ion nào sau đây thuộc loại ion đa nguyên tử?
		\choice
		{\True $NH_4^{+}$}
		{$\mathrm{Na}^{+}$}
		{$\mathrm{Ca}^{2+}$}
		{$\mathrm{Cl}^{-}$}
		\loigiai{}
	\end{ex}
	
	%%%=============EX_16=============%%%
	\begin{ex}
		Phân tử nào sau đây có chứa ion đa nguyên tử?
		\choice
		{$K_2\mathrm{~S}$}
		{\True $NH_4\mathrm{Cl}$}
		{$\mathrm{AlBr}_3$}
		{ZnO}
		\loigiai{}
	\end{ex}
	
	%%%=============EX_17=============%%%
	\begin{ex}
		Oxide của nguyên tố nào sau đây có liên kết ion?
		\choice
		{Nitrogen}
		{Carbon}
		{Sulfur}
		{\True Calcium}
		\loigiai{}
	\end{ex}
	
	%%%=============EX_18=============%%%
	\begin{ex}
		Phân tử nào dưới đây chỉ chứa ion đơn nguyên tử?
		\choice
		{$K_2SO_4$}
		{$NH_4\mathrm{Cl}$}
		{\True NaCl}
		{$\mathrm{Zn}\left(NO_3\right)_2$}
		\loigiai{}
	\end{ex}
	
	%%%=============EX_19=============%%%
	\begin{ex}
		Hợp chất nào sau đây có liên kết ion?
		\choice
		{$H_2\mathrm{~S}$}
		{$H_2O$}
		{\True $\mathrm{MgCl}_2$}
		{$CO_2$}
		\loigiai{}
	\end{ex}
	
	%%%=============EX_20=============%%%
	\begin{ex}
		Quá trình tạo thành ion nào sau đây được viết đúng?
		\choice
		{$K+1e \to K^{+}$}
		{$\mathrm{Cl}_2\to 2\mathrm{Cl}^{-}+2e$}
		{$O_2+2e \to 2O^{2-}$}
		{\True $\mathrm{Al} \to \mathrm{Al}^{3+}+3e$}
		\loigiai{}
	\end{ex}
	
	%%%=============EX_21=============%%%
	\begin{ex}
		Ion nào sau đây là ion đa nguyên tử?
		\choice
		{$\mathrm{Na}^{+}$}
		{\True $NO_3^{-}$}
		{$\mathrm{Cl}^{-}$}
		{$O^{2-}$}
		\loigiai{}
	\end{ex}
	
	%%%=============EX_22=============%%%
	\begin{ex}
		Sodium chloride là một hợp chất có thể tan trong nước lạnh và có nhiệt độ nóng chảy cao. Liên kết trong phân tử sodium chloride là liên kết
		\choice
		{cộng hóa trị không phân cực}
		{\True liên kết ion}
		{hydrogen}
		{cộng hóa trị phân cực}
		\loigiai{}
	\end{ex}
	
	%%%=============EX_23=============%%%
	\begin{ex}
		Tính chất nào sau đây không đúng với hợp chất ion?
		\choice
		{Nhiệt độ nóng chảy cao}
		{Tan tốt trong nước}
		{\True Không dẫn điện}
		{Rắn chắc nhưng khá giòn}
		\loigiai{}
	\end{ex}
	
	%%%=============EX_24=============%%%
	\begin{ex}
		Tính chất nào sau đây là tính chất của hợp chất ion?
		\choice
		{Nhiệt độ nóng chảy thấp}
		{\True Tan nhiều trong nước}
		{Dễ bay hơi}
		{Nhiệt độ sôi thấp}
		\loigiai{}
	\end{ex}
	
	%%%=============EX_25=============%%%
	\begin{ex}
		Trong các hợp chất sau: $\mathrm{CaO}, \mathrm{Ba}\left(NO_3\right)_2, \mathrm{Na}_2O, KF, K_2SO_4, NH_4\mathrm{Cl}$, số hợp chất chứa ion đa nguyên tử là
		\choice
		{$2$}
		{\True $3$}
		{$4$}
		{$5$}
		\loigiai{}
	\end{ex}
	
	%%%=============EX_26=============%%%
	\begin{ex}
		Liên kết trong phân tử muối clorua của kim loại kiềm nào sau đây mang nhiều đặc tính ion nhất?
		\choice
		{CsCl}
		{LiCl}
		{\True KCl}
		{RbCl}
		\loigiai{}
	\end{ex}
	
	%%%=============EX_27=============%%%
	\begin{ex}
		Sự kết hợp của các nguyên tử nào sau đây không tạo hợp chất dạng $X_2^{+} Y^{2-}$ hoặc $X^{2+} Y_2^{-2}$?
		\choice
		{$_{11} \mathrm{Na}$ và $_8O$}
		{$_{12} \mathrm{Mg}$ và $_{17} \mathrm{Cl}$}
		{\True $_{20} \mathrm{Ca}$ và $_8O$}
		{$_{20} \mathrm{Ca}$ và $_{17} \mathrm{Cl}$}
		\loigiai{}
	\end{ex}
	
	%%%=============EX_28=============%%%
	\begin{ex}
		Cặp nguyên tử có cấu hình electron nào sau đây có thể tạo liên kết ion?
		\choice
		{$1s^22s^22p^3$ và $1s^22s^22p^5$}
		{\True $1s^22s^22p^63s^1$ và $1s^22s^22p^5$}
		{$1s^22s^22p^3$ và $1s^22s^22p^4$}
		{$1s^2$ và $1s^22s^22p^4$}
		\loigiai{}
	\end{ex}
	
	%%%=============EX_29=============%%%
	\begin{ex}
		Nguyên tử $X$ có 12 electron, nguyên tử $Y$ có 17 electron. Công thức hợp chất và loại liên kết hình thành giữa hai nguyên tử này là
		\choice
		{\True $XY_2$, liên kết ion}
		{$X_3Y_2$, liên kết cộng hóa trị}
		{$X_2Y$, liên kết cộng hóa trị}
		{XY, liên kết ion}
		\loigiai{}
	\end{ex}
	
	%%%=============EX_30=============%%%
	\begin{ex}
		Cho cấu hình electron nguyên tử của các nguyên tố sau: $X(1s^22s^22p^63s^23p^64s^1), Y(1s^22s^22p^4)$. Hợp chất ion được tạo thành từ X và Y có công thức là
		\choice
		{XY}
		{$X_2Y_3$}
		{$XY_2$}
		{\True $X_2Y$}
		\loigiai{}
	\end{ex}
\Closesolutionfile{ans}
\Closesolutionfile{ansex}
%\bangdapan{Ans-C03_B02_LKION.tex}
\phan{Trắc nghiệm đúng sai}
%%%=============SOẠN EXTF===============%%%
\Opensolutionfile{ansex}[Ans/LGTF-C03_B02_LKION.tex]
\Opensolutionfile{ansbook}[Ansbook/AnsTF-C03_B02_LKION.tex]
\Opensolutionfile{ans}[Ans/Tempt-C03_B02_LKION.tex]
%%%=============EX_1=============%%%
\begin{ex}
	Liên kết ion được hình thành do lực hút tĩnh điện giữa các ion mang điện tích trái dấu.
	\choiceTF
	{\True Liên kết ion được hình thành do lực hút tĩnh điện giữa các ion mang điện tích trái dấu.}
	{Liên kết ion được hình thành do lực hút tĩnh điện giữa các ion mang điện tích cùng dấu.}
	{\True Liên kết ion chỉ được hình thành giữa kim loại điển hình và phi kim điển hình.}
	{Liên kết ion có tính định hướng.}
	\loigiai{
		\begin{itemchoice}[T1,F2,T3,F4]
			\itemch Lực hút tĩnh điện giữa các ion mang điện tích trái dấu là bản chất của liên kết ion.
			\itemch Các ion mang điện tích cùng dấu sẽ đẩy nhau.
			\itemch Kim loại điển hình dễ nhường electron tạo cation, phi kim điển hình dễ nhận electron tạo anion, tạo điều kiện hình thành liên kết ion.
			\itemch Liên kết ion không có tính định hướng, lực hút tĩnh điện giữa các ion trái dấu không theo một hướng nhất định.
		\end{itemchoice}
	}
\end{ex}
%%%=============EX_2=============%%%
\begin{ex}
	Về cấu trúc và tính chất của hợp chất ion, cho biết tính đúng/ sai các phát biểu sau?
	\choiceTF
	{Các ion trong hợp chất ion sắp xếp tự do, không theo trật tự nào.}
	{\True Các ion trong hợp chất ion sắp xếp theo một trật tự nhất định, tạo thành mạng tinh thể ion.}
	{Mạng tinh thể ion không bền vững.}
	{\True Hợp chất ion có nhiệt độ nóng chảy và nhiệt độ sôi cao.}
	\loigiai{
		\begin{itemchoice}[F1,T2,F3,T4]
			\itemch Các ion trong hợp chất ion được sắp xếp theo trật tự nhất định trong mạng tinh thể.
			\itemch  Các ion trong hợp chất ion sắp xếp có trật tự, lực hút tĩnh điện mạnh, tạo thành mạng tinh thể ion.
			\itemch Mạng tinh thể ion rất bền vững do lực hút tĩnh điện mạnh giữa các ion.
			\itemch Do mạng tinh thể ion bền vững nên hợp chất ion có nhiệt độ nóng chảy và nhiệt độ sôi cao.
		\end{itemchoice}
	}
\end{ex}
%%%=============EX_3=============%%%
\begin{ex}
	Phân tích các phát biểu sau về hợp chất natri clorua ($NaCl$)
	\choiceTF
	{\True $NaCl$ được tạo thành từ kim loại điển hình $Na$ và phi kim điển hình $Cl$.}
	{Liên kết trong $NaCl$ là liên kết cộng hóa trị.}
	{\True $Na$ dễ nhường 1 electron tạo $Na^+$, $Cl$ dễ nhận 1 electron tạo $Cl^-$.}
	{Phân tử $NaCl$ ở trạng thái rắn, lỏng, khí đều dẫn điện.}
	\loigiai{
		\begin{itemchoice}[T1,F2,T3,F4]
			\itemch $NaCl$ được tạo từ kim loại điển hình $Na$ và phi kim điển hình $Cl$, nên có liên kết ion.
			\itemch Liên kết trong $NaCl$ là liên kết ion.
			\itemch $Na$ dễ nhường 1e tạo $Na^+$, $Cl$ dễ nhận 1e tạo $Cl^-$, tạo ra lực hút tĩnh điện hình thành liên kết ion.
			\itemch $NaCl$ chỉ dẫn điện ở trạng thái nóng chảy và dung dịch.
		\end{itemchoice}
	}
\end{ex}
%%%=============EX_4=============%%%
\begin{ex}
	Về tính chất của liên kết ion và hợp chất ion
	\choiceTF
	{\True Độ bền của liên kết ion tỉ lệ thuận với hiệu độ âm điện giữa các nguyên tố tham gia liên kết.}
	{Liên kết ion thể hiện tính định hướng trong không gian}
	{Tất cả các hợp chất ion đều tan tốt trong nước.}
	{\True Hợp chất ion không dẫn điện ở trạng thái rắn.}
	\loigiai{
		\begin{itemchoice}[T1,F2,T3,T4]
			\itemch Sự chênh lệch độ âm điện lớn dẫn đến sự hình thành ion dễ dàng hơn, liên kết ion bền vững hơn.
			\itemch Liên kết ion là lực hút tĩnh điện giữa các ion trái dấu.Lực này tác dụng theo mọi hướng trong không gian (đẳng hướng)
			\itemch Có một số hợp chất ion khó tan trong nước như: AgCl, BaSO$_4$, PbCl$_2$.
			\itemch  Ở trạng thái rắn, các ion bị cố định trong mạng tinh thể, không di chuyển được nên hợp chất ion không dẫn điện.
		\end{itemchoice}
	}
\end{ex}
%%%=============EX_5=============%%%
\begin{ex}
	Về hợp chất ion.
	\choiceTF
	{\True  $KCl$ tan tốt trong nước do nước là dung môi phân cực mạnh.}
	{Các hợp chất ion đều tan tốt trong nước.}
	{\True Liên kết ion trong $KCl$ bị phá vỡ khi hòa tan vào nước.}
	{\True Dung dịch $KCl$ dẫn điện.}
	\loigiai{
		\begin{itemchoice}[T1,F2,T3,T4]
			\itemch Nước là dung môi phân cực mạnh, có khả năng hydrat hóa các ion.
			\itemch Không phải tất cả các hợp chất ion đều tan tốt trong nước.
			\itemch Khi hòa tan vào nước, nước sẽ hydrat hóa các ion, phá vỡ liên kết ion.
			\itemch Dung dịch $KCl$ dẫn điện do có các ion tự do di chuyển.
		\end{itemchoice}
	}
\end{ex}
%%%=============EX_6=============%%%
\begin{ex}
	Nghiên cứu tính chất vật lí của canxi clorua ($CaCl_2$)
	\choiceTF
	{$CaCl_2$ là hợp chất phân tử nên dễ bay hơi ở nhiệt độ thường.}
	{\True Để nóng chảy $CaCl_2$ cần nhiệt độ rất cao do lực hút tĩnh điện mạnh giữa các ion trong mạng tinh thể.}
	{Tinh thể $CaCl_2$ có khả năng dẫn điện do chứa các ion $Ca^{2+}$ và $Cl^-$.}
	{\True $CaCl_2$ nóng chảy dẫn điện tốt vì các ion được giải phóng khỏi mạng tinh thể và chuyển động tự do.}
	\loigiai{
		\begin{itemchoice}[F1,T2,F3,T4]
			\itemch Về bản chất của $CaCl_2$ là hợp chất ion, không phải hợp chất phân tử
			\itemch Về nhiệt độ nóng chảy của $CaCl_2$ cao do lực hút tĩnh điện mạnh giữa ion $Ca^{2+}$ và $Cl^-$ và cấu trúc mạng tinh thể ion bền vững
			\itemch Mặc dù chứa các ion nhưng tinh thể $CaCl_2$ không dẫn điện. Nguyên nhân: các ion được sắp xếp có trật tự và cố định trong mạng tinh thể do đó các ion không thể di chuyển tự do nên không thể dẫn điện
			\itemch  Về khả năng dẫn điện của $CaCl_2$ nóng chảy: Khi nóng chảy, mạng tinh thể ion bị phá vỡ.Các ion $Ca^{2+}$ và $Cl^-$ được giải phóng.Các ion có thể di chuyển tự do do đó, $CaCl_2$ nóng chảy dẫn điện tốt
		\end{itemchoice}
	}
\end{ex}
%%%=============EX_7=============%%%
\begin{ex}
	Một nguyên tố X có cấu hình electron lớp ngoài cùng là $3s^23p^5$.
	\choiceTF
	{\True X là phi kim và có xu hướng nhận 1 electron}
	{Hợp chất của X với Na có công thức $Na_2X$}
	{Ion $X^-$ có cấu hình electron của khí hiếm Kr}
	{\True X có thể tạo liên kết cộng hóa trị với H theo tỉ lệ $1:1$}
	\loigiai{%
		\begin{itemchoice}[T1,F2,F3,T4]
			\itemch X có 7e lớp ngoài cùng (2e ở 3s và 5e ở 3p) nên là phi kim mạnh, có xu hướng nhận 1e để đạt cấu hình electron khí hiếm.
			\itemch Na cho 1e, X nhận 1e nên tỉ lệ $Na:X = 1:1$, công thức là NaX.
			\itemch $X^-$: $[Ne]3s^23p^6 = [Ar]$, không phải Kr.
			\itemch X có thể dùng chung 1e với H tạo liên kết cộng hóa trị, tạo thành phân tử HX.
		\end{itemchoice}
	}
\end{ex}
%%%=============EX_8=============%%%
\begin{ex}
	Cho các phát biểu sau về liên kết ion:
	\choiceTF
	{\True Hình thành do lực hút tĩnh điện giữa các ion trái dấu.}
	{Chỉ tồn tại giữa kim loại và phi kim.}
	{\True Hợp chất ion thường là chất rắn ở điều kiện thường.}
	{Hợp chất ion luôn dẫn điện tốt ở trạng thái rắn.}
	\loigiai{%
		\begin{itemchoice}[T1,F2,T3,F4]
			\itemch Liên kết ion hình thành do lực hút tĩnh điện giữa các ion trái dấu.
			\itemch Liên kết ion có thể tồn tại giữa kim loại và phi kim, hoặc giữa ion kim loại và nhóm phi kim (ví dụ $NH_4^+$, $SO_4^{2-}$).
			\itemch Hợp chất ion thường là chất rắn ở điều kiện thường.
			\itemch Hợp chất ion dẫn điện tốt ở trạng thái nóng chảy và dung dịch.
		\end{itemchoice}
	}
\end{ex}
%%%=============EX_9=============%%%
\begin{ex}
	Cho phản ứng: $2\text{ Na }+\text{ Cl }_2\to 2\text{ NaCl }$. Trong hợp chất $\text{ NaCl }$:
	\choiceTF
	{\True Ion $\text{ Na }^+$ có cấu hình electron của $\text{ Ne }$}
	{$\text{ Na }$ nhận thêm 1 electron từ $\text{ Cl }$}
	{Ion $\text{ Cl }^-$ có cấu hình electron của $\text{ Kr }$}
	{\True Liên kết được hình thành do lực hút tĩnh điện}
	\loigiai{
		\begin{itemchoice}[T1,F2,F3,T4]
			\itemch $\text{ Na }$ ($Z=11$): $[\text{ Ne }]3s^1\to \text{ Na }^+: [\text{ Ne }]$, có cấu hình electron của $\text{ Ne }$.
			\itemch $\text{ Na }$ cho 1e (không phải nhận) để tạo thành ion $\text{ Na }^+$.
			\itemch Ion $\text{ Cl }^-$ có cấu hình của $\text{ Ar }$ (không phải $\text{ Kr }$).
			\itemch Liên kết ion trong $\text{ NaCl }$ là lực hút tĩnh điện giữa $\text{ Na }^+$ và $\text{ Cl }^-$.
		\end{itemchoice}
	}
\end{ex}
%%%=============EX_10=============%%%
\begin{ex}
	Cho nguyên tử M có cấu hình electron $[\text{ Ne }]3s^2$. Hợp chất của M với ion $\text{ SO }_4^{2-}$:
	\choiceTF
	{Có công thức $\text{ MSO }_4$}
	{\True Có công thức $\text{ M }_2\text{ SO }_4$}
	{\True M là kim loại kiềm thổ}
	{\True M cho 2 electron tạo ion $\text{ M }^{2+}$}
	\loigiai{
		\begin{itemchoice}[F1,T2,T3,T4]
			\itemch $\text{ MSO }_4$ sai vì không cân bằng điện tích ($\text{ M }^{2+}$ và $\text{ SO }_4^{2-}$).
			\itemch $\text{ M }_2\text{ SO }_4$ đúng vì $2(\text{ M }^{2+})$ cân bằng điện tích với $\text{ SO }_4^{2-}$.
			\itemch $[\text{ Ne }]3s^2$ là cấu hình của kim loại kiềm thổ (nhóm IIA).
			\itemch M có 2e hóa trị nên cho 2e để tạo ion $\text{ M }^{2+}$.
		\end{itemchoice}
	}
\end{ex}
%%%=============EX_11=============%%%
\begin{ex}
	Cho phản ứng: $2\text{K} + \text{Br}_2 \rightarrow 2\text{KBr}$. Xét quá trình tạo liên kết:
	\choiceTF
	{\True $\text{K}$ có năng lượng ion hóa nhỏ hơn $\text{Na}$}
	{Ion $\text{K}^+$ và $\text{Br}^-$ có cùng kích thước}
	{\True $\text{Br}$ có ái lực electron lớn hơn $\text{Cl}$}
	{\True Liên kết trong $\text{KBr}$ yếu hơn trong $\text{NaCl}$}
	\loigiai{
		\begin{itemchoice}[T1,F2,T3,T4]
			\itemch K ở chu kỳ lớn hơn nên năng lượng ion hóa nhỏ hơn Na.
			\itemch $\text{K}^+$ có kích thước nhỏ hơn $\text{Br}^-$ do mất 1e.
			\itemch Br ở chu kỳ lớn hơn nên có ái lực electron lớn hơn Cl.
			\itemch Do khoảng cách giữa các ion lớn hơn nên lực hút tĩnh điện yếu hơn.
		\end{itemchoice}
	}
\end{ex}
%%%=============EX_12=============%%%
\begin{ex}
	Cho hai hợp chất $\text{MgO}$ và $\text{CaO}$. Xét các nhận định:
	\choiceTF
	{\True Cả hai đều là hợp chất ion}
	{$\text{MgO}$ bền hơn $\text{CaO}$}
	{\True $\text{Mg}^{2+}$ nhỏ hơn $\text{Ca}^{2+}$}
	{\True Cả hai đều có nhiệt độ nóng chảy cao}
	\loigiai{
		\begin{itemchoice}[T1,F2,T3,T4]
			\itemch $\text{Mg}^{2+}$, $\text{Ca}^{2+}$ và $\text{O}^{2-}$ tạo liên kết ion.
			\itemch $\text{CaO}$ bền hơn do $\text{Ca}^{2+}$ có năng lượng ion hóa nhỏ hơn.
			\itemch $\text{Mg}^{2+}$ ở chu kỳ 3, $\text{Ca}^{2+}$ ở chu kỳ 4.
			\itemch Liên kết ion mạnh nên nhiệt độ nóng chảy cao.
		\end{itemchoice}
	}
\end{ex}
%%%=============EX_13=============%%%
\begin{ex}
	Ion $\text{Al}^{3+}$ trong hợp chất với ion $\text{PO}_4^{3-}$:
	\choiceTF
	{Tạo thành hợp chất $\text{AlPO}_4$}
	{\True Có cấu hình electron của $\text{Ne}$}
	{\True Tạo thành hợp chất $\text{AlPO}_4$ không tan trong nước}
	{\True Tạo liên kết ion với $\text{PO}_4^{3-}$}
	\loigiai{
		\begin{itemchoice}[F1,T2,T3,T4]
			\itemch $\text{AlPO}_4$ đúng vì điện tích cân bằng, $\text{Al}^{3+}:\text{PO}_4^{3-} = 1:1$.
			\itemch $\text{Al}^{3+}$ có cấu hình electron $1s^22s^22p^6$ giống $\text{Ne}$.
			\itemch $\text{AlPO}_4$ là hợp chất ion có độ tan thấp trong nước.
			\itemch Liên kết giữa $\text{Al}^{3+}$ và $\text{PO}_4^{3-}$ là liên kết ion.
		\end{itemchoice}
	}
\end{ex}
%%%=============EX_14=============%%%
\begin{ex}
	Cho hai hợp chất $\text{LiF}$ và $\text{CsF}$. So sánh:
	\choiceTF
	{\True $\text{LiF}$ có tính ion mạnh hơn}
	{Ion $\text{Li}^+$ lớn hơn ion $\text{Cs}^+$}
	{\True $\text{LiF}$ có nhiệt độ nóng chảy cao hơn}
	{\True $\text{F}^-$ đều có cấu hình của $\text{Ne}$}
	\loigiai{
		\begin{itemchoice}[T1,F2,T3,T4]
			\itemch LiF có độ chênh lệch điện tích lớn hơn nên tính ion mạnh hơn.
			\itemch $\text{Li}^+$ (chu kỳ 2) nhỏ hơn $\text{Cs}^+$ (chu kỳ 6).
			\itemch Lực hút tĩnh điện mạnh hơn nên nhiệt độ nóng chảy cao hơn.
			\itemch $\text{F}^-$ có cấu hình electron $1s^22s^22p^6$ giống $\text{Ne}$.
		\end{itemchoice}
	}
\end{ex}
%%%=============EX_15=============%%%
\begin{ex}
	Cho dung dịch chứa $\text{Ba}^{2+}$ tác dụng với dung dịch chứa $\text{SO}_4^{2-}$:
	\choiceTF
	{\True Tạo thành kết tủa $\text{BaSO}_4$}
	{Kết tủa có màu}
	{\True $\text{BaSO}_4$ không tan trong nước}
	{\True Liên kết trong $\text{BaSO}_4$ là liên kết ion}
	\loigiai{
		\begin{itemchoice}[T1,F2,T3,T4]
			\itemch $\text{Ba}^{2+}$ kết hợp với $\text{SO}_4^{2-}$ tạo kết tủa $\text{BaSO}_4$.
			\itemch $\text{BaSO}_4$ là kết tủa màu trắng.
			\itemch $\text{BaSO}_4$ có độ tan rất nhỏ trong nước.
			\itemch Liên kết giữa $\text{Ba}^{2+}$ và $\text{SO}_4^{2-}$ là liên kết ion.
		\end{itemchoice}
	}
\end{ex}
%%%=============EX_16=============%%%
\begin{ex}
	Các ion $\text{Na}^+$, $\text{K}^+$, $\text{Mg}^{2+}$, $\text{Ca}^{2+}$ trong nước:
	\choiceTF
	{\True Đều được hình thành từ kim loại}
	{Có cùng kích thước với nhau}
	{\True Đều tạo được hợp chất với $\text{Cl}^-$}
	{\True Đều có cấu hình electron của khí hiếm}
	\loigiai{
		\begin{itemchoice}[T1,F2,T3,T4]
			\itemch Đều là ion của các kim loại kiềm và kiềm thổ.
			\itemch Kích thước khác nhau do khác chu kỳ và điện tích.
			\itemch Tạo thành $\text{NaCl}$, $\text{KCl}$, $\text{MgCl}_2$, $\text{CaCl}_2$.
			\itemch $\text{Na}^+$, $\text{K}^+$ có cấu hình $\text{Ne}$, $\text{Ar}$; $\text{Mg}^{2+}$, $\text{Ca}^{2+}$ có cấu hình $\text{Ne}$, $\text{Ar}$.
		\end{itemchoice}
	}
\end{ex}
%%%=============EX_17=============%%%
\begin{ex}
	Xét các tính chất của hợp chất $\text{CaCl}_2$:
	\choiceTF
	{\True Hút ẩm mạnh}
	{Không tan trong nước}
	{\True Tinh thể có màu trắng}
	{\True Dẫn điện khi nóng chảy}
	\loigiai{
		\begin{itemchoice}[T1,F2,T3,T4]
			\itemch $\text{CaCl}_2$ là chất hút ẩm mạnh, được dùng làm chất hút ẩm.
			\itemch $\text{CaCl}_2$ tan tốt trong nước.
			\itemch $\text{CaCl}_2$ là tinh thể ion màu trắng.
			\itemch Khi nóng chảy, các ion tự do chuyển động nên dẫn điện.
		\end{itemchoice}
	}
\end{ex}
%%%=============EX_18=============%%%
\begin{ex}
	Trong phản ứng tạo $\text{MgCl}_2$:
	\choiceTF
	{\True $\text{Mg}$ mất 2 electron thành $\text{Mg}^{2+}$}
	{$\text{Cl}_2$ nhận 1 electron thành $2\text{Cl}^-$}
	{\True Liên kết ion được tạo thành sau khi trao đổi electron}
	{\True Ion $\text{Mg}^{2+}$ có kích thước nhỏ hơn nguyên tử $\text{Mg}$}
	\loigiai{
		\begin{itemchoice}[T1,F2,T3,T4]
			\itemch $\text{Mg}$ mất 2e để đạt cấu hình electron của $\text{Ne}$.
			\itemch Mỗi nguyên tử $\text{Cl}$ nhận 1e (không phải $\text{Cl}_2$ nhận 1e).
			\itemch Sau khi trao đổi electron, các ion tạo liên kết ion.
			\itemch Do mất 2e nên $\text{Mg}^{2+}$ có kích thước nhỏ hơn nguyên tử $\text{Mg}$.
		\end{itemchoice}
	}
\end{ex}
%%%=============EX_19=============%%%
\begin{ex}
	So sánh $\text{NaCl}$ và $\text{AgCl}$:
	\choiceTF
	{\True Đều là hợp chất ion}
	{Có cùng độ tan trong nước}
	{\True $\text{AgCl}$ có màu trắng}
	{\True $\text{Ag}^+$ và $\text{Na}^+$ có cùng điện tích}
	\loigiai{
		\begin{itemchoice}[T1,F2,T3,T4]
			\itemch Cả hai đều được tạo thành từ kim loại và phi kim.
			\itemch $\text{NaCl}$ tan tốt trong nước, $\text{AgCl}$ không tan.
			\itemch $\text{AgCl}$ là kết tủa màu trắng.
			\itemch Đều là ion +1 do mất 1 electron.
		\end{itemchoice}
	}
\end{ex}
%%%=============EX_20=============%%%
\begin{ex}
	Xét sự phân ly của $\text{Na}_2\text{CO}_3$ trong nước:
	\choiceTF
	{\True Tạo thành ion $\text{Na}^+$ và $\text{CO}_3^{2-}$}
	{Tỉ lệ số ion $\text{Na}^+$ và $\text{CO}_3^{2-}$ là 1:1}
	{\True Dung dịch dẫn điện}
	{\True Phản ứng được với dung dịch $\text{CaCl}_2$}
	
	\loigiai{
		\begin{itemchoice}[T1,F2,T3,T4]
			\itemch $\text{Na}_2\text{CO}_3 \rightarrow 2\text{Na}^+ + \text{CO}_3^{2-}$
			\itemch Tỉ lệ số ion $\text{Na}^+:\text{CO}_3^{2-}$ = 2:1.
			\itemch Các ion tự do trong dung dịch dẫn điện tốt.
			\itemch Tạo kết tủa $\text{CaCO}_3$ khi tác dụng với $\text{CaCl}_2$.
		\end{itemchoice}
	}
\end{ex}
%%%=============EX_21=============%%%
\begin{ex}
	Các phát biểu sau là đúng hay sai khi nói về tính chất của hợp chất ion?
	\choiceTF
	{\True Có nhiệt độ nóng chảy cao}
	{\True Tan nhiều trong nước và tạo ra dung dịch dẫn điện tốt}
	{\True Không dẫn điện ở trạng thái rắn}
	{\True Ở trạng thái nóng chảy, dẫn điện rất tốt}
	\loigiai{}
\end{ex}

%%%=============EX_22=============%%%
\begin{ex}
	Các phát biểu sau là đúng hay sau khi nói về hợp chất sodium oxide $\left(\mathrm{Na}_2O\right)$?
	\choiceTF
	{\True Trong phân tử $\mathrm{Na}_2O$, các ion $\mathrm{Na}^{+}$ và ion $O^{2-}$ đều đạt cấu hình electron bền vững của khí hiếm neon}
	{\True Mỗi nguyên tử Na nhường 1 electron cho nguyên tử O để tạo thành ion dương}
	{\True Ở điều kiện thường, là chất rắn, khó nóng chảy, khó bay hơi}
	{Không tan trong nước, chỉ tan trong dung môi không phân cực như benzene, carbon tetrachloride,\ldots}
	\loigiai{}
\end{ex}

%%%=============EX_23=============%%%
\begin{ex}
	Các phát biểu sau là đúng hay sai khi nói về hợp chất magnesium oxide $(\mathrm{MgO})$?
	\choiceTF
	{\True Là hợp chất ion}
	{\True Là chất rắn ở điều kiện thường}
	{\True Có nhiệt độ nóng chảy rất cao (khoảng $2852^{\circ} C$)}
	{\True Phân tử tạo bởi lực hút tĩnh điện giữa ion $\mathrm{Mg}^{2+}$ và $O^{2-}$}
	\loigiai{}
\end{ex}
\Closesolutionfile{ans}
\Closesolutionfile{ansbook}
\Closesolutionfile{ansex}
%\bangdapanTF{AnsTF-C03_B02_LKION.tex}
\phan{Bài tập tự luận}
%%%=============SOẠN BT===============%%%
\Opensolutionfile{ansbth}[Ans/LGBT-C03_B02_LKION.tex]
\Opensolutionfile{ansbt}[Ans/AnsBT-C03_B02_LKION.tex]
%%%==============Bai_BT1==============%%%
\begin{bt}
	Cho các ion sau: $K^{+}; \mathrm{Be}^{2+}; \mathrm{Cr}^{3+}; F^{-}; \mathrm{Se}^{2-}; N^{3-}$.
	Viết phương trình biểu diễn sự hình thành mỗi ion trên.
	\loigiai{%
	Phương trình biểu diễn sự hình thành các ion:\\
	\begin{tabular}{*{3}{L{0.2\linewidth}}}
		$\mathrm{K} \rightarrow \mathrm{~K}^{+}+\mathrm{le}$ & $\mathrm{Be} \rightarrow \mathrm{Be}^{2+}+2 \mathrm{e}$ & $\mathrm{Cr} \rightarrow \mathrm{Cr}^{3+}+3 \mathrm{e}$ \\
		$\mathrm{~F}+\mathrm{e} \rightarrow \mathrm{~F}^{-}$ & $\mathrm{Se}+2 \mathrm{e} \rightarrow \mathrm{Se}^{2-}$ & $\mathrm{N}+3 \mathrm{e} \rightarrow \mathrm{~N}^{3-}$
	\end{tabular}
	}
\end{bt}
%%==============HetBai_BT1==============%%%

%%%==============Bai_BT2==============%%%
\begin{bt}
	Cho các ion sau: $_{20}\mathrm{Ca}^{2+}; _{13}\mathrm{Al}^{3+}; _{\phantom{9}9}\mathrm{F}^{-}; _{16}\mathrm{S}^{2-}; _{\phantom{7}7}\mathrm{N}^{3-}$.
	\begin{enumerate}
		\item Viết cấu hình electron của mỗi ion.
		\item Mỗi cấu hình đã viết giống với cấu hình electron của nguyên tử nào?
	\end{enumerate}
	\loigiai{%
		\begin{enumerate}
			\item Cấu hình electron:\\
			\begin{tabular}{L{0.35\linewidth}L{0.1\linewidth}L{0.35\linewidth}L{0.1\linewidth}}
				$_{20}\mathrm{Ca}^{2+}:$ $1 s^2 2 s^2 2 p^6 3 s^2 3 \mathrm{p}^6$ & \text {(I)} & $_{13}\mathrm{Al}^{3+}:$  $1\mathrm{~s}^22\mathrm{~s}^22\mathrm{p}^6$ &\text {(II)}\\
				 $_{\phantom{9}9}\mathrm{~F}^{-}: 1 \mathrm{~s}^2 2 \mathrm{~s}^2 2 \mathrm{p}^6$ & \text {(III)} & ${ }_{16} \mathrm{~S}^{2-}: 1 s^2 2 s^2 2 p^6 3 s^2 3 p^6$&\text {(IV)}\\
				 ${ }_7 \mathrm{~N}^{3-}: 1 \mathrm{~s}^2 2 \mathrm{~s}^2 2 \mathrm{p}^6$&\text {(V)}&&
			\end{tabular}
			\item Các cấu hình (II), (III), (V) giống cấu hình electron của khí hiếm 10 Ne . Các cấu hình (I), (IV) giống cấu hình electron của khí hiếm 18 Ar .
		\end{enumerate}
	}
\end{bt}
%%%==============HetBai_BT2==============%%%

%%%==============Bai_BT3==============%%%
\begin{bt}
	Vì sao các hợp chất ion thường là chất rắn ở nhiệt độ phòng?
	\loigiai{Các hợp chất ion thường là chất rắn ở nhiệt độ phòng vì hợp chất ion có cấu trúc mạng tinh thể ion. Lực tĩnh điện mạnh giữa các phần tử mạng với nhau làm cho khoảng cách giữa các phần tử ngắn lại.}
\end{bt}
%%%==============HetBai_BT3==============%%%

%%%==============Bai_BT4==============%%%
\begin{bt}
	Cho các chất sau: $K_2O, H_2O, H_2\mathrm{~S}, SO_2, \mathrm{NaCl}, K_2\mathrm{~S}, \mathrm{CaF}_2, \mathrm{HCl}$.
	Trong phân tử chất nào có liên kết ion?
	\loigiai{Những phân tử có liên kết ion là: $\mathrm{K}_2 \mathrm{O}, \mathrm{K}_2 \mathrm{~S}, \mathrm{NaCl}_{,} \mathrm{CaF}_2$.}
\end{bt}
%%%==============HetBai_BT4==============%%%

%%%==============Bai_BT5==============%%%
\begin{bt}
	Kể ra những hợp chất ion tạo thành từ các ion sau: $F^{-}, K^{+}, O^{2-}, \mathrm{Ca}^{2+}$.
	%%%VD
	\loigiai{Các hợp chất ion là: $\mathrm{KF}, \mathrm{K}_2 \mathrm{O}, \mathrm{CaF}_2, \mathrm{CaO}$.}
\end{bt}
%%%==============HetBai_BT5==============%%%

%%%==============Bai_BT6==============%%%
\begin{bt}
	Dùng sơ đồ để biểu diễn sự hình thành liên kết trong mỗi hợp chất ion sau đây:
	\begin{enumerate}
		\item magnesium fluoride $\left(\mathrm{MgF}_2\right)$;
		\item potassium fluoride (KF);
		\item sodium oxide $\left(\mathrm{Na}_2O\right)$;
		\item calcium oxide $(\mathrm{CaO})$.
	\end{enumerate}
	\loigiai{%
	\begin{enumerate}
		\item  Magnesium fluoride:\\
		\schemestart
			\chemfig{\charge{[.radius=0.2ex]0:2pt=\.,90:2pt=\:,-90:2pt=\:,180:2pt=\:}{F}} 
			\+ 
			\chemfig{\charge{[.radius=0.2ex]0:2pt=\.,180:2pt=\.}{Mg}} 
			\+
			\chemfig{\charge{[.radius=0.2ex]0:2pt=\:,90:2pt=\:,-90:2pt=\:,180:2pt=\.}{F}} 
			\arrow{->}[,,,-stealth]
			\khungion[-]{\chemfig{\charge{[.radius=0.2ex]0:2pt=\:,90:2pt=\:,-90:2pt=\:,180:2pt=\:}{F}}}
			\+ 
			\khungion[2+][-1.5]{\chemfig{Mg}} 
			\+
			\khungion[-]{\chemfig{\charge{[.radius=0.2ex]0:2pt=\:,90:2pt=\:,-90:2pt=\:,180:2pt=\:}{F}}}
			\arrow{->}[,,,-stealth]
			\chemfig{MgF_2}
		\schemestop
		\item  Potassium fluoride:\\
		\schemestart 
			\chemfig{\charge{[.radius=0.2ex]0:2pt=\.}{K}} 
			\+
			\chemfig{\charge{[.radius=0.2ex]0:2pt=\:,90:2pt=\:,-90:2pt=\:,180:2pt=\.}{F}} 
			\arrow{->}[,,,-stealth]
			\khungion[+][-1.5]{\chemfig{K}} 
			\+
			\khungion[-]{\chemfig{\charge{[.radius=0.2ex]0:2pt=\:,90:2pt=\:,-90:2pt=\:,180:2pt=\:}{F}}}
			\arrow{->}[,,,-stealth]
			\chemfig{KF}
		\schemestop
		\item  Sodium oxide:\\
		\schemestart
			\chemfig{\charge{[.radius=0.2ex]0:2pt=\.}{Na}} 
			\+ 
			\chemfig{\charge{[.radius=0.2ex]0:2pt=\.,180:2pt=\.,90:2pt=\:,-90:2pt=\:}{O}} 
			\+
			\chemfig{\charge{[.radius=0.2ex]180:2pt=\.}{Na}} 
			\arrow{->}[,,,-stealth]
			\khungion[+]{\chemfig{Na}}
			\+ 
			\khungion[2-][-1.5]{			\chemfig{\charge{[.radius=0.2ex]0:2pt=\:,180:2pt=\:,90:2pt=\:,-90:2pt=\:}{O}} } 
			\+
			\khungion[+]{\chemfig{Na}}
			\arrow{->}[,,,-stealth]
			\chemfig{Na_2O}
		\schemestop
		\item  Calcium oxide:\\
		\schemestart 
			\chemfig{\charge{[.radius=0.2ex]0:2pt=\:}{Ca}} 
			\+
			\chemfig{\charge{[.radius=0.2ex]0:2pt=\:,90:2pt=\:,-90:2pt=\:}{O}} 
			\arrow{->}[,,,-stealth]
			\khungion[2+][-1.5]{\chemfig{Ca}} 
			\+
			\khungion[2-]{\chemfig{\charge{[.radius=0.2ex]0:2pt=\:,180:2pt=\:,90:2pt=\:,-90:2pt=\:}{O}}}
			\arrow{->}[,,,-stealth]
			\chemfig{CaO}
		\schemestop
	\end{enumerate}
	}
\end{bt}
%%%==============HetBai_BT6==============%%%

%%%==============Bai_BT7==============%%%
\begin{bt}
	Anion $X^{-}$ có cấu hình electron nguyên tử ở phân lớp ngoài cùng là $3p^6$.
	\begin{enumerate}
		\item Viết cấu hình electron của nguyên tử $X$. Cho biết $X$ là nguyên tố kim loại hay phi kim.
		\item Giải thích bản chất liên kết giữa X với barium.
	\end{enumerate}
	\loigiai{\begin{enumerate}
			\item  Khi nhận electron, nguyên tử $X$ biến thành anion $X^{-}$. Cấu hình electron của X là $1 \mathrm{~s}^2 2 \mathrm{~s}^2 2 \mathrm{p}^6 3 \mathrm{~s}^2 3 \mathrm{p}^5$, X là chlorine. X là phi kim điển hình.
			\item  Barium là nguyên tố kim loại điển hình ở chu kì 6 , nhóm IIA. Barium dễ nhường 2 electron hoá trị và tạo ra cation có điện tích $2+$. Khi chlorine kết hợp với barium, nguyên tử barium nhường 2 electron cho hai nguyên tử chlorine (mỗi nguyên tử chlorine nhận 1 electron), tạo thành các ion $\mathrm{Ba}^{2+}$ và $\mathrm{Cl}^{-}$. Các ion này mang điện trái dấu sẽ hút nhau bằng lực hút tĩnh điện
	\end{enumerate}}
\end{bt}
%%%==============HetBai_BT7==============%%%

%%%==============Bai_BT8==============%%%
\begin{bt}
	Nguyên tố X tích luỹ trong các tế bào thực vật nên rau và trái cây tươi là nguồn cung cấp tốt nguyên tố $X$ cho cơ thể. Các nghiên cứu chỉ ra khẩu phần ăn chứa nhiều X có thể giảm nguy cơ cao huyết áp và đột quy. Nguyên tố Z được dùng chế tạo dược phẩm, phẩm nhuộm và chất nhạy với ánh sáng. Nguyên tử X chỉ có 7 electron trên phân lớp s; còn nguyên tử Z chỉ có 17 electron trên phân lớp p.
	\begin{enumerate}
		\item Viết công thức hoá học của hợp chất tạo bởi $X$ và $Z$.
		\item Hợp chất tạo bởi X và Z có tính dẫn điện không? Vì sao?
		\item Trong thực tế cuộc sống, hợp chất tạo bởi X và Z được dùng để làm gi?
	\end{enumerate}
	\loigiai{\begin{enumerate}
			\item  Nguyên tử X chỉ có 7 electron trên phân lớp s nên cấu hình electron của $X$ là: $1s^22s^22p^63s^23p^64s^1$.
			
			Nguyên tử $Z$ chỉ có 17 e trên phân lớp p nên cấu hình electron của $Z$ là:
			\[1\mathrm{s}^22\mathrm{s}^22\mathrm{p}^63\mathrm{s}^23\mathrm{p}^64\mathrm{s}^23\mathrm{d}^{10}4\mathrm{p}^5\]
			$\Rightarrow \mathrm{X}$ là ${ }_{19} \mathrm{~K}$ và Z là ${ }_{35} \mathrm{Br}$.
			$\Rightarrow$ Công thức hoá học của hợp chất tạo bởi X và Z là KBr .
			\item  Hợp chất KBr có tính dẫn điện khi nóng chảy hoặc tan trong dung dịch vì nó là hợp chất ion.
			\item  Trong thực tế, KBr được sử dụng rộng rãi như thuốc chống co giật và an thần, nó là muối ion điển hình, hoàn toàn phân cực và đạt độ $\mathrm{pH}=7$ trong dung dịch nước.
	\end{enumerate}}
\end{bt}

%%%=============BT_9=============%%%
\begin{bt}
	Cho dãy các ion: $K^{+}$, $S^{2-}$, $\mathrm{Al}^{3+}$, $SO_4^{2-}$, $NH_4^{+}$, $CO_3^{2-}$, $\mathrm{Na}^{+}$, $NO_3^{-}$, $\mathrm{Cl}^{-}$, $\mathrm{Mg}^{2+}$. Có bao nhiêu ion thuộc loại cation?
	\shortans{}
	\loigiai{}
\end{bt}

%%%=============BT_10=============%%%
\begin{bt}
	Cho dãy các ion: $K^{+}, S^{2-}, \mathrm{Al}^{3+}, SO_4^{2-}, NH_4^{+}, CO_3^{2-}, \mathrm{Na}^{+}, NO_3^{-}, \mathrm{Cl}^{-}, \mathrm{Mg}^{2+}$. Có bao nhiêu ion thuộc loại anion?
	\shortans{}
	\loigiai{}
\end{bt}

%%%=============BT_11=============%%%
\begin{bt}
	Cho các ion sau: $\mathrm{Na}^{+}, \mathrm{Mg}^{2+}, \mathrm{Ca}^{2+}, F^{-}, \mathrm{Al}^{3+}, O^{2-}, N^{3-}$. Có bao nhiêu ion có cấu hình electron của khí hiếm neon?
	\shortans{}
	\loigiai{}
\end{bt}

%%%=============BT_12=============%%%
\begin{bt}
	Cho các hợp chất sau: $\mathrm{NaCl}, \mathrm{MgO}, \mathrm{KBr}, \mathrm{CaF}_2, CH_4, HNO_3$. Có bao nhiêu hợp chất chứa liên kết ion trong phân tử?
	\shortans{}
	\loigiai{}
\end{bt}

%%%=============BT_13=============%%%
\begin{bt}
	Cho các hợp chất sau: $\mathrm{MgCl}_2, \mathrm{CaO}, \mathrm{HBr}, NH_4NO_3, \mathrm{CCl}_4, \mathrm{PCl}_5$. Có bao nhiêu hợp chất chứa liên kết ion trong phân tử?
	\shortans{}
	\loigiai{}
\end{bt}
\Closesolutionfile{ansbt}
\Closesolutionfile{ansbth}
%\bangdapanSA{AnsBT-C03_B02_LKION.tex}