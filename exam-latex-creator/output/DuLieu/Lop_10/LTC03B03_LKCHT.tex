\subsubsection{Sự tạo thành liên kết cộng hóa trị}
	\Noibat{Tìm hiểu sự hình thành liên kết đơn, đôi, ba trong một số phân tử}
	\Noibat[\maunhan][][\faStar][]{Sự hình thành liên kết đơn, liên kết cho - nhận}
	\begin{center}
		\schemestart
			\chemfig{\charge{[.radius=0.2ex]0:2pt=\.}{H}}
			\+
			\chemfig{\charge{[.radius=0.2ex]0:2pt=\:,-90:2pt=\:,90:2pt=\:,180:2pt=\.}{Cl}}
			\arrow(c1.east--c2.mid west){->}[,0.9,,-stealth]
			\chemname{			\chemfig{H-[,0.52,,,draw=none]\charge{[.radius=0.2ex]0:2pt=\:,-90:2pt=\:,90:2pt=\:,180:2pt=\:}{Cl}}}{Công thức electron}
		\schemestop 
		\hspace{1.5cm};\hspace{1.5cm}
		\chemname{\chemfig{H-[,0.7]\charge{[.radius=0.2ex]0:2pt=\:,-90:2pt=\:,90:2pt=\:}{Cl}}}{Công thức Lewis} \hspace{1.5cm};\hspace{1.5cm} \chemname{\chemfig{H-[,0.7]Cl}}{Công thức cấu tạo}
		\captionof{figure}{Sự tạo thành liên kết đơn trong phân tử HCl}
	\end{center}
	%%%Liên kết cho - nhận
	\vspace*{0.25cm}
	\begin{center}
		\schemestart
		\chemfig{H-[,0.52,,,draw=none]\charge{[.radius=0.2ex]0:2pt=\:,90:2pt=\:,180:2pt=\:,270:2pt=\:}{N}(-[:-90,0.52,,,draw=none]H)-[:90,0.52,,,draw=none]H}
		\+
		\chemfig{H^+}
		\arrow{->}[,0.9,,-stealth]
		\chembelow[1.2cm]{\khungion{\chemfig{H-[,0.52,,,draw=none]\charge{[.radius=0.2ex]0:2pt=\:,90:2pt=\:,180:2pt=\:,270:2pt=\:}{N}(-[:-90,0.52,,,draw=none]H)(-[:90,0.52,,,draw=none]H)-[,0.52,,,draw=none]H}}}{\text{Công thức electron}}
		\hspace{1.5cm};\hspace{1.5cm}
		\chembelow[1.2cm]{\khungion{\chemfig{H-[,0.55]N(-[:-90,0.55]H)(-[:90,0.55]H)-[,0.55,,,-stealth]H}}}{\text{Công thức cấu tạo}}
		\schemestop
		\vspace*{1.2cm}
		\captionof{figure}{Sự tạo thành liên kết cho - nhận trong ion ${NH_4}^+$}
	\end{center}
	\begin{luuy}
		\indam[\maunhan]{Liên kết cho - nhận} là một trường hợp đặc biệt của liên kết cộng hoá trị, trong đó cặp electron chung chỉ do một nguyên tử đóng góp.
	\end{luuy}
	\Noibat[\maunhan][][\faStar][]{Sự hình thành liên kết đôi}
		%%%Phân tử O2
		\begin{center}
			\schemestart
			\chemfig{\charge{[.radius=0.2ex]0:2pt=\:,180:2pt=\:,90:2pt=\:}{O}}
			\+
			\chemfig{\charge{[.radius=0.2ex]0:2pt=\:,180:2pt=\:,90:2pt=\:}{O}}
			\arrow(.east--.mid west){->}[,0.9,,-stealth]
			\chemname{\chemfig{\charge{[.radius=0.2ex]0:2pt=\:,180:2pt=\:,90:2pt=\:}{O}-[,0.58,,,draw=none]\charge{[.radius=0.2ex]0:2pt=\:,180:2pt=\:,90:2pt=\:}{O}}}{Công thức electron}
			\schemestop 
			\hspace{1.5cm};\hspace{1.5cm}
			\chemname{\chemfig{\charge{[.radius=0.2ex]90:2pt=\:,180:2pt=\:}{O}=[,0.58]\charge{[.radius=0.2ex]0:2pt=\:,90:2pt=\:}{O}}}{Công thức Lewis} \hspace{1.5cm};\hspace{1.5cm}
			 \chemname{\chemfig{O=[,0.58]O}}{Công thức cấu tạo}
			\captionof{figure}{Sự tạo thành liên kết đôi trong phân tử $O_2$}
		\end{center}
		%%%Phân tử CO2
		\vspace{0.25cm}
		\begin{center}
			\schemestart
			\chemfig{\charge{[.radius=0.2ex]0:2pt=\.,180:2pt=\.,90:2pt=\:,-90:2pt=\:}{O}}
			\+
			\chemfig{\charge{[.radius=0.2ex]0:2pt=\.,180:2pt=\.,90:2pt=\:}{C}}
			\+
			\chemfig{\charge{[.radius=0.2ex]0:2pt=\.,180:2pt=\.,90:2pt=\:,-90:2pt=\:}{O}}
			\arrow(.east--.mid west){->}[,0.9,,-stealth]
			\chemname{\chemfig{\charge{[.radius=0.2ex]0:2pt=\:,180:2pt=\:,90:2pt=\:}{O}=[,0.58,,,draw=none]\charge{[.radius=0.2ex]0:3pt=\:,180:3pt=\:}{C}=[,0.58,,,draw=none]\charge{[.radius=0.2ex]0:2pt=\:,180:2pt=\:,90:2pt=\:}{O}}}{Công thức electron}
			\schemestop 
			\hspace{1.5cm};\hspace{1.5cm}
			\chemname{\chemfig{\charge{[.radius=0.2ex]180:2pt=\:,90:2pt=\:}{O}=[,0.58]C=[,0.58]\charge{[.radius=0.2ex]0:2pt=\:,90:2pt=\:}{O}}}{Công thức Lewis} \hspace{1.5cm};\hspace{1.5cm}
			\chemname{\chemfig{O=[,0.58]C=[,0.58]O}}{Công thức cấu tạo}
			\captionof{figure}{Sự tạo thành liên kết đôi trong phân tử $CO_2$}
		\end{center}
	\Noibat[\maunhan][][\faStar][]{Sự hình thành liên kết ba}
	%%%Phân tử N2
	\begin{center}
		\schemestart
		\chemfig{\charge{[.radius=0.2ex]0:2pt=\.,180:2pt=\:,90:2pt=\.,-90:2pt=\.}{N}}
		\+
		\chemfig{\charge{[.radius=0.2ex]0:2pt=\:,180:2pt=\.,90:2pt=\.,-90:2pt=\.}{N}}
		\arrow(.east--.mid west){->[][][1.2pt]}[,0.9,,-stealth]
		\chemname{\chemfig{\charge{[.radius=0.2ex]180:2pt=\:,0:2pt=\.,23:2.20pt=\.,-23:2.20pt=\.}{N}-[,0.58,,,draw=none]\charge{[.radius=0.2ex]0:2pt=\:,180:2pt=\.,157:2.20pt=\.,-157:2.20pt=\.}{N}}}{Công thức electron}
		\schemestop 
		\hspace{1.5cm};\hspace{1.5cm}
		\chemname{\chemfig{\charge{[.radius=0.2ex]180:2pt=\:}{N}~[,0.58]\charge{[.radius=0.2ex]0:2pt=\:}{N}}}{Công thức Lewis} \hspace{1.5cm};\hspace{1.5cm}
		\chemname{\chemfig{N~[,0.58]N}}{Công thức cấu tạo}
		\captionof{figure}{Sự tạo thành liên kết đôi trong phân tử $N_2$}
	\end{center}
	\Noibat{Tìm hiểu cách viết công thức Lewis}
	\begin{hopdongian}
		\indam[\maunhan]{Công thức Lewis} của một phân tử được xây dựng từ công thức electron của phân tử, trong đó mỗi cặp electron chung giữa hai nguyên tử tham gia liên kết được thay bằng một gạch nối "-".
	\end{hopdongian}
	\begin{phuongphap}
		\begin{itemize}
			\item  \indam{Bước 1.} Xác định tổng số electron hóa trị bằng cách cộng số nhóm của tất cả các nguyên tử trong phân tử.
			\item  \indam{Bước 2.} Xác định nguyên tử trung tâm. Đây thường là nguyên tử có độ âm điện nhỏ nhất.
			\item  \indam{Bước 3.} Nối các nguyên tử bằng liên kết đơn.
			\item  \indam{Bước 4.} Hoàn thành octet cho các nguyên tử liên kết với nguyên tử trung tâm. Lưu ý rằng hydro chỉ cần 2 electron để hoàn thành octet.
			\item  \indam{Bước 5.} Đặt bất kỳ electron còn lại nào lên nguyên tử trung tâm.
			\item  \indam{Bước 6.} Nếu nguyên tử trung tâm không có octet, hãy thử tạo liên kết đôi hoặc liên kết ba.
			\item  \indam{Bước 7.} Tính điện tích hình thức trên mỗi nguyên tử.Điện tích hình thức là hiệu số giữa số electron hóa trị mà một nguyên tử có ở trạng thái tự do và số electron mà nó "sở hữu" trong cấu trúc Lewis. Mục tiêu là có điện tích hình thức bằng 0 trên mỗi nguyên tử, hoặc càng gần 0 càng tốt.
		\end{itemize}
	\end{phuongphap}
	%%%==============Vidu1==============%%%
	\begin{vd}[Công thức lewis]
		Viết công thức Lewis của phân tử $CO_2$.
		\begin{itemize}
			\item \indam{Bước 1:} Tổng số electron hoá trị của phân tử $CO_2$ là:
			\[4+6\times 2=16\]
			\item  \indam{Bước 2:} Sơ đồ khung biểu diễn liên kết của phân tử $CO_2$:
			\[\mathrm{O}-\mathrm{C}-\mathrm{O}\]
			\item \indam{Bước 3:} Số electron hoá trị chưa tham gia liên kết trong sơ đồ là:
			\[16-2\times 2=12\]
			\item \indam{Bước 4,5,6:}Hoàn thiện octet cho các nguyên tử có độ âm điện lớn hơn trong sơ đồ:
			Số electron hoá trị còn lại: $12-6\times 2=0$.
			Nguyên tử trung tâm C có 4 electron hoá trị, chưa đạt octet.
			\item \indam{Bước 7:} Vì C chưa đạt octet, cẩn chuyển một cặp electron của mỗi nguyên tử oxygen thành cặp electron chung giữa C và O để C đạt octet. Công thức Lewis của phân tử $CO_2$ thu được là:
		\end{itemize}
		\[\chemfig{\charge{[.radius=0.2ex]180:2pt=\:,90:2pt=\:}{O}=[,0.58]C=[,0.58]\charge{[.radius=0.2ex]0:2pt=\:,90:2pt=\:}{O}}\]
		\end{vd}
	%%%==============HetVidu1==============%%%
\subsubsection{Độ âm điện và liên kết hóa học}
\Noibat[\maunhan][][\faStar][]{Phân biệt liên kết cộng hoá trị phân cực và không phân cực}

\begin{ghinho}
	\indam[\maunhan]{Liên kết cộng hoá trị không phân cực} là liên kết cộng hoá trị trong đó \indam[\maunhan]{cặp electron chung không lệch} về phía nguyên tử nào.
	
	\indam[\maunhan]{Liên kết cộng hoá trị phân cực} là liên kết cộng hoá trị trong đó \indam[\maunhan]{cặp electron chung lệch} về phía nguyên tử có \indam[\maunhan]{độ âm điện lớn hơn}.
\end{ghinho}

\Noibat[\maunhan][][\faStar][]{Phân biệt loại liên kết trong phân tử dựa trên giá trị hiệu độ âm điện}

\begin{center}
	\begin{tabular}{|l|l|}
	\hline \multicolumn{1}{|c|}{\textbf{Hiệu độ âm điện $(\Delta \chi)$}}  &  \multicolumn{1}{c|}{\textbf{Loại liên kết}}  \\
	\hline $0 \leq \Delta \chi<0,4$ & Cộng hoá trị không phân cực \\
	\hline $0,4 \leq \Delta \chi<1,7$ & Cộng hoá trị phân cực \\
	\hline$\Delta \chi \geq 1,7$ & lon \\
	\hline
	\end{tabular}
	\captionof{table}{Hiệu độ âm điện $(\Delta \chi)$ và loại liên kết tương ứng}
\end{center}
\subsubsection{Mô tả liên kết cộng hóa trị bằng sự xen phủ orbital nguyên tử}
	\Noibat{Sự xen phủ trục các orbital nguyên tử tạo liên kết $\sigma$ (sigma)}
		\Noibat[\maunhan][][\faStar][0.5]{Xen phủ s - s}
			%%%============= Sự xen phủ AO S-S =====================%%%
			\begin{center}
				\begin{tikzpicture}[declare function={d=0.9cm;h=.05*d;},node font=\bfseries\sffamily]
						\tikzstyle{linestyle} = [line width=2pt,\maunhan!80,->,>=stealth]
						\tikzstyle{myshapestyle} = [line width=1pt,ball color = \maunhan!90,opacity=.65]
						\draw[\maunhan,dashed,line width=1pt] ([xshift=-d]0,0)--([xshift=d]0,0);
						\fill[myshapestyle] (0,0) circle (.75*d);
   						\node at (1.5*d,0) (plus) {\color{\maunhan}\LARGE\bfseries\sffamily+};
						\draw[\maunhan,dashed,line width=1pt] ([xshift=-d]3*d,0)--([xshift=d]3*d,0);
						\fill[myshapestyle] (3*d,0) circle (.75*d);
						\draw[linestyle] (4.5*d,0)--(6*d,0);
						\draw[\maunhan,dashed,line width=1pt] ([xshift=-1.5*d]8*d,0)--([xshift=1.5*d]8*d,0);
						\fill[myshapestyle] (7.5*d,0) circle (.75*d);
						\fill[myshapestyle] (8.5*d,0) circle (.75*d);
						\node at ([shift={(-90:1.2*d)}]0,0) {AO s};
						\node at ([shift={(-90:1.2*d)}]3*d,0) {AO s};
						\node at ([shift={(-90:1.2*d)}]8.0*d,0) {Xen phủ trục s-s};
					\end{tikzpicture}
			\end{center}
		\Noibat[\maunhan][][\faStar][0.5]{Xen phủ s - p}
		
		\begin{tikzpicture}[declare function={d=1.8cm;r=.55*d;h=.125*d;R=.36*d;k=0.65*d;}]
			\tikzstyle{linestyle} = [line width=1pt,\maunhan!80,dashed]
			\tikzstyle{myshapestyle} = [line width=1pt,opacity=.90,ball color =\mauphu!90]
			\tikzset{
				pics/.cd,
				AOs/.style args={#1/#2}{code={%
					\if\relax\detokenize{#1}\relax
					\def\ballcolor{red}
					\else
					\def\ballcolor{#1}
					\fi,
					\if\relax\detokenize{#2}\relax
					\def\opacity{0.8}
					\else
					\def\opacity{#2}
					\fi
					\draw[linestyle] ([xshift=-1.2*R]0*d,0)--([xshift=1.2*R]0*d,0);
					\fill[myshapestyle,ball color = \ballcolor,opacity=\opacity] (0*d,0) circle (R);
					}
				},
				AOp/.style={code={%
					\draw[linestyle] (0,{-d - h})--(0,{d + h});
					\path[myshapestyle,pic actions] (0,0)..controls +(0:{.25*r}) and +(0:r)..(0,d)--
					(0,d)..controls +(180:r) and +(180:{.25*r})..(0,0);
					\path[myshapestyle,pic actions] (0,-d)..controls +(180:r) and +(180:{.25*r})..(0,0)--
					(0,0)..controls +(0:{.25*r}) and +(0:r)..(0,-d);
					} 
				}
			}
			\path (0,0) coordinate (A)
			(1.5*k,0) coordinate (B)
			(4*k,0) coordinate (C)
			(10*k,0) coordinate (D)
			;
			\pic [local bounding box=AOsa] at (A) {AOs={\maunhan}/{}};
			\node(plus) at (B) {\color{\maunhan}\LARGE\bfseries\sffamily +};
			\pic [local bounding box=AOPy,rotate= 90] at (C) {AOp};
			\pic [local bounding box=AOPz,rotate= 90] at (D) {AOp};
			\pic [local bounding box=AOsb,opacity=0.3] at ([xshift=-1.1*d]D) {AOs={\maunhan}/{}};
			\path[draw,->,-latex,line width=.035*d,\maunhan!80] ([xshift=0.25cm]AOPy.east)--([xshift=-0.25cm]AOsb.west);
			\node [font=\bfseries\sffamily,shift={(-90:0.75*d)} ]at (A){AO s};
			\node [font=\bfseries\sffamily,shift={(-90:0.75*d)} ]at (C){AO p};
			\node [font=\bfseries\sffamily,shift={(-90:0.75*d)} ]at ($(AOsb.west)!0.5!(AOPz.east)$){Xen phủ trục s-p};
		\end{tikzpicture}
		\Noibat[\maunhan][][\faStar][0.5]{Xen phủ p - p}
		%%%============= Sự xen phủ AO p-p =====================%%%
		\begin{center}
			\begin{tikzpicture}[declare function={d=1.5cm;r=.55*d;h=0.125*d;},node font=\bfseries\sffamily]
				\tikzstyle{linestyle} = [->,>=stealth,line width=1pt,\maunhan!80]
				\tikzstyle{myshapestyle} = [line width=1pt,ball color = \maudam!90,opacity=.90]
				\draw[line width=1pt,\maunhan!80] ([xshift=-1.1*d]0*d,0)--([xshift=1.1*d]0*d,0);
				\path[myshapestyle] (-d,0)..controls +(90:r) and +(90:{.25*r})..(0,0)--
				(0,0)..controls +(-90:{.25*r}) and +(-90:r)..(-d,0);
				\path[myshapestyle] (d,0)..controls +(90:r) and +(90:{.25*r})..(0,0)--
				(0,0)..controls +(-90:{.25*r}) and +(-90:r)..(d,0);
				%%%==================================%%%
				\draw[line width=1pt,\maunhan!80] ([xshift=-1.1*d]3*d,0)--([xshift=1.1*d]3*d,0);
				\path[myshapestyle] (2*d,0)..controls +(90:r) and +(90:{.25*r})..(3*d,0)--
				(3*d,0)..controls +(-90:{.25*r}) and +(-90:r)..(2*d,0);
				\path[myshapestyle] (4*d,0)..controls +(90:r) and +(90:{.25*r})..(3*d,0)--
				(3*d,0)..controls +(-90:{.25*r}) and +(-90:r)..(4*d,0);
				%%%==================================%%%
				\draw[line width=1pt,\maunhan!80] ([xshift=-2.05*d]7.875*d,0)--([xshift=2.05*d]7.875*d,0);
				\path[myshapestyle] (6*d,0)..controls +(90:r) and +(90:{.25*r})..(7*d,0)--
				(7*d,0)..controls +(-90:{.25*r}) and +(-90:r)..(6*d,0);
				\path[myshapestyle] (8*d,0)..controls +(90:r) and +(90:{.25*r})..(7*d,0)--
				(7*d,0)..controls +(-90:{.25*r}) and +(-90:r)..(8*d,0);
				%%%==================================%%%
				\path[myshapestyle] (7.75*d,0)..controls +(90:r) and +(90:{.25*r})..(8.75*d,0)--
				(8.75*d,0)..controls +(-90:{.25*r}) and +(-90:r)..(7.75*d,0);
				\path[myshapestyle] (9.75*d,0)..controls +(90:r) and +(90:{.25*r})..(8.75*d,0)--
				(8.75*d,0)..controls +(-90:{.25*r}) and +(-90:r)..(9.75*d,0);
				\draw[linestyle,line width=0.05*d] (4.5*d,0)--(5.5*d,0);
				\node at (1.5*d,0) {\color{\maunhan}\LARGE\bfseries\sffamily +};
				\node at ([shift=(-90:.55*d)]0*d,0) {AO p};
				\node at ([shift=(-90:.55*d)]3*d,0) {AO p};
				\node at ([shift=(-90:.55*d)]7.875*d,0) {Xen phủ trục p-p};
			\end{tikzpicture}
		\end{center}
	\Noibat{Sự xen phủ bên các orbital nguyên tử tạo liên kết $\pi$ (pi)}
	%
	%%%%=============Sự xen phủ bên=====================%%%
	\begin{center}
			\begin{tikzpicture}[declare function={d=2cm;r=.55*d;h=.125*d;R=.36*d;k=0.65*d;}]
			\tikzstyle{linestyle} = [line width=1pt,\maunhan]
			\tikzstyle{myshapestyle} = [line width=1pt,ball color = \maudam]
			\tikzset{
				pics/.cd,
					AOs/.style={code={
						\draw[linestyle] ([xshift=-1.2*R]0*d,0)--([xshift=1.2*R]0*d,0);
						\path[myshapestyle,pic actions] (0*d,0) circle (R);
					}
				},
					AOP/.style={code={
						\draw[linestyle] (0,{-d - h})--(0,{d + h});
						\path[myshapestyle,pic actions] (0,0)..controls +(0:{.25*r}) and +(0:r)..(0,d)--
						(0,d)..controls +(180:r) and +(180:{.25*r})..(0,0);
						\path[myshapestyle,pic actions] (0,-d)..controls +(180:r) and +(180:{.25*r})..(0,0)--
						(0,0)..controls +(0:{.25*r}) and +(0:r)..(0,-d);
					}
				}
			}
			\path (0,0) coordinate (A)
			(1.5*k,0) coordinate (B)
			(3*k,0) coordinate (C)
			(7*k,0) coordinate (D)
			;
			\pic [local bounding box=AOPx] at (A) {AOP};
			\node(plus) at (B) {\color{\maunhan}\LARGE\bfseries\sffamily +};
			\pic [local bounding box=AOPy] at (C) {AOP};
			\pic [local bounding box=AOPxH,opacity=0.9] at (D) {AOP};
			\pic [local bounding box=AOPyH,opacity=0.9] at ([xshift=0.355*d]D) {AOP};
			\path[draw,->,-latex,line width=.035*d,\maunhan!80] (AOPy)--(AOPxH);
			\node [font=\bfseries\sffamily,shift={(-90:1.3*d)} ]at (A){AO p};
			\node [font=\bfseries\sffamily,shift={(-90:1.3*d)} ]at (C){AO p};
			\node [font=\bfseries\sffamily,shift={(-90:1.3*d)} ]at ($(AOPxH)!0.5!(AOPyH)$){Xen phủ bên p-p};
		\end{tikzpicture}
	\end{center}
	\begin{ghinho}
		\begin{itemize}
			\item  \indam[\maunhan]{Liên kết $\sigma$} là loại liên kết cộng hoá trị được hình thành do sự \indam[\maunhan]{xen phủ trục} của hai orbital. Vùng xen phủ nẳm trên đường nối tâm hai nguyên tử.
			\item  \indam[\maunhan]{Liên kết $\pi$} là loại liên kết cộng hoá trị được hình thành do sự \indam[\maunhan]{xen phủ bên} của hai orbital. Vùng xen phủ nằm hai bên đường nối tâm hai nguyên tử.
			\item Do mức độ xen phủ trục lớn hơn mức độ xen phủ bên nên \indam[\maunhan]{liên kết $\sigma$ bền hơn liên kết $\pi$}
		\end{itemize}
	\end{ghinho}
	\begin{vidu}Giải thích sự hình thành phân tử $O_2$ bằng sự xen phủ
		%%%===============GIẢI THÍCH SỤ HÌNH THÀNH PHÂN TỬ OXI================%%%
		\par\begin{tikzpicture}[declare function={d=2.5cm;r=.55*d;h=.125*d;R=.85*d;k=.035*d;},node distance= k and k]
			\node (AOPX){%
				\begin{tikzpicture}[scale=.5]
					\tikzstyle{linestyle} = [->,>=stealth,line width=.02*d,\maunhan!80]
					\tikzstyle{myshapestyle} = [ball color = \maudam!90,opacity=.90]
					\draw[line width=.02*d,\maunhan!80] ({-1*d - h},0)--({1*d + h},0);
					\draw[line width=.02*d,\maunhan!80] (0*R,{-d - h})--(0*R,{d + h});
					\path[myshapestyle] (-d,0)..controls +(90:r) and +(90:{.25*r})..(0,0)--
					(0,0)..controls +(-90:{.25*r}) and +(-90:r)..(-d,0);
					\path[myshapestyle] (d,0)..controls +(90:r) and +(90:{.25*r})..(0,0)--
					(0,0)..controls +(-90:{.25*r}) and +(-90:r)..(d,0);
					%%============================================================%%
					\path[myshapestyle] (0*R,0*d)..controls +(0:{.25*r}) and +(0:r)..(0*R,d)--
					(0*R,d)..controls +(180:r) and +(180:{.25*r})..(0*R,0);
					\path[myshapestyle] (0*R,-d)..controls +(180:r) and +(180:{.25*r})..(0*R,0*d)--
					(0*R,0*d)..controls +(0:{.25*r}) and +(0:r)..(0*R,-d);
					\end{tikzpicture}
				} ;
			\node [right=of AOPX] (plus) {\color{\maunhan}\LARGE\bfseries\sffamily +};
			\node [right=of plus] (AOPY){%
				\begin{tikzpicture}[scale=.5]
						\tikzstyle{linestyle} = [->,>=stealth,line width=.02*d,\maunhan!80]
						\tikzstyle{myshapestyle} = [ball color = \maudam!90,opacity=.90]
						\draw[line width=.02*d,\maunhan!80] ({-1*d - h},0)--({1*d + h},0);
						\draw[line width=.02*d,\maunhan!80] (0*R,{-d - h})--(0*R,{d + h});
						\path[myshapestyle] (-d,0)..controls +(90:r) and +(90:{.25*r})..(0,0)--
						(0,0)..controls +(-90:{.25*r}) and +(-90:r)..(-d,0);
						\path[myshapestyle] (d,0)..controls +(90:r) and +(90:{.25*r})..(0,0)--
						(0,0)..controls +(-90:{.25*r}) and +(-90:r)..(d,0);
						%%============================================================%%
						\path[myshapestyle] (0*R,0*d)..controls +(0:{.25*r}) and +(0:r)..(0*R,d)--
						(0*R,d)..controls +(180:r) and +(180:{.25*r})..(0*R,0);
						\path[myshapestyle] (0*R,-d)..controls +(180:r) and +(180:{.25*r})..(0*R,0*d)--
						(0*R,0*d)..controls +(0:{.25*r}) and +(0:r)..(0*R,-d);
				\end{tikzpicture}
			};
			\node[right=of AOPY](muiten){\tikz{\path[draw,->,-latex,line width=.075*d,\maunhan!80] (0,0)--(3,0);}};
			\node [right=of muiten] (AOXENPHU){%
				\begin{tikzpicture}[scale=.5]
					\tikzstyle{linestyle} = [->,>=stealth,line width=.02*d,\maunhan!80]
					\tikzstyle{myshapestyle} = [ball color = \maudam!90,opacity=.90]
					\draw[line width=.02*d,\maunhan!80] ({-d - h},0)--({2.8*d + h},0);
					\draw[line width=.02*d,\maunhan!80] (0*R,{-d - h})--(0*R,{d + h});
					\path[myshapestyle] ([shift={(.49cm,-.085cm)}]0,1*d)..controls +(-30:.49*d) and +(-150:.49*d)..([shift={(-.49cm,-.085cm)}]1.8*d,1*d)--
					(1.8*d,d)..controls +(180:r) and +(180:{.25*r})..(1.8*d,0)--
					(1.8*d,0*d)..controls +(135:.82*d) and +(45:.82*d)..(0,0*d)--
					(0*R,0*d)..controls +(0:{.25*r}) and +(0:r)..(0*R,d)--cycle;
					\begin{scope}[transform canvas={yscale=-1}]
						\path[myshapestyle] ([shift={(.49cm,-.085cm)}]0,1*d)..controls +(-30:.49*d) and +(-150:.49*d)..([shift={(-.49cm,-.085cm)}]1.8*d,1*d)--
						(1.8*d,d)..controls +(180:r) and +(180:{.25*r})..(1.8*d,0)--
						(1.8*d,0*d)..controls +(135:.82*d) and +(45:.82*d)..(0,0*d)--
						(0*R,0*d)..controls +(0:{.25*r}) and +(0:r)..(0*R,d)--cycle;
					\end{scope}
					\path[myshapestyle] (-d,0)..controls +(90:r) and +(90:{.25*r})..(0,0)--
					(0,0)..controls +(-90:{.25*r}) and +(-90:r)..(-d,0);
					\path[myshapestyle] (d,0)..controls +(90:r) and +(90:{.25*r})..(0,0)--
					(0,0)..controls +(-90:{.25*r}) and +(-90:r)..(d,0);
					%%============================================================%%
					\path[myshapestyle] (0*R,0*d)..controls +(0:{.25*r}) and +(0:r)..(0*R,d)--
					(0*R,d)..controls +(180:r) and +(180:{.25*r})..(0*R,0);
					\path[myshapestyle] (0*R,-d)..controls +(180:r) and +(180:{.25*r})..(0*R,0*d)--
					(0*R,0*d)..controls +(0:{.25*r}) and +(0:r)..(0*R,-d);
					%%%=============================================================%%%
					\path[myshapestyle] (.8*d,0)..controls +(90:r) and +(90:{.25*r})..(1.8*d,0)--
					(1.8*d,0)..controls +(-90:{.25*r}) and +(-90:r)..(.8*d,0);
					\path[myshapestyle] (2.8*d,0)..controls +(90:r) and +(90:{.25*r})..(1.8*d,0)--
					(1.8*d,0)..controls +(-90:{.25*r}) and +(-90:r)..(2.8*d,0);
					%%==============================================%%
					\draw[line width=.02*d,\maunhan!80] (1.8*d,{-d - h})--(1.8*d,{d + h});
					\path[myshapestyle] (1.8*d,0*d)..controls +(0:{.25*r}) and +(0:r)..(1.8*d,d)--
					(1.8*d,d)..controls +(180:r) and +(180:{.25*r})..(1.8*d,0);
					\path[myshapestyle] (1.8*d,-d)..controls +(180:r) and +(180:{.25*r})..(1.8*d,0*d)--
					(1.8*d,0*d)..controls +(0:{.25*r}) and +(0:r)..(1.8*d,-d);
					%%================================================%%
				\end{tikzpicture}
				} ;
			\node[below=of AOPX,yshift=.15cm]{\color{\maunhan}\bfseries\sffamily O};
			\node[below=of AOPY,yshift=.15cm]{\color{\maunhan}\bfseries\sffamily O};
			\node[below=of AOXENPHU,yshift=.15cm]{\color{\maunhan}\bfseries\sffamily phân tử O$\mathbf{_2}$};
			\node[above=of AOXENPHU,yshift=-1.2cm]{\color{\maunhan}\LARGE $\mathbf{\pi}$};
			\node[below=of AOXENPHU,yshift=2.0cm]{\color{\maunhan}\LARGE $\mathbf{\sigma}$};
		\end{tikzpicture}
	\end{vidu}
	%%%==============VD_1==============%%%
	\begin{vidu}
		Tổng năng lượng liên kết trong phân tử $\mathrm{CH}_4$ là $1660 \mathrm{kJ} / \mathrm{mol}$.
		$$ \mathrm{CH}_4(\mathrm{g}) \rightarrow \mathrm{C}(\mathrm{g})+4 \mathrm{H}(\mathrm{g}) \quad \mathrm{E}_{\mathrm{b}}=1660\mathrm{kJ}/\mathrm{mol}$$. Do đó, năng lượng liên kết trung bình của một liên kết $\mathrm{C}-\mathrm{H}$ là $\dfrac{1660}{4}=415 \mathrm{kJ} / \mathrm{mol}$.
	\end{vidu}
	%%%==============HetVD_1==============%%%
\subsubsection{Năng lượng liên kết cộng hóa trị}
\begin{ghinho}
	\indam[\maunhan]{Năng lượng của một liên kết hoá học} là năng lượng cần thiết để  \indam[\maunhan]{phá vỡ 1 mol liên kết đó ở thể khí}, tạo thành các \indam[\maunhan]{nguyên tử ở thể khí}.
	
	Giá trị năng lượng của một liên kết hoá học là thước đo \indam[\maunhan]{độ bền liên kết}.
\end{ghinho}
%%%
\begin{vidu}
	Cho các phương trình phản ứng sau:
	$$
	\begin{array}{ll}
		\mathrm{H}_2(\mathrm{g}) \rightarrow 2\mathrm{H}(\mathrm{g}) & \mathrm{E}_{\mathrm{b}}=432 \mathrm{~kJ}/\mathrm{mol} \quad (1) \\
		\mathrm{N}_2(\mathrm{g}) \rightarrow 2\mathrm{N}(\mathrm{g}) & \mathrm{E}_{\mathrm{b}}=945 \mathrm{~kJ}/\mathrm{mol} \quad (2)
	\end{array}
	$$
	Ta nói năng lượng liên kết ${E}_{b}$ trong phân tử $\mathrm{H}_2$ và $N_2$ lần lượt là $432$ $kJ/mol$ và $945$ $kJ/mol$. Điều này có nghĩa cần cung cấp $432$ kJ và $945$ kJ để lần lượt phá vỡ $1$ mol khí $H_2$ và $1$ mol khí ${N}_2$ thành các nguyên tử ở thể khí.
\end{vidu}
%%%
\begin{ghinho}
	\begin{itemize}
		\item \indam[\maunhan]{Liên kết đơn} luôn luôn là \indam[\maunhan]{liên kết $\sigma$}, được tạo thành từ sự xen phủ trục và \indam[\maunhan]{thường bền vững}
		\item \indam[\maunhan]{Liên kết đôi} gồm  \indam[\maunhan]{1 liên kết $\sigma$} và  \indam[\maunhan]{1 liên kết $\pi$}.
		\item \indam[\maunhan]{Liên kết ba} gồm \indam[\maunhan]{1 liên kết $\sigma$} và \indam[\maunhan]{2 liên kết $\pi$}
	\end{itemize}
\end{ghinho}
\begin{luuy}
	Liên kết giữa hai nguyên tử được thực hiện bởi \indam[\maunhan]{một liên kết $\sigma$} và \indam[\maunhan]{một hay nhiều  liên kết $\pi$} được gọi là \indam[\maunhan]{liên kết bội}
\end{luuy}