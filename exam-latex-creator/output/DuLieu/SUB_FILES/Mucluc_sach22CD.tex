\documentclass[Main.tex]{subfiles}
\renewcommand*\printatom[1]{\ensuremath{\mathsf{#1}}}
\setchemfig{bond style={\mycolor,line width=0.8pt},atom sep=2.2em,atom style={\mycolor},bond join=true}
\NewDocumentCommand{\nodefill}{O{\mycolor}mm}{
 \path[
	fill=#1,
	rounded corners=2pt,
	fill opacity=0.5,
	text opacity=1
	]
	([shift={(135:3pt)}]#2.north west)
	rectangle
	([shift={(-45:3pt)}]#3.south east) ;
}
\setlist[itemize,1]{itemsep=-3pt,topsep=0pt,label=\protect{\small\color{\mycolor}\rhombusdot},wide=1cm,leftmargin=1cm}
%%%%MỤc lục phần%%%
\makeatletter
\newif\ifnotstarredversion
\renewcommand{\numberline}[1]{#1.~}
\renewcommand{\thechapter}{\Roman{chapter}}%\arabic{chapter}
\renewcommand*\l@chapter[2]{%
	\ifnum \c@tocdepth >-2\relax
	\addpenalty{-\@highpenalty}%
	\addvspace{10pt \@plus\p@}%
	\setlength\@tempdima{3em}%
	\begingroup
	\tikz[remember picture]{
		\path (0,0) coordinate (A)--(\linewidth,0) coordinate (B);%
		\path ($ (A)!0.5!(B) $) node[font=\bfseries\sffamily\Large,align=center,text width=\linewidth] (TEXT)
		%{{Chủ đề~#1 - Trang~#2}};
		{{#1}};
		\fill[dnvang!65!black,rounded corners=3.5pt] ([xshift=-2pt]TEXT.north west) rectangle ([xshift=2pt]TEXT.south east);
		\fill[dnvang!15,rounded corners=3pt] ([yshift=0.15pt]TEXT.north west) rectangle ([yshift=-0.15pt]TEXT.south east);
		\hypersetup{linkcolor=dnvang!85!black}
		\path ($ (A)!0.5!(B) $) node[font=\bfseries\sffamily\Large\color{\mycolor!85!black},align=center,text width=\linewidth] (TEXT)
		{{#1}};
		%		{{Chủ đề~#1 - Trang~#2}};
	}
	\par\bigskip
	\penalty\@highpenalty
	\setcounter{section}{0}
	\endgroup
	\fi
}
\makeatother

\begin{document}
\renewcommand{\thesection}{Chuyên đề \arabic{section}.\,}
%%%Nhớ tắt 3 lệnh này khi chạy filemain
\setcounter{tocdepth}{1}
\setcounter{secnumdepth}{4}
\tableofcontents
\titlespacing*{\subsubsection}{0cm}{0pt}{0pt}
%\part{Hóa hữu cơ}
%\chapter{Tập một (10 chuyên đề)}
%\section{Sử dụng quy tắc hóa trị để lập phương trình hóa học\quad}
%\section{Sơ đồ phản ứng - Xác định chất dựa theo kết quả định tính}
%\section{Phương pháp  giải bài tập độ tan tinh thể hidrat (muối ngậm nước)}
%\section{Bài toán có chất dư - Chứng minh hỗn hợp phản ứng hết}
%\section{Sử dụng phương pháp hợp thức}
%\section{Phân dạng và phương pháp giải bài tập về $\mathsf{CO_2}$ hoặc $\mathsf{SO_2}$ tác dụng với kiềm}
%\section{Kĩ thuật tự chọn lượng chất}
%\section{Bài tập nhận biết - Dự đoán và giải thích hiện tượng hóa học}
%\section{Phương pháp xác định CTHH qua phép tính theo phương trình hóa học}
%\section{Phương pháp bảo toàn khối lượng và bảo toàn số mol nguyên tố}
\setcounter{chapter}{1}
\chapter{Tập hai (12 chuyên đề)}
\setcounter{section}{10}
\section{Toán kim loại tác dụng với dung dịch muối}
\section{Bài tập về cấu tạo nguyên tử, Bảng tuần hoàn các nguyên tố hóa học}
\section{Biện luận tìm CTPT và viết CTCT dựa vào một số đặc điểm của hợp chất hữu cơ}
\section{Tìm CTPT hợp chất hữu cơ qua phép tính theo PTHH}
\section{Xác định lượng chất trong hỗn hợp bằng cách tính theo phương trình hóa học}
\section{Phương pháp giải toán cộng $\mathsf{H_2}$, $\mathsf{Br_2}$ vào hợp chất HC không no}
\section{Xác định khoảng biến thiên của lượng chất}
\section{Bài toán về hiệu suất phản ứng}
\section{Chuỗi phản ứng và phương trình hóa học hữu cơ}
\section{Bài toán nhận biết- dự đoán, giải thích hiện tượng hóa học (Phần hữu cơ)}
\section{Điều chế - tinh chế, tách chất ra khỏi hỗn hợp}
\section{Vận dụng một số kĩ thuật giải nhanh các bài tập trắc nghiệm định lượng}
\end{document}

