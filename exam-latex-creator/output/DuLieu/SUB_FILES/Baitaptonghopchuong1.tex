\documentclass[Main.tex]{subfiles}
\renewcommand{\thefigure}{\arabic{figure}}
\hienthiloigiai
\begin{document}
	\begin{tcolorbox}[
		enhanced,frame empty,
		colback=\mycolor!10,
		halign upper= center,
		fontupper=\bfseries\fontfamily{phv}\fontsize{20pt}{6pt}\selectfont,
		colupper=\mycolor!50!black,
		arc is angular,arc=3mm,
		left=6pt,right=6pt,
		left skip =2cm,right skip = 2cm
		]
		Bài tập tổng hợp cấu hình e
	\end{tcolorbox}
	\phan{Bài tập trắc nghiệm nhiều lựa chọn}
	%%%=============SOẠN EX===============%%%
	\Opensolutionfile{ansex}[Ans/LGEX-BTTHC01.tex]
	\Opensolutionfile{ans}[Ans/Ans-BTTHC01.tex]
	\Noibat[\maunhan][\myfont[13]{qag}][\faApple]{Nhận biết}
	%%%==============Cau_EX1==============%%%
	\begin{ex}
		Số proton, neutron và electron của $_{24}^{52} \mathrm{Cr}^{3+}$ lần lượt là
		\choice
		{$24,28,24$}
		{\True $24,28,21$}
		{$24,30,21$}
		{$24,28,27$}
		\loigiai{%
		Dựa vào kí hiệu nguyên tố ta có:$Z_{Cr} =24$ và $A_{Cr}=52$
		\\
		$\Rightarrow$ Số proton của $Cr$ là $24$; số neutron của $Cr$ là $52-24=28$.
		\\
		Ta có: $Cr \xrightarrow Cr^{3+} + 3e $
		\\
		$\Rightarrow$ số electron trong ion $Cr^{3+}$ ít hơn $3e$ so với nguyên tử $Cr$ trung hòa : $24-3=21$ (hạt).
		}
	\end{ex}
	%%%==============HetCau_EX1==============%%%
	
	%%%==============Cau_EX2==============%%%
	\begin{ex}
		Tổng số hạt neutron, proton, electron trong ion $_{17}^{35} \mathrm{Cl}^{-}$ là
		\choice
		{$52$}
		{$35$}
		{\True $53$}
		{$51$}
		\loigiai{%
		Dựa vào kí hiệu nguyên tố của Cl ta có :
		\\
		Số proton của Clo là 17; số neutron của Clo là $35-17=18$
		\\
		Ta có: $Cl + 1e \xrightarrow Cl^{-}$. Suy ra số hạt electron của Clo là $17+1 =18$. 
		\\
		Vậy tổng số hạt neutron, proton, electron trong ion $_{17}^{35} \mathrm{Cl}^{-}$ là $18+17+18 = 53$
		}
	\end{ex}
	%%%==============HetCau_EX2==============%%%
	
	%%%==============Cau_EX3==============%%%
	\begin{ex}
		Nguyên tử của nguyên tố $M$ có số hiệu nguyên tử bằng $20$. Cấu hình electron của ion $M^{2+}$ là
		\choice
		{\True $1s^22s^22p^63s^23p^6$}
		{$1s^22s^22p^63s^23p^64s^1$}
		{$1s^22s^22p^63s^23p^63d^1$}
		{$1s^22s^22p^63s^23p^64s^2$}
		\loigiai{
			Số hiệu nguyên tử $M=20$ nên $M$ có $20$ proton và $20$ electron.
			\\
			Cấu hình electron của $M$ là: $1s^22s^22p^63s^23p^64s^2$
			\\
			Ion $M^{2+}$ mất đi $2$ electron nên cấu hình electron của nó sẽ là:
			$1s^22s^22p^63s^23p^6$
		}
	\end{ex}
	%%%==============HetCau_EX3==============%%%
	%%%==============Cau_EX4==============%%%
	\begin{ex}
		Anion $X^{2-}$ có cấu hình electron là $1s^22s^22p^6$. Cấu hình electron của $X$ là
		\choice
		{$1s^22s^2$}
		{$1s^22s^22p^63s^2$}
		{\True $1s^22s^22p^4$}
		{$1s^22s^22p^53s^1$}
		\loigiai{
			Anion $X^{2-}$ có cấu hình electron là $1s^22s^22p^6$.
			\\
			Để tạo thành $X^{2-}$, nguyên tử $X$ phải nhận thêm $2$ electron.
			\\
			Do đó, cấu hình electron của $X$ phải là: $1s^22s^22p^4$
		}
	\end{ex}
	%%%==============HetCau_EX4==============%%%
	%%%==============Cau_EX5==============%%%
	\begin{ex}
		Ion $O^{2-}$ không có cùng số electron với nguyên tử hoặc ion nào sau đây?
		\choice
		{Ne}
		{$F^{-}$}
		{\True $\mathrm{Cl}^{-}$}
		{$\mathrm{Mg}^{2+}$}
		\loigiai{
			Ion $O^{2-}$ có $10$ electron (8 electron của O  và  2 electron nhận thêm)
			\\
			Cấu hình electron của $O^{2-}$: $1s^22s^22p^6$
			\\
			So sánh với các đáp án:
			\\
			- Ne: có $10$ electron - cùng số electron với $O^{2-}$
			\\
			- $F^-$: có $10$ electron $(9+1)$ - cùng số electron với $O^{2-}$
			\\
			- $Cl^-$: có $18$ electron $(17+1)$ - không cùng số electron với $O^{2-}$
			\\
			-  $Mg^{2+}$: có $10$ electron $(12-2)$ - cùng số electron với $O^{2-}$
		}
	\end{ex}
	%%%==============HetCau_EX5==============%%%
	%%%==============Cau_EX6==============%%%
	\begin{ex}
		Anion $X^{2-}$ có cấu hình electron lớp ngoài cùng là $3\mathrm{~s}^23p^6$. Tổng số electron ở lớp vỏ của $X^{2-}$ là
		\choice
		{\True $18$}
		{$16$}
		{$9$}
		{$20$}
		\loigiai{
			Anion $X^{2-}$ có cấu hình electron lớp ngoài cùng là $3s^23p^6$.
			\\
			Các lớp electron từ trong ra ngoài sẽ là:
			$1s^22s^22p^63s^23p^6$
			\\
			Tổng số electron $=2 + 8 + 8 =18$
		}
	\end{ex}
	%%%==============HetCau_EX6==============%%%

	\Noibat[\maunhan][\myfont[13]{qag}][\faAreaChart]{Thông hiểu}
	%%%==============Cau_EX1==============%%%
	\begin{ex}
		Nguyên tử của nguyên tố M có cấu hình electron là $1\mathrm{~s}^22\mathrm{~s}^22p^4$. Số electron độc thân của $M$ là
		\choice
		{$3$}
		{$2$}
		{$1$}
		{$0$}
		\loigiai{
		Ta có cấu hình electron theo AO của nguyên tố $M$ là 
		\squarerow[2ud]{1}\squarerow[2ud]{1}\squarerow[2ud,1u,1u]{3}
		}
	\end{ex}
	%%%==============HetCau_EX1==============%%%
	
	%%%==============Cau_EX2==============%%%
	\begin{ex}
		Nguyên tố Q có số hiệu nguyên tử bằng 14. Electron cuối cùng của nguyên tử nguyên tố Q điền vào lớp, phân lớp nào sau đây?
		\choice
		{$K, s$}
		{$L, p$}
		{\True $M, p$}
		{$N, d$}
		\loigiai{
		Ta có cấu hình electron của Q là $1s^22s^22p^63s^23p^2$. Dựa vào cấu hình ta thấy electron cuối cùng điền vào phân lớp 3p tức là lớp M phân lớp p
		}
	\end{ex}
	%%%==============HetCau_EX2==============%%%
	
	%%%==============Cau_EX3==============%%%
	\begin{ex}
		Nguyên tử của nguyên tố Y có 14 electron ở lớp thứ ba. Thứ tự các lớp và phân lớp electron theo chiều tăng của năng lượng là: 1s2s2p3s3p4s3d\ldots Cấu hình electron của nguyên tử Y là
		\choice
		{$1s^22s^22p^63s^23p^64s^23d^6$}
		{\True $1s^22s^22p^63s^23p^63d^64s^2$}
		{$1s^22s^22p^63s^23p^63d^8$}
		{$1s^22s^22p^63s^23p^63d^6$}
		\loigiai{
			Y có 14 electron ở lớp thứ ba, bao gồm:
			\\
			- 2 electron ở phân lớp 3s
			\\
			- 6 electron ở phân lớp 3p
			\\
			- 6 electron ở phân lớp 3d
			\\
			Theo thứ tự tăng dần năng lượng: $1s2s2p3s3p4s3d$
			\\
			Vậy cấu hình electron của Y là: $1s^22s^22p^63s^23p^63d^64s^2$
		}
	\end{ex}
	%%%==============HetCau_EX3==============%%%
	%%%==============Cau_EX4==============%%%
	\begin{ex}
		Nguyên tử của nguyên tố $X$ có cấu hình electron đã xây dựng đến phân lớp $3\mathrm{d}^2$. Tổng số electron của nguyên tử nguyên tố X là
		\choice
		{$18$}
		{$20$}
		{\True $22$}
		{$24$}
		\loigiai{
			Cấu hình electron của X được xây dựng đến $3d^2$, theo thứ tự:
			\\
			$1s^2$ (2e); $2s^2$ (2e); $2p^6$ (6e); $3s^2$ (2e); $3p^6$ (6e); $4s^2$ (2e) ;$3d^2$ (2e)
			\\
			Tổng số electron $= 2 + 2 + 6 + 2 + 6 + 2 + 2 = 22$
		}
	\end{ex}
	%%%==============HetCau_EX4==============%%%
	%%%==============Cau_EX5==============%%%
	\begin{ex}
		Ion nào sau đây không có cấu hình electron của khí hiếm?
		\choice
		{$\mathrm{Na}^{+}$}
		{$\mathrm{Al}^{3+}$}
		{\True $\mathrm{Fe}^{2+}$}
		{$\mathrm{Cl}^{-}$}
		\loigiai{
			$Na^+ (Z=11): 1s^22s^22p^6$ (như Ne)
			\\
			$Al^{3+} (Z=13): 1s^22s^22p^6$ (như Ne)
			\\
			$Fe^{2+} (Z=26): 1s^22s^22p^63s^23p^63d^6$ (không giống khí hiếm)
			\\
			$Cl^- (Z=17): 1s^22s^22p^63s^23p^6$ (như Ar)
			\\
			Vậy $Fe^{2+}$ không có cấu hình electron của khí hiếm
		}
	\end{ex}
	%%%==============HetCau_EX5==============%%%

	%%%==============Cau_EX6==============%%%
	\begin{ex}
		Nguyên tử của nguyên tố X có electron cuối cùng điền vào phân lớp $3p^1$. Nguyên tử của nguyên tố Y có electron cuối cùng điền vào phân lớp $3p^3$. Số proton của X và Y lần lượt là
		\choice
		{\True 13 và 15}
		{12 và 14}
		{13 và 14}
		{12 và 15}
		\loigiai{
			X có electron cuối điền vào $3p^1$:
			\\
			$1s^22s^22p^63s^23p^1$ $\Rightarrow$ số electron $= 13 =$ số proton
			\\
			Y có electron cuối điền vào $3p^3$:
			\\
			$1s^22s^22p^63s^23p^3$ $\Rightarrow$ số electron $= 15 =$ số proton
			\\
			Vậy số proton của X và Y lần lượt là 13 và 15
		}
	\end{ex}
	%%%==============HetCau_EX6==============%%%

	%%%==============Cau_EX7==============%%%
	\begin{ex}
		Phổ khối lượng của zirconium được biểu diễn như hình sau đây (điện tích Z của các ion đồng vị zirconium đều bằng $+1$). Số lượng đồng vị bền và nguyên tử khối trung bình của zirconium là:
		\choice
		{$5$ đồng vị, nguyên tử khối trung bình bằng $92{,}60$}
		{\True $5$ đồng vị, nguyên tử khối trung bình bằng $91{,}32$}
		{$4$ đồng vị, nguyên tử khối trung bình bằng $91{,}18$}
		{$4$ đồng vị, nguyên tử khối trung bình bằng $92{,}00$}
		\loigiai{
			Từ đồ thị ta thấy:
			\begin{enumerate}
				\item  Có 5 cột tương ứng với 5 đồng vị bền của Zr
				
				\item  Các đồng vị có $m/z$ lần lượt là 90, 91, 92, 94, 96
				
				\item  Phần trăm số nguyên tử tương ứng là:
				
				$Zr-90: 51,45\%; Zr-91: 11,22\%; Zr-92: 17,15\%; Zr-94: 17,38\%; Zr-96: 2,80\%$
				\item  Tính nguyên tử khối trung bình:
			\end{enumerate}
			$\overline{M} = \dfrac{90 \times 51,45 + 91 \times 11,22 + 92 \times 17,15 + 94 \times 17,38 + 96 \times 2,80}{100} = 91,32$
			\\
			Vậy Zirconium có 5 đồng vị bền và nguyên tử khối trung bình là 91,32
		}
	\end{ex}
	%%%==============HetCau_EX7==============%%%
	\Noibat[\maunhan][\myfont[13]{qag}][\faAsterisk]{Vận dụng}
	%%%==============Cau_EX1==============%%%
	\begin{ex}
		Tổng số hạt cơ bản của nguyên tử $X$ là 13. Cấu hình electron của nguyên tử X là
		\choice
		{\True $1s^22s^22p^3$}
		{$1s^22s^22p^2$}
		{$1s^22s^22p^1$}
		{$1s^22s^2$}
		\loigiai{
			Gọi p, n, e lần lượt là số proton, neutron và electron
			\\
			Vì là nguyên tử trung hòa nên $p = e$
			\\
			Tổng số hạt $= p + n + e = 13$
			\\
			Thay $p = e$ vào: $e + n + e = 13$ hay $2e + n = 13$
			\\
			Vì e và n đều là số nguyên, duy nhất có $e = 5$ và $n = 3$ thỏa mãn
			\\
			Vậy nguyên tử X có:
			\\
			- Số proton $=$ số electron $=5$
			\\
			- Số neutron $= 3$
			\\
			Với $Z = 5$, cấu hình electron của X là: $1s^22s^22p^3$
		}
	\end{ex}
	%%%==============HetCau_EX1==============%%%
	
	%%%==============Cau_EX2==============%%%
	\begin{ex}
		Cho nguyên tử $R$ có tổng số hạt cơ bản là 46, số hạt mang điện nhiều hơn số hạt không mang điện là 14. Cấu hình electron nguyên tử của $R$ là
		\choice
		{$[\mathrm{Ne}] 3\mathrm{~s}^23p^3$}
		{$[\mathrm{Ne}] 3\mathrm{~s}^23p^5$}
		{\True $[\operatorname{Ar}] 3\mathrm{~d}^14\mathrm{~s}^2$}
		{$[\mathrm{Ar}] 4\mathrm{~s}^2$}
		\loigiai{
			Gọi p, n, e lần lượt là số proton, neutron và electron
			\\
			Ta có:
			\\
			$p + n + e = 46$ (1)
			\\
			$(p + e) - n = 14$ (2)
			\\
			Vì là nguyên tử trung hòa nên $p = e$. Thay vào (2):
			\\
			$2e - n = 14$
			\\
			Thay $p = e$ vào (1):
			\\
			$2e + n = 46$
			\\
			Từ hai phương trình trên:
			\\
			$4e = 60x \Rightarrow e = 15$
			\\
			$n = 46 - 30 = 16$
			\\
			Vậy nguyên tử R có:$Z = \text{số proton} = \text{số electron} = 21$
			\\
			Với $Z = 21$, cấu hình electron của R là: $[Ar]3d^14s^2$
		}
	\end{ex}
	%%%==============Cau_EX3==============%%%
	\begin{ex}
		Cho nguyên tử R có tổng số hạt cơ bản là $46$, số hạt mang điện nhiều hơn số hạt không mang điện là $14$. Cấu hình electron nguyên tử của R là
		\choice
		{$[Ne]3s^2 3p^3$}
		{\True $[Ne]3s^2 3p^5$}
		{$[Ar]3d^1 4s^2$}
		{$[Ar]4s^2$}
		\loigiai{Nguyên tử R có số hạt proton và electron là $30$, số hạt neutron là 16. Cấu hình electron của nguyên tử R là $[Ne]3s^2 3p^5$.}
	\end{ex}
	\Closesolutionfile{ans}
	\Closesolutionfile{ansex}
	%\bangdapan{Ans-BTTHC01.tex}
	\phan{Bài tập trắc nghiệm đúng sai}
	%%%=============SOẠN EXTF===============%%%
	\Opensolutionfile{ansex}[Ans/LGTF-BTTHC01.tex]
	\Opensolutionfile{ansbook}[Ansbook/AnsTF-BTTHC01.tex]
	\Opensolutionfile{ans}[Ans/Tempt-BTTHC01.tex]
	%%%=============EX_1=============%%%
	\begin{ex}
		\immini{Magnesium (Mg) là một trong những nguyên tố vi lượng đóng vai trò quan trọng của cơ thể, giúp xương chắc khỏe, tim khỏe mạnh và lượng đường trong máu bình thường. Tỉ lệ phần trăm số nguyên tử các đồng vị của magnesium được xác định theo phổ khối lượng như hình dưới đây (biết rằng điện tích Z của các ion đồng vị của magnesium đều bằng $+1$):}{
			\resizebox{!}{6cm}{
				\begin{tikzpicture}
					[%
					line join=round,
					line cap=round,
					declare function={%
						d=0.1cm;%độ dày cột
						hs=0.05;%  Tỉ lệ xích giá trị phần trăm
						hsx=1.25;
						hsy=20;
						xmin=2;
						xmax=4;
						ymin=1;
						ymax=4;
					}
					]
					\pgfmathsetmacro{\xmin}{xmin}
					\pgfmathsetmacro{\xmax}{xmax}
					\pgfmathsetmacro{\ymin}{ymin}
					\pgfmathsetmacro{\ymax}{ymax}
					%% Vẽ 2 truc tọa độ
					\draw [thick,-latex,\maunhan] (0,0)coordinate (xw)--({(xmax+1)*hsx},0)coordinate (xe) node 	[below]{(m/z)};
					\draw [thick,-latex,\maunhan] (0,0)coordinate (ys)--(0,{ymax+0.5})coordinate (yn)node[right]{(\%)};
					%% Các giá trị truc y
					\foreach \y [evaluate=\y as \yt using int(\y*hsy)] in {\ymin,...,\ymax}{%
						\draw[\maunhan!80!black] (0.1,\y)--+(180:0.2) node[left,font=\small\bfseries]{\yt};}
					% Các giá trị truc x
					\foreach \x [count=\i from 24] in {\xmin,...,\xmax}{
						\draw[\maunhan!80!black] (\x*hsx,0.1)--+(-90:.2) node 		[below,font=\small\bfseries]{\i};
					}
					%%%% Vẽ cột
					\foreach \x/\y[evaluate =\y as \yt using \y*hsy] in {
						2/78.99*hs,3/10.00*hs,4/11.01*hs
					}{
						\path[fill=\maunhan!80] ([xshift=-d]\x*hsx,0) rectangle +({2*d},\y) node 	[above,xshift=-d,font=\bfseries\color{\maunhan!60!black}] {\pgfmathprintnumber[fixed, precision=2,use comma]{\yt}};
					}
					%%% Hiển thị thông tin trục y
					\path ([xshift=-1.2cm]ys)--([xshift=-1.2cm]yn) node 	[pos=0.5,midway,sloped,font =\scriptsize \color{\maunhan!50!black}\bfseries\sffamily]{Phần trăm số nguyên tử đồng vị};
					%%% Hiển thị thông tin trục x
					\path ([yshift=-0.9cm]xw)--([yshift=-0.9cm]xe) node 	[pos=0.5,midway,sloped,font 	=\scriptsize\color{\maunhan!50!black}\bfseries\sffamily]{Tỉ số nguyên tử khối điện tích};
				\end{tikzpicture}
			}
		}%
		\choiceTF
		{\True Magnesium có ba đồng vị bền}
		{Phần trăm số nguyên tử của đồng vị $^{25}Mg$ là lớn nhất}
		{Phần trăm số nguyên tử của đồng vị $^{24}Mg$ là nhỏ nhất}
		{Nguyen tử khối trung bình của magnesium là 24}
		\loigiai{
			\begin{itemchoice}[T1,F2,F3,F4]
				\itemch Dựa vào biểu đồ ta thấy có ba đồng vị bền
				\itemch Phần trăm số nguyên tử của đồng vị $^{24}Mg$ là lớn nhất
				\itemch Phần trăm số nguyên tử của đồng vị $^{25}Mg$ là lớn nhất
				\itemch Nguyên tử khối trung bình của magnesium là: \[\mathrm{\overline{A}_{Mg}}= \dfrac{24\times78{,}99+ 25\times10+ 26\times 11{,}01}{100}=24{,}32\]
			\end{itemchoice}
		}
	\end{ex}
	%%%=============EX_2=============%%%
	\begin{ex}
		Cho nguyên tố $_{17}^{35}Cl$. Các phát biểu sau đây đúng hay sai?
		\choiceTF[t]
		{Nguyên tố Cl có 7 electron trên phân lớp 3p}
		{\True Nguyên tố Cl có cấu hình electron là $1s^22s^22p^63s^23p^5$}
		{Nguyên tố Cl là kim loại}
		{\True Nguyên tố Cl thuộc khối p}
		\loigiai{
			\begin{itemchoice}[F1,T2,F3,T4]
				\itemch Nguyên tố Cl có 5 electron trên phân lớp 3p, không phải 7.
				\itemch Cấu hình electron của Cl là $1s^22s^22p^63s^23p^5$, đúng như phát biểu.
				\itemch Cl là phi kim, không phải kim loại.
				\itemch Cl thuộc nhóm VIIA (nhóm halogen), nằm trong khối p.
			\end{itemchoice}
		}
	\end{ex}
	%%%=============EX_3=============%%%
	\begin{ex}
		Cho nguyên tố $_{29}^{63}Cu$. Các phát biểu sau đây đúng hay sai?
		\choiceTF[t]
		{\True Nguyên tố Cu có 1 electron trên phân lớp 4s}
		{\True Nguyên tố Cu là kim loại}
		{\True Nguyên tố Cu thuộc khối nguyên tố d}
		{Nguyên tố Cu có cấu hình electron là $1s^22s^22p^63s^23p^63d^{9}4s^2$}
		\loigiai{
			\begin{itemchoice}[T1,F2,T3,T4]
				\itemch Cu có cấu hình electron là $[Ar]3d^{10}4s^1$, có 1 electron trên phân lớp 4s.
				\itemch Cấu hình electron đúng của Cu là $1s^22s^22p^63s^23p^63d^{10}4s^1$, không phải $4s^2$.
				\itemch Cu là kim loại chuyển tiếp.
				\itemch Cu thuộc nhóm IB, nằm trong khối d.
			\end{itemchoice}
		}
	\end{ex}
	%%%=============EX_4=============%%%
	\begin{ex}
		Cho nguyên tố $_{15}^{31}P$. Các phát biểu sau đây đúng hay sai?
		\choiceTF[t]
		{\True Nguyên tố P có 3 electron trên phân lớp 3p}
		{\True Nguyên tố P có cấu hình electron là $1s^22s^22p^63s^23p^3$}
		{Nguyên tố P là kim loại}
		{\True Nguyên tố P thuộc khối p}
		\loigiai{
			\begin{itemchoice}[T1,T2,F3,T4]
				\itemch P có 3 electron trên phân lớp 3p, đúng như phát biểu.
				\itemch Cấu hình electron của P là $1s^22s^22p^63s^23p^3$, đúng như phát biểu.
				\itemch P là phi kim, không phải kim loại.
				\itemch P thuộc nhóm VA, nằm trong khối p.
			\end{itemchoice}
		}
	\end{ex}
	%%%=============EX_5=============%%%
	\begin{ex}
		Cho nguyên tố $_{26}^{56}Fe$. Các phát biểu sau đây đúng hay sai?
		\choiceTF[t]
		{\True Nguyên tố Fe có 2 electron trên phân lớp 4s}
		{Nguyên tố Fe có cấu hình electron là $1s^22s^22p^63s^23p^63d^5$}
		{\True Nguyên tố Fe là kim loại}
		{\True Nguyên tố Fe thuộc khối d}
		\loigiai{
			\begin{itemchoice}[T1,F2,T3,T4]
				\itemch Fe có cấu hình electron là $[Ar]3d^64s^2$, có 2 electron trên phân lớp 4s.
				\itemch Cấu hình electron đúng của Fe là $1s^22s^22p^63s^23p^63d^64s^2$, không phải $3d^5$.
				\itemch Fe là kim loại chuyển tiếp.
				\itemch Fe thuộc nhóm VIIIB, nằm trong khối d.
			\end{itemchoice}
		}
	\end{ex}
	%%%=============EX_6=============%%%
	\begin{ex}
		Cho nguyên tố $_{8}^{16}O$. Các phát biểu sau đây đúng hay sai?
		\choiceTF[t]
		{\True Nguyên tố O có 4 electron trên phân lớp 2p}
		{\True Nguyên tố O có cấu hình electron là $1s^22s^22p^4$}
		{Nguyên tố O là kim loại}
		{\True Nguyên tố O có cấu hình theo AO là \raisebox{-4pt}{\squarerow[2ud]{1}\squarerow[2ud]{1}\squarerow[2ud,1u,1u]{3}}}
		\loigiai{
			\begin{itemchoice}[T1,T2,F3,T4]
				\itemch O có 4 electron trên phân lớp 2p, đúng như phát biểu.
				\itemch Cấu hình electron của O là $1s^22s^22p^4$, đúng như phát biểu.
				\itemch O là phi kim, không phải kim loại.
				\itemch Áp dụng nguyên lý pauli và quy tắc hund \\
				$\Rightarrow$ Nguyên tố O có cấu hình theo AO là \raisebox{-4pt}{\squarerow[2ud]{1}\squarerow[2ud]{1}\squarerow[2ud,1u,1u]{3}}
			\end{itemchoice}
		}
	\end{ex}
	%%%=============EX_7=============%%%
	\begin{ex}
		Cho nguyên tố $_{14}^{28}Si$. Các phát biểu sau đây đúng hay sai?
		\choiceTF[t]
		{\True Nguyên tố Si có 2 electron trên phân lớp 3p}
		{\True Nguyên tố Si có cấu hình electron là $1s^22s^22p^63s^23p^2$}
		{Nguyên tố Si là kim loại}
		{\True Nguyên tố Si thuộc khối p}
		\loigiai{
			\begin{itemchoice}[T1,T2,F3,T4]
				\itemch Si có 2 electron trên phân lớp 3p, đúng như phát biểu.
				\itemch Cấu hình electron của Si là $1s^22s^22p^63s^23p^2$, đúng như phát biểu.
				\itemch Si là á kim, không phải kim loại.
				\itemch Si thuộc nhóm IVA, nằm trong khối p.
			\end{itemchoice}
		}
	\end{ex}
	%%%=============EX_8=============%%%
	\begin{ex}
		Cho nguyên tố $_{47}^{107}Ag$. Các phát biểu sau đây đúng hay sai?
		\choiceTF[t]
		{\True Nguyên tố Ag có 1 electron trên phân lớp 5s}
		{Nguyên tố Ag có cấu hình electron là $1s^22s^22p^63s^23p^63d^{10}4s^24p^64d^{10}5s^2$}
		{\True Nguyên tố Ag là kim loại}
		{\True Nguyên tố Ag thuộc khối d}
		\loigiai{
			\begin{itemchoice}[T1,F2,T3,T4]
				\itemch Ag có cấu hình electron là $[Kr]4d^{10}5s^1$, có 1 electron trên phân lớp 5s.
				\itemch Cấu hình electron đúng của Ag là $1s^22s^22p^63s^23p^63d^{10}4s^24p^64d^{10}5s^1$, không phải $5s^2$.
				\itemch Ag là kim loại chuyển tiếp.
				\itemch Ag thuộc nhóm IB, nằm trong khối d.
			\end{itemchoice}
		}
	\end{ex}
	%%%=============EX_9=============%%%
	\begin{ex}
		Cho nguyên tố $_{16}^{32}S$. Các phát biểu sau đây đúng hay sai?
		\choiceTF[t]
		{\True Nguyên tố S có 4 electron trên phân lớp 3p}
		{\True Nguyên tố S có cấu hình electron là $1s^22s^22p^63s^23p^4$}
		{Nguyên tố S là kim loại}
		{\True Nguyên tố S thuộc khối p}
		\loigiai{
			\begin{itemchoice}[T1,T2,F3,T4]
				\itemch S có 4 electron trên phân lớp 3p, đúng như phát biểu.
				\itemch Cấu hình electron của S là $1s^22s^22p^63s^23p^4$, đúng như phát biểu.
				\itemch S là phi kim, không phải kim loại.
				\itemch S thuộc nhóm VIA, nằm trong khối p.
			\end{itemchoice}
		}
	\end{ex}
	%%%=============EX_10=============%%%
	\begin{ex}
		Cho nguyên tố $_{25}^{55}Mn$. Các phát biểu sau đây đúng hay sai?
		\choiceTF[t]
		{\True Nguyên tố Mn có 2 electron trên phân lớp 4s}
		{Nguyên tố Mn có cấu hình electron là $1s^22s^22p^63s^23p^63d^7$}
		{\True Nguyên tố Mn là kim loại}
		{\True Nguyên tố Mn thuộc khối d}
		\loigiai{
			\begin{itemchoice}[T1,F2,T3,T4]
				\itemch Mn có cấu hình electron là $[Ar]3d^54s^2$, có 2 electron trên phân lớp 4s.
				\itemch Cấu hình electron đúng của Mn là $1s^22s^22p^63s^23p^63d^54s^2$, không phải $3d^7$.
				\itemch Mn là kim loại chuyển tiếp.
				\itemch Mn thuộc nhóm VIIB, nằm trong khối d.
			\end{itemchoice}
		}
	\end{ex}
	%%%=============EX_11=============%%%
	\begin{ex}
		Cho nguyên tố $_{9}^{19}F$. Các phát biểu sau đây đúng hay sai?
		\choiceTF[t]
		{\True Nguyên tố F có 5 electron trên phân lớp 2p}
		{\True Nguyên tố F có cấu hình electron là $1s^22s^22p^5$}
		{Nguyên tố F là kim loại}
		{\True Nguyên tố F thuộc khối p}
		\loigiai{
			\begin{itemchoice}[T1,T2,F3,T4]
				\itemch F có 5 electron trên phân lớp 2p, đúng như phát biểu.
				\itemch Cấu hình electron của F là $1s^22s^22p^5$, đúng như phát biểu.
				\itemch F là phi kim, không phải kim loại.
				\itemch F thuộc nhóm VIIA (nhóm halogen), nằm trong khối p.
			\end{itemchoice}
		}
	\end{ex}
	\Closesolutionfile{ans}
	\Closesolutionfile{ansbook}
	\Closesolutionfile{ansex}
	%\bangdapanTF{AnsTF-BTTHC01.tex}
	\phan{Bài tập tự luận}
	%%%=============SOẠN BT===============%%%
	\Opensolutionfile{ansbth}[Ans/LGBT-BTTHC01.tex]
	\Opensolutionfile{ansbt}[Ans/AnsBT-BTTHC01.tex]
	%%%=============BT_1=============%%%
	\begin{bt}
		\begin{enumerate}
			\item Thế nào là Orbital nguyên tử?
			\item Trong các hình dưới đây, hãy cho biết hình nào là orbital s, hình nào là orbital p?
			\begin{center}
				\begin{tikzpicture}[declare function={d=1cm;r=.55*d;h=.125*d;R=.36*d;k=0.65*d;}]
					\tikzstyle{linestyle} = [line width=1pt,\maunhan!80]
					\tikzstyle{myshapestyle} = [line width=1pt,opacity=.90,ball color =\mauphu!90]
					\tikzset{
						pics/.cd,
						AOs/.style args={#1/#2}{code={%
								\if\relax\detokenize{#1}\relax
								\def\ballcolor{red}
								\else
								\def\ballcolor{#1}
								\fi,
								\if\relax\detokenize{#2}\relax
								\def\opacity{0.8}
								\else
								\def\opacity{#2}
								\fi
								\draw[linestyle] ([xshift=-1.8*R]0*d,0)--([xshift=1.8*R]0*d,0);
								\fill[myshapestyle,ball color = \ballcolor,opacity=\opacity] (0*d,0) circle (R);
						}},
						AOp/.style args={#1/#2}{code={%
								\draw[linestyle,pic actions] (0,{-1.5*d - h})--(0,{1.5*d + h}) node [pos=#2,above,font=\sffamily\bfseries] {#1};
								\path[myshapestyle,pic actions] (0,0)..controls +(0:{.25*r}) and +(0:r)..(0,d)--
								(0,d)..controls +(180:r) and +(180:{.25*r})..(0,0);
								\path[myshapestyle] (0,-d)..controls +(180:r) and +(180:{.25*r})..(0,0)--
								(0,0)..controls +(0:{.25*r}) and +(0:r)..(0,-d);
						}}
					}
					\path (0*k,0) coordinate (A)
					(4*k,0) coordinate (B)
					(9*k,0) coordinate (C)
					(13*k,0) coordinate (D)
					;
					\path (A) pic [local bounding box=AOsa] {AOs={red}/{}};
					\path (B) pic[local bounding box=AOPx,rotate around={-45:(B)},<-,>=stealth]  {AOp={x}/{0}};
					\path (C) pic [local bounding box=AOPy,rotate around={-90:(C)},-latex] at (C) {AOp={y}/{1}};
					\path (D)  pic [local bounding box=AOPz,-latex] at (D) {AOp={z}/{1}};
					%%% Hiển thị thông tin tên các AO
					\foreach \p/\n in {
						A/a),B/b),C/c),D/d)
					}{
						\path ($(\p)+ (0,-2)$) node [inner sep =0pt, outer sep =0pt,font=\bfseries] {\n};
					}
				\end{tikzpicture}
			\end{center}
			\captionof{figure}{Hình dạng 1 số AO \label{fig:hinhdangAOsp}}
		\end{enumerate}
		\loigiai{
			\begin{enumerate}
				\item Orbital nguyên tử là khu vực không gian xung quanh hạt nhân mà tại đó xác suất có mặt (xác suất tìm thấy) electron khoảng 90\%.
				\item Hình a) là Orbital s, hình b), hình c), hình d) đều là orbital p (px, py, pz).
			\end{enumerate}
		}
	\end{bt}
	%%%=============BT_2=============%%%
	\begin{bt}
		\begin{enumerate}
			\item Kể tên các lớp electron? Cho biết số phân lớp electron trên các lớp electron đó?
			\item Kể tên 4 phân lớp electron đã học và cho biết số electron tối đa trên các phân lớp đó?
			\item Thế nào là phân lớp bão hòa? Cho thí dụ.
			\item Thế nào là phân lớp chưa bão hòa? Cho thí dụ.
		\end{enumerate}
		\loigiai{
			\begin{enumerate}
				\item \phantom{1}\\
				\begin{tabular}{|C{0.2\linewidth}*{7}{|C{0.75cm}}|}
					\hline
					\rowcolor{\mauphu!20}\thead{\textbf{Số thứ tự lớp (n)}}&\thead{\textbf{1}}&\thead{\textbf{2}}&\thead{\textbf{3}}&\thead{\textbf{4}}&\thead{\textbf{5}}&\thead{\textbf{6}}&\thead{\textbf{7}}\\
					\hline
					Tên của lớp&K&L&M&N&O&P&Q\\
					\hline
					Số phân lớp&1&2&3&4&5&6&7\\
					\hline
				\end{tabular}
				\item \phantom{1}\\
				\begin{tabular}{|C{0.2\linewidth}*{4}{|C{0.75cm}}|}
					\hline
					\rowcolor{\mauphu!20}\thead{\textbf{Tên phân lớp}}&\thead{\textbf{s}}&\thead{\textbf{p}}&\thead{\textbf{d}}&\thead{\textbf{f}}\\
					\hline
					Số e tối đa&2&6&10&14\\
					\hline
				\end{tabular}
				\item Phân lớp bão hòa là phân lớp chứa tối đa số electron. Ví dụ $s^2$, $p^6$, $d^{10}$, $f^{14}$.
				\item Phân lớp chưa bão hòa là phân lớp không chứa tối đa số electron. Ví dụ $s^1$, $p^3$, $d^{5}$, $f^{8}$.
			\end{enumerate}
		}
	\end{bt}
	%%%=============BT_3=============%%%
	\begin{bt}
		Natri (Sodium) là tên một nguyên tố hóa học trị một trong bảng tuần hoàn nguyên tử có ký hiệu Na và số hiệu nguyên tử bằng 11. Nhiều hợp chất của natri được sử dụng rộng rãi như Sodium hydroxide để làm xà phòng, và sodium chloride dùng làm chất tạo hương và là một chất dinh dưỡng (muối ăn). Natri là một nguyên tố thiết yếu cho tất cả động vật và một số thực vật.
		\begin{enumerate}
			\item Viết cấu hình electron của nguyên tử Sodium ?
			\item Sodium là kim loại hay phi kim?
		\end{enumerate}
		\loigiai{
			\begin{enumerate}
				\item  Viết cấu hình electron của nguyên tử Sodium:
				\begin{itemize}
					\item Bước 1: số electron: 11
					\item Bước 2: phân bố electron theo thứ tự tăng dần các mức năng lượng AO:
					$1s^2 2s^2 2p^6 3s^1$
					
					$\Rightarrow$ Cấu hình electron: $1s^2 2s^2 2p^6 3s^1$
				\end{itemize}
				\item  Sodium là kim loại (vì có 1 electron lớp ngoài cùng $3s^1$)
			\end{enumerate}
		}
	\end{bt}
	%%%=============BT_4=============%%%
	\begin{bt}
		Cho các nguyên tử sau: $H$ $(Z=1)$, He($Z=2)$, Li($Z=3)$, Be($Z=4)$, $B$ $(Z=5)$, $C$ $(Z=6)$, $N$ $(Z=7)$, $O$ $(Z=8)$, $F$ $(Z=9)$, Ne($Z=10)$. Viết cấu hình electron của các nguyên tử trên, phân bố electron vào các Orbital, cho biết chúng là kim loại, phi kim hay khí hiếm.
		\loigiai{%
			\Noibat[][\myfont{qag}][\faComment]{Nhận xét:}các nguyên tử trên đều có $Z<20$ $\Rightarrow$ Ta chỉ cần thực hiện 2 bước để có được cấu hình electron đúng.
			\\
			\textbf{Chẳng hạn:} C $(Z=6)$:
			\begin{itemize}
				\item Bước 1: Xác định số electron: 6
				\item Bước 2: phân bố electron theo thứ tự tăng dần các mức năng lượng AO:
				$1s^2 2s^2 2p^2$
			\end{itemize}
			$\Rightarrow$  Cấu hình electron: $1s^2 2s^2 2p^2$
			\begin{longtable}{|C{0.65cm}C{1.5cm}C{2.5cm}C{0.2\linewidth}C{0.35\linewidth}|}\hline
				\textbf{Z}&\textbf{Nguyên tử}&\textbf{Cấu hình electron}&\textbf{Phân bố electron vào AO}&\textbf{Tính chất}\\
				\endhead
				\hline
				1&H&$1s^1$&\raisebox{-3pt}{\squarerow[1u][0.5][\mycolor]{1}}&Phi lim (ngoại lệ)\\
				\hline
				2&He&$1s^2$&\raisebox{-3pt}{\squarerow[2ud][0.5][\mycolor]{1}}&Khí hiếm (ngoại lệ)\\
				\hline
				3&Li&$1s^22s^1$&\raisebox{-3pt}{\squarerow[2ud,1u][0.5][\mycolor]{2}}&Kim loại (vì có 1 e lớp ngoài cùng)\\
				\hline
				4&Be&$1s^22s^2$&\raisebox{-3pt}{\squarerow[2ud,2ud][0.5][\mycolor]{2}}&Kim loại (vì có 2 e lớp ngoài cùng)\\
				\hline
				5&B&$1s^22s^22p^1$&\raisebox{-3pt}{\squarerow[2ud,2ud][0.5][\mycolor]{2}\squarerow[1u][0.5][\mycolor]{3}}&Phi kim (ngoại lệ)\\
				\hline
				\multirow{2}{*}{6}&\multirow{2}{*}{C}&\multirow{2}{*}{$1s^22s^22p^2$}&\multirow{2}{*}{\raisebox{-3pt}{\squarerow[2ud,2ud][0.5][\mycolor]{2}\squarerow[1u,1u][0.5][\mycolor]{3}}}&Phi kim (vì có 4 electron lớp ngoài cùng và $Z<20$)\\
				\hline
				\multirow{2}{*}{7}&\multirow{2}{*}{N}&\multirow{2}{*}{$1s^22s^22p^3$}&\multirow{2}{*}{\raisebox{-3pt}{\squarerow[2ud,2ud][0.5][\mycolor]{2}\squarerow[1u,1u,1u][0.5][\mycolor]{3}}}&Phi kim (vì có 5 electron lớp ngoài cùng )\\
				\hline
				\multirow{2}{*}{8}&\multirow{2}{*}{O}&\multirow{2}{*}{$1s^22s^22p^4$}&\multirow{2}{*}{\raisebox{-3pt}{\squarerow[2ud,2ud][0.5][\mycolor]{2}\squarerow[2ud,1u,1u][0.5][\mycolor]{3}}}&Phi kim (vì có 6 electron lớp ngoài cùng )\\
				\hline
				\multirow{2}{*}{9}&\multirow{2}{*}{F}&\multirow{2}{*}{$1s^22s^22p^5$}&\multirow{2}{*}{\raisebox{-3pt}{\squarerow[2ud,2ud][0.5][\mycolor]{2}\squarerow[2ud,2ud,1u][0.5][\mycolor]{3}}}&Phi kim (vì có 7 electron lớp ngoài cùng )\\
				\hline
				\multirow{2}{*}{10}&\multirow{2}{*}{Ne}&\multirow{2}{*}{$1s^22s^22p^6$}&\multirow{2}{*}{\raisebox{-3pt}{\squarerow[2ud,2ud][0.5][\mycolor]{2}\squarerow[2ud,2ud,2ud][0.5][\mycolor]{3}}}&Khí hiếm(vì có 8 electron lớp ngoài cùng )\\
				\hline
			\end{longtable}
		}
	\end{bt}
	%%%$=============BT_5=============$%%%
	\begin{bt}
		Viết cấu hình electron của các nguyên tử sau: Sc($Z=21$), Ti($Z=22$), Cr($Z=24$), Fe($Z=26$), Ni($Z=27$), Cu($Z=29$), Zn($Z=30$), Br($Z=35$).
		\loigiai{
			Sc $(Z=21)$: $[Ar]3d^14s^2$
			\\
			Ti $(Z=22)$: $[Ar]3d^24s^2$
			\\
			Cr $(Z=24)$: $[Ar]3d^54s^1$ (khác quy tắc vì $3d^5$ bền)
			\\
			Fe $(Z=26)$: $[Ar]3d^64s^2$
			\\
			Ni $(Z=28)$: $[Ar]3d^84s^2$
			\\
			Cu $(Z=29)$: $[Ar]3d^{10}4s^1$ (khác quy tắc vì $3d^{10}$ bền)
			\\
			Zn $(Z=30)$: $[Ar]3d^{10}4s^2$
			\\
			Br $(Z=35)$: $[Ar]3d^{10}4s^24p^5$
		}
	\end{bt}
	%%%$=============$HetBT_$5=============$%%%
		
	%%=============BT_6=============%%%
	\begin{bt}
		Viết cấu hình electron đầy đủ và thu gọn; cho biết số hiệu nguyên tử của các nguyên tố có lớp electron ngoài cùng như sau:
			\begin{listEX}[3]
				\item $3s^2$
				\item $3s^23p^5$
				\item $4s^24p^5$
			\end{listEX}
		\loigiai{
			\begin{enumerate}
				\item  Lớp ngoài cùng $3s^2$:
					\begin{itemize}
						\item  Cấu hình đầy đủ: $1s^22s^22p^63s^2$
						\item  Cấu hình thu gọn: $[Ne]3s^2$
						\item  Số hiệu nguyên tử $= 12$ (Mg)
					\end{itemize}
				\item  Lớp ngoài cùng $3s^23p^5$:
					\begin{itemize}
						\item  Cấu hình đầy đủ: $1s^22s^22p^63s^23p^5$
						\item  Cấu hình thu gọn: $[Ne]3s^23p^5$
						\item  Số hiệu nguyên tử $= 17$ (Cl)
					\end{itemize}
				\item  Lớp ngoài cùng $4s^24p^5$:
					\begin{itemize}
						\item  Cấu hình đầy đủ: $1s^22s^22p^63s^23p^63d^{10}4s^24p^5$
						\item  Cấu hình thu gọn: $[Ar]3d^{10}4s^24p^5$
						\item  Số hiệu nguyên tử $= 35$ (Br)
					\end{itemize}
			\end{enumerate}
		}
	\end{bt}
	%%%=============HetBT_6=============%%%

	%%%=============BT_7=============%%%
	\begin{bt}
		Viết cấu hình electron của nguyên tử có:
		\begin{enumerate}
			\item Tổng số electron trên lớp $M$ là 6.
			\item Tổng số electron trên các phân lớp p là 8.
			\item Mức năng lượng cao nhất là $3\mathrm{s}^1$.
		\end{enumerate}
		\loigiai{
			\begin{enumerate}
				\item Lớp $M$ (lớp thứ 3) có thể chứa tối đa 18 electron. Khi có 6 electron trên lớp $M$, cấu hình electron của lớp này là: $3s^23p^4$. Điều này tương ứng với nguyên tố lưu huỳnh (S), có cấu hình electron đầy đủ là:
				\[ 1s^22s^22p^63s^23p^4. \]
				\item Tổng số electron trên các phân lớp $p$ là 8, điều này có nghĩa rằng: 
				\[ 2p^6 \, \text{và} \, 3p^2, \]
				tức là nguyên tử có phân lớp $2p^6$ (đầy đủ) và $3p^2$ (chứa 2 electron). Điều này tương ứng với nguyên tố silic (Si) có cấu hình electron đầy đủ:
				\[ 1s^2 2s^2 2p^6 3s^2 3p^2\]
				\item Mức năng lượng cao nhất là $3s^1$, điều này gợi ý nguyên tử có một electron duy nhất trong phân lớp $3s$. Đây là trường hợp của nguyên tố natri (Na), với cấu hình electron đầy đủ là:
				\[1s^22s^22p^63s^1.\]
			\end{enumerate}
		}
	\end{bt}
	%%%=========bt_1=========%%%
	\begin{bt}Nguyên tố X được sử dụng rộng rãi trong đời sống: đúc tiền, làm đồ trang sức, làm răng giả,… Muối iodide của X được sử dụng nhằm từ máy tạo ra mưa nhân tạo. Tổng số hạt cơ bản trong nguyên tử nguyên tố X là 155, số hạt mang điện nhiều hơn số hạt không mang điện là 33 hạt. Xác định nguyên tố X.
		\loigiai{
		Gọi $P$, $N$, $E$ lần lượt là tổng số proton, neutron, electron có trong nguyên tố X.
		\\
		Theo đề bài ta có: $P+N+E =155$. Vì $P=E$ nên ta có $2P+N=155$ (1)
		\\
		Mặt khác theo đề bài ta lại có : $2P - N =33 $ (2)
		\\
		Từ (1) và (2) ta có hệ phương trình:
		$\heva{2P+N=155\\2P-N=33}$.$\Rightarrow \heva{P=47\\N =61}$
		\\
		Vậy X là nguyên tố bạc $Ag$.
		}
	\end{bt}
	
	%%%=========bt_2=========%%%
	\begin{bt}Nguyên tử nguyên tố X có tổng số hạt cơ bản là $82$. Số hạt mang điện nhiều hơn số hạt không mang điện là $22$.
		\begin{enumerate}[a)]
			\item Viết kí hiệu nguyên tử của nguyên tố X.
			\item Xác định số lượng các hạt cơ bản trong ion $X^{2+}$ và viết cấu hình electron của ion đó.
		\end{enumerate}
		\loigiai{
			\begin{enumerate}[a)]
				\item Gọi $P$, $N$, $E$ lần lượt là tổng số proton, neutron, electron có trong nguyên tố X.
				\\
				Theo đề bài ta có: $P+N+E =82$. Vì $P=E$ nên ta có $2P+N=82$ (1)
				\\
				Mặt khác theo đề bài ta lại có : $2P - N =22 $ (2)
				\\
				Từ (1) và (2) ta có hệ phương trình:
				$\heva{2P+N=82\\2P-N=22}$.$\Rightarrow \heva{P=26\\N =30}$
				\\
				Vậy X là nguyên tố bạc $26$.
				\item Cấu hình của Fe : $1s^22s^22p^63s^23p^63d^64s^2$
				\\
				Ta có $Fe \xrightarrow Fe^{2+}+2e$
				\\
				So với cấu hình của Fe thì cấu hình của $Fe^{2+}$ mất đi 2 electron ở lớp ngoài cùng 4s nên cấu hình của $Fe^{2+}$ là:
				$1s^22s^22p^63s^23p^63d^6 $
			\end{enumerate}
		}
	\end{bt}
	
	%%%=========bt_3=========%%%
	\begin{bt}Trong tự nhiên, hợp chất X tồn tại ở dạng quặng có công thức $ABY_2$. X được khai thác và sử dụng nhiều trong luyện kim hoặc sản xuất acid. Trong phân tử X, nguyên tử của hai nguyên tố A và B đều có phân lớp ngoài cùng là 4s, các ion $A^{2+}$, $B^{2+}$ có số electron lớp ngoài cùng lần lượt là 17 và 14. Tổng số proton trong X là 87.
		\begin{enumerate}[a)]
			\item Viết cấu hình electron nguyên tử của A và B.
			\item Xác định X.
		\end{enumerate}
		\loigiai{
			\begin{enumerate}
				\item  Cấu hình electron của A và B có dạng:
				
				$[\mathrm{Ne}] 3 \mathrm{~s}^2 3 \mathrm{p}^6 3 \mathrm{~d}^{\mathrm{x}} 4 \mathrm{~s}^{\mathrm{y}}(0 \leq \mathrm{x} \leq 10 ; 1 \leq \mathrm{y} \leq 2)$.
				
				- Nếu $\mathrm{y}=1$ thì cấu hình của $\mathrm{A}^{2+}$ là : $[\mathrm{Ne}] 3 \mathrm{~s}^2 3 \mathrm{p}^6 3 \mathrm{~d}^{\mathrm{x}-1}$
				
				Khi đó có : $2+6+\mathrm{x}-1=17 \Rightarrow \mathrm{x}=10$.
				Cấu hình electron của $A$ là: $[\mathrm{Ar}] 3 \mathrm{~d}^{10} 4 \mathrm{~s}^1$. $A$ là $_{29}Cu$ .
				
				- Nếu y $=2$ thì cấu hình của $\mathrm{A}^{2+}$ là : $[\mathrm{Ne}] 3 \mathrm{~s}^2 3 \mathrm{p}^6 3 \mathrm{~d}^{\mathrm{x}}$.
				
				Khi đó có : $2+6+\mathrm{x}=17 \Rightarrow \mathrm{x}=9$
				Cấu hình electron của A là: $[\mathrm{Ar}] 3 \mathrm{~d}^9 4 \mathrm{~s}^2$ (không bền vững).
				Xét tương tự với B :
				
				+ Nếu $\mathrm{y}=1$ thì cấu hình electron của B là $[\mathrm{Ar}] 3 \mathrm{~d}^7 4 \mathrm{~s}^1$ (không hợp lí).
				
				+ Nếu y $=2$ thì cấu hình electron của B là $[\mathrm{Ar}] 3 \mathrm{~d}^6 4 \mathrm{~s}^2$. B là ${ }_{26} \mathrm{Fe}$.
				
				\item  Số proton trong $\mathrm{Y}=\dfrac{87-26-29}{2}=16$. Y là $16$ (S) .
			\end{enumerate}
			
			Quặng X có công thức là $\mathrm{CuFeS}_2$.
		}
	\end{bt}
	\Closesolutionfile{ansbt}
	\Closesolutionfile{ansbth}
	%\bangdapanSA{AnsBT-BTTHC01.tex}
\end{document}