\documentclass[Main.tex]{subfiles}
\usepackage{adjustbox}
\usepackage{paracol}
\columnratio{0.65}
\globalcounter{kp}
\setlength{\columnseprule}{0pt}
\def\gachngang{\resizebox{0.3cm}{!}{\_}}
\NewDocumentCommand{\Rightpicture}{O{0.45}O{4pt}mm}{%
	\adjustbox{valign=t}{
		\begin{minipage}[htp!]{#1\linewidth}
		\vspace*{#2}
			#3
		\end{minipage}
	}
	\hfill
	\adjustbox{valign=t}{
		\begin{minipage}[htp!]{0.98\linewidth-#1\linewidth}
			#4
		\end{minipage}
	}
}

\AtBeginEnvironment{ex}{\renewcommand{\dotEX}{}}

\newenvironment{vidu}[1][\mycolor]{
	\begin{hopvidu}[#1]
	\par\noindent\indam[#1]{Ví dụ:}
	}{\end{hopvidu}}
\begin{document}
%%%Lệnh hiển thị mục lục%%%
%\setcounter{tocdepth}{1}
%\tableofcontents
%%%Lệnh hiển thị nội dung đến cấp mấy %%%
\setcounter{secnumdepth}{4} % Hiển thị đến \paragraph
\titlespacing*{\subsubsection}{0cm}{0pt}{0pt}
\titlespacing*{\paragraph}{0cm}{0pt}{0pt}
\phan{Trắc nghiệm 1 phương án đúng}
\Opensolutionfile{ans}[Ans/Ans_EX_C1B3]
\Opensolutionfile{ansbook}[Ansbook/DapanTFChuong]
\hienthiloigiaiex
\hienthiloigiaibt
%%%=============EX_1=============%%%
\begin{ex}
	Nguyên tố hóa học là những nguyên tử có cùng
	\choice
	{số khối}
	{số neutron}
	{\True số proton}
	{số electron}
	\loigiai{Nguyên tố hóa học là tập hợp những nguyên tử  có cùng số proton trong hạt nhân}
\end{ex}
%%%=============EX_2=============%%%
\begin{ex}
	Kí hiệu nguyên tử biểu thị đầy đủ dặc trưng cho một nguyên tử của một nguyên tố hóa học vì nó cho biết
	\choice
	{số khối A}
	{nguyên tử khối của nguyên tử}
	{số hiệu nguyên tử Z}
	{\True số khối A và số hiệu nguyên tử Z}
	\loigiai{Kí hiệu nguyên tử có dạng ${}^{A}_{Z}X$ trong đó X là kí hiệu nguyen tố, A là số khối , Z là số proton }
\end{ex}
%%%=============EX_3=============%%%
\begin{ex}
	Số nơtron trong nguyên tử ${}^{39}_{19}K$ là
	\choice
	{\True$20$}
	{$39$}
	{$19$}
	{$58$}
	\loigiai{ Ta có số neutron được tính theo công thức  $N=A-Z =39-19=20$}
\end{ex}
%%%=============EX_4=============%%%
\begin{ex}
	Trong dãy ký hiệu các nguyên tử sau, dãy nào chỉ cùng một nguyên tố hóa học?
	\choice
	{\True ${}^{12}_{\phantom{0}6}X$, ${}^{13}_{\phantom{0}6}Y$}
	{${}^{18}_{10}Z$, ${}^{14}_{\phantom{0}7}T$}
	{${}^{56}_{26}G$, ${}^{31}_{15}H$}
	{${}^{40}_{20}G$, ${}^{27}_{13}H$}
	\loigiai{Các kí hiệu nguyên tử của cùng một nguyên tố hóa học có cùng số Z (chỉ số góc dưới bên trái kí hiệu nguyên tố)}
\end{ex}
%%%=============EX_5=============%%%
\begin{ex}
	Clo có hai đồng vị là ${}^{35}_{17}Cl$ và ${}^{37}_{17}Cl$. Cho biết khối lượng nguyên tử trung bình của clo là $35{,}5$. Phần trăm số nguyên tử của đồng vị ${}^{37}_{17}Cl$ trong hỗn hợp là
	\choice
	{$75\%$}
	{$40\%$}
	{$60\%$}
	{\True$25\%$}
	\loigiai{
		Gọi phần trăm số nguyên tử của đồng vị ${}^{37}_{17}Cl$ trong hỗn hợp là $x \%$\\
		Ta có
		\begin{eqnarray*}
			& \overset{\_}{A}_{Cl} & = 35{,}5\\
			\Leftrightarrow &\dfrac{37 \cdot x + 35 \cdot (100-x)}{100} & = 35{,}5\\
			\Leftrightarrow & x & = 25
		\end{eqnarray*}
		Vậy phần trăm số nguyên tử của đồng vị ${}^{37}_{17}Cl$ là $25\%$
	}
\end{ex}
%%%%=============EX_6=============%%%
%\begin{ex}
%	Trong tự nhiên, nguyên tố Clo có 2 đồng vị bền là ${}^{35}_{17}Cl$ và ${}^{37}_{17}Cl$, trong đó đồng vị ${}^{35}_{17}Cl$ chiếm $75{,}00\%$ về số nguyên tử. Phần trăm khối lượng của ${}^{37}_{17}Cl$ trong $CaCl_2$ là
%	\choice
%	{$15{,}99\%$}
%	{$15{,}77\%$}
%	{\True $16{,}67\%$}
%	{$47{,}97\%$}
%	\loigiai{
%		Ta có $\% ^{35}\mathrm{Cl} + \% ^{37}\mathrm{Cl} =100 \%  \Rightarrow \% ^{37}\mathrm{Cl}=100\% - 75\% =25\%$
%		\\
%		Nguyên tử khối trung bình của Cl là: $\overset{\_}{A}_{Cl}= \dfrac{35\cdot75+37\cdot25}{100}=35{,}5$
%		\\
%		Gọi số mol của $CaCl_2$ là $1$ mol $\Rightarrow n _{Cl} =1\cdot2=2$ mol
%		\\
%		Do đó: $n_{^{37}Cl} =2\cdot0{,}25=0{,}5$ mol
%		\begin{eqnarray*}
%			\Rightarrow \% m_{^{37}Cl} &=&\dfrac{m_{^{37}Cl}}{m_{CaCl_2}}\cdot100\% \\
%			&=&\dfrac{37 \cdot 0{,}5}{(40+35{,}5\cdot2)}\cdot100\% = 16{,}67
%		\end{eqnarray*}
%		Vậy phần trăm khối lượng của ${}^{37}_{17}Cl$ trong $CaCl_2$ là $16{,}67 \%$
%	}
%\end{ex}
%%%%=============EX_7=============%%%
\begin{ex}
	Hỗn hợp 2 đồng vị bền của một nguyên tố có nguyên tử khối trung bình là $40{,}08$. Hai đồng vị này có số nơtron hơn kém nhau hai hạt. Đồng vị có số khối lớn hơn chiếm $4\%$ về số nguyên tử. Số khối lớn là
	\choice
	{$40$}
	{\True$42$}
	{$41$}
	{$43$}
	\loigiai{
		Giả sử nguyên tố  có 2 đồng vị bền là $_{\phantom{Z}Z}^{A_1}X$ và $_{\phantom{Z}Z}^{A_2}X$ với ($A_2>A_1$).
		\\
		Ta có {\renewcommand{\arraystretch}{0.85}
			$\left.
			\begin{aligned}
				A_1=Z+N_1\\
				A_2=Z+N_2
			\end{aligned}
			\right\}$}
		$\Rightarrow A_2-A_1=N_2-N_1 =2$ hay $-A_1+A_2=2$ (1)
		\\
		Mặt khác theo đề bài ta có \[
		\begin{aligned}
			&&\overset{\_}{A}_{X}& = 40{,}08&\\
			&\Leftrightarrow &\dfrac{ A_1\times 96 + A_2 \times 4}{100}& = 40{,}08&\\
			&\Leftrightarrow &96A_1+4A_2& = 4008 & \quad (2)
		\end{aligned}
		\]
		Từ (1) và (2) ta có hệ phương trình $\heva{&-A_1+A_2&=&\phantom{x}2\\&96A_1+4A_2&=&\phantom{x}4008}$.
		Giải hệ ta được $\heva{&A_1=40\\&A_2=42}$
		\\
		Vậy Số khối lớn hơn là $42$.
	}
\end{ex}
%%%=============EX_8=============%%%
\begin{ex}
	Một nguyên tố X có 3 đồng vị là $X_1$, $X_2$ và $X_3$. Đồng vị $X_1$ chiếm $92{,}23\%$, $X_2$ chiếm $4{,}67\%$ và $X_3$ chiếm $3{,}10\%$ số nguyên tử. Tổng số khối của 3 đồng vị bằng $87$. Số nơtron trong đồng vị $X_2$ nhiều hơn số nơtron trong đồng vị $X_1$ là một hạt. Nguyên tử khối trung bình của X là $28{,}0855$ và trong $X_1$ có số nơtron bằng số proton. Số nơtron của $X_2$ là
	\choice
	{$14$}
	{$15$}
	{$13$}
	{$16$}
	\loigiai{
		Gọi ba đồng vị của X lần lượt là $_{\phantom{z}Z}^{A_1}X_1$, $_{\phantom{z}Z}^{A_2}X_2$ và $_{\phantom{z}Z}^{A_3}X_3$
		\\
		Theo giả thiết ta có $A_1+A_2+A_3=87$ (1)
		\\
		Vì $N_2-N_1=1$ $\Rightarrow$ $A_2-A_1=1$ (2)
		\\
		Lại có $\overset{\_}{A}_X=28{,}0855 \Leftrightarrow 0{,}9223A_1+0{,}0467A_2+0{,}031A_3=28{,}0855$ (3).
		\\
		Từ (1), (2) và (3) ta có hệ phương trình $\heva{&A_1+A_2+A_3=87\\&-A_1+A_2=1\\&0{,}9223A_1+0{,}0467A_2+0{,}031A_3=28{,}0855}$.
		\\
		Giải hệ ta được $\heva{&A_1=28\\&A_2=29\\&A_3=30}$
		$\Rightarrow N_1 =Z =\dfrac{A_1}{2}=\dfrac{28}{2}=14\Rightarrow N_2 =15$
		\\
		Vậy số neutron của $X_2 = 15 $ (hạt)
	}
\end{ex}
%%%=============EX_9=============%%%
\begin{ex}
	Trong tự nhiên, Clo có hai đồng vị bền là ${}^{37}_{17}Cl$ chiếm $24{,}23\%$ tổng số nguyên tử, vậy còn lại là ${}^{35}_{17}Cl$. Thành phần phần trăm theo khối lượng của ${}^{35}_{17}Cl$ trong $HClO_4$ là
	\choice
	{$8{,}43\%$}
	{\True $8{,}79\%$}
	{$8{,}92\%$}
	{$8{,}5\%$}
	\loigiai{
		Ta có $\% ^{37}\mathrm{Cl} = 24{,}23\% \Rightarrow \% ^{35}\mathrm{Cl} = 100\% - 24{,}23\% = 75{,}77\%$
		\\
		Nguyên tử khối trung bình của Cl là:
		\begin{eqnarray*}
			\overset{\_}{A}_{Cl} &=& \dfrac{35 \cdot 75{,}77 + 37 \cdot 24{,}23}{100} \\
			&=& \dfrac{2651{,}95 + 896{,}51}{100} \\
			&=& 35{,}4846
		\end{eqnarray*}
		Gọi số mol của $HClO_4$ là $1$ mol $\Rightarrow n_{Cl} = 1$ mol
		\\
		Do đó: $n_{^{35}Cl} = 1 \cdot 0{,}7577 = 0{,}7577$ mol
		\\
		Khối lượng phân tử của $HClO_4$:
		\begin{eqnarray*}
			M_{HClO_4} &=& 1 + 35{,}4846 + 4 \cdot 16 \\
			&=& 100{,}4846 \text{ g/mol}
		\end{eqnarray*}
		\\
		Tính phần trăm khối lượng của ${}^{35}_{17}Cl$ trong $HClO_4$:
		\begin{eqnarray*}
			\% m_{^{35}Cl} &=& \dfrac{m_{^{35}Cl}}{m_{HClO_4}} \cdot 100\% \\
			&=& \dfrac{35 \cdot 0{,}7577}{100{,}4846} \cdot 100\% \\
			&=& 8{,}79\%
		\end{eqnarray*}
		\\
		Vậy thành phần phần trăm theo khối lượng của ${}^{35}_{17}Cl$ trong $HClO_4$ là $8{,}79\%$
	}
\end{ex}
%%%=============EX_10=============%%%
\begin{ex}
	Nguyên tố $Cu$ có hai đồng vị, nguyên tử khối trung bình là $63{,}62$.Một trong hai đồng vị là $^{63}Cu$ (chiếm $69{,}17\%$).Nguyên tử khối của đồng vị thứ hai là
	\choice
	{$66$}
	{$64$}
	{$67$}
	{\True$65$}
	\loigiai{Gọi nguyên tử khối của đồng vị thứ hai là $x$.
		Ta có phương trình:
		\[
		\begin{aligned}
			&&63 \cdot 0{,}6917 + x\cdot(1-0{,}6917)= 63{,}62 &\\
			\Leftrightarrow	&&x = 65 &
		\end{aligned}
		\]
		Vậy nguyên tử khối của đồng vị thứ hai là $65$ (amu).
	}
\end{ex}
%%%=============EX_11=============%%%
\begin{ex}
	Nguyên tố $Cl$ có hai đồng vị, nguyên tử khối trung bình là $35{,}48$.Một trong hai đồng vị là $^{35}Cl$ (chiếm $75{,}78\%$).Nguyên tử khối của đồng vị thứ hai là
	\choice
	{$38$}
	{$36$}
	{$39$}
	{\True$37$}
	\loigiai{Gọi nguyên tử khối của đồng vị thứ hai là $x$.
		Ta có phương trình:
		\[
		\begin{aligned}
			&&35 \cdot 0{,}7578 + x\cdot(1-0{,}7578)= 35{,}48 &\\
			\Leftrightarrow	&&x = 37 &
		\end{aligned}
		\]
		Vậy nguyên tử khối của đồng vị thứ hai là $37$ (amu).
	}
\end{ex}
%%%=============EX_12=============%%%
\begin{ex}
	Nguyên tố $K$ có ba đồng vị, nguyên tử khối trung bình là $39{,}13$.Hai trong ba đồng vị là $^{39}K$ (chiếm $93{,}2581\%$) và $^{40}K$ (chiếm $0{,}0117\%$).Nguyên tử khối của đồng vị thứ ba là
	\choice
	{$40$}
	{$43$}
	{\True$41$}
	{$42$}
	\loigiai{Gọi nguyên tử khối của đồng vị thứ ba là $x$.
		Ta có phương trình:
		\[
		\begin{aligned}
			&&39 \cdot 0{,}932581 +40 \cdot 0{,}000117 + x\cdot(1-0{,}932581-0{,}000117)= 39{,}13 &\\
			\Leftrightarrow	&&x = 41 &
		\end{aligned}
		\]
		Vậy nguyên tử khối của đồng vị thứ ba là $41$ (amu).
	}
\end{ex}
%%%%=============EX_13=============%%%
%\begin{ex}
%	Carbon có hai đồng vị bền là $_6^{12} \mathrm{C}$ và $_6^{13} \mathrm{C}$. Oxygen có ba đồng vị bền là $_8^{16} O,_8^{17} O$ và $_8^{18} O$. Số hợp chất $CO_2$ tạo bởi các đồng vị trên là
%	\choice
%	{$9$}
%	{\True$12$}
%	{$18$}
%	{$27$}
%	\loigiai{Xét hợp chất $CO_2$ có dạng \canhdong{\chemfig[atom sep=3em]{O_a-[:30]C-[:-30]O_b}}
%		\\
%		Để tạo ra một phân tử $CO_2$ cần 1 nguyên tử $C$ và 2 nguyên tử $O$
%		\begin{itemize}
%			\item Chọn 1 nguyên tử C trong 2 đồng vị C có 2 cách chọn
%			\item Chọn 2 nguyên tử O
%			\begin{itemize}
%				\item TH1 : $O_a \equiv O_b$ có 3 cách chọn
%				\\
%				$\Rightarrow$ Số loại phân tử $CO_2$ được tạo ra là $2\times3 =6$ (phân tử)
%				\item TH2 :$O_a \neq O_b$ có 3 cách chọn
%				\\
%				$\Rightarrow$ Số loại phân tử $CO_2$ được tạo ra là $2\times3 =6$ (phân tử)
%			\end{itemize}
%			
%		\end{itemize}
%		Vậy có tất cả  $6+6=12$ (phân tử)
%	}
%\end{ex}
%%%%=============EX_14=============%%%
\begin{ex}
	Nitrogen có hai đồng vị bền là $_7^{14} \mathrm{N}$ và $_7^{15} \mathrm{N}$. Hydrogen có ba đồng vị bền là $_1^{1} \mathrm{H}$ , $_1^{2} \mathrm{H}$ và $_1^{3} \mathrm{H}$. Số hợp chất $NH_3$ tạo bởi các đồng vị trên là
	\choice
	{$12$}
	{\True $14$}
	{$24$}
	{$32$}
	\loigiai{
		Xét hợp chất $NH_3$ có dạng \chemfig[atom sep=4em,bond offset=4pt]{H_a-[:35]N(<[:-60]H_c)<:[:-20]H_b}
		\\
		Để tạo ra một phân tử $NH_3$ cần 1 nguyên tử $N$ và 3 nguyên tử $H$
		\begin{itemize}
			\item Chọn 1 nguyên tử N trong 2 đồng vị N có 2 cách chọn
			\item Chọn 3 nguyên tử H
			\begin{itemize}
				\item TH1 : $H_a \equiv H_b \equiv H_c $ có 3 cách chọn
				\\
				$\Rightarrow$ Số loại phân tử $NH_3$ được tạo ra là $2\times3 =6$ (phân tử)
				\item TH2 :$H_a \equiv H_b \neq H_c $ có 3 cách chọn
				\\
				$\Rightarrow$ Số loại phân tử $NH_3$ được tạo ra là $2\times3 =6$ (phân tử)
				\item TH3 :$H_a \neq H_b \neq H_c $ có 1 cách chọn
				\\
				$\Rightarrow$ Số loại phân tử $NH_3$ được tạo ra là $2\times1 =2$ (phân tử)
			\end{itemize}
			
		\end{itemize}
		Vậy có tất cả  $6+6+2=14$ (phân tử)
	}
\end{ex}
\phan{Trắc nghiệm đúng sai}

%%%=============EX_15=============%%%
\begin{ex}
	Nguyên tố Mg có KHNT là ${}^{24}_{12}Mg$.Hãy cho biết tính đúng, sai của các phát biểu sau:
	\choiceTF[t]
	{\True Magnesium có 12 proton và 12 electron}
	{\True Magnesium có số khối là 24}
	{Magnesium có nguyên tử khối là 24 amu}
	{Ion $Mg^{2+}$ có 10 electron trong hạt nhân}
	\loigiai{
		\begin{itemchoice}
			\itemch Đúng. Trong một nguyên tử trung hòa tổng số proton = tổng số electron
			\itemch Đúng. Theo kí hiệu nguyên tố ${}^A_ZX$ trong đó $A$ là số khối, $Z$ là số hiệu nguyên tử
			\itemch Sai. Nguyên tử khối  không có đơn vị
			\itemch Sai. Electron nằm ở vỏ nguyên tử, proton và neutron  nằm trong hạt nhân. Nguyên tử trung hòa khi mất đi electron ở lớp vỏ ngoài cùng sẽ tạo thành ion dương (cation).
		\end{itemchoice}
	}
\end{ex}
\Closesolutionfile{ans}
\Closesolutionfile{ansbook}
\phan{Bài tập tự luận}
%%%=============BT_1=============%%%
\begin{bt}
	Chromium (Cr), có khối lượng các đồng vị và độ phổ biến được cho ở bảng sau (Bảng \ref{tab:dvCr}). Hãy tính nguyên tử khối trung bình của Chromium
	\begin{center}
		\begin{tabular}{*{3}{C{0.26\linewidth}}}
			\hline\rowcolor{\maunhan!5}
			\begin{tabular}{l}
				\textsf{\textbf{Số khối}}
			\end{tabular}
			&
			\begin{tabular}{l}
				\textsf{\textbf{Khối lượng đồng vị}}
			\end{tabular}
			&
			\begin{tabular}{l}
				\textsf{\textbf{Độ phổ biến}}
			\end{tabular}
			\\
			\hline
			50 & 49{,}9461 & 0{,}0435 \\
			52 & 51{,}9405 & 0{,}8379 \\
			53 & 52{,}9407 & 0{,}0950 \\
			54 & 53{,}9389 & 0{,}0236 \\
			\hline
		\end{tabular}
		\captionof{table}{Các đồng vị phổ biến của 	Chromium } \label{tab:dvCr}
	\end{center}
	\loigiai{Nguyên tử khối của Cr là:
		\[
		\begin{aligned}
			\overset{\_}{A}_{Cr}&=49{,}9461\cdot0{,}0435+49{,}9461\cdot0{,}0435+52{,}9407\cdot0{,}0950+53{,}9389\cdot 0{,}0236 \\
			&= 51{,}9959 \approx 52
		\end{aligned}
		\]
		Vậy khối lượng nguyên tử trung bình của Cr là 52 (amu)
	}
\end{bt}
%%%=============BT_2=============%%%
\begin{bt}
	Hoàn thành những thông tin còn thiếu trong bảng sau:\\
	\begin{tabular}{|c|c|c|c|}
		\hline
		\rowcolor{\maunhan!8}
		Nguyên tử & Kí hiệu nguyên tử & Số hiệu nguyên tử & Số khối \\
		\hline
		Zinc & $_{30}^{65} \mathrm{Zn}$ & $?$ & $?$ \\
		\hline
		Carbon & $?$ & $6$ & $14$ \\
		\hline
		Lead & $_{82}^? \mathrm{Pb}$ & $?$ & $207$ \\
		\hline
		Oxygen & $_{8}^{16} \mathrm{O}$ & $?$ & $?$ \\
		\hline
		Copper & $?$ & $29$ & $64$ \\
		\hline
		Iron & $_{26}^? \mathrm{Fe}$ & $?$ & $56$ \\
		\hline
	\end{tabular}
	\loigiai{
		\begin{tabular}{|c|c|c|c|}
			\hline
			\rowcolor{\maunhan!8}
			Nguyên tử & Kí hiệu nguyên tử & Số hiệu nguyên tử & Số khối \\
			\hline
			Zinc & $_{30}^{65} \mathrm{Zn}$ & $30$ & $65$ \\
			\hline
			Carbon & $_{6}^{14} \mathrm{C}$ & $6$ & $14$ \\
			\hline
			Lead & $_{82}^{207} \mathrm{Pb}$ & $82$ & $207$ \\
			\hline
			Oxygen & $_{8}^{16} \mathrm{O}$ & $8$ & $16$ \\
			\hline
			Copper & $_{29}^{64} \mathrm{Cu}$ & $29$ & $64$ \\
			\hline
			Iron & $_{26}^{56} \mathrm{Fe}$ & $26$ & $56$ \\
			\hline
		\end{tabular}
	}
\end{bt}
%%%=============BT_3=============%%%
\begin{bt}
	Hoàn thành những thông tin còn thiếu trong bảng sau:
	\\
	\begin{tabular}{|c|c|c|c|}
		\hline
		Nguyên tử & Ki hiệu nguyên tử & Số hiệu nguyên tử & Số khối \\
		\hline
		Europium & $_{63}^{152} \mathrm{Eu}$ & $?$ & $?$ \\
		\hline
		Silver & $?$ & 47 & 108 \\
		\hline
		Tellurium & $_{52}^? \mathrm{Te}$ & $?$ & 128 \\
		\hline
	\end{tabular}
	\loigiai{
		\begin{tabular}{|c|c|c|c|}
			\hline
			Nguyên tử & Ki hiệu nguyên tử & Số hiệu nguyên tử & Số khối \\
			\hline
			Europium & $_{63}^{152} \mathrm{Eu}$ & $63$ & $152$ \\
			\hline
			Silver & $_{47}^{108} \mathrm{Ag}$ & $47$ & $108$ \\
			\hline
			Tellurium & $_{52}^{128} \mathrm{Te}$ & $52$ & $128$ \\
			\hline
		\end{tabular}
	}
\end{bt}
%%%=============BT_4=============%%%
\begin{bt}
	Hoàn thành những thông tin còn thiếu trong bảng sau:
	\\
	\begin{tabular}{|c|c|c|c|}
		\hline
		\rowcolor{\maunhan!8}
		Nguyên tử & Kí hiệu nguyên tử & Số hiệu nguyên tử & Số khối \\
		\hline
		Iodine & $_{\phantom{x}53}^{127} \mathrm{I}$ & $?$ & $?$ \\
		\hline
		Gold & $?$ & $79$ & $197$ \\
		\hline
		Platinum & $_{78}^? \mathrm{Pt}$ & $?$ & $195$ \\
		\hline
		Sulfur & $_{16}^{32} \mathrm{S}$ & $?$ & $?$ \\
		\hline
		Tin & $?$ & $50$ & $119$ \\
		\hline
		Barium & $_{56}^? \mathrm{Ba}$ & $?$ & $137$ \\
		\hline
	\end{tabular}
	\loigiai{
		\begin{tabular}{|c|c|c|c|}
			\hline
			\rowcolor{\maunhan!8}
			Nguyên tử & Kí hiệu nguyên tử & Số hiệu nguyên tử & Số khối \\
			\hline
			Iodine & $_{\phantom{x}53}^{127} \mathrm{I}$ & $53$ & $127$ \\
			\hline
			Gold & $_{\phantom{x}79}^{197}\mathrm{Au}$ & $79$ & $197$ \\
			\hline
			Platinum & $_{78}^{195} \mathrm{Pt}$ & $78$ & $195$ \\
			\hline
			Sulfur & $_{16}^{32} \mathrm{S}$ & $16$ & $32$ \\
			\hline
			Tin & $_{50}^{119} \mathrm{Sn}$ & $50$ & $119$ \\
			\hline
			Barium & $_{56}^{137} \mathrm{Ba}$ & $56$ & $137$ \\
			\hline
		\end{tabular}
	}
\end{bt}
%%%=============BT_5=============%%%
\begin{bt}
	Oxide của kim loại $M$ $(M_2O_3)$ được ứng dụng rộng rãi trong công nghiệp, đặc biệt trong sản xuất thép không gỉ và làm chất xúc tác. Trong phòng thí nghiệm, $M_2O_3$ thường có màu xanh lục đến xám đen và ít tan trong nước. Tổng số hạt cơ bản trong phân tử $X$ có công thức $M_2O_3$ là 280, trong đó số hạt mang điện nhiều hơn số hạt không mang điện là 80. Xác định công thức phân tử của $M_2O_3$.
	\loigiai{
		\begin{enumerate}
			\item Gọi số proton của $M$ là $x$
			\item Tổng số hạt cơ bản $= 280$
			\begin{itemize}
				\item Số hạt mang điện $= (2x + 3 \times 8) \times 2 = 4x + 48$ (proton và electron)
				\item Số hạt không mang điện $= 280 - (4x + 48) = 232 - 4x$ (neutron)
			\end{itemize}
			\item Số hạt mang điện nhiều hơn số hạt không mang điện là 80
			\begin{align*}
				(4x + 48) - (232 - 4x) &= 80 \\
				8x - 184 &= 80 \\
				8x &= 264 \\
				x &= 24
			\end{align*}
			\item Vậy $M$ có số proton là 24, đó là nguyên tố Cr (crom)
			\item Công thức phân tử là $Cr_2O_3$
		\end{enumerate}
	}
\end{bt}
%%%=============BT_6=============%%%
\begin{bt}
	Hợp chất $AB_3$ là một chất quan trọng trong công nghiệp hóa chất, được sử dụng làm chất xúc tác trong nhiều phản ứng hữu cơ. Mỗi phân tử $AB_3$ có tổng số hạt proton, neutron và electron bằng 224. Trong đó, số hạt mang điện nhiều hơn số hạt không mang điện là 64. Số hạt mang điện của $A$ nhiều hơn tổng số hạt mang điện của ba nguyên tử $B$ là 4. Hãy xác định kí hiệu hóa học của $A$ và $B$.
	\loigiai{
		\begin{enumerate}
			\item Gọi số proton của $A$ là $x$, của $B$ là $y$
			\item Tổng số hạt $= 224$
			\begin{itemize}
				\item Số hạt mang điện $= (x + 3y) \times 2 = 2x + 6y$ (proton và electron)
				\item Số hạt không mang điện $= 224 - (2x + 6y)$ (neutron)
			\end{itemize}
			\item Số hạt mang điện nhiều hơn số hạt không mang điện là 64
			\begin{align*}
				(2x + 6y) - (224 - 2x - 6y) &= 64 \\
				4x + 12y - 224 &= 64 \\
				4x + 12y &= 288
			\end{align*}
			\item Số hạt mang điện của $A$ nhiều hơn tổng số hạt mang điện của ba $B$ là 4
			\[2x - 6y = 4\]
			\item Giải hệ phương trình:
			\begin{align*}
				4x + 12y &= 288 \\
				2x - 6y &= 4 \\
				\Rightarrow x &= 33, y = 9
			\end{align*}
			\item Vậy $A$ có số proton là 33 (As - Asen), $B$ có số proton là 9 (F - Flo)
			\item Công thức hóa học là $AsF_3$
		\end{enumerate}
	}
\end{bt}
\end{document}

