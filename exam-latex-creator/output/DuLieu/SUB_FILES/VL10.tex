\documentclass[Main.tex]{subfiles}
\begin{document}
	\Noibat[][\myfont[17]{qag}][\faBank]{BÀI TẬP ĐỘ DỊCH CHUYỂN QUÃNG ĐƯỜNG}
	\sodongke[5]{ex}
	\vspace{3mm}
	%%%==============Cau_EX1==============%%%
	\begin{ex}
		[NB] Chất điểm là
		\choice
		{một vật có kích thước bất kỳ}
		{một vật có hình học}
		{một điểm khi ta nghiên cứu chuyển động của nó trong một khoảng rất nhỏ}
		{một vật có kích thước rất nhỏ so với độ dài đường đi}
		\loigiai{}
	\end{ex}
	%%%==============HetCau_EX1==============%%%
	
	%%%==============Cau_EX2==============%%%
	\begin{ex}
		[NB] Chọn phát biểu đúng:
		\choice
		{Vecto độ dịch chuyển thay đổi phương liên tục khi vật chuyển động}
		{Vecto độ dịch chuyển có độ lớn luôn bằng quãng đường đi được của chất điểm}
		{Trong chuyển động thẳng độ dịch chuyển bằng độ biến thiên tọa độ}
		{Độ dịch chuyển bằng giá trị độ dời}
		\loigiai{}
	\end{ex}
	%%%==============HetCau_EX2==============%%%
	
	%%%==============Cau_EX3==============%%%
	\begin{ex}
		[NB] Chọn phát biểu sai: Độ dịch chuyển
		\choice
		{là đại lượng vecto}
		{luôn có độ lớn bằng quãng đường đi được.a}
		{phụ thuộc vào điểm đầu và điểm cuối của vật chuyển động, không phụ thuộc vào hình dạng quỹ đạo}
		{có đơn vị là mét}
		\loigiai{}
	\end{ex}
	%%%==============HetCau_EX3==============%%%
	
	%%%==============Cau_EX4==============%%%
	\begin{ex}
		[NB] Độ dịch chuyển và quãng đường đi được của vật có độ lớn bằng nhau khi vật
		\choice
		{chuyển động tròn}
		{chuyển động thẳng và không đổi chiều}
		{chuyển động thẳng và chỉ đổi chiều 1 lần}
		{chuyển động thẳng và chỉ đổi chiều 2 lần}
		\loigiai{}
	\end{ex}
	%%%==============HetCau_EX4==============%%%
	
	%%%==============Cau_EX5==============%%%
	\begin{ex}
		[NB] Để xác định thời điểm của vật chuyển động
		\choice
		{ta dùng hệ trục tọa độ có gốc là vị trí của vật mốc}
		{ta chọn mốc thời gian và đồng hồ đo thời gian}
		{ta cần xác định vị trí của vật chuyển động tại các thời điểm}
		{ta chỉ cần chọn vật làm mốc}
		\loigiai{}
	\end{ex}
	%%%==============HetCau_EX5==============%%%
	
	%%%==============Cau_EX6==============%%%
	\begin{ex}
		[NB] Để xác định vị trí của vật chuyển động tại các thời điểm
		\choice
		{ta dùng hệ trục tọa độ có gốc là vị trí của vật mốc}
		{ta chọn mốc thời gian và đồng hồ đo thời gian}
		{ta chỉ cần chọn vật làm mốc}
		{ta dùng hệ quy chiếu: là hệ tọa độ kết hợp với thời gian và đồng hồ đo thời gian}
		\loigiai{}
	\end{ex}
	%%%==============HetCau_EX6==============%%%
	
	%%%==============Cau_EX7==============%%%
	\begin{ex}
		[NB] Trường hợp nào sau đây không thể coi là chất điểm?
		\choice
		{Ô tô chuyển động từ Hà Nội đi Hà Nam}
		{Hà Nội trên bản đồ Việt Nam}
		{Một học sinh đi chuyển từ nhà đến trường}
		{Một phát biểu di từ cuối lớp đến đầu lớp}
		\loigiai{}
	\end{ex}
	%%%==============HetCau_EX7==============%%%
	
	%%%==============Cau_EX8==============%%%
	\begin{ex}
		[NB] Phát sinh nào sau đây là chính xác nhất?
		\choice
		{Chuyển động cơ học là sự thay đổi khoảng cách của vật chuyển động so với vật mốc}
		{Quỹ đạo là đường thẳng mà vật chuyển động vạch ra trong không gian}
		{Chuyển động cơ học là sự thay đổi vị trí của vật so với vật mốc}
		{Khi khoảng cách từ vật đến vật làm mốc là không đổi thì vật đứng yên}
		\loigiai{}
	\end{ex}
	%%%==============HetCau_EX8==============%%%
	
	%%%==============Cau_EX9==============%%%
	\begin{ex}
		[NB] Chọn câu sai
		\choice
		{Độ dịch chuyển là vecto nối vị trí đầu và vị trí cuối của chất điểm chuyển động}
		{Độ dịch chuyển có độ lớn bằng quãng đường đi được của chất điểm}
		{Chất điểm di trên một đường thẳng rồi quay về vị trí ban đầu thì có độ dịch chuyển bằng không}
		{Độ dịch chuyển có thể dương hoặc âm}
		\loigiai{}
	\end{ex}
	%%%==============HetCau_EX9==============%%%
	
	%%%==============Cau_EX10==============%%%
	\begin{ex}
		[TH] Kết luận nào sau đây là đúng khi nói về độ dịch chuyển và quãng đường đi được của một vật:
		\choice
		{Độ dịch chuyển và quãng đường đi được đều là đại lượng vô hướng}
		{Độ dịch chuyển là đại lượng vecto còn quãng đường đi được là đại lượng vô hướng}
		{Độ dịch chuyển và quãng đường đi được đều là đại lượng vecto}
		{Độ dịch chuyển và quãng đường đi được đều là đại lượng không âm}
		\loigiai{}
	\end{ex}
	%%%==============HetCau_EX10==============%%%
	
	%%%==============Cau_EX11==============%%%
	\begin{ex}
		[TH] Độ dịch chuyển và quãng đường đi được của vật có độ lớn bằng nhau khi vật
		\choice
		{chuyển động tròn}
		{chuyển động thẳng và không đổi chiều}
		{chuyển động thẳng và chỉ đổi chiều 1 lần}
		{chuyển động thẳng và chỉ đổi chiều 2 lần}
		\loigiai{}
	\end{ex}
	%%%==============HetCau_EX11==============%%%
	
	%%%==============Cau_EX12==============%%%
	\begin{ex}
		[TH] Kết luận nào sau đây là sai khi nói về độ dịch chuyển của một vật:
		\choice
		{Khi vật chuyển động thẳng, không đổi chiều thì độ lớn của độ dịch chuyển và quãng đường đi được bằng nhau (d=s)}
		{Độ dịch chuyển được biểu diễn bằng một mũi tên nối vị trí đầu và vị trí cuối của chuyển động, có chiều nhận giá trị dương, âm hoặc bằng 0}
		{Độ lớn chính bằng khoảng cách giữa vị trí đầu và vị trí cuối. Kí hiệu là d}
		{Khi vật chuyển động thẳng, có đổi chiều thì độ lớn của độ dịch chuyển và quãng đường đi được bằng nhau (d=s)}
		\loigiai{}
	\end{ex}
	%%%==============HetCau_EX12==============%%%
	
	%%%==============Cau_EX13==============%%%
	\begin{ex}
		[TH] Chọn phát biểu đúng:
		\choice
		{Vecto độ dịch chuyển thay đổi phương liên tục khi vật chuyển động}
		{Vecto độ dịch chuyển có độ lớn luôn bằng quãng đường đi được của vật}
		{Trong chuyển động thẳng độ dịch chuyển bằng độ biến thiên tọa độ}
		{Độ dịch chuyển có giá trị luôn dương}
		\loigiai{}
	\end{ex}
	%%%==============HetCau_EX13==============%%%
	
		%%%==============Cau_EX1==============%%%
		\begin{ex}
			[TH] Chọn phát biểu sai:
			\choice
			{Vecto độ dịch chuyển là một vecto nối vị trí đầu và vị trí cuối của một vật chuyển động}
			{Vật đi từ A đến B, từ B đến C rồi từ C về A thì có độ dịch chuyển bằng $AB+BC+CD.$}
			{Vật đi từ A đến B, từ B đến C rồi từ C về A thì có độ dịch chuyển bằng 0}
			{Độ dịch chuyển có thể dương, âm hoặc bằng 0}
			\loigiai{}
		\end{ex}
		%%%==============HetCau_EX1==============%%%
		%%%==============Cau_EX1==============%%%
		\begin{ex}
			[TH] Một vật bắt đầu chuyển động từ điểm O đến điểm A Sau đó chuyển động về điểm B (hình vẽ). Quãng đường và độ dịch chuyển của vật tương ứng bằng
			\choice
			{8m;-8m}
			{2m; 2m}
			{2m;-2m}
			{8m; 2m}
			\loigiai{}
		\end{ex}
		%%%==============HetCau_EX1==============%%%
			
			%%%==============Cau_EX16==============%%%
			\begin{ex}
				[TH] Chọn phát biểu sai khi nói về độ dịch chuyển của sự thay đổi vị trí
				\choice
				{Độ dịch chuyển chỉ cho biết độ dài, không cho biết hướng của sự thay đổi vị trí}
				{Độ dịch chuyển là một đại lượng vecto}
				{Độ dịch chuyển được biểu diễn bằng một mũi tên, nối vị trí đầu và vị trí cuối của chuyển động}
				{Độ dài tỉ lệ với độ lớn của độ dịch chuyển}
				\loigiai{}
			\end{ex}
			%%%==============HetCau_EX16==============%%%
			
			%%%==============Cau_EX17==============%%%
			\begin{ex}
				[TH] Độ dịch chuyển và quãng đường đi được có thể bằng nhau trong trường hợp đặc biệt không phải của mọi vật chuyển động, đặc điểm nào sau đây chỉ là của quãng đường đi được.
				\choice
				{Có điểm ví và chiều xác định}
				{Có phương và mét}
				{Không thể có độ lớn bằng 0}
				{Có thể có độ lớn bằng 0}
				\loigiai{}
			\end{ex}
			%%%==============HetCau_EX17==============%%%
			
			%%%==============Cau_EX18==============%%%
			\begin{ex}
				[TH] Chọn phát biểu sai
				\choice
				{Vecto độ dịch chuyển là một vecto nối vị trí đầu và vị trí cuối của vật chuyển động}
				{Vecto độ dịch chuyển luôn có độ lớn bằng quãng đường đi được của vật}
				{Vật đi từ điểm A đến điểm B, sau đó đến điểm C rồi quay về điểm A thì độ dịch chuyển của vật có độ lớn bằng 0}
				{Độ dịch chuyển có thể có giá trị âm, dương hoặc bằng 0}
				\loigiai{}
			\end{ex}
			%%%==============HetCau_EX18==============%%%
			
			%%%==============Cau_EX19==============%%%
			\begin{ex}
				[VD] Hình vẽ bên dưới mô tả độ dịch chuyển của 3 vật. Chọn câu đúng:
				\choice
				{Vật 1 đi 200 m theo hướng Nam}
				{Vật 2 đi 200 m theo hướng $45^\circ$ Đông – Bắc}
				{Vật 3 đi 30 m theo hướng Đông}
				{Vật 4 đi 100 m theo hướng Đông}
				\loigiai{}
			\end{ex}
			%%%==============HetCau_EX19==============%%%
			
			%%%==============Cau_EX20==============%%%
			\begin{ex}
				[VD] Một người lái ô tô đi thẳng 6 km theo hướng Tây, sau đó rẽ trái đi thẳng theo hướng Nam 4 km rồi quay sang hướng Đông đi 3 km. Quãng đường đi được và độ dịch chuyển của ô tô lần lượt là
				\choice
				{13 km; 5km}
				{13 km; 13 km}
				{7 km; 7 km}
				{4 km; 13km}
				\loigiai{}
			\end{ex}
			%%%==============HetCau_EX20==============%%%
			
			%%%==============Cau_EX21==============%%%
			\begin{ex}
				[VD] Một người bơi ngang từ bờ này sang bờ kia của một dòng sông rộng 50 m có dòng chảy theo hướng Bắc xuống Nam. Do nước sông chảy mạnh nên khi sang đến bờ bên kia thì người đó đã trôi xuôi theo dòng nước 50 m. Độ dịch chuyển của người đó là
				\choice
				{$50$ m}
				{$50\sqrt{2}$ m}
				{$100$ m}
				{$100\sqrt{2}$ m}
				\loigiai{}
			\end{ex}
			%%%==============HetCau_EX21==============%%%
			
			%%%==============Cau_EX22==============%%%
			\begin{ex}
				[VD] Một người đi thang máy từ tầng G xuống tầng hầm cách tầng G 5 m rồi tiếp tục lên tới tầng cao nhất của tòa nhà cách tầng G 50 m. Xác định quãng đường đi được và độ dịch chuyển của người đó
				\choice
				{s=60 m và d=50m}
				{s=60 m và d=60m}
				{s=55 m và d=55m}
				{s=60 m và d=55m}
				\loigiai{}
			\end{ex}
			%%%==============HetCau_EX22==============%%%
			
			%%%==============Cau_EX23==============%%%
			\begin{ex}
				[VDC] Một người bơi từ bờ này sang bờ kia của một con sông rộng 50 m theo hướng vuông góc với bờ sông. Do nước chảy mạnh nên quãng đường người đó phải bơi dài gấp 2 lần chiều rộng của con sông. Hỏi vị trí điểm tới cách điểm đối diện với điểm khởi hành của người bơi là bao nhiêu mét?
				\choice
				{50m}
				{86,6 m}
				{100 m}
				{141,4 m}
				\loigiai{}
			\end{ex}
			%%%==============HetCau_EX23==============%%%
			
			%%%==============Cau_EX24==============%%%
			\begin{ex}
				[VDC] Một người lái xe ô tô đi thẳng 4 km theo hướng Đông, sau đó rẽ Phải đi thẳng theo hướng Nam 6 km rồi quay sang hướng Đông đi 4 km. Xác định quãng đường đi được và độ dịch chuyển của ô tô.
				\choice
				{s=14 m và d=8m}
				{s=10 m và d=14m}
				{s=14 m và d=14m}
				{s=14 m và d=10m}
				\loigiai{}
			\end{ex}
			%%%==============HetCau_EX24==============%%%
			
			%%%==============Cau_EX25==============%%%
			\begin{ex}
				[VDC] Em của An chơi trò chơi tìm kho báu ở ngoài vườn với các bạn của mình. Em của An giấu kho báu của mình là 1 chiếc vòng nhựa vào trong một chiếc giấy rồi viết mật thư tìm kho báu như sau: Bắt đầu từ gốc cây, đi 10 bước về phía bắc, sau đó 4 bước về phía tây, 1,5 bước về phía nam, 5 bước về phía đông và 5 bước về phía bắc là tới chỗ giấu kho báu. Hãy tính quãng đường đi (theo bước) để tìm ra kho báu và vị trí giấu kho báu.
				\choice
				{s=34 bước và vị trí giấu kho báu cách cây ôi 2 bước, theo hướng đông}
				{s=39 bước và vị trí giấu kho báu cách cây ôi 1 bước, theo hướng bắc}
				{s=39 bước và vị trí giấu kho báu cách cây ôi 1 bước, theo hướng đông}
				{s=34 bước và vị trí giấu kho báu cách cây ôi 2 bước, theo hướng bắc}
				\loigiai{}
			\end{ex}
			%%%==============HetCau_EX25==============%%%
\end{document}

