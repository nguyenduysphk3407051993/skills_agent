\documentclass[Main.tex]{subfiles}
\begin{document}
\hienthiloigiaivd
\subsection{Các dạng bài tập}
\begin{dang}{Câu hỏi lý thuyết}
\end{dang}
%%%=============Ví dụ mấu dạng 1=================%%%
\Noibat[][][\faBookmark]{Ví dụ mẫu}
%%%==============VDM1==============%%%
\hienthiloigiaivd
\begin{vd}
	Chất nào sau đây là chất điện ly?
	\choice
	{Đường saccharose}
	{Ethanol}
	{\True Natri clorua}
	{Dầu hỏa}
	\loigiai{Natri clorua (NaCl) là một chất điện ly vì nó phân ly thành các ion $Na^+$ và $Cl^-$ khi hòa tan trong nước. Các chất còn lại như đường saccharose, ethanol và dầu hỏa không phân ly thành ion trong dung dịch nước, do đó không phải là chất điện ly.}
\end{vd}
%%%=============EX_2=============%%%
\begin{vd}
	Theo thuyết Brønsted-Lowry, acid là chất:
	\choice
	{Nhận proton}
	{\True Cho proton}
	{Nhận electron}
	{Cho electron}
	\loigiai{Theo thuyết Brønsted-Lowry, acid được định nghĩa là chất cho proton $(H^+)$. Base, ngược lại, là chất nhận proton. Định nghĩa này khác với thuyết Lewis về acid-base, trong đó acid được xem là chất nhận cặp electron và base là chất cho cặp electron.}
\end{vd}
%%%%=============EX_3=============%%%
%%%==============HetVDM1==============%%%
\Noibat[][][\faBank]{Bài tập tự luyện dạng \thedang}

%%%=============SOẠN EX===============%%%
\Opensolutionfile{ansex}[Ans/LGEX-H11C01B02-BTTL01]
\Opensolutionfile{ans}[Ans/Ans-H11C01B02-BTTL01]
\hienthiloigiaiex
%\tatloigiaiex
%\luuloigiaiex
%%%=========ex_1=========%%%
%%%=============EX_1=============%%%
\begin{ex}
	Giá trị pH của máu người khỏe mạnh thường nằm trong khoảng:
	\choice
	{$4{,}5-6{,}5$}
	{$6{,}5-7{,}0$}
	{\True $7{,}35-7{,}45$}
	{$8{,}0-9{,}0$}
	\loigiai{Máu của người khỏe mạnh có pH nằm trong khoảng hẹp từ $7{,}35$ đến $7{,}45$. Đây là một khoảng pH hơi kiềm nhẹ, rất quan trọng cho sự hoạt động bình thường của các enzyme và protein trong cơ thể. Sự thay đổi nhỏ trong pH máu có thể dẫn đến các vấn đề sức khỏe nghiêm trọng.}
\end{ex}
%%%=============EX_4=============%%%
\begin{ex}
	Chất chỉ thị nào sau đây chuyển màu từ không màu sang hồng trong môi trường kiềm?
	\choice
	{Quỳ tím}
	{Methyl da cam}
	{\True Phenolphthalein}
	{Bromothymol blue}
	\loigiai{Phenolphthalein là một chất chỉ thị acid-base phổ biến, không màu trong môi trường acid và trung tính (pH $<8{,}2)$ và chuyển sang màu hồng trong môi trường kiềm (pH $>8{,}2)$. Đặc tính này làm cho phenolphthalein trở thành một chất chỉ thị hữu ích trong các phép chuẩn độ acid-base, đặc biệt khi chuẩn độ acid yếu bằng base mạnh.}
\end{ex}

%%%=============EX_6=============%%%
\begin{ex}
	Trong dung dịch nước, ion $\mathrm{Al}^{3+}$ tồn tại dưới dạng cân bằng:
	\choice
	{$\mathrm{Al}^{3+} + 3\mathrm{H}_2\mathrm{O} \rightleftharpoons \mathrm{Al(OH)}_3 + 3\mathrm{H}^+$}
	{$\mathrm{Al}^{3+} + 2\mathrm{H}_2\mathrm{O} \rightleftharpoons \mathrm{Al(OH)}_2^+ + 2\mathrm{H}^+$}
	{\True $\mathrm{Al}^{3+} + \mathrm{H}_2\mathrm{O} \rightleftharpoons \mathrm{Al(OH)}^{2+} + \mathrm{H}^+$}
	{$\mathrm{Al}^{3+} + 4\mathrm{H}_2\mathrm{O} \rightleftharpoons \mathrm{Al(OH)}_4^- + 4\mathrm{H}^+$}
	\loigiai{Trong dung dịch nước, ion $\mathrm{Al}^{3+}$ tham gia vào phản ứng thủy phân, tạo ra cân bằng với ion $\mathrm{Al(OH)}^{2+}$. Phương trình cân bằng chính xác là $\mathrm{Al}^{3+} + \mathrm{H}_2\mathrm{O} \rightleftharpoons \mathrm{Al(OH)}^{2+} + \mathrm{H}^+$. Đây là bước đầu tiên trong quá trình thủy phân của $\mathrm{Al}^{3+}$, và là cân bằng chủ yếu trong dung dịch nước.}
\end{ex}
%%%$=============EX_7=============$%%%
\begin{ex}
	Giá trị pH của một dung dịch acid mạnh $0{,}001M$ là:
	\choice
	{$1$}
	{$2$}
	{\True $3$}
	{$4$}
	\loigiai{Đối với acid mạnh, ta giả định rằng nó phân ly hoàn toàn trong nước. Với nồng độ $0{,}001M$, nồng độ ion $H+$ sẽ bằng $0{,}001M$.
		Áp dụng công thức pH $= -log[H+]$, ta có:
		pH $=-log(0{,}001)= -log(10^-3)=3$
		Vì vậy, giá trị pH của dung dịch acid mạnh $0{,}001M$ là $3$.}
\end{ex}
%%%$=============EX_8=============$%%%
\begin{ex}
	Trong phản ứng: $\mathrm{NH}_3 + \mathrm{H}_2\mathrm{O} \rightleftharpoons \mathrm{NH}_4^+ + \mathrm{OH}^-$, theo thuyết Brønsted-Lowry, $\mathrm{NH}_3$ đóng vai trò là:
	\choice
	{Acid}
	{\True Base}
	{Vừa acid vừa base}
	{Không phải acid cũng không phải base}
	\loigiai{Trong phản ứng này, $\mathrm{NH}_3$ (amoniac) nhận một proton $(H+)$ từ $\mathrm{H}_2\mathrm{O}$ để tạo thành $\mathrm{NH}_4^+$. Theo thuyết Brønsted-Lowry, chất nhận proton được định nghĩa là base. Do đó, trong phản ứng này, $\mathrm{NH}_3$ đóng vai trò là base Brønsted-Lowry.}
\end{ex}
%%%$=============EX_9=============$%%%
\begin{ex}
	Quỳ tím chuyển sang màu gì khi nhúng vào dung dịch có pH $=3$?
	\choice
	{Xanh}
	{\True Đỏ}
	{Tím}
	{Không đổi màu}
	\loigiai{Quỳ tím là một chất chỉ thị acid-base phổ biến. Nó có màu tím ở pH trung tính (khoảng $7$), chuyển sang màu đỏ trong môi trường acid (pH $<7)$ và màu xanh trong môi trường kiềm (pH $>7)$. Với pH $=3$, dung dịch có tính acid mạnh, do đó quỳ tím sẽ chuyển sang màu đỏ.}
\end{ex}
%%%%=============EX_10=============%%%
\begin{ex}
	Trong chuẩn độ acid-base, để xác định chính xác điểm tương đương, nên chọn chất chỉ thị có khoảng đổi màu:
	\choice
	{Trùng với pH tại điểm tương đương}
	{Cao hơn nhiều so với pH tại điểm tương đương}
	{Thấp hơn nhiều so với pH tại điểm tương đương}
	{\True Gần với pH tại điểm tương đương}
	\loigiai{Để xác định chính xác điểm tương đương trong chuẩn độ acid-base, nên chọn chất chỉ thị có khoảng đổi màu gần với pH tại điểm tương đương. Điều này đảm bảo rằng sự thay đổi màu sắc của chất chỉ thị xảy ra càng gần với điểm tương đương thực tế càng tốt, giúp tăng độ chính xác của phép chuẩn độ. Nếu khoảng đổi màu quá cao hoặc quá thấp so với pH tại điểm tương đương, có thể dẫn đến sai số đáng kể trong kết quả chuẩn độ.}
\end{ex}
%%%$=============EX_11=============$%%%
\begin{ex}
	Chất nào sau đây là chất điện ly?
	\choice
	{Đường saccharose}
	{Dầu hỏa}
	{\True Natri clorua}
	{Cồn ethylic nguyên chất}
	\loigiai{Natri clorua (NaCl) là một chất điện ly, khi hòa tan trong nước nó sẽ phân ly thành các ion $Na^+$ và $Cl^-$. Các chất còn lại không phân ly thành ion trong dung dịch nên không phải là chất điện ly.}
\end{ex}
%%%$=============EX_12=============$%%%
\begin{ex}
	Theo thuyết Brønsted - Lowry, acid là chất:
	\choice
	{Nhận proton}
	{\True Cho proton}
	{Nhận electron}
	{Cho electron}
	\loigiai{Theo thuyết Brønsted - Lowry, acid được định nghĩa là chất có khả năng cho proton $H^+$ trong phản ứng. Base là chất có khả năng nhận proton.}
\end{ex}
%%%$=============EX_13=============$%%%
\begin{ex}
	pH của một dung dịch trung tính ở $25\circ C$ là:
	\choice
	{$0$}
	{$14$}
	{\True $7$}
	{$1$}
	\loigiai{Ở $25\circ C$, một dung dịch được coi là trung tính khi có pH $=7$. Khi pH $<7$, dung dịch là acid; khi pH $>7$, dung dịch là base.}
\end{ex}
%%%$=============EX_14=============$%%%
\begin{ex}
	Công thức tính pH của dung dịch acid mạnh là:
	\choice
	{pH $= -log[OH_-]$}
	{pH $=14+ log[H+]$}
	{\True $pH = -log[H^+]$}
	{pH $= log[H^+]$}
	\loigiai{Công thức tính pH của dung dịch acid mạnh là $pH = -log[H+]$, trong đó $[H^+]$ là nồng độ ion hydro trong dung dịch.}
\end{ex}
%%%=============EX_15=============%%%
\begin{ex}
	Chất chỉ thị nào sau đây chuyển màu trong môi trường base?
	\choice
	{Methyl cam}
	{Methyl đỏ}
	{\True Phenolphthalein}
	{Xanh bromothymol}
	\loigiai{Phenolphthalein là chất chỉ thị chuyển màu trong môi trường base. Nó không màu trong môi trường acid và chuyển sang màu hồng trong môi trường base.}
\end{ex}
%%%=============EX_16=============%%%
\begin{ex}
	Trong phương pháp chuẩn độ acid - base, điểm tương đương là điểm mà:
	\choice
	{pH của dung dịch bằng 7}
	{Chất chỉ thị chuyển màu}
	{\True Số mol acid đã phản ứng hết với số mol base}
	{Thể tích dung dịch chuẩn độ bằng thể tích dung dịch được chuẩn độ}
	\loigiai{Điểm tương đương trong chuẩn độ acid - base là điểm mà số mol acid đã phản ứng hết với số mol base, tức là tại điểm này, lượng acid và base đã phản ứng với nhau theo tỉ lệ phản ứng hóa học.}
\end{ex}
%%%=============EX_17=============%%%
\begin{ex}
	Ion Al3+ trong dung dịch nước có tính:
	\choice
	{Trung tính}
	{Base}
	{\True Acid}
	{Lưỡng tính}
	\loigiai{Ion $Al^3+$ trong dung dịch nước có tính acid. Nó có thể tham gia phản ứng thủy phân tạo ra ion H+, làm tăng nồng độ H+ trong dung dịch: $Al^{3+}+ + H_2O \xrightleftharpoons Al(OH)^{2+} + H^+$}
\end{ex}
%%%=============EX_18=============%%%
\begin{ex}
	Quỳ tím chuyển màu gì trong môi trường acid?
	\choice
	{Xanh}
	{\True Đỏ}
	{Tím}
	{Không màu}
	\loigiai{Quỳ tím chuyển sang màu đỏ trong môi trường acid. Trong môi trường base, nó chuyển sang màu xanh, còn trong môi trường trung tính, nó giữ nguyên màu tím.}
\end{ex}
%%%=============EX_19=============%%%
\begin{ex}
	Phản ứng nào sau đây thể hiện tính chất của acid theo Brønsted - Lowry?
	\choice
	{$NaOH + HCl \rightarrow NaCl + H2O$}
	{$2Na + 2H_2O \rightarrow 2NaOH + H_2$}
	{\True $HCl + H_2O \rightarrow H_3O+ + Cl-$}
	{$CuO + 2HCl \rightarrow CuCl_2 + H_2O$}
	\loigiai{Phản ứng $HCl + H_2O → H_3O^+ + Cl^-$ thể hiện tính chất của acid theo Brønsted - Lowry. Trong phản ứng này, HCl đóng vai trò là acid (cho proton) và $H_2O$ đóng vai trò là base (nhận proton).}
\end{ex}
%%%=============EX_20=============%%%
\begin{ex}
	pH của máu người khỏe mạnh thường nằm trong khoảng:
	\choice
	{$4{,}5-5{,}5$}
	{$6{,}0-7{,}0$}
	{\True $7{,}35-7{,}45$}
	{$8{,}0-9{,}0$}
	\loigiai{pH của máu người khỏe mạnh thường nằm trong khoảng $7{,}35-7{,}45$. Đây là một khoảng pH hẹp và rất quan trọng cho sự hoạt động bình thường của các enzyme và quá trình trao đổi chất trong cơ thể.}
\end{ex}
%%%=============EX_21=============%%%
\begin{ex}
	Chất nào sau đây không phải là chất điện ly?
	\choice
	{NaCl}
	{$H_2SO_4$}
	{KOH}
	{\True $C_6H_{12}O_6$ (glucose)}
	\loigiai{$C_6H_{12}O_6$ (glucose) không phải là chất điện ly. Khi hòa tan trong nước, glucose không phân ly thành ion mà tồn tại dưới dạng phân tử. Các chất còn lại (NaCl, $H_2SO_4$, KOH) đều là chất điện ly, phân ly thành ion khi hòa tan trong nước.}
\end{ex}
%%%=============EX_22=============%%%
\begin{ex}
	Theo thuyết Brønsted - Lowry, trong phản ứng $NH3 + H2O \xrightleftharpoons{} NH4+ + OH-$, $NH_3$ đóng vai trò là:
	\choice
	{Acid}
	{\True Base}
	{Chất oxi hóa}
	{Chất khử}
	\loigiai{Trong phản ứng $NH3 + H2O \xrightleftharpoons{} NH4+ + OH-$, $NH_3$ đóng vai trò là base theo thuyết Brønsted - Lowry vì nó nhận proton ($H^+$) từ $H_2O$ để tạo thành $NH_4^+$. $H_2O$ trong trường hợp này đóng vai trò là acid, cho proton.}
\end{ex}
%%%=============EX_23=============%%%
\begin{ex}
	pH của nước cất tinh khiết ở $25^\circ C$ là:
	\choice
	{$0$}
	{$14$}
	{\True $7$}
	{$1$}
	\loigiai{pH của nước cất tinh khiết ở $25^\circ C$ là 7. Ở nhiệt độ này, nồng độ ion $H^+$ và $OH^-$ trong nước tinh khiết đều bằng $10^{-7} mol/L$, do đó $pH = -log[H+] = -log(10^{-7}) = 7$.}
\end{ex}
%%%=============EX_24=============%%%
\begin{ex}
	Công thức tính pOH của một dung dịch là:
	\choice
	{$pOH = -log[H^+]$}
	{$pOH = 14 + log[OH^-]$}
	{\True $pOH = -log[OH^-]$}
	{$pOH = log[OH^-]$}
	\loigiai{Công thức tính pOH của một dung dịch là $pOH = -log[OH^-]$, trong đó $[OH^-]$ là nồng độ ion hydroxide trong dung dịch. Chú ý rằng $pH + pOH = 14$ ở $25^\circ C$.}
\end{ex}
%%%=============EX_25=============%%%
\begin{ex}
	Chất chỉ thị nào sau đây chuyển màu trong khoảng pH từ $8{,}2$ đến $10$?
	\choice
	{Methyl cam}
	{Methyl đỏ}
	{\True Phenolphthalein}
	{Xanh bromothymol}
	\loigiai{Phenolphthalein là chất chỉ thị chuyển màu trong khoảng pH từ $8{,}2$ đến $10$. Nó không màu khi pH $<8{,}2$ và chuyển sang màu hồng khi pH $>8{,}2$.}
\end{ex}
%%%=============EX_26=============%%%
\begin{ex}
	Trong phương pháp chuẩn độ acid - base, điểm kết thúc chuẩn độ là:
	\choice
	{Điểm mà số mol acid bằng số mol base}
	{\True Điểm mà chất chỉ thị chuyển màu}
	{Điểm mà pH của dung dịch bằng 7}
	{Điểm mà thể tích dung dịch chuẩn độ bằng thể tích dung dịch được chuẩn độ}
	\loigiai{Điểm kết thúc chuẩn độ là điểm mà chất chỉ thị chuyển màu. Đây là điểm mà người thực hiện chuẩn độ quan sát được và dừng việc thêm dung dịch chuẩn độ. Điểm này thường gần với điểm tương đương nhưng không nhất thiết trùng khớp.}
\end{ex}
%%%=============EX_27=============%%%
\begin{ex}
	Ion Fe3+ trong dung dịch nước có tính:
	\choice
	{Trung tính}
	{Base}
	{\True Acid}
	{Lưỡng tính}
	\loigiai{Ion Fe3+ trong dung dịch nước có tính acid. Nó tham gia phản ứng thủy phân tạo ra ion H+, làm tăng nồng độ H+ trong dung dịch: $Fe^{3+} + H_2O \xrightleftharpoons{} Fe(OH)_2+ + H^+$}
\end{ex}
%%%=============EX_28=============%%%
\begin{ex}
	Giấy quỳ đỏ chuyển màu gì trong môi trường base?
	\choice
	{Đỏ}
	{\True Xanh}
	{Tím}
	{Không màu}
	\loigiai{Giấy quỳ đỏ chuyển sang màu xanh trong môi trường base. Trong môi trường acid, nó giữ nguyên màu đỏ.}
\end{ex}
%%%=============EX_29=============%%%
\begin{ex}
	Phản ứng nào sau đây thể hiện tính chất của base theo Brønsted - Lowry?
	\choice
	{$NaOH + HCl \xrightarrow{} NaCl + H_2O$}
	{$2Na + 2H_2O \xrightarrow{} 2NaOH + H_2$}
	{\True $NH_3 + H_2O \xrightleftharpoons{} NH_4^+ + OH^-$}
	{$CuO + 2HCl \xrightarrow{} CuCl_2 + H_2O$}
	\loigiai{Phản ứng $NH3 + H_2O \xrightleftharpoons{} NH_4^+ + OH^-$ thể hiện tính chất của base theo Brønsted - Lowry. Trong phản ứng này, NH3 đóng vai trò là base (nhận proton) và H2O đóng vai trò là acid (cho proton).}
\end{ex}
%%%=============EX_30=============%%%
\begin{ex}
	pH của nước mưa tự nhiên thường nằm trong khoảng:
	\choice
	{$1 - 2$}
	{\True $5{,}6 - 6{,}5$}
	{$7{,}0 - 8{,}0$}
	{$9{,}0 - 10{,}0$}
	\loigiai{pH của nước mưa tự nhiên thường nằm trong khoảng $5{,}6 - 6{,}5$. Nước mưa có tính acid nhẹ do hòa tan $CO_2$ từ không khí tạo thành acid carbonic ($H_2CO_3$).}
\end{ex}
%%%%=============EX_31=============%%%
\begin{ex}
	Dung dịch nào sau đây có $pH < 7$?
	\choice
	{Dung dịch NaOH}
	{Dung dịch $Na_2CO_3$}
	{\True Dung dịch HCl}
	{Dung dịch $NH_3$}
	\loigiai{Dung dịch HCl có $pH < 7$ vì HCl là một acid mạnh. Khi hòa tan trong nước, nó phân ly hoàn toàn tạo ra ion H+, làm tăng nồng độ $H^+$ trong dung dịch, dẫn đến $pH < 7$.}
\end{ex}
%%%=============EX_32=============%%%
\begin{ex}
	Trong phương trình $pH + pOH = 14$ (ở $25^\circ C$), 14 là giá trị của:
	\choice
	{pH của nước tinh khiết}
	{pOH của nước tinh khiết}
	{\True pKw (hằng số phân ly của nước)}
	{Nồng độ ion $H^+$ trong nước tinh khiết}
	\loigiai{Trong phương trình $pH + pOH = 14$ (ở $25^\circ C$), 14 là giá trị của pKw - hằng số phân ly của nước. Kw là tích ion của nước ($Kw = [H+][OH-] = 10^-14$ ở $25^\circ C$), và $pKw = -logKw = 14$.}
\end{ex}
%%%=============EX_33=============%%%
\begin{ex}
	Ion $CO_3^{2-}$ trong dung dịch nước có tính:
	\choice
	{Acid}
	{\True Base}
	{Trung tính}
	{Lưỡng tính}
	\loigiai{Ion $CO_3^{2-}$ trong dung dịch nước có tính base. Nó tham gia phản ứng thủy phân tạo ra ion $OH^-$, làm tăng nồng độ $OH^-$ trong dung dịch: $CO_3^{2-} + H_2O \xrightleftharpoons{} HCO_3^- + OH^-$}
\end{ex}
%%%=============EX_34=============%%%
\begin{ex}
	Trong phản ứng chuẩn độ giữa NaOH và HCl, chất chỉ thị nào sau đây phù hợp nhất?
	\choice
	{Phenolphthalein}
	{\True Methyl cam}
	{Xanh bromothymol}
	{Phenol đỏ}
	\loigiai{Methyl cam là chất chỉ thị phù hợp nhất cho phản ứng chuẩn độ giữa NaOH và HCl. Nó có khoảng chuyển màu từ pH $3{,}1$ đến $4{,}4$, gần với điểm tương đương của phản ứng giữa acid mạnh và base mạnh (pH khoảng $7$).}
\end{ex}
%%%=============EX_35=============%%%
\begin{ex}
	Nếu pH của một dung dịch là 4, thì pOH của dung dịch đó là bao nhiêu (ở 25°C)?
	\choice
	{4}
	{7}
	{\True 10}
	{14}
	\loigiai{Ở $25^\circ C$, ta có $pH + pOH = 14$. Nếu $pH = 4$, thì $pOH = 14 - pH = 14 - 4 = 10$.}
\end{ex}
%%%=============EX_36=============%%%
\begin{ex}
	Trong phản ứng $Al(OH)_3 + 3H+ \xrightleftharpoons{} Al3+ + 3H2O$, Al(OH)3 đóng vai trò là:
	\choice
	{Acid}
	{\True Base}
	{Chất oxi hóa}
	{Chất khử}
	\loigiai{Trong phản ứng $Al(OH)_3 + 3H^+ \xrightleftharpoons{} Al^{3+} + 3H_2O$, $Al(OH)_3$ đóng vai trò là base theo thuyết Brønsted - Lowry vì nó nhận proton $(H^+)$ từ acid.}
\end{ex}
%%%=============EX_37=============%%%
\begin{ex}
	Nồng độ ion H+ trong một dung dịch là $10^{-}3$ mol/L. pH của dung dịch này là:
	\choice
	{-3}
	{\True 3}
	{11}
	{13}
	\loigiai{$pH = -log[H+] = -log(10^-3) = 3$}
\end{ex}
%%%=============EX_38=============%%%
\begin{ex}
	Chất nào sau đây là chất lưỡng tính?
	\choice
	{NaOH}
	{HCl}
	{\True $Al(OH)_3$}
	{$H_2SO_4$}
	\loigiai{$Al(OH)_3$ là chất lưỡng tính. Nó có thể đóng vai trò là acid (cho proton) trong phản ứng với base mạnh, và đóng vai trò là base (nhận proton) trong phản ứng với acid mạnh.}
\end{ex}
%%%=============EX_39=============%%%
\begin{ex}
	Phương pháp chuẩn độ acid - base được sử dụng để xác định:
	\choice
	{Khối lượng riêng của acid hoặc base}
	{Nhiệt độ sôi của acid hoặc base}
	{\True Nồng độ của acid hoặc base}
	{Điện tích của ion trong dung dịch acid hoặc base}
	\loigiai{Phương pháp chuẩn độ acid - base được sử dụng để xác định nồng độ của acid hoặc base. Bằng cách sử dụng một dung dịch chuẩn đã biết nồng độ, ta có thể xác định được nồng độ chính xác của dung dịch acid hoặc base cần phân tích.}
\end{ex}
%%%=============EX_40=============%%%
\begin{ex}
	Trong phản ứng $NH_4^+ + OH^- \xrightleftharpoons{} NH_3 + H_2O$, $NH_4^+$ đóng vai trò là:
	\choice
	{\True Acid}
	{Base}
	{Chất oxi hóa}
	{Chất khử}
	\loigiai{Trong phản ứng $NH_4^+ + OH^- \xrightleftharpoons{} NH_3 + H_2O$, $NH_4^+$ đóng vai trò là acid theo thuyết Brønsted - Lowry vì nó cho proton ($H^+$) cho $OH^-$ để tạo thành H2O.}
\end{ex}
%%%%=============EX_41=============%%%
\begin{ex}
	pH của đất ảnh hưởng như thế nào đến sự phát triển của cây trồng?
	\choice
	{pH không ảnh hưởng đến sự phát triển của cây trồng}
	{Cây trồng phát triển tốt nhất ở pH rất thấp ($< 3$)}
	{Cây trồng phát triển tốt nhất ở pH rất cao ($> 10$)}
	{\True pH ảnh hưởng đến khả năng hấp thu dinh dưỡng của cây trồng}
	\loigiai{pH của đất ảnh hưởng đến khả năng hấp thu dinh dưỡng của cây trồng. Hầu hết các cây trồng phát triển tốt nhất ở pH từ $6{,}0$ đến $7{,}5$. Ở pH quá thấp hoặc quá cao, một số chất dinh dưỡng có thể trở nên không tan hoặc không sẵn có cho cây hấp thu.}
\end{ex}
%%%=============EX_42=============%%%
\begin{ex}
	Nồng độ ion $OH^-$ trong một dung dịch là $10^{-11}$ mol/L. pH của dung dịch này là (ở $25^\circ C$):
	\choice
	{3}
	{11}
	{\True 13}
	{1}
	\loigiai{%
		Bước 1: Tính pOH
		$pOH = -log[OH-] = -log(10^-11) = 11$
		\\
		Bước 2: Sử dụng mối quan hệ $pH + pOH = 14$
		$pH = 14 - pOH = 14 - 11 = 3$
	}
\end{ex}
%%%=============EX_43=============%%%
\begin{ex}
	Trong phản ứng chuẩn độ giữa $CH_3COOH$ và NaOH, pH tại điểm tương đương là:
	\choice
	{Nhỏ hơn 7}
	{Bằng 7}
	{\True Lớn hơn 7}
	{Bằng 0}
	\loigiai{Trong phản ứng chuẩn độ giữa $CH_3COOH$ (acid yếu) và NaOH (base mạnh), pH tại điểm tương đương lớn hơn 7. Điều này là do muối tạo thành ($CH_3COONa$) có tính base yếu do ion CH3COO- thủy phân trong nước tạo ra OH-.}
\end{ex}
%%%=============EX_44=============%%%
\begin{ex}
	Chất nào sau đây là acid theo Brønsted $-$ Lowry?
	\choice
	{NaOH}
	{$CH_3COONa$}
	{\True $H_2O$}
	{$NH_3$}
	\loigiai{$H_2$Ocó thể đóng vai trò là acid theo Brønsted $-$ Lowry. Trong một số phản ứng, nước có thể cho proton $(H+)$, ví dụ trong phản ứng: $H_2O+NH_3$ $\xrightleftharpoons{}$ $NH_4^++OH^-$}
\end{ex}
%%%=============EX_45=============%%%
\begin{ex}
	Nồng độ ion $H^+$ trong máu người khỏe mạnh khoảng:
	\choice
	{$10^-1$ mol/L}
	{$10^-5$ mol/L}
	{\True $10^-7$ mol/L}
	{$10^-14$ mol/L}
	\loigiai{Nồng độ ion $H^+$ trong máu người khỏe mạnh khoảng $10^{-7}$ mol/L, tương ứng với pH khoảng $7{,}4$. Điều này đảm bảo môi trường slightly alkaline cần thiết cho các quá trình sinh hóa trong cơ thể.}
\end{ex}
%%%=============EX_46=============%%%
\begin{ex}
	Trong phản ứng $HCO_3^- + H_2O \xrightleftharpoons{} H_2CO_3 + OH^-$, $HCO_3^-$ đóng vai trò là:
	\choice
	{Acid}
	{\True Base}
	{Chất oxi hóa}
	{Chất khử}
	\loigiai{Trong phản ứng $HCO_3^- + H_2O \xrightleftharpoons{} H_2CO_3 + OH^-$, $HCO_3^-$ đóng vai trò là base theo thuyết Brønsted - Lowry vì nó nhận proton ($H^+$) từ $H_2O$ để tạo thành $H_2CO_3$.}
\end{ex}
%%%=============EX_47=============%%%
\begin{ex}
	pH của nước biển thường nằm trong khoảng:
	\choice
	{$4-5$}
	{$6-7$}
	{\True $7{,}5-8{,}4$}
	{$9-10$}
	\loigiai{pH của nước biển thường nằm trong khoảng $7{,}5-8{,}4$. Nước biển có tính base nhẹ do sự hiện diện của các ion carbonate và bicarbonate.}
\end{ex}
%%%=============EX_48=============%%%
\begin{ex}
	Chất nào sau đây không thể được sử dụng làm chất chỉ thị acid - base?
	\choice
	{Phenolphthalein}
	{Methyl cam}
	{Quỳ tím}
	{\True Glucose}
	\loigiai{Glucose không thể được sử dụng làm chất chỉ thị acid - base vì nó không thay đổi màu sắc theo sự thay đổi của pH. Các chất chỉ thị acid - base phải có khả năng thay đổi màu sắc rõ ràng trong các khoảng pH khác nhau.}
\end{ex}
%%%=============EX_49=============%%%
\begin{ex}
	Trong quá trình chuẩn độ HCl bằng NaOH, tại điểm tương đương:
	\choice
	{Dung dịch có tính acid}
	{Dung dịch có tính base}
	{\True Dung dịch có pH $=7$}
	{Không thể xác định được pH của dung dịch}
	\loigiai{Trong quá trình chuẩn độ HCl (acid mạnh) bằng NaOH (base mạnh), tại điểm tương đương, dung dịch có pH $=7$. Điều này là do HCl và NaOH phản ứng hoàn toàn với nhau theo tỉ lệ $1:1$, tạo ra muối NaCl (không thủy phân) và nước.}
\end{ex}
%%%=============EX_50=============%%%
\begin{ex}
	Phát biểu nào sau đây về sự điện ly là đúng?
	\choice
	{Tất cả các chất tan trong nước đều điện ly}
	{Chỉ có các chất tan trong nước mới điện ly}
	{\True Sự điện ly là quá trình phân ly các chất thành ion khi hòa tan trong dung môi phân cực}
	{Sự điện ly chỉ xảy ra với các chất có liên kết ion}
	\loigiai{Sự điện ly là quá trình phân ly các chất thành ion khi hòa tan trong dung môi phân cực. Điều này có thể xảy ra với cả chất có liên kết ion và một số chất có liên kết cộng hóa trị phân cực.}
\end{ex}
%%%%=============EX_51=============%%%
\begin{ex}
	Trong phản ứng $NH_4^+ + H_2O \xrightleftharpoons{} NH_3 + H_3O^+$, $H_2O$ đóng vai trò là:
	\choice
	{Acid}
	{\True Base}
	{Chất oxi hóa}
	{Chất khử}
	\loigiai{Trong phản ứng $NH_4^+ + H_2O \xrightleftharpoons{} NH_3 + H_3O^+$, $H_2O$ đóng vai trò là base theo thuyết Brønsted - Lowry vì nó nhận proton ($H^+$) từ $NH_4^+$ để tạo thành $H_3O^+$.}
\end{ex}
%%%%=============EX_52=============%%%
\begin{ex}
	Một dung dịch có pH $=2$. Nồng độ ion $H+$ trong dung dịch này là:
	\choice
	{$10^{-2}$ mol/L}
	{\True $10^{-2}$ mol/L}
	{$2$ mol/L}
	{$10^{-12}$ mol/L}
	\loigiai{
		$pH =-log[H+]$;
		$2 = -log[H+]$;
		$[H^+] =10^{-2}$ mol/L
	}
\end{ex}
%%%%=============EX_53=============%%%
\begin{ex}
	Chất nào sau đây là chất điện ly mạnh?
	\choice
	{$CH_3COOH$}
	{$NH_3$}
	{\True $KOH$}
	{$H_2CO_3$}
	\loigiai{$KOH$ là chất điện ly mạnh. Khi hòa tan trong nước, nó phân ly hoàn toàn thành các ion $K^+$ và $OH^-$. Các chất còn lại $(CH_3COOH$, $NH_3$, $H_2CO_3)$ là chất điện ly yếu.}
\end{ex}
%%%%=============EX_54=============%%%
\begin{ex}
	Trong chuẩn độ acid - base, đường cong chuẩn độ biểu diễn sự phụ thuộc của:
	\choice
	{Nhiệt độ vào thể tích dung dịch chuẩn độ}
	{\True pH vào thể tích dung dịch chuẩn độ}
	{Áp suất vào thể tích dung dịch chuẩn độ}
	{Nồng độ vào nhiệt độ dung dịch}
	\loigiai{Trong chuẩn độ acid - base, đường cong chuẩn độ biểu diễn sự phụ thuộc của pH vào thể tích dung dịch chuẩn độ. Đường cong này cho phép xác định điểm tương đương và lựa chọn chất chỉ thị phù hợp.}
\end{ex}
%%%%=============EX_55=============%%%
\begin{ex}
	pH của nước mưa acid thường:
	\choice
	{Lớn hơn 7}
	{Bằng 7}
	{\True Nhỏ hơn $5{,}6$}
	{Bằng 14}
	\loigiai{pH của nước mưa acid thường nhỏ hơn $5{,}6$. Nước mưa tự nhiên có pH khoảng $5{,}6$ do hòa tan CO2 từ không khí. Khi pH nhỏ hơn $5{,}6$, nước mưa được coi là acid do ảnh hưởng của các chất ô nhiễm như $SO_2$, $NO_x$.}
\end{ex}
%%%=============EX_56=============%%%
\begin{ex}
	Chất nào sau đây có thể đóng vai trò vừa là acid vừa là base theo Brønsted - Lowry?
	\choice
	{NaOH}
	{HCl}
	{\True $H_2O$}
	{CH4}
	\loigiai{$H_2O$ có thể đóng vai trò vừa là acid vừa là base theo Brønsted - Lowry. Nó có thể cho proton (ví dụ: $H2O + NH_3 \xrightleftharpoons{} NH_4^+ + OH^-$) hoặc nhận proton (ví dụ: $H2O + HCl \xrightleftharpoons{} H_3O^+ + Cl^-$) tùy thuộc vào chất phản ứng với nó.}
\end{ex}
%%%=============EX_57=============%%%
\begin{ex}
	Trong phản ứng chuẩn độ giữa $CH_3COOH$ và $NaOH$, chất chỉ thị nào sau đây phù hợp nhất?
	\choice
	{Methyl cam}
	{\True Phenolphthalein}
	{Methyl đỏ}
	{Xanh bromothymol}
	\loigiai{Phenolphthalein là chất chỉ thị phù hợp nhất cho phản ứng chuẩn độ giữa $CH_3COOH$ và NaOH. Nó có khoảng chuyển màu từ pH $8{,}3$ đến $10$, gần với điểm tương đương của phản ứng giữa acid yếu và base mạnh ($pH > 7$).}
\end{ex}
%%%=============EX_58=============%%%
\begin{ex}
Ý nghĩa thực tiễn của cân bằng trong dung dịch nước của ion$Fe^{3+}$ là gì?
\choice
{Tạo ra màu sắc đẹp cho dung dịch}
{Làm tăng độ dẫn điện của dung dịch}
{\True Ảnh hưởng đến quá trình xử lý nước và ăn mòn kim loại}
{Không có ý nghĩa thực tiễn}
\loigiai{Cân bằng trong dung dịch nước của ion $Fe^{3+}$ có ý nghĩa thực tiễn quan trọng trong việc ảnh hưởng đến quá trình xử lý nước và ăn mòn kim loại. Ion $Fe^{3+}$ thủy phân tạo ra acid, có thể gây ăn mòn và ảnh hưởng đến chất lượng nước.}
\end{ex}
%%%=============EX_59=============%%%
\begin{ex}
Phát biểu nào sau đây về thuyết Brønsted - Lowry là không đúng?
\choice
{Acid là chất cho proton}
{Base là chất nhận proton}
{Một chất có thể vừa là acid vừa là base}
{\True Chỉ có các chất có H trong công thức mới là acid}
\loigiai{Phát biểu "Chỉ có các chất có H trong công thức mới là acid" là không đúng theo thuyết Brønsted - Lowry. Theo thuyết này, acid là chất có khả năng cho proton, không nhất thiết phải có H trong công thức (ví dụ: $NH_4^+$ là acid Brønsted - Lowry).}
\end{ex}

\Closesolutionfile{ans}
\Closesolutionfile{ansex}
%\bangdapan{Ans-H11C01B02-BTTL1}


%%================Dạng 3==============%%%
\begin{dang}{Viết phương trình điện li}
\end{dang}
\begin{pp}
\begin{itemize}
	\item Các chất điênli mạnh dùng mũi tên 1 chiều
	\item Đối với chất điện li yếu dùng mũi tên thuận nghịch
\end{itemize}
\end{pp}
%%=============Ví dụ mấu dạng 3=================%%%
\Noibat[][][\faBookmark]{Ví dụ mẫu}
%%==============VDM1==============%%%
\begin{vd}
Viết phương trình điện li trong nước của các chất sau: $\mathrm{HClO}_4$, $\mathrm{CH_3COONa}$, $\mathrm{Na}_2\mathrm{SO}_4$, $\mathrm{NH}_4\mathrm{Cl}$
\loigiai{
	\begin{itemize}
		\item $\mathrm{HClO}_4$: \\
	 $\mathrm{HClO}_4 \rightarrow \mathrm{H}^+ + \mathrm{ClO}_4^-$
		\item $\mathrm{CH_3COONa}$: \\
	 $\mathrm{CH_3COONa} \rightarrow \mathrm{CH_3COO}^- + \mathrm{Na}^+$
		\item $\mathrm{Na}_2\mathrm{SO}_4$: \\
	 $\mathrm{Na}_2\mathrm{SO}_4 \rightarrow 2\mathrm{Na}^+ + \mathrm{SO}_4^{2-}$
		\item $\mathrm{NH}_4\mathrm{Cl}$: \\
	 $\mathrm{NH}_4\mathrm{Cl} \rightarrow \mathrm{NH}_4^+ + \mathrm{Cl}^-$
	\end{itemize}
}
\end{vd}

%%==============HetVDM1==============%%%
\Noibat[][][\faBank]{Bài tập tự luyện dạng \thedang}
\phan{Bài tập tự luận}
%%=============SOẠN BT===============%%%
\Opensolutionfile{ansbth}[Ans/LGBT-H11C01B01-BTTL03]
\Opensolutionfile{ansbt}[Ans/AnsBT-H11C01B01-BTTL03]
\luuloigiaibt
\hienthiloigiaibt
%%==============Bai_BT1==============%%%
\begin{bt}
Viết phương trình điện li trong nước của các chất sau: $\mathrm{NaHCO}_3, \mathrm{CuCl}_2$, $\left(NH_4\right)_2SO_4, \mathrm{Fe}\left(NO_3\right)_3$
\loigiai{
	\begin{itemize}
		\item $\mathrm{NaHCO}_3 \rightarrow \mathrm{Na}^+ + \mathrm{HCO}_3^-$ 
		\item $\mathrm{CuCl}_2 \rightarrow \mathrm{Cu}^{2+} + 2\mathrm{Cl}^-$ 
		\item $\left(NH_4\right)_2SO_4 \rightarrow 2\mathrm{NH}_4^+ + \mathrm{SO}_4^{2-}$ 
		\item $\mathrm{Fe}\left(NO_3\right)_3 \rightarrow \mathrm{Fe}^{3+} + 3\mathrm{NO}_3^-$ 
	\end{itemize}
}
\end{bt}
%%==============HetBai_BT1==============%%%

%%==============Bai_BT2==============%%%
\begin{bt}
Viết phương trình điện li trong nước của các chất sau: $\mathrm{CH}_3\mathrm{COOH}, \mathrm{Ba(OH)}_2, \mathrm{NH}_4\mathrm{Cl}, \mathrm{H}_2\mathrm{CO}_3$
\loigiai{
	\begin{itemize}
		\item $\mathrm{CH}_3\mathrm{COOH} \rightleftharpoons \mathrm{CH}_3\mathrm{COO}^- + \mathrm{H}^+$ 
		\item $\mathrm{Ba(OH)}_2 \rightarrow \mathrm{Ba}^{2+} + 2\mathrm{OH}^-$ 
		\item $\mathrm{NH}_4\mathrm{Cl} \rightarrow \mathrm{NH}_4^+ + \mathrm{Cl}^-$ 
		\item $\mathrm{H}_2\mathrm{CO}_3 \rightleftharpoons \mathrm{H}^+ + \mathrm{HCO}_3^-$ 
	\end{itemize}
}
\end{bt}
%%==============HetBai_BT2==============%%%

%%==============Bai_BT3==============%%%
\begin{bt}
Viết phương trình điện li trong nước của các chất sau: $\mathrm{KNO}_3, \mathrm{H}_2\mathrm{S}, \mathrm{Mg(ClO}_4)_2, \mathrm{HClO}$
\loigiai{
	\begin{itemize}
		\item $\mathrm{KNO}_3 \rightarrow \mathrm{K}^+ + \mathrm{NO}_3^-$ 
		\item $\mathrm{H}_2\mathrm{S} \rightleftharpoons \mathrm{H}^+ + \mathrm{HS}^-$ 
		\item $\mathrm{Mg(ClO}_4)_2 \rightarrow \mathrm{Mg}^{2+} + 2\mathrm{ClO}_4^-$ 
		\item $\mathrm{HClO} \rightleftharpoons \mathrm{H}^+ + \mathrm{ClO}^-$ 
	\end{itemize}
}
\end{bt}
%%==============HetBai_BT3==============%%%

%%==============Bai_BT4==============%%%
\begin{bt}
Viết phương trình điện li trong nước của các chất sau: $\mathrm{Al}_2(\mathrm{SO}_4)_3, \mathrm{HF}, \mathrm{Na}_3\mathrm{PO}_4, \mathrm{NH}_4\mathrm{NO}_3$
\loigiai{
	\begin{itemize}
		\item $\mathrm{Al}_2(\mathrm{SO}_4)_3 \rightarrow 2\mathrm{Al}^{3+} + 3\mathrm{SO}_4^{2-}$ 
		\item $\mathrm{HF} \rightleftharpoons \mathrm{H}^+ + \mathrm{F}^-$ 
		\item $\mathrm{Na}_3\mathrm{PO}_4 \rightarrow 3\mathrm{Na}^+ + \mathrm{PO}_4^{3-}$ 
		\item $\mathrm{NH}_4\mathrm{NO}_3 \rightarrow \mathrm{NH}_4^+ + \mathrm{NO}_3^-$ 
	\end{itemize}
}
\end{bt}
%%==============HetBai_BT4==============%%%

%%==============Bai_BT5==============%%%
\begin{bt}
Viết phương trình điện li trong nước của các chất sau: $\mathrm{Pb(NO}_3)_2, \mathrm{CH}_3\mathrm{NH}_2, \mathrm{K}_2\mathrm{Cr}_2\mathrm{O}_7, \mathrm{H}_3\mathrm{PO}_4$
\loigiai{
	\begin{itemize}
		\item $\mathrm{Pb(NO}_3)_2 \rightarrow \mathrm{Pb}^{2+} + 2\mathrm{NO}_3^-$ 
		\item $\mathrm{CH}_3\mathrm{NH}_2 + \mathrm{H}_2\mathrm{O} \rightleftharpoons \mathrm{CH}_3\mathrm{NH}_3^+ + \mathrm{OH}^-$ 
		\item $\mathrm{K}_2\mathrm{Cr}_2\mathrm{O}_7 \rightarrow 2\mathrm{K}^+ + \mathrm{Cr}_2\mathrm{O}_7^{2-}$ 
		\item $\mathrm{H}_3\mathrm{PO}_4 \rightleftharpoons \mathrm{H}^+ + \mathrm{H}_2\mathrm{PO}_4^-$ 
	\end{itemize}
}
\end{bt}
%%==============HetBai_BT5==============%%%

%%==============Bai_BT6==============%%%
\begin{bt}
Viết phương trình điện li trong nước của các chất sau: $\mathrm{AgNO}_3, \mathrm{H}_2\mathrm{SO}_3, \mathrm{Ca(OH)}_2, \mathrm{NH}_4\mathrm{HCO}_3$
\loigiai{
	\begin{itemize}
		\item $\mathrm{AgNO}_3 \rightarrow \mathrm{Ag}^+ + \mathrm{NO}_3^-$ 
		\item $\mathrm{H}_2\mathrm{SO}_3 \rightleftharpoons \mathrm{H}^+ + \mathrm{HSO}_3^-$ 
		\item $\mathrm{Ca(OH)}_2 \rightarrow \mathrm{Ca}^{2+} + 2\mathrm{OH}^-$ 
		\item $\mathrm{NH}_4\mathrm{HCO}_3 \rightarrow \mathrm{NH}_4^+ + \mathrm{HCO}_3^-$ 
	\end{itemize}
}
\end{bt}
%%==============HetBai_BT6==============%%%

%%==============Bai_BT7==============%%%
\begin{bt}
Viết phương trình điện li trong nước của các chất sau: $\mathrm{ZnSO}_4, \mathrm{HCOOH}, \mathrm{LiOH}, \mathrm{NaHSO}_4$
\loigiai{
	\begin{itemize}
		\item $\mathrm{ZnSO}_4 \rightarrow \mathrm{Zn}^{2+} + \mathrm{SO}_4^{2-}$ 
		\item $\mathrm{HCOOH} \rightleftharpoons \mathrm{HCOO}^- + \mathrm{H}^+$ 
		\item $\mathrm{LiOH} \rightarrow \mathrm{Li}^+ + \mathrm{OH}^-$ 
		\item $\mathrm{NaHSO}_4 \rightarrow \mathrm{Na}^+ + \mathrm{HSO}_4^-$ 
	\end{itemize}
}
\end{bt}
%%==============HetBai_BT7==============%%%

%%==============Bai_BT8==============%%%
\begin{bt}
Viết phương trình điện li trong nước của các chất sau: $\mathrm{Fe}_2(\mathrm{SO}_4)_3, \mathrm{HNO}_2, \mathrm{KMnO}_4, \mathrm{(NH}_4)_2\mathrm{CO}_3$
\loigiai{
	\begin{itemize}
		\item $\mathrm{Fe}_2(\mathrm{SO}_4)_3 \rightarrow 2\mathrm{Fe}^{3+} + 3\mathrm{SO}_4^{2-}$ 
		\item $\mathrm{HNO}_2 \rightleftharpoons \mathrm{H}^+ + \mathrm{NO}_2^-$ 
		\item $\mathrm{KMnO}_4 \rightarrow \mathrm{K}^+ + \mathrm{MnO}_4^-$ 
		\item $\mathrm{(NH}_4)_2\mathrm{CO}_3 \rightarrow 2\mathrm{NH}_4^+ + \mathrm{CO}_3^{2-}$ 
	\end{itemize}
}
\end{bt}
%%==============HetBai_BT8==============%%%

%%==============Bai_BT9==============%%%
\begin{bt}
Viết phương trình điện li trong nước của các chất sau: $\mathrm{CuSO}_4, \mathrm{H}_2\mathrm{S}, \mathrm{Ba(ClO}_3)_2, \mathrm{CH}_3\mathrm{COONH}_4$
\loigiai{
	\begin{itemize}
		\item $\mathrm{CuSO}_4 \rightarrow \mathrm{Cu}^{2+} + \mathrm{SO}_4^{2-}$ 
		\item $\mathrm{H}_2\mathrm{S} \rightleftharpoons \mathrm{H}^+ + \mathrm{HS}^-$ )
		\item $\mathrm{Ba(ClO}_3)_2 \rightarrow \mathrm{Ba}^{2+} + 2\mathrm{ClO}_3^-$ 
		\item $\mathrm{CH}_3\mathrm{COONH}_4 \rightarrow \mathrm{CH}_3\mathrm{COO}^- + \mathrm{NH}_4^+$ 
	\end{itemize}
}
\end{bt}
%%==============HetBai_BT9==============%%%

%%==============Bai_BT10==============%%%
\begin{bt}
Viết phương trình điện li trong nước của các chất sau: $\mathrm{Na}_2\mathrm{HPO}_4, \mathrm{HClO}_4, \mathrm{AlCl}_3, \mathrm{NH}_4\mathrm{OH}$
\loigiai{
	\begin{itemize}
		\item $\mathrm{Na}_2\mathrm{HPO}_4 \rightarrow 2\mathrm{Na}^+ + \mathrm{HPO}_4^{2-}$ 
		\item $\mathrm{HClO}_4 \rightarrow \mathrm{H}^+ + \mathrm{ClO}_4^-$ 
		\item $\mathrm{AlCl}_3 \rightarrow \mathrm{Al}^{3+} + 3\mathrm{Cl}^-$ 
		\item $\mathrm{NH}_4\mathrm{OH} \rightleftharpoons \mathrm{NH}_4^+ + \mathrm{OH}^-$ 
	\end{itemize}
}
\end{bt}
%%==============HetBai_BT10==============%%%
\Closesolutionfile{ansbt}
\Closesolutionfile{ansbth}
%\bangdapanSA{AnsBT-H11C01B01-BTTL03}


\phan{Bài tập trắc nghiệm}
%%%=============SOẠN EX===============%%%
\Opensolutionfile{ansex}[Ans/LGEX-filename]
\Opensolutionfile{ans}[Ans/Ans-filename]
\hienthiloigiaiex
%\tatloigiaiex
%\luuloigiaiex
%%%==============EX_01==================%%%
\begin{ex}
	Phương trình điện li nào sau đây là đúng cho một hợp chất điện li yếu trong dung dịch nước?
	\choice
	{$Na_2SO_4 \rightarrow 2\text{Na}^+ + \text{SO}_4^{2-}$}
	{$BaCl_2 \rightarrow \text{Ba}^{2+} + 2\text{Cl}^-$}
	{$H_2SO_4 \rightarrow 2\text{H}^+ + \text{SO}_4^{2-}$}
	{\True $CH_3COOH \rightleftharpoons \text{CH}_3\text{COO}^- + \text{H}^+$}
	\loigiai{Phương trình điện li đúng cho một hợp chất điện li yếu là:
	$CH_3COOH \rightleftharpoons \text{CH}_3\text{COO}^- + \text{H}^+$
	\\
	$CH_3COOH$ là acid yếu, chỉ phân li một phần trong dung dịch nước. Các chất còn lại là điện li mạnh, phân li hoàn toàn.}
\end{ex}
%%%==============EX_02==================%%%
\begin{ex}
	Phương trình điện li nào sau đây là \textbf{đúng} ?
	\choice
	{\True $CH_3COONa \rightarrow \text{CH}_3\text{COO}^- + \text{Na}^+$}
	{$HClO \rightarrow \text{H}^+ + \text{ClO}^-$}
	{$H_2CO_3 \rightarrow 2\text{H}^+ + \text{CO}_3^{2-}$}
	{$H_2S \rightarrow 2\text{H}^+ + \text{S}^-$}
	\loigiai{
	$CH_3COOH$ chất điện li mạnh nên dùng mũi tên một chiều, còn $HClO$,$H_2CO_3$, $H_2S$là chất điện li yếu nên dùng mũi tên hai chiều	}
\end{ex}
%%%==============EX_03==================%%%
\begin{ex}
	Phương trình điện li nào sau đây là \textbf{không đúng} ?
	\choice
	{$HCl \xrightarrow{}  H^+ + Cl^{-}$}
	{$Al_2{(SO_4)}_3 \xrightarrow  2Al^{3+} + 3SO_4^{2-}$}
	{$NaOH \xrightarrow{}  Na^+ + OH^{-}$}
	{\True $H_2SO_4 \xrightarrow  H^+ + HSO_4^{-}$}
	\loigiai{
		$H_2SO_4$ là acid mạnh nên phân li hoàn toàn thành ion $H^+$ và $SO_4^{2-}$
		\[\mathrm{H}_2\mathrm{SO}_4 \xrightarrow  2\mathrm{H}^+ + \mathrm{SO}_4^{-}\]
		}
\end{ex}
\Closesolutionfile{ans}
\Closesolutionfile{ansex}

%%%============Dạng 4================%%%
\begin{dang}{Tính pH của dung dịch}
\end{dang}
\Noibat[][][\faCoffee]{Bài toán 1 pH của dung dịch acid/base mạnh}
\begin{pp}
Đối với bài toán pha trộn thì mới làm thêm bước 2, bước 3
\begin{cacbuoc}
	\item Tính số mol $H^+$ hoặc $OH^-$ trong mỗi dung dịch ban đầu
	\item Tính tổng số mol $H^+$ hoặc $OH^-$ sau khi trộn hoặc pha loãng
	\item Tính nồng độ mới của các ion: $C_M=\dfrac{n}{V_s}$ ($V_s$: Thể tích dung dịch ssau khi trộn, hoặc sau pha loãng).
	\item Tính $pH=-lg[H^+]$ hoặc $pH=14-pOH$ với $pOH =-lg[OH^-]$
\end{cacbuoc}
\end{pp}
{\indam[\maunhan]{Lưu ý:} Khi pha loãng, thể tích tăng bao nhiêu lần thì nồng độ giảm bấy nhiêu lần.}
\begin{center}
\boxct{$\dfrac{C_1}{C_2}=\dfrac{V_2}{V_1}$}
\end{center}
%%=============Ví dụ mấu dạng 4=================%%%
\Noibat[][][\faBookmark]{Ví dụ mẫu}
%%==============VDM1==============%%%
\begin{vd}
Tính pH của các dung dịch sau:
\begin{enumEX}{2}
	\item Dung dịch $\mathrm{NaOH}$ $0,001\;M$;
	\item Dung dịch $\mathrm{HCl}$ $0,01\;M$;
	\item Dung dịch $\mathrm{Mg{(OH)}_2}$ $0,002\;M$.
\end{enumEX}
\loigiai{
	\begin{enumerate}
		\item Dung dịch $\mathrm{NaOH}$ $0{,}001\;M$:
		\[\begin{array}{ccccc}
			NaOH& \xrightarrow& Na^+& +& OH^-\\
			0{,}001&&&\rightarrow&0{,}001
		\end{array}\]
		\begin{align*}
			[\mathrm{OH^-}] &= 0,001 \mathrm{M} \\
			\mathrm{pOH} &= -\log[\mathrm{OH^-}] = -\log(0,001) = 3 \\
			\mathrm{pH} &= 14 - \mathrm{pOH} = 14 - 3 = 11
		\end{align*}
		\item Dung dịch $\mathrm{HCl}$ $0,01\;M$:
		\[\begin{array}{ccccc}
			HCl& \xrightarrow& H^+&+& Cl^-\\
			0{,}01&\rightarrow &0{,}01&&
		\end{array}\]
		\begin{align*}
			[\mathrm{H^+}] &= 0,01 \mathrm{M} \\
			\mathrm{pH} &= -\log[\mathrm{H^+}] = -\log(0,01) = 2
		\end{align*}
		\item Dung dịch $\mathrm{Mg{(OH)}_2}$ $0{,}002\;M$:
		\[\begin{array}{ccccc}
			Mg(OH)_2& \xrightarrow& Mg^{2+}&+& 2OH^-\\
			0{,}002&&&\rightarrow &0{,}004
		\end{array}\]
		\begin{align*}
			[\mathrm{OH^-}] &= 2 \times 0,002 = 0,004 \mathrm{M} \\
			\mathrm{pOH} &= -\log[\mathrm{OH^-}] = -\log(0,004) = 2,4 \\
			\mathrm{pH} &= 14 - \mathrm{pOH} = 14 - 2,4 = 11,6
		\end{align*}
	\end{enumerate}
}
\end{vd}
%%==============HetVDM1==============%%%
%%==============VDM2==============%%%
\begin{vdex}
Dung dịch X là hỗn hợp $Ba{(OH)}_2$ $0{.}1$ M và $NaOH$ $0{.}1$ M. Dung dịch Y là hỗn hợp của $H_2SO_4$ $0{,}0375$ M; $HCl$ $0{,}0125$ M. Trộn $100$ ml dung dịch X với $400$ ml dung dịch Y thu được dung dịch Z. pH của dung dịch Z là
\choice
{$1$}
{$7$}
{$2$}
{$6$}
\loigiai{
\indam{Phân tích:} Bài toán trộn dung dịch, lưu ý phải tính lại nồng độ các chất vì thể tích dung dịch  thay đổi. Xác định chất dư để tính pH theo chất đó.
\\[5pt]
$\left.\begin{aligned}
	Ba(OH)_2:[OH^-]=0{,}2 \;M\\
	NaOH :[OH^-]=0{,}1\; M
\end{aligned}\right\}$ $\Rightarrow$ $\sum[OH^-]=0{,}3$ M $\Rightarrow$ $nOH^-=0{,}3\cdot0{,}1=0{,}03$ mol.
\\
$\left.\begin{aligned}
	H_2SO_4 :[H^+]=0{,}075\;M\\
	HCl:[H^+]=0{,}0125\;M
\end{aligned}\right\}$ $\Rightarrow$ $\sum[H^+]=0{,}0875$ M $\Rightarrow$ $nH^+=0{,}0875\cdot0{,}4=0{,}035$ mol.
\[\begin{matrix}
	& H^+&+& OH^- & \xrightarrow & H_2O\\
	&0{,}03\;\text{mol}&\xleftarrow&0{,}03\;\text{mol}&&
\end{matrix}\]
$\Rightarrow$ $n_{H^+\text{dư}}=\dfrac{0{,}005}{0{,}5}=0{,}01\;M$ $\Rightarrow pH =2$.
}
\end{vdex}
%%==============HetVDM2==============%%%
\Noibat[][][\faCoffee]{Bài toán 2 Tính pH của dung dịch acid/base yếu}
\begin{pp}
\begin{cacbuoc}
	\item Viết phương trình điện li
	\begin{itemize}[wide=0.65cm]
	\item  Đối với acid yếu:
	\[\mathrm{HA} + \mathrm{H}_2\mathrm{O} \xrightleftharpoons{} \mathrm{A}^- + \mathrm{H}_3\mathrm{O}^+ \quad K_a=\dfrac{[A^-][H_3O^+]}{[HA]}\]
	\item Đối với bazo yếu :
		\[\mathrm{B} + \mathrm{H}_2\mathrm{O} \xrightleftharpoons{} \mathrm{BH^+}^- + \mathrm{OH}^- \quad K_b=\dfrac{[BH^+][OH^-]}{[B]}\]
	\end{itemize}
	\item tính nồng độ $H^+$ hoặc $[OH^-]$ thông qua hằng số phân li $K_a$ hoặc $K_b$ theo phương pháp "3 dòng"\\
	\begin{tabular}{cp{1cm}c}
		$\begin{matrix}
		&\mathrm{HA}& +& \mathrm{H}_2\mathrm{O}& \xrightleftharpoons{}& \mathrm{A}^-& +& \mathrm{H}_3\mathrm{O}^+&\\
	\text{ban đầu:}	&a&&&&&&&\\
	\text{phản ứng:}&-x&&&&+x&&+x&\\
	\text{cân bằng:}&a-x&&&&x&&x&
	\end{matrix}$
	&&
	$\begin{matrix}
		&\mathrm{B}& +& \mathrm{H}_2\mathrm{O}& \xrightleftharpoons{}& \mathrm{BH}^+& +& \mathrm{OH}^-&\\
		\text{ban đầu:}	&b&&&&&&&\\
		\text{phản ứng:}&-x&&&&+x&&+x&\\
		\text{cân bằng:}&a-x&&&&x&&x&
	\end{matrix}$\\
		$K_a=\dfrac{x \cdot x}{a-x}$ \quad (1)
	&&
	$K_b=\dfrac{x \cdot x}{b-x}$ \quad (2)
	\end{tabular}
	\item Từ (1) và (2) tính được $[H^+]$ hoặc $[OH^-]$ $\Rightarrow pH$  
\end{cacbuoc}
\end{pp}
\Noibat[][][\faBookmark]{Ví dụ mẫu}
\begin{vd}Tính pH của các dung dịch sau:
	\begin{enumerate}
		\item $CH_3COOH$ $0{,}1$M có $K_a=1{,}75\cdot10^{-5}$.
		\item $NH_3$ $0{,}10$M có $K_b=1{,}80\cdot10^{-5}$.
	\end{enumerate}
	\loigiai{
	\begin{enumerate}
		\item \phantom{x} 
		
		$\begin{matrix}
			&CH_3COOH& +& \mathrm{H}_2\mathrm{O}& \xrightleftharpoons{}& CH_3COO^-&+& H_3O^+&\\
			\text{ban đầu:}	&0{,}1&&&&&&&\\
			\text{phản ứng:}&-x&&&&+x&&+x&\\
			\text{cân bằng:}&0{,}1-x&&&&x&&x&
		\end{matrix}$\\
		Ta có $K_a=\dfrac{x \cdot x}{0{,}1-x} = 1{,}75\cdot10^{-5} \Rightarrow [H^+] = x = 1{,}31\cdot10^{-3} $ $\Rightarrow pH =-log(1{,}31\cdot10^{-3}) = 2{,}88$
		\item \phantom{x} 
		
		$\begin{matrix}
			&\mathrm{NH_3}& +& \mathrm{H}_2\mathrm{O}& \xrightleftharpoons{}& \mathrm{NH_4}^+& +& \mathrm{OH}^-&\\
			\text{ban đầu:}	&0{,}1&&&&&&&\\
			\text{phản ứng:}&-x&&&&+x&&+x&\\
			\text{cân bằng:}&0{,}1-x&&&&x&&x&
		\end{matrix}$\\
		$K_b=\dfrac{x \cdot x}{0{,}1-x} = 1{,}80\cdot10^{-5} $ 
		$\Rightarrow [OH^-] = x = 1{,}33\cdot10^{-3} $ $\Rightarrow [H^+]=\dfrac{10^{-14}}{1{,}33\cdot10^{-3}} = 7{,}5\cdot10^{-12}\\ \Rightarrow pH =-log(7{,}5\cdot10^{-12}) = 11{,}12$
	\end{enumerate}
	}
\end{vd}
\Noibat[][][\faBank]{Bài tập tự luyện dạng \thedang}
\phan{Bài tập tự luận}
%%=============SOẠN BT===============%%%
\Opensolutionfile{ansbth}[Ans/LGBT-H11C01B02-BTTL4]
\Opensolutionfile{ansbt}[Ans/AnsBT-H11C01B02-BTTL4]
\hienthiloigiaibt
%%==============BT_2==============%%%
\begin{bt}
Tính pH của các dung dịch sau:
\begin{enumEX}{2}
	\item Dung dịch $\mathrm{Ca{(OH)}_2}$ $0,02\;M$;
	\item Dung dịch $\mathrm{HNO_3}$ $0,05\;M$;
	\item Dung dịch $\mathrm{LiOH}$ $0,1\;M$;
	\item Dung dịch $\mathrm{H_3PO_4}$ $0,01\;M$;
	\item Dung dịch $\mathrm{Sr{(OH)}_2}$ $0,005\;M$.
\end{enumEX}
\loigiai{
	\begin{enumerate}
		\item Dung dịch $\mathrm{Ca{(OH)}_2}$ $0,02\;M$:
		\[\begin{array}{ccccc}
			Ca(OH)_2& \xrightarrow& Ca^{2+}&+& 2OH^-\\
			0{,}02&&&\rightarrow &0{,}04
		\end{array}\]
		\begin{align*}
			[\mathrm{OH^-}] &= 2 \times 0,02 = 0,04 \mathrm{M} \\
			\mathrm{pOH} &= -\log[\mathrm{OH^-}] = -\log(0,04) = 1,4 \\
			\mathrm{pH} &= 14 - \mathrm{pOH} = 14 - 1,4 = 12,6
		\end{align*}
		\item Dung dịch $\mathrm{HNO_3}$ $0,05\;M$:
		\[\begin{array}{ccccc}
			HNO_3& \xrightarrow& H^+&+& NO_3^-\\
			0{,}05&\rightarrow &0{,}05&&
		\end{array}\]
		\begin{align*}
			[\mathrm{H^+}] &= 0,05 \mathrm{M} \\
			\mathrm{pH} &= -\log[\mathrm{H^+}] = -\log(0,05) = 1,3
		\end{align*}
		\item Dung dịch $\mathrm{LiOH}$ $0,1\;M$:
		\[\begin{array}{ccccc}
			LiOH& \xrightarrow& Li^+& +& OH^-\\
			0{,}1&&&\rightarrow&0{,}1
		\end{array}\]
		\begin{align*}
			[\mathrm{OH^-}] &= 0,1 \mathrm{M} \\
			\mathrm{pOH} &= -\log[\mathrm{OH^-}] = -\log(0,1) = 1 \\
			\mathrm{pH} &= 14 - \mathrm{pOH} = 14 - 1 = 13
		\end{align*}
		\item Dung dịch $\mathrm{H_3PO_4}$ $0,01\;M$:
		\[\begin{array}{ccccc}
			H_3PO_4& \xrightarrow& H^+&+& H_2PO_4^-\\
			0{,}01&\rightarrow&0{,}01& &
		\end{array}\]
		\begin{align*}
			[\mathrm{H^+}] &\approx 0,01 \mathrm{M} \text{ (giả sử phân ly hoàn toàn)} \\
			\mathrm{pH} &= -\log[\mathrm{H^+}] = -\log(0,01) = 2
		\end{align*}
		\item Dung dịch $\mathrm{Sr{(OH)}_2}$ $0,005\;M$:
		\[\begin{array}{ccccc}
			Sr(OH)_2& \xrightarrow& Sr^{2+}&+& 2OH^-\\
			0{,}005&&&\rightarrow &0{,}01
		\end{array}\]
		\begin{align*}
			[\mathrm{OH^-}] &= 2 \times 0,005 = 0,01 \mathrm{M} \\
			\mathrm{pOH} &= -\log[\mathrm{OH^-}] = -\log(0,01) = 2 \\
			\mathrm{pH} &= 14 - \mathrm{pOH} = 14 - 2 = 12
		\end{align*}
	\end{enumerate}
}
\end{bt}

%%==============BT_3==============%%%
\begin{bt}
Tính pH của các dung dịch sau:
\begin{enumEX}{2}
	\item Dung dịch $\mathrm{KOH}$ $0,005\;M$;
	\item Dung dịch $\mathrm{HCl}$ $0,2\;M$;
	\item Dung dịch $\mathrm{Ba{(OH)}_2}$ $0,01\;M$;
	\item Dung dịch $\mathrm{HClO_4}$ $0,001\;M$.
\end{enumEX}
\loigiai{
	\begin{enumerate}
		\item Dung dịch $\mathrm{KOH}$ $0,005\;M$:
		\[\begin{array}{ccccc}
			KOH& \xrightarrow& K^+& +& OH^-\\
			0{,}005&&&\rightarrow&0{,}005
		\end{array}\]
		\begin{align*}
			[\mathrm{OH^-}] &= 0,005 \mathrm{M} \\
			\mathrm{pOH} &= -\log[\mathrm{OH^-}] = -\log(0,005) = 2,3 \\
			\mathrm{pH} &= 14 - \mathrm{pOH} = 14 - 2,3 = 11,7
		\end{align*}
		\item Dung dịch $\mathrm{HCl}$ $0,2\;M$:
		\[\begin{array}{ccccc}
			HCl& \xrightarrow& H^+&+& Cl^-\\
			0{,}2&\rightarrow &0{,}2&&
		\end{array}\]
		\begin{align*}
			[\mathrm{H^+}] &= 0,2 \mathrm{M} \\
			\mathrm{pH} &= -\log[\mathrm{H^+}] = -\log(0,2) = 0,7
		\end{align*}
		\item Dung dịch $\mathrm{Ba{(OH)}_2}$ $0,01\;M$:
		\[\begin{array}{ccccc}
			Ba(OH)_2& \xrightarrow& Ba^{2+}&+& 2OH^-\\
			0{,}01&&&\rightarrow &0{,}02
		\end{array}\]
		\begin{align*}
			[\mathrm{OH^-}] &= 2 \times 0,01 = 0,02 \mathrm{M} \\
			\mathrm{pOH} &= -\log[\mathrm{OH^-}] = -\log(0,02) = 1,7 \\
			\mathrm{pH} &= 14 - \mathrm{pOH} = 14 - 1,7 = 12,3
		\end{align*}
		\item Dung dịch $\mathrm{HClO_4}$ $0,001\;M$:
		\[\begin{array}{ccccc}
			HClO_4& \xrightarrow& H^+&+& ClO_4^-\\
			0{,}001&\rightarrow &0{,}001&&
		\end{array}\]
		\begin{align*}
			[\mathrm{H^+}] &= 0,001 \mathrm{M} \\
			\mathrm{pH} &= -\log[\mathrm{H^+}] = -\log(0,001) = 3
		\end{align*}
	\end{enumerate}
}
\end{bt}

%%==============BT_4==============%%%
\begin{bt}
Tính pH của các dung dịch sau:
\begin{enumEX}{2}
	\item Dung dịch $\mathrm{NaOH}$ $0,02\;M$;
	\item Dung dịch $\mathrm{HNO_3}$ $0,005\;M$;
	\item Dung dịch $\mathrm{Ca{(OH)}_2}$ $0,008\;M$.
\end{enumEX}
\loigiai{
	\begin{enumerate}
		\item Dung dịch $\mathrm{NaOH}$ $0,02\;M$:
		\[\begin{array}{ccccc}
			NaOH& \xrightarrow& Na^+& +& OH^-\\
			0{,}02&&&\rightarrow&0{,}02
		\end{array}\]
		\begin{align*}
			[\mathrm{OH^-}] &= 0,02 \mathrm{M} \\
			\mathrm{pOH} &= -\log[\mathrm{OH^-}] = -\log(0,02) = 1,7 \\
			\mathrm{pH} &= 14 - \mathrm{pOH} = 14 - 1,7 = 12,3
		\end{align*}
		\item Dung dịch $\mathrm{HNO_3}$ $0,005\;M$:
		\[\begin{array}{ccccc}
			HNO_3& \xrightarrow& H^+&+& NO_3^-\\
			0{,}005&\rightarrow &0{,}005&&
		\end{array}\]
		\begin{align*}
			[\mathrm{H^+}] &= 0,005 \mathrm{M} \\
			\mathrm{pH} &= -\log[\mathrm{H^+}] = -\log(0,005) = 2,3
		\end{align*}
		\item Dung dịch $\mathrm{Ca{(OH)}_2}$ $0,008\;M$:
		\[\begin{array}{ccccc}
			Ca(OH)_2& \xrightarrow& Ca^{2+}&+& 2OH^-\\
			0{,}008&&&\rightarrow &0{,}016
		\end{array}\]
		\begin{align*}
			[\mathrm{OH^-}] &= 2 \times 0,008 = 0,016 \mathrm{M} \\
			\mathrm{pOH} &= -\log[\mathrm{OH^-}] = -\log(0,016) = 1,8 \\
			\mathrm{pH} &= 14 - \mathrm{pOH} = 14 - 1,8 = 12,2
		\end{align*}
	\end{enumerate}
}
\end{bt}
%%==============BT_5==============%%%
\begin{bt}
Tính pH của các dung dịch sau:
\begin{enumEX}{2}
	\item Dung dịch $\mathrm{LiOH}$ $0,05\;M$;
	\item Dung dịch $\mathrm{HBr}$ $0,02\;M$;
	\item Dung dịch $\mathrm{Al{(OH)}_3}$ $0,003\;M$;
	\item Dung dịch $\mathrm{HClO_3}$ $0,008\;M$.
\end{enumEX}
\loigiai{
	\begin{enumerate}
		\item Dung dịch $\mathrm{LiOH}$ $0,05\;M$:
		\[\begin{array}{ccccc}
			LiOH& \xrightarrow& Li^+& +& OH^-\\
			0{,}05&&&\rightarrow&0{,}05
		\end{array}\]
		\begin{align*}
			[\mathrm{OH^-}] &= 0,05 \mathrm{M} \\
			\mathrm{pOH} &= -\log[\mathrm{OH^-}] = -\log(0,05) = 1,3 \\
			\mathrm{pH} &= 14 - \mathrm{pOH} = 14 - 1,3 = 12,7
		\end{align*}
		\item Dung dịch $\mathrm{HBr}$ $0,02\;M$:
		\[\begin{array}{ccccc}
			HBr& \xrightarrow& H^+&+& Br^-\\
			0{,}02&\rightarrow &0{,}02&&
		\end{array}\]
		\begin{align*}
			[\mathrm{H^+}] &= 0,02 \mathrm{M} \\
			\mathrm{pH} &= -\log[\mathrm{H^+}] = -\log(0,02) = 1,7
		\end{align*}
		\item Dung dịch $\mathrm{Al{(OH)}_3}$ $0,003\;M$:
		\[\begin{array}{ccccc}
			Al(OH)_3& \xrightarrow& Al^{3+}&+& 3OH^-\\
			0{,}003&&&\rightarrow &0{,}009
		\end{array}\]
		\begin{align*}
			[\mathrm{OH^-}] &= 3 \times 0,003 = 0,009 \mathrm{M} \\
			\mathrm{pOH} &= -\log[\mathrm{OH^-}] = -\log(0,009) = 2,05 \\
			\mathrm{pH} &= 14 - \mathrm{pOH} = 14 - 2,05 = 11,95
		\end{align*}
		\item Dung dịch $\mathrm{HClO_3}$ $0,008\;M$:
		\[\begin{array}{ccccc}
			HClO_3& \xrightarrow& H^+&+& ClO_3^-\\
			0{,}008&\rightarrow &0{,}008&&
		\end{array}\]
		\begin{align*}
			[\mathrm{H^+}] &= 0,008 \mathrm{M} \\
			\mathrm{pH} &= -\log[\mathrm{H^+}] = -\log(0,008) = 2,1
		\end{align*}
	\end{enumerate}
}
\end{bt}
%%%==============Bai_BT1==============%%%
\begin{bt}Tính pH của dung dịch $HClO$ (axit hypochlorous) $0,05$M biết $K_a = 3,0 \cdot 10^{-8}$.
\loigiai{
	$\begin{matrix}
		&HClO& +& \mathrm{H}_2\mathrm{O}& \xrightleftharpoons{}& ClO^-&+& H_3O^+&\\
		\text{ban đầu:}	&0{,}05&&&&&&&\\
		\text{phản ứng:}&-x&&&&+x&&+x&\\
		\text{cân bằng:}&0{,}05-x&&&&x&&x&
	\end{matrix}$
	
	Ta có: $K_a=\dfrac{x \cdot x}{0{,}05-x} = 3{,}0\cdot10^{-8}$
	
	Giả sử $x \ll 0{,}05$, ta có:
	
	$x^2 = 3{,}0\cdot10^{-8} \cdot 0{,}05 = 1{,}5\cdot10^{-9}$
	
	$x = \sqrt{1{,}5\cdot10^{-9}} = 1{,}22\cdot10^{-5}$
	
	Kiểm tra giả thiết: $\dfrac{1{,}22\cdot10^{-5}}{0{,}05} = 2{,}44\cdot10^{-4} \ll 1$ (giả thiết đúng)
	
	Vậy $[H^+] = 1{,}22\cdot10^{-5}$
	
	$pH = -\log[H^+] = -\log(1{,}22\cdot10^{-5}) = 4{,}91$
}
\end{bt}
%%%==============HetBai_BT1==============%%%

%%%==============Bai_BT2==============%%%
\begin{bt}Tính pH của dung dịch $CH_3NH_2$ (methylamine) $0,2$M biết $K_b = 4,38 \cdot 10^{-4}$.
\loigiai{
	$\begin{matrix}
		&CH_3NH_2& +& \mathrm{H}_2\mathrm{O}& \xrightleftharpoons{}& CH_3NH_3^+&+& OH^-&\\
		\text{ban đầu:}	&0{,}2&&&&&&&\\
		\text{phản ứng:}&-x&&&&+x&&+x&\\
		\text{cân bằng:}&0{,}2-x&&&&x&&x&
	\end{matrix}$
	
	Ta có: $K_b=\dfrac{x \cdot x}{0{,}2-x} = 4{,}38\cdot10^{-4}$
	
	Giải phương trình: $x^2 + 4{,}38\cdot10^{-4}x - 8{,}76\cdot10^{-5} = 0$
	
	$x = \dfrac{-4{,}38\cdot10^{-4} + \sqrt{(4{,}38\cdot10^{-4})^2 + 4\cdot8{,}76\cdot10^{-5}}}{2} = 8{,}85\cdot10^{-3}$
	
	Vậy $[OH^-] = 8{,}85\cdot10^{-3}$
	
	$pOH = -\log[OH^-] = -\log(8{,}85\cdot10^{-3}) = 2{,}05$
	
	$pH = 14 - pOH = 14 - 2{,}05 = 11{,}95$
}
\end{bt}
%%%==============HetBai_BT2==============%%%

%%%==============Bai_BT3==============%%%
\begin{bt}Tính pH của dung dịch $HCOOH$ (axit formic) $0,1$M biết $K_a = 1,8 \cdot 10^{-4}$.
\loigiai{
	$\begin{matrix}
		&HCOOH& +& \mathrm{H}_2\mathrm{O}& \xrightleftharpoons{}& HCOO^-&+& H_3O^+&\\
		\text{ban đầu:}	&0{,}1&&&&&&&\\
		\text{phản ứng:}&-x&&&&+x&&+x&\\
		\text{cân bằng:}&0{,}1-x&&&&x&&x&
	\end{matrix}$
	
	Ta có: $K_a=\dfrac{x \cdot x}{0{,}1-x} = 1{,}8\cdot10^{-4}$
	
	Giải phương trình: $x^2 + 1{,}8\cdot10^{-4}x - 1{,}8\cdot10^{-5} = 0$
	
	$x = \dfrac{-1{,}8\cdot10^{-4} + \sqrt{(1{,}8\cdot10^{-4})^2 + 4\cdot1{,}8\cdot10^{-5}}}{2} = 4{,}02\cdot10^{-3}$
	
	Vậy $[H^+] = 4{,}02\cdot10^{-3}$
	
	$pH = -\log[H^+] = -\log(4{,}02\cdot10^{-3}) = 2{,}40$
}
\end{bt}
%%%==============HetBai_BT3==============%%%

%%%==============Bai_BT4==============%%%
\begin{bt}Tính pH của dung dịch $C_6H_5COOH$ (axit benzoic) $0,02$M biết $K_a = 6,3 \cdot 10^{-5}$.
	\loigiai{
		$\begin{matrix}
			&C_6H_5COOH& +& \mathrm{H}_2\mathrm{O}& \xrightleftharpoons{}& C_6H_5COO^-&+& H_3O^+&\\
			\text{ban đầu:}	&0{,}02&&&&&&&\\
			\text{phản ứng:}&-x&&&&+x&&+x&\\
			\text{cân bằng:}&0{,}02-x&&&&x&&x&
		\end{matrix}$
		
		Ta có: $K_a=\dfrac{x\cdot x}{0{,}02-x} = 6{,}3\cdot 10^{-5}$
		
		Giả sử $x << 0{,}02$, ta có:
		
		$x^2 = 6{,}3\cdot 10^{-5} \cdot 0{,}02 = 1{,}26\cdot 10^{-6}$
			
			$x = \sqrt{1{,}26 \cdot 10^{-6}} = 1{,}12\cdot 10^{-3}$
			\\
			Kiểm tra giả thiết: $\dfrac{1{,}12\cdot 10^{-3}}{0{,}02} = 0{,}056 < 0{,}05$ (giả thiết đúng)
			\\
			Vậy $[H^+] = 1{,}12\cdot 10^{-3}$ $\Rightarrow$
			$pH = -\log[H^+] = -\log(1{,}12\cdot 10^{-3}) = 2{,}95$
		}
	\end{bt}
	%%%==============HetBai_BT4==============%%%
	
%%%==============Bai_BT6==============%%%
\begin{bt}Tính pH của dung dịch $C_5H_5N$ (pyridine) $0,15$M biết $K_b = 1,7 \cdot 10^{-9}$.
	\loigiai{
		$\begin{matrix}
			&C_5H_5N& +& \mathrm{H}_2\mathrm{O}& \xrightleftharpoons{}& C_5H_5NH^+&+& OH^-&\\
			\text{ban đầu:}	&0{,}15&&&&&&&\\
			\text{phản ứng:}&-x&&&&+x&&+x&\\
			\text{cân bằng:}&0{,}15-x&&&&x&&x&
		\end{matrix}$
		
		Ta có: $K_b=\dfrac{x \cdot x}{0{,}15-x} = 1{,}7\cdot10^{-9}$
		
		Giả sử $x \ll 0{,}15$, ta có:
		
		$x^2 = 1{,}7\cdot10^{-9} \cdot 0{,}15 = 2{,}55\cdot10^{-10}$
		
		$x = \sqrt{2{,}55\cdot10^{-10}} = 1{,}60\cdot10^{-5}$
		
		Kiểm tra giả thiết: $\dfrac{1{,}60\cdot10^{-5}}{0{,}15} = 1{,}07\cdot10^{-4} \ll 1$ (giả thiết đúng)
		
		Vậy $[OH^-] = 1{,}60\cdot10^{-5}$
		
		$pOH = -\log[OH^-] = -\log(1{,}60\cdot10^{-5}) = 4{,}80$
		$\Rightarrow$
		$pH = 14 - pOH = 14 - 4{,}80 = 9{,}20$
	}
\end{bt}
%%%==============HetBai_BT6==============%%%
%%%=====================BT_07==================%%%
\begin{bt}
	Hệ đệm bicarbonate là một trong những hệ đệm quan trọng nhất trong cơ thể người, đóng vai trò thiết yếu trong việc duy trì pH máu ổn định. Hệ đệm này hoạt động bằng cách ngăn chặn những thay đổi đột ngột trong nồng độ ion hydrogen ($H^+$), giúp bảo vệ các protein và enzyme khỏi biến tính do pH thay đổi.
	\\
	Hệ đệm bicarbonate trong máu được mô tả bởi phương trình:
	\[\mathrm{CO}_2+\mathrm{H}_2 \mathrm{O} \rightleftarrows \mathrm{H}_2 \mathrm{CO}_3 \rightleftarrows \mathrm{HCO}_3^{-}+\mathrm{H}^{+}\]
	Cho biết:
	\begin{itemize}
		\item pKa của $\mathrm{H}_2 \mathrm{CO}_3$ là $6{,}1$.
		\item Nồng độ $\left[\mathrm{HCO}_3^{-}\right]$ trong máu là 24 mM.
		\item Nồng độ $\left[\mathrm{H}_2 \mathrm{CO}_3\right]$ trong máu là $1{,}2$ mM.
	\end{itemize}
	\begin{enumerate}
		\item Tính pH của máu.
		\item Tại sao pH của máu của người bình thường lại giữ ở mức bình thường 
	\end{enumerate}
	\loigiai{
\begin{enumerate}
		\item Tính pH của máu:
		Sử dụng phương trình Henderson-Hasselbalch:
		\[ \text{pH} = \text{pKa} + \log\frac{[\text{bazơ}]}{[\text{axit}]} \]
		\\
		Thay số:
		\begin{align*}
			\text{pH} &= 6{,}1 + \log\frac{[\mathrm{HCO}_3^-]}{[\mathrm{H}_2\mathrm{CO}_3]} \\
			&= 6{,}1 + \log(\frac{24}{1{,}2}) \\
			&= 6{,}1 + \log(20) \\
			&= 6{,}1 + 1{,}3 \\
			&= 7{,}4
		\end{align*}
		\\
		Vậy pH của máu là 7,4.
		\item pH của máu của người bình thường được giữ ở mức bình thường (khoảng 7,35 - 7,45) nhờ các cơ chế sau:
		\begin{itemize}
				\item Hệ đệm hóa học:
			\begin{itemize}
				\item Hệ đệm bicarbonate ($\mathrm{HCO}_3^- / \mathrm{H}_2\mathrm{CO}_3$) là hệ đệm chính trong máu.
				\item Khi có axit được thêm vào máu: $\mathrm{H}^+ + \mathrm{HCO}_3^- \rightarrow \mathrm{H}_2\mathrm{CO}_3 \rightarrow \mathrm{CO}_2 + \mathrm{H}_2\mathrm{O}$
				\item Khi có bazơ được thêm vào máu: $\mathrm{OH}^- + \mathrm{H}_2\mathrm{CO}_3 \rightarrow \mathrm{HCO}_3^- + \mathrm{H}_2\mathrm{O}$
			\end{itemize}
				\item Điều chỉnh hô hấp:
				\begin{itemize}
					\item Khi pH máu giảm, trung tâm hô hấp được kích thích, tăng tần số và độ sâu hô hấp để thải $\mathrm{CO}_2$, làm tăng pH máu.
					\item Khi pH máu tăng, giảm tần số và độ sâu hô hấp để giữ $\mathrm{CO}_2$, làm giảm pH máu.
				\end{itemize}
				
				\item Điều chỉnh thận:
			\begin{itemize}
				\item Thận điều chỉnh nồng độ $\mathrm{HCO}_3^-$ bằng cách tái hấp thu hoặc bài tiết $\mathrm{HCO}_3^-$.
				\item Khi pH máu giảm, thận tăng tái hấp thu và tổng hợp $\mathrm{HCO}_3^-$, đồng thời tăng bài tiết $\mathrm{H}^+$.
				\item Khi pH máu tăng, thận giảm tái hấp thu $\mathrm{HCO}_3^-$ và tăng bài tiết $\mathrm{HCO}_3^-$.
			\end{itemize}
				Sự phối hợp của ba cơ chế trên tạo nên một hệ thống điều hòa pH máu hiệu quả, giúp duy trì pH máu ổn định trong khoảng hẹp 7,35 - 7,45, đảm bảo hoạt động bình thường của các quá trình sinh lý trong cơ thể.
		\end{itemize}
\end{enumerate}
	}
\end{bt}
%%%$==============$Bai_$BT8==============$%%%
\begin{bt}
	Một nông dân có một khu đất trồng rau diện tích $0.5$ hecta. Sau khi kiểm tra, họ nhận thấy pH của đất hiện tại là $5{,}5$, trong khi loại rau họ muốn trồng phát triển tốt nhất ở pH $6.5$. Họ quyết định sử dụng vôi nông nghiệp ($CaCO_3$) để tăng pH của đất.
	\begin{enumerate}
		\item Giải thích tại sao khi bón vôi làm cho pH của đất tăng
		\item Biết rằng:$1$ tấn vôi nông nghiệp trên $1$ hecta đất sẽ làm tăng pH lên $0.5$ đơn vị.Giá vôi nông nghiệp là $1{,}500{,}000$ đồng/tấn.Hãy tính:
		\begin{enumerate}[a)]
			\item Tính lượng vôi (tính theo kg) cần thiết để tăng pH của toàn bộ khu đất từ $5.5$ lên $6.5$.
			\item Tính tổng chi phí để mua đủ lượng vôi cần thiết.
			\item Nếu nông dân chỉ có ngân sách $500{,}000$ đồng, họ có thể điều chỉnh pH của bao nhiêu phần trăm diện tích đất?
		\end{enumerate}
	\end{enumerate}
	\loigiai{
	\begin{enumerate}
		\item Khi bón vôi $\left(\mathrm{CaCO}_3\right)$ vào đất, nó phản ứng với nước và $\mathrm{CO}_2$ trong đất theo phương trình:
		\[
		\mathrm{CaCO}_3+\mathrm{H}_2 \mathrm{O}+\mathrm{CO}_2 \rightarrow \mathrm{Ca}\left(\mathrm{HCO}_3\right)_2
		\]
		Canxi bicacbonat $\left(\mathrm{Ca}\left(\mathrm{HCO}_3\right)_2\right)$ được tạo ra sẽ điện ly trong dung dịch đất:
		\[
		\mathrm{Ca}\left(\mathrm{HCO}_3\right)_2 \rightarrow \mathrm{Ca}^{2+}+2 \mathrm{HCO}_3^{-}
		\]
		Ion bicacbonat $\left(\mathrm{HCO}_3^{-}\right)$phản ứng với ion $\mathrm{H}^{+}$trong đất:
		\[
		\mathrm{HCO}_3^{-}+\mathrm{H}^{+} \rightarrow \mathrm{H}_2 \mathrm{O}+\mathrm{CO}_2
		\]
		Quá trình này làm giảm nồng đô ion $\mathrm{H}^{+}$trong đất, dẫn đến tăng pH (giảm độ. chua).
		\item Số tấn vôi cần thiết để tằng pH lên 1 đơn vị từ $5{,}5$ lên $6{,}5$ là:
		\\
		$\begin{array}{cccc}
			\text{Cứ}& pH\; \uparrow\; 0{,}5 \; \text{đơn vị} & \xleftarrow & \text{1 tấn vôi} / \text{1 hecta }\\
			\text{Vậy}&pH\; \uparrow\; 6{,}5-5{,}5=1 \; \text{đơn vị} & \xrightarrow & \dfrac{1}{0{,}5} = 2\; \text{tấn vôi} / \text{1 hecta }
		\end{array}$
		\\
		- Lượng vôi cần bón cho $0{,}5$ hecta là $0{,}5 \cdot 2 = 1 $ (tấn)
		\\
		- Chi phí phải trả là $1{,}500{,}000 \cdot 1 = 1{,}500{,}000$ (đồng)
		\\
		- Phần trăm diện tích có thể điều chỉnh với ngân sách 500,000 đồng:
		\begin{itemize}
			\item Lượng vôi mua được: $500,000 \div 1,500,000=1 / 3$ tấn
			\item Diện tích có thể điều chỉnh: $1 / 3 \div 1 \times 0.5$ ha $=1 / 6$ ha
			\item Phần trăm diện tích: $(1 / 6 \div 0.5) \times 100 \%=33.33 \%$
		\end{itemize}
	\end{enumerate}
	}
\end{bt}
%%%$==============$HetBai_$BT1==============$%%%
\Closesolutionfile{ansbt}
\Closesolutionfile{ansbth}
%\bangdapanSA{AnsBT-H11C01B02-BTTL4}
%
\phan{Bài tập trắc nghiệm nhiều lựa chọn}
%%%=============SOẠN EX===============%%%
\Opensolutionfile{ansex}[Ans/LGEX-H11C01B02-BTTL4]
\Opensolutionfile{ans}[Ans/Ans-H11C01B02-BTTL4]
\hienthiloigiaiex
%%\tatloigiaiex
%%\luuloigiaiex
%%%=============EX_1=============%%%
\begin{ex}
	Dung dịch X chứa $HCl$ $0{,}2$ M. Dung dịch Y chứa $NaOH$ $0{,}15$ M. Trộn $200$ ml dung dịch X với $300$ ml dung dịch Y thu được dung dịch Z. pH của dung dịch Z là
	\choice
	{$1{,}3$}
	{$2{,}3$}
	{\True $11{,}7$}
	{$12{,}7$}
	\loigiai{%
		\indam{Phân tích:} Đây là trường hợp trộn 1 axit với 1 bazơ. Ta cần tính số mol của $H^+$ và $OH^-$, xác định chất dư và tính pH.
		\\[5pt]
		$n_{H^+} = 0{,}2 \cdot 0{,}2 = 0{,}04$ mol
		\\
		$n_{OH^-} = 0{,}15 \cdot 0{,}3 = 0{,}045$ mol
		\[
		\begin{matrix}
			& H^+&+& OH^- & \xrightarrow & H_2O\\
			&0{,}04\;\text{mol}&\xleftarrow&0{,}04\;\text{mol}&&
		\end{matrix}
		\]
		$\Rightarrow$ $n_{OH^-\text{dư}}=0{,}005$ mol
		\\
		$[OH^-] = \dfrac{0{,}005}{0{,}5} = 0{,}01$ M
		\\
		$pOH = -\log(0{,}01) = 2$
		\\
		$pH = 14 - pOH = 14 - 2 = 12$
	}
\end{ex}
%%%=============EX_2=============%%%
\begin{ex}
	Dung dịch X là hỗn hợp $H_2SO_4$ $0{,}1$ M và $HNO_3$ $0{,}2$ M. Dung dịch Y chứa $Ca(OH)_2$ $0{,}15$ M. Trộn $200$ ml dung dịch X với $300$ ml dung dịch Y thu được dung dịch Z. pH của dung dịch Z là
	\choice
	{$1{,}7$}
	{$2{,}3$}
	{\True $12{,}3$}
	{$13{,}7$}
	\loigiai{%
		\indam{Phân tích:} Đây là trường hợp trộn hỗn hợp 2 axit với 1 bazơ. Ta cần tính tổng số mol $H^+$ từ cả hai axit và số mol $OH^-$ từ bazơ.
		\\[5pt]
		$n_{H^+} = (2 \cdot 0{,}1 + 0{,}2) \cdot 0{,}2 = 0{,}08$ mol
		\\
		$n_{OH^-} = 2 \cdot 0{,}15 \cdot 0{,}3 = 0{,}09$ mol
		\[
		\begin{matrix}
			& H^+&+& OH^- & \xrightarrow & H_2O\\
			&0{,}08\;\text{mol}&\xleftarrow&0{,}08\;\text{mol}&&
		\end{matrix}
		\]
		$\Rightarrow$ $n_{OH^-\text{dư}}=0{,}01$ mol
		\\
		$[OH^-] = \dfrac{0{,}01}{0{,}5} = 0{,}02$ M
		\\
		$pOH = -\log(0{,}02) = 1{,}7$
		\\
		$pH = 14 - pOH = 14 - 1{,}7 = 12{,}3$
	}
\end{ex}
%%%=============EX_3=============%%%
\begin{ex}
	Dung dịch X chứa $HCl$ $0{,}25$ M. Dung dịch Y là hỗn hợp $NaOH$ $0{,}1$ M và $KOH$ $0{,}15$ M. Trộn $300$ ml dung dịch X với $200$ ml dung dịch Y thu được dung dịch Z. pH của dung dịch Z là
	\choice
	{\True $1{,}3$}
	{$2{,}7$}
	{$12{,}7$}
	{$13{,3}$}
	\loigiai{%
		\indam{Phân tích:} Đây là trường hợp trộn 1 axit với hỗn hợp 2 bazơ. Ta cần tính số mol $H^+$ từ axit và tổng số mol $OH^-$ từ cả hai bazơ.
		\\[5pt]
		$n_{H^+} = 0{,}25 \cdot 0{,}3 = 0{,}075$ mol
		\\
		$n_{OH^-} = (0{,}1 + 0{,}15) \cdot 0{,}2 = 0{,}05$ mol
		\[
		\begin{matrix}
			& H^+&+& OH^- & \xrightarrow & H_2O\\
			&0{,}05\;\text{mol}&\xleftarrow&0{,}05\;\text{mol}&&
		\end{matrix}
		\]
		$\Rightarrow$ $n_{H^+\text{dư}}=0{,}025$ mol
		\\
		$[H^+] = \dfrac{0{,}025}{0{,}5} = 0{,}05$ M
		\\
		$pH = -\log(0{,}05) = 1{,}3$
	}
\end{ex}
%%%=============EX_4=============%%%
\begin{ex}
	Dung dịch X là hỗn hợp $H_2SO_4$ $0{,}05$ M và $HNO_3$ $0{,}1$ M. Dung dịch Y là hỗn hợp $Ba(OH)_2$ $0{,}04$ M và $NaOH$ $0{,}1$ M. Trộn $400$ ml dung dịch X với $100$ ml dung dịch Y thu được dung dịch Z. pH của dung dịch Z gần với giá trị nào nhất?
	\choice
	{\True $1$}
	{$2$}
	{$3$}
	{$4$}
	\loigiai{%
		\indam{Phân tích:} Đây là trường hợp trộn hỗn hợp 2 axit với hỗn hợp 2 bazơ. Ta cần tính tổng số mol $H^+$ từ cả hai axit và tổng số mol $OH^-$ từ cả hai bazơ.
		\\[5pt]
		$n_{H^+} = (2 \cdot 0{,}05 + 0{,}1) \cdot 0{,}4 = 0{,}08$ mol
		\\
		$n_{OH^-} = (2 \cdot 0{,}04 + 0{,}1) \cdot 0{,}1 = 0{,}018$ mol
		\[
		\begin{matrix}
			& H^+&+& OH^- & \xrightarrow & H_2O\\
			&0{,}018\;\text{mol}&\xleftarrow&0{,}018\;\text{mol}&&
		\end{matrix}
		\]
		$\Rightarrow$ $n_{H^+\text{dư}}=0{,}062$ mol
		\\
		$[H^+] = \dfrac{0{,}062}{0{,}5} = 0{,}124$ M
		\\
		$pH = -\log(0{,}124) = 0{,}907$
	}
\end{ex}

%%%=============EX_6=============%%%
\begin{ex}[Chuyên Bắc Giang-2018]
	Cho V ml dung dịch $\mathrm{NaOH}$ $0,01\;M$ vào V ml dung dịch $\mathrm{HCl}$ $0,03\;M$ được $2V$ ml dung dịch Y. Dung dịch Y có pH là
	\choice
	{$1$}
	{\True $2$} 
	{$3$}
	{$4$}
	\loigiai{\[\begin{matrix}
		HCl& + &NaOH &\xrightarrow &NaCl& + &H_2O\\
		0{,}01V&\leftarrow&0{,}01V&&&
	\end{matrix}\]
	$\Rightarrow$ $n_{H^+\text{dư}}=0{,}03V - 0{,}01V =0{,}02V$ $\Rightarrow$ $[H^+] = \dfrac{0{,}02V}{2V}=0{,}01$ M $\Rightarrow$ $pH=2$
	}
\end{ex}
%%%$=============EX_7=============$%%%
\begin{ex}[Chuyên Lam Sơn-Thanh Hóa-Lần $1-2018$]
	Dung dịch $HNO_3$ $0{,}1$ M có pH bằng
	\choice
	{$3{,}00$}
	{$2{,}00$}
	{$4{,}00$}
	{\True $1{,}00$}
	\loigiai{\[\begin{matrix}
			HNO_3&\xrightarrow &H^+& + &NO_3^-\\
			0{,}1&\xrightarrow &0{,}1&&
		\end{matrix}\]
		$\Rightarrow$ $pH=-log(0{,}1)=1$
	}
\end{ex}
%%%=============EX_8=============%%%
\begin{ex}
	Dung dịch NaOH có $\mathrm{pH}=10$. Pha loãng dung dịch 10 lần bằng nước thì dung dịch mới pH bằng
	\choice
	{$6$}
	{$7$}
	{$8$}
	{\True $9$}
	\loigiai{%
	$pH = 10 \Rightarrow [H^+]=10^{-10}$ M $\Rightarrow [OH^-] =10^{-4}$ M.
	\\
	Khi pha loãng dung dịch 10 lần thì nồng độ dung dịch giảm đi 10 lần
	\\ 
	$\Rightarrow [OH^-]^\prime = 10^{-5}$ M $\Rightarrow [H^+]^\prime = 10^{-9}$ M $\Rightarrow pH =9$.
	}
\end{ex}
%%%=============EX_9=============$%%%
\begin{ex}
	Cho $200$ ml $H_2SO_4$ $0{,}05$ M vào $300$ ml dung dịch $\mathrm{NaOH}$ $0{,}06$ M. pH của dung dịch tạo thành là
	\choice
	{$2{,}7$}
	{$1{,}6$}
	{$1{,}9$}
	{\True $2{,}4$}
	\loigiai{%
	$n_{H^+}=2n_{H_2SO_4}=2\cdot0{,}2\cdot0{,}05=0{,}02$ mol;
	$n_{OH^-}=2n_{NaOH} =0{,}3\cdot0{,}06=0{,}018$ mol
	\[\begin{matrix}
		H^+&+& OH^-& \xrightarrow & H_2O\\
		0{,}018&\leftarrow&0{,}018&&
	\end{matrix}\]
	$\Rightarrow$ $n_{H^+\text{dư}}= 0{,}02-0{,}018=0{,}002$ mol $\Rightarrow [H^+]=\dfrac{0{,}002}{0{,}5}=0{,}004$ M 
	\\
	$\Rightarrow pH=-log[H^+]=-log(0{,}004)=2{,}4$
	}
\end{ex}
%%=============EX_10=============%%%
\begin{ex}
	Dung dịch có $\mathrm{pH}=3$. Pha loãng dung dịch bằng cách thêm vào $90$ ml nước cất thì dung dịch mới có $\mathrm{pH}=4$. Tính thể tich dung dịch trước khi pha loãng?
	\choice
	{\True $10$ ml}
	{$910$ ml}
	{$100$ ml}
	{$110$ ml}
	\loigiai{%
	Vì khi pha loãng số mol $H^+$ trước và sau pha loãng không đổi
	nên ta có phương trình:
	\\
	$\begin{aligned}
		nH^+ &=1000V\cdot10^{-3} = 1000(V+90)\cdot10^{-4}\\
		\Rightarrow V &= 10\;(ml)
	\end{aligned}$
	}
\end{ex}
%%%=============EX_11=============%%%
\begin{ex}
	Cho mẫu hợp kim K-Ba tác dụng với nước dư thu được dung dịch X và 4,48 lít khí ở đktc. Trung hoà X cần V lít dung dịch HCl có $\mathrm{pH}=1$. Giá trị của V là
	\choice
	{$2$}
	{\True $4$}
	{$6$}
	{$8$}
	\loigiai{
	\begin{equation}\label{eq:Kpunuoc}
		\mathrm{K} + \mathrm{H}_2\mathrm{O} \xrightarrow \mathrm{KOH} + \tfrac{1}{2}\mathrm{H_2}
	\end{equation}
	\begin{equation}\label{eq:Bapunuoc}
		\mathrm{Ba} + \mathrm{2H_2O} \xrightarrow \mathrm{Ba}{(\mathrm{OH})}_\mathrm{2} + \mathrm{H}_2
	\end{equation}
	Theo phương trình (\ref{eq:Kpunuoc}) và (\ref{eq:Bapunuoc}) ta có $n_{OH^-}=2n_{H_2} =2\cdot\dfrac{4{,}48}{22{,}4}=0{,}4$ mol
	\\
	Phản ứng trung hòa : $H^+  + OH^-\xrightarrow H_2O$
	\\
	$n_{H^+}=n_{OH^-} =0{,}4$ mol. $pH =1 \Rightarrow [H^+] =0{,}1$ M.
	$\Rightarrow$ $V_{HCl}= \dfrac{0{,}4}{0{,}1}= 4$ (lít)
	}
\end{ex}
%%%=============EX_12=============%%%
\begin{ex}[Đ$HB-2009$]
	Cho $100$ ml dung dịch hỗn hợp gồm $H_2SO_4$ $0{,}05\;M$ và $\mathrm{HCl}$ $0{,}1\;M$ vào $100$ ml dung dịch hỗn hợp gồm $\mathrm{NaOH}$ $0{,}2\;M$ và $\mathrm{Ba}(OH)_2$ $0{,}1\;M$, thu được dung dịch $X$. Dung dịch $X$ có pH là
	\choice
	{\True $13{,}0$}
	{$1{,}2$}
	{$1{,}0$}
	{$12{,}8$}
	\loigiai{%
	$\begin{aligned}
		n_{H^+}&=2n_{H_2SO_4} + n_{HCl}\\
		&=2\cdot0{,}1\cdot0{,}05 + 0{,}1\cdot0{,}1\\
		&=0{,}02\;\text (mol)
	\end{aligned}$
	\hspace{2cm}
	$\begin{aligned}
		n_{OH^-}&=n_{NaOH} + 2n_{Ba{(OH)}_2}\\
		&=0{,}1\cdot0{,}2 + 2\cdot0{,}1\cdot0{,}1\\
		&=0{,}04\;\text (mol)
	\end{aligned}$
	\\
	$\begin{matrix}
		H^+&+&OH^-&\xrightarrow& H_2O\\
		0{,}02&\xrightarrow&0{,}02&&
	\end{matrix}$
	\\
	$\Rightarrow n_{OH^-\text{dư}}=0{,}04-0{,}02=0{,}02$ (mol)
	$\Rightarrow [OH^-]=\dfrac{0{,}02}{0{,}2}=0{,}1$ (mol)
	\\
	$\Rightarrow pH=14+log[OH^-]=14+log(0{,}1)=13$
	}
\end{ex}
%
%%%$=============EX_13=============$%%%
\begin{ex}
	A là dung dịch $\mathrm{Ba}(OH)_2$ có $\mathrm{pH}=12$. $B$ là dung dịch HCl có $\mathrm{pH}=2$. Phản ứng vừa đủ $V_1$ lít $A$ cần $V_2$ lít. Tìm $V_1/ V_2$?
	\choice
	{\True $1{,}0$}
	{$2{,}0$}
	{$0{,}5$}
	{$2{,}5$}
	\loigiai{%
	dung dịch $Ba{(OH)}_2$ có $pH=12$ $\Rightarrow [OH^-]=\dfrac{10^{-14}}{10^{-12}}=10^{-2}$ (M)
	\\
	$\begin{matrix}
		H^+ & + & OH^-&\xrightarrow &H_2O\\
		10^{-2}V_1&\xrightarrow&10^{-2}V_2&&
	\end{matrix}$
	\\
	$\Rightarrow 10^{-2}V_1 = 10^{-2}V_2\;\text{hay}\;\dfrac{V_1}{V_2}=1$
	}
\end{ex}
%%=============EX_14=============%%%
\begin{ex}[DH B-2008]
	Cho V ml dung dịch $\mathrm{NaOH}$ $0,01$ M vào V ml dung dịch HCl $0,03$ M được $2V$ ml dung dich Y. Dung dich Y có pH là
	\choice
	{$4$}
	{$3$}
	{\True $2$}
	{$1$}
	\loigiai{%
	Phương trình ion thu gọn
	\[\begin{matrix}
		H^+ &+ &OH^-&\xrightarrow & H_2O\\
		10^{-3}V\cdot0{,}01 & \xleftarrow & 10^{-3}V\cdot0{,}01& &
	\end{matrix}\]
	$\Rightarrow n_{H^+\text{dư}} = 3\cdot10^{-5}V-10^{-5}V = 2\cdot10^{-5}V$ (mol)
	$\Rightarrow [H^+] = \dfrac{2\cdot10^{-5}V}{2\cdot10^{-3}V} = 10^{-2}$ (M)
	$\Rightarrow pH= 2$
	}
\end{ex}
%%%=============EX_15=============%%%
\begin{ex}
	Trộn dung dịch $H_2SO_4$ $0{,}1$ M; $HNO_3$ $0{,}2$ M và $\mathrm{HCl}$ $0{,}3$ M với những thể tích bằng nhau thu được dung dịch A.
	Lấy $300$ ml dung dịch A phản ứng với V lít dung dịch B gồm $\mathrm{NaOH}$ $0{,}2$ M và $KOH$ $0{,}29$ M thu được dung dịch $C$ có $\mathrm{pH}=2$. Giá trị của V là
	\choice
	{$0{,}134$}
	{$0{,}112$}
	{$0{,}067$}
	{\True $0{,}414$}
	\loigiai{%
		\indam{Phân tích:} Đây là bài toán trộn 3 axit với 2 bazơ. Ta cần tính tổng số mol $H^+$ từ các axit và tổng số mol $OH^-$ từ các bazơ, sau đó giải phương trình để tìm thể tích V.
		\\[5pt]
		$n_{H^+} = \left(2 \cdot 0{,}1 + 0{,}2 + 0{,}3\right) \cdot 0{,}3 = 0{,}21$ mol
		\\
		$n_{OH^-} = \left(0{,}2 + 0{,}29\right) \cdot V = 0{,}49V$ mol
		\\
		Dung dịch có pH = 2 $\Rightarrow$ $[H^+] = 10^{-2}$ M
		\\
		Ta có phương trình cân bằng số mol:
		\[
		0{,}21 - 0{,}49V = 10^{-2} \times \left(0{,}3 + V\right)
		\]
		\\
		Giải phương trình trên, ta tìm được $V = 0{,}414$ lít
	}
\end{ex}
%%%$==============$Cau_$EX1==============$%%%
\begin{ex}[ĐHA-2009]
	Nung $6{,}58$ gam $\mathrm{Cu}\left(NO_3\right)_2$ trong bình kín không chứa không khí, sau một thời gian thu được $4{,}96$ gam chất rắn và hỗn hợp khí $X$. Hấp thụ hoàn toàn $X$ vào nước để được $300$ ml dung dịch $Y$. Dung dịch $Y$ có pH bằng
	\choice
	{\True $1$}
	{$4$}
	{$3$}
	{$2$}
	\loigiai{%
	$Cu{(NO_3)}_2\xrightarrow[$t^\circ$] CuO + NO_2 + O_2$
	\\
	$\begin{matrix}
		4NO_2 &+& O_2& + &2H_2O& \xrightarrow& 4HNO_3\\
		4x&&x&&&&4x
	\end{matrix}$
	\\
	Áp dụng định luật bỏa toàn khối lượng ta có:
	$m_X=46\cdot4x +32x =6{,}58-4{,}96=1{,}62 \Rightarrow x =0{,}0075$ (mol).
	\\
	$\Rightarrow n_{HNO_3} = n_{NO_2}=4x =0{,}03$ (mol) $\Rightarrow pH =1 $
	}
\end{ex}
%%%$==============$HetCau_$EX1==============$%%%

%%%$==============$Cau_$EX2==============$%%%
\begin{ex}
	Dung dịch HCl có $\mathrm{pH}=5\left(\mathrm{~V}_1\right)$ cho vào dung dịch $KOH$ $\mathrm{pH}=9\left(\mathrm{~V}_2\right)$. Tính $V_1/ V_2$ để dung dịch mới có $\mathrm{pH}=8$?
	\choice
	{$0{,}1$}
	{$10$}
	{$2/ 9$}
	{\True $9/ 11$}
	\loigiai{%
	Dung dịch $HCl$ có $pH=5 \Rightarrow [H^+]=10^{-5}$ (M) $\Rightarrow n_{H^+} = 10^{-5}V_1$ (mol)
	\\
	Dung dịch $KOH$ có $pH=9 \Rightarrow [H^+]=10^{-9}$ (M)
	$\Rightarrow [OH^-]=10^{-5}$ (M)
	$\Rightarrow n_{OH^-} = 10^{-5}V_2$ (mol)
	\\
	Dung dịch sau phản ứng có $pH = 8$ chứng tỏ $KOH$ còn dư sau phản ứng và $n_{OH^-\text{dư}}= 10^{-5} (V_2-V_1)$ (mol)
	\\
	Ta có $pH =8 \Rightarrow [H^+] = 10^{-8}$ M $\Rightarrow [OH^-]_{\text{dư}} = 10^{-6}$ M
	\\
	$\Rightarrow$ $10^{-5} (V_2-V_1)=10^{-6}(V_1+V_2)$ $V_1/V_2=9/11$
	}
\end{ex}
%%%$==============$HetCau_$EX2==============$%%%

%%%$==============$Cau_$EX3==============$%%%
\begin{ex}
	Pha loãng $100$ ml dung dịch NaOH có $\mathrm{pH}=12$ với $900$ ml nước cất thu được dung dịch mới có pH là
	\choice
	{$2$}
	{$12$}
	{\True $11$}
	{$1$}
	\loigiai{%
	Dung dịch có $pH = 12$ $\Rightarrow [H^+] =10^{-12}$ M $\Rightarrow [OH^-]=10^{-2}$ M $\Rightarrow $ $n_{OH^-} = 0{,}1\cdot10^{-2}=10^{-3}$ mol
	\\
	$[OH^-]^\prime = \dfrac{10^{-3}}{1} = 10^{-3}$ M $\Rightarrow [H^+]^\prime = 10^{-11}$ $\Rightarrow pH=11$
	}
\end{ex}
%%%$==============$HetCau_$EX3==============$%%%

%%%$==============$Cau_$EX4==============$%%%
\begin{ex}[$\mathbf{CD}-\mathbf{2011}$]
Cho $a$ lít dung dịch $KOH$ có $\mathrm{pH}=12{,}0$ vào $8{,}00$ lít dung dịch HCl có $\mathrm{pH}=3{,}0$ thu được dung dịch $Y$ có $\mathrm{pH}=11{,}0$. Giá trị của $a$ là
	\choice
	{$1{,}60$}
	{$0{,}80$}
	{\True $1{,}78$}
	{$0{,}12$}
	\loigiai{%
	Dug dịch $KOH$ có $pH=12{,}0$ $\Rightarrow[H^+] =10^{-12}$ M $\Rightarrow [OH^-]= 10^{-2}$ M$\Rightarrow n_{OH^-} = 10^{-2}a$ (mol)
	\\
	Dug dịch $HCl$ có $pH=3{,}0$ $\Rightarrow[H^+] =10^{-3}$ M $\Rightarrow n_{H^+} = 8\cdot10^{-3}$ (mol)
	\\
	Dung dịch sau phản ứng có $pH =11{,}0$ nên $KOH$ còn dư sau phản ứng. $n_{OH^-\text{dư}} = 10^{-2}a-8\cdot10^{-3}$ (mol)\\
	$[OH^-]_{\text{dư}} = \dfrac{10^{-2}a-8\cdot10^{-3}}{a+8}=10^{-2}$
	$\Rightarrow a=1{,}78$
	}
\end{ex}
%%%$==============$HetCau_$EX4==============$%%%

%%%$==============$Cau_$EX5==============$%%%
\begin{ex}[$KB-2008$]
	Trộn $100$ ml dung dịch có $\mathrm{pH}=1$ gồm HCl và $HNO_3$ với $100$ ml dung dịch NaOH nồng độ $a$ $\mathrm{mol} /$ lít, thu được $200$ ml dung dịch có $\mathrm{pH}=12$. Giá trị của $a$ là
	\choice
	{$0{,}15$}
	{$0{,}30$}
	{$0{,}03$}
	{\True $0{,}12$}
	\loigiai{%
	Dung dịch có $pH=1 \Rightarrow [H^+] = 0{,}1$ M $\Rightarrow$ $n_{H^+} = 0{,}1\cdot0{,}1 = 0{,}01$ (mol);
	$n_{NaOH} =0{,}1a$ (mol)
	\\
	Dung dịch sau khi trộn có $pH =12$ nên NaOH còn dư và $n_{NaOH\;\text{dư}} = 0{,}1a - 0{,}01$ (mol)
	\\
	Theo đề bài ta có phương trình $0{,}1a - 0{,}01 = 0{,}2\cdot0{,}02 =0{,}04$ $\Rightarrow a =0{,}12$
	}
\end{ex}
%%%$==============$HetCau_$EX5==============$%%%
\Closesolutionfile{ans}
\Closesolutionfile{ansex}
%\bangdapan{Ans-H11C01B02-BTTL4}
\end{document}