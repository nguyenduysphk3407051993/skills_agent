\documentclass[Main.tex]{subfiles}
\renewcommand{\thefigure}{\arabic{figure}}
\begin{document}
	\begin{tcolorbox}[
		enhanced,frame empty,
		colback=\mycolor!10,
		halign upper= center,
		fontupper=\bfseries\fontfamily{phv}\fontsize{20pt}{6pt}\selectfont,
		colupper=\mycolor!50!black,
		arc is angular,arc=3mm,
		left=6pt,right=6pt,
		left skip =2cm,right skip = 2cm
		]
		Bài tập Kiểm tra giữa kì 1 hóa 10
	\end{tcolorbox}
	\phan{Bài tập trắc nghiệm nhiều lựa chọn}
	%%%=============SOẠN EX===============%%%
	\Opensolutionfile{ansex}[Ans/LGEX-KTGK1HOA10.tex]
	\Opensolutionfile{ans}[Ans/Ans-KTGK1HOA10.tex]
	\begin{ex}
	Hình vẽ (Hình \ref{fig:Sutimraelectron}) sau mô tả thí nghiêm tìm ra môt hạt cấu tao nên nguyên tử. Đó là
		\begin{center}
			\includegraphics[width=6cm]{Images/anhhoahoc10/SU_TIM_RA_ELECTRON.png}
		\captionof{figure}{Thí nghiệm tia âm cực  \label{fig:Sutimraelectron}}
		\end{center}
		\choice
		{Thí nghiệm tìm ra proton}
		{Thí nghiêm tìm ra neutron}
		{Thí nghiệm tìm ra hạt nhân}
		{\True Thí nghiêm tìm ra electron}
		\loigiai{Thí nghiệm tia âm cực tìm ra electron}
	\end{ex}
	%%%==============Cau_EX1==============%%%
	\begin{ex}
		\immini{Phổ khối lượng của Silver (bạc, Ag) như hình bên. Trong tự nhiên Ag có bao nhiêu đồng vị bền?}{
		\resizebox{!}{6cm}{%
			\begin{tikzpicture}
				[%
				line join=round,
				line cap=round,
				declare function={%
					d=0.1cm;%độ dày cột
					hs=0.1;%  Tỉ lệ xích giá trị phần trăm
					hsx=1.25;
					hsy=10;
					xmin=1;
					xmax=5;
					ymin=1;
					ymax=6;
				}
				]
				\pgfmathsetmacro{\xmin}{xmin}
				\pgfmathsetmacro{\xmax}{xmax}
				\pgfmathsetmacro{\ymin}{ymin}
				\pgfmathsetmacro{\ymax}{ymax}
				%% Vẽ 2 truc tọa độ
				\draw [thick,-latex,\maunhan] (0,0)coordinate (xw)--({(xmax+1)*hsx},0)coordinate (xe) node 	[below]{(m/z)};
				\draw [thick,-latex,\maunhan] (0,0)coordinate (ys)--(0,{ymax+0.5})coordinate (yn)node[right]{(\%)};
				%% Các giá trị truc y
				\foreach \y [evaluate=\y as \yt using int(\y*hsy)] in {\ymin,...,\ymax}{%
					\draw[\maunhan!80!black] (0.1,\y)--+(180:0.2) node[left,font=\small\bfseries]{\yt};}
				% Các giá trị truc x
				\foreach \x [count=\i from 106] in {\xmin,...,\xmax}{
					\draw[\maunhan!80!black] (\x*hsx,0.1)--+(-90:.2) node 		[below,font=\small\bfseries]{\i};
				}
				%%%% Vẽ cột
				\foreach \x/\y[evaluate =\y as \yt using \y*hsy] in {
					2/51.8*hs,4/48.2*hs
				}{
					\path[fill=\maunhan!80] ([xshift=-d]\x*hsx,0) rectangle +({2*d},\y) node 	[above,xshift=-d,font=\bfseries\color{\maunhan!60!black}] {\pgfmathprintnumber[fixed, precision=2,use comma]{\yt}};
				}
				%%% Hiển thị thông tin trục y
				\path ([xshift=-1.2cm]ys)--([xshift=-1.2cm]yn) node 	[pos=0.5,midway,sloped,font =\scriptsize \color{\maunhan!50!black}\bfseries\sffamily]{Phần trăm số nguyên tử đồng vị};
				%%% Hiển thị thông tin trục x
				\path ([yshift=-0.9cm]xw)--([yshift=-0.9cm]xe) node 	[pos=0.5,midway,sloped,font 	=\scriptsize\color{\maunhan!50!black}\bfseries\sffamily]{Tỉ số nguyên tử khối điện tích};
			\end{tikzpicture}
			}
		}
		\choice
		{$3$}
		{\True $2$}
		{$5$}
		{$1$}
		\loigiai{
		Dựa vào đồ thị ta thấy $Ag$ có hai đồng vị bền
		}
	\end{ex}
	%%%==============HetCau_EX1==============%%%
	\begin{ex}
		Sự phân bố electron theo orbital nào dưới đây là đúng
		\choice[4pt]
		{\raisebox{-6pt}{\squarerow[2ud,2ud]{3}}}
		{\raisebox{-6pt}{\squarerow[2ud,2du,1u]{3}}}
		{\raisebox{-6pt}{\squarerow[2ud,2du,2ud]{3}}}
		{\True \raisebox{-6pt}{\squarerow[1u,1u]{3}}}
		\loigiai{Sự phân electron vào AO theo đúng nguyên lí  pauli và quy tắc Hund là \raisebox{-6pt}{\squarerow[1u,1u]{3}} }
	\end{ex}
	%%%==============Cau_EX1==============%%%
	\begin{ex}
		Trong nguyên tử, hạt mang điện là
		\choice
		{proton và neutron}
		{\True proton và electron}
		{electron}
		{electron và neutron}
		\loigiai{Trong nguyên tử, hạt mang điện là proton (mang điện dương) và electron (mang điện âm). Neutron là hạt không mang điện.}
	\end{ex}
	%%%==============HetCau_EX1==============%%%
	
	%%%==============Cau_EX2==============%%%
	\begin{ex}
		Hạt nhân của nguyên tử nào có số hạt neutron là 12?
		\choice
		{$^{32}_{15}P$}
		{\True $^{23}_{11}Na$}
		{$^{39}_{19}K$}
		{$^{54}_{26}Fe$}
		\loigiai{Số neutron = Số khối - Số proton. Đối với $^{23}_{11}Na$, số neutron = 23 - 11 = 12.}
	\end{ex}
	%%%==============HetCau_EX2==============%%%
	
	%%%==============Cau_EX3==============%%%
	\begin{ex}
		Số hiệu nguyên tử cho biết thông tin nào sau đây?
		\choice
		{\True Số proton}
		{Số khối}
		{Số neutron}
		{Nguyên tử khối}
		\loigiai{Số hiệu nguyên tử chính là số proton trong hạt nhân của nguyên tử đó.}
	\end{ex}
	%%%==============HetCau_EX3==============%%%
	
	%%%==============Cau_EX4==============%%%
	\begin{ex}
		Cho các kí hiệu nguyên tử sau: $^{39}_{19}X$ và $^{40}_{19}Y$. Nhận xét nào sau đây không đúng?
		\choice
		{X và Y có số khối khác nhau}
		{X và Y có cùng số electron}
		{X và Y là 2 nguyên tử đồng vị}
		{\True X và Y đều có 19 neutron}
		\loigiai{X có 20 neutron (39 - 19), Y có 21 neutron (40 - 19). Vì vậy, nhận xét "X và Y đều có 19 neutron" là không đúng.}
	\end{ex}
	%%%==============HetCau_EX4==============%%%
	
	%%%==============Cau_EX5==============%%%
	\begin{ex}
		Phân lớp 3d có số electron tối đa là
		\choice
		{$6$}
		{$18$}
		{\True $10$}
		{$14$}
		\loigiai{Phân lớp d có 5 orbital, mỗi orbital chứa tối đa 2 electron. Vậy phân lớp 3d có số electron tối đa là 5 x 2 = 10.}
	\end{ex}
	%%%==============HetCau_EX5==============%%%
	
	%%%==============Cau_EX6==============%%%
	\begin{ex}
		Sự phân bố electron vào các lớp và phân lớp căn cứ vào
		\choice
		{số khối tăng dần}
		{điện tích hạt nhân tăng dần}
		{\True mức năng lượng electron}
		{nguyên tử khối tăng dần}
		\loigiai{Sự phân bố electron vào các lớp và phân lớp tuân theo nguyên lý vững bền, theo đó electron sẽ lấp đầy các orbital có mức năng lượng thấp trước.}
	\end{ex}
	%%%==============HetCau_EX6==============%%%
	
	%%%==============Cau_EX7==============%%%
	\begin{ex}
		Số orbital trong các phân lớp s, p, d lần lượt bằng
		\choice
		{2,6, 10}
		{1,2, 3}
		{3,5, 7}
		{\True 1,3, 5}
		\loigiai{Phân lớp s có 1 orbital, phân lớp p có 3 orbital, và phân lớp d có 5 orbital.}
	\end{ex}
	%%%==============HetCau_EX7==============%%%
	
	%%%==============Cau_EX8==============%%%
	\begin{ex}
		Cặp nguyên tử nào dưới đây thuộc cùng một nguyên tố hóa học?
		\choice
		{$^{14}_{\phantom{0}7}G$; $^{16}_{\phantom{0}8}E$}
		{\True $^{16}_{\phantom{0}8}M$; $^{16}_{\phantom{0}8}L$}
		{$^{15}_{\phantom{0}7}D$; $^{22}_{10}Q$}
		{$^{16}_{\phantom{0}8}M$; $^{23}_{11}L$}
		\loigiai{Các nguyên tử của cùng một nguyên tố hóa học phải có cùng số proton (số hiệu nguyên tử). Cặp $^{16}_8M$ và $^{16}_8L$ có cùng số proton là 8, nên thuộc cùng một nguyên tố hóa học.}
	\end{ex}
	%%%==============HetCau_EX8==============%%%
	
	%%%==============Cau_EX9==============%%%
	\begin{ex}
		Bảng tuần hoàn hiện nay không áp dụng nguyên tắc sắp xếp nào sau đây?
		\choice
		{Mỗi nguyên tố hóa học được xếp vào một ô trong bảng tuần hoàn}
		{Các nguyên tố có cùng số lớp electron trên nguyên tử được xếp thành một hàng}
		{Các nguyên tố có cùng số electron hóa trị trong nguyên tử được xếp thành một cột}
		{\True Các nguyên tố được sắp xếp theo chiều tăng dần khối lượng nguyên tử}
		\loigiai{Bảng tuần hoàn hiện nay được sắp xếp theo chiều tăng dần của số hiệu nguyên tử, không phải theo khối lượng nguyên tử.}
	\end{ex}
	%%%==============HetCau_EX9==============%%%
	
	%%%==============Cau_EX10==============%%%
	\begin{ex}
		Ô nguyên tố trong bảng tuần hoàn không cho biết thông tin nào sau đây?
		\choice
		{Kí hiệu nguyên tố}
		{Tên nguyên tố}
		{Số hiệu nguyên tử}
		{\True Số khối của hạt nhân}
		\loigiai{Ô nguyên tố trong bảng tuần hoàn thường chứa kí hiệu nguyên tố, tên nguyên tố và số hiệu nguyên tử. Số khối của hạt nhân không được hiển thị trực tiếp trong ô nguyên tố.}
	\end{ex}
	%%%==============HetCau_EX10==============%%%
	
	%%%==============Cau_EX11==============%%%
	\begin{ex}
		Ngành nào sau đây không liên quan đến hóa học?
		\choice
		{Mĩ phẩm}
		{Năng lượng}
		{Dược phẩm}
		{\True Vũ trụ}
		\loigiai{Mặc dù hóa học có ứng dụng trong nhiều lĩnh vực, bao gồm cả nghiên cứu vũ trụ, nhưng ngành vũ trụ học chủ yếu liên quan đến vật lý và thiên văn học hơn là hóa học.}
	\end{ex}
	%%%==============HetCau_EX11==============%%%
	
	%%%==============Cau_EX12==============%%%
	\begin{ex}
		Nội dung nào dưới đây thuộc đối tượng nghiên cứu của hóa học?
		\choice
		{Sự vận chuyển của máu trong hệ tuần hoàn}
		{Sự tự quay của Trái Đất quanh trục riêng}
		{Sự ra đời và phát triển của nền văn minh lúa nước}
		{\True Sự phá hủy tầng ozone bởi freon-12}
		\loigiai{Sự phá hủy tầng ozone bởi freon-12 là một quá trình hóa học, liên quan đến phản ứng giữa các phân tử freon-12 và ozone trong khí quyển.}
	\end{ex}
	%%%==============HetCau_EX12==============%%%
	
	%%%==============Cau_EX13==============%%%
	\begin{ex}
		Ở trạng thái cơ bản, cấu hình electron của nguyên tử Na (Z=11) là
		\choice
		{$1s^22s^22p^53s^2$}
		{\True $1s^22s^22p^63s^1$}
		{$1s^22s^22p^63s^2$}
		{$1s^22s^22p^43s^1$}
		\loigiai{Na có 11 electron. Cấu hình electron ở trạng thái cơ bản là $1s^22s^22p^63s^1$, tuân theo nguyên lý vững bền và quy tắc Hund.}
	\end{ex}
	%%%==============HetCau_EX13==============%%%
	
	%%%==============Cau_EX14==============%%%
	\begin{ex}
		Các electron của nguyên tử X được phân bố trên 3 lớp, lớp thứ 3 có 6 electron. Số hạt proton trong hạt nhân của nguyên tử của nguyên tố X là
		\choice
		{$6$}
		{$8$}
		{$14$}
		{\True $16$}
		\loigiai{Cấu hình electron của X là $1s^22s^22p^63s^23p^4$. Tổng số electron là 2 + 8 + 6 = 16. Vì nguyên tử trung hòa, số proton cũng bằng 16.}
	\end{ex}
	%%%==============HetCau_EX14==============%%%
	
	%%%==============Cau_EX15==============%%%
	\begin{ex}
		Nguyên tử của nguyên tố X có electron cuối cùng điền vào phân lớp $3p^1$. Nguyên tử của nguyên tố Y có electron cuối cùng điền vào phân lớp $3p^3$. Số hiệu nguyên tử của X và Y lần lượt là
		\choice
		{12 và 15}
		{12 và 14}
		{\True 13 và 15}
		{13 và 15}
		\loigiai{X có cấu hình electron $1s^22s^22p^63s^23p^1$, tổng 13 electron. Y có cấu hình $1s^22s^22p^63s^23p^3$, tổng 15 electron. Số hiệu nguyên tử bằng số electron, nên X và Y có số hiệu nguyên tử lần lượt là 13 và 15.}
	\end{ex}
	%%%==============HetCau_EX15==============%%%
	
	%%%==============Cau_EX16==============%%%
	\begin{ex}
		Bảng tuần hoàn hiện nay có số cột, số nhóm A và số nhóm B lần lượt là
		\choice
		{18,8, 8}
		{\True 18,8, 10}
		{18,10,8}
		{16,8, 8}
		\loigiai{Bảng tuần hoàn hiện đại có 18 cột (nhóm), trong đó có 8 nhóm A (nhóm chính) và 10 nhóm B (nhóm phụ).}
	\end{ex}
	%%%==============HetCau_EX16==============%%%
	
	%%%==============Cau_EX17==============%%%
	\begin{ex}
		Trong các chất dưới đây, chất nào có tính acid yếu nhất?
		\choice
		{$H_2SO_4$}
		{$HClO_4$}
		{\True $H_2SiO_3$}
		{$H_3PO_4$}
		\loigiai{Trong các acid trên, $H_2SiO_3$ (acid silicic) là acid yếu nhất. $H_2SO_4$ và $HClO_4$ là acid mạnh, $H_3PO_4$ là acid trung bình.}
	\end{ex}
	%%%==============HetCau_EX17==============%%%
	\Closesolutionfile{ans}
	\Closesolutionfile{ansex}
	%\bangdapan{Ans-KTGK1HOA10.tex}
	\phan{Bài tập trắc nghiệm đúng sai}
	%%%=============SOẠN EXTF===============%%%
	\Opensolutionfile{ansex}[Ans/LGTF-KTGK1HOA10.tex]
	\Opensolutionfile{ansbook}[Ansbook/AnsTF-KTGK1HOA10.tex]
	\Opensolutionfile{ans}[Ans/Tempt-KTGK1HOA10.tex]
		%%%==============Cau_EX1==============%%%
		\begin{ex}
			Cho các thông tin về nguyên tố Mg trong tự nhiên như sau:
			\choiceTF
			{Trong tự nhiên nguyên tố Magnesium có ba đồng vị bền}
			{Đồng vị $_{25}Mg$ phổ biến nhất so với các đồng vị còn lại}
			{Ba đồng vị bền trên đều ở cùng 1 ô thứ $24$ trong bảng tuần hoàn}
			{Nguyên tử khối trung bình của Mg là $24{,}327$}
			\loigiai{}
		\end{ex}
		%%%==============HetCau_EX1==============%%%
		
		%%%==============Cau_EX2==============%%%
		\begin{ex}
			Hình dưới mô tả orbital (a) và orbital (b) chứa electron trong nguyên tử sodium (Na) ở trạng thái cơ bản. Mức năng lượng của orbital (a) cao hơn orbital (b).
			Cho các phát biểu sau:
			\choiceTF
			{Electron trong các orbital (a) và (b) thuộc cùng lớp electron}
			{Số electron trong 1 orbital (b) gấp ba số electron trong orbital (a)}
			{Electron trên orbital (a) nằm gần hạt nhân hơn electron trên orbital (b)}
			{orbital (a) và (b) Khác nhau về định hướng trong không gian}
			\loigiai{}
		\end{ex}
		%%%==============HetCau_EX2==============%%%
		
		%%%==============Cau_EX3==============%%%
		\begin{ex}
			Cho các nguyên tố X, Y, T, R cùng một chu kỳ và thuộc nhóm A trong bảng tuần hoàn hóa học. Bán kính nguyên tử như hình vẽ sau:
			\choiceTF
			{Nguyên tử có giá trị độ âm điện lớn nhất là T}
			{Nguyên tố có tính kim loại mạnh nhất là Y}
			{Nguyên tử của các nguyên tố này đều có cùng số lớp electron}
			{Hidroxide của X có tính lưỡng tính thì Hidroxide của T có tính base}
			\loigiai{}
		\end{ex}
		%%%==============HetCau_EX3==============%%%
		
		%%%==============Cau_EX4==============%%%
		\begin{ex}
			Tìm hiểu các nguyên tố hóa học Natri (sodium, 11Na) và Potasium(19K) trong bảng tuần hoàn.
			\choiceTF
			{Theo xu hướng biến đổi tính kim loại, K có tính kim loại mạnh hơn Na}
			{Đều thuộc chu kì 3 trong bảng tuần hoàn}
			{Tính base của sodium hydroxide yếu hơn tính base của potasium hydroxide}
			{Na và K đều có tính chất hóa học cơ bản giống nhau}
			\loigiai{}
		\end{ex}
		%%%==============HetCau_EX4==============%%%
	\Closesolutionfile{ans}
	\Closesolutionfile{ansbook}
	\Closesolutionfile{ansex}
	%\bangdapanTF{AnsTF-BTTHC01.tex}
	\phan{Bài tập tự luận}
	%%%=============SOẠN BT===============%%%
	\Opensolutionfile{ansbth}[Ans/LGBT-KTGK1HOA10.tex]
	\Opensolutionfile{ansbt}[Ans/AnsBT-KTGK1HOA10.tex]
		%%%==============Bai_BT1==============%%%
		\begin{bt}
			Nguyên tử của nguyên tố \(X \) có tổng số hạt là 34. Tổng số hạt mang điện nhiều hơn tổng số hạt không mang điện là 10 hạt. Xác định số khối của nguyên tử \(X \)?
			\shortans{}
			\loigiai{}
		\end{bt}
		%%%==============HetBai_BT1==============%%%
		
		%%%==============Bai_BT2==============%%%
		\begin{bt}
			Trong tự nhiên, carbon có hai đồng vị bền là \(^{12}C \) và \(^{13}C \); oxygen có ba đồng vị bền là \(^{16}O \); \(^{17}O \) và \(^{18}O \). Xác định số lượng tối đa loại phân tử \(CO_2 \) có thể tạo ra từ các đồng vị này.
			\shortans{}
			\loigiai{}
		\end{bt}
		%%%==============HetBai_BT2==============%%%
		
		%%%==============Bai_BT3==============%%%
		\begin{bt}
			Nguyên tử cobalt có cấu hình electron ngoài cùng là \(3d^7 4s^2 \). Xác định số hiệu nguyên tử của cobalt?
			\shortans{}
			\loigiai{}
		\end{bt}
		%%%==============HetBai_BT3==============%%%
		
		%%%==============Bai_BT4==============%%%
		\begin{bt}
			Trong tự nhiên đồng có hai đồng vị \(^{63}Cu \) và \(^{65}Cu \). Nguyên tử khối trung bình của đồng là 63,54. Tính số phần trăm khối lượng của \(^{63}Cu \) trong phân tử \(Cu_2O \) (Nguyên tử khối của \(O=16 \)).
			\shortans{}
			\loigiai{}
		\end{bt}
		%%%==============HetBai_BT4==============%%%
		
		%%%==============Bai_BT5==============%%%
		\begin{bt}
			Nguyên tố \(Y \) thuộc chu kì 2 trong bảng tuần hoàn các nguyên tố hóa học. Công thức oxide cao nhất của \(Y \) là \(Y_2O_5 \). Khi cho 1 mol \(Y_2O_5 \) tác dụng với dung dịch NaOH dư thì số mol \(NaOH \) phản ứng là bao nhiêu?
			\shortans{}
			\loigiai{}
		\end{bt}
		%%%==============HetBai_BT5==============%%%
		
		%%%==============Bai_BT6==============%%%
		\begin{bt}
			Có bao nhiêu nguyên tố thuộc chu kì 4 mà nguyên tử có cấu hình electron ở lớp ngoài cùng là \(4s^2 \)?
			\shortans{}
			\loigiai{}
		\end{bt}
		%%%==============HetBai_BT6==============%%%
	\Closesolutionfile{ansbt}
	\Closesolutionfile{ansbth}
	%\bangdapanSA{AnsBT-KTGK1HOA10.tex}
\end{document}