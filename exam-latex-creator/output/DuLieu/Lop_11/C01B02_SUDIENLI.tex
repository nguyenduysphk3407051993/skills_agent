\section{Cân bằng trong dung dịch nước}
\begin{Muctieu}
	\begin{itemize}
		\item  Nêu được khái niệm sự điện li, chất điện li, chất không điện li.
		\item  Trình bày được thuyết Brønsted - Lowry (Bron-stêt - Lau-ri) về acid - base.
		\item  Nêu được khái niệm và ý nghĩa của pH trong thực tiễn (liên hệ giá trị pH ở các bộ phận trong cơ thể với sức khoẻ con người, pH của đất, nước tới sự phát triển của động thực vật, ...).
		\item  Viết được biểu thức tính pH và biết cách sử dụng các chất chỉ thị để xác định pH (môi trường acid, base, trung tính) bẳng các chất chỉ thị phổ biến như giấy chi thị màu, quỳ tím, phenolphthalein, ...
		\item  Nêu được nguyên tắc xác định nồng độ acid, base mạnh bằng phương pháp chuẩn độ.
		\item  Thực hiện được thí nghiệm chuẩn độ acid - base: Chuẩn độ dung dịch base mạnh (sodium hydroxide) bằng dung dich acid mạnh (hydrochloric acid).
		\item  Trình bày được ý nghĩa thực tiễn cân bằng trong dung dịch nước của ion $\mathrm{Al}^{3+}, \mathrm{Fe}^{3+}$ $\mathrm{Và}_{\mathrm{Cl}} \mathrm{CO}_3^{2-}$.
	\end{itemize}
\end{Muctieu}
\subsection{Nội dung bài học}
\begin{kd}
	\immini{Có một anh kĩ sư nông nghiệp trồng cà chua. Anh ấy nhận thấy cây cà chua của mình không phát triển tốt như mong đợi. Sau khi tìm hiểu, anh ấy biết rằng cà chua thích hợp với đất có độ pH từ $6{,}0$ đến $6{,}8$. Anh ấy đo pH của đất trong vườn và thấy nó là $5{,}5$.
		Câu hỏi đặt ra là:
		\begin{itemize}
			\item Tại sao độ pH của đất lại quan trọng đối với sự phát triển của cây trồng?
			\item Điều gì khiến độ pH của đất thay đổi?
			\item Làm thế nào để anh kĩ sư nông nghiệp có thể điều chỉnh pH của đất?
		\end{itemize}
		}{\includegraphics[width=5cm]{Images/anhhoa11/dopH.jpg}}
		Để trả lời những câu hỏi này, chúng ta cần hiểu về cân bằng axit-bazơ trong dung dịch, đặc biệt là trong môi trường đất. Đất là một hệ thống phức tạp, trong đó có nhiều phản ứng hóa học xảy ra đồng thời, bao gồm cả các phản ứng cân bằng.
\end{kd}
\subsubsection{Sự điện li}
\Noibat[\maunhan]{Hiện tượng điện li}
\Noibat[][][\faApple]{Tìm hiểu về sự điện li}
\begin{figure}[!htp]
	\begin{hopdongian}[\mauphu]
		{\indam[\mauphu]{\faAsterisk\;Thí nghiệm về sự điện li}
			\begin{itemize}
				\item \textbf{Chuẩn bị hóa chất:} Kim loại: $Al$, $Cu$; muối: $NaCl$, $CaCl_2$, $CuSO_4$; hạt nhựa PVC, đường, rượu etylic, silic dioxit.
				\item \textbf{Cách tiến hành:}
				\begin{itemize}
					\item Thí nghiệm 1: Đặt điện cực tiếp xúc trực tiếp với các chất trước khi cho nước cất vào.
					\item Thí nghiệm 2: Đặt điện cực tiếp xúc với nước sau khi rót nước vào và khuấy đều.
				\end{itemize}
				Quan sát hiện tượng.
			\end{itemize}
		}
		\begin{center}
			\begin{subfigure}[b]{0.23\textwidth}
				\centering
				\includegraphics[height=0.7\textwidth]{Images/anhhoa11/dodandien_1.png}
				\caption{Trước khi rót nước vào}
				\label{subfig:TNsudienli1}
			\end{subfigure}
			\begin{subfigure}[b]{0.23\textwidth}
				\centering
				\includegraphics[height=0.7\textwidth]{Images/anhhoa11/dodandien_2.png}
				\caption{Đã rót nước và khuấy}
				\label{subfig:TNsudienli2}
			\end{subfigure}
			\hspace*{0.5cm}
			\begin{subfigure}[b]{0.23\textwidth}
				\centering
				\includegraphics[height=0.7\textwidth]{Images/anhhoa11/dodandien_3.png}
				\caption{Trước khi rót nước vào}
				\label{subfig:TNsudienli3}
			\end{subfigure}
			\hspace*{0.25cm}
			\begin{subfigure}[b]{0.23\textwidth}
				\centering
				\includegraphics[height=0.7\textwidth]{Images/anhhoa11/dodandien_4.png}
				\caption{Đã cho nước và khuấy}
				\label{subfig:TNsudienli4}
			\end{subfigure}
			\caption{Thí nghiệm về sự điện li}
			\label{fig:TNsudienli}
		\end{center}
	\end{hopdongian}
\end{figure}
\begin{hoivadap}
	\begin{cauhoi}
		Quan sát ở hình \ref{fig:TNsudienli} hãy trả lời các câu hỏi sau:
	    \begin{enumerate}[(1)]
	    	\item Trước khỉ rót nước vào (thí nghiệm 1), bóng đèn có sáng khi nhúng cặp điện cực vào các chất nào? Tại sao?
	    	Đối với các chất không làm sáng bóng đèn trong thí nghiệm 1, bạn có nhận xét gì về trạng thái của chúng (rắn, lỏng, khí)?
	    	\item Sau khi hòa tan các chất vào nước và khuấy đều (thí nghiệm 2), bóng đèn có sáng khi nhúng cặp điện cực vào dung dịch nào? 
	    	\item So sánh kết quả của thí nghiệm 1 và thí nghiệm 2. Bạn nhận thấy điều gì khác biệt?
	    	\item Đối với các chất làm sáng bóng đèn trong thí nghiệm 2 nhưng không làm sáng trong thí nghiệm 1, điều gì đã xảy ra khi chúng được hòa tan vào nước?Theo bạn, tại sao một số chất chỉ dẫn điện khi được hòa tan vào nước?
	    	\item Nếu một chất khi hòa tan vào nước làm cho dung dịch dẫn điện, bạn nghĩ trong dung dịch đó có gì?
	    \end{enumerate}
	\end{cauhoi}
	\loigiai{%
		\begin{enumerate}[(1)]
			\item Trong thí nghiệm 1 trước khỉ rót nước vào, bóng đèn sáng khi nhúng cặp điện cực vào kim loại $Al$, $Cu$. Đây là do kim loại  có cấu trúc tinh thể với các electron tự do có thể di chuyển, cho phép dòng điện đi qua.
			Các chất không làm sáng bóng đèn như (nước cất,nhựa PVC, rượu, canxi clorua rắn, đồng sunfat khan,...) đều ở trạng thái rắn hoặc lỏng, nhưng không có cấu trúc cho phép các hạt mang điện di chuyển tự do.
			\item Trong thí nghiệm 2, sau khi hòa tan các chất vào nước và khuấy đều, bóng đèn sáng khi nhúng cặp điện cực vào dung dịch Natri clorua,canxi clorua và dung dịch đồng sunfat.
			\item So sánh kết quả của hai thí nghiệm, ta nhận thấy:
			\begin{itemize}
				\item Thí nghiệm 1: Chỉ có kim loại đồng, nhôm dẫn điện.
				\item Thí nghiệm 2: dung dịch natri clorua,canxi clorua và đồng sunfat  dẫn điện.
			\end{itemize}
			Điều này cho thấy một số chất không dẫn điện ở trạng thái rắn, nhưng lại dẫn điện khi hòa tan trong nước.
			\item Khi Natriclorua, canxi clorua và đồng sunfat được hòa tan vào nước, chúng phân li thành các ion. Các ion này có khả năng di chuyển tự do trong dung dịch, cho phép dòng điện đi qua. Một số chất chỉ dẫn điện khi hòa tan trong nước vì nước có khả năng phân li các chất thành các ion mang điện.
			\item Khi một chất hòa tan vào nước làm cho dung dịch dẫn điện, trong dung dịch đó có các ion. Các ion này là các hạt mang điện có khả năng di chuyển tự do trong dung dịch, cho phép dòng điện đi qua. Ví dụ, trong dung dịch canxi clorua, ta có các ion $Ca^{2+}$ và $Cl^-$, còn trong dung dịch đồng sunfat, ta có các ion $Cu^{2+}$ và $SO4^{2-}$.
		\end{enumerate}}
\end{hoivadap}
\begin{tomtat}
	Quá trình phân li các chất trong nước tạo thành ion được gọi là \indam{sự điện li}.
\end{tomtat}
\Noibat[\maunhan]{Chất điện li}
\Noibat[\mauphu][][\faArrowCircleORight]{Chất điện li và chất không điện li}
\\
Thí nghiệm trên cho thấy: các chất như natri clorua, canxi clorua,... tan trong nước phân li ra các ion nên chúng là chất điện li. Saccarose, ethanol,... không phân li ra các ion nên chúng là chất không điện li.
\\
Như vậy trong dung dịch $NaCl$ có chứa các ion $Na^+$ và $Cl^-$ mang điện còn ở dung dịch đường chức các phân tử đường không mang điện
\begin{equation}\label{eq:dienliNaCl}
	\mathrm{NaCl}(s) \rightarrow \mathrm{Na}^{+}(a q)+\mathrm{Cl}^{-}(a q)
\end{equation}
\begin{equation}
	\mathrm{C}_{12} \mathrm{H}_{22} \mathrm{O}_{11}(s) \rightarrow \mathrm{C}_{12} \mathrm{H}_{22} \mathrm{O}_{11}(a q)
\end{equation}
Phương trình (\ref{eq:dienliNaCl}) được gọi là \indam{phương trình điện li}
\vspace{0.25cm}
\begin{tomtat}
	 \begin{itemize}
	 	\item Những chất khi tan trong nước phân li ra các ion được gọi là \indam{chất điện li}
	 	\item Những chất khi tan trong nước không phân li thành các ion được gọi là \indam{Chất không diện li} .
	 \end{itemize}
\end{tomtat}
\Noibat[\mauphu][][\faArrowCircleORight]{Chất điện li mạnh và chất điện li yếu}
\begin{figure}[!htp]
	\begin{hopdongian}[\mauphu]
		{\indam[\mauphu]{\faAsterisk\;Thí nghiệm so sánh khả năng phân li trong nước của $HCl$ và $CH_3COOH$}\\
		- Chuẩn bị 2 dung dịch $\mathrm{HCl}$ $0{,}1\mathrm{M}$ và dung dịch $\mathrm{CH}_3\mathrm{COOH}$ $0{,}1\mathrm{M}$, cắm điện cực vào 2 dung dịch , quan sát hiện tượng xảy ra?
		}
		\begin{center}
			\includegraphics[width=5.5cm,trim={6cm 0cm 0cm 0cm},clip]{Images/anhhoa11/dodandien.jpg}
			\caption{Thí nghiệm so sánh khả năng phân li của $HCl$ và $CH_3COOH$}
			\label{fig:ssdodienli}
		\end{center}
	\end{hopdongian}
\end{figure}
\begin{hoivadap}
	\begin{cauhoi}
		Quan sát và nêu hiện tượng của thí nghiệm?Giải thích
	\end{cauhoi}
		\loigiai{%
		\\
		\textbf{Hiện tượng}:
		Cả hai dung dịch đều dẫn điện, nhưng dung dịch $\mathrm{HCl} 0,1 \mathrm{M}$ dẫn điện tốt hơn dung dịch $\mathrm{CH}_3\mathrm{COOH} 0,1 \mathrm{M}$.\\
		\textbf{Giải thích}:
	\begin{itemize}
		\item Cả $\mathrm{HCl}$ và $\mathrm{CH}_3\mathrm{COOH}$ đều là axit, khi hòa tan trong nước sẽ điện ly tạo ra các ion:
		\begin{align*}
		\mathrm{HCl} \rightarrow \mathrm{H}^+ + \mathrm{Cl}^-\\
		\mathrm{CH}_3\mathrm{COOH} \rightleftharpoons \mathrm{CH}_3\mathrm{COO}^- + \mathrm{H}^+
		\end{align*}
		\item Độ dẫn điện của dung dịch phụ thuộc vào số lượng ion trong dung dịch. Dung dịch có nhiều ion hơn sẽ dẫn điện tốt hơn.Khi nhúng điện cực vào dung dịch $HCl$ đèn sáng mạnh hơn so với dung dịch $CH_3COOH$, điều đó chứng tỏ $HCl$ khi hòa tan vào nước phân li ra nhiều ion hơn so với $CH_3COOH$
		\item \textbf{Kết luận:} Dung dịch $\mathrm{HCl} 0{,}1 \mathrm{M}$ dẫn điện tốt hơn dung dịch $\mathrm{CH}_3\mathrm{COOH} 0{,}1 \mathrm{M}$ vì $\mathrm{HCl}$ là axit mạnh, điện ly hoàn toàn, tạo ra nhiều ion hơn trong dung dịch so với $\mathrm{CH}_3\mathrm{COOH}$ - một axit yếu chỉ điện ly một phần.
	\end{itemize}
	}
\end{hoivadap}
\vspace*{0.25cm}
\begin{tomtat}
	Dựa vào mức độ phân li thành các ion, chất điện li được chia thành hai loại:
	\begin{enumerate}
	\item \indam{Chất điện li mạnh} là chất khi tan trong nước, hầu hết các phân tử chất tan đều phân li ra ion. Các chất điện li mạnh thường gặp là:
\begin{itemize}
		\item \textbf{Các acid mạnh:} $\mathrm{HCl}, \mathrm{HNO}_3, \mathrm{H}_2 \mathrm{SO}_4, \ldots$
		\item \textbf{Các base mạnh:} $\mathrm{NaOH}, \mathrm{KOH}, \mathrm{Ca}(\mathrm{OH})_2, \mathrm{Ba}(\mathrm{OH})_2, \ldots$
		\item \textbf{Hầu hết các muối.}
		
	Quá trình phân li của chất điện li mạnh xảy ra gần như hoàn toàn và được biểu diễn bằng \textbf{mũi tên một chiều}.
	\[
	\begin{aligned}
		& \mathrm{HNO}_3 \longrightarrow \mathrm{H}^{+}+\mathrm{NO}_3^{-} \\
		& \mathrm{NaOH} \longrightarrow \mathrm{Na}^{+}+\mathrm{OH}^{-} \\
		& \mathrm{Na}_2 \mathrm{CO}_3 \longrightarrow 2 \mathrm{Na}^{+}+\mathrm{CO}_3^{2-}
	\end{aligned}
	\]
	\end{itemize}
	\item \indam{Chất điện li yếu} là chất khi tan trong nước chỉ có một phần số phân tử chất tan phân li ra ion, phần còn lại vẫn tồn tại ở dạng phân tử trong dung dịch.
	\\
	Ví dụ: trong dung dịch $\mathrm{CH}_3 \mathrm{COOH} 0,1 \mathrm{M}$, cứ 1000 phân tử hoà tan thì chỉ có 3 phân tử phân li thành ion, còn lại tồn tại ở dạng phân tử.
	\\
	Những chất điện li yếu gồm :
	\begin{itemize}
		\item \textbf{Các acid yếu:} $\mathrm{CH}_3 \mathrm{COOH}, \mathrm{HClO}, \mathrm{HF}, \mathrm{H}_2 \mathrm{CO}_3, \ldots$  
		\item \textbf{Các base yếu:} $\mathrm{Cu}(\mathrm{OH})_2, \mathrm{Fe}(\mathrm{OH})_2, \ldots$
	\end{itemize}
	Quá trình phân li của chất điện li yếu là một phản ứng thuận nghịch và được biểu diễn bằng \textbf{hai nửa mũi tên ngược chiều nhau}
	$$
	\mathrm{CH}_3 \mathrm{COOH} \rightleftharpoons \mathrm{H}^{+}+\mathrm{CH}_3 \mathrm{COO}^{-} .
	$$
	\end{enumerate}
\end{tomtat}
\begin{hopvidu}
	\Noibat[][][\faArchive]{Cơ chế quá trình điện li}\\
	Nước đóng vai trò quan trọng trong sự điện li của một chất. Điều này được giải thích bởi nước là phân tử phân cực (các nguyên tử H mang một phần điện tích dương và nguyên tử O mang một phần điện tích âm) nên khi hoà tan một chất điện li vào nước, xuất hiện tương tác của nước với các ion. Tương tác này sẽ bứt các ion khỏi tinh thể (hoặc phân tử) đế tan vào nước.
	\begin{center}
		\includegraphics[height=4cm]{Images/anhhoa11/hoatanNaCl_2.jpg}
		\captionof{figure}{Quá trình điện li NaCl trong nước}
	\end{center}
\end{hopvidu}
\subsubsection{Thuyết acid-base của bronsted - lowry}
\Noibat[\maunhan]{Khái niệm acid và base theo bronsted - lowry}
%%%==============Bai_BT1==============%%%
\begin{hoivadap}
	\begin{cauhoi}
		Cho các dung dịch: $\mathrm{HCl}, \mathrm{NaOH}, \mathrm{Na_2CO_3}$.
		\begin{enumerate}
			\item Viết phương trình điện li của các chất trên.
			\item Theo khái niệm acid-base trong môn Khoa học tự nhiên ở lớp 8, trong những chất cho ở trên: Chất nào là acid? Chất nào là base?
			\item Sử dụng máy đo pH (hoặc giấy pH) xác định pH, môi trường (acid/base) của các dung dịch trên.
		\end{enumerate}
	\end{cauhoi}
	\loigiai{
		\begin{enumerate}
			\item Phương trình điện li của các chất:
			\[
			\mathrm{HCl} \rightarrow \mathrm{H}^+ + \mathrm{Cl}^-
			\]
			\[
			\mathrm{NaOH} \rightarrow \mathrm{Na}^+ + \mathrm{OH}^-
			\]
			\[
			\mathrm{Na_2CO_3} \rightarrow 2\mathrm{Na}^+ + \mathrm{CO_3}^{2-}
			\]
			\item Theo khái niệm acid-base trong môn Khoa học tự nhiên ở lớp 8:
			\begin{itemize}
				\item HCl là acid vì nó giải phóng ion H\(^+\) khi hòa tan trong nước.
				\item NaOH là base vì nó giải phóng ion OH\(^-\) khi hòa tan trong nước.
				\item Na\(_2\)CO\(_3\) không phải là bazơ vì khi tan vào nước nó không phân ly ra $OH^-$.
			\end{itemize}
			\item Sử dụng máy đo pH (hoặc giấy pH) xác định:
			\begin{itemize}
				\item Dung dịch HCl: pH < 7, môi trường acid.
				\item Dung dịch NaOH: pH > 7, môi trường base.
				\item Dung dịch Na\(_2\)CO\(_3\): pH > 7, môi trường base.
			\end{itemize}
	\end{enumerate}}
\end{hoivadap}
\begin{hoivadap}
	\begin{cauhoi}
	Nếu theo quan điểm cũ, Na\(_2\)CO\(_3\) không phải là bazơ vì không phân ly ra ion OH\(^-\). Tuy nhiên, khi đo pH, ta lại thấy dung dịch Na\(_2\)CO\(_3\) có tính bazơ. Điều này có mâu thuẫn không?
	\end{cauhoi}
	\loigiai{Khái niệm acid-base đề cập ở lớp 8 chỉ đúng với dung môi nước và chưa phản ánh đầy đủ bản chất acid/base. Năm 1923, nhà hoá học người Đan Mạch J. Brønsted (Bronstết) và nhà hoá học người Anh T. Lowry (Lao-ri) đã đưa ra một định nghĩa tổng quát hơn về acid, base.}
\end{hoivadap}
\vspace{0.25cm}
\begin{tomtat}
	\indam{Thuyết Brønsted - Lowry:} \indam[\mauphu]{acid} là chất \indam[\mauphu]{cho proton} $\left(\mathrm{H}^{+}\right)$ và \indam[\mauphu]{base} là chất \indam[\mauphu]{nhận proton}. Acid và base có thể là phân tử hoặc ion.
\end{tomtat}

\begin{vidu}
	\begin{enumerate}
		\item \begin{tikzpicture}[anchor=base, baseline]
		\tikzstyle{mynode} =[
		font=\bfseries,
		anchor=base,
		align =center,
		minimum width = 1cm,
		minimum height = .65cm
		]
		%%%==================%%%
		\tikzstyle{mymatrix} = [
		matrix of nodes,
		nodes in empty cells,
		nodes={mynode},
		column sep=-\pgflinewidth,
		row sep = -\pgflinewidth
		]
		%%%=======================================================%%%
		\matrix(m) [mymatrix]{
		$\mathsf{\color{\maunhan}HCl}$	& + & $\mathsf{H_2O}$ & $\xleftrightarrow$ & $\mathsf{\color{\maunhan}H_3O^+}$ &  + & $\mathsf{Cl^-}$\\
		};
		\path [draw,-stealth,line width=1pt] ([xshift=-0.25cm]m-1-1.south) |-([yshift=-0.5cm]m-1-2.south)node[above]{$H^+$}-|(m-1-3.south) ;
	\end{tikzpicture}
	\\
	Trong phản ứng trên: HCl cho $\mathrm{H}^{+}, \mathrm{HCl}$ là acid; $\mathrm{H}_2 \mathrm{O}$ nhận $\mathrm{H}^{+}, \mathrm{H}_2 \mathrm{O}$ là base.
	%%%
	\item \begin{tikzpicture}[anchor=base, baseline]
		\tikzstyle{mynode} =[
		font=\bfseries,
		anchor=base,
		align =center,
		minimum width = 1cm,
		minimum height = .65cm
		]
		%%%==================%%%
		\tikzstyle{mymatrix} = [
		matrix of nodes,
		nodes in empty cells,
		nodes={mynode},
		column sep=-\pgflinewidth,
		row sep = -\pgflinewidth
		]
		%%%=======================================================%%%
		\matrix(m) [mymatrix]{
			$\mathsf{NH_3}$	& + & $\mathsf{\color{\maunhan}H_2O}$ & $\xleftrightarrow$ & $\mathsf{\color{\maunhan}NH_4^+}$ &  + & $\mathsf{OH^-}$\\
		};
		\path [draw,<-,line width=1pt,>=stealth] ([xshift=-0.25cm]m-1-1.south) |-([yshift=-0.5cm]m-1-2.south)node[above,xshift=-0.25cm]{$H^+$}-|([xshift=-0.25cm]m-1-3.south) ;
	\end{tikzpicture}
	\\
	Trong phản ứng trên: $H_2O$ cho $\mathrm{H}^{+}, \mathrm{H_2O}$ là acid; $\mathrm{NH}_3$ nhận $\mathrm{H}^{+}$, $\mathrm{NH}_3$ là base.
	%%%
		%%%
	\item \begin{tikzpicture}[anchor=base, baseline]
		\tikzstyle{mynode} =[
		font=\bfseries,
		anchor=base,
		align =center,
		minimum width = 1cm,
		minimum height = .65cm
		]
		%%%==================%%%
		\tikzstyle{mymatrix} = [
		matrix of nodes,
		nodes in empty cells,
		nodes={mynode},
		column sep=-\pgflinewidth,
		row sep = -\pgflinewidth
		]
		%%%=======================================================%%%
		\matrix(m) [mymatrix]{
			$\mathsf{CO_3^{2-}}$	& + & $\mathsf{\color{\maunhan}H_2O}$ & $\xleftrightarrow$ & $\mathsf{\color{\maunhan}HCO_3^-}$ &  + & $\mathsf{OH^-}$\\
		};
		\path [draw,<-,line width=1pt,>=stealth] ([xshift=-0.25cm]m-1-1.south) |-([yshift=-0.5cm]m-1-2.south)node[above,xshift=-0.25cm]{$H^+$}-|([xshift=-0.25cm]m-1-3.south) ;
		\path [draw,->,line width=1pt,>=stealth] ([xshift=-0.45cm]m-1-5.north) |-([yshift=0.5cm]m-1-6.north)node[below,xshift=-0.35cm]{$H^+$}-|([xshift=-0.25cm]m-1-7.north) ;
	\end{tikzpicture}
	\\
	Trong phản ứng thuận, $\mathrm{CO}_3^{2-}$ nhận $\mathrm{H}^{+}$của $\mathrm{H}_2 \mathrm{O}, \mathrm{CO}_3^{2-}$ là base, $\mathrm{H}_2 \mathrm{O}$ là acid. Trong phản ứng nghịch, ion $\mathrm{HCO}_3^{-}$là acid, ion $\mathrm{OH}^{-}$là base.
	\end{enumerate}
\end{vidu}
%%%==============Bai_BT1==============%%%
\begin{hoivadap}
	\begin{cauhoi}
		Dựa vào thuyết acid-base của Brønsted-Lowry, hãy xác định chất nào là acid, chất nào là base trong các phản ứng sau:
		\begin{enumerate}
			\item $\mathrm{CH_3COOH} + \mathrm{H_2O} \rightleftharpoons \mathrm{CH_3COO^{-}} + \mathrm{H_3O^{+}}$
			\item $\mathrm{S^{2-}} + \mathrm{H_2O} \rightleftharpoons \mathrm{HS^{-}} + \mathrm{OH^{-}}$
		\end{enumerate}
	\end{cauhoi}
	\loigiai{
		Dựa vào thuyết acid-base của Brønsted-Lowry, một acid là chất cho proton (H\(^+\)) và một base là chất nhận proton. Áp dụng định nghĩa này vào từng phản ứng:
		\begin{enumerate}
			\item $\mathrm{CH_3COOH} + \mathrm{H_2O} \rightleftharpoons \mathrm{CH_3COO^{-}} + \mathrm{H_3O^{+}}$
			\begin{itemize}
				\item $\mathrm{CH_3COOH}$ là acid vì nó cho proton (H\(^+\)) để tạo thành $\mathrm{CH_3COO^{-}}$.
				\item $\mathrm{H_2O}$ là base vì nó nhận proton (H\(^+\)) để tạo thành $\mathrm{H_3O^{+}}$.
			\end{itemize}
			\item $\mathrm{S^{2-}} + \mathrm{H_2O} \rightleftharpoons \mathrm{HS^{-}} + \mathrm{OH^{-}}$
			\begin{itemize}
				\item $\mathrm{S^{2-}}$ là base vì nó nhận proton (H\(^+\)) từ nước để tạo thành $\mathrm{HS^{-}}$.
				\item $\mathrm{H_2O}$ là acid vì nó cho proton (H\(^+\)) để tạo thành $\mathrm{OH^{-}}$.
			\end{itemize}
		\end{enumerate}
	}
\end{hoivadap}
\Noibat[\maunhan]{Ưu điểm của thuyết bronsted - lowry }
\vspace{0.25cm}
\begin{tomtat}
	Ưu điểm của thuyết acid-base Bronsted-Lowry so với thuyết Arrhenius:
	\begin{enumerate}
		\item  Phạm vi áp dụng rộng hơn: Thuyết Bronsted-Lowry không chỉ áp dụng cho các phản ứng trong dung dịch nước như thuyết Arrhenius, mà còn có thể áp dụng cho các phản ứng trong môi trường khác như dung dịch amoniac, dung dịch acid anhhydric, etc.
		\item  Định nghĩa acid-base cụ thể hơn: Theo Bronsted-Lowry, acid là chất cho proton (H+), base là chất nhận proton. Định nghĩa này rõ ràng hơn và bao quát hơn so với định nghĩa acid-base của Arrhenius chỉ dựa trên sự tan trong nước.
		\item  Phản ứng acid-base không chỉ xảy ra trong dung dịch, mà còn có thể xảy ra trong các chất rắn, khí.
		\item  Thuyết Bronsted-Lowry giải thích được nhiều hiện tượng mà thuyết Arrhenius không thể giải thích, như phản ứng tạo phức men-chất nền, phản ứng trao đổi prôton.
	\end{enumerate}
\end{tomtat}
\subsubsection{pH và chất chỉ thị acid - Base}
\Noibat[][][]{Tìm hiểu về pH}
\\
Nước là chất điện li yếu: 
\[
\mathrm{H}_2 \mathrm{O} \rightleftharpoons \mathrm{H}^{+}+\mathrm{OH}^{-}
\]
Tích số ion của nước, kí hiệu $\mathrm{K}_{\mathrm{w}}$ :
\begin{center}
	\boxct{$\mathrm{K}_{\mathrm{w}}=\left[\mathrm{H}^{+}\right]\left[\mathrm{OH}^{-}\right]$}
\end{center}
Ở $25^{\circ} \mathrm{C}, \mathrm{K}_{\mathrm{w}}=\left[\mathrm{H}^{+}\right]\left[\mathrm{OH}^{-}\right]=10^{-14}$. Đối với nước nguyên chất có $ [H^+] = [OH^-] =10^{-7}$.
\\
Nồng độ ion $\mathrm{H}^{+}$hoặc ion $\mathrm{OH}^{-}$được dùng để đánh giá tính acid hoặc tính base của các dung dịch. Tuy nhiên, nếu các dung dịch có nồng độ $\mathrm{H}^{+}$, nồng độ $\mathrm{OH}^{-}$thấp, chúng là những số có số mũ âm hoặc có nhiều chữ số thập phân. Vì vậy, để tiện sử dụng, người ta dùng đại lượng pH với quy ước như sau:
\[
\hopcttoan{\mathrm{pH}=-\lg \left[\mathrm{H}^{+}\right] \text {hoặc }\left[\mathrm{H}^{+}\right]=10^{-\mathrm{pH}}}
\]
Trong đó $\left[\mathrm{H}^{+}\right]$là nồng độ mol của ion $\mathrm{H}^{+}$.
Nếu dung dịch có $\left[\mathrm{H}^{+}\right]=10^{-\mathrm{a}} \mathrm{mol} / \mathrm{L}$ thì $\mathrm{pH}=\mathrm{a}$.
\begin{hopdongian}
	Dựa vào nồng độ $H^+$  có thể đánh giá môi trường của dung dịch
	\begin{enumerate}
		\item  Môi trường acid là môi trường có $\left[\mathrm{H}^{+}\right]>\left[\mathrm{OH}^{-}\right]$nên $\left[\mathrm{H}^{+}\right]>10^{-7}\mathrm{~mol}/\mathrm{L}$ hay $\mathrm{pH}<7$.
		\item  Môi trường base là môi trường có $\left[\mathrm{H}^{+}\right]<\left[\mathrm{OH}^{-}\right]$nên $\left[\mathrm{H}^{+}\right]<10^{-7}\mathrm{~mol}/\mathrm{L}$ hay $\mathrm{pH}>7$.
		\item  Môi trường trung tính là môi truờng có $\left[\mathrm{H}^{+}\right]=\left[\mathrm{OH}^{-}\right]=10^{-7} \mathrm{~mol}/\mathrm{L}$ hay $\mathrm{pH}=7$.
	\end{enumerate}
\end{hopdongian}
\noindent Thang pH thường dùng có giá trị từ 1 đến 14
	\begin{hopdongian}[\mauphu]
		\begin{center}
		\includegraphics[width=12cm]{Images/anhhoa11/pHscale.png}
		\captionof{figure}{thang pH\label{fig:pHScale}}
		\end{center}
	\end{hopdongian}
\Noibat[][][]{Ý nghĩa pH trong thực tiễn}
\begin{tcolorbox}[
	enhanced jigsaw,
	frame hidden,
	colback=\mycolor!15,
	arc is angular,
	breakable
	]
	Thang đo pH là một công cụ quan trọng để duy trì sự cân bằng và an toàn trong nhiều lĩnh vực của cuộc sống hàng ngày.
\begin{enumerate}
	\item \indam{Kiểm soát chất lượng nước:}
	pH của nước ảnh hưởng đến sự sống của các sinh vật trong nước. Ví dụ, nước có pH quá thấp (acid) có thể gây hại cho cá và các sinh vật thủy sinh. Trong xử lý nước, kiểm tra và điều chỉnh pH là cần thiết để đảm bảo nước an toàn cho con người và động vật.
	\item \indam{Sản xuất thực phẩm và đồ uống:}
	 pH ảnh hưởng đến hương vị, màu sắc và độ an toàn của thực phẩm. Ví dụ, pH của rượu vang, sữa chua và pho mát phải được kiểm soát để đảm bảo chất lượng sản phẩm. Ngoài ra, trong ngành công nghiệp đồ uống, pH còn ảnh hưởng đến sự phát triển của vi khuẩn và quá trình lên men.
	\item \indam{Dược phẩm và y học:}
	pH có vai trò quan trọng trong việc hấp thu và hiệu quả của thuốc. Một số loại thuốc chỉ hoạt động tốt ở một mức pH nhất định, do đó việc điều chỉnh pH của môi trường là cần thiết để đảm bảo thuốc có tác dụng.
	\item \indam{Nông nghiệp:}
	 Đất có pH phù hợp là yếu tố quan trọng đối với sự phát triển của cây trồng. Đất có pH quá cao hoặc quá thấp có thể ảnh hưởng đến khả năng hấp thụ dinh dưỡng của cây. Nông dân thường kiểm tra pH của đất để điều chỉnh phân bón và cải tạo đất.
	\item \indam{Mỹ phẩm:}
	pH của các sản phẩm chăm sóc da và tóc ảnh hưởng đến tính an toàn và hiệu quả của chúng. Các sản phẩm như sữa rửa mặt, dầu gội, và kem dưỡng da cần có pH tương thích với da và tóc để tránh kích ứng và bảo vệ lớp màng acid tự nhiên.
	\item \indam{Xử lý chất thải:}
	pH của chất thải phải được kiểm soát trước khi xả ra môi trường để ngăn ngừa ô nhiễm. Nước thải công nghiệp thường phải được trung hòa pH trước khi xả ra hệ thống thoát nước hoặc sông ngòi.
\end{enumerate}
\end{tcolorbox}
\Noibat[][][]{Chất chỉ thị axit-base}
\vspace{0.5cm}
\begin{tomtat}
	Chất chỉ thị acid - base là chất có màu sắc biến đổi theo giá trị pH của dung dịch.\\
	Một số chất chỉ thị axit-bazo phổ biến:
	\begin{itemize}
		\item Quỳ tím Trong môi trường acid (pH < 7), quỳ tím chuyển sang màu đỏ.Trong môi trường base (pH > 7), quỳ tím chuyển sang màu xanh.
		\item  Phenolphthalein: Chuyển từ không màu (pH < 7) sang hồng (pH > 8.2).
		\item  Methyl orange: Chuyển từ đỏ (pH < 3.1) sang vàng (pH > 4.4).
		\item  Bromothymol blue: Chuyển từ vàng (pH < 6.0) sang xanh dương (pH > 7.6).
	\end{itemize}
\end{tomtat}
%%%
\begin{Bancobiet}
	\begin{center}
		\Large\indam{Mối quan hệ giữa pH và màu sắc của hoa cẩm tú cầu}
	\end{center}
	\immini{Hoa cẩm tú cầu (Hydrangea) có thể thay đổi màu sắc dựa trên độ pH của đất nơi chúng được trồng. Sự thay đổi màu sắc chủ yếu được điều chỉnh bởi các hợp chất anthocyanins trong hoa và ion nhôm ($Al^{3+}$) trong đất. 
		\begin{itemize}
			\item Trong đất acid (pH dưới $6.0$): Hoa cẩm tú cầu thường có màu xanh dương.
			\item Trong đất trung tính (pH khoảng $6.0 - 7.0$): Hoa có thể có màu tím nhạt hoặc xanh dương nhạt. 
			\item Trong đất kiềm (pH trên $7.0$): Hoa cẩm tú cầu thường chuyển sang màu hồng hoặc đỏ. 
		\end{itemize}
		Điều chỉnh độ pH của đất có thể giúp thay đổi màu sắc của hoa: thêm các vật liệu làm acid đất như nhôm sulfate để tạo màu xanh dương, hoặc thêm vôi nông nghiệp để làm tăng pH và tạo màu hồng. Các yếu tố khác như loại đất, tình trạng dinh dưỡng và lượng nước cũng ảnh hưởng đến màu sắc của hoa.
		}{\includegraphics[height=7.5cm,trim={4.2cm 0cm 4.2cm 0cm},clip]{Images/anhhoa11/hoacamtucau.png}}
\end{Bancobiet}
\subsubsection{Sự thủy phân của muối}
Trong dung dịch nước, một số ion như $\mathrm{Al}^{3+}, \mathrm{Fe}^{3+}$ và $\mathrm{CO}_3^{2-}$ phản ứng với nước tạo ra các dung dịch có môi trường acid/base.
\begin{vidu}
	\begin{enumerate}
		\item Trong dung dịch $\mathrm{Na}_2 \mathrm{CO}_3$, ion $\mathrm{Na}^{+}$ không bị thủy phân, còn $\mathrm{CO}_3^{2-}$ thủy phân trong nước tạo ion $\mathrm{OH}^{-}$ theo phương trình:
		\[
		\mathrm{CO}_3^{2-}+\mathrm{H}_2 \mathrm{O} \rightleftharpoons \mathrm{HCO}_3^{-}+\mathrm{OH}^{-}
		\]
		Vì vậy, dung dịch $\mathrm{Na}_2 \mathrm{CO}_3$ có môi trường base.
		\item Trong dung dịch $\mathrm{AlCl}_3$ và $\mathrm{FeCl}_3$, ion $\mathrm{Cl}^{-}$không bị thuỷ phân, các ion $\mathrm{Al}^{3+}$ và $\mathrm{Fe}^{3+}$ bị thuỷ phân trong nước tạo ion $\mathrm{H}^{+}$theo phương trình ở dạng đơn giản như sau:
		\[
		\begin{aligned}
			& \mathrm{Al}^{3+}+\mathrm{H}_2 \mathrm{O} \rightleftharpoons \mathrm{Al}(\mathrm{OH})^{2+}+\mathrm{H}^{+} \\
			& \mathrm{Fe}^{3+}+\mathrm{H}_2 \mathrm{O} \rightleftharpoons \mathrm{Fe}(\mathrm{OH})^{2+}+\mathrm{H}^{+}
		\end{aligned}
		\]
		Do đó, dung dịch $\mathrm{AlCl}_3, \mathrm{FeCl}_3$ có môi trường acid. Trong thực tế, các loại đất có chứa nhiều ion $\mathrm{Al}^{3+}, \mathrm{Fe}^{3+}$ có giá trị pH thấp hay còn gọi là đất chua. Để khử chua, người ta bón vôi cho đất.
		
		Các muối nhôm và sắt, ví dụ: phèn nhôm $\left(\left(\mathrm{NH}_4\right)_2 \mathrm{SO}_4 \cdot \mathrm{Al}_2\left(\mathrm{SO}_4\right)_3 \cdot 24 \mathrm{H}_2 \mathrm{O}\right)$ và phèn sắt $\left(\left(\mathrm{NH}_4\right)_2 \mathrm{SO}_4 \cdot \mathrm{Fe}_2\left(\mathrm{SO}_4\right)_3 \cdot 24 \mathrm{H}_2 \mathrm{O}\right)$ được sử dụng làm chất keo tụ trong quá trình xử lí nước, dùng làm chất cầm màu trong công nghiệp dệt, nhuộm, hoặc làm chất kết dính, chống nhoè trong công nghiệp giấy,...
	\end{enumerate}
\end{vidu}
\subsubsection{Chuẩn độ acid -base}
Chuẩn độ là phương pháp xác định nồng độ của một chất bằng một dung dịch chuần đã biết nồng độ. Dựa vào thể tích của các dung dịch khi phản ứng vừa đủ với nhau, xác định được nồng độ dung dịch chất cần chuần độ.
\\
Trong phòng thí nghiệm, nồng độ của dung dịch base mạnh (ví dụ NaOH ) được xác định bằng một dung dịch acid mạnh (ví dụ HCl ) đã biết trước nồng độ mol dựa trên phản ứng:
\[
\mathrm{NaOH}+\mathrm{HCl} \longrightarrow \mathrm{NaCl}+\mathrm{H}_2 \mathrm{O}
\]
Khi các chất phản ứng vừa đủ với nhau, số mol HCl phản ứng bằng số mol NaOH .
Ta có:
\[
\mathrm{V}_{\mathrm{HCl}} \cdot \mathrm{C}_{\mathrm{HCl}}=\mathrm{V}_{\mathrm{NaOH}} \cdot \mathrm{C}_{\mathrm{NaOH}}
\]
Trong đó:
	\begin{itemize}
		\item $\mathrm{C}_{\mathrm{HCl}}$ và $\mathrm{C}_{\mathrm{NaOH}}$ lần lượt là nồng độ mol của dung dịch HCl và dung dịch NaOH ;
		\item $\mathrm{V}_{\mathrm{HCl}}$ và $\mathrm{V}_{\mathrm{NaOH}}$ lần lượt là thể tích của dung dịch HCl và dung dịch NaOH (cùng đơn vị đo).
	\end{itemize}
Khi biết $\mathrm{V}_{\mathrm{HCl}}, \mathrm{V}_{\mathrm{NaOH}}$ trong quá trình chuẩn độ và biết $\mathrm{C}_{\mathrm{HCl}}$ sẽ tính được $\mathrm{C}_{\mathrm{NaOH}}$.\\
Thời điểm để kết thúc chuẩn độ được xác định bằng sự đổi màu của chất chỉ thị phenolphthalein.

%%%%%%%%%%%%%%%%

\subsection{Các dạng bài tập}
\begin{dang}{Nhận biết và phân loại chất điện li, acid, base, chất lưỡng tính}\end{dang}
\begin{phuongphap}
Để nhận biết và phân loại các chất, cần nắm vững các khái niệm sau:
\begin{itemize}
    \item \textbf{Sự điện li:} Quá trình chất tan trong nước phân li ra ion.
    \item \textbf{Chất điện li:} Chất khi tan trong nước phân li ra ion (gồm acid, base, muối).
        \begin{itemize}
	            \item \textbf{Chất điện li mạnh:} Phân li hoàn toàn ra ion (acid mạnh, base mạnh, hầu hết các muối). Phương trình dùng dấu $\rightarrow$.
	            \item \textbf{Chất điện li yếu:} Chỉ một phần phân li ra ion (acid yếu, base yếu, $H_2O$). Phương trình dùng dấu $\rightleftharpoons$.
	        \end{itemize}
    \item \textbf{Chất không điện li:} Chất khi tan trong nước không phân li ra ion (ví dụ: saccharose, glucose, ethanol).
    \item \textbf{Thuyết Arrhenius:}
        \begin{itemize}
	            \item Acid: Chất phân li ra $H^+$ trong nước.
	            \item Base: Chất phân li ra $OH^-$ trong nước.
	        \end{itemize}
    \item \textbf{Thuyết Brønsted-Lowry:}
        \begin{itemize}
	            \item Acid: Chất cho proton ($H^+$).
	            \item Base: Chất nhận proton ($H^+$).
	            \item Chất lưỡng tính: Chất vừa có khả năng cho proton, vừa có khả năng nhận proton.
	            \item Cặp acid-base liên hợp: Acid $\rightleftharpoons$ Base liên hợp + $H^+$; Base + $H^+$ $\rightleftharpoons$ Acid liên hợp.
	        \end{itemize}
\end{itemize}
\end{phuongphap}

\Noibat[\maunhan][][\faBookmark][]{Ví dụ mẫu}
%%%%%==========VD_01==========%%%%%
\begin{vd}
	Theo thuyết Brønsted-Lowry, trong phản ứng: $NH_3 + H_2O \rightleftharpoons NH_4^+ + OH^-$, nước đóng vai trò là
	\choice
	{chất khử}
	{chất oxi hóa}
	{\True acid}
	{base}
	\loigiai{Trong phản ứng $NH_3 + H_2O \rightleftharpoons NH_4^+ + OH^-$, phân tử $H_2O$ đã nhường proton ($H^+$) cho $NH_3$ để tạo thành $OH^-$. Theo thuyết Brønsted-Lowry, chất cho proton là acid. Vậy, $H_2O$ đóng vai trò là acid.}
\end{vd}

%%%%%==========VD_02==========%%%%%
\begin{vd}
	Cho các chất sau: $HCl, NaOH, CH_3COOH, C_2H_5OH, NaCl, Al(OH)_3, HCO_3^-, H_2O$.
	\begin{enumerate}
		   \item Phân loại các chất trên thành chất điện li mạnh, chất điện li yếu, chất không điện li.
		   \item Xác định chất nào là acid, base, lưỡng tính theo thuyết Brønsted-Lowry.
		\end{enumerate}
	\loigiai{
		\begin{enumerate}
			    \item Phân loại:
			    \begin{itemize}
				        \item Chất điện li mạnh: $HCl, NaOH, NaCl$.
				        \item Chất điện li yếu: $CH_3COOH, Al(OH)_3, HCO_3^-, H_2O$.
				        \item Chất không điện li: $C_2H_5OH$.
				    \end{itemize}
			    \item Xác định vai trò theo Brønsted-Lowry:
			    \begin{itemize}
				        \item Acid: $HCl$ (cho $H^+$), $CH_3COOH$ (cho $H^+$).
				        \item Base: $NaOH$ (phân li ra $OH^-$, $OH^-$ nhận $H^+$).
				        \item Lưỡng tính: $Al(OH)_3$ (vừa có thể cho $H^+$ dạng $HAlO_2.H_2O$, vừa có thể nhận $H^+$), $HCO_3^-$ (vừa cho $H^+$ thành $CO_3^{2-}$, vừa nhận $H^+$ thành $H_2CO_3$), $H_2O$ (vừa cho $H^+$ thành $OH^-$, vừa nhận $H^+$ thành $H_3O^+$).
				    \end{itemize}
			\end{enumerate}
		}
\end{vd}

%%%%%=====================Bài tập tự luyện Dạng 1==========================%%%
\Noibat[\maunhan][][\faBook][]{Bài tập tự luyện}

\phan{Bài tập tự luận}
%%%=============SOẠN BT===============%%%
% Giả sử chương này là chương 1, bài 2
\Opensolutionfile{ansbth}[Ans/LGBT-C01B02_Dang1]
\Opensolutionfile{ansbt}[Ans/AnsBT-C01B02_Dang1]
%%%%%============BT_01================%%%%%%
\begin{bt}
	Viết các cặp acid-base liên hợp trong các phản ứng sau theo thuyết Brønsted-Lowry:
	\begin{enumerate}
			\item $HF + H_2O \rightleftharpoons F^- + H_3O^+$
			\item $CH_3COO^- + H_2O \rightleftharpoons CH_3COOH + OH^-$
			\item $H_2SO_4 + H_2O \rightarrow HSO_4^- + H_3O^+$
			\item $S^{2-} + H_2O \rightleftharpoons HS^- + OH^-$
	        \item $NH_4^+ + CO_3^{2-} \rightleftharpoons NH_3 + HCO_3^-$
		\end{enumerate}
	\loigiai{
		\begin{enumerate}
				\item Cặp acid/base liên hợp: $HF/F^-$ và $H_3O^+/H_2O$.
				\item Cặp acid/base liên hợp: $CH_3COOH/CH_3COO^-$ và $H_2O/OH^-$.
				\item Cặp acid/base liên hợp: $H_2SO_4/HSO_4^-$ và $H_3O^+/H_2O$.
				\item Cặp acid/base liên hợp: $HS^-/S^{2-}$ và $H_2O/OH^-$.
		        \item Cặp acid/base liên hợp: $NH_4^+/NH_3$ và $HCO_3^-/CO_3^{2-}$.
			\end{enumerate}
		}
\end{bt}
%%%%%============BT_02================%%%%%%
\begin{bt}
	Giải thích tại sao $Zn(OH)_2$ là một hydroxide lưỡng tính. Viết phương trình hóa học minh họa.
	\loigiai{
		$Zn(OH)_2$ là một hydroxide lưỡng tính vì nó có thể phản ứng với cả acid và base mạnh.
		\begin{itemize}
			    \item Tác dụng với acid (thể hiện tính base): $Zn(OH)_2(s) + 2H^+(aq) \rightarrow Zn^{2+}(aq) + 2H_2O(l)$
			    (Ví dụ: $Zn(OH)_2 + 2HCl \rightarrow ZnCl_2 + 2H_2O$)
			    \item Tác dụng với base mạnh (thể hiện tính acid): $Zn(OH)_2(s) + 2OH^-(aq) \rightarrow [Zn(OH)_4]^{2-}(aq)$ (ion zincate)
			    (Ví dụ: $Zn(OH)_2 + 2NaOH \rightarrow Na_2[Zn(OH)_4]$ hoặc $Na_2ZnO_2 + 2H_2O$)
			\end{itemize}
		Theo Brønsted-Lowry:
		\begin{itemize}
			    \item Khi phản ứng với acid, $Zn(OH)_2$ nhận $H^+$ (nếu coi $OH^-$ trong $Zn(OH)_2$ nhận $H^+$ để tạo $H_2O$).
			    \item Khi phản ứng với base mạnh, $Zn(OH)_2$ cho $H^+$ (dưới dạng $H_2ZnO_2 \rightleftharpoons 2H^+ + ZnO_2^{2-}$).
			\end{itemize}
		}
\end{bt}
%%%%%============BT_03================%%%%%%
\begin{bt}
    Dựa vào thuyết Brønsted-Lowry, hãy xác định vai trò (acid, base, lưỡng tính, không phải acid/base) của các phân tử và ion sau trong dung dịch nước: $H_2S, CO_3^{2-}, NH_4^+, Cl^-, H_2PO_4^-, Al^{3+}(aq)$. Viết phương trình phản ứng minh họa (nếu có).
	\loigiai{
	    \begin{itemize}
		        \item $H_2S$: Acid (cho $H^+$). $H_2S + H_2O \rightleftharpoons HS^- + H_3O^+$
		        \item $CO_3^{2-}$: Base (nhận $H^+$). $CO_3^{2-} + H_2O \rightleftharpoons HCO_3^- + OH^-$
		        \item $NH_4^+$: Acid (cho $H^+$). $NH_4^+ + H_2O \rightleftharpoons NH_3 + H_3O^+$
		        \item $Cl^-$: Base liên hợp của acid mạnh $HCl$, nên là base rất yếu, thường coi là trung tính trong nước (không nhận $H^+$ từ $H_2O$ đáng kể).
		        \item $H_2PO_4^-$: Lưỡng tính.
		            \begin{itemize}
			                \item Acid: $H_2PO_4^- + H_2O \rightleftharpoons HPO_4^{2-} + H_3O^+$
			                \item Base: $H_2PO_4^- + H_2O \rightleftharpoons H_3PO_4 + OH^-$
			            \end{itemize}
		        \item $Al^{3+}(aq)$: Ion kim loại hydrated có tính acid. $Al^{3+} + H_2O \rightleftharpoons [Al(OH)]^{2+} + H^+$ (Viết đơn giản) hoặc $[Al(H_2O)_6]^{3+} + H_2O \rightleftharpoons [Al(H_2O)_5(OH)]^{2+} + H_3O^+$
		    \end{itemize}
		}
\end{bt}
%%%%%============BT_04================%%%%%%
\begin{bt}
	So sánh khái niệm acid và base theo thuyết Arrhenius và thuyết Brønsted-Lowry. Nêu ưu điểm của thuyết Brønsted-Lowry. Cho ví dụ minh họa.
	\loigiai{
	    \textbf{So sánh:}
	    \begin{itemize}
		        \item \textbf{Giống nhau:} Cả hai thuyết đều mô tả acid là chất liên quan đến ion $H^+$ và base liên quan đến ion $OH^-$ (hoặc khả năng tạo ra $OH^-$).
		        \item \textbf{Khác nhau:}
		            \begin{itemize}
			                \item \textbf{Phạm vi áp dụng:} Arrhenius chỉ áp dụng cho dung môi nước. Brønsted-Lowry áp dụng rộng hơn, cho cả dung môi khác và pha khí.
			                \item \textbf{Định nghĩa base:} Arrhenius định nghĩa base là chất phân li ra $OH^-$. Brønsted-Lowry định nghĩa base là chất nhận $H^+$, bao gồm cả những chất không chứa $OH^-$ như $NH_3, CO_3^{2-}$.
			                \item \textbf{Vai trò của dung môi:} Arrhenius không đề cập rõ vai trò của dung môi. Brønsted-Lowry cho thấy dung môi (như nước) có thể đóng vai trò acid hoặc base.
			                \item \textbf{Khái niệm acid:} Arrhenius coi acid là chất tạo $H^+$. Brønsted-Lowry mở rộng cho cả các ion có khả năng cho $H^+$ (ví dụ $NH_4^+$).
			            \end{itemize}
		    \end{itemize}
	    \textbf{Ưu điểm của thuyết Brønsted-Lowry:}
	    \begin{itemize}
		        \item \textbf{Tổng quát hơn:} Giải thích được tính acid-base của nhiều loại chất hơn, bao gồm cả ion và các phân tử không chứa $H^+$ (cho acid) hoặc $OH^-$ (cho base) trong công thức. Ví dụ: $NH_3$ là base vì $NH_3 + H_2O \rightleftharpoons NH_4^+ + OH^-$. $CO_3^{2-}$ là base vì $CO_3^{2-} + H_2O \rightleftharpoons HCO_3^- + OH^-$. $NH_4^+$ là acid vì $NH_4^+ \rightleftharpoons NH_3 + H^+$.
		        \item \textbf{Không phụ thuộc dung môi nước:} Có thể giải thích phản ứng acid-base trong dung môi khác nước hoặc pha khí.
		        \item \textbf{Làm rõ vai trò của nước:} Nước có thể là acid hoặc base tùy thuộc vào chất phản ứng cùng.
		        \item \textbf{Giới thiệu khái niệm cặp acid-base liên hợp:} Giúp hiểu rõ hơn về bản chất của phản ứng acid-base.
		    \end{itemize}
		}
\end{bt}
%%%%%============BT_05================%%%%%%
\begin{bt}
    Cho dung dịch $X$ chứa các ion: $Na^+, K^+, Cl^-, SO_4^{2-}$. Khi cô cạn dung dịch $X$, có thể thu được những muối nào? Giải thích tại sao các muối đó là chất điện li mạnh.
	\loigiai{
	    Khi cô cạn dung dịch $X$, các ion $Na^+, K^+, Cl^-, SO_4^{2-}$ sẽ kết hợp với nhau để tạo thành các muối. Các muối có thể thu được là:
	    \begin{itemize}
		        \item $NaCl$ (Sodium chloride)
		        \item $KCl$ (Potassium chloride)
		        \item $Na_2SO_4$ (Sodium sulfate)
		        \item $K_2SO_4$ (Potassium sulfate)
		    \end{itemize}
	    Cũng có thể tạo thành các muối kép như $NaKSO_4$ (ít phổ biến hơn khi cô cạn đơn giản). Tuy nhiên, thường xét các muối đơn giản.
	
	    Các muối trên ($NaCl, KCl, Na_2SO_4, K_2SO_4$) đều là chất điện li mạnh vì:
	    \begin{itemize}
		        \item Chúng là các hợp chất ion được tạo thành từ cation của kim loại mạnh ($Na^+, K^+$) và anion gốc acid mạnh ($Cl^-, SO_4^{2-}$).
		        \item Khi hòa tan trong nước, các liên kết ion trong mạng lưới tinh thể của chúng bị phá vỡ hoàn toàn bởi sự hydrat hóa của các ion bởi các phân tử nước, dẫn đến sự phân li hoàn toàn ra các ion tự do.
		        \item Ví dụ: $NaCl(s) \xrightarrow{H_2O} Na^+(aq) + Cl^-(aq)$
		        \\ $K_2SO_4(s) \xrightarrow{H_2O} 2K^+(aq) + SO_4^{2-}(aq)$
		    \end{itemize}
		}
\end{bt}
\Closesolutionfile{ansbt}
\Closesolutionfile{ansbth}
%\bangdapanSA{AnsBT-C01B02_Dang1}

\phan{Bài tập trả lời ngắn}
%%%=============SOẠN BT===============%%%
\Opensolutionfile{ansbth}[Ans/LGSA-C01B02_Dang1]
\Opensolutionfile{ansbt}[Ans/AnsSA-C01B02_Dang1]
%%%%%============SA_01================%%%%%%
\begin{bt}
	Chất nào sau đây không phải là chất điện li: $H_2SO_4, Ba(OH)_2, C_6H_{12}O_6$ (glucose), $Fe(NO_3)_3$? (Chỉ ghi công thức hóa học)
	\shortans{$C_6H_{12}O_6$}
	\loigiai{Glucose ($C_6H_{12}O_6$) là chất hữu cơ, khi tan trong nước không phân li ra ion.}
\end{bt}
%%%%%============SA_02================%%%%%%
\begin{bt}
	Theo thuyết Brønsted-Lowry, ion $HCO_3^-$ có thể đóng vai trò là acid hay base? (Trả lời: Acid, Base, hoặc Lưỡng tính)
	\shortans{Lưỡng tính}
	\loigiai{$HCO_3^-$ có thể cho $H^+$ (tạo $CO_3^{2-}$) nên là acid. $HCO_3^-$ có thể nhận $H^+$ (tạo $H_2CO_3$) nên là base. Vậy $HCO_3^-$ lưỡng tính.}
\end{bt}
%%%%%============SA_03================%%%%%%
\begin{bt}
	Trong dung dịch $CH_3COOH$ 0,1M, nếu bỏ qua sự điện li của nước, có bao nhiêu loại tiểu phân (phân tử và ion) khác nhau tồn tại?
	\shortans{4}
	\loigiai{
		Các tiểu phân gồm: $CH_3COOH$ (chưa điện li), $H^+$, (${H_3O}^+$)
	}
\end{bt}
%%%%%============SA_04================%%%%%%
\begin{bt}
	Acid liên hợp của $NH_3$ là gì? (Chỉ ghi công thức hóa học và điện tích nếu có)
	\shortans{${NH_4}^+$}
	\loigiai{$NH_3$ (base) + $H^+$ $\rightleftharpoons NH_4^+$ (acid liên hợp).}
\end{bt}
%%%%%============SA_05================%%%%%%
\begin{bt}
	Chất nào sau đây là chất điện li yếu: $HNO_3, H_2S, BaCl_2, KOH$? (Chỉ ghi công thức hóa học)
	\shortans{H2S}
	\loigiai{$H_2S$ là một acid yếu.}
\end{bt}
%%%%%============SA_06================%%%%%%
\begin{bt}
	Theo thuyết Arrhenius, dung dịch base là dung dịch chứa ion gì đặc trưng? (Ghi công thức ion)
	\shortans{OH-}
	\loigiai{Theo Arrhenius, base là chất khi tan trong nước phân li ra ion $OH^-$.}
\end{bt}
%%%%%============SA_07================%%%%%%
\begin{bt}
	Trong phản ứng $S^{2-} + H_2O \rightleftharpoons HS^- + OH^-$, $S^{2-}$ đóng vai trò là acid hay base theo Brønsted-Lowry? (Trả lời: Acid hoặc Base)
	\shortans{Base}
	\loigiai{$S^{2-}$ nhận $H^+$ từ $H_2O$ để tạo thành $HS^-$, do đó $S^{2-}$ là base.}
\end{bt}
%%%%%============SA_08================%%%%%%
\begin{bt}
    Số chất điện li mạnh trong dãy sau là bao nhiêu: $H_2SO_3, MgCl_2, HF, CH_3COONa, Al(OH)_3, HClO_4$?
	\shortans{3}
	\loigiai{Các chất điện li mạnh là $MgCl_2$ (muối tan), $CH_3COONa$ (muối tan), $HClO_4$ (acid mạnh). $H_2SO_3, HF$ là acid yếu. $Al(OH)_3$ là base yếu, ít tan.}
\end{bt}
%%%%%============SA_09================%%%%%%
\begin{bt}
    Base liên hợp của $HSO_4^-$ là gì? (Chỉ ghi công thức hóa học và điện tích)
	\shortans{${SO_4}^{2-}$}
	\loigiai{$HSO_4^-$ (acid) $\rightleftharpoons SO_4^{2-}$ (base liên hợp) + $H^+$.}
\end{bt}
%%%%%============SA_10================%%%%%%
\begin{bt}
    Theo Brønsted-Lowry, một chất được coi là lưỡng tính nếu nó có thể cho và nhận loại hạt nào? (Ghi tên hạt)
	\shortans{proton}
	\loigiai{Chất lưỡng tính theo Brønsted-Lowry là chất vừa có khả năng cho proton ($H^+$), vừa có khả năng nhận proton ($H^+$).}
\end{bt}
\Closesolutionfile{ansbt}
\Closesolutionfile{ansbth}
%\bangdapanSA{AnsSA-C01B02_Dang1}


%%%%============Phần trắc nghiệm============%%%
\phan{Trắc nghiệm nhiều lựa chọn}
%%%=============SOẠN EX===============%%%
\Opensolutionfile{ansex}[Ans/LGEX-C01B02_Dang1]
\Opensolutionfile{ans}[Ans/Ans-C01B02_Dang1]
%%%%%============EX_01================%%%%%%
\begin{ex}
	Dung dịch chất nào sau đây không dẫn điện?
	\choice
	{$HCl$ trong nước.}
	{$Na_2SO_4$ trong nước.}
	{\True $C_6H_{12}O_6$ (glucose) trong nước.}
	{$Ca(OH)_2$ trong nước.}
	\loigiai{Glucose ($C_6H_{12}O_6$) là chất hữu cơ, khi tan trong nước không phân li ra ion, do đó dung dịch glucose không dẫn điện. Các chất còn lại là acid, muối, base là những chất điện li.}
\end{ex}
%%%%%============EX_02================%%%%%%
\begin{ex}
	Theo thuyết Arrhenius, chất nào sau đây là acid?
	\choice
	{$NaOH$}
	{\True $HNO_3$}
	{$KCl$}
	{$C_2H_5OH$}
	\loigiai{Theo Arrhenius, acid là chất khi tan trong nước phân li ra ion $H^+$. $HNO_3$ khi tan trong nước phân li ra $H^+$ và $NO_3^-$.}
\end{ex}
%%%%%============EX_03================%%%%%%
\begin{ex}
	Chất nào sau đây là chất điện li mạnh?
	\choice
	{$CH_3COOH$}
	{$H_2O$}
	{\True $Ba(NO_3)_2$}
	{$HF$}
	\loigiai{$Ba(NO_3)_2$ là muối tan, thuộc loại chất điện li mạnh. $CH_3COOH, HF$ là acid yếu. $H_2O$ là chất điện li rất yếu.}
\end{ex}
%%%%%============EX_04================%%%%%%
\begin{ex}
	Theo thuyết Brønsted-Lowry, base là chất
	\choice
	{cho proton.}
	{\True nhận proton.}
	{cho electron.}
	{nhận electron.}
	\loigiai{Theo thuyết Brønsted-Lowry, base là chất có khả năng nhận proton ($H^+$).}
\end{ex}
%%%%%============EX_05================%%%%%%
\begin{ex}
	Trong phản ứng: $S^{2-} + H_2O \rightleftharpoons HS^- + OH^-$. Ion $S^{2-}$ đóng vai trò là
	\choice
	{acid.}
	{\True base.}
	{chất oxi hóa.}
	{chất khử.}
	\loigiai{Trong phản ứng, ion $S^{2-}$ nhận proton ($H^+$) từ $H_2O$ để tạo thành $HS^-$. Theo Brønsted-Lowry, chất nhận proton là base.}
\end{ex}
%%%%%============EX_06================%%%%%%
\begin{ex}
	Chất nào sau đây là chất điện li yếu?
	\choice
	{$NaCl$}
	{$KOH$}
	{\True $H_2CO_3$}
	{$H_2SO_4$}
	\loigiai{$H_2CO_3$ là một acid yếu. $NaCl$ là muối tan (điện li mạnh). $KOH$ là base mạnh. $H_2SO_4$ là acid mạnh.}
\end{ex}
%%%%%============EX_07================%%%%%%
\begin{ex}
	Dãy chất nào sau đây chỉ gồm các chất điện li mạnh?
	\choice
	{$HCl, NaOH, CH_3COOH, NaCl$}
	{\True $HNO_3, Mg(OH)_2, K_2SO_4, BaCl_2$}
	{$H_2SO_4, Cu(OH)_2, FeCl_3, HClO$}
	{$H_3PO_4, NaOH, AgCl, KNO_3$}
	\loigiai{$Mg(OH)_2$ là base mạnh, $K_2SO_4, BaCl_2$ là muối tan, $HNO_3$ là acid mạnh.
	    A sai vì $CH_3COOH$ yếu. C sai vì $Cu(OH)_2$ yếu, $HClO$ yếu. D sai vì $H_3PO_4$ yếu, $AgCl$ không tan (điện li không đáng kể).}
\end{ex}
%%%%%============EX_08================%%%%%%
\begin{ex}
	Theo thuyết Brønsted-Lowry, ion nào sau đây là lưỡng tính?
	\choice
	{$CO_3^{2-}$}
	{$NH_4^+$}
	{\True $HPO_4^{2-}$}
	{$SO_4^{2-}$}
	\loigiai{Ion $HPO_4^{2-}$ có thể cho proton ($HPO_4^{2-} \rightleftharpoons PO_4^{3-} + H^+$) và nhận proton ($HPO_4^{2-} + H^+ \rightleftharpoons H_2PO_4^-$).
	    $CO_3^{2-}$ là base. $NH_4^+$ là acid. $SO_4^{2-}$ là base rất yếu (trung tính).}
\end{ex}
%%%%%============EX_09================%%%%%%
\begin{ex}
	Acid liên hợp của $HCO_3^-$ là
	\choice
	{$CO_3^{2-}$}
	{\True $H_2CO_3$}
	{$OH^-$}
	{$H_3O^+$}
	\loigiai{$HCO_3^-$ (base) + $H^+$ $\rightleftharpoons H_2CO_3$ (acid liên hợp).}
\end{ex}
%%%%%============EX_10================%%%%%%
\begin{ex}
	Base liên hợp của $NH_4^+$ là
	\choice
	{$NH_2^-$}
	{\True $NH_3$}
	{$N_2H_4$}
	{$OH^-$}
	\loigiai{$NH_4^+$ (acid) $\rightleftharpoons NH_3$ (base liên hợp) + $H^+$.}
\end{ex}
%%%%%============EX_11================%%%%%%
\begin{ex}
	Trong dung dịch acid acetic ($CH_3COOH$) có những phần tử nào sau đây (bỏ qua sự điện li của nước)?
	\choice
	{$H^+, CH_3COO^-$}
	{$CH_3COOH, H^+$}
	{\True $CH_3COOH, H^+, CH_3COO^-$}
	{$CH_3COOH, CH_3COO^-, OH^-$}
	\loigiai{$CH_3COOH$ là acid yếu, điện li một phần: $CH_3COOH \rightleftharpoons CH_3COO^- + H^+$. Do đó trong dung dịch có $CH_3COOH$ (chưa điện li), $H^+$ và $CH_3COO^-$.}
\end{ex}
%%%%%============EX_12================%%%%%%
\begin{ex}
	Chất nào sau đây không phải là chất lưỡng tính theo Brønsted-Lowry?
	\choice
	{$Al_2O_3$}
	{$NaHCO_3$}
	{\True $H_2SO_4$}
	{$Zn(OH)_2$}
	\loigiai{$H_2SO_4$ là acid mạnh, chỉ có khả năng cho proton, không có khả năng nhận proton. $Al_2O_3, NaHCO_3, Zn(OH)_2$ là các chất lưỡng tính.}
\end{ex}
%%%%%============EX_13================%%%%%%
\begin{ex}
	Theo thuyết Brønsted-Lowry, $H_2O$ đóng vai trò là acid trong phản ứng nào sau đây?
	\choice
	{$H_2O + HCl \rightarrow H_3O^+ + Cl^-$}
	{\True $H_2O + NH_3 \rightleftharpoons NH_4^+ + OH^-$}
	{$H_2O + H_2O \rightleftharpoons H_3O^+ + OH^-$}
	{$H_2O + SO_3 \rightarrow H_2SO_4$}
	\loigiai{Trong phản ứng $H_2O + NH_3 \rightleftharpoons NH_4^+ + OH^-$, $H_2O$ nhường proton ($H^+$) cho $NH_3$, do đó $H_2O$ là acid.
	    Trong A, $H_2O$ là base. Trong C, một $H_2O$ là acid, một $H_2O$ là base. Trong D, $SO_3$ là anhydride acid, phản ứng với nước.}
\end{ex}
%%%%%============EX_14================%%%%%%
\begin{ex}
    Dãy các chất nào sau đây được sắp xếp theo chiều tăng dần tính acid?
    \choice
    {$HClO < HClO_2 < HClO_3 < HClO_4$}
    {\True $CH_3COOH < H_2CO_3 < HCl < H_2SO_4$}
    {$HF > HCl > HBr > HI$}
    {$H_3PO_4 < H_2SO_4 < HClO_4 < H_2S$}
    \loigiai{Thứ tự tính acid thường gặp.
	    A: Đúng, tính acid của dãy oxoacid của Cl tăng theo số nguyên tử O.
	    \\
	    B: $CH_3COOH$ (yếu) < $H_2CO_3$ (yếu) < $HCl$ (mạnh) < $H_2SO_4$ (mạnh). Tuy nhiên, $H_2CO_3$ yếu hơn $CH_3COOH$. $pK_{a1}(H_2CO_3) \approx 6.35$, $pK_a(CH_3COOH) \approx 4.76$. Vậy $CH_3COOH$ mạnh hơn $H_2CO_3$. Phương án B sai.
	    \\
	    C: Sai, tính acid của dãy hydrohalogenic acid tăng từ HF đến HI.
	    \\
	    D: $H_2S$ là acid rất yếu.
	    }
\end{ex}
%%%%%============EX_15================%%%%%%
\begin{ex}
    Chất nào sau đây khi tan trong nước tạo dung dịch có môi trường base?
    \choice
    {$NaCl$}
    {$NH_4Cl$}
    {\True $Na_2CO_3$}
    {$CH_3COOH$}
    \loigiai{$Na_2CO_3$ là muối của base mạnh ($NaOH$) và acid yếu ($H_2CO_3$). Ion $CO_3^{2-}$ bị thủy phân tạo môi trường base: $CO_3^{2-} + H_2O \rightleftharpoons HCO_3^- + OH^-$.
	    $NaCl$ (trung tính). $NH_4Cl$ (acid). $CH_3COOH$ (acid).}
\end{ex}
%%%%%============EX_16================%%%%%%
\begin{ex}
    Phản ứng nào sau đây không phải là phản ứng acid-base theo Brønsted-Lowry?
    \choice
    {$HCl + KOH \rightarrow KCl + H_2O$}
    {$CH_3COOH + NaHCO_3 \rightarrow CH_3COONa + CO_2 + H_2O$}
    {$NH_3 + HCl \rightarrow NH_4Cl$}
    {\True $2Na + 2H_2O \rightarrow 2NaOH + H_2$}
    \loigiai{Phản ứng $2Na + 2H_2O \rightarrow 2NaOH + H_2$ là phản ứng oxi hóa-khử, không có sự cho nhận proton theo nghĩa Brønsted-Lowry một cách trực tiếp giữa Na và $H_2O$ (mặc dù $H_2O$ có thể coi là cho $H^+$ cho electron từ Na). Các phản ứng còn lại đều có sự trao đổi proton.}
\end{ex}
%%%%%============EX_17================%%%%%%
\begin{ex}
    Dung dịch chất nào sau đây làm quỳ tím hóa xanh?
    \choice
    {$H_2SO_4$}
    {$KCl$}
    {\True $CH_3COONa$}
    {$AlCl_3$}
    \loigiai{$CH_3COONa$ là muối của base mạnh ($NaOH$) và acid yếu ($CH_3COOH$). Ion $CH_3COO^-$ bị thủy phân tạo môi trường base, làm quỳ tím hóa xanh: $CH_3COO^- + H_2O \rightleftharpoons CH_3COOH + OH^-$.
	    $H_2SO_4$ (hóa đỏ). $KCl$ (không đổi màu). $AlCl_3$ (hóa đỏ do $Al^{3+}$ thủy phân).}
\end{ex}
%%%%%============EX_18================%%%%%%
\begin{ex}
    Trong các cặp chất sau, cặp chất nào không phải là cặp acid-base liên hợp?
    \choice
    {$HCl/Cl^-$}
    {$NH_4^+/NH_3$}
    {\True $H_2SO_4/SO_4^{2-}$}
    {$CH_3COOH/CH_3COO^-$}
    \loigiai{$H_2SO_4$ cho proton tạo $HSO_4^-$, sau đó $HSO_4^-$ cho proton tạo $SO_4^{2-}$. Vậy $H_2SO_4/HSO_4^-$ là một cặp, và $HSO_4^-/SO_4^{2-}$ là một cặp. $H_2SO_4/SO_4^{2-}$ không phải là một cặp acid-base liên hợp trực tiếp (khác nhau 2 proton).}
\end{ex}
%%%%%============EX_19================%%%%%%
\begin{ex}
    Sự điện li hoàn toàn là
    \choice
    {sự phân li một phần chất tan thành ion.}
    {\True sự phân li toàn bộ chất tan thành ion.}
    {sự phân li chất tan thành nguyên tử.}
    {sự kết hợp ion thành phân tử.}
    \loigiai{Sự điện li hoàn toàn xảy ra ở các chất điện li mạnh, khi đó toàn bộ các phân tử chất tan trong dung dịch đều phân li ra ion.}
\end{ex}
%%%%%============EX_20================%%%%%%
\begin{ex}
    Nước đóng vai trò là base Brønsted-Lowry khi tương tác với chất nào sau đây?
    \choice
    {$NH_3$}
    {$CH_3COO^-$}
    {\True $H_2SO_4$}
    {$CO_3^{2-}$}
    \loigiai{Khi tương tác với $H_2SO_4$ (acid), $H_2O$ nhận proton để tạo $H_3O^+$: $H_2SO_4 + H_2O \rightarrow HSO_4^- + H_3O^+$. Vậy $H_2O$ là base.
	    Với $NH_3, CH_3COO^-, CO_3^{2-}$ (là base), $H_2O$ sẽ đóng vai trò là acid.}
\end{ex}
\Closesolutionfile{ans}
\Closesolutionfile{ansex}
%\bangdapan{Ans-C01B02_Dang1}

%%%%%%%%%%%%%%%Trắc nghiệm đúng sai%%%%%%%%%%%%%%%%%%%%%%%%
\phan{Bài tập trắc nghiệm Đúng Sai}
%%%=============SOẠN EXTF===============%%%
\Opensolutionfile{ansex}[Ans/LGTF-C01B02_Dang1]
\Opensolutionfile{ansbook}[Ansbook/AnsTF-C01B02_Dang1]
\Opensolutionfile{ans}[Ans/Tempt-C01B02_Dang1]
%%%%%============TF_01================%%%%%%
\begin{ex}
	Xét các phát biểu sau về sự điện li:
	\choiceTF
	{\True Chất điện li là chất khi tan trong nước phân li ra ion.}
	{Tất cả các acid đều là chất điện li mạnh.}
	{\True Muối ăn ($NaCl$) là một chất điện li mạnh.}
	{Nước cất dẫn điện tốt vì chứa $H_2O$.}
	\loigiai{
			\begin{itemchoice}[T1,F2,T3,F4]
					\itemch  theo định nghĩa chất điện li.
					\itemch  Có nhiều acid yếu như $CH_3COOH, H_2S, H_2CO_3$.
					\itemch  $NaCl$ là muối tan của acid mạnh và base mạnh.
					\itemch  Nước cất là chất điện li rất yếu, hầu như không dẫn điện.
				\end{itemchoice}
		}
\end{ex}
%%%%%============TF_02================%%%%%%
\begin{ex}
	Theo thuyết Brønsted-Lowry:
	\choiceTF
	{\True Acid là chất cho proton ($H^+$).}
	{Base là chất cho proton ($H^+$).}
	{\True Trong phản ứng $HF + H_2O \rightleftharpoons F^- + H_3O^+$, $HF$ là acid.}
	{Mọi base theo Arrhenius đều là base theo Brønsted-Lowry.}
	\loigiai{
			\begin{itemchoice}[T1,F2,T3,T4]
					\itemch  theo định nghĩa acid của Brønsted-Lowry.
					\itemch  Base là chất nhận proton ($H^+$).
					\itemch  $HF$ nhường $H^+$ cho $H_2O$.
					\itemch  Base Arrhenius phân li ra $OH^-$, ion $OH^-$ có khả năng nhận $H^+$ nên là base Brønsted-Lowry.
				\end{itemchoice}
		}
\end{ex}
%%%%%============TF_03================%%%%%%
\begin{ex}
	Về chất lưỡng tính:
	\choiceTF
	{\True $Al(OH)_3$ là một hydroxide lưỡng tính.}
	{Ion $NH_4^+$ là lưỡng tính.}
	{\True Nước ($H_2O$) được coi là một chất lưỡng tính.}
	{Chất lưỡng tính chỉ phản ứng với acid, không phản ứng với base.}
	\loigiai{
			\begin{itemchoice}[T1,F2,T3,F4]
					\itemch  $Al(OH)_3$ tác dụng được với cả acid mạnh và base mạnh.
					\itemch  Ion $NH_4^+$ chỉ có khả năng cho $H^+$ (là acid), không có khả năng nhận $H^+$.
					\itemch  $H_2O$ có thể cho $H^+$ (thành $OH^-$) hoặc nhận $H^+$ (thành $H_3O^+$).
					\itemch  Chất lưỡng tính có khả năng phản ứng với cả acid và base.
				\end{itemchoice}
		}
\end{ex}
%%%%%============TF_04================%%%%%%
\begin{ex}
	Xét các cặp acid-base liên hợp:
	\choiceTF
	{\True Cặp $H_3O^+/H_2O$ là một cặp acid-base liên hợp.}
	{Base liên hợp của $HCl$ là $OH^-$.}
	{\True Acid liên hợp của $CH_3COO^-$ là $CH_3COOH$.}
	{Một acid mạnh có base liên hợp mạnh.}
	\loigiai{
			\begin{itemchoice}[T1,F2,T3,F4]
					\itemch  $H_3O^+$ (acid) $\rightleftharpoons H_2O$ (base liên hợp) + $H^+$.
					\itemch  Base liên hợp của $HCl$ là $Cl^-$.
					\itemch  $CH_3COO^-$ (base) + $H^+$ $\rightleftharpoons CH_3COOH$ (acid liên hợp).
					\itemch  Một acid mạnh có base liên hợp rất yếu.
				\end{itemchoice}
		}
\end{ex}
%%%%%============TF_05================%%%%%%
\begin{ex}
	Phân loại các chất điện li:
	\choiceTF
	{$H_2SO_3$ là chất điện li mạnh.}
	{\True $KNO_3$ là chất điện li mạnh.}
	{$C_2H_5OH$ (ethanol) là chất điện li yếu.}
	{\True Hầu hết các muối đều là chất điện li mạnh.}
	\loigiai{
			\begin{itemchoice}[F1,T2,F3,T4]
					\itemch  $H_2SO_3$ là acid yếu.
					\itemch  $KNO_3$ là muối tan.
					\itemch  $C_2H_5OH$ là chất không điện li.
					\itemch  Ngoại trừ một số muối rất ít tan hoặc muối của kim loại yếu và acid yếu.
				\end{itemchoice}
		}
\end{ex}
%%%%%============TF_06================%%%%%%
\begin{ex}
	Liên quan đến thuyết Arrhenius:
	\choiceTF
	{\True Dung dịch $NaOH$ là một dung dịch base.}
	{Theo Arrhenius, $NH_3$ là một acid.}
	{Chất tạo ra $H^+$ khi tan trong nước là base.}
	{\True Thuyết Arrhenius chỉ áp dụng cho dung môi là nước.}
	\loigiai{
			\begin{itemchoice}[T1,F2,F3,T4]
					\itemch  $NaOH$ phân li ra $OH^-$.
					\itemch  $NH_3$ không trực tiếp phân li ra $H^+$ hay $OH^-$ theo Arrhenius (mặc dù dung dịch $NH_3$ có tính base do $NH_3 + H_2O \rightleftharpoons NH_4^+ + OH^-$).
					\itemch  Chất tạo ra $H^+$ khi tan trong nước là acid.
					\itemch  Đây là một hạn chế của thuyết Arrhenius.
				\end{itemchoice}
		}
\end{ex}
%%%%%============TF_07================%%%%%%
\begin{ex}
	Xác định acid, base theo Brønsted-Lowry:
	\choiceTF
	{Trong phản ứng $CO_3^{2-} + H_2O \rightleftharpoons HCO_3^- + OH^-$, $H_2O$ là base.}
	{\True Ion $CH_3NH_3^+$ là một acid.}
	{Ion $Cl^-$ là một base mạnh.}
	{\True Mọi acid theo Brønsted-Lowry đều chứa hydrogen.}
	\loigiai{
			\begin{itemchoice}[F1,T2,F3,T4]
					\itemch  $H_2O$ cho $H^+$ nên là acid.
					\itemch  $CH_3NH_3^+$ có khả năng cho $H^+$: $CH_3NH_3^+ \rightleftharpoons CH_3NH_2 + H^+$.
					\itemch  $Cl^-$ là base liên hợp của acid mạnh $HCl$, nên là base rất yếu.
					\itemch  Vì acid theo Brønsted-Lowry là chất cho proton ($H^+$), nên phải chứa $H$ có khả năng phân li.
				\end{itemchoice}
		}
\end{ex}
%%%%%============TF_08================%%%%%%
\begin{ex}
	Về tính chất của các dung dịch:
	\choiceTF
	{\True Dung dịch $NH_4NO_3$ có tính acid.}
	{Dung dịch $K_2S$ có tính acid.}
	{Dung dịch $Na_2SO_4$ có tính base.}
	{\True Ion $HCO_3^-$ có thể làm dung dịch có tính acid hoặc base yếu tùy thuộc vào điều kiện.}
	\loigiai{
			\begin{itemchoice}[T1,F2,F3,T4]
					\itemch  $NH_4^+$ thủy phân tạo $H^+$.
					\itemch  $S^{2-}$ thủy phân tạo $OH^-$, dung dịch có tính base.
					\itemch  $Na_2SO_4$ tạo từ acid mạnh và base mạnh, dung dịch trung tính.
					\itemch  $HCO_3^-$ lưỡng tính, $HCO_3^- \rightleftharpoons H^+ + CO_3^{2-}$ ($K_{a2}$) và $HCO_3^- + H_2O \rightleftharpoons H_2CO_3 + OH^-$ ($K_b$). Trong dung dịch $NaHCO_3$, $K_b > K_{a2}$ nên dung dịch có tính base yếu.
				\end{itemchoice}
		}
\end{ex}
%%%%%============TF_09================%%%%%%
\begin{ex}
	Xét các chất sau: $H_2O, HCl, Cl^-, NH_4^+, NH_3$.
	\choiceTF
	{\True $HCl$ là acid Brønsted-Lowry.}
	{$Cl^-$ là acid Brønsted-Lowry.}
	{$NH_3$ là acid Brønsted-Lowry.}
	{\True $H_2O$ có thể đóng vai trò là acid hoặc base Brønsted-Lowry.}
	\loigiai{
			\begin{itemchoice}[T1,F2,F3,T4]
					\itemch  $HCl$ cho $H^+$.
					\itemch  $Cl^-$ là base liên hợp của $HCl$, là base rất yếu.
					\itemch  $NH_3$ là base Brønsted-Lowry (nhận $H^+$).
					\itemch  $H_2O$ là chất lưỡng tính.
				\end{itemchoice}
		}
\end{ex}
%%%%%============TF_10================%%%%%%
\begin{ex}
	Liên quan đến độ mạnh acid-base:
	\choiceTF
	{\True Acid càng mạnh, base liên hợp của nó càng yếu.}
	{Base càng yếu, acid liên hợp của nó càng yếu.}
	{\True $HClO_4$ là một trong những acid mạnh nhất.}
	{$CH_3COOH$ là một base mạnh.}
	\loigiai{
			\begin{itemchoice}[T1,F2,T3,F4]
					\itemch  Đây là một quy tắc quan trọng.
					\itemch  Base càng yếu, acid liên hợp của nó càng mạnh (tương đối).
					\itemch  $HClO_4$ (perchloric acid) là một acid rất mạnh.
					\itemch  $CH_3COOH$ (acetic acid) là một acid yếu.
				\end{itemchoice}
		}
\end{ex}
\Closesolutionfile{ans}
\Closesolutionfile{ansbook}
\Closesolutionfile{ansex}
%\bangdapanTF{AnsTF-C01B02_Dang1}

%%%%========================DẠNG 2=============================%=%%%%
\begin{dang}{Viết phương trình điện li và xác định nồng độ mol ion trong dung dịch chất điện li mạnh}\end{dang}
\begin{phuongphap}
\begin{itemize}
    \item \textbf{Viết phương trình điện li của chất điện li mạnh:}
        \begin{itemize}
	            \item Chất điện li mạnh bao gồm các acid mạnh ($HCl, HNO_3, H_2SO_4,...$), base mạnh ($NaOH, KOH, Ba(OH)_2,...$) và hầu hết các muối tan.
	            \item Khi tan trong nước, chất điện li mạnh phân li hoàn toàn ra ion.
	            \item Sử dụng mũi tên một chiều ($\rightarrow$) trong phương trình điện li.
	            \item Ví dụ: $H_2SO_4 \rightarrow 2H^+ + SO_4^{2-}$
	            \item Ví dụ: $Ba(OH)_2 \rightarrow Ba^{2+} + 2OH^-$
	            \item Ví dụ: $Al_2(SO_4)_3 \rightarrow 2Al^{3+} + 3SO_4^{2-}$
	        \end{itemize}
    \item \textbf{Xác định nồng độ mol ion trong dung dịch chất điện li mạnh:}
        \begin{itemize}
	            \item Từ phương trình điện li, xác định tỉ lệ mol giữa chất điện li và các ion tạo thành.
	            \item Vì chất điện li mạnh phân li hoàn toàn, nồng độ mol của mỗi ion được tính bằng nồng độ mol ban đầu của chất điện li nhân với hệ số tỉ lượng của ion đó trong phương trình điện li.
	            \item Ví dụ: Nếu dung dịch $H_2SO_4$ có nồng độ $C_M$, thì:
				\[
					\begin{array}{ccccl}
						H_2SO_4 & \rightarrow & 2H^+& + &SO_4^{2-} \\
						C_M & \rightarrow & 2C_M& \rightarrow& C_M \quad \text{(mol/L)}
					\end{array}
				\]
	            \\ Vậy $[H^+] = 2C_M$; $[SO_4^{2-}] = C_M$.
	            \item Ví dụ: Dung dịch $Al_2(SO_4)_3$ nồng độ 0,1 M:
				\[
				\begin{array}{ccccl}
					Al_2{SO_4}_3 & \rightarrow  & 2Al^{3+}  & + &3{SO_4}^{2-} \\
					0{,}1        & \rightarrow & 0{,}2      & \rightarrow& 0{,}3 \; \text{(mol/L)}
				\end{array}
				\]
	            \\ Vậy $[Al^{3+}] = 0{,}2 M$; $[SO_4^{2-}] = 0{,}3 M$.
	        \end{itemize}
    \item \textbf{Đối với dung dịch chứa nhiều chất điện li mạnh không phản ứng với nhau:}
        \begin{itemize}
	            \item Viết phương trình điện li của từng chất.
	            \item Tính nồng độ mol của mỗi ion do từng chất điện li tạo ra.
	            \item Nếu có cùng một loại ion được tạo ra từ nhiều chất điện li khác nhau, nồng độ của ion đó trong dung dịch là tổng nồng độ của ion đó do các chất tạo ra.
	            \item Ví dụ: Dung dịch chứa $NaCl$ 0,1 M và $CaCl_2$ 0,2M.
	                \[
	                \begin{array}{lclcl}
	                	NaCl   & \rightarrow & Na^+        & + & Cl^- \\
	                	0,1    & \rightarrow & 0,1         &\rightarrow& 0,1 \\[1em]
	                	CaCl_2 & \rightarrow & Ca^{2+}     & + & 2Cl^- \\
	                	0,2    & \rightarrow & 0,2         &\rightarrow& 2 \times 0,2 = 0,4
	                \end{array}
	                \]
	            \\ Trong dung dịch: $[Na^+] = 0,1 M$; $[Ca^{2+}] = 0,2 M$; $[Cl^-] = 0,1 + 0,4 = 0,5 M$.
	        \end{itemize}
\end{itemize}
\end{phuongphap}
%
\Noibat[\maunhan][][\faBookmark][]{Ví dụ mẫu}
%%%%%==========VD_01==========%%%%%
\begin{vd}
	Viết phương trình điện li và tính nồng độ các ion trong dung dịch $HNO_3$ 0,02 M.
	\choice
	{$[H^+] = 0,01 M; [NO_3^-] = 0,01 M$}
	{\True $[H^+] = 0,02 M; [NO_3^-] = 0,02 M$}
	{$[H^+] = 0,02 M; [NO_3^-] = 0,01 M$}
	{$[H^+] = 0,04 M; [NO_3^-] = 0,02 M$}
	\loigiai{
		$HNO_3$ là acid mạnh, điện li hoàn toàn:
		\[ \text{HNO}_3 \xrightarrow \text{H}^+ + \text{NO}_3^- \]
		Theo phương trình, 1 mol $HNO_3$ điện li ra 1 mol $H^+$ và 1 mol $NO_3^-$.
		Vậy, dung dịch $HNO_3$ 0,02 M có: $[H^+] = 0,02 M$, $[NO_3^-] = 0,02 M$
		}
\end{vd}

%%%%%==========VD_02==========%%%%%
\begin{vd}
	Tính nồng độ mol của ion $OH^-$ trong 200 ml dung dịch chứa 0,01 mol $Ba(OH)_2$. (Coi $Ba(OH)_2$ điện li hoàn toàn).
	\loigiai{
		Nồng độ mol ban đầu của $Ba(OH)_2$: $C_{M_{Ba(OH)_2}} = \frac{0,01 \text{ mol}}{0,2 \text{ L}} = 0,05 M$.
		Phương trình điện li của $Ba(OH)_2$:
		\[ 
			\begin{array}{ccccl}
				Ba(OH)_2 &\rightarrow& Ba^{2+}& + &2OH^- 
			\\
				0{,}05 M & \rightarrow & 0{,}05 M & \rightarrow & 0{,}1 M 
			\end{array}
		\]
		  
		Vậy, nồng độ ion $OH^-$ trong dung dịch là:
		$[OH^-] = 2 \times 0,05 M = 0,1 M$.
		}
\end{vd}

%%%%%==========VD_03==========%%%%%
\begin{vd}
    Một dung dịch chứa đồng thời $Fe_2(SO_4)_3$ 0,05M và $K_2SO_4$ 0,1M. Tính nồng độ ion $SO_4^{2-}$ trong dung dịch.
    \loigiai{
	    Phương trình điện li:
	    \[
	    \begin{array}{ccccc}
	    	Fe_2(SO_4)_3 & \rightarrow & 2Fe^{3+} & + & 3SO_4^{2-}\\
	    	0{,}05 M & \rightarrow & 0{,}1 M &\rightarrow & 0{,}15 M
	    \end{array}
	    \]
	    $[SO_4^{2-}]_{Fe_2(SO_4)_3} = 3 \times 0,05 M = 0,15 M$.
		\[
		\begin{array}{ccccc}
			K_2SO_4 & \rightarrow & 2K^+ & + & SO_4^{2-}\\
			0{,}1 M & \rightarrow & 0{,}2 M &\rightarrow & 0{,}1 M
		\end{array}
		\]
	    $\Rightarrow$ $[SO_4^{2-}]_{K_2SO_4} = 0,1 M$.
		\\
	    Tổng nồng độ ion $SO_4^{2-}$ trong dung dịch:
	    \[
	    [SO_4^{2-}]_{\text{tổng}} = [SO_4^{2-}]_{do Fe_2(SO_4)_3} + [SO_4^{2-}]_{K_2SO_4} = 0,15 M + 0,1 M = 0,25 M.
	    \]
	}
\end{vd}


%%%%%=====================Bài tập tự luyện Dạng 2==========================%%%
\Noibat[\maunhan][][\faBook][]{Bài tập tự luyện}

\phan{Bài tập tự luận}
%%%=============SOẠN BT===============%%%
% Giả sử chương này là chương 1, bài 2
\Opensolutionfile{ansbth}[Ans/LGBT-C01B02_Dang2]
\Opensolutionfile{ansbt}[Ans/AnsBT-C01B02_Dang2]
%%%%%============BT_01================%%%%%%
\begin{bt}
	Viết phương trình điện li của các chất sau trong dung dịch: $HBr, Ca(OH)_2, K_3PO_4, Fe(NO_3)_3, HClO_4$.
	\loigiai{
		\begin{itemize}
			    \item $HBr \rightarrow H^+ + Br^-$
			    \item $Ca(OH)_2 \rightarrow Ca^{2+} + 2OH^-$
			    \item $K_3PO_4 \rightarrow 3K^+ + PO_4^{3-}$
			    \item $Fe(NO_3)_3 \rightarrow Fe^{3+} + 3NO_3^-$
			    \item $HClO_4 \rightarrow H^+ + ClO_4^-$
		\end{itemize}
	}
\end{bt}
%%%%%============BT_02================%%%%%%
\begin{bt}
	Tính nồng độ mol của các ion trong các dung dịch sau:
	\begin{enumerate}
			\item Dung dịch $HCl$ 0,1 M.
			\item Dung dịch $Ba(OH)_2$ 0,02 M.
			\item Dung dịch $Na_2SO_4$ 0,05 M.
		\end{enumerate}
	\loigiai{
		\begin{enumerate}
				\item $HCl \rightarrow H^+ + Cl^-$
				\\ $[H^+] = 0,1 M$; $[Cl^-] = 0,1 M$.
				\item $Ba(OH)_2 \rightarrow Ba^{2+} + 2OH^-$
				\\ $[Ba^{2+}] = 0,02 M$; $[OH^-] = 2 \times 0,02 M = 0,04 M$.
				\item $Na_2SO_4 \rightarrow 2Na^+ + SO_4^{2-}$
				\\ $[Na^+] = 2 \times 0,05 M = 0,1 M$; $[SO_4^{2-}] = 0,05 M$.
			\end{enumerate}
		}
\end{bt}
%%%%%============BT_03================%%%%%%
\begin{bt}
    Hòa tan 11,1 gam $CaCl_2$ vào nước thu được 500 ml dung dịch A.
    \begin{enumerate}
	        \item Viết phương trình điện li của $CaCl_2$ trong dung dịch.
	        \item Tính nồng độ mol của $CaCl_2$ trong dung dịch A.
	        \item Tính nồng độ mol của mỗi ion trong dung dịch A. (Cho $Ca=40, Cl=35,5$)
	    \end{enumerate}
	\loigiai{
	    \begin{enumerate}
		        \item Phương trình điện li: $CaCl_2 \rightarrow Ca^{2+} + 2Cl^-$
		        \item Số mol $CaCl_2$: $n_{CaCl_2} = \frac{11,1}{40 + 2 \times 35,5} = \frac{11,1}{111} = 0,1$ mol.
		        \\ Nồng độ mol của $CaCl_2$: $C_{M_{CaCl_2}} = \frac{0,1 \text{ mol}}{0,5 \text{ L}} = 0,2 M$.
		        \item Theo phương trình điện li:
		        \\ $[Ca^{2+}] = C_{M_{CaCl_2}} = 0,2 M$.
		        \\ $[Cl^-] = 2 \times C_{M_{CaCl_2}} = 2 \times 0,2 M = 0,4 M$.
		 \end{enumerate}
		}
\end{bt}
%%%%%============BT_04================%%%%%%
\begin{bt}
    Trộn 100 ml dung dịch $NaCl$ 0,2 M với 100 ml dung dịch $K_2SO_4$ 0,1 M. Tính nồng độ mol các ion trong dung dịch thu được sau khi trộn (bỏ qua sự thay đổi thể tích không đáng kể khi trộn).
	\loigiai{
	    Thể tích dung dịch sau khi trộn: $V_{dd} = 100 ml + 100 ml = 200 ml = 0,2 L$.
	    Số mol các chất ban đầu:
	    $n_{NaCl} = 0,1 L \times 0,2 M = 0,02$ mol.
	    $n_{K_2SO_4} = 0,1 L \times 0,1 M = 0,01$ mol.
	
	    Phương trình điện li:
	    $NaCl \rightarrow Na^+ + Cl^-$
	    $0,02 \text{ mol} \rightarrow 0,02 \text{ mol} \quad 0,02 \text{ mol}$
	
	    $K_2SO_4 \rightarrow 2K^+ + SO_4^{2-}$
	    $0,01 \text{ mol} \rightarrow 2 \times 0,01 \text{ mol} \quad 0,01 \text{ mol}$
	    $n_{K^+} = 0,02$ mol.
	
	    Nồng độ các ion trong dung dịch sau khi trộn:
	    $[Na^+] = \frac{0,02 \text{ mol}}{0,2 \text{ L}} = 0,1 M$.
	    $[Cl^-] = \frac{0,02 \text{ mol}}{0,2 \text{ L}} = 0,1 M$.
	    $[K^+] = \frac{0,02 \text{ mol}}{0,2 \text{ L}} = 0,1 M$.
	    $[SO_4^{2-}] = \frac{0,01 \text{ mol}}{0,2 \text{ L}} = 0,05 M$.
		}
\end{bt}
%%%%%============BT_05================%%%%%%
\begin{bt}
    Một dung dịch X chứa $a$ mol $Al^{3+}$, $b$ mol $SO_4^{2-}$ và $c$ mol $Cl^-$. Viết biểu thức liên hệ giữa $a, b, c$ theo định luật bảo toàn điện tích. Nếu dung dịch X được tạo thành từ việc hòa tan hai muối là $AlCl_3$ và $Al_2(SO_4)_3$, hãy tìm số mol mỗi muối biết $a=0,5; b=0,6$.
	\loigiai{
	    Theo định luật bảo toàn điện tích, tổng điện tích dương bằng tổng điện tích âm trong dung dịch:
	    $3a = 2b + c$.
	
	    Nếu dung dịch X được tạo thành từ $x$ mol $AlCl_3$ và $y$ mol $Al_2(SO_4)_3$:
	    $AlCl_3 \rightarrow Al^{3+} + 3Cl^-$
	    $x \quad \rightarrow \quad x \quad \quad 3x$
	    $Al_2(SO_4)_3 \rightarrow 2Al^{3+} + 3SO_4^{2-}$
	    $y \quad \rightarrow \quad 2y \quad \quad 3y$
	
	    Ta có:
	    Số mol $Al^{3+}$: $a = x + 2y = 0,5$ (1)
	    Số mol $SO_4^{2-}$: $b = 3y = 0,6 \Rightarrow y = 0,2$ mol.
	    Thay $y = 0,2$ vào (1): $x + 2 \times 0,2 = 0,5 \Rightarrow x + 0,4 = 0,5 \Rightarrow x = 0,1$ mol.
	    Vậy, số mol $AlCl_3$ là 0,1 mol và số mol $Al_2(SO_4)_3$ là 0,2 mol.
		}
\end{bt}
\Closesolutionfile{ansbt}
\Closesolutionfile{ansbth}
%\bangdapanSA{AnsBT-C01B02_Dang2}

\phan{Bài tập trả lời ngắn}
%%%=============SOẠN BT===============%%%
\Opensolutionfile{ansbth}[Ans/LGSA-C01B02_Dang2]
\Opensolutionfile{ansbt}[Ans/AnsSA-C01B02_Dang2]
%%%%%============SA_01================%%%%%%
\begin{bt}
	Dung dịch $H_2SO_4$ 0,05 M có nồng độ ion $H^+$ là bao nhiêu M? (Coi $H_2SO_4$ điện li hoàn toàn cả hai nấc).
	\shortans{0.1}
	\loigiai{$H_2SO_4 \rightarrow 2H^+ + SO_4^{2-}$. $[H^+] = 2 \times 0,05 = 0,1 M$.}
\end{bt}
%%%%%============SA_02================%%%%%%
\begin{bt}
	Nồng độ ion $Na^+$ trong dung dịch $Na_3PO_4$ 0,2 M là bao nhiêu M?
	\shortans{0.6}
	\loigiai{$Na_3PO_4 \rightarrow 3Na^+ + PO_4^{3-}$. $[Na^+] = 3 \times 0,2 = 0,6 M$.}
\end{bt}
%%%%%============SA_03================%%%%%%
\begin{bt}
    Trong dung dịch $FeCl_3$ 0,1M, nồng độ ion $Cl^-$ là bao nhiêu M?
	\shortans{0.3}
	\loigiai{$FeCl_3 \rightarrow Fe^{3+} + 3Cl^-$. $[Cl^-] = 3 \times 0,1 = 0,3 M$.}
\end{bt}
%%%%%============SA_04================%%%%%%
\begin{bt}
    Hòa tan 0,02 mol $KOH$ vào nước thu được 400 ml dung dịch. Nồng độ ion $OH^-$ trong dung dịch là bao nhiêu M?
	\shortans{0.05}
	\loigiai{$C_{M_{KOH}} = \frac{0,02}{0,4} = 0,05 M$. $KOH \rightarrow K^+ + OH^-$. $[OH^-] = 0,05 M$.}
\end{bt}
%%%%%============SA_05================%%%%%%
\begin{bt}
	Dung dịch $Al_2(SO_4)_3$ 0,01 M có tổng nồng độ các ion là bao nhiêu M?
	\shortans{0.05}
	\loigiai{$Al_2(SO_4)_3 \rightarrow 2Al^{3+} + 3SO_4^{2-}$. $[Al^{3+}] = 2 \times 0,01 = 0,02 M$. $[SO_4^{2-}] = 3 \times 0,01 = 0,03 M$. Tổng nồng độ ion $= 0,02 + 0,03 = 0,05 M$.}
\end{bt}
%%%%%============SA_06================%%%%%%
\begin{bt}
    Nồng độ ion $Ba^{2+}$ trong dung dịch $Ba(NO_3)_2$ 0,15 M là bao nhiêu M?
	\shortans{0.15}
	\loigiai{$Ba(NO_3)_2 \rightarrow Ba^{2+} + 2NO_3^-$. $[Ba^{2+}] = 0,15 M$.}
\end{bt}
%%%%%============SA_07================%%%%%%
\begin{bt}
    Trộn 50 ml dung dịch $HCl$ 0,2 M với 50 ml dung dịch $NaCl$ 0,2 M. Nồng độ ion $Cl^-$ trong dung dịch thu được là bao nhiêu M?
	\shortans{0.2}
	\loigiai{$n_{Cl^-(HCl)} = 0,05 \times 0,2 = 0,01$ mol. $n_{Cl^-(NaCl)} = 0,05 \times 0,2 = 0,01$ mol. Tổng $n_{Cl^-} = 0,02$ mol. $V_{dd} = 0,1 L$. $[Cl^-] = \frac{0,02}{0,1} = 0,2 M$.}
\end{bt}
%%%%%============SA_08================%%%%%%
\begin{bt}
    Trong dung dịch $K_2CrO_4$ 0,02M, nồng độ ion $K^+$ gấp mấy lần nồng độ ion $CrO_4^{2-}$?
	\shortans{2}
	\loigiai{$K_2CrO_4 \rightarrow 2K^+ + CrO_4^{2-}$. $[K^+] = 2 \times 0,02 = 0,04 M$. $[CrO_4^{2-}] = 0,02 M$. Vậy $[K^+]$ gấp 2 lần $[CrO_4^{2-}]$.}
\end{bt}
%%%%%============SA_09================%%%%%%
\begin{bt}
    Để có dung dịch chứa $Mg^{2+}$ 0,1M; $NO_3^-$ 0,2M thì nồng độ mol của $Mg(NO_3)_2$ cần dùng là bao nhiêu M?
	\shortans{0.1}
	\loigiai{$Mg(NO_3)_2 \rightarrow Mg^{2+} + 2NO_3^-$. Nếu $[Mg^{2+}] = 0,1M$ thì $[NO_3^-] = 0,2M$. Vậy $C_{M_{Mg(NO_3)_2}} = 0,1M$.}
\end{bt}
%%%%%============SA_10================%%%%%%
\begin{bt}
    Dung dịch $A$ chứa 0,1 mol $Na^+$, 0,2 mol $Mg^{2+}$, 0,1 mol $Cl^-$ và $x$ mol $SO_4^{2-}$. Giá trị của $x$ là bao nhiêu?
	\shortans{0.2}
	\loigiai{Bảo toàn điện tích: $1 \times 0,1 + 2 \times 0,2 = 1 \times 0,1 + 2x \Rightarrow 0,1 + 0,4 = 0,1 + 2x \Rightarrow 0,4 = 2x \Rightarrow x = 0,2$.}
\end{bt}
\Closesolutionfile{ansbt}
\Closesolutionfile{ansbth}
%\bangdapanSA{AnsSA-C01B02_Dang2}

%
%%%%============Phần trắc nghiệm============%%%
\phan{Trắc nghiệm nhiều lựa chọn}
%%%=============SOẠN EX===============%%%
\Opensolutionfile{ansex}[Ans/LGEX-C01B02_Dang2]
\Opensolutionfile{ans}[Ans/Ans-C01B02_Dang2]
%%%%%============EX_01================%%%%%%
\begin{ex}
	Phương trình điện li nào sau đây được viết đúng?
	\choice
	{$H_2CO_3 \rightarrow 2H^+ + CO_3^{2-}$}
	{$CH_3COOH \rightarrow CH_3COO^- + H^+$}
	{\True $Na_2SO_4 \rightarrow 2Na^+ + SO_4^{2-}$}
	{$Mg(OH)_2 \rightarrow Mg^{2+} + (OH)_2^-$}
	\loigiai{$Na_2SO_4$ là chất điện li mạnh, phân li hoàn toàn. A sai vì $H_2CO_3$ là acid yếu, điện li hai chiều và từng nấc. B sai vì $CH_3COOH$ là acid yếu, dùng mũi tên hai chiều. D sai công thức ion $OH^-$.}
\end{ex}
%%%%%============EX_02================%%%%%%
\begin{ex}
	Trong dung dịch $KCl$ 0,1M, nồng độ của ion $K^+$ là
	\choice
	{$0,05 M$}
	{\True $0,1 M$}
	{$0,2 M$}
	{$0 M$}
	\loigiai{$KCl \rightarrow K^+ + Cl^-$. $KCl$ là chất điện li mạnh. $[K^+] = C_{M_{KCl}} = 0,1 M$.}
\end{ex}
%%%%%============EX_03================%%%%%%
\begin{ex}
	Dung dịch $Ba(OH)_2$ 0,005M có nồng độ ion $OH^-$ là
	\choice
	{$0,0025 M$}
	{$0,005 M$}
	{\True $0,01 M$}
	{$0,015 M$}
	\loigiai{$Ba(OH)_2 \rightarrow Ba^{2+} + 2OH^-$. $[OH^-] = 2 \times C_{M_{Ba(OH)_2}} = 2 \times 0,005 = 0,01 M$.}
\end{ex}
%%%%%============EX_04================%%%%%%
\begin{ex}
	Hòa tan $x$ mol $AlCl_3$ vào nước, thu được dung dịch chứa
	\choice
	{$x$ mol $Al^{3+}$ và $x$ mol $Cl^-$}
	{$x$ mol $Al^{3+}$ và $2x$ mol $Cl^-$}
	{\True $x$ mol $Al^{3+}$ và $3x$ mol $Cl^-$}
	{$3x$ mol $Al^{3+}$ và $x$ mol $Cl^-$}
	\loigiai{$AlCl_3 \rightarrow Al^{3+} + 3Cl^-$. Từ $x$ mol $AlCl_3$ tạo ra $x$ mol $Al^{3+}$ và $3x$ mol $Cl^-$.}
\end{ex}
%%%%%============EX_05================%%%%%%
\begin{ex}
	Dung dịch $X$ chứa $Na_2SO_4$ 0,02M. Nồng độ ion $Na^+$ và $SO_4^{2-}$ lần lượt là
	\choice
	{$0,02M$ và $0,02M$}
	{$0,04M$ và $0,04M$}
	{\True $0,04M$ và $0,02M$}
	{$0,02M$ và $0,04M$}
	\loigiai{$Na_2SO_4 \rightarrow 2Na^+ + SO_4^{2-}$. $[Na^+] = 2 \times 0,02 = 0,04 M$. $[SO_4^{2-}] = 0,02 M$.}
\end{ex}
%%%%%============EX_06================%%%%%%
\begin{ex}
	Cho dung dịch $(NH_4)_2SO_4$ 1M. Nồng độ mol của ion $NH_4^+$ và $SO_4^{2-}$ tương ứng là
	\choice
	{$1M$ và $1M$}
	{$1M$ và $2M$}
	{$2M$ và $2M$}
	{\True $2M$ và $1M$}
	\loigiai{$(NH_4)_2SO_4 \rightarrow 2NH_4^+ + SO_4^{2-}$. $[NH_4^+] = 2 \times 1 = 2M$. $[SO_4^{2-}] = 1M$.}
\end{ex}
%%%%%============EX_07================%%%%%%
\begin{ex}
	Trong dung dịch $Fe_2(SO_4)_3$ 0,05M, tổng nồng độ các ion là
	\choice
	{$0,05M$}
	{$0,10M$}
	{$0,15M$}
	{\True $0,25M$}
	\loigiai{$Fe_2(SO_4)_3 \rightarrow 2Fe^{3+} + 3SO_4^{2-}$. $[Fe^{3+}] = 2 \times 0,05 = 0,1M$. $[SO_4^{2-}] = 3 \times 0,05 = 0,15M$. Tổng nồng độ ion $= 0,1 + 0,15 = 0,25M$.}
\end{ex}
%%%%%============EX_08================%%%%%%
\begin{ex}
    Hòa tan 0,1 mol $MgCl_2$ và 0,2 mol $NaCl$ vào nước thu được dung dịch A. Nồng độ ion $Cl^-$ trong dung dịch A là (giả sử thể tích dung dịch là 1 lít).
	\choice
	{$0,2M$}
	{$0,3M$}
	{\True $0,4M$}
	{$0,5M$}
	\loigiai{$MgCl_2 \rightarrow Mg^{2+} + 2Cl^-$. $n_{Cl^-(MgCl_2)} = 2 \times 0,1 = 0,2$ mol.
	    $NaCl \rightarrow Na^+ + Cl^-$. $n_{Cl^-(NaCl)} = 0,2$ mol.
	    Tổng $n_{Cl^-} = 0,2 + 0,2 = 0,4$ mol.
	    Nếu $V=1L$, $[Cl^-] = 0,4M$.}
\end{ex}
%%%%%============EX_09================%%%%%%
\begin{ex}
    Dung dịch $X$ chứa $K^+$ 0,1M; $Mg^{2+}$ 0,2M; $Cl^-$ 0,3M và $SO_4^{2-}$ $x$M. Giá trị của $x$ là
	\choice
	{$0,05$}
	{$0,1$}
	{\True $0,1$} % Sửa lại, 0.1*1 + 0.2*2 = 0.5. 0.3*1 + x*2 = 0.5 => 2x = 0.2 => x = 0.1
	{$0,2$}
	\loigiai{Bảo toàn điện tích: Tổng điện tích dương = Tổng điện tích âm.
	    $0,1 \times (+1) + 0,2 \times (+2) = 0,3 \times (-1) + x \times (-2)$
	    $0,1 + 0,4 = -0,3 - 2x$
	    $0,5 = -0,3 - 2x$ là sai.
	    Phải là $0,1 \times 1 + 0,2 \times 2 = 0,3 \times 1 + x \times 2$
	    $0,1 + 0,4 = 0,3 + 2x$
	    $0,5 = 0,3 + 2x \Rightarrow 2x = 0,2 \Rightarrow x = 0,1$.}
\end{ex}
%%%%%============EX_10================%%%%%%
\begin{ex}
    Phương trình điện li nào không đúng cho chất điện li mạnh?
	\choice
	{$HClO_4 \rightarrow H^+ + ClO_4^-$}
	{$BaCl_2 \rightarrow Ba^{2+} + 2Cl^-$}
	{\True $H_2SO_4 \rightleftharpoons 2H^+ + SO_4^{2-}$}
	{$KNO_3 \rightarrow K^+ + NO_3^-$}
	\loigiai{$H_2SO_4$ là acid mạnh, điện li hoàn toàn, dùng mũi tên một chiều ($\rightarrow$). Mũi tên hai chiều ($\rightleftharpoons$) dùng cho chất điện li yếu.}
\end{ex}
%%%%%============EX_11================%%%%%%
\begin{ex}
	Khi hòa tan $NaCl$ vào nước, ion $Na^+$ được bao quanh bởi các phân tử nước định hướng như thế nào?
	\choice
	{Nguyên tử $H$ của $H_2O$ hướng về phía $Na^+$.}
	{\True Nguyên tử $O$ của $H_2O$ hướng về phía $Na^+$.}
	{Cả $H$ và $O$ đều hướng về $Na^+$.}
	{Các phân tử $H_2O$ định hướng ngẫu nhiên.}
	\loigiai{Ion $Na^+$ mang điện tích dương. Nguyên tử $O$ trong $H_2O$ mang một phần điện tích âm (do O có độ âm điện lớn hơn H), do đó nguyên tử $O$ sẽ bị hút về phía ion $Na^+$.}
\end{ex}
%%%%%============EX_12================%%%%%%
\begin{ex}
    Trong dung dịch $CH_3COONa$ 0,1M (muối của acid yếu và base mạnh), nồng độ ion $Na^+$ là
	\choice
	{Nhỏ hơn 0,1M do $Na^+$ bị thủy phân.}
	{\True Bằng 0,1M.}
	{Lớn hơn 0,1M.}
	{Không xác định được.}
	\loigiai{$CH_3COONa$ là chất điện li mạnh: $CH_3COONa \rightarrow CH_3COO^- + Na^+$. Do đó $[Na^+] = 0,1M$. Ion $Na^+$ không bị thủy phân đáng kể.}
\end{ex}
%%%%%============EX_13================%%%%%%
\begin{ex}
    Một dung dịch chứa 0,02 mol $Cu^{2+}$, 0,03 mol $K^+$, $x$ mol $Cl^-$ và $y$ mol $SO_4^{2-}$. Tổng khối lượng các muối khan thu được khi cô cạn dung dịch là 5,435 gam. Giá trị của $x$ và $y$ lần lượt là (Cho $Cu=64, K=39, Cl=35,5, S=32, O=16$)
	\choice
	{$0,02$ và $0,03$}
	{$0,03$ và $0,02$}
	{\True $0,02$ và $0,025$} % Sửa lại
	{$0,01$ và $0,035$}
	\loigiai{}
\end{ex}
%%%%%============EX_14================%%%%%%
\begin{ex}
    Nồng độ mol của ion $NO_3^-$ trong dung dịch $Ca(NO_3)_2$ 0,2M là:
    \choice
    {$0,1M$}
    {$0,2M$}
    {\True $0,4M$}
    {$0,6M$}
    \loigiai{$Ca(NO_3)_2 \rightarrow Ca^{2+} + 2NO_3^-$. $[NO_3^-] = 2 \times 0,2M = 0,4M$.}
\end{ex}
%%%%%============EX_15================%%%%%%
\begin{ex}
    Dung dịch nào sau đây có nồng độ ion $H^+$ lớn nhất (coi các acid điện li hoàn toàn)?
    \choice
    {Dung dịch $HCl$ 0,1M.}
    {\True Dung dịch $H_2SO_4$ 0,1M.}
    {Dung dịch $HNO_3$ 0,1M.}
    {Dung dịch $CH_3COOH$ 0,1M.}
    \loigiai{$HCl \rightarrow H^+ + Cl^- \Rightarrow [H^+] = 0,1M$.
	    $H_2SO_4 \rightarrow 2H^+ + SO_4^{2-} \Rightarrow [H^+] = 2 \times 0,1M = 0,2M$.
	    $HNO_3 \rightarrow H^+ + NO_3^- \Rightarrow [H^+] = 0,1M$.
	    $CH_3COOH$ là acid yếu, $[H^+] < 0,1M$.
	    Vậy dung dịch $H_2SO_4$ 0,1M có $[H^+]$ lớn nhất.}
\end{ex}
%%%%%============EX_16================%%%%%%
\begin{ex}
    Nếu một dung dịch chứa $FeCl_3$ có $[Cl^-] = 0,6M$ thì nồng độ của $FeCl_3$ là:
    \choice
    {$0,1M$}
    {\True $0,2M$}
    {$0,3M$}
    {$0,6M$}
    \loigiai{$FeCl_3 \rightarrow Fe^{3+} + 3Cl^-$. Gọi nồng độ $FeCl_3$ là $C_M$.
	    $[Cl^-] = 3 \times C_M = 0,6M \Rightarrow C_M = \frac{0,6}{3} = 0,2M$.}
\end{ex}
%%%%%============EX_17================%%%%%%
\begin{ex}
    Trộn 200ml dung dịch $NaOH$ 0,1M với 300ml dung dịch $KOH$ 0,2M. Nồng độ ion $OH^-$ trong dung dịch thu được là (bỏ qua sự thay đổi thể tích):
    \choice
    {$0,12M$}
    {$0,14M$}
    {\True $0,16M$}
    {$0,15M$}
    \loigiai{$n_{OH^-(NaOH)} = 0,2L \times 0,1M = 0,02$ mol.
	    $n_{OH^-(KOH)} = 0,3L \times 0,2M = 0,06$ mol.
	    Tổng $n_{OH^-} = 0,02 + 0,06 = 0,08$ mol.
	    $V_{dd \text{ sau trộn}} = 200ml + 300ml = 500ml = 0,5L$.
	    $[OH^-] = \frac{0,08 \text{ mol}}{0,5 \text{ L}} = 0,16M$.}
\end{ex}
%%%%%============EX_18================%%%%%%
\begin{ex}
    Khi pha loãng dung dịch $HCl$ 10 lần thì nồng độ ion $H^+$
    \choice
    {tăng 10 lần.}
    {\True giảm 10 lần.}
    {không đổi.}
    {giảm 100 lần.}
    \loigiai{$HCl$ là acid mạnh, $[H^+]$ bằng nồng độ $HCl$. Khi pha loãng dung dịch 10 lần, nồng độ $HCl$ giảm 10 lần, do đó nồng độ $H^+$ cũng giảm 10 lần.}
\end{ex}
%%%%%============EX_19================%%%%%%
\begin{ex}
    Dung dịch $X$ có chứa $a$ mol $K^+$, $b$ mol $Mg^{2+}$, $c$ mol $NO_3^-$ và $d$ mol $Cl^-$. Biểu thức nào sau đây đúng theo định luật bảo toàn điện tích?
    \choice
    {$a + b = c + d$}
    {$a + 2b = c + d$}
    {\True $a + 2b = c + d$} % Trùng với B, sửa lại
    {$a + b = 2c + d$}
    \loigiai{Tổng điện tích dương: $a \times (+1) + b \times (+2) = a + 2b$.
	    Tổng điện tích âm: $c \times (-1) + d \times (-1) = -(c+d)$.
	    Độ lớn tổng điện tích dương bằng độ lớn tổng điện tích âm: $a + 2b = c + d$.
	    B và C giống nhau. Chọn B.
	    \choice
		{$a + b = c + d$}
		{\True $a + 2b = c + d$}
		{$a - 2b = c - d$} %Sửa C
		{$a + b = 2c + d$}
	    }
\end{ex}
%%%%%============EX_20================%%%%%%
\begin{ex}
    Chất điện li mạnh là chất khi tan trong nước:
    \choice
    {Chỉ một phần số phân tử hòa tan phân li ra ion.}
    {Không phân li ra ion.}
    {\True Tất cả các phân tử hòa tan đều phân li hoàn toàn ra ion.}
    {Tạo ra số mol ion dương bằng số mol ion âm.}
    \loigiai{Theo định nghĩa, chất điện li mạnh là chất khi tan trong nước, tất cả các phân tử hòa tan đều phân li hoàn toàn ra ion. D không phải lúc nào cũng đúng (ví dụ $Al_2(SO_4)_3$).}
\end{ex}
\Closesolutionfile{ans}
\Closesolutionfile{ansex}
%\bangdapan{Ans-C01B02_Dang2}

%%%%%%%%%%%%%%%Trắc nghiệm đúng sai%%%%%%%%%%%%%%%%%%%%%%%%
\phan{Bài tập trắc nghiệm Đúng Sai}
%%%=============SOẠN EXTF===============%%%
\Opensolutionfile{ansex}[Ans/LGTF-C01B02_Dang2]
\Opensolutionfile{ansbook}[Ansbook/AnsTF-C01B02_Dang2]
\Opensolutionfile{ans}[Ans/Tempt-C01B02_Dang2]
%%%%%============TF_01================%%%%%%
\begin{ex}
	Về phương trình điện li của chất điện li mạnh:
	\choiceTF
	{\True $HNO_3 \rightarrow H^+ + NO_3^-$}
	{$H_2SO_4 \rightarrow H^+ + HSO_4^-$ (là phương trình điện li hoàn toàn).}
	{\True $Ba(OH)_2 \rightarrow Ba^{2+} + 2OH^-$}
	{$Fe_2(SO_4)_3 \rightarrow Fe^{3+} + (SO_4)_3^{2-}$}
	\loigiai{
			\begin{itemchoice}[T1,F2,T3,F4]
					\itemch Đúng. $HNO_3$ là acid mạnh.
					\itemch Sai. $H_2SO_4$ điện li hoàn toàn thành $2H^+$ và $SO_4^{2-}$ (nếu xét tổng thể). Phương trình $H_2SO_4 \rightarrow H^+ + HSO_4^-$ là nấc 1, sau đó $HSO_4^-$ tiếp tục điện li (nếu là acid mạnh thì HSO4- cũng điện li mạnh). Tuy nhiên, $H_2SO_4 \rightarrow 2H^+ + SO_4^{2-}$ là biểu diễn tổng quát cho sự điện li hoàn toàn. Phát biểu "là phương trình điện li hoàn toàn" cho nấc 1 là chưa đủ.
					\itemch Đúng. $Ba(OH)_2$ là base mạnh.
					\itemch Sai. Sai công thức ion sulfate và cân bằng. Phải là $Fe_2(SO_4)_3 \rightarrow 2Fe^{3+} + 3SO_4^{2-}$.
				\end{itemchoice}
		}
\end{ex}
%%%%%============TF_02================%%%%%%
\begin{ex}
	Nồng độ ion trong dung dịch $HCl$ 0,2M:
	\choiceTF
	{\True $[H^+] = 0,2M$.}
	{$[Cl^-] = 0,1M$.}
	{Tổng nồng độ các ion là 0,2M.}
	{\True $[HCl]_{\text{chưa điện li}} \approx 0M$.}
	\loigiai{
			\begin{itemchoice}[T1,F2,F3,T4]
					\itemch Đúng. $HCl \rightarrow H^+ + Cl^-$, $[H^+]=0,2M$.
					\itemch Sai. $[Cl^-]=0,2M$.
					\itemch Sai. Tổng nồng độ ion $= [H^+] + [Cl^-] = 0,2 + 0,2 = 0,4M$.
					\itemch Đúng. $HCl$ là acid mạnh, điện li hoàn toàn.
				\end{itemchoice}
		}
\end{ex}
%%%%%============TF_03================%%%%%%
\begin{ex}
	Trong dung dịch $K_2SO_4$ 0,05M:
	\choiceTF
	{\True $[K^+] = 0,1M$.}
	{$[SO_4^{2-}] = 0,1M$.}
	{Nồng độ $K^+$ bằng nồng độ $SO_4^{2-}$.}
	{\True $K_2SO_4$ là chất điện li mạnh.}
	\loigiai{
			\begin{itemchoice}[T1,F2,F3,T4]
					\itemch Đúng. $K_2SO_4 \rightarrow 2K^+ + SO_4^{2-}$. $[K^+] = 2 \times 0,05 = 0,1M$.
					\itemch Sai. $[SO_4^{2-}] = 0,05M$.
					\itemch Sai. $[K^+] = 2[SO_4^{2-}]$.
					\itemch Đúng. $K_2SO_4$ là muối tan.
				\end{itemchoice}
		}
\end{ex}
%%%%%============TF_04================%%%%%%
\begin{ex}
	Xét dung dịch $Al(NO_3)_3$ nồng độ $C$ (mol/L):
	\choiceTF
	{\True Nồng độ ion $Al^{3+}$ là $C$ (mol/L).}
	{Nồng độ ion $NO_3^-$ là $C$ (mol/L).}
	{\True Tổng nồng độ các ion trong dung dịch là $4C$ (mol/L).}
	{Phản ứng điện li là: $Al(NO_3)_3 \rightleftharpoons Al^{3+} + 3NO_3^-$.}
	\loigiai{
			\begin{itemchoice}[T1,F2,T3,F4]
					\itemch Đúng. $Al(NO_3)_3 \rightarrow Al^{3+} + 3NO_3^-$. $[Al^{3+}] = C$.
					\itemch Sai. $[NO_3^-] = 3C$.
					\itemch Đúng. Tổng nồng độ ion $= [Al^{3+}] + [NO_3^-] = C + 3C = 4C$.
					\itemch Sai. $Al(NO_3)_3$ là chất điện li mạnh, dùng mũi tên $\rightarrow$.
				\end{itemchoice}
		}
\end{ex}
%%%%%============TF_05================%%%%%%
\begin{ex}
	Khi hòa tan các chất điện li mạnh vào nước:
	\choiceTF
	{\True Số mol ion tạo thành phụ thuộc vào hệ số trong phương trình điện li.}
	{Nồng độ của chất điện li không thay đổi.}
	{\True Dung dịch thu được luôn dẫn điện.}
	{Tất cả các phân tử chất tan đều tồn tại dưới dạng ion.}
	\loigiai{
			\begin{itemchoice}[T1,F2,T3,T4]
					\itemch Đúng.
					\itemch Sai. Nồng độ chất điện li (phân tử ban đầu) giảm xuống gần bằng 0.
					\itemch Đúng. Do có các ion tự do.
					\itemch Đúng. Đây là đặc điểm của chất điện li mạnh.
				\end{itemchoice}
		}
\end{ex}
%%%%%============TF_06================%%%%%%
\begin{ex}
	Một dung dịch chứa $NaCl$ 0,1M và $MgCl_2$ 0,1M.
	\choiceTF
	{\True Nồng độ ion $Na^+$ là 0,1M.}
	{Nồng độ ion $Mg^{2+}$ là 0,2M.}
	{\True Nồng độ ion $Cl^-$ là 0,3M.}
	{Tổng nồng độ cation bằng tổng nồng độ anion.}
	\loigiai{
			\begin{itemchoice}[T1,F2,T3,F4]
					\itemch Đúng. Từ $NaCl \rightarrow Na^+ + Cl^-$.
					\itemch Sai. Từ $MgCl_2 \rightarrow Mg^{2+} + 2Cl^-$, $[Mg^{2+}] = 0,1M$.
					\itemch Đúng. $[Cl^-] = [Cl^-]_{NaCl} + [Cl^-]_{MgCl_2} = 0,1M + 2 \times 0,1M = 0,3M$.
					\itemch Sai. Tổng điện tích dương bằng tổng điện tích âm. Tổng nồng độ cation (mol ion/L) không nhất thiết bằng tổng nồng độ anion (mol ion/L). Ở đây, tổng nồng độ cation $= [Na^+] + [Mg^{2+}] = 0,1 + 0,1 = 0,2M$. Tổng nồng độ anion $= [Cl^-] = 0,3M$.
				\end{itemchoice}
		}
\end{ex}
%%%%%============TF_07================%%%%%%
\begin{ex}
	Về sự điện li của $Ca(OH)_2$:
	\choiceTF
	{\True $Ca(OH)_2$ là một base mạnh.}
	{$Ca(OH)_2 \rightarrow Ca^{2+} + OH_2^{2-}$.}
	{Trong dung dịch $Ca(OH)_2$ 0,01M, $[Ca^{2+}] = 0,02M$.}
	{\True Trong dung dịch $Ca(OH)_2$ 0,01M, $[OH^-] = 0,02M$.}
	\loigiai{
			\begin{itemchoice}[T1,F2,F3,T4]
					\itemch Đúng. $Ca(OH)_2$ là base mạnh (tuy ít tan nhưng phần tan điện li hoàn toàn).
					\itemch Sai. Phương trình đúng: $Ca(OH)_2 \rightarrow Ca^{2+} + 2OH^-$.
					\itemch Sai. $[Ca^{2+}] = 0,01M$.
					\itemch Đúng. $[OH^-] = 2 \times 0,01 = 0,02M$.
				\end{itemchoice}
		}
\end{ex}
%%%%%============TF_08================%%%%%%
\begin{ex}
	Dung dịch muối $Fe_2(SO_4)_3$:
	\choiceTF
	{\True Là chất điện li mạnh.}
	{Phương trình điện li: $Fe_2(SO_4)_3 \rightarrow Fe_2^{6+} + 3SO_4^{2-}$.}
	{\True Trong dung dịch 0,1M $Fe_2(SO_4)_3$, nồng độ $Fe^{3+}$ là 0,2M.}
	{Nồng độ $SO_4^{2-}$ gấp 2 lần nồng độ $Fe^{3+}$.}
	\loigiai{
			\begin{itemchoice}[T1,F2,T3,F4]
					\itemch Đúng. Muối tan là chất điện li mạnh.
					\itemch Sai. Ion sắt là $Fe^{3+}$. PT: $Fe_2(SO_4)_3 \rightarrow 2Fe^{3+} + 3SO_4^{2-}$.
					\itemch Đúng. $[Fe^{3+}] = 2 \times 0,1 = 0,2M$.
					\itemch Sai. Nồng độ $SO_4^{2-}$ là $3 \times C_M$, nồng độ $Fe^{3+}$ là $2 \times C_M$. Tỉ lệ mol là $3SO_4^{2-} : 2Fe^{3+}$. Vậy $[SO_4^{2-}] = \frac{3}{2} [Fe^{3+}]$.
				\end{itemchoice}
		}
\end{ex}
%%%%%============TF_09================%%%%%%
\begin{ex}
	Đối với dung dịch $HNO_2$ (acid yếu) và $HNO_3$ (acid mạnh) cùng nồng độ 0,1M:
	\choiceTF
	{\True Nồng độ $H^+$ trong dung dịch $HNO_3$ lớn hơn trong dung dịch $HNO_2$.}
	{Cả hai dung dịch đều có $[H^+] = 0,1M$.}
	{\True $HNO_3$ điện li hoàn toàn, còn $HNO_2$ điện li một phần.}
	{Số ion trong dung dịch $HNO_2$ nhiều hơn trong dung dịch $HNO_3$.}
	\loigiai{
			\begin{itemchoice}[T1,F2,T3,F4]
					\itemch Đúng. $HNO_3$ điện li hoàn toàn, $HNO_2$ điện li yếu.
					\itemch Sai. Chỉ $HNO_3$ có $[H^+] = 0,1M$. Trong $HNO_2$, $[H^+] < 0,1M$.
					\itemch Đúng. Theo định nghĩa chất điện li mạnh và yếu.
					\itemch Sai. Do $HNO_3$ điện li hoàn toàn nên tạo ra nhiều ion hơn $HNO_2$ (điện li một phần).
				\end{itemchoice}
		}
\end{ex}
%%%%%============TF_10================%%%%%%
\begin{ex}
	Bảo toàn điện tích trong dung dịch:
	\choiceTF
	{\True Tổng điện tích dương của các cation bằng tổng điện tích âm của các anion.}
	{Tổng số mol cation luôn bằng tổng số mol anion.}
	{Trong dung dịch $NaCl$, $[Na^+]$ + $[Cl^-] = 0$.}
	{\True Trong dung dịch $MgSO_4$, $[Mg^{2+}] = [SO_4^{2-}]$.}
	\loigiai{
			\begin{itemchoice}[T1,F2,F3,T4]
					\itemch Đúng. Đây là nguyên tắc cơ bản của dung dịch trung hòa điện.
					\itemch Sai. Ví dụ $Al_2(SO_4)_3 \rightarrow 2Al^{3+} + 3SO_4^{2-}$. Số mol $Al^{3+}$ là 2, số mol $SO_4^{2-}$ là 3 (nếu xét từ 1 mol muối).
					\itemch Sai. Phải là $1 \times [Na^+] + (-1) \times [Cl^-] = 0 \Rightarrow [Na^+] = [Cl^-]$ về nồng độ. Tổng điện tích bằng 0, không phải tổng nồng độ.
					\itemch Đúng. $MgSO_4 \rightarrow Mg^{2+} + SO_4^{2-}$. Do đó nồng độ của chúng bằng nhau.
				\end{itemchoice}
		}
\end{ex}
\Closesolutionfile{ans}
\Closesolutionfile{ansbook}
\Closesolutionfile{ansex}
%\bangdapanTF{AnsTF-C01B02_Dang2}
%%%%%%%%==========DẠNG 3================%%%%%%%%%%%%%%
\begin{dang}{pH của dung dịch, ý nghĩa của pH, phản ứng trao đổi ion trong dung dịch}
\end{dang}
\begin{phuongphap}
\begin{itemize}
    \item \textbf{Sự điện li của nước và pH:}
        \begin{itemize}
	            \item Nước là chất điện li rất yếu: $H_2O \rightleftharpoons H^+ + OH^-$.
	            \item Tích số ion của nước: $K_w = [H^+][OH^-]$. Ở $25^\circ C$, $K_w = 1,0 \cdot 10^{-14}$.
	            \item Trong nước tinh khiết (môi trường trung tính), $[H^+] = [OH^-] = 1,0 \cdot 10^{-7} M$ (ở $25^\circ C$).
	            \item Môi trường acid: $[H^+] > [OH^-] \Rightarrow [H^+] > 1,0 \cdot 10^{-7} M$.
	            \item Môi trường base (kiềm): $[H^+] < [OH^-] \Rightarrow [H^+] < 1,0 \cdot 10^{-7} M$.
	            \item Khái niệm pH: $pH = -\log[H^+]$ hoặc $[H^+] = 10^{-pH}$.
	            \item Khái niệm pOH: $pOH = -\log[OH^-]$.
	            \item Mối quan hệ: $pH + pOH = 14$ (ở $25^\circ C$).
	            \item Đánh giá môi trường dựa vào pH (ở $25^\circ C$):
	                \begin{itemize}
		                    \item Môi trường acid: $pH < 7$.
		                    \item Môi trường trung tính: $pH = 7$.
		                    \item Môi trường base: $pH > 7$.
		                \end{itemize}
	            \item \textbf{Tính pH dung dịch:}
	                \begin{itemize}
		                    \item Acid mạnh ($HCl, H_2SO_4,...$ nồng độ $C_a$): Tính $[H^+]$ theo sự điện li hoàn toàn, rồi tính $pH$. Ví dụ $HCl \rightarrow H^+ + Cl^-$, $[H^+]=C_a$. $H_2SO_4 \rightarrow 2H^+ + SO_4^{2-}$, $[H^+]=2C_a$.
		                    \item Base mạnh ($NaOH, Ba(OH)_2,...$ nồng độ $C_b$): Tính $[OH^-]$ theo sự điện li hoàn toàn, rồi tính $pOH \Rightarrow pH = 14 - pOH$. Ví dụ $NaOH \rightarrow Na^+ + OH^-$, $[OH^-]=C_b$. $Ba(OH)_2 \rightarrow Ba^{2+} + 2OH^-$, $[OH^-]=2C_b$.
		                    \item Acid yếu ($HA$ nồng độ $C_a$, hằng số $K_a$): Tính $[H^+]$ từ cân bằng $HA \rightleftharpoons H^+ + A^-$. Thường $[H^+] \approx \sqrt{K_a \cdot C_a}$ (nếu $C_a/K_a$ lớn). Rồi tính $pH$.
		                    \item Base yếu ($B$ nồng độ $C_b$, hằng số $K_b$): Tính $[OH^-]$ từ cân bằng $B + H_2O \rightleftharpoons BH^+ + OH^-$. Thường $[OH^-] \approx \sqrt{K_b \cdot C_b}$. Rồi tính $pOH \Rightarrow pH$.
		                \end{itemize}
	            \item \textbf{Chất chỉ thị acid-base:} Là chất có màu biến đổi phụ thuộc vào giá trị pH của dung dịch. Ví dụ: quỳ tím (đỏ trong acid, xanh trong base, không đổi màu trong trung tính), phenolphthalein (không màu trong acid và trung tính, hồng trong base có $pH \ge 8,3$).
	        \end{itemize}
    \item \textbf{Phản ứng trao đổi ion trong dung dịch các chất điện li:}
        \begin{itemize}
	            \item \textbf{Điều kiện xảy ra phản ứng:} Phản ứng trao đổi ion trong dung dịch các chất điện li chỉ xảy ra khi các ion kết hợp được với nhau tạo thành ít nhất một trong các loại chất sau:
	                \begin{enumerate}
		                    \item Chất kết tủa (chất không tan hoặc ít tan). (Dựa vào bảng tính tan).
		                    \item Chất khí (bay hơi ra khỏi dung dịch). (Ví dụ: $CO_2, SO_2, H_2S, NH_3$).
		                    \item Chất điện li yếu (ví dụ: $H_2O$, acid yếu $CH_3COOH$, base yếu).
		                \end{enumerate}
	            \item \textbf{Bản chất của phản ứng:} Là sự tương tác giữa các ion trong dung dịch để tạo ra sản phẩm thỏa mãn một trong các điều kiện trên, làm giảm nồng độ của một số ion trong dung dịch.
	            \item \textbf{Phương trình ion rút gọn:}
	                \begin{itemize}
		                    \item Cho biết bản chất của phản ứng trong dung dịch các chất điện li.
		                    \item Trong phương trình ion rút gọn, các chất kết tủa, chất khí, chất điện li yếu được viết dưới dạng phân tử. Các chất điện li mạnh tan được viết dưới dạng ion.
		                    \item Lược bỏ các ion không tham gia trực tiếp vào phản ứng (các ion "khán giả").
		                \end{itemize}
	            \item \textbf{Các bước viết phương trình ion rút gọn:}
	                \begin{enumerate}
		                    \item Viết phương trình hóa học dạng phân tử (nếu cần).
		                    \item Chuyển các chất điện li mạnh tan thành ion (phương trình ion đầy đủ). Giữ nguyên dạng phân tử đối với chất rắn, chất khí, chất điện li yếu.
		                    \item Lược bỏ các ion giống nhau ở cả hai vế (ion không tham gia phản ứng) để được phương trình ion rút gọn.
		                \end{enumerate}
	        \end{itemize}
\end{itemize}
\end{phuongphap}

\Noibat[\maunhan][][\faBookmark][]{Ví dụ mẫu}
%%%%%==========VD_01==========%%%%%
\begin{vd}
	Tính pH của dung dịch $HCl$ 0,001M và dung dịch $NaOH$ 0,01M (ở $25^\circ C$).
	\loigiai{
		\begin{itemize}
			    \item Dung dịch $HCl$ $0{,}001$:
				\[\begin{array}{rccccl}
					\text{Phương trình điện li:\, } \text{HCl} & \rightarrow &\text{H}^+& + & \text{Cl}^- &\\
					 0{,}001 &\rightarrow & 0{,}001 &  &&\text{(M)} \\
				\end{array}\]
				$\Rightarrow$ $[H^+] = 0{,}001 $ (M) $\Rightarrow$	$pH = -log([H^+]) =-log(0{,}001)=3$
			    \item Dung dịch $NaOH$ 0,01M:
			    \[\begin{array}{rccccl}
			    	\text{Phương trình điện li:\, } \text{NaOH} & \rightarrow &\text{Na}^+& + & \text{OH}^- &\\
			    	0{,}01 &\rightarrow & 0{,}01 & \rightarrow  &0{,}01& \text{(M)}\\
			    \end{array}\]
			    $\Rightarrow$ $[{OH}^-] = 0{,}01 $ (M) $\Rightarrow$	$pOH = -log([{OH}^-]) =-log(0{,}01)=2$ $\Rightarrow$ $pH=14-2=12$
			\end{itemize}
		}
\end{vd}

%%%%%==========VD_02==========%%%%%
\begin{vd}
	Viết phương trình phân tử, phương trình ion đầy đủ và phương trình ion rút gọn cho phản ứng giữa dung dịch $BaCl_2$ và dung dịch $Na_2SO_4$.
	\loigiai{
		\begin{itemize}
			    \item Phương trình phân tử:
			    \begin{eqnarray*}
			    	BaCl_2(aq) + Na_2SO_4(aq) \xrightarrow BaSO_4(s) + 2NaCl(aq)
			    \end{eqnarray*}
			    (Điều kiện: $BaSO_4$ là chất kết tủa)
			    \item Phương trình ion đầy đủ (các chất điện li mạnh tan được viết dưới dạng ion):
				\begin{eqnarray*}
					Ba^{2+}(aq) + 2Cl^-(aq) + 2Na^+(aq) + SO_4^{2-}(aq) \rightarrow BaSO_4(s) + 2Na^+(aq) + 2Cl^-(aq)
				\end{eqnarray*}
			    \item Phương trình ion rút gọn (lược bỏ các ion không tham gia phản ứng là $Na^+$ và $Cl^-$):
			    \begin{eqnarray*}
			      Ba^{2+}(aq) + SO_4^{2-}(aq) \rightarrow BaSO_4(s)
			    \end{eqnarray*}
			\end{itemize}
		}
\end{vd}

%%%%%==========VD_03==========%%%%%
\begin{vd}
    Dung dịch muối $K_2CO_3$ có môi trường gì (acid, base, hay trung tính)? Giải thích.
    \loigiai{
	    $K_2CO_3$ là muối được tạo bởi base mạnh ($KOH$) và acid yếu ($H_2CO_3$).
	    Khi tan trong nước, $K_2CO_3$ điện li hoàn toàn: $K_2CO_3 \rightarrow 2K^+ + CO_3^{2-}$.
	    Ion $K^+$ là cation của base mạnh, không bị thủy phân.
	    Ion $CO_3^{2-}$ là anion của acid yếu, bị thủy phân trong nước theo phương trình:
	    $CO_3^{2-} + H_2O \rightleftharpoons HCO_3^- + OH^-$
	    Sự thủy phân của ion $CO_3^{2-}$ tạo ra ion $OH^-$, làm cho nồng độ $OH^-$ trong dung dịch tăng lên, do đó dung dịch $K_2CO_3$ có môi trường base ($pH > 7$).
	    }
\end{vd}


%%%%%=====================Bài tập tự luyện Dạng 4==========================%%%
\Noibat[\maunhan][][\faBook][]{Bài tập tự luyện}

\phan{Bài tập tự luận}
%%%=============SOẠN BT===============%%%
\Opensolutionfile{ansbth}[Ans/LGBT-C01B01_Dang4]
\Opensolutionfile{ansbt}[Ans/AnsBT-C01B01_Dang4]
%%%%%============BT_01================%%%%%%
\begin{bt}
	Tính pH của các dung dịch sau ở $25^\circ C$:
	\begin{enumerate}
			\item Dung dịch $H_2SO_4$ 0,005M (coi $H_2SO_4$ điện li hoàn toàn cả hai nấc).
			\item Dung dịch $KOH$ 0,002M.
			\item Dung dịch $CH_3COOH$ 0,1M, biết $K_a = 1,75 \cdot 10^{-5}$.
		\end{enumerate}
	\loigiai{
		\begin{enumerate}
				\item $H_2SO_4 \rightarrow 2H^+ + SO_4^{2-}$
				$[H^+] = 2 \times 0,005 M = 0,01 M = 10^{-2} M$.
				$pH = -\log(10^{-2}) = 2$.
				\item $KOH \rightarrow K^+ + OH^-$
				$[OH^-] = 0,002 M = 2 \cdot 10^{-3} M$.
				$pOH = -\log(2 \cdot 10^{-3}) \approx 2,7$.
				$pH = 14 - pOH = 14 - 2,7 = 11,3$.
				\item $CH_3COOH \rightleftharpoons CH_3COO^- + H^+$
				$[H^+] \approx \sqrt{K_a \cdot C_0} = \sqrt{1,75 \cdot 10^{-5} \cdot 0,1} = \sqrt{1,75 \cdot 10^{-6}} \approx 1,32 \cdot 10^{-3} M$.
				$pH = -\log(1,32 \cdot 10^{-3}) \approx 2,88$.
			\end{enumerate}
		}
\end{bt}
%%%%%============BT_02================%%%%%%
\begin{bt}
	Viết phương trình phân tử, ion đầy đủ và ion rút gọn cho các phản ứng (nếu có) xảy ra khi trộn các cặp dung dịch sau:
	\begin{enumerate}
			\item $FeCl_3$ và $NaOH$.
			\item $Na_2CO_3$ và $HCl$.
			\item $CuSO_4$ và $Ba(NO_3)_2$.
	        \item $KNO_3$ và $NaCl$.
		\end{enumerate}
	\loigiai{
		\begin{enumerate}
				\item $FeCl_3$ và $NaOH$:
				PT phân tử: $FeCl_3(aq) + 3NaOH(aq) \rightarrow Fe(OH)_3(s) + 3NaCl(aq)$
				PT ion đầy đủ: $Fe^{3+}(aq) + 3Cl^-(aq) + 3Na^+(aq) + 3OH^-(aq) \rightarrow Fe(OH)_3(s) + 3Na^+(aq) + 3Cl^-(aq)$
				PT ion rút gọn: $Fe^{3+}(aq) + 3OH^-(aq) \rightarrow Fe(OH)_3(s)$
				\item $Na_2CO_3$ và $HCl$:
				PT phân tử: $Na_2CO_3(aq) + 2HCl(aq) \rightarrow 2NaCl(aq) + H_2O(l) + CO_2(g)$
				PT ion đầy đủ: $2Na^+(aq) + CO_3^{2-}(aq) + 2H^+(aq) + 2Cl^-(aq) \rightarrow 2Na^+(aq) + 2Cl^-(aq) + H_2O(l) + CO_2(g)$
				PT ion rút gọn: $CO_3^{2-}(aq) + 2H^+(aq) \rightarrow H_2O(l) + CO_2(g)$
				\item $CuSO_4$ và $Ba(NO_3)_2$:
				PT phân tử: $CuSO_4(aq) + Ba(NO_3)_2(aq) \rightarrow BaSO_4(s) + Cu(NO_3)_2(aq)$
				PT ion đầy đủ: $Cu^{2+}(aq) + SO_4^{2-}(aq) + Ba^{2+}(aq) + 2NO_3^-(aq) \rightarrow BaSO_4(s) + Cu^{2+}(aq) + 2NO_3^-(aq)$
				PT ion rút gọn: $Ba^{2+}(aq) + SO_4^{2-}(aq) \rightarrow BaSO_4(s)$
		        \item $KNO_3$ và $NaCl$:
		        Không có phản ứng xảy ra vì không tạo kết tủa, khí hay chất điện li yếu. Các ion vẫn tồn tại độc lập trong dung dịch.
			\end{enumerate}
		}
\end{bt}
%%%%%============BT_03================%%%%%%
\begin{bt}
    Dự đoán môi trường (acid, base, hay trung tính) của các dung dịch muối sau ở $25^\circ C$. Giải thích ngắn gọn.
    \begin{enumerate}
	        \item $NH_4Cl$
	        \item $CH_3COONa$
	        \item $KNO_3$
	        \item $Na_2S$
	    \end{enumerate}
	\loigiai{
	    \begin{enumerate}
		        \item $NH_4Cl$: Muối tạo bởi acid mạnh ($HCl$) và base yếu ($NH_3$). Ion $NH_4^+$ bị thủy phân: $NH_4^+ + H_2O \rightleftharpoons NH_3 + H_3O^+$. Môi trường acid ($pH < 7$).
		        \item $CH_3COONa$: Muối tạo bởi base mạnh ($NaOH$) và acid yếu ($CH_3COOH$). Ion $CH_3COO^-$ bị thủy phân: $CH_3COO^- + H_2O \rightleftharpoons CH_3COOH + OH^-$. Môi trường base ($pH > 7$).
		        \item $KNO_3$: Muối tạo bởi acid mạnh ($HNO_3$) và base mạnh ($KOH$). Cả $K^+$ và $NO_3^-$ đều không bị thủy phân đáng kể. Môi trường trung tính ($pH = 7$).
		        \item $Na_2S$: Muối tạo bởi base mạnh ($NaOH$) và acid yếu ($H_2S$). Ion $S^{2-}$ bị thủy phân (chủ yếu nấc 1): $S^{2-} + H_2O \rightleftharpoons HS^- + OH^-$. Môi trường base ($pH > 7$).
		    \end{enumerate}
		}
\end{bt}
%%%%%============BT_04================%%%%%%
\begin{bt}
    Một học sinh thực hiện thí nghiệm cho từ từ dung dịch $NaOH$ vào dung dịch $AlCl_3$. Hãy mô tả hiện tượng quan sát được và viết các phương trình ion rút gọn của các phản ứng xảy ra.
	\loigiai{
	    Hiện tượng:
	    \begin{itemize}
		        \item Ban đầu, khi cho từ từ dung dịch $NaOH$ vào dung dịch $AlCl_3$, xuất hiện kết tủa keo trắng $Al(OH)_3$.
		        $Al^{3+}(aq) + 3OH^-(aq) \rightarrow Al(OH)_3(s)$
		        \item Nếu tiếp tục cho dư dung dịch $NaOH$ vào, kết tủa keo trắng $Al(OH)_3$ sẽ tan dần tạo dung dịch trong suốt.
		        $Al(OH)_3(s) + OH^-(aq) \rightarrow [Al(OH)_4]^-(aq)$ (hoặc $AlO_2^-(aq) + 2H_2O(l)$)
		    \end{itemize}
	    Phương trình ion rút gọn:
	    \begin{enumerate}
		        \item Tạo kết tủa: $Al^{3+} + 3OH^- \rightarrow Al(OH)_3\downarrow$
		        \item Hòa tan kết tủa (khi $OH^-$ dư): $Al(OH)_3 + OH^- \rightarrow [Al(OH)_4]^-$
		    \end{enumerate}
		}
\end{bt}
%%%%%============BT_05================%%%%%%
\begin{bt}
    Trong nông nghiệp, người ta thường dùng vôi ($CaO$ hoặc $Ca(OH)_2$) để cải tạo đất chua (đất có pH thấp). Giải thích cơ sở khoa học của việc làm này bằng các phương trình hóa học (hoặc ion rút gọn). Tại sao không nên bón vôi cùng lúc với phân đạm ammonium ($NH_4NO_3, (NH_4)_2SO_4$)?
	\loigiai{
	    Cơ sở khoa học của việc dùng vôi cải tạo đất chua:
	    Đất chua chứa nhiều ion $H^+$. Vôi khi bón vào đất sẽ phản ứng với nước (nếu là $CaO$) hoặc trực tiếp (nếu là $Ca(OH)_2$) để tạo ra $Ca(OH)_2$, là một base.
	    $CaO(s) + H_2O(l) \rightarrow Ca(OH)_2(s, aq)$
	    $Ca(OH)_2$ tan một phần trong nước tạo ion $OH^-$:
	    $Ca(OH)_2(s) \rightleftharpoons Ca^{2+}(aq) + 2OH^-(aq)$
	    Ion $OH^-$ sẽ trung hòa ion $H^+$ trong đất, làm giảm độ chua của đất (tăng pH):
	    $H^+(aq) + OH^-(aq) \rightarrow H_2O(l)$
	
	    Không nên bón vôi cùng lúc với phân đạm ammonium vì:
	    Khi có mặt $OH^-$ từ vôi, ion $NH_4^+$ trong phân đạm sẽ phản ứng:
	    $NH_4^+(aq) + OH^-(aq) \rightarrow NH_3(g) + H_2O(l)$
	    Phản ứng này tạo ra khí $NH_3$ bay hơi, làm mất đạm, giảm hiệu quả của phân bón.
		}
\end{bt}
\Closesolutionfile{ansbt}
\Closesolutionfile{ansbth}
%\bangdapanSA{AnsBT-C01B01_Dang4}

\phan{Bài tập trả lời ngắn}
%%%=============SOẠN BT===============%%%
\Opensolutionfile{ansbth}[Ans/LGSA-C01B01_Dang4]
\Opensolutionfile{ansbt}[Ans/AnsSA-C01B01_Dang4]
%%%%%============SA_01================%%%%%%
\begin{bt}
	Dung dịch có $[H^+] = 10^{-5} M$ thì có pH bằng bao nhiêu?
	\shortans{5}
	\loigiai{$pH = -\log(10^{-5}) = 5$.}
\end{bt}
%%%%%============SA_02================%%%%%%
\begin{bt}
	Nếu pH của một dung dịch là 9, môi trường của dung dịch đó là gì? (Acid, Base, Trung tính)
	\shortans{Base}
	\loigiai{$pH = 9 > 7$, môi trường base.}
\end{bt}
%%%%%============SA_03================%%%%%%
\begin{bt}
    Trong phản ứng $Zn(s) + 2HCl(aq) \rightarrow ZnCl_2(aq) + H_2(g)$, ion nào được giữ nguyên (không tham gia phản ứng) trong phương trình ion rút gọn? (Ghi công thức ion)
	\shortans{Cl-}
	\loigiai{PT ion rút gọn: $Zn(s) + 2H^+(aq) \rightarrow Zn^{2+}(aq) + H_2(g)$. Ion $Cl^-$ không tham gia.}
\end{bt}
%%%%%============SA_04================%%%%%%
\begin{bt}
    Chất nào là kết tủa khi trộn dung dịch $AgNO_3$ với dung dịch $NaCl$? (Ghi công thức hóa học)
	\shortans{AgCl}
	\loigiai{$Ag^+(aq) + Cl^-(aq) \rightarrow AgCl(s)$.}
\end{bt}
%%%%%============SA_05================%%%%%%
\begin{bt}
    pH của dung dịch $NaOH$ 0,0001M ở $25^\circ C$ là bao nhiêu?
	\shortans{10}
	\loigiai{$[OH^-] = 10^{-4} M \Rightarrow pOH = 4 \Rightarrow pH = 14 - 4 = 10$.}
\end{bt}
%%%%%============SA_06================%%%%%%
\begin{bt}
    Chất khí nào thoát ra khi cho dung dịch $Na_2SO_3$ tác dụng với dung dịch $H_2SO_4$? (Ghi công thức hóa học)
	\shortans{SO2}
	\loigiai{$SO_3^{2-}(aq) + 2H^+(aq) \rightarrow H_2O(l) + SO_2(g)$.}
\end{bt}
%%%%%============SA_07================%%%%%%
\begin{bt}
    Dung dịch $X$ có $pOH = 3$. Môi trường của dung dịch $X$ là gì? (Acid, Base, Trung tính)
	\shortans{Base}
	\loigiai{$pOH = 3 \Rightarrow pH = 14 - 3 = 11 > 7$. Môi trường base.}
\end{bt}
%%%%%============SA_08================%%%%%%
\begin{bt}
    Phương trình ion rút gọn của phản ứng giữa dung dịch $CH_3COOH$ và dung dịch $KOH$ là gì? (Viết phương trình)
	\shortans{CH3COOH+OH->CH3COO-+H2O}
	\loigiai{$CH_3COOH(aq) + OH^-(aq) \rightarrow CH_3COO^-(aq) + H_2O(l)$.}
\end{bt}
%%%%%============SA_09================%%%%%%
\begin{bt}
    Tích số ion của nước ($K_w$) ở $25^\circ C$ có giá trị bằng bao nhiêu? (Dạng $x \cdot 10^{-k}$)
	\shortans{1.0E-14}
	\loigiai{$K_w = [H^+][OH^-] = 1,0 \cdot 10^{-14}$ ở $25^\circ C$.}
\end{bt}
%%%%%============SA_10================%%%%%%
\begin{bt}
    Dung dịch muối $AlCl_3$ có pH lớn hơn, nhỏ hơn hay bằng 7? (Trả lời: Lớn hơn, Nhỏ hơn, Bằng)
	\shortans{Nhỏ hơn}
	\loigiai{Ion $Al^{3+}$ bị thủy phân tạo $H^+$: $Al^{3+} + H_2O \rightleftharpoons [Al(OH)]^{2+} + H^+$. Do đó dung dịch có môi trường acid, $pH < 7$.}
\end{bt}
\Closesolutionfile{ansbt}
\Closesolutionfile{ansbth}
%\bangdapanSA{AnsSA-C01B01_Dang4}


%%%%============Phần trắc nghiệm============%%%
\phan{Trắc nghiệm nhiều lựa chọn}
%%%=============SOẠN EX===============%%%
\Opensolutionfile{ansex}[Ans/LGEX-C01B01_Dang4]
\Opensolutionfile{ans}[Ans/Ans-C01B01_Dang4]
%%%%%============EX_01================%%%%%%
\begin{ex}
	Giá trị pH của dung dịch $HCl$ 0,01M là:
	\choice
	{1}
	{\True 2}
	{12}
	{13}
	\loigiai{$HCl$ là acid mạnh, $[H^+] = 0,01M = 10^{-2}M$. $pH = -\log(10^{-2}) = 2$.}
\end{ex}
%%%%%============EX_02================%%%%%%
\begin{ex}
	Dung dịch có $pH = 11$ có môi trường:
	\choice
	{Acid.}
	{\True Base.}
	{Trung tính.}
	{Lưỡng tính.}
	\loigiai{$pH = 11 > 7$, do đó dung dịch có môi trường base.}
\end{ex}
%%%%%============EX_03================%%%%%%
\begin{ex}
	Phản ứng nào sau đây có phương trình ion rút gọn là $H^+ + OH^- \rightarrow H_2O$?
	\choice
	{$HCl + Na_2CO_3$}
	{$CH_3COOH + NaOH$}
	{\True $H_2SO_4 + Ba(OH)_2$}
	{$NH_4Cl + KOH$}
	\loigiai{Đây là phản ứng trung hòa giữa acid mạnh và base mạnh.
	    A: $2H^+ + CO_3^{2-} \rightarrow H_2O + CO_2$.
	    B: $CH_3COOH + OH^- \rightarrow CH_3COO^- + H_2O$.
	    C: $H_2SO_4$ (acid mạnh) và $Ba(OH)_2$ (base mạnh). Phản ứng $2H^+ + 2OH^- \rightarrow 2H_2O$ (rút gọn $H^+ + OH^- \rightarrow H_2O$) và $Ba^{2+} + SO_4^{2-} \rightarrow BaSO_4(s)$.
	    Tuy nhiên, nếu chỉ xét sự trung hòa thì $H^+ + OH^- \rightarrow H_2O$ là bản chất.
	    D: $NH_4^+ + OH^- \rightarrow NH_3 + H_2O$.
	    Để có $H^+ + OH^- \rightarrow H_2O$ là phương trình ion rút gọn duy nhất (không có kết tủa hay khí khác), cần acid mạnh và base mạnh không tạo kết tủa khác. Ví dụ: $HCl + NaOH \rightarrow NaCl + H_2O$.
	    Trong các lựa chọn:
	    $H_2SO_4 + Ba(OH)_2 \rightarrow BaSO_4(s) + 2H_2O$. Ion rút gọn còn $Ba^{2+} + SO_4^{2-} \rightarrow BaSO_4$.
	    Nếu câu hỏi ý là "phản ứng nào có sự tạo thành $H_2O$ từ $H^+$ và $OH^-$" thì nhiều đáp án đúng.
	    Nếu "phương trình ion rút gọn CHỈ LÀ", thì cần phản ứng giữa acid mạnh đơn chức và base mạnh đơn chức không tạo kết tủa.
	    Giả sử câu hỏi muốn tìm phản ứng mà bản chất là sự kết hợp của $H^+$ và $OH^-$.
	    Với $CH_3COOH + NaOH \rightarrow CH_3COONa + H_2O$, ion rút gọn là $CH_3COOH + OH^- \rightarrow CH_3COO^- + H_2O$.
	    Với $H_2SO_4 + Ba(OH)_2$, bản chất có $H^+ + OH^- \rightarrow H_2O$ nhưng còn có $Ba^{2+} + SO_4^{2-} \rightarrow BaSO_4$.
	    Để phương trình ion rút gọn là $H^+ + OH^- \rightarrow H_2O$, phải là phản ứng giữa acid mạnh và base mạnh, sản phẩm muối tan. Ví dụ: $HCl + NaOH$.
	    Trong các phương án, nếu $H_2SO_4$ tác dụng với một base mạnh tan không tạo kết tủa $SO_4^{2-}$ thì có thể.
	    Tuy nhiên, nếu $H_2SO_4$ (2 $H^+$) và $Ba(OH)_2$ (2 $OH^-$), tỉ lệ 1:1.
	    $2H^+ + SO_4^{2-} + Ba^{2+} + 2OH^- \rightarrow BaSO_4(s) + 2H_2O$.
	    Phương trình ion rút gọn sẽ là cả hai: $Ba^{2+} + SO_4^{2-} \rightarrow BaSO_4(s)$ và $H^+ + OH^- \rightarrow H_2O$.
	    Có lẽ đáp án C được chọn vì có cả $H^+$ từ acid mạnh và $OH^-$ từ base mạnh.
	    Nếu chọn câu có bản chất $H^+ + OH^- \rightarrow H_2O$ là chính thì C có thể được xem xét.
	    Tuy nhiên, để PT ion rút gọn CHỈ LÀ $H^+ + OH^- \rightarrow H_2O$, thì phải là acid mạnh và base mạnh tạo muối tan.
	    Ví dụ $HCl + NaOH$.
	    Xem xét lại:
	    A. $2H^+ + CO_3^{2-} \rightarrow H_2O + CO_2$
	    B. $CH_3COOH + OH^- \rightarrow CH_3COO^- + H_2O$
	    C. $H_2SO_4 + Ba(OH)_2$: $2H^+ + SO_4^{2-} + Ba^{2+} + 2OH^- \rightarrow BaSO_4 \downarrow + 2H_2O$. Rút gọn: $H^+ + OH^- \rightarrow H_2O$ và $Ba^{2+} + SO_4^{2-} \rightarrow BaSO_4 \downarrow$.
	    D. $NH_4^+ + OH^- \rightarrow NH_3 \uparrow + H_2O$.
	    Không có phương án nào có PƯ ion rút gọn CHỈ LÀ $H^+ + OH^- \rightarrow H_2O$.
	    Có lẽ ý hỏi là phản ứng nào có sự trung hòa acid-base mạnh.
	    Nếu vậy, C là acid mạnh ($H_2SO_4$) và base mạnh ($Ba(OH)_2$).
	    Chọn C với giả định là có phản ứng $H^+ + OH^- \rightarrow H_2O$ xảy ra giữa acid mạnh và base mạnh.
	    }
\end{ex}
%%%%%============EX_04================%%%%%%
\begin{ex}
	Điều kiện để xảy ra phản ứng trao đổi ion trong dung dịch là sản phẩm tạo thành phải có:
	\choice
	{Chất rắn hoặc chất lỏng.}
	{Chất khí hoặc chất không màu.}
	{\True Chất kết tủa, chất khí hoặc chất điện li yếu.}
	{Chất điện li mạnh hoặc chất dễ tan.}
	\loigiai{Phản ứng trao đổi ion xảy ra khi có sự tạo thành chất kết tủa, chất khí hoặc chất điện li yếu (như nước, acid yếu, base yếu).}
\end{ex}
%%%%%============EX_05================%%%%%%
\begin{ex}
	Dung dịch muối nào sau đây có pH < 7 (môi trường acid)?
	\choice
	{$NaCl$}
	{$K_2CO_3$}
	{\True $NH_4Cl$}
	{$CH_3COONa$}
	\loigiai{$NH_4Cl$ là muối của base yếu ($NH_3$) và acid mạnh ($HCl$). Ion $NH_4^+$ bị thủy phân tạo môi trường acid: $NH_4^+ + H_2O \rightleftharpoons NH_3 + H_3O^+$.}
\end{ex}
%%%%%============EX_06================%%%%%%
\begin{ex}
	Phương trình ion rút gọn của phản ứng giữa dung dịch $AgNO_3$ và $KCl$ là:
	\choice
	{$K^+ + NO_3^- \rightarrow KNO_3$}
	{$AgNO_3 + Cl^- \rightarrow AgCl + NO_3^-$}
	{\True $Ag^+ + Cl^- \rightarrow AgCl(s)$}
	{$AgNO_3 + KCl \rightarrow AgCl(s) + KNO_3$}
	\loigiai{Phản ứng: $AgNO_3(aq) + KCl(aq) \rightarrow AgCl(s) + KNO_3(aq)$.
	    Ion đầy đủ: $Ag^+(aq) + NO_3^-(aq) + K^+(aq) + Cl^-(aq) \rightarrow AgCl(s) + K^+(aq) + NO_3^-(aq)$.
	    Lược bỏ ion khán giả ($K^+, NO_3^-$), ta có: $Ag^+(aq) + Cl^-(aq) \rightarrow AgCl(s)$.}
\end{ex}
%%%%%============EX_07================%%%%%%
\begin{ex}
	Khi trộn dung dịch $Na_2CO_3$ với dung dịch $CaCl_2$, hiện tượng quan sát được là:
	\choice
	{Sủi bọt khí.}
	{Dung dịch đổi màu.}
	{\True Xuất hiện kết tủa trắng.}
	{Không có hiện tượng gì.}
	\loigiai{Phản ứng: $Na_2CO_3(aq) + CaCl_2(aq) \rightarrow CaCO_3(s) + 2NaCl(aq)$. $CaCO_3$ là chất kết tủa màu trắng.}
\end{ex}
%%%%%============EX_07.1================%%%%%%
\begin{ex}
	Khi trộn dung dịch $Na_2CO_3$ với dung dịch $Fe_Cl_3$, hiện tượng quan sát được là:
	\choice
	{\True Sủi bọt khí và  kết tảu màu nâu đỏ.}
	{Dung dịch đổi màu.}
	{Xuất hiện kết tủa trắng.}
	{Không có hiện tượng gì.}
	\loigiai{Phản ứng: $Na_2CO_3(aq) + CaCl_2(aq) \rightarrow CaCO_3(s) + 2NaCl(aq)$. $CaCO_3$ là chất kết tủa màu trắng.}
\end{ex}
%%%%%============EX_08================%%%%%%
\begin{ex}
	Chất chỉ thị nào sau đây chuyển sang màu hồng trong dung dịch có $pH = 10$?
	\choice
	{Quỳ tím}
	{\True Phenolphthalein}
	{Methyl da cam}
	{Giấy pH vạn năng (chỉ hiện màu, không chữ)}
	\loigiai{Phenolphthalein không màu trong môi trường acid và trung tính ($pH < 8,3$), chuyển sang màu hồng trong môi trường base ($pH \ge 8,3$). $pH = 10$ là môi trường base.}
\end{ex}
%%%%%============EX_09================%%%%%%
\begin{ex}
	Tích số ion của nước ($K_w$) ở $25^\circ C$ bằng:
	\choice
	{$10^{-7}$}
	{$10^{-1}$}
	{\True $10^{-14}$}
	{$10^{14}$}
	\loigiai{Ở $25^\circ C$, tích số ion của nước $K_w = [H^+][OH^-] = 1,0 \cdot 10^{-14}$.}
\end{ex}
%%%%%============EX_10================%%%%%%
\begin{ex}
    Phản ứng nào sau đây không xảy ra trong dung dịch?
	\choice
	{$Ba(OH)_2 + H_2SO_4$}
	{$FeS + HCl$}
	{$NaHCO_3 + NaOH$}
	{\True $KCl + NaNO_3$}
	\loigiai{Khi trộn dung dịch $KCl$ và $NaNO_3$, các ion $K^+, Cl^-, Na^+, NO_3^-$ không kết hợp với nhau để tạo thành chất kết tủa, chất khí hay chất điện li yếu. Do đó, không có phản ứng xảy ra.}
\end{ex}
%%%%%============EX_11================%%%%%%
\begin{ex}
    pH của dung dịch $Ba(OH)_2$ 0,005M là:
	\choice
	{2}
	{12}
	{\True 12} % Sửa lại, Ba(OH)2 -> Ba2+ + 2OH- => [OH-] = 0.01 => pOH=2 => pH=12
	{11}
	\loigiai{$Ba(OH)_2 \rightarrow Ba^{2+} + 2OH^-$. $[OH^-] = 2 \times 0,005M = 0,01M = 10^{-2}M$.
	    $pOH = -\log(10^{-2}) = 2$.
	    $pH = 14 - pOH = 14 - 2 = 12$.}
\end{ex}
%%%%%============EX_12================%%%%%%
\begin{ex}
    Dung dịch $X$ có $[OH^-] = 10^{-4}M$. pH của dung dịch $X$ là:
	\choice
	{4}
	{\True 10}
	{8}
	{6}
	\loigiai{$pOH = -\log[OH^-] = -\log(10^{-4}) = 4$.
	    $pH = 14 - pOH = 14 - 4 = 10$.}
\end{ex}
%%%%%============EX_13================%%%%%%
\begin{ex}
    Phương trình ion rút gọn $CO_3^{2-} + 2H^+ \rightarrow CO_2 \uparrow + H_2O$ tương ứng với phản ứng giữa các cặp chất nào sau đây?
	\choice
	{$CaCO_3 + HCl$}
	{\True $K_2CO_3 + HNO_3$}
	{$MgCO_3 + H_2SO_4$}
	{$BaCO_3 + CH_3COOH$}
	\loigiai{$K_2CO_3$ tan, điện li ra $CO_3^{2-}$. $HNO_3$ là acid mạnh, điện li ra $H^+$. Sản phẩm $CO_2$ là khí, $H_2O$ là chất điện li yếu.
	    A: $CaCO_3$ là chất rắn.
	    C: $MgCO_3$ là chất rắn, $H_2SO_4$ còn tạo $SO_4^{2-}$.
	    D: $BaCO_3$ rắn, $CH_3COOH$ là acid yếu.}
\end{ex}
%%%%%============EX_14================%%%%%%
\begin{ex}
    Trộn dung dịch chứa $a$ mol $NaOH$ với dung dịch chứa $a$ mol $HCl$. Dung dịch thu được có pH là:
	\choice
	{$< 7$}
	{$> 7$}
	{\True $= 7$}
	{Không xác định.}
	\loigiai{$NaOH$ (base mạnh) và $HCl$ (acid mạnh) phản ứng vừa đủ theo tỉ lệ 1:1.
	    $NaOH + HCl \rightarrow NaCl + H_2O$.
	    $OH^- + H^+ \rightarrow H_2O$.
	    Dung dịch thu được chứa $NaCl$ (muối của acid mạnh và base mạnh) nên có môi trường trung tính, $pH = 7$.}
\end{ex}
%%%%%============EX_15================%%%%%%
\begin{ex}
    Để trung hòa 100ml dung dịch $H_2SO_4$ 0,1M cần V ml dung dịch $NaOH$ 0,2M. Giá trị của V là:
	\choice
	{50 ml}
	{\True 100 ml}
	{150 ml}
	{200 ml}
	\loigiai{$H_2SO_4 + 2NaOH \rightarrow Na_2SO_4 + 2H_2O$.
	    $n_{H_2SO_4} = 0,1L \times 0,1M = 0,01$ mol.
	    Theo phương trình, $n_{NaOH} = 2 \times n_{H_2SO_4} = 2 \times 0,01 = 0,02$ mol.
	    $V_{NaOH} = \frac{n_{NaOH}}{C_{M_{NaOH}}} = \frac{0,02 \text{ mol}}{0,2 M} = 0,1 L = 100 ml$.}
\end{ex}
%%%%%============EX_16================%%%%%%
\begin{ex}
    Môi trường của dung dịch $CH_3COOK$ là:
	\choice
	{Acid.}
	{\True Base.}
	{Trung tính.}
	{Lưỡng tính.}
	\loigiai{$CH_3COOK$ là muối của acid yếu ($CH_3COOH$) và base mạnh ($KOH$). Ion $CH_3COO^-$ bị thủy phân: $CH_3COO^- + H_2O \rightleftharpoons CH_3COOH + OH^-$, tạo môi trường base.}
\end{ex}
%%%%%============EX_17================%%%%%%
\begin{ex}
    Phản ứng nào sau đây tạo ra chất khí?
	\choice
	{$BaCl_2 + H_2SO_4$}
	{$Fe(NO_3)_2 + NaOH$}
	{\True $(NH_4)_2SO_4 + Ba(OH)_2$ (đun nóng)}
	{$CuSO_4 + KCl$}
	\loigiai{$(NH_4)_2SO_4 + Ba(OH)_2 \xrightarrow{t^\circ} BaSO_4(s) + 2NH_3(g) + 2H_2O(l)$. Khí $NH_3$ thoát ra.
	    A tạo kết tủa $BaSO_4$. B tạo kết tủa $Fe(OH)_2$. D không phản ứng.}
\end{ex}
%%%%%============EX_18================%%%%%%
\begin{ex}
    Thêm từ từ dung dịch $HCl$ vào dung dịch $Na_2CO_3$. Hiện tượng ban đầu (khi $HCl$ còn ít) là:
	\choice
	{Có khí $CO_2$ thoát ra ngay.}
	{\True Chưa có khí thoát ra, tạo thành $NaHCO_3$.}
	{Có kết tủa trắng.}
	{Dung dịch chuyển sang màu hồng.}
	\loigiai{Khi $HCl$ còn ít, xảy ra phản ứng: $CO_3^{2-} + H^+ \rightarrow HCO_3^-$. Chưa có khí $CO_2$ thoát ra. Khi $HCl$ dư, mới có: $HCO_3^- + H^+ \rightarrow H_2O + CO_2 \uparrow$.}
\end{ex}
%%%%%============EX_19================%%%%%%
\begin{ex}
    Quỳ tím chuyển màu gì khi nhúng vào dung dịch $NH_4Cl$?
	\choice
	{Xanh}
	{\True Đỏ (hoặc hồng)}
	{Không đổi màu}
	{Mất màu}
	\loigiai{Dung dịch $NH_4Cl$ có môi trường acid do ion $NH_4^+$ thủy phân. Quỳ tím hóa đỏ trong môi trường acid.}
\end{ex}
%%%%%============EX_20================%%%%%%
\begin{ex}
    Dung dịch muối nào sau đây có pH = 7?
	\choice
	{$Na_2S$}
	{$FeCl_3$}
	{\True $KBr$}
	{$Al_2(SO_4)_3$}
	\loigiai{$KBr$ là muối của acid mạnh ($HBr$) và base mạnh ($KOH$), nên dung dịch có môi trường trung tính, $pH = 7$.
	    $Na_2S$ (base). $FeCl_3, Al_2(SO_4)_3$ (acid).}
\end{ex}
\Closesolutionfile{ans}
\Closesolutionfile{ansex}
%\bangdapan{Ans-C01B01_Dang4}
%
%%%%%%%%%%%%%%%Trắc nghiệm đúng sai%%%%%%%%%%%%%%%%%%%%%%%%
\phan{Bài tập trắc nghiệm Đúng Sai}
%%%=============SOẠN EXTF===============%%%
\Opensolutionfile{ansex}[Ans/LGTF-C01B01_Dang4]
\Opensolutionfile{ansbook}[Ansbook/AnsTF-C01B01_Dang4]
\Opensolutionfile{ans}[Ans/Tempt-C01B01_Dang4]
%%%%%============TF_01================%%%%%%
\begin{ex}
	Về pH và môi trường dung dịch ở $25^\circ C$:
	\choiceTF
	{\True Dung dịch có $pH = 5$ là môi trường acid.}
	{Dung dịch có $[H^+] = 10^{-9} M$ có $pH = -9$.}
	{\True Nếu $pH > 7$ thì dung dịch có tính base.}
	{Nước tinh khiết có $pH = 0$.}
	\loigiai{
			\begin{itemchoice}[T1,F2,T3,F4]
					\itemch Đúng. $pH = 5 < 7$.
					\itemch Sai. $pH = -\log(10^{-9}) = 9$.
					\itemch Đúng.
					\itemch Sai. Nước tinh khiết có $pH = 7$.
				\end{itemchoice}
		}
\end{ex}
%%%%%============TF_02================%%%%%%
\begin{ex}
	Điều kiện xảy ra phản ứng trao đổi ion trong dung dịch:
	\choiceTF
	{\True Sản phẩm phải có chất kết tủa.}
	{Sản phẩm phải có chất khí.}
	{\True Sản phẩm phải có chất điện li yếu (ví dụ $H_2O$).}
	{Chỉ cần các chất tham gia là chất điện li mạnh.}
	\loigiai{
			\begin{itemchoice}[T1,T2,T3,F4] % Sửa lại 1,2,3 đều là điều kiện, nhưng yêu cầu là "ít nhất một trong các". Vậy F4 là đúng.
					\itemch Đúng. Đây là một trong các điều kiện.
					\itemch Đúng. Đây là một trong các điều kiện.
					\itemch Đúng. Đây là một trong các điều kiện.
					\itemch Sai. Điều kiện là sản phẩm tạo thành phải thỏa mãn: kết tủa, khí, hoặc điện li yếu. Các chất tham gia là chất điện li.
		            Phát biểu T1, T2, T3 đúng theo nghĩa là "Sản phẩm CÓ THỂ LÀ chất kết tủa/khí/điện li yếu".
		            Nếu hiểu theo "điều kiện là sản phẩm BẮT BUỘC PHẢI LÀ" thì không đúng.
		            Yêu cầu là "ít nhất một trong".
		            Vậy T1, T2, T3 là các trường hợp có thể.
		            Phát biểu 4 sai rõ ràng.
		            Tôi sẽ hiểu T1,T2,T3 là các trường hợp dẫn đến phản ứng xảy ra.
				\end{itemchoice}
		}
\end{ex}
%%%%%============TF_03================%%%%%%
\begin{ex}
	Phương trình ion rút gọn:
	\choiceTF
	{\True Cho biết bản chất của phản ứng trong dung dịch chất điện li.}
	{Luôn bao gồm tất cả các ion có mặt trong dung dịch.}
	{\True Các chất kết tủa, khí, điện li yếu được viết dưới dạng phân tử.}
	{Phản ứng $NaOH + HCl \rightarrow NaCl + H_2O$ có phương trình ion rút gọn là $Na^+ + Cl^- \rightarrow NaCl$.}
	\loigiai{
			\begin{itemchoice}[T1,F2,T3,F4]
					\itemch Đúng.
					\itemch Sai. Lược bỏ các ion không tham gia phản ứng.
					\itemch Đúng.
					\itemch Sai. Phương trình ion rút gọn là $H^+ + OH^- \rightarrow H_2O$.
				\end{itemchoice}
		}
\end{ex}
%%%%%============TF_04================%%%%%%
\begin{ex}
	Môi trường của dung dịch muối:
	\choiceTF
	{\True Dung dịch $Na_2CO_3$ có môi trường base.}
	{Dung dịch $NH_4NO_3$ có môi trường trung tính.}
	{\True Dung dịch $KCl$ có môi trường trung tính.}
	{Muối tạo bởi acid yếu và base yếu luôn có môi trường trung tính.}
	\loigiai{
			\begin{itemchoice}[T1,F2,T3,F4]
					\itemch Đúng. $CO_3^{2-}$ thủy phân tạo $OH^-$.
					\itemch Sai. $NH_4^+$ thủy phân tạo $H^+$, môi trường acid.
					\itemch Đúng. Muối của acid mạnh và base mạnh.
					\itemch Sai. Môi trường phụ thuộc vào $K_a$ của cation acid và $K_b$ của anion base. Nếu $K_a > K_b$ (môi trường acid), $K_a < K_b$ (môi trường base), $K_a \approx K_b$ (môi trường gần trung tính).
				\end{itemchoice}
		}
\end{ex}
%%%%%============TF_05================%%%%%%
\begin{ex}
	Tính pH của dung dịch:
	\choiceTF
	{\True Dung dịch $HNO_3$ 0,01M có $pH = 2$.}
	{Dung dịch $Ba(OH)_2$ 0,005M có $pH = 2$.}
	{\True Nếu $pOH = 5$ thì $pH = 9$.}
	{Dung dịch có $[H^+] = [OH^-]$ thì $pH < 7$.}
	\loigiai{
			\begin{itemchoice}[T1,F2,T3,F4]
					\itemch Đúng. $[H^+] = 10^{-2}M \Rightarrow pH = 2$.
					\itemch Sai. $[OH^-] = 2 \times 0,005 = 0,01M \Rightarrow pOH = 2 \Rightarrow pH = 12$.
					\itemch Đúng. $pH = 14 - pOH = 14 - 5 = 9$.
					\itemch Sai. Nếu $[H^+] = [OH^-]$ thì dung dịch trung tính, $pH = 7$.
				\end{itemchoice}
		}
\end{ex}
%%%%%============TF_06================%%%%%%
\begin{ex}
	Phản ứng trong dung dịch:
	\choiceTF
	{\True Phản ứng giữa $CuSO_4$ và $H_2S$ tạo kết tủa $CuS$.}
	{Dung dịch $NaHCO_3$ không phản ứng với dung dịch $NaOH$.}
	{\True $HCl$ có thể phản ứng với $Fe(OH)_2$.}
	{Trộn dung dịch $K_2SO_4$ và $BaCl_2$ không có hiện tượng gì.}
	\loigiai{
			\begin{itemchoice}[T1,F2,T3,F4]
					\itemch Đúng. $CuS$ là kết tủa đen, không tan trong acid loãng.
					\itemch Sai. $NaHCO_3 + NaOH \rightarrow Na_2CO_3 + H_2O$ (nếu $NaHCO_3$ thể hiện tính acid).
					\itemch Đúng. $Fe(OH)_2$ là base, $HCl$ là acid. $Fe(OH)_2 + 2HCl \rightarrow FeCl_2 + 2H_2O$.
					\itemch Sai. Tạo kết tủa trắng $BaSO_4$.
				\end{itemchoice}
		}
\end{ex}
%%%%%============TF_07================%%%%%%
\begin{ex}
	Về chất chỉ thị pH:
	\choiceTF
	{\True Quỳ tím hóa đỏ trong dung dịch có $pH = 3$.}
	{Phenolphthalein có màu hồng trong dung dịch $HCl$.}
	{\True Giấy pH vạn năng cho phép xác định giá trị pH gần đúng của dung dịch.}
	{Màu của chất chỉ thị không phụ thuộc vào pH của dung dịch.}
	\loigiai{
			\begin{itemchoice}[T1,F2,T3,F4]
					\itemch Đúng. $pH=3$ là môi trường acid.
					\itemch Sai. Phenolphthalein không màu trong dung dịch acid.
					\itemch Đúng.
					\itemch Sai. Màu của chất chỉ thị thay đổi theo pH.
				\end{itemchoice}
		}
\end{ex}
%%%%%============TF_08================%%%%%%
\begin{ex}
	Sự thủy phân của muối:
	\choiceTF
	{\True Muối $Na_2CO_3$ làm dung dịch có tính kiềm do ion $CO_3^{2-}$ bị thủy phân.}
	{Muối $NH_4Cl$ làm dung dịch có tính kiềm do ion $Cl^-$ bị thủy phân.}
	{Muối $NaCl$ có tính acid do cả $Na^+$ và $Cl^-$ đều bị thủy phân.}
	{\True Muối tạo từ acid mạnh và base yếu có môi trường acid.}
	\loigiai{
			\begin{itemchoice}[T1,F2,F3,T4]
					\itemch Đúng. $CO_3^{2-} + H_2O \rightleftharpoons HCO_3^- + OH^-$.
					\itemch Sai. Dung dịch $NH_4Cl$ có tính acid do $NH_4^+$ thủy phân. $Cl^-$ không bị thủy phân đáng kể.
					\itemch Sai. $NaCl$ có môi trường trung tính.
					\itemch Đúng. Cation của base yếu sẽ thủy phân tạo $H^+$.
				\end{itemchoice}
		}
\end{ex}
%%%%%============TF_09================%%%%%%
\begin{ex}
	Xét phản ứng: $CH_3COOH + NaHCO_3 \rightarrow CH_3COONa + H_2O + CO_2$.
	\choiceTF
	{\True Đây là phản ứng trao đổi ion.}
	{Sản phẩm có chất kết tủa.}
	{\True $CH_3COOH$ là chất điện li yếu hơn $H_2CO_3$ (nấc 1).} % Sửa: $CH_3COOH$ mạnh hơn $H_2CO_3$ (nấc 1)
	{Phương trình ion rút gọn là $H^+ + HCO_3^- \rightarrow H_2O + CO_2$.}
	\loigiai{
			\begin{itemchoice}[T1,F2,F3,F4] % Sửa F3
					\itemch Đúng. Có sự trao đổi ion tạo chất khí và chất điện li yếu.
					\itemch Sai. Sản phẩm có khí $CO_2$ và nước.
					\itemch Sai. $CH_3COOH$ ($K_a \approx 1,8 \cdot 10^{-5}$) mạnh hơn nấc 1 của $H_2CO_3$ ($K_{a1} \approx 4,3 \cdot 10^{-7}$), nên phản ứng xảy ra.
					\itemch Sai. $CH_3COOH$ là acid yếu, viết dạng phân tử: $CH_3COOH + HCO_3^- \rightarrow CH_3COO^- + H_2O + CO_2$.
				\end{itemchoice}
		}
\end{ex}
%%%%%============TF_10================%%%%%%
\begin{ex}
	Ứng dụng của kiến thức pH:
	\choiceTF
	{\True pH đất ảnh hưởng đến sự sinh trưởng của cây trồng.}
	{pH máu người luôn được duy trì ổn định ở khoảng 7,35 - 7,45.}
	{Trong xử lý nước thải, việc điều chỉnh pH là không cần thiết.}
	{\True Nhiều enzyme trong cơ thể hoạt động tối ưu ở một khoảng pH nhất định.}
	\loigiai{
			\begin{itemchoice}[T1,T2,F3,T4]
					\itemch Đúng. Mỗi loại cây thích hợp với một khoảng pH đất nhất định.
					\itemch Đúng. Hệ đệm trong máu giúp duy trì pH ổn định.
					\itemch Sai. Điều chỉnh pH là một bước quan trọng trong nhiều quy trình xử lý nước thải.
					\itemch Đúng. Sự thay đổi pH có thể làm mất hoạt tính của enzyme.
				\end{itemchoice}
		}
\end{ex}
\Closesolutionfile{ans}
\Closesolutionfile{ansbook}
\Closesolutionfile{ansex}
%%\bangdapanTF{AnsTF-C01B01_Dang4}





































%%%=============OLD==================%%%%%%%

%\begin{dang}{Câu hỏi lý thuyết}
%\end{dang}
%%%%=============Ví dụ mấu dạng 1=================%%%
%\Noibat[][][\faBookmark]{Ví dụ mẫu}
%%%%==============VDM1==============%%%
%\begin{vd}
%	Chất nào sau đây là chất điện ly?
%	\choice
%	{Đường saccharose}
%	{Ethanol}
%	{\True Natri clorua}
%	{Dầu hỏa}
%	\loigiai{Natri clorua (NaCl) là một chất điện ly vì nó phân ly thành các ion $Na^+$ và $Cl^-$ khi hòa tan trong nước. Các chất còn lại như đường saccharose, ethanol và dầu hỏa không phân ly thành ion trong dung dịch nước, do đó không phải là chất điện ly.}
%\end{vd}
%%%%=============EX_2=============%%%
%\begin{vd}
%	Theo thuyết Brønsted-Lowry, acid là chất:
%	\choice
%	{Nhận proton}
%	{\True Cho proton}
%	{Nhận electron}
%	{Cho electron}
%	\loigiai{Theo thuyết Brønsted-Lowry, acid được định nghĩa là chất cho proton $(H^+)$. Base, ngược lại, là chất nhận proton. Định nghĩa này khác với thuyết Lewis về acid-base, trong đó acid được xem là chất nhận cặp electron và base là chất cho cặp electron.}
%\end{vd}
%%%%%=============EX_3=============%%%
%%%%==============HetVDM1==============%%%
%\Noibat[][][\faBank]{Bài tập tự luyện dạng \thedang}
%
%%%%=============SOẠN EX===============%%%
%\Opensolutionfile{ansex}[Ans/LGEX-H11C01B02-BTTL01]
%\Opensolutionfile{ans}[Ans/Ans-H11C01B02-BTTL01]
%%\tatloigiaiex
%%\luuloigiaiex
%%%%=========ex_1=========%%%
%%%%=============EX_1=============%%%
%\begin{ex}
%	Giá trị pH của máu người khỏe mạnh thường nằm trong khoảng:
%	\choice
%	{$4{,}5-6{,}5$}
%	{$6{,}5-7{,}0$}
%	{\True $7{,}35-7{,}45$}
%	{$8{,}0-9{,}0$}
%	\loigiai{Máu của người khỏe mạnh có pH nằm trong khoảng hẹp từ $7{,}35$ đến $7{,}45$. Đây là một khoảng pH hơi kiềm nhẹ, rất quan trọng cho sự hoạt động bình thường của các enzyme và protein trong cơ thể. Sự thay đổi nhỏ trong pH máu có thể dẫn đến các vấn đề sức khỏe nghiêm trọng.}
%\end{ex}
%%%%=============EX_4=============%%%
%\begin{ex}
%	Chất chỉ thị nào sau đây chuyển màu từ không màu sang hồng trong môi trường kiềm?
%	\choice
%	{Quỳ tím}
%	{Methyl da cam}
%	{\True Phenolphthalein}
%	{Bromothymol blue}
%	\loigiai{Phenolphthalein là một chất chỉ thị acid-base phổ biến, không màu trong môi trường acid và trung tính (pH $<8{,}2)$ và chuyển sang màu hồng trong môi trường kiềm (pH $>8{,}2)$. Đặc tính này làm cho phenolphthalein trở thành một chất chỉ thị hữu ích trong các phép chuẩn độ acid-base, đặc biệt khi chuẩn độ acid yếu bằng base mạnh.}
%\end{ex}
%
%%%%=============EX_6=============%%%
%\begin{ex}
%	Trong dung dịch nước, ion $\mathrm{Al}^{3+}$ tồn tại dưới dạng cân bằng:
%	\choice
%	{$\mathrm{Al}^{3+} + 3\mathrm{H}_2\mathrm{O} \rightleftharpoons \mathrm{Al(OH)}_3 + 3\mathrm{H}^+$}
%	{$\mathrm{Al}^{3+} + 2\mathrm{H}_2\mathrm{O} \rightleftharpoons \mathrm{Al(OH)}_2^+ + 2\mathrm{H}^+$}
%	{\True $\mathrm{Al}^{3+} + \mathrm{H}_2\mathrm{O} \rightleftharpoons \mathrm{Al(OH)}^{2+} + \mathrm{H}^+$}
%	{$\mathrm{Al}^{3+} + 4\mathrm{H}_2\mathrm{O} \rightleftharpoons \mathrm{Al(OH)}_4^- + 4\mathrm{H}^+$}
%	\loigiai{Trong dung dịch nước, ion $\mathrm{Al}^{3+}$ tham gia vào phản ứng thủy phân, tạo ra cân bằng với ion $\mathrm{Al(OH)}^{2+}$. Phương trình cân bằng chính xác là $\mathrm{Al}^{3+} + \mathrm{H}_2\mathrm{O} \rightleftharpoons \mathrm{Al(OH)}^{2+} + \mathrm{H}^+$. Đây là bước đầu tiên trong quá trình thủy phân của $\mathrm{Al}^{3+}$, và là cân bằng chủ yếu trong dung dịch nước.}
%\end{ex}
%%%%$=============EX_7=============$%%%
%\begin{ex}
%	Giá trị pH của một dung dịch acid mạnh $0{,}001M$ là:
%	\choice
%	{$1$}
%	{$2$}
%	{\True $3$}
%	{$4$}
%	\loigiai{Đối với acid mạnh, ta giả định rằng nó phân ly hoàn toàn trong nước. Với nồng độ $0{,}001M$, nồng độ ion $H+$ sẽ bằng $0{,}001M$.
%		Áp dụng công thức pH $= -log[H+]$, ta có:
%		pH $=-log(0{,}001)= -log(10^-3)=3$
%		Vì vậy, giá trị pH của dung dịch acid mạnh $0{,}001M$ là $3$.}
%\end{ex}
%%%%$=============EX_8=============$%%%
%\begin{ex}
%	Trong phản ứng: $\mathrm{NH}_3 + \mathrm{H}_2\mathrm{O} \rightleftharpoons \mathrm{NH}_4^+ + \mathrm{OH}^-$, theo thuyết Brønsted-Lowry, $\mathrm{NH}_3$ đóng vai trò là:
%	\choice
%	{Acid}
%	{\True Base}
%	{Vừa acid vừa base}
%	{Không phải acid cũng không phải base}
%	\loigiai{Trong phản ứng này, $\mathrm{NH}_3$ (amoniac) nhận một proton $(H+)$ từ $\mathrm{H}_2\mathrm{O}$ để tạo thành $\mathrm{NH}_4^+$. Theo thuyết Brønsted-Lowry, chất nhận proton được định nghĩa là base. Do đó, trong phản ứng này, $\mathrm{NH}_3$ đóng vai trò là base Brønsted-Lowry.}
%\end{ex}
%%%%$=============EX_9=============$%%%
%\begin{ex}
%	Quỳ tím chuyển sang màu gì khi nhúng vào dung dịch có pH $=3$?
%	\choice
%	{Xanh}
%	{\True Đỏ}
%	{Tím}
%	{Không đổi màu}
%	\loigiai{Quỳ tím là một chất chỉ thị acid-base phổ biến. Nó có màu tím ở pH trung tính (khoảng $7$), chuyển sang màu đỏ trong môi trường acid (pH $<7)$ và màu xanh trong môi trường kiềm (pH $>7)$. Với pH $=3$, dung dịch có tính acid mạnh, do đó quỳ tím sẽ chuyển sang màu đỏ.}
%\end{ex}
%%%%%=============EX_10=============%%%
%\begin{ex}
%	Trong chuẩn độ acid-base, để xác định chính xác điểm tương đương, nên chọn chất chỉ thị có khoảng đổi màu:
%	\choice
%	{Trùng với pH tại điểm tương đương}
%	{Cao hơn nhiều so với pH tại điểm tương đương}
%	{Thấp hơn nhiều so với pH tại điểm tương đương}
%	{\True Gần với pH tại điểm tương đương}
%	\loigiai{Để xác định chính xác điểm tương đương trong chuẩn độ acid-base, nên chọn chất chỉ thị có khoảng đổi màu gần với pH tại điểm tương đương. Điều này đảm bảo rằng sự thay đổi màu sắc của chất chỉ thị xảy ra càng gần với điểm tương đương thực tế càng tốt, giúp tăng độ chính xác của phép chuẩn độ. Nếu khoảng đổi màu quá cao hoặc quá thấp so với pH tại điểm tương đương, có thể dẫn đến sai số đáng kể trong kết quả chuẩn độ.}
%\end{ex}
%%%%$=============EX_11=============$%%%
%\begin{ex}
%	Chất nào sau đây là chất điện ly?
%	\choice
%	{Đường saccharose}
%	{Dầu hỏa}
%	{\True Natri clorua}
%	{Cồn ethylic nguyên chất}
%	\loigiai{Natri clorua (NaCl) là một chất điện ly, khi hòa tan trong nước nó sẽ phân ly thành các ion $Na^+$ và $Cl^-$. Các chất còn lại không phân ly thành ion trong dung dịch nên không phải là chất điện ly.}
%\end{ex}
%%%%$=============EX_12=============$%%%
%\begin{ex}
%	Theo thuyết Brønsted - Lowry, acid là chất:
%	\choice
%	{Nhận proton}
%	{\True Cho proton}
%	{Nhận electron}
%	{Cho electron}
%	\loigiai{Theo thuyết Brønsted - Lowry, acid được định nghĩa là chất có khả năng cho proton $H^+$ trong phản ứng. Base là chất có khả năng nhận proton.}
%\end{ex}
%%%%$=============EX_13=============$%%%
%\begin{ex}
%	pH của một dung dịch trung tính ở $25\circ C$ là:
%	\choice
%	{$0$}
%	{$14$}
%	{\True $7$}
%	{$1$}
%	\loigiai{Ở $25\circ C$, một dung dịch được coi là trung tính khi có pH $=7$. Khi pH $<7$, dung dịch là acid; khi pH $>7$, dung dịch là base.}
%\end{ex}
%%%%$=============EX_14=============$%%%
%\begin{ex}
%	Công thức tính pH của dung dịch acid mạnh là:
%	\choice
%	{pH $= -log[OH_-]$}
%	{pH $=14+ log[H+]$}
%	{\True $pH = -log[H^+]$}
%	{pH $= log[H^+]$}
%	\loigiai{Công thức tính pH của dung dịch acid mạnh là $pH = -log[H+]$, trong đó $[H^+]$ là nồng độ ion hydro trong dung dịch.}
%\end{ex}
%%%%=============EX_15=============%%%
%\begin{ex}
%	Chất chỉ thị nào sau đây chuyển màu trong môi trường base?
%	\choice
%	{Methyl cam}
%	{Methyl đỏ}
%	{\True Phenolphthalein}
%	{Xanh bromothymol}
%	\loigiai{Phenolphthalein là chất chỉ thị chuyển màu trong môi trường base. Nó không màu trong môi trường acid và chuyển sang màu hồng trong môi trường base.}
%\end{ex}
%%%%=============EX_16=============%%%
%\begin{ex}
%	Trong phương pháp chuẩn độ acid - base, điểm tương đương là điểm mà:
%	\choice
%	{pH của dung dịch bằng 7}
%	{Chất chỉ thị chuyển màu}
%	{\True Số mol acid đã phản ứng hết với số mol base}
%	{Thể tích dung dịch chuẩn độ bằng thể tích dung dịch được chuẩn độ}
%	\loigiai{Điểm tương đương trong chuẩn độ acid - base là điểm mà số mol acid đã phản ứng hết với số mol base, tức là tại điểm này, lượng acid và base đã phản ứng với nhau theo tỉ lệ phản ứng hóa học.}
%\end{ex}
%%%%=============EX_17=============%%%
%\begin{ex}
%	Ion Al3+ trong dung dịch nước có tính:
%	\choice
%	{Trung tính}
%	{Base}
%	{\True Acid}
%	{Lưỡng tính}
%	\loigiai{Ion $Al^3+$ trong dung dịch nước có tính acid. Nó có thể tham gia phản ứng thủy phân tạo ra ion H+, làm tăng nồng độ H+ trong dung dịch: $Al^{3+}+ + H_2O \xrightleftharpoons Al(OH)^{2+} + H^+$}
%\end{ex}
%%%%=============EX_18=============%%%
%\begin{ex}
%	Quỳ tím chuyển màu gì trong môi trường acid?
%	\choice
%	{Xanh}
%	{\True Đỏ}
%	{Tím}
%	{Không màu}
%	\loigiai{Quỳ tím chuyển sang màu đỏ trong môi trường acid. Trong môi trường base, nó chuyển sang màu xanh, còn trong môi trường trung tính, nó giữ nguyên màu tím.}
%\end{ex}
%%%%=============EX_19=============%%%
%\begin{ex}
%	Phản ứng nào sau đây thể hiện tính chất của acid theo Brønsted - Lowry?
%	\choice
%	{$NaOH + HCl \rightarrow NaCl + H2O$}
%	{$2Na + 2H_2O \rightarrow 2NaOH + H_2$}
%	{\True $HCl + H_2O \rightarrow H_3O+ + Cl-$}
%	{$CuO + 2HCl \rightarrow CuCl_2 + H_2O$}
%	\loigiai{Phản ứng $HCl + H_2O → H_3O^+ + Cl^-$ thể hiện tính chất của acid theo Brønsted - Lowry. Trong phản ứng này, HCl đóng vai trò là acid (cho proton) và $H_2O$ đóng vai trò là base (nhận proton).}
%\end{ex}
%%%%=============EX_20=============%%%
%\begin{ex}
%	pH của máu người khỏe mạnh thường nằm trong khoảng:
%	\choice
%	{$4{,}5-5{,}5$}
%	{$6{,}0-7{,}0$}
%	{\True $7{,}35-7{,}45$}
%	{$8{,}0-9{,}0$}
%	\loigiai{pH của máu người khỏe mạnh thường nằm trong khoảng $7{,}35-7{,}45$. Đây là một khoảng pH hẹp và rất quan trọng cho sự hoạt động bình thường của các enzyme và quá trình trao đổi chất trong cơ thể.}
%\end{ex}
%%%%=============EX_21=============%%%
%\begin{ex}
%	Chất nào sau đây không phải là chất điện ly?
%	\choice
%	{NaCl}
%	{$H_2SO_4$}
%	{KOH}
%	{\True $C_6H_{12}O_6$ (glucose)}
%	\loigiai{$C_6H_{12}O_6$ (glucose) không phải là chất điện ly. Khi hòa tan trong nước, glucose không phân ly thành ion mà tồn tại dưới dạng phân tử. Các chất còn lại (NaCl, $H_2SO_4$, KOH) đều là chất điện ly, phân ly thành ion khi hòa tan trong nước.}
%\end{ex}
%%%%=============EX_22=============%%%
%\begin{ex}
%	Theo thuyết Brønsted - Lowry, trong phản ứng $NH3 + H2O \xrightleftharpoons{} NH4+ + OH-$, $NH_3$ đóng vai trò là:
%	\choice
%	{Acid}
%	{\True Base}
%	{Chất oxi hóa}
%	{Chất khử}
%	\loigiai{Trong phản ứng $NH3 + H2O \xrightleftharpoons{} NH4+ + OH-$, $NH_3$ đóng vai trò là base theo thuyết Brønsted - Lowry vì nó nhận proton ($H^+$) từ $H_2O$ để tạo thành $NH_4^+$. $H_2O$ trong trường hợp này đóng vai trò là acid, cho proton.}
%\end{ex}
%%%%=============EX_23=============%%%
%\begin{ex}
%	pH của nước cất tinh khiết ở $25^\circ C$ là:
%	\choice
%	{$0$}
%	{$14$}
%	{\True $7$}
%	{$1$}
%	\loigiai{pH của nước cất tinh khiết ở $25^\circ C$ là 7. Ở nhiệt độ này, nồng độ ion $H^+$ và $OH^-$ trong nước tinh khiết đều bằng $10^{-7} mol/L$, do đó $pH = -log[H+] = -log(10^{-7}) = 7$.}
%\end{ex}
%%%%=============EX_24=============%%%
%\begin{ex}
%	Công thức tính pOH của một dung dịch là:
%	\choice
%	{$pOH = -log[H^+]$}
%	{$pOH = 14 + log[OH^-]$}
%	{\True $pOH = -log[OH^-]$}
%	{$pOH = log[OH^-]$}
%	\loigiai{Công thức tính pOH của một dung dịch là $pOH = -log[OH^-]$, trong đó $[OH^-]$ là nồng độ ion hydroxide trong dung dịch. Chú ý rằng $pH + pOH = 14$ ở $25^\circ C$.}
%\end{ex}
%%%%=============EX_25=============%%%
%\begin{ex}
%	Chất chỉ thị nào sau đây chuyển màu trong khoảng pH từ $8{,}2$ đến $10$?
%	\choice
%	{Methyl cam}
%	{Methyl đỏ}
%	{\True Phenolphthalein}
%	{Xanh bromothymol}
%	\loigiai{Phenolphthalein là chất chỉ thị chuyển màu trong khoảng pH từ $8{,}2$ đến $10$. Nó không màu khi pH $<8{,}2$ và chuyển sang màu hồng khi pH $>8{,}2$.}
%\end{ex}
%%%%=============EX_26=============%%%
%\begin{ex}
%	Trong phương pháp chuẩn độ acid - base, điểm kết thúc chuẩn độ là:
%	\choice
%	{Điểm mà số mol acid bằng số mol base}
%	{\True Điểm mà chất chỉ thị chuyển màu}
%	{Điểm mà pH của dung dịch bằng 7}
%	{Điểm mà thể tích dung dịch chuẩn độ bằng thể tích dung dịch được chuẩn độ}
%	\loigiai{Điểm kết thúc chuẩn độ là điểm mà chất chỉ thị chuyển màu. Đây là điểm mà người thực hiện chuẩn độ quan sát được và dừng việc thêm dung dịch chuẩn độ. Điểm này thường gần với điểm tương đương nhưng không nhất thiết trùng khớp.}
%\end{ex}
%%%%=============EX_27=============%%%
%\begin{ex}
%	Ion Fe3+ trong dung dịch nước có tính:
%	\choice
%	{Trung tính}
%	{Base}
%	{\True Acid}
%	{Lưỡng tính}
%	\loigiai{Ion Fe3+ trong dung dịch nước có tính acid. Nó tham gia phản ứng thủy phân tạo ra ion H+, làm tăng nồng độ H+ trong dung dịch: $Fe^{3+} + H_2O \xrightleftharpoons{} Fe(OH)_2+ + H^+$}
%\end{ex}
%%%%=============EX_28=============%%%
%\begin{ex}
%	Giấy quỳ đỏ chuyển màu gì trong môi trường base?
%	\choice
%	{Đỏ}
%	{\True Xanh}
%	{Tím}
%	{Không màu}
%	\loigiai{Giấy quỳ đỏ chuyển sang màu xanh trong môi trường base. Trong môi trường acid, nó giữ nguyên màu đỏ.}
%\end{ex}
%%%%=============EX_29=============%%%
%\begin{ex}
%	Phản ứng nào sau đây thể hiện tính chất của base theo Brønsted - Lowry?
%	\choice
%	{$NaOH + HCl \xrightarrow{} NaCl + H_2O$}
%	{$2Na + 2H_2O \xrightarrow{} 2NaOH + H_2$}
%	{\True $NH_3 + H_2O \xrightleftharpoons{} NH_4^+ + OH^-$}
%	{$CuO + 2HCl \xrightarrow{} CuCl_2 + H_2O$}
%	\loigiai{Phản ứng $NH3 + H_2O \xrightleftharpoons{} NH_4^+ + OH^-$ thể hiện tính chất của base theo Brønsted - Lowry. Trong phản ứng này, NH3 đóng vai trò là base (nhận proton) và H2O đóng vai trò là acid (cho proton).}
%\end{ex}
%%%%=============EX_30=============%%%
%\begin{ex}
%	pH của nước mưa tự nhiên thường nằm trong khoảng:
%	\choice
%	{$1 - 2$}
%	{\True $5{,}6 - 6{,}5$}
%	{$7{,}0 - 8{,}0$}
%	{$9{,}0 - 10{,}0$}
%	\loigiai{pH của nước mưa tự nhiên thường nằm trong khoảng $5{,}6 - 6{,}5$. Nước mưa có tính acid nhẹ do hòa tan $CO_2$ từ không khí tạo thành acid carbonic ($H_2CO_3$).}
%\end{ex}
%%%%%=============EX_31=============%%%
%\begin{ex}
%	Dung dịch nào sau đây có $pH < 7$?
%	\choice
%	{Dung dịch NaOH}
%	{Dung dịch $Na_2CO_3$}
%	{\True Dung dịch HCl}
%	{Dung dịch $NH_3$}
%	\loigiai{Dung dịch HCl có $pH < 7$ vì HCl là một acid mạnh. Khi hòa tan trong nước, nó phân ly hoàn toàn tạo ra ion H+, làm tăng nồng độ $H^+$ trong dung dịch, dẫn đến $pH < 7$.}
%\end{ex}
%%%%=============EX_32=============%%%
%\begin{ex}
%	Trong phương trình $pH + pOH = 14$ (ở $25^\circ C$), 14 là giá trị của:
%	\choice
%	{pH của nước tinh khiết}
%	{pOH của nước tinh khiết}
%	{\True pKw (hằng số phân ly của nước)}
%	{Nồng độ ion $H^+$ trong nước tinh khiết}
%	\loigiai{Trong phương trình $pH + pOH = 14$ (ở $25^\circ C$), 14 là giá trị của pKw - hằng số phân ly của nước. Kw là tích ion của nước ($Kw = [H+][OH-] = 10^-14$ ở $25^\circ C$), và $pKw = -logKw = 14$.}
%\end{ex}
%%%%=============EX_33=============%%%
%\begin{ex}
%	Ion $CO_3^{2-}$ trong dung dịch nước có tính:
%	\choice
%	{Acid}
%	{\True Base}
%	{Trung tính}
%	{Lưỡng tính}
%	\loigiai{Ion $CO_3^{2-}$ trong dung dịch nước có tính base. Nó tham gia phản ứng thủy phân tạo ra ion $OH^-$, làm tăng nồng độ $OH^-$ trong dung dịch: $CO_3^{2-} + H_2O \xrightleftharpoons{} HCO_3^- + OH^-$}
%\end{ex}
%%%%=============EX_34=============%%%
%\begin{ex}
%	Trong phản ứng chuẩn độ giữa NaOH và HCl, chất chỉ thị nào sau đây phù hợp nhất?
%	\choice
%	{Phenolphthalein}
%	{\True Methyl cam}
%	{Xanh bromothymol}
%	{Phenol đỏ}
%	\loigiai{Methyl cam là chất chỉ thị phù hợp nhất cho phản ứng chuẩn độ giữa NaOH và HCl. Nó có khoảng chuyển màu từ pH $3{,}1$ đến $4{,}4$, gần với điểm tương đương của phản ứng giữa acid mạnh và base mạnh (pH khoảng $7$).}
%\end{ex}
%%%%=============EX_35=============%%%
%\begin{ex}
%	Nếu pH của một dung dịch là 4, thì pOH của dung dịch đó là bao nhiêu (ở 25°C)?
%	\choice
%	{4}
%	{7}
%	{\True 10}
%	{14}
%	\loigiai{Ở $25^\circ C$, ta có $pH + pOH = 14$. Nếu $pH = 4$, thì $pOH = 14 - pH = 14 - 4 = 10$.}
%\end{ex}
%%%%=============EX_36=============%%%
%\begin{ex}
%	Trong phản ứng $Al(OH)_3 + 3H+ \xrightleftharpoons{} Al3+ + 3H2O$, Al(OH)3 đóng vai trò là:
%	\choice
%	{Acid}
%	{\True Base}
%	{Chất oxi hóa}
%	{Chất khử}
%	\loigiai{Trong phản ứng $Al(OH)_3 + 3H^+ \xrightleftharpoons{} Al^{3+} + 3H_2O$, $Al(OH)_3$ đóng vai trò là base theo thuyết Brønsted - Lowry vì nó nhận proton $(H^+)$ từ acid.}
%\end{ex}
%%%%=============EX_37=============%%%
%\begin{ex}
%	Nồng độ ion H+ trong một dung dịch là $10^{-}3$ mol/L. pH của dung dịch này là:
%	\choice
%	{-3}
%	{\True 3}
%	{11}
%	{13}
%	\loigiai{$pH = -log[H+] = -log(10^-3) = 3$}
%\end{ex}
%%%%=============EX_38=============%%%
%\begin{ex}
%	Chất nào sau đây là chất lưỡng tính?
%	\choice
%	{NaOH}
%	{HCl}
%	{\True $Al(OH)_3$}
%	{$H_2SO_4$}
%	\loigiai{$Al(OH)_3$ là chất lưỡng tính. Nó có thể đóng vai trò là acid (cho proton) trong phản ứng với base mạnh, và đóng vai trò là base (nhận proton) trong phản ứng với acid mạnh.}
%\end{ex}
%%%%=============EX_39=============%%%
%\begin{ex}
%	Phương pháp chuẩn độ acid - base được sử dụng để xác định:
%	\choice
%	{Khối lượng riêng của acid hoặc base}
%	{Nhiệt độ sôi của acid hoặc base}
%	{\True Nồng độ của acid hoặc base}
%	{Điện tích của ion trong dung dịch acid hoặc base}
%	\loigiai{Phương pháp chuẩn độ acid - base được sử dụng để xác định nồng độ của acid hoặc base. Bằng cách sử dụng một dung dịch chuẩn đã biết nồng độ, ta có thể xác định được nồng độ chính xác của dung dịch acid hoặc base cần phân tích.}
%\end{ex}
%%%%=============EX_40=============%%%
%\begin{ex}
%	Trong phản ứng $NH_4^+ + OH^- \xrightleftharpoons{} NH_3 + H_2O$, $NH_4^+$ đóng vai trò là:
%	\choice
%	{\True Acid}
%	{Base}
%	{Chất oxi hóa}
%	{Chất khử}
%	\loigiai{Trong phản ứng $NH_4^+ + OH^- \xrightleftharpoons{} NH_3 + H_2O$, $NH_4^+$ đóng vai trò là acid theo thuyết Brønsted - Lowry vì nó cho proton ($H^+$) cho $OH^-$ để tạo thành H2O.}
%\end{ex}
%%%%%=============EX_41=============%%%
%\begin{ex}
%	pH của đất ảnh hưởng như thế nào đến sự phát triển của cây trồng?
%	\choice
%	{pH không ảnh hưởng đến sự phát triển của cây trồng}
%	{Cây trồng phát triển tốt nhất ở pH rất thấp ($< 3$)}
%	{Cây trồng phát triển tốt nhất ở pH rất cao ($> 10$)}
%	{\True pH ảnh hưởng đến khả năng hấp thu dinh dưỡng của cây trồng}
%	\loigiai{pH của đất ảnh hưởng đến khả năng hấp thu dinh dưỡng của cây trồng. Hầu hết các cây trồng phát triển tốt nhất ở pH từ $6{,}0$ đến $7{,}5$. Ở pH quá thấp hoặc quá cao, một số chất dinh dưỡng có thể trở nên không tan hoặc không sẵn có cho cây hấp thu.}
%\end{ex}
%%%%=============EX_42=============%%%
%\begin{ex}
%	Nồng độ ion $OH^-$ trong một dung dịch là $10^{-11}$ mol/L. pH của dung dịch này là (ở $25^\circ C$):
%	\choice
%	{3}
%	{11}
%	{\True 13}
%	{1}
%	\loigiai{%
%		Bước 1: Tính pOH
%		$pOH = -log[OH-] = -log(10^-11) = 11$
%		\\
%		Bước 2: Sử dụng mối quan hệ $pH + pOH = 14$
%		$pH = 14 - pOH = 14 - 11 = 3$
%	}
%\end{ex}
%%%%=============EX_43=============%%%
%\begin{ex}
%	Trong phản ứng chuẩn độ giữa $CH_3COOH$ và NaOH, pH tại điểm tương đương là:
%	\choice
%	{Nhỏ hơn 7}
%	{Bằng 7}
%	{\True Lớn hơn 7}
%	{Bằng 0}
%	\loigiai{Trong phản ứng chuẩn độ giữa $CH_3COOH$ (acid yếu) và NaOH (base mạnh), pH tại điểm tương đương lớn hơn 7. Điều này là do muối tạo thành ($CH_3COONa$) có tính base yếu do ion CH3COO- thủy phân trong nước tạo ra OH-.}
%\end{ex}
%%%%=============EX_44=============%%%
%\begin{ex}
%	Chất nào sau đây là acid theo Brønsted $-$ Lowry?
%	\choice
%	{NaOH}
%	{$CH_3COONa$}
%	{\True $H_2O$}
%	{$NH_3$}
%	\loigiai{$H_2$Ocó thể đóng vai trò là acid theo Brønsted $-$ Lowry. Trong một số phản ứng, nước có thể cho proton $(H+)$, ví dụ trong phản ứng: $H_2O+NH_3$ $\xrightleftharpoons{}$ $NH_4^++OH^-$}
%\end{ex}
%%%%=============EX_45=============%%%
%\begin{ex}
%	Nồng độ ion $H^+$ trong máu người khỏe mạnh khoảng:
%	\choice
%	{$10^-1$ mol/L}
%	{$10^-5$ mol/L}
%	{\True $10^-7$ mol/L}
%	{$10^-14$ mol/L}
%	\loigiai{Nồng độ ion $H^+$ trong máu người khỏe mạnh khoảng $10^{-7}$ mol/L, tương ứng với pH khoảng $7{,}4$. Điều này đảm bảo môi trường slightly alkaline cần thiết cho các quá trình sinh hóa trong cơ thể.}
%\end{ex}
%%%%=============EX_46=============%%%
%\begin{ex}
%	Trong phản ứng $HCO_3^- + H_2O \xrightleftharpoons{} H_2CO_3 + OH^-$, $HCO_3^-$ đóng vai trò là:
%	\choice
%	{Acid}
%	{\True Base}
%	{Chất oxi hóa}
%	{Chất khử}
%	\loigiai{Trong phản ứng $HCO_3^- + H_2O \xrightleftharpoons{} H_2CO_3 + OH^-$, $HCO_3^-$ đóng vai trò là base theo thuyết Brønsted - Lowry vì nó nhận proton ($H^+$) từ $H_2O$ để tạo thành $H_2CO_3$.}
%\end{ex}
%%%%=============EX_47=============%%%
%\begin{ex}
%	pH của nước biển thường nằm trong khoảng:
%	\choice
%	{$4-5$}
%	{$6-7$}
%	{\True $7{,}5-8{,}4$}
%	{$9-10$}
%	\loigiai{pH của nước biển thường nằm trong khoảng $7{,}5-8{,}4$. Nước biển có tính base nhẹ do sự hiện diện của các ion carbonate và bicarbonate.}
%\end{ex}
%%%%=============EX_48=============%%%
%\begin{ex}
%	Chất nào sau đây không thể được sử dụng làm chất chỉ thị acid - base?
%	\choice
%	{Phenolphthalein}
%	{Methyl cam}
%	{Quỳ tím}
%	{\True Glucose}
%	\loigiai{Glucose không thể được sử dụng làm chất chỉ thị acid - base vì nó không thay đổi màu sắc theo sự thay đổi của pH. Các chất chỉ thị acid - base phải có khả năng thay đổi màu sắc rõ ràng trong các khoảng pH khác nhau.}
%\end{ex}
%%%%=============EX_49=============%%%
%\begin{ex}
%	Trong quá trình chuẩn độ HCl bằng NaOH, tại điểm tương đương:
%	\choice
%	{Dung dịch có tính acid}
%	{Dung dịch có tính base}
%	{\True Dung dịch có pH $=7$}
%	{Không thể xác định được pH của dung dịch}
%	\loigiai{Trong quá trình chuẩn độ HCl (acid mạnh) bằng NaOH (base mạnh), tại điểm tương đương, dung dịch có pH $=7$. Điều này là do HCl và NaOH phản ứng hoàn toàn với nhau theo tỉ lệ $1:1$, tạo ra muối NaCl (không thủy phân) và nước.}
%\end{ex}
%%%%=============EX_50=============%%%
%\begin{ex}
%	Phát biểu nào sau đây về sự điện ly là đúng?
%	\choice
%	{Tất cả các chất tan trong nước đều điện ly}
%	{Chỉ có các chất tan trong nước mới điện ly}
%	{\True Sự điện ly là quá trình phân ly các chất thành ion khi hòa tan trong dung môi phân cực}
%	{Sự điện ly chỉ xảy ra với các chất có liên kết ion}
%	\loigiai{Sự điện ly là quá trình phân ly các chất thành ion khi hòa tan trong dung môi phân cực. Điều này có thể xảy ra với cả chất có liên kết ion và một số chất có liên kết cộng hóa trị phân cực.}
%\end{ex}
%%%%%=============EX_51=============%%%
%\begin{ex}
%	Trong phản ứng $NH_4^+ + H_2O \xrightleftharpoons{} NH_3 + H_3O^+$, $H_2O$ đóng vai trò là:
%	\choice
%	{Acid}
%	{\True Base}
%	{Chất oxi hóa}
%	{Chất khử}
%	\loigiai{Trong phản ứng $NH_4^+ + H_2O \xrightleftharpoons{} NH_3 + H_3O^+$, $H_2O$ đóng vai trò là base theo thuyết Brønsted - Lowry vì nó nhận proton ($H^+$) từ $NH_4^+$ để tạo thành $H_3O^+$.}
%\end{ex}
%%%%%=============EX_52=============%%%
%\begin{ex}
%	Một dung dịch có pH $=2$. Nồng độ ion $H+$ trong dung dịch này là:
%	\choice
%	{$10^{-2}$ mol/L}
%	{\True $10^{-2}$ mol/L}
%	{$2$ mol/L}
%	{$10^{-12}$ mol/L}
%	\loigiai{
%		$pH =-log[H+]$;
%		$2 = -log[H+]$;
%		$[H^+] =10^{-2}$ mol/L
%	}
%\end{ex}
%%%%%=============EX_53=============%%%
%\begin{ex}
%	Chất nào sau đây là chất điện ly mạnh?
%	\choice
%	{$CH_3COOH$}
%	{$NH_3$}
%	{\True $KOH$}
%	{$H_2CO_3$}
%	\loigiai{$KOH$ là chất điện ly mạnh. Khi hòa tan trong nước, nó phân ly hoàn toàn thành các ion $K^+$ và $OH^-$. Các chất còn lại $(CH_3COOH$, $NH_3$, $H_2CO_3)$ là chất điện ly yếu.}
%\end{ex}
%%%%%=============EX_54=============%%%
%\begin{ex}
%	Trong chuẩn độ acid - base, đường cong chuẩn độ biểu diễn sự phụ thuộc của:
%	\choice
%	{Nhiệt độ vào thể tích dung dịch chuẩn độ}
%	{\True pH vào thể tích dung dịch chuẩn độ}
%	{Áp suất vào thể tích dung dịch chuẩn độ}
%	{Nồng độ vào nhiệt độ dung dịch}
%	\loigiai{Trong chuẩn độ acid - base, đường cong chuẩn độ biểu diễn sự phụ thuộc của pH vào thể tích dung dịch chuẩn độ. Đường cong này cho phép xác định điểm tương đương và lựa chọn chất chỉ thị phù hợp.}
%\end{ex}
%%%%%=============EX_55=============%%%
%\begin{ex}
%	pH của nước mưa acid thường:
%	\choice
%	{Lớn hơn 7}
%	{Bằng 7}
%	{\True Nhỏ hơn $5{,}6$}
%	{Bằng 14}
%	\loigiai{pH của nước mưa acid thường nhỏ hơn $5{,}6$. Nước mưa tự nhiên có pH khoảng $5{,}6$ do hòa tan CO2 từ không khí. Khi pH nhỏ hơn $5{,}6$, nước mưa được coi là acid do ảnh hưởng của các chất ô nhiễm như $SO_2$, $NO_x$.}
%\end{ex}
%%%%=============EX_56=============%%%
%\begin{ex}
%	Chất nào sau đây có thể đóng vai trò vừa là acid vừa là base theo Brønsted - Lowry?
%	\choice
%	{NaOH}
%	{HCl}
%	{\True $H_2O$}
%	{CH4}
%	\loigiai{$H_2O$ có thể đóng vai trò vừa là acid vừa là base theo Brønsted - Lowry. Nó có thể cho proton (ví dụ: $H2O + NH_3 \xrightleftharpoons{} NH_4^+ + OH^-$) hoặc nhận proton (ví dụ: $H2O + HCl \xrightleftharpoons{} H_3O^+ + Cl^-$) tùy thuộc vào chất phản ứng với nó.}
%\end{ex}
%%%%=============EX_57=============%%%
%\begin{ex}
%	Trong phản ứng chuẩn độ giữa $CH_3COOH$ và $NaOH$, chất chỉ thị nào sau đây phù hợp nhất?
%	\choice
%	{Methyl cam}
%	{\True Phenolphthalein}
%	{Methyl đỏ}
%	{Xanh bromothymol}
%	\loigiai{Phenolphthalein là chất chỉ thị phù hợp nhất cho phản ứng chuẩn độ giữa $CH_3COOH$ và NaOH. Nó có khoảng chuyển màu từ pH $8{,}3$ đến $10$, gần với điểm tương đương của phản ứng giữa acid yếu và base mạnh ($pH > 7$).}
%\end{ex}
%%%%=============EX_58=============%%%
%\begin{ex}
%	Ý nghĩa thực tiễn của cân bằng trong dung dịch nước của ion$Fe^{3+}$ là gì?
%	\choice
%	{Tạo ra màu sắc đẹp cho dung dịch}
%	{Làm tăng độ dẫn điện của dung dịch}
%	{\True Ảnh hưởng đến quá trình xử lý nước và ăn mòn kim loại}
%	{Không có ý nghĩa thực tiễn}
%	\loigiai{Cân bằng trong dung dịch nước của ion $Fe^{3+}$ có ý nghĩa thực tiễn quan trọng trong việc ảnh hưởng đến quá trình xử lý nước và ăn mòn kim loại. Ion $Fe^{3+}$ thủy phân tạo ra acid, có thể gây ăn mòn và ảnh hưởng đến chất lượng nước.}
%\end{ex}
%%%%=============EX_59=============%%%
%\begin{ex}
%	Phát biểu nào sau đây về thuyết Brønsted - Lowry là không đúng?
%	\choice
%	{Acid là chất cho proton}
%	{Base là chất nhận proton}
%	{Một chất có thể vừa là acid vừa là base}
%	{\True Chỉ có các chất có H trong công thức mới là acid}
%	\loigiai{Phát biểu "Chỉ có các chất có H trong công thức mới là acid" là không đúng theo thuyết Brønsted - Lowry. Theo thuyết này, acid là chất có khả năng cho proton, không nhất thiết phải có H trong công thức (ví dụ: $NH_4^+$ là acid Brønsted - Lowry).}
%\end{ex}
%
%\Closesolutionfile{ans}
%\Closesolutionfile{ansex}
%%\bangdapan{Ans-H11C01B02-BTTL1}
%
%
%%%================Dạng 3==============%%%
%\begin{dang}{Viết phương trình điện li}
%\end{dang}
%\begin{pp}
%	\begin{itemize}
%		\item Các chất điênli mạnh dùng mũi tên 1 chiều
%		\item Đối với chất điện li yếu dùng mũi tên thuận nghịch
%	\end{itemize}
%\end{pp}
%%%=============Ví dụ mấu dạng 3=================%%%
%\Noibat[][][\faBookmark]{Ví dụ mẫu}
%%%==============VDM1==============%%%
%\begin{vd}
%	Viết phương trình điện li trong nước của các chất sau: $\mathrm{HClO}_4$, $\mathrm{CH_3COONa}$, $\mathrm{Na}_2\mathrm{SO}_4$, $\mathrm{NH}_4\mathrm{Cl}$
%	\loigiai{
%		\begin{itemize}
%			\item $\mathrm{HClO}_4$: \\
%			$\mathrm{HClO}_4 \rightarrow \mathrm{H}^+ + \mathrm{ClO}_4^-$
%			\item $\mathrm{CH_3COONa}$: \\
%			$\mathrm{CH_3COONa} \rightarrow \mathrm{CH_3COO}^- + \mathrm{Na}^+$
%			\item $\mathrm{Na}_2\mathrm{SO}_4$: \\
%			$\mathrm{Na}_2\mathrm{SO}_4 \rightarrow 2\mathrm{Na}^+ + \mathrm{SO}_4^{2-}$
%			\item $\mathrm{NH}_4\mathrm{Cl}$: \\
%			$\mathrm{NH}_4\mathrm{Cl} \rightarrow \mathrm{NH}_4^+ + \mathrm{Cl}^-$
%		\end{itemize}
%	}
%\end{vd}
%
%%%==============HetVDM1==============%%%
%\Noibat[][][\faBank]{Bài tập tự luyện dạng \thedang}
%\phan{Bài tập tự luận}
%%%=============SOẠN BT===============%%%
%\Opensolutionfile{ansbth}[Ans/LGBT-H11C01B01-BTTL03]
%\Opensolutionfile{ansbt}[Ans/AnsBT-H11C01B01-BTTL03]
%%\hienthiloigiaibt
%%%==============Bai_BT1==============%%%
%\begin{bt}
%	Viết phương trình điện li trong nước của các chất sau: $\mathrm{NaHCO}_3, \mathrm{CuCl}_2$, $\left(NH_4\right)_2SO_4, \mathrm{Fe}\left(NO_3\right)_3$
%	\loigiai{
%		\begin{itemize}
%			\item $\mathrm{NaHCO}_3 \rightarrow \mathrm{Na}^+ + \mathrm{HCO}_3^-$ 
%			\item $\mathrm{CuCl}_2 \rightarrow \mathrm{Cu}^{2+} + 2\mathrm{Cl}^-$ 
%			\item $\left(NH_4\right)_2SO_4 \rightarrow 2\mathrm{NH}_4^+ + \mathrm{SO}_4^{2-}$ 
%			\item $\mathrm{Fe}\left(NO_3\right)_3 \rightarrow \mathrm{Fe}^{3+} + 3\mathrm{NO}_3^-$ 
%		\end{itemize}
%	}
%\end{bt}
%%%==============HetBai_BT1==============%%%
%
%%%==============Bai_BT2==============%%%
%\begin{bt}
%	Viết phương trình điện li trong nước của các chất sau: $\mathrm{CH}_3\mathrm{COOH}, \mathrm{Ba(OH)}_2, \mathrm{NH}_4\mathrm{Cl}, \mathrm{H}_2\mathrm{CO}_3$
%	\loigiai{
%		\begin{itemize}
%			\item $\mathrm{CH}_3\mathrm{COOH} \rightleftharpoons \mathrm{CH}_3\mathrm{COO}^- + \mathrm{H}^+$ 
%			\item $\mathrm{Ba(OH)}_2 \rightarrow \mathrm{Ba}^{2+} + 2\mathrm{OH}^-$ 
%			\item $\mathrm{NH}_4\mathrm{Cl} \rightarrow \mathrm{NH}_4^+ + \mathrm{Cl}^-$ 
%			\item $\mathrm{H}_2\mathrm{CO}_3 \rightleftharpoons \mathrm{H}^+ + \mathrm{HCO}_3^-$ 
%		\end{itemize}
%	}
%\end{bt}
%%%==============HetBai_BT2==============%%%
%
%%%==============Bai_BT3==============%%%
%\begin{bt}
%	Viết phương trình điện li trong nước của các chất sau: $\mathrm{KNO}_3, \mathrm{H}_2\mathrm{S}, \mathrm{Mg(ClO}_4)_2, \mathrm{HClO}$
%	\loigiai{
%		\begin{itemize}
%			\item $\mathrm{KNO}_3 \rightarrow \mathrm{K}^+ + \mathrm{NO}_3^-$ 
%			\item $\mathrm{H}_2\mathrm{S} \rightleftharpoons \mathrm{H}^+ + \mathrm{HS}^-$ 
%			\item $\mathrm{Mg(ClO}_4)_2 \rightarrow \mathrm{Mg}^{2+} + 2\mathrm{ClO}_4^-$ 
%			\item $\mathrm{HClO} \rightleftharpoons \mathrm{H}^+ + \mathrm{ClO}^-$ 
%		\end{itemize}
%	}
%\end{bt}
%%%==============HetBai_BT3==============%%%
%
%%%==============Bai_BT4==============%%%
%\begin{bt}
%	Viết phương trình điện li trong nước của các chất sau: $\mathrm{Al}_2(\mathrm{SO}_4)_3, \mathrm{HF}, \mathrm{Na}_3\mathrm{PO}_4, \mathrm{NH}_4\mathrm{NO}_3$
%	\loigiai{
%		\begin{itemize}
%			\item $\mathrm{Al}_2(\mathrm{SO}_4)_3 \rightarrow 2\mathrm{Al}^{3+} + 3\mathrm{SO}_4^{2-}$ 
%			\item $\mathrm{HF} \rightleftharpoons \mathrm{H}^+ + \mathrm{F}^-$ 
%			\item $\mathrm{Na}_3\mathrm{PO}_4 \rightarrow 3\mathrm{Na}^+ + \mathrm{PO}_4^{3-}$ 
%			\item $\mathrm{NH}_4\mathrm{NO}_3 \rightarrow \mathrm{NH}_4^+ + \mathrm{NO}_3^-$ 
%		\end{itemize}
%	}
%\end{bt}
%%%==============HetBai_BT4==============%%%
%
%%%==============Bai_BT5==============%%%
%\begin{bt}
%	Viết phương trình điện li trong nước của các chất sau: $\mathrm{Pb(NO}_3)_2, \mathrm{CH}_3\mathrm{NH}_2, \mathrm{K}_2\mathrm{Cr}_2\mathrm{O}_7, \mathrm{H}_3\mathrm{PO}_4$
%	\loigiai{
%		\begin{itemize}
%			\item $\mathrm{Pb(NO}_3)_2 \rightarrow \mathrm{Pb}^{2+} + 2\mathrm{NO}_3^-$ 
%			\item $\mathrm{CH}_3\mathrm{NH}_2 + \mathrm{H}_2\mathrm{O} \rightleftharpoons \mathrm{CH}_3\mathrm{NH}_3^+ + \mathrm{OH}^-$ 
%			\item $\mathrm{K}_2\mathrm{Cr}_2\mathrm{O}_7 \rightarrow 2\mathrm{K}^+ + \mathrm{Cr}_2\mathrm{O}_7^{2-}$ 
%			\item $\mathrm{H}_3\mathrm{PO}_4 \rightleftharpoons \mathrm{H}^+ + \mathrm{H}_2\mathrm{PO}_4^-$ 
%		\end{itemize}
%	}
%\end{bt}
%%%==============HetBai_BT5==============%%%
%
%%%==============Bai_BT6==============%%%
%\begin{bt}
%	Viết phương trình điện li trong nước của các chất sau: $\mathrm{AgNO}_3, \mathrm{H}_2\mathrm{SO}_3, \mathrm{Ca(OH)}_2, \mathrm{NH}_4\mathrm{HCO}_3$
%	\loigiai{
%		\begin{itemize}
%			\item $\mathrm{AgNO}_3 \rightarrow \mathrm{Ag}^+ + \mathrm{NO}_3^-$ 
%			\item $\mathrm{H}_2\mathrm{SO}_3 \rightleftharpoons \mathrm{H}^+ + \mathrm{HSO}_3^-$ 
%			\item $\mathrm{Ca(OH)}_2 \rightarrow \mathrm{Ca}^{2+} + 2\mathrm{OH}^-$ 
%			\item $\mathrm{NH}_4\mathrm{HCO}_3 \rightarrow \mathrm{NH}_4^+ + \mathrm{HCO}_3^-$ 
%		\end{itemize}
%	}
%\end{bt}
%%%==============HetBai_BT6==============%%%
%
%%%==============Bai_BT7==============%%%
%\begin{bt}
%	Viết phương trình điện li trong nước của các chất sau: $\mathrm{ZnSO}_4, \mathrm{HCOOH}, \mathrm{LiOH}, \mathrm{NaHSO}_4$
%	\loigiai{
%		\begin{itemize}
%			\item $\mathrm{ZnSO}_4 \rightarrow \mathrm{Zn}^{2+} + \mathrm{SO}_4^{2-}$ 
%			\item $\mathrm{HCOOH} \rightleftharpoons \mathrm{HCOO}^- + \mathrm{H}^+$ 
%			\item $\mathrm{LiOH} \rightarrow \mathrm{Li}^+ + \mathrm{OH}^-$ 
%			\item $\mathrm{NaHSO}_4 \rightarrow \mathrm{Na}^+ + \mathrm{HSO}_4^-$ 
%		\end{itemize}
%	}
%\end{bt}
%%%==============HetBai_BT7==============%%%
%
%%%==============Bai_BT8==============%%%
%\begin{bt}
%	Viết phương trình điện li trong nước của các chất sau: $\mathrm{Fe}_2(\mathrm{SO}_4)_3, \mathrm{HNO}_2, \mathrm{KMnO}_4, \mathrm{(NH}_4)_2\mathrm{CO}_3$
%	\loigiai{
%		\begin{itemize}
%			\item $\mathrm{Fe}_2(\mathrm{SO}_4)_3 \rightarrow 2\mathrm{Fe}^{3+} + 3\mathrm{SO}_4^{2-}$ 
%			\item $\mathrm{HNO}_2 \rightleftharpoons \mathrm{H}^+ + \mathrm{NO}_2^-$ 
%			\item $\mathrm{KMnO}_4 \rightarrow \mathrm{K}^+ + \mathrm{MnO}_4^-$ 
%			\item $\mathrm{(NH}_4)_2\mathrm{CO}_3 \rightarrow 2\mathrm{NH}_4^+ + \mathrm{CO}_3^{2-}$ 
%		\end{itemize}
%	}
%\end{bt}
%%%==============HetBai_BT8==============%%%
%
%%%==============Bai_BT9==============%%%
%\begin{bt}
%	Viết phương trình điện li trong nước của các chất sau: $\mathrm{CuSO}_4, \mathrm{H}_2\mathrm{S}, \mathrm{Ba(ClO}_3)_2, \mathrm{CH}_3\mathrm{COONH}_4$
%	\loigiai{
%		\begin{itemize}
%			\item $\mathrm{CuSO}_4 \rightarrow \mathrm{Cu}^{2+} + \mathrm{SO}_4^{2-}$ 
%			\item $\mathrm{H}_2\mathrm{S} \rightleftharpoons \mathrm{H}^+ + \mathrm{HS}^-$ )
%			\item $\mathrm{Ba(ClO}_3)_2 \rightarrow \mathrm{Ba}^{2+} + 2\mathrm{ClO}_3^-$ 
%			\item $\mathrm{CH}_3\mathrm{COONH}_4 \rightarrow \mathrm{CH}_3\mathrm{COO}^- + \mathrm{NH}_4^+$ 
%		\end{itemize}
%	}
%\end{bt}
%%%==============HetBai_BT9==============%%%
%
%%%==============Bai_BT10==============%%%
%\begin{bt}
%	Viết phương trình điện li trong nước của các chất sau: $\mathrm{Na}_2\mathrm{HPO}_4, \mathrm{HClO}_4, \mathrm{AlCl}_3, \mathrm{NH}_4\mathrm{OH}$
%	\loigiai{
%		\begin{itemize}
%			\item $\mathrm{Na}_2\mathrm{HPO}_4 \rightarrow 2\mathrm{Na}^+ + \mathrm{HPO}_4^{2-}$ 
%			\item $\mathrm{HClO}_4 \rightarrow \mathrm{H}^+ + \mathrm{ClO}_4^-$ 
%			\item $\mathrm{AlCl}_3 \rightarrow \mathrm{Al}^{3+} + 3\mathrm{Cl}^-$ 
%			\item $\mathrm{NH}_4\mathrm{OH} \rightleftharpoons \mathrm{NH}_4^+ + \mathrm{OH}^-$ 
%		\end{itemize}
%	}
%\end{bt}
%%%==============HetBai_BT10==============%%%
%\Closesolutionfile{ansbt}
%\Closesolutionfile{ansbth}
%%\bangdapanSA{AnsBT-H11C01B01-BTTL03}
%
%
%\phan{Bài tập trắc nghiệm}
%%%%=============SOẠN EX===============%%%
%\Opensolutionfile{ansex}[Ans/LGEX-C01B02-BTTL03]
%\Opensolutionfile{ans}[Ans/Ans-C01B02-BTTL03]
%%\hienthiloigiaiex
%%\tatloigiaiex
%%\luuloigiaiex
%%%%==============EX_01==================%%%
%\begin{ex}
%	Phương trình điện li nào sau đây là đúng cho một hợp chất điện li yếu trong dung dịch nước?
%	\choice
%	{$Na_2SO_4 \rightarrow 2\text{Na}^+ + \text{SO}_4^{2-}$}
%	{$BaCl_2 \rightarrow \text{Ba}^{2+} + 2\text{Cl}^-$}
%	{$H_2SO_4 \rightarrow 2\text{H}^+ + \text{SO}_4^{2-}$}
%	{\True $CH_3COOH \rightleftharpoons \text{CH}_3\text{COO}^- + \text{H}^+$}
%	\loigiai{Phương trình điện li đúng cho một hợp chất điện li yếu là:
%		$CH_3COOH \rightleftharpoons \text{CH}_3\text{COO}^- + \text{H}^+$
%		\\
%		$CH_3COOH$ là acid yếu, chỉ phân li một phần trong dung dịch nước. Các chất còn lại là điện li mạnh, phân li hoàn toàn.}
%\end{ex}
%%%%==============EX_02==================%%%
%\begin{ex}
%	Phương trình điện li nào sau đây là \textbf{đúng} ?
%	\choice
%	{\True $CH_3COONa \rightarrow \text{CH}_3\text{COO}^- + \text{Na}^+$}
%	{$HClO \rightarrow \text{H}^+ + \text{ClO}^-$}
%	{$H_2CO_3 \rightarrow 2\text{H}^+ + \text{CO}_3^{2-}$}
%	{$H_2S \rightarrow 2\text{H}^+ + \text{S}^-$}
%	\loigiai{
%		$CH_3COOH$ chất điện li mạnh nên dùng mũi tên một chiều, còn $HClO$,$H_2CO_3$, $H_2S$là chất điện li yếu nên dùng mũi tên hai chiều	}
%\end{ex}
%%%%==============EX_03==================%%%
%\begin{ex}
%	Phương trình điện li nào sau đây là \textbf{không đúng} ?
%	\choice
%	{$HCl \xrightarrow{}  H^+ + Cl^{-}$}
%	{$Al_2{(SO_4)}_3 \xrightarrow  2Al^{3+} + 3SO_4^{2-}$}
%	{$NaOH \xrightarrow{}  Na^+ + OH^{-}$}
%	{\True $H_2SO_4 \xrightarrow  H^+ + HSO_4^{-}$}
%	\loigiai{
%		$H_2SO_4$ là acid mạnh nên phân li hoàn toàn thành ion $H^+$ và $SO_4^{2-}$
%		\[\mathrm{H}_2\mathrm{SO}_4 \xrightarrow  2\mathrm{H}^+ + \mathrm{SO}_4^{-}\]
%	}
%\end{ex}
%\Closesolutionfile{ans}
%\Closesolutionfile{ansex}
%
%%%%============Dạng 4================%%%
%\begin{dang}{Tính pH của dung dịch}
%\end{dang}
%\Noibat[][][\faCoffee]{Bài toán 1 pH của dung dịch acid/base mạnh}
%\begin{pp}
%	Đối với bài toán pha trộn thì mới làm thêm bước 2, bước 3
%	\begin{cacbuoc}
%		\item Tính số mol $H^+$ hoặc $OH^-$ trong mỗi dung dịch ban đầu
%		\item Tính tổng số mol $H^+$ hoặc $OH^-$ sau khi trộn hoặc pha loãng
%		\item Tính nồng độ mới của các ion: $C_M=\dfrac{n}{V_s}$ ($V_s$: Thể tích dung dịch ssau khi trộn, hoặc sau pha loãng).
%		\item Tính $pH=-lg[H^+]$ hoặc $pH=14-pOH$ với $pOH =-lg[OH^-]$
%	\end{cacbuoc}
%\end{pp}
%{\indam[\maunhan]{Lưu ý:} Khi pha loãng, thể tích tăng bao nhiêu lần thì nồng độ giảm bấy nhiêu lần.}
%\begin{center}
%	\boxct{$\dfrac{C_1}{C_2}=\dfrac{V_2}{V_1}$}
%\end{center}
%%%=============Ví dụ mấu dạng 4=================%%%
%\Noibat[][][\faBookmark]{Ví dụ mẫu}
%%%==============VDM1==============%%%
%\begin{vd}
%	Tính pH của các dung dịch sau:
%	\begin{enumEX}{2}
%		\item Dung dịch $\mathrm{NaOH}$ $0,001\;M$;
%		\item Dung dịch $\mathrm{HCl}$ $0,01\;M$;
%		\item Dung dịch $\mathrm{Mg{(OH)}_2}$ $0,002\;M$.
%	\end{enumEX}
%	\loigiai{
%		\begin{enumerate}
%			\item Dung dịch $\mathrm{NaOH}$ $0{,}001\;M$:
%			\[\begin{array}{ccccc}
%				NaOH& \xrightarrow& Na^+& +& OH^-\\
%				0{,}001&&&\rightarrow&0{,}001
%			\end{array}\]
%			\begin{align*}
%				[\mathrm{OH^-}] &= 0,001 \mathrm{M} \\
%				\mathrm{pOH} &= -\log[\mathrm{OH^-}] = -\log(0,001) = 3 \\
%				\mathrm{pH} &= 14 - \mathrm{pOH} = 14 - 3 = 11
%			\end{align*}
%			\item Dung dịch $\mathrm{HCl}$ $0,01\;M$:
%			\[\begin{array}{ccccc}
%				HCl& \xrightarrow& H^+&+& Cl^-\\
%				0{,}01&\rightarrow &0{,}01&&
%			\end{array}\]
%			\begin{align*}
%				[\mathrm{H^+}] &= 0,01 \mathrm{M} \\
%				\mathrm{pH} &= -\log[\mathrm{H^+}] = -\log(0,01) = 2
%			\end{align*}
%			\item Dung dịch $\mathrm{Mg{(OH)}_2}$ $0{,}002\;M$:
%			\[\begin{array}{ccccc}
%				Mg(OH)_2& \xrightarrow& Mg^{2+}&+& 2OH^-\\
%				0{,}002&&&\rightarrow &0{,}004
%			\end{array}\]
%			\begin{align*}
%				[\mathrm{OH^-}] &= 2 \times 0,002 = 0,004 \mathrm{M} \\
%				\mathrm{pOH} &= -\log[\mathrm{OH^-}] = -\log(0,004) = 2,4 \\
%				\mathrm{pH} &= 14 - \mathrm{pOH} = 14 - 2,4 = 11,6
%			\end{align*}
%		\end{enumerate}
%	}
%\end{vd}
%%%==============HetVDM1==============%%%
%%%==============VDM2==============%%%
%\begin{vdex}
%	Dung dịch X là hỗn hợp $Ba{(OH)}_2$ $0{.}1$ M và $NaOH$ $0{.}1$ M. Dung dịch Y là hỗn hợp của $H_2SO_4$ $0{,}0375$ M; $HCl$ $0{,}0125$ M. Trộn $100$ ml dung dịch X với $400$ ml dung dịch Y thu được dung dịch Z. pH của dung dịch Z là
%	\choice
%	{$1$}
%	{$7$}
%	{$2$}
%	{$6$}
%	\loigiai{
%		\indam{Phân tích:} Bài toán trộn dung dịch, lưu ý phải tính lại nồng độ các chất vì thể tích dung dịch  thay đổi. Xác định chất dư để tính pH theo chất đó.
%		\\[5pt]
%		$\left.\begin{aligned}
%			Ba(OH)_2:[OH^-]=0{,}2 \;M\\
%			NaOH :[OH^-]=0{,}1\; M
%		\end{aligned}\right\}$ $\Rightarrow$ $\sum[OH^-]=0{,}3$ M $\Rightarrow$ $nOH^-=0{,}3\cdot0{,}1=0{,}03$ mol.
%		\\
%		$\left.\begin{aligned}
%			H_2SO_4 :[H^+]=0{,}075\;M\\
%			HCl:[H^+]=0{,}0125\;M
%		\end{aligned}\right\}$ $\Rightarrow$ $\sum[H^+]=0{,}0875$ M $\Rightarrow$ $nH^+=0{,}0875\cdot0{,}4=0{,}035$ mol.
%		\[\begin{matrix}
%			& H^+&+& OH^- & \xrightarrow & H_2O\\
%			&0{,}03\;\text{mol}&\xleftarrow&0{,}03\;\text{mol}&&
%		\end{matrix}\]
%		$\Rightarrow$ $n_{H^+\text{dư}}=\dfrac{0{,}005}{0{,}5}=0{,}01\;M$ $\Rightarrow pH =2$.
%	}
%\end{vdex}
%%%==============HetVDM2==============%%%
%\Noibat[][][\faCoffee]{Bài toán 2 Tính pH của dung dịch acid/base yếu}
%\begin{pp}
%	\begin{cacbuoc}
%		\item Viết phương trình điện li
%		\begin{itemize}[wide=0.65cm]
%			\item  Đối với acid yếu:
%			\[\mathrm{HA} + \mathrm{H}_2\mathrm{O} \xrightleftharpoons{} \mathrm{A}^- + \mathrm{H}_3\mathrm{O}^+ \quad K_a=\dfrac{[A^-][H_3O^+]}{[HA]}\]
%			\item Đối với bazo yếu :
%			\[\mathrm{B} + \mathrm{H}_2\mathrm{O} \xrightleftharpoons{} \mathrm{BH^+}^- + \mathrm{OH}^- \quad K_b=\dfrac{[BH^+][OH^-]}{[B]}\]
%		\end{itemize}
%		\item tính nồng độ $H^+$ hoặc $[OH^-]$ thông qua hằng số phân li $K_a$ hoặc $K_b$ theo phương pháp "3 dòng"\\
%		\begin{tabular}{cp{1cm}c}
%			$\begin{matrix}
%				&\mathrm{HA}& +& \mathrm{H}_2\mathrm{O}& \xrightleftharpoons{}& \mathrm{A}^-& +& \mathrm{H}_3\mathrm{O}^+&\\
%				\text{ban đầu:}	&a&&&&&&&\\
%				\text{phản ứng:}&-x&&&&+x&&+x&\\
%				\text{cân bằng:}&a-x&&&&x&&x&
%			\end{matrix}$
%			&&
%			$\begin{matrix}
%				&\mathrm{B}& +& \mathrm{H}_2\mathrm{O}& \xrightleftharpoons{}& \mathrm{BH}^+& +& \mathrm{OH}^-&\\
%				\text{ban đầu:}	&b&&&&&&&\\
%				\text{phản ứng:}&-x&&&&+x&&+x&\\
%				\text{cân bằng:}&a-x&&&&x&&x&
%			\end{matrix}$\\
%			$K_a=\dfrac{x \cdot x}{a-x}$ \quad (1)
%			&&
%			$K_b=\dfrac{x \cdot x}{b-x}$ \quad (2)
%		\end{tabular}
%		\item Từ (1) và (2) tính được $[H^+]$ hoặc $[OH^-]$ $\Rightarrow pH$  
%	\end{cacbuoc}
%\end{pp}
%\Noibat[][][\faBookmark]{Ví dụ mẫu}
%\begin{vd}Tính pH của các dung dịch sau:
%	\begin{enumerate}
%		\item $CH_3COOH$ $0{,}1$M có $K_a=1{,}75\cdot10^{-5}$.
%		\item $NH_3$ $0{,}10$M có $K_b=1{,}80\cdot10^{-5}$.
%	\end{enumerate}
%	\loigiai{
%		\begin{enumerate}
%			\item \phantom{x} 
%			
%			$\begin{matrix}
%				&CH_3COOH& +& \mathrm{H}_2\mathrm{O}& \xrightleftharpoons{}& CH_3COO^-&+& H_3O^+&\\
%				\text{ban đầu:}	&0{,}1&&&&&&&\\
%				\text{phản ứng:}&-x&&&&+x&&+x&\\
%				\text{cân bằng:}&0{,}1-x&&&&x&&x&
%			\end{matrix}$\\
%			Ta có $K_a=\dfrac{x \cdot x}{0{,}1-x} = 1{,}75\cdot10^{-5} \Rightarrow [H^+] = x = 1{,}31\cdot10^{-3} $ $\Rightarrow pH =-log(1{,}31\cdot10^{-3}) = 2{,}88$
%			\item \phantom{x} 
%			
%			$\begin{matrix}
%				&\mathrm{NH_3}& +& \mathrm{H}_2\mathrm{O}& \xrightleftharpoons{}& \mathrm{NH_4}^+& +& \mathrm{OH}^-&\\
%				\text{ban đầu:}	&0{,}1&&&&&&&\\
%				\text{phản ứng:}&-x&&&&+x&&+x&\\
%				\text{cân bằng:}&0{,}1-x&&&&x&&x&
%			\end{matrix}$\\
%			$K_b=\dfrac{x \cdot x}{0{,}1-x} = 1{,}80\cdot10^{-5} $ 
%			$\Rightarrow [OH^-] = x = 1{,}33\cdot10^{-3} $ $\Rightarrow [H^+]=\dfrac{10^{-14}}{1{,}33\cdot10^{-3}} = 7{,}5\cdot10^{-12}\\ \Rightarrow pH =-log(7{,}5\cdot10^{-12}) = 11{,}12$
%		\end{enumerate}
%	}
%\end{vd}
%\Noibat[][][\faBank]{Bài tập tự luyện dạng \thedang}
%\phan{Bài tập tự luận}
%%%=============SOẠN BT===============%%%
%\Opensolutionfile{ansbth}[Ans/LGBT-H11C01B02-BTTL4]
%\Opensolutionfile{ansbt}[Ans/AnsBT-H11C01B02-BTTL4]
%%\hienthiloigiaibt
%%%==============BT_2==============%%%
%\begin{bt}
%	Tính pH của các dung dịch sau:
%	\begin{enumEX}{2}
%		\item Dung dịch $\mathrm{Ca{(OH)}_2}$ $0,02\;M$;
%		\item Dung dịch $\mathrm{HNO_3}$ $0,05\;M$;
%		\item Dung dịch $\mathrm{LiOH}$ $0,1\;M$;
%		\item Dung dịch $\mathrm{H_3PO_4}$ $0,01\;M$;
%		\item Dung dịch $\mathrm{Sr{(OH)}_2}$ $0,005\;M$.
%	\end{enumEX}
%	\loigiai{
%		\begin{enumerate}
%			\item Dung dịch $\mathrm{Ca{(OH)}_2}$ $0,02\;M$:
%			\[\begin{array}{ccccc}
%				Ca(OH)_2& \xrightarrow& Ca^{2+}&+& 2OH^-\\
%				0{,}02&&&\rightarrow &0{,}04
%			\end{array}\]
%			\begin{align*}
%				[\mathrm{OH^-}] &= 2 \times 0,02 = 0,04 \mathrm{M} \\
%				\mathrm{pOH} &= -\log[\mathrm{OH^-}] = -\log(0,04) = 1,4 \\
%				\mathrm{pH} &= 14 - \mathrm{pOH} = 14 - 1,4 = 12,6
%			\end{align*}
%			\item Dung dịch $\mathrm{HNO_3}$ $0,05\;M$:
%			\[\begin{array}{ccccc}
%				HNO_3& \xrightarrow& H^+&+& NO_3^-\\
%				0{,}05&\rightarrow &0{,}05&&
%			\end{array}\]
%			\begin{align*}
%				[\mathrm{H^+}] &= 0,05 \mathrm{M} \\
%				\mathrm{pH} &= -\log[\mathrm{H^+}] = -\log(0,05) = 1,3
%			\end{align*}
%			\item Dung dịch $\mathrm{LiOH}$ $0,1\;M$:
%			\[\begin{array}{ccccc}
%				LiOH& \xrightarrow& Li^+& +& OH^-\\
%				0{,}1&&&\rightarrow&0{,}1
%			\end{array}\]
%			\begin{align*}
%				[\mathrm{OH^-}] &= 0,1 \mathrm{M} \\
%				\mathrm{pOH} &= -\log[\mathrm{OH^-}] = -\log(0,1) = 1 \\
%				\mathrm{pH} &= 14 - \mathrm{pOH} = 14 - 1 = 13
%			\end{align*}
%			\item Dung dịch $\mathrm{H_3PO_4}$ $0,01\;M$:
%			\[\begin{array}{ccccc}
%				H_3PO_4& \xrightarrow& H^+&+& H_2PO_4^-\\
%				0{,}01&\rightarrow&0{,}01& &
%			\end{array}\]
%			\begin{align*}
%				[\mathrm{H^+}] &\approx 0,01 \mathrm{M} \text{ (giả sử phân ly hoàn toàn)} \\
%				\mathrm{pH} &= -\log[\mathrm{H^+}] = -\log(0,01) = 2
%			\end{align*}
%			\item Dung dịch $\mathrm{Sr{(OH)}_2}$ $0,005\;M$:
%			\[\begin{array}{ccccc}
%				Sr(OH)_2& \xrightarrow& Sr^{2+}&+& 2OH^-\\
%				0{,}005&&&\rightarrow &0{,}01
%			\end{array}\]
%			\begin{align*}
%				[\mathrm{OH^-}] &= 2 \times 0,005 = 0,01 \mathrm{M} \\
%				\mathrm{pOH} &= -\log[\mathrm{OH^-}] = -\log(0,01) = 2 \\
%				\mathrm{pH} &= 14 - \mathrm{pOH} = 14 - 2 = 12
%			\end{align*}
%		\end{enumerate}
%	}
%\end{bt}
%
%%%==============BT_3==============%%%
%\begin{bt}
%	Tính pH của các dung dịch sau:
%	\begin{enumEX}{2}
%		\item Dung dịch $\mathrm{KOH}$ $0,005\;M$;
%		\item Dung dịch $\mathrm{HCl}$ $0,2\;M$;
%		\item Dung dịch $\mathrm{Ba{(OH)}_2}$ $0,01\;M$;
%		\item Dung dịch $\mathrm{HClO_4}$ $0,001\;M$.
%	\end{enumEX}
%	\loigiai{
%		\begin{enumerate}
%			\item Dung dịch $\mathrm{KOH}$ $0,005\;M$:
%			\[\begin{array}{ccccc}
%				KOH& \xrightarrow& K^+& +& OH^-\\
%				0{,}005&&&\rightarrow&0{,}005
%			\end{array}\]
%			\begin{align*}
%				[\mathrm{OH^-}] &= 0,005 \mathrm{M} \\
%				\mathrm{pOH} &= -\log[\mathrm{OH^-}] = -\log(0,005) = 2,3 \\
%				\mathrm{pH} &= 14 - \mathrm{pOH} = 14 - 2,3 = 11,7
%			\end{align*}
%			\item Dung dịch $\mathrm{HCl}$ $0,2\;M$:
%			\[\begin{array}{ccccc}
%				HCl& \xrightarrow& H^+&+& Cl^-\\
%				0{,}2&\rightarrow &0{,}2&&
%			\end{array}\]
%			\begin{align*}
%				[\mathrm{H^+}] &= 0,2 \mathrm{M} \\
%				\mathrm{pH} &= -\log[\mathrm{H^+}] = -\log(0,2) = 0,7
%			\end{align*}
%			\item Dung dịch $\mathrm{Ba{(OH)}_2}$ $0,01\;M$:
%			\[\begin{array}{ccccc}
%				Ba(OH)_2& \xrightarrow& Ba^{2+}&+& 2OH^-\\
%				0{,}01&&&\rightarrow &0{,}02
%			\end{array}\]
%			\begin{align*}
%				[\mathrm{OH^-}] &= 2 \times 0,01 = 0,02 \mathrm{M} \\
%				\mathrm{pOH} &= -\log[\mathrm{OH^-}] = -\log(0,02) = 1,7 \\
%				\mathrm{pH} &= 14 - \mathrm{pOH} = 14 - 1,7 = 12,3
%			\end{align*}
%			\item Dung dịch $\mathrm{HClO_4}$ $0,001\;M$:
%			\[\begin{array}{ccccc}
%				HClO_4& \xrightarrow& H^+&+& ClO_4^-\\
%				0{,}001&\rightarrow &0{,}001&&
%			\end{array}\]
%			\begin{align*}
%				[\mathrm{H^+}] &= 0,001 \mathrm{M} \\
%				\mathrm{pH} &= -\log[\mathrm{H^+}] = -\log(0,001) = 3
%			\end{align*}
%		\end{enumerate}
%	}
%\end{bt}
%
%%%==============BT_4==============%%%
%\begin{bt}
%	Tính pH của các dung dịch sau:
%	\begin{enumEX}{2}
%		\item Dung dịch $\mathrm{NaOH}$ $0,02\;M$;
%		\item Dung dịch $\mathrm{HNO_3}$ $0,005\;M$;
%		\item Dung dịch $\mathrm{Ca{(OH)}_2}$ $0,008\;M$.
%	\end{enumEX}
%	\loigiai{
%		\begin{enumerate}
%			\item Dung dịch $\mathrm{NaOH}$ $0,02\;M$:
%			\[\begin{array}{ccccc}
%				NaOH& \xrightarrow& Na^+& +& OH^-\\
%				0{,}02&&&\rightarrow&0{,}02
%			\end{array}\]
%			\begin{align*}
%				[\mathrm{OH^-}] &= 0,02 \mathrm{M} \\
%				\mathrm{pOH} &= -\log[\mathrm{OH^-}] = -\log(0,02) = 1,7 \\
%				\mathrm{pH} &= 14 - \mathrm{pOH} = 14 - 1,7 = 12,3
%			\end{align*}
%			\item Dung dịch $\mathrm{HNO_3}$ $0,005\;M$:
%			\[\begin{array}{ccccc}
%				HNO_3& \xrightarrow& H^+&+& NO_3^-\\
%				0{,}005&\rightarrow &0{,}005&&
%			\end{array}\]
%			\begin{align*}
%				[\mathrm{H^+}] &= 0,005 \mathrm{M} \\
%				\mathrm{pH} &= -\log[\mathrm{H^+}] = -\log(0,005) = 2,3
%			\end{align*}
%			\item Dung dịch $\mathrm{Ca{(OH)}_2}$ $0,008\;M$:
%			\[\begin{array}{ccccc}
%				Ca(OH)_2& \xrightarrow& Ca^{2+}&+& 2OH^-\\
%				0{,}008&&&\rightarrow &0{,}016
%			\end{array}\]
%			\begin{align*}
%				[\mathrm{OH^-}] &= 2 \times 0,008 = 0,016 \mathrm{M} \\
%				\mathrm{pOH} &= -\log[\mathrm{OH^-}] = -\log(0,016) = 1,8 \\
%				\mathrm{pH} &= 14 - \mathrm{pOH} = 14 - 1,8 = 12,2
%			\end{align*}
%		\end{enumerate}
%	}
%\end{bt}
%%%==============BT_5==============%%%
%\begin{bt}
%	Tính pH của các dung dịch sau:
%	\begin{enumEX}{2}
%		\item Dung dịch $\mathrm{LiOH}$ $0,05\;M$;
%		\item Dung dịch $\mathrm{HBr}$ $0,02\;M$;
%		\item Dung dịch $\mathrm{Al{(OH)}_3}$ $0,003\;M$;
%		\item Dung dịch $\mathrm{HClO_3}$ $0,008\;M$.
%	\end{enumEX}
%	\loigiai{
%		\begin{enumerate}
%			\item Dung dịch $\mathrm{LiOH}$ $0,05\;M$:
%			\[\begin{array}{ccccc}
%				LiOH& \xrightarrow& Li^+& +& OH^-\\
%				0{,}05&&&\rightarrow&0{,}05
%			\end{array}\]
%			\begin{align*}
%				[\mathrm{OH^-}] &= 0,05 \mathrm{M} \\
%				\mathrm{pOH} &= -\log[\mathrm{OH^-}] = -\log(0,05) = 1,3 \\
%				\mathrm{pH} &= 14 - \mathrm{pOH} = 14 - 1,3 = 12,7
%			\end{align*}
%			\item Dung dịch $\mathrm{HBr}$ $0,02\;M$:
%			\[\begin{array}{ccccc}
%				HBr& \xrightarrow& H^+&+& Br^-\\
%				0{,}02&\rightarrow &0{,}02&&
%			\end{array}\]
%			\begin{align*}
%				[\mathrm{H^+}] &= 0,02 \mathrm{M} \\
%				\mathrm{pH} &= -\log[\mathrm{H^+}] = -\log(0,02) = 1,7
%			\end{align*}
%			\item Dung dịch $\mathrm{Al{(OH)}_3}$ $0,003\;M$:
%			\[\begin{array}{ccccc}
%				Al(OH)_3& \xrightarrow& Al^{3+}&+& 3OH^-\\
%				0{,}003&&&\rightarrow &0{,}009
%			\end{array}\]
%			\begin{align*}
%				[\mathrm{OH^-}] &= 3 \times 0,003 = 0,009 \mathrm{M} \\
%				\mathrm{pOH} &= -\log[\mathrm{OH^-}] = -\log(0,009) = 2,05 \\
%				\mathrm{pH} &= 14 - \mathrm{pOH} = 14 - 2,05 = 11,95
%			\end{align*}
%			\item Dung dịch $\mathrm{HClO_3}$ $0,008\;M$:
%			\[\begin{array}{ccccc}
%				HClO_3& \xrightarrow& H^+&+& ClO_3^-\\
%				0{,}008&\rightarrow &0{,}008&&
%			\end{array}\]
%			\begin{align*}
%				[\mathrm{H^+}] &= 0,008 \mathrm{M} \\
%				\mathrm{pH} &= -\log[\mathrm{H^+}] = -\log(0,008) = 2,1
%			\end{align*}
%		\end{enumerate}
%	}
%\end{bt}
%%%%==============Bai_BT1==============%%%
%\begin{bt}Tính pH của dung dịch $HClO$ (axit hypochlorous) $0,05$M biết $K_a = 3,0 \cdot 10^{-8}$.
%	\loigiai{
%		$\begin{matrix}
%			&HClO& +& \mathrm{H}_2\mathrm{O}& \xrightleftharpoons{}& ClO^-&+& H_3O^+&\\
%			\text{ban đầu:}	&0{,}05&&&&&&&\\
%			\text{phản ứng:}&-x&&&&+x&&+x&\\
%			\text{cân bằng:}&0{,}05-x&&&&x&&x&
%		\end{matrix}$
%		
%		Ta có: $K_a=\dfrac{x \cdot x}{0{,}05-x} = 3{,}0\cdot10^{-8}$
%		
%		Giả sử $x \ll 0{,}05$, ta có:
%		
%		$x^2 = 3{,}0\cdot10^{-8} \cdot 0{,}05 = 1{,}5\cdot10^{-9}$
%		
%		$x = \sqrt{1{,}5\cdot10^{-9}} = 1{,}22\cdot10^{-5}$
%		
%		Kiểm tra giả thiết: $\dfrac{1{,}22\cdot10^{-5}}{0{,}05} = 2{,}44\cdot10^{-4} \ll 1$ (giả thiết đúng)
%		
%		Vậy $[H^+] = 1{,}22\cdot10^{-5}$
%		
%		$pH = -\log[H^+] = -\log(1{,}22\cdot10^{-5}) = 4{,}91$
%	}
%\end{bt}
%%%%==============HetBai_BT1==============%%%
%
%%%%==============Bai_BT2==============%%%
%\begin{bt}Tính pH của dung dịch $CH_3NH_2$ (methylamine) $0,2$M biết $K_b = 4,38 \cdot 10^{-4}$.
%	\loigiai{
%		$\begin{matrix}
%			&CH_3NH_2& +& \mathrm{H}_2\mathrm{O}& \xrightleftharpoons{}& CH_3NH_3^+&+& OH^-&\\
%			\text{ban đầu:}	&0{,}2&&&&&&&\\
%			\text{phản ứng:}&-x&&&&+x&&+x&\\
%			\text{cân bằng:}&0{,}2-x&&&&x&&x&
%		\end{matrix}$
%		
%		Ta có: $K_b=\dfrac{x \cdot x}{0{,}2-x} = 4{,}38\cdot10^{-4}$
%		
%		Giải phương trình: $x^2 + 4{,}38\cdot10^{-4}x - 8{,}76\cdot10^{-5} = 0$
%		
%		$x = \dfrac{-4{,}38\cdot10^{-4} + \sqrt{(4{,}38\cdot10^{-4})^2 + 4\cdot8{,}76\cdot10^{-5}}}{2} = 8{,}85\cdot10^{-3}$
%		
%		Vậy $[OH^-] = 8{,}85\cdot10^{-3}$
%		
%		$pOH = -\log[OH^-] = -\log(8{,}85\cdot10^{-3}) = 2{,}05$
%		
%		$pH = 14 - pOH = 14 - 2{,}05 = 11{,}95$
%	}
%\end{bt}
%%%%==============HetBai_BT2==============%%%
%
%%%%==============Bai_BT3==============%%%
%\begin{bt}Tính pH của dung dịch $HCOOH$ (axit formic) $0,1$M biết $K_a = 1,8 \cdot 10^{-4}$.
%	\loigiai{
%		$\begin{matrix}
%			&HCOOH& +& \mathrm{H}_2\mathrm{O}& \xrightleftharpoons{}& HCOO^-&+& H_3O^+&\\
%			\text{ban đầu:}	&0{,}1&&&&&&&\\
%			\text{phản ứng:}&-x&&&&+x&&+x&\\
%			\text{cân bằng:}&0{,}1-x&&&&x&&x&
%		\end{matrix}$
%		
%		Ta có: $K_a=\dfrac{x \cdot x}{0{,}1-x} = 1{,}8\cdot10^{-4}$
%		
%		Giải phương trình: $x^2 + 1{,}8\cdot10^{-4}x - 1{,}8\cdot10^{-5} = 0$
%		
%		$x = \dfrac{-1{,}8\cdot10^{-4} + \sqrt{(1{,}8\cdot10^{-4})^2 + 4\cdot1{,}8\cdot10^{-5}}}{2} = 4{,}02\cdot10^{-3}$
%		
%		Vậy $[H^+] = 4{,}02\cdot10^{-3}$
%		
%		$pH = -\log[H^+] = -\log(4{,}02\cdot10^{-3}) = 2{,}40$
%	}
%\end{bt}
%%%%==============HetBai_BT3==============%%%
%
%%%%==============Bai_BT4==============%%%
%\begin{bt}Tính pH của dung dịch $C_6H_5COOH$ (axit benzoic) $0,02$M biết $K_a = 6,3 \cdot 10^{-5}$.
%	\loigiai{
%		$\begin{matrix}
%			&C_6H_5COOH& +& \mathrm{H}_2\mathrm{O}& \xrightleftharpoons{}& C_6H_5COO^-&+& H_3O^+&\\
%			\text{ban đầu:}	&0{,}02&&&&&&&\\
%			\text{phản ứng:}&-x&&&&+x&&+x&\\
%			\text{cân bằng:}&0{,}02-x&&&&x&&x&
%		\end{matrix}$
%		
%		Ta có: $K_a=\dfrac{x\cdot x}{0{,}02-x} = 6{,}3\cdot 10^{-5}$
%		
%		Giả sử $x << 0{,}02$, ta có:
%		
%		$x^2 = 6{,}3\cdot 10^{-5} \cdot 0{,}02 = 1{,}26\cdot 10^{-6}$
%		
%		$x = \sqrt{1{,}26 \cdot 10^{-6}} = 1{,}12\cdot 10^{-3}$
%		\\
%		Kiểm tra giả thiết: $\dfrac{1{,}12\cdot 10^{-3}}{0{,}02} = 0{,}056 < 0{,}05$ (giả thiết đúng)
%		\\
%		Vậy $[H^+] = 1{,}12\cdot 10^{-3}$ $\Rightarrow$
%		$pH = -\log[H^+] = -\log(1{,}12\cdot 10^{-3}) = 2{,}95$
%	}
%\end{bt}
%%%%==============HetBai_BT4==============%%%
%
%%%%==============Bai_BT6==============%%%
%\begin{bt}Tính pH của dung dịch $C_5H_5N$ (pyridine) $0,15$M biết $K_b = 1,7 \cdot 10^{-9}$.
%	\loigiai{
%		$\begin{matrix}
%			&C_5H_5N& +& \mathrm{H}_2\mathrm{O}& \xrightleftharpoons{}& C_5H_5NH^+&+& OH^-&\\
%			\text{ban đầu:}	&0{,}15&&&&&&&\\
%			\text{phản ứng:}&-x&&&&+x&&+x&\\
%			\text{cân bằng:}&0{,}15-x&&&&x&&x&
%		\end{matrix}$
%		
%		Ta có: $K_b=\dfrac{x \cdot x}{0{,}15-x} = 1{,}7\cdot10^{-9}$
%		
%		Giả sử $x \ll 0{,}15$, ta có:
%		
%		$x^2 = 1{,}7\cdot10^{-9} \cdot 0{,}15 = 2{,}55\cdot10^{-10}$
%		
%		$x = \sqrt{2{,}55\cdot10^{-10}} = 1{,}60\cdot10^{-5}$
%		
%		Kiểm tra giả thiết: $\dfrac{1{,}60\cdot10^{-5}}{0{,}15} = 1{,}07\cdot10^{-4} \ll 1$ (giả thiết đúng)
%		
%		Vậy $[OH^-] = 1{,}60\cdot10^{-5}$
%		
%		$pOH = -\log[OH^-] = -\log(1{,}60\cdot10^{-5}) = 4{,}80$
%		$\Rightarrow$
%		$pH = 14 - pOH = 14 - 4{,}80 = 9{,}20$
%	}
%\end{bt}
%%%%==============HetBai_BT6==============%%%
%%%%=====================BT_07==================%%%
%\begin{bt}
%	Hệ đệm bicarbonate là một trong những hệ đệm quan trọng nhất trong cơ thể người, đóng vai trò thiết yếu trong việc duy trì pH máu ổn định. Hệ đệm này hoạt động bằng cách ngăn chặn những thay đổi đột ngột trong nồng độ ion hydrogen ($H^+$), giúp bảo vệ các protein và enzyme khỏi biến tính do pH thay đổi.
%	\\
%	Hệ đệm bicarbonate trong máu được mô tả bởi phương trình:
%	\[\mathrm{CO}_2+\mathrm{H}_2 \mathrm{O} \rightleftarrows \mathrm{H}_2 \mathrm{CO}_3 \rightleftarrows \mathrm{HCO}_3^{-}+\mathrm{H}^{+}\]
%	Cho biết:
%	\begin{itemize}
%		\item pKa của $\mathrm{H}_2 \mathrm{CO}_3$ là $6{,}1$.
%		\item Nồng độ $\left[\mathrm{HCO}_3^{-}\right]$ trong máu là 24 mM.
%		\item Nồng độ $\left[\mathrm{H}_2 \mathrm{CO}_3\right]$ trong máu là $1{,}2$ mM.
%	\end{itemize}
%	\begin{enumerate}
%		\item Tính pH của máu.
%		\item Tại sao pH của máu của người bình thường lại giữ ở mức bình thường 
%	\end{enumerate}
%	\loigiai{
%		\begin{enumerate}
%			\item Tính pH của máu:
%			Sử dụng phương trình Henderson-Hasselbalch:
%			\[ \text{pH} = \text{pKa} + \log\frac{[\text{bazơ}]}{[\text{axit}]} \]
%			\\
%			Thay số:
%			\begin{align*}
%				\text{pH} &= 6{,}1 + \log\frac{[\mathrm{HCO}_3^-]}{[\mathrm{H}_2\mathrm{CO}_3]} \\
%				&= 6{,}1 + \log(\frac{24}{1{,}2}) \\
%				&= 6{,}1 + \log(20) \\
%				&= 6{,}1 + 1{,}3 \\
%				&= 7{,}4
%			\end{align*}
%			\\
%			Vậy pH của máu là 7,4.
%			\item pH của máu của người bình thường được giữ ở mức bình thường (khoảng 7,35 - 7,45) nhờ các cơ chế sau:
%			\begin{itemize}
%				\item Hệ đệm hóa học:
%				\begin{itemize}
%					\item Hệ đệm bicarbonate ($\mathrm{HCO}_3^- / \mathrm{H}_2\mathrm{CO}_3$) là hệ đệm chính trong máu.
%					\item Khi có axit được thêm vào máu: $\mathrm{H}^+ + \mathrm{HCO}_3^- \rightarrow \mathrm{H}_2\mathrm{CO}_3 \rightarrow \mathrm{CO}_2 + \mathrm{H}_2\mathrm{O}$
%					\item Khi có bazơ được thêm vào máu: $\mathrm{OH}^- + \mathrm{H}_2\mathrm{CO}_3 \rightarrow \mathrm{HCO}_3^- + \mathrm{H}_2\mathrm{O}$
%				\end{itemize}
%				\item Điều chỉnh hô hấp:
%				\begin{itemize}
%					\item Khi pH máu giảm, trung tâm hô hấp được kích thích, tăng tần số và độ sâu hô hấp để thải $\mathrm{CO}_2$, làm tăng pH máu.
%					\item Khi pH máu tăng, giảm tần số và độ sâu hô hấp để giữ $\mathrm{CO}_2$, làm giảm pH máu.
%				\end{itemize}
%				
%				\item Điều chỉnh thận:
%				\begin{itemize}
%					\item Thận điều chỉnh nồng độ $\mathrm{HCO}_3^-$ bằng cách tái hấp thu hoặc bài tiết $\mathrm{HCO}_3^-$.
%					\item Khi pH máu giảm, thận tăng tái hấp thu và tổng hợp $\mathrm{HCO}_3^-$, đồng thời tăng bài tiết $\mathrm{H}^+$.
%					\item Khi pH máu tăng, thận giảm tái hấp thu $\mathrm{HCO}_3^-$ và tăng bài tiết $\mathrm{HCO}_3^-$.
%				\end{itemize}
%				Sự phối hợp của ba cơ chế trên tạo nên một hệ thống điều hòa pH máu hiệu quả, giúp duy trì pH máu ổn định trong khoảng hẹp 7,35 - 7,45, đảm bảo hoạt động bình thường của các quá trình sinh lý trong cơ thể.
%			\end{itemize}
%		\end{enumerate}
%	}
%\end{bt}
%%%%$==============$Bai_$BT8==============$%%%
%\begin{bt}
%	Một nông dân có một khu đất trồng rau diện tích $0.5$ hecta. Sau khi kiểm tra, họ nhận thấy pH của đất hiện tại là $5{,}5$, trong khi loại rau họ muốn trồng phát triển tốt nhất ở pH $6.5$. Họ quyết định sử dụng vôi nông nghiệp ($CaCO_3$) để tăng pH của đất.
%	\begin{enumerate}
%		\item Giải thích tại sao khi bón vôi làm cho pH của đất tăng
%		\item Biết rằng:$1$ tấn vôi nông nghiệp trên $1$ hecta đất sẽ làm tăng pH lên $0.5$ đơn vị.Giá vôi nông nghiệp là $1{,}500{,}000$ đồng/tấn.Hãy tính:
%		\begin{enumerate}[a)]
%			\item Tính lượng vôi (tính theo kg) cần thiết để tăng pH của toàn bộ khu đất từ $5.5$ lên $6.5$.
%			\item Tính tổng chi phí để mua đủ lượng vôi cần thiết.
%			\item Nếu nông dân chỉ có ngân sách $500{,}000$ đồng, họ có thể điều chỉnh pH của bao nhiêu phần trăm diện tích đất?
%		\end{enumerate}
%	\end{enumerate}
%	\loigiai{
%		\begin{enumerate}
%			\item Khi bón vôi $\left(\mathrm{CaCO}_3\right)$ vào đất, nó phản ứng với nước và $\mathrm{CO}_2$ trong đất theo phương trình:
%			\[
%			\mathrm{CaCO}_3+\mathrm{H}_2 \mathrm{O}+\mathrm{CO}_2 \rightarrow \mathrm{Ca}\left(\mathrm{HCO}_3\right)_2
%			\]
%			Canxi bicacbonat $\left(\mathrm{Ca}\left(\mathrm{HCO}_3\right)_2\right)$ được tạo ra sẽ điện ly trong dung dịch đất:
%			\[
%			\mathrm{Ca}\left(\mathrm{HCO}_3\right)_2 \rightarrow \mathrm{Ca}^{2+}+2 \mathrm{HCO}_3^{-}
%			\]
%			Ion bicacbonat $\left(\mathrm{HCO}_3^{-}\right)$phản ứng với ion $\mathrm{H}^{+}$trong đất:
%			\[
%			\mathrm{HCO}_3^{-}+\mathrm{H}^{+} \rightarrow \mathrm{H}_2 \mathrm{O}+\mathrm{CO}_2
%			\]
%			Quá trình này làm giảm nồng đô ion $\mathrm{H}^{+}$trong đất, dẫn đến tăng pH (giảm độ. chua).
%			\item Số tấn vôi cần thiết để tằng pH lên 1 đơn vị từ $5{,}5$ lên $6{,}5$ là:
%			\\
%			$\begin{array}{cccc}
%				\text{Cứ}& pH\; \uparrow\; 0{,}5 \; \text{đơn vị} & \xleftarrow & \text{1 tấn vôi} / \text{1 hecta }\\
%				\text{Vậy}&pH\; \uparrow\; 6{,}5-5{,}5=1 \; \text{đơn vị} & \xrightarrow & \dfrac{1}{0{,}5} = 2\; \text{tấn vôi} / \text{1 hecta }
%			\end{array}$
%			\\
%			- Lượng vôi cần bón cho $0{,}5$ hecta là $0{,}5 \cdot 2 = 1 $ (tấn)
%			\\
%			- Chi phí phải trả là $1{,}500{,}000 \cdot 1 = 1{,}500{,}000$ (đồng)
%			\\
%			- Phần trăm diện tích có thể điều chỉnh với ngân sách 500,000 đồng:
%			\begin{itemize}
%				\item Lượng vôi mua được: $500,000 \div 1,500,000=1 / 3$ tấn
%				\item Diện tích có thể điều chỉnh: $1 / 3 \div 1 \times 0.5$ ha $=1 / 6$ ha
%				\item Phần trăm diện tích: $(1 / 6 \div 0.5) \times 100 \%=33.33 \%$
%			\end{itemize}
%		\end{enumerate}
%	}
%\end{bt}
%%%%$==============$HetBai_$BT1==============$%%%
%\Closesolutionfile{ansbt}
%\Closesolutionfile{ansbth}
%%\bangdapanSA{AnsBT-H11C01B02-BTTL4}
%%
%\phan{Bài tập trắc nghiệm nhiều lựa chọn}
%%%%=============SOẠN EX===============%%%
%\Opensolutionfile{ansex}[Ans/LGEX-H11C01B02-BTTL4]
%\Opensolutionfile{ans}[Ans/Ans-H11C01B02-BTTL4]
%%\hienthiloigiaiex
%%%\tatloigiaiex
%%%\luuloigiaiex
%%%%=============EX_1=============%%%
%\begin{ex}
%	Dung dịch X chứa $HCl$ $0{,}2$ M. Dung dịch Y chứa $NaOH$ $0{,}15$ M. Trộn $200$ ml dung dịch X với $300$ ml dung dịch Y thu được dung dịch Z. pH của dung dịch Z là
%	\choice
%	{$1{,}3$}
%	{$2{,}3$}
%	{\True $11{,}7$}
%	{$12{,}7$}
%	\loigiai{%
%		\indam{Phân tích:} Đây là trường hợp trộn 1 axit với 1 bazơ. Ta cần tính số mol của $H^+$ và $OH^-$, xác định chất dư và tính pH.
%		\\[5pt]
%		$n_{H^+} = 0{,}2 \cdot 0{,}2 = 0{,}04$ mol
%		\\
%		$n_{OH^-} = 0{,}15 \cdot 0{,}3 = 0{,}045$ mol
%		\[
%		\begin{matrix}
%			& H^+&+& OH^- & \xrightarrow & H_2O\\
%			&0{,}04\;\text{mol}&\xleftarrow&0{,}04\;\text{mol}&&
%		\end{matrix}
%		\]
%		$\Rightarrow$ $n_{OH^-\text{dư}}=0{,}005$ mol
%		\\
%		$[OH^-] = \dfrac{0{,}005}{0{,}5} = 0{,}01$ M
%		\\
%		$pOH = -\log(0{,}01) = 2$
%		\\
%		$pH = 14 - pOH = 14 - 2 = 12$
%	}
%\end{ex}
%%%%=============EX_2=============%%%
%\begin{ex}
%	Dung dịch X là hỗn hợp $H_2SO_4$ $0{,}1$ M và $HNO_3$ $0{,}2$ M. Dung dịch Y chứa $Ca(OH)_2$ $0{,}15$ M. Trộn $200$ ml dung dịch X với $300$ ml dung dịch Y thu được dung dịch Z. pH của dung dịch Z là
%	\choice
%	{$1{,}7$}
%	{$2{,}3$}
%	{\True $12{,}3$}
%	{$13{,}7$}
%	\loigiai{%
%		\indam{Phân tích:} Đây là trường hợp trộn hỗn hợp 2 axit với 1 bazơ. Ta cần tính tổng số mol $H^+$ từ cả hai axit và số mol $OH^-$ từ bazơ.
%		\\[5pt]
%		$n_{H^+} = (2 \cdot 0{,}1 + 0{,}2) \cdot 0{,}2 = 0{,}08$ mol
%		\\
%		$n_{OH^-} = 2 \cdot 0{,}15 \cdot 0{,}3 = 0{,}09$ mol
%		\[
%		\begin{matrix}
%			& H^+&+& OH^- & \xrightarrow & H_2O\\
%			&0{,}08\;\text{mol}&\xleftarrow&0{,}08\;\text{mol}&&
%		\end{matrix}
%		\]
%		$\Rightarrow$ $n_{OH^-\text{dư}}=0{,}01$ mol
%		\\
%		$[OH^-] = \dfrac{0{,}01}{0{,}5} = 0{,}02$ M
%		\\
%		$pOH = -\log(0{,}02) = 1{,}7$
%		\\
%		$pH = 14 - pOH = 14 - 1{,}7 = 12{,}3$
%	}
%\end{ex}
%%%%=============EX_3=============%%%
%\begin{ex}
%	Dung dịch X chứa $HCl$ $0{,}25$ M. Dung dịch Y là hỗn hợp $NaOH$ $0{,}1$ M và $KOH$ $0{,}15$ M. Trộn $300$ ml dung dịch X với $200$ ml dung dịch Y thu được dung dịch Z. pH của dung dịch Z là
%	\choice
%	{\True $1{,}3$}
%	{$2{,}7$}
%	{$12{,}7$}
%	{$13{,3}$}
%	\loigiai{%
%		\indam{Phân tích:} Đây là trường hợp trộn 1 axit với hỗn hợp 2 bazơ. Ta cần tính số mol $H^+$ từ axit và tổng số mol $OH^-$ từ cả hai bazơ.
%		\\[5pt]
%		$n_{H^+} = 0{,}25 \cdot 0{,}3 = 0{,}075$ mol
%		\\
%		$n_{OH^-} = (0{,}1 + 0{,}15) \cdot 0{,}2 = 0{,}05$ mol
%		\[
%		\begin{matrix}
%			& H^+&+& OH^- & \xrightarrow & H_2O\\
%			&0{,}05\;\text{mol}&\xleftarrow&0{,}05\;\text{mol}&&
%		\end{matrix}
%		\]
%		$\Rightarrow$ $n_{H^+\text{dư}}=0{,}025$ mol
%		\\
%		$[H^+] = \dfrac{0{,}025}{0{,}5} = 0{,}05$ M
%		\\
%		$pH = -\log(0{,}05) = 1{,}3$
%	}
%\end{ex}
%%%%=============EX_4=============%%%
%\begin{ex}
%	Dung dịch X là hỗn hợp $H_2SO_4$ $0{,}05$ M và $HNO_3$ $0{,}1$ M. Dung dịch Y là hỗn hợp $Ba(OH)_2$ $0{,}04$ M và $NaOH$ $0{,}1$ M. Trộn $400$ ml dung dịch X với $100$ ml dung dịch Y thu được dung dịch Z. pH của dung dịch Z gần với giá trị nào nhất?
%	\choice
%	{\True $1$}
%	{$2$}
%	{$3$}
%	{$4$}
%	\loigiai{%
%		\indam{Phân tích:} Đây là trường hợp trộn hỗn hợp 2 axit với hỗn hợp 2 bazơ. Ta cần tính tổng số mol $H^+$ từ cả hai axit và tổng số mol $OH^-$ từ cả hai bazơ.
%		\\[5pt]
%		$n_{H^+} = (2 \cdot 0{,}05 + 0{,}1) \cdot 0{,}4 = 0{,}08$ mol
%		\\
%		$n_{OH^-} = (2 \cdot 0{,}04 + 0{,}1) \cdot 0{,}1 = 0{,}018$ mol
%		\[
%		\begin{matrix}
%			& H^+&+& OH^- & \xrightarrow & H_2O\\
%			&0{,}018\;\text{mol}&\xleftarrow&0{,}018\;\text{mol}&&
%		\end{matrix}
%		\]
%		$\Rightarrow$ $n_{H^+\text{dư}}=0{,}062$ mol
%		\\
%		$[H^+] = \dfrac{0{,}062}{0{,}5} = 0{,}124$ M
%		\\
%		$pH = -\log(0{,}124) = 0{,}907$
%	}
%\end{ex}
%
%%%%=============EX_6=============%%%
%\begin{ex}[Chuyên Bắc Giang-2018]
%	Cho V ml dung dịch $\mathrm{NaOH}$ $0,01\;M$ vào V ml dung dịch $\mathrm{HCl}$ $0,03\;M$ được $2V$ ml dung dịch Y. Dung dịch Y có pH là
%	\choice
%	{$1$}
%	{\True $2$} 
%	{$3$}
%	{$4$}
%	\loigiai{\[\begin{matrix}
%			HCl& + &NaOH &\xrightarrow &NaCl& + &H_2O\\
%			0{,}01V&\leftarrow&0{,}01V&&&
%		\end{matrix}\]
%		$\Rightarrow$ $n_{H^+\text{dư}}=0{,}03V - 0{,}01V =0{,}02V$ $\Rightarrow$ $[H^+] = \dfrac{0{,}02V}{2V}=0{,}01$ M $\Rightarrow$ $pH=2$
%	}
%\end{ex}
%%%%$=============EX_7=============$%%%
%\begin{ex}[Chuyên Lam Sơn-Thanh Hóa-Lần $1-2018$]
%	Dung dịch $HNO_3$ $0{,}1$ M có pH bằng
%	\choice
%	{$3{,}00$}
%	{$2{,}00$}
%	{$4{,}00$}
%	{\True $1{,}00$}
%	\loigiai{\[\begin{matrix}
%			HNO_3&\xrightarrow &H^+& + &NO_3^-\\
%			0{,}1&\xrightarrow &0{,}1&&
%		\end{matrix}\]
%		$\Rightarrow$ $pH=-log(0{,}1)=1$
%	}
%\end{ex}
%%%%=============EX_8=============%%%
%\begin{ex}
%	Dung dịch NaOH có $\mathrm{pH}=10$. Pha loãng dung dịch 10 lần bằng nước thì dung dịch mới pH bằng
%	\choice
%	{$6$}
%	{$7$}
%	{$8$}
%	{\True $9$}
%	\loigiai{%
%		$pH = 10 \Rightarrow [H^+]=10^{-10}$ M $\Rightarrow [OH^-] =10^{-4}$ M.
%		\\
%		Khi pha loãng dung dịch 10 lần thì nồng độ dung dịch giảm đi 10 lần
%		\\ 
%		$\Rightarrow [OH^-]^\prime = 10^{-5}$ M $\Rightarrow [H^+]^\prime = 10^{-9}$ M $\Rightarrow pH =9$.
%	}
%\end{ex}
%%%%=============EX_9=============$%%%
%\begin{ex}
%	Cho $200$ ml $H_2SO_4$ $0{,}05$ M vào $300$ ml dung dịch $\mathrm{NaOH}$ $0{,}06$ M. pH của dung dịch tạo thành là
%	\choice
%	{$2{,}7$}
%	{$1{,}6$}
%	{$1{,}9$}
%	{\True $2{,}4$}
%	\loigiai{%
%		$n_{H^+}=2n_{H_2SO_4}=2\cdot0{,}2\cdot0{,}05=0{,}02$ mol;
%		$n_{OH^-}=2n_{NaOH} =0{,}3\cdot0{,}06=0{,}018$ mol
%		\[\begin{matrix}
%			H^+&+& OH^-& \xrightarrow & H_2O\\
%			0{,}018&\leftarrow&0{,}018&&
%		\end{matrix}\]
%		$\Rightarrow$ $n_{H^+\text{dư}}= 0{,}02-0{,}018=0{,}002$ mol $\Rightarrow [H^+]=\dfrac{0{,}002}{0{,}5}=0{,}004$ M 
%		\\
%		$\Rightarrow pH=-log[H^+]=-log(0{,}004)=2{,}4$
%	}
%\end{ex}
%%%=============EX_10=============%%%
%\begin{ex}
%	Dung dịch có $\mathrm{pH}=3$. Pha loãng dung dịch bằng cách thêm vào $90$ ml nước cất thì dung dịch mới có $\mathrm{pH}=4$. Tính thể tich dung dịch trước khi pha loãng?
%	\choice
%	{\True $10$ ml}
%	{$910$ ml}
%	{$100$ ml}
%	{$110$ ml}
%	\loigiai{%
%		Vì khi pha loãng số mol $H^+$ trước và sau pha loãng không đổi
%		nên ta có phương trình:
%		\\
%		$\begin{aligned}
%			nH^+ &=1000V\cdot10^{-3} = 1000(V+90)\cdot10^{-4}\\
%			\Rightarrow V &= 10\;(ml)
%		\end{aligned}$
%	}
%\end{ex}
%%%%=============EX_11=============%%%
%\begin{ex}
%	Cho mẫu hợp kim K-Ba tác dụng với nước dư thu được dung dịch X và 4,48 lít khí ở đktc. Trung hoà X cần V lít dung dịch HCl có $\mathrm{pH}=1$. Giá trị của V là
%	\choice
%	{$2$}
%	{\True $4$}
%	{$6$}
%	{$8$}
%	\loigiai{
%		\begin{equation}\label{eq:Kpunuoc}
%			\mathrm{K} + \mathrm{H}_2\mathrm{O} \xrightarrow \mathrm{KOH} + \tfrac{1}{2}\mathrm{H_2}
%		\end{equation}
%		\begin{equation}\label{eq:Bapunuoc}
%			\mathrm{Ba} + \mathrm{2H_2O} \xrightarrow \mathrm{Ba}{(\mathrm{OH})}_\mathrm{2} + \mathrm{H}_2
%		\end{equation}
%		Theo phương trình (\ref{eq:Kpunuoc}) và (\ref{eq:Bapunuoc}) ta có $n_{OH^-}=2n_{H_2} =2\cdot\dfrac{4{,}48}{22{,}4}=0{,}4$ mol
%		\\
%		Phản ứng trung hòa : $H^+  + OH^-\xrightarrow H_2O$
%		\\
%		$n_{H^+}=n_{OH^-} =0{,}4$ mol. $pH =1 \Rightarrow [H^+] =0{,}1$ M.
%		$\Rightarrow$ $V_{HCl}= \dfrac{0{,}4}{0{,}1}= 4$ (lít)
%	}
%\end{ex}
%%%%=============EX_12=============%%%
%\begin{ex}[ĐHKB-2009][2 điểm]
%	Cho $100$ ml dung dịch hỗn hợp gồm $H_2SO_4$ $0{,}05\;M$ và $\mathrm{HCl}$ $0{,}1\;M$ vào $100$ ml dung dịch hỗn hợp gồm $\mathrm{NaOH}$ $0{,}2\;M$ và $\mathrm{Ba}(OH)_2$ $0{,}1\;M$, thu được dung dịch $X$. Dung dịch $X$ có pH là
%	\choice
%	{\True $13{,}0$}
%	{$1{,}2$}
%	{$1{,}0$}
%	{$12{,}8$}
%	\loigiai{%
%		$\begin{aligned}
%			n_{H^+}&=2n_{H_2SO_4} + n_{HCl}\\
%			&=2\cdot0{,}1\cdot0{,}05 + 0{,}1\cdot0{,}1\\
%			&=0{,}02\;\text (mol)
%		\end{aligned}$
%		\hspace{2cm}
%		$\begin{aligned}
%			n_{OH^-}&=n_{NaOH} + 2n_{Ba{(OH)}_2}\\
%			&=0{,}1\cdot0{,}2 + 2\cdot0{,}1\cdot0{,}1\\
%			&=0{,}04\;\text (mol)
%		\end{aligned}$
%		\\
%		$\begin{matrix}
%			H^+&+&OH^-&\xrightarrow& H_2O\\
%			0{,}02&\xrightarrow&0{,}02&&
%		\end{matrix}$
%		\\
%		$\Rightarrow n_{OH^-\text{dư}}=0{,}04-0{,}02=0{,}02$ (mol)
%		$\Rightarrow [OH^-]=\dfrac{0{,}02}{0{,}2}=0{,}1$ (mol)
%		\\
%		$\Rightarrow pH=14+log[OH^-]=14+log(0{,}1)=13$
%	}
%\end{ex}
%%
%%%%$=============EX_13=============$%%%
%\begin{ex}
%	A là dung dịch $\mathrm{Ba}(OH)_2$ có $\mathrm{pH}=12$. $B$ là dung dịch HCl có $\mathrm{pH}=2$. Phản ứng vừa đủ $V_1$ lít $A$ cần $V_2$ lít. Tìm $V_1/ V_2$?
%	\choice
%	{\True $1{,}0$}
%	{$2{,}0$}
%	{$0{,}5$}
%	{$2{,}5$}
%	\loigiai{%
%		dung dịch $Ba{(OH)}_2$ có $pH=12$ $\Rightarrow [OH^-]=\dfrac{10^{-14}}{10^{-12}}=10^{-2}$ (M)
%		\\
%		$\begin{matrix}
%			H^+ & + & OH^-&\xrightarrow &H_2O\\
%			10^{-2}V_1&\xrightarrow&10^{-2}V_2&&
%		\end{matrix}$
%		\\
%		$\Rightarrow 10^{-2}V_1 = 10^{-2}V_2\;\text{hay}\;\dfrac{V_1}{V_2}=1$
%	}
%\end{ex}
%%%=============EX_14=============%%%
%\begin{ex}[c][ĐH B-2008]
%	Cho V ml dung dịch $\mathrm{NaOH}$ $0,01$ M vào V ml dung dịch HCl $0,03$ M được $2V$ ml dung dich Y. Dung dich Y có pH là
%	\choice
%	{$4$}
%	{$3$}
%	{\True $2$}
%	{$1$}
%	\loigiai{%
%		Phương trình ion thu gọn
%		\[\begin{matrix}
%			H^+ &+ &OH^-&\xrightarrow & H_2O\\
%			10^{-3}V\cdot0{,}01 & \xleftarrow & 10^{-3}V\cdot0{,}01& &
%		\end{matrix}\]
%		$\Rightarrow n_{H^+\text{dư}} = 3\cdot10^{-5}V-10^{-5}V = 2\cdot10^{-5}V$ (mol)
%		$\Rightarrow [H^+] = \dfrac{2\cdot10^{-5}V}{2\cdot10^{-3}V} = 10^{-2}$ (M)
%		$\Rightarrow pH= 2$
%	}
%\end{ex}
%%%%=============EX_15=============%%%
%\begin{ex}
%	Trộn dung dịch $H_2SO_4$ $0{,}1$ M; $HNO_3$ $0{,}2$ M và $\mathrm{HCl}$ $0{,}3$ M với những thể tích bằng nhau thu được dung dịch A.
%	Lấy $300$ ml dung dịch A phản ứng với V lít dung dịch B gồm $\mathrm{NaOH}$ $0{,}2$ M và $KOH$ $0{,}29$ M thu được dung dịch $C$ có $\mathrm{pH}=2$. Giá trị của V là
%	\choice
%	{$0{,}134$}
%	{$0{,}112$}
%	{$0{,}067$}
%	{\True $0{,}414$}
%	\loigiai{%
%		\indam{Phân tích:} Đây là bài toán trộn 3 axit với 2 bazơ. Ta cần tính tổng số mol $H^+$ từ các axit và tổng số mol $OH^-$ từ các bazơ, sau đó giải phương trình để tìm thể tích V.
%		\\[5pt]
%		$n_{H^+} = \left(2 \cdot 0{,}1 + 0{,}2 + 0{,}3\right) \cdot 0{,}3 = 0{,}21$ mol
%		\\
%		$n_{OH^-} = \left(0{,}2 + 0{,}29\right) \cdot V = 0{,}49V$ mol
%		\\
%		Dung dịch có pH = 2 $\Rightarrow$ $[H^+] = 10^{-2}$ M
%		\\
%		Ta có phương trình cân bằng số mol:
%		\[
%		0{,}21 - 0{,}49V = 10^{-2} \times \left(0{,}3 + V\right)
%		\]
%		\\
%		Giải phương trình trên, ta tìm được $V = 0{,}414$ lít
%	}
%\end{ex}
%%%%$==============$Cau_$EX1==============$%%%
%\begin{ex}[ĐHA-2009][2 điểm]
%	Nung $6{,}58$ gam $\mathrm{Cu}\left(NO_3\right)_2$ trong bình kín không chứa không khí, sau một thời gian thu được $4{,}96$ gam chất rắn và hỗn hợp khí $X$. Hấp thụ hoàn toàn $X$ vào nước để được $300$ ml dung dịch $Y$. Dung dịch $Y$ có pH bằng
%	\choice
%	{\True $1$}
%	{$4$}
%	{$3$}
%	{$2$}
%	\loigiai{%
%		$Cu{(NO_3)}_2\xrightarrow[$t^\circ$] CuO + NO_2 + O_2$
%		\\
%		$\begin{matrix}
%			4NO_2 &+& O_2& + &2H_2O& \xrightarrow& 4HNO_3\\
%			4x&&x&&&&4x
%		\end{matrix}$
%		\\
%		Áp dụng định luật bỏa toàn khối lượng ta có:
%		$m_X=46\cdot4x +32x =6{,}58-4{,}96=1{,}62 \Rightarrow x =0{,}0075$ (mol).
%		\\
%		$\Rightarrow n_{HNO_3} = n_{NO_2}=4x =0{,}03$ (mol) $\Rightarrow pH =1 $
%	}
%\end{ex}
%%%%$==============$HetCau_$EX1==============$%%%
%
%%%%$==============$Cau_$EX2==============$%%%
%\begin{ex}
%	Dung dịch HCl có $\mathrm{pH}=5\left(\mathrm{~V}_1\right)$ cho vào dung dịch $KOH$ $\mathrm{pH}=9\left(\mathrm{~V}_2\right)$. Tính $V_1/ V_2$ để dung dịch mới có $\mathrm{pH}=8$?
%	\choice
%	{$0{,}1$}
%	{$10$}
%	{$2/ 9$}
%	{\True $9/ 11$}
%	\loigiai{%
%		Dung dịch $HCl$ có $pH=5 \Rightarrow [H^+]=10^{-5}$ (M) $\Rightarrow n_{H^+} = 10^{-5}V_1$ (mol)
%		\\
%		Dung dịch $KOH$ có $pH=9 \Rightarrow [H^+]=10^{-9}$ (M)
%		$\Rightarrow [OH^-]=10^{-5}$ (M)
%		$\Rightarrow n_{OH^-} = 10^{-5}V_2$ (mol)
%		\\
%		Dung dịch sau phản ứng có $pH = 8$ chứng tỏ $KOH$ còn dư sau phản ứng và $n_{OH^-\text{dư}}= 10^{-5} (V_2-V_1)$ (mol)
%		\\
%		Ta có $pH =8 \Rightarrow [H^+] = 10^{-8}$ M $\Rightarrow [OH^-]_{\text{dư}} = 10^{-6}$ M
%		\\
%		$\Rightarrow$ $10^{-5} (V_2-V_1)=10^{-6}(V_1+V_2)$ $V_1/V_2=9/11$
%	}
%\end{ex}
%%%%$==============$HetCau_$EX2==============$%%%
%
%%%%$==============$Cau_$EX3==============$%%%
%\begin{ex}
%	Pha loãng $100$ ml dung dịch NaOH có $\mathrm{pH}=12$ với $900$ ml nước cất thu được dung dịch mới có pH là
%	\choice
%	{$2$}
%	{$12$}
%	{\True $11$}
%	{$1$}
%	\loigiai{%
%		Dung dịch có $pH = 12$ $\Rightarrow [H^+] =10^{-12}$ M $\Rightarrow [OH^-]=10^{-2}$ M $\Rightarrow $ $n_{OH^-} = 0{,}1\cdot10^{-2}=10^{-3}$ mol
%		\\
%		$[OH^-]^\prime = \dfrac{10^{-3}}{1} = 10^{-3}$ M $\Rightarrow [H^+]^\prime = 10^{-11}$ $\Rightarrow pH=11$
%	}
%\end{ex}
%%%%$==============$HetCau_$EX3==============$%%%
%
%%%%$==============$Cau_$EX4==============$%%%
%\begin{ex}[$\mathbf{CD}-\mathbf{2011}$]
%	Cho $a$ lít dung dịch $KOH$ có $\mathrm{pH}=12{,}0$ vào $8{,}00$ lít dung dịch HCl có $\mathrm{pH}=3{,}0$ thu được dung dịch $Y$ có $\mathrm{pH}=11{,}0$. Giá trị của $a$ là
%	\choice
%	{$1{,}60$}
%	{$0{,}80$}
%	{\True $1{,}78$}
%	{$0{,}12$}
%	\loigiai{%
%		Dug dịch $KOH$ có $pH=12{,}0$ $\Rightarrow[H^+] =10^{-12}$ M $\Rightarrow [OH^-]= 10^{-2}$ M$\Rightarrow n_{OH^-} = 10^{-2}a$ (mol)
%		\\
%		Dug dịch $HCl$ có $pH=3{,}0$ $\Rightarrow[H^+] =10^{-3}$ M $\Rightarrow n_{H^+} = 8\cdot10^{-3}$ (mol)
%		\\
%		Dung dịch sau phản ứng có $pH =11{,}0$ nên $KOH$ còn dư sau phản ứng. $n_{OH^-\text{dư}} = 10^{-2}a-8\cdot10^{-3}$ (mol)\\
%		$[OH^-]_{\text{dư}} = \dfrac{10^{-2}a-8\cdot10^{-3}}{a+8}=10^{-2}$
%		$\Rightarrow a=1{,}78$
%	}
%\end{ex}
%%%%$==============$HetCau_$EX4==============$%%%
%
%%%%$==============$Cau_$EX5==============$%%%
%\begin{ex}[$KB-2008$]
%	Trộn $100$ ml dung dịch có $\mathrm{pH}=1$ gồm HCl và $HNO_3$ với $100$ ml dung dịch NaOH nồng độ $a$ $\mathrm{mol} /$ lít, thu được $200$ ml dung dịch có $\mathrm{pH}=12$. Giá trị của $a$ là
%	\choice
%	{$0{,}15$}
%	{$0{,}30$}
%	{$0{,}03$}
%	{\True $0{,}12$}
%	\loigiai{%
%		Dung dịch có $pH=1 \Rightarrow [H^+] = 0{,}1$ M $\Rightarrow$ $n_{H^+} = 0{,}1\cdot0{,}1 = 0{,}01$ (mol);
%		$n_{NaOH} =0{,}1a$ (mol)
%		\\
%		Dung dịch sau khi trộn có $pH =12$ nên NaOH còn dư và $n_{NaOH\;\text{dư}} = 0{,}1a - 0{,}01$ (mol)
%		\\
%		Theo đề bài ta có phương trình $0{,}1a - 0{,}01 = 0{,}2\cdot0{,}02 =0{,}04$ $\Rightarrow a =0{,}12$
%	}
%\end{ex}
%%%%$==============$HetCau_$EX5==============$%%%
%\Closesolutionfile{ans}
%\Closesolutionfile{ansex}
%%\bangdapan{Ans-H11C01B02-BTTL4}
%%%%