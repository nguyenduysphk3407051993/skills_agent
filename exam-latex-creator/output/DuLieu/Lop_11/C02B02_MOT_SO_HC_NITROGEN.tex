\setcounter{section}{1}
\section{Một số hợp chất quan trọng của nitrogen}
	\begin{Muctieu}
		\begin{itemize}
			\item Nêu được công thức hóa học, tính chất vật lí và tính chất hóa học cơ bản của ammonia ($NH_3$) (tính bazơ, tính khử).
			\item Viết được phương trình hóa học minh họa các tính chất của ammonia.
			\item Trình bày được ứng dụng của ammonia và muối amoni.
			\item Nêu được tính chất hóa học của acid nitric ($HNO_3$) (tính acid mạnh, tính oxi hóa mạnh).
			\item Viết được phương trình hóa học minh họa các tính chất của acid nitric.
			\item Nêu được ứng dụng của acid nitric và muối nitrat.
			\item Giải thích được một số hiện tượng trong đời sống liên quan đến các hợp chất này.
		\end{itemize}
	\end{Muctieu}
	%%%
	\begin{kd}
		\immini{Trong nông nghiệp, phân bón đạm đóng vai trò vô cùng quan trọng giúp cây trồng phát triển xanh tốt. Em có biết loại phân đạm nào được sản xuất từ khí nitrogen trong không khí và có công thức hóa học là gì không?}{}
	\end{kd}

\subsection{Nội dung bài học}
%%%Phần lý thuyết
\subsubsection{Ammonia ($NH_3$)}
\Noibat[\maunhan][][][]{Cấu tạo phân tử và tính chất vật lí}

\begin{figure}[!htp]
	\begin{center}
		%% Hình phụ 1
		\subcaptionbox{\label{ctlewis}}[6cm]{\chemfig[atom sep =5em]{H-\charge{90:2.0pt =\:[{.style={draw=none,fill=\mycolor}}]}{N}(-[:-90]H)-H}}
		%% Hình phụ 2
		\subcaptionbox{\label{ctphantu}}[6cm]{\chemfig[atom sep =3em]{H?[a]-[:40,2]N(<[:-70,2]H?[a,,dashed]?[b])<:[:-30,2]H?[a,,dashed]?[b,,dashed]}}
		\caption{Công thức Lewis \ref{ctlewis} và dạng hình học \ref{ctphantu}  của $NH_3$ \label{NH3}}
	\end{center}
\end{figure}
Ammonia ($NH_3$) là một chất khí không màu, có mùi khai đặc trưng, nhẹ hơn không khí ($M_{NH_3} = 17$ g/mol; $M_{\text{không khí}} = 29$ g/mol). Ammonia tan rất nhiều trong nước, tạo thành dung dịch amoniac. Khí ammonia dễ hóa lỏng ở nhiệt độ thấp ($-33,4^\circ C$) và áp suất cao.

\begin{hoivadap}
	Dựa vào cấu tạo phân tử của ammonia, giải thích tại sao khí $NH_3$ tan rất nhiều trong nước.
\end{hoivadap}

\Noibat[\maunhan][][][]{Tính chất hóa học của Ammonia}

\Noibat[\maunhan][][\faBook][]{Tính base}

Ammonia thể hiện tính bazơ yếu do cặp electron chưa liên kết trên nguyên tử nitrogen. Nó có khả năng nhận proton ($H^+$).
\begin{itemize}
	\item \textbf{Tác dụng với nước:} Một phần nhỏ ammonia phản ứng với nước tạo ion amoni ($NH_4^+$) và ion hiđroxit ($OH^-$), làm dung dịch có môi trường bazơ (làm quỳ tím hóa xanh, phenolphtalein hóa hồng).
	\[
	\text{NH}_3\text{ (g)} + \text{H}_2\text{O}\text{ (l)} \xharpoonarrow[][][1] \text{NH}_4^+\text{ (aq)} + \text{OH}^-\text{ (aq)}
	\]
	\item \textbf{Tác dụng với acid:} Ammonia phản ứng với acid tạo thành muối amoni.
	\[
	\text{NH}_3\text{ (g)} + \text{HCl}\text{ (g)} \xrightarrow \text{NH}_4\text{Cl}\text{ (s)}
	\]
	(Phản ứng tạo khói trắng amoni clorua khi hai khí tiếp xúc).
	\[
	3\text{NH}_3\text{ (aq)} + \text{H}_3\text{PO}_4\text{ (aq)} \xrightarrow (\text{NH}_4)_3\text{PO}_4\text{ (aq)}
	\]
	\item \textbf{Tác dụng với dung dịch muối của kim loại yếu (kết tủa hiđroxit):} Dung dịch amoniac tác dụng với dung dịch muối của một số kim loại tạo kết tủa hiđroxit.
	\[
	3\text{NH}_3\text{ (aq)} + \text{AlCl}_3\text{ (aq)} + 3\text{H}_2\text{O}\text{ (l)} \xrightarrow \text{Al(OH)}_3 \downarrow + 3\text{NH}_4\text{Cl}\text{ (aq)}
	\]
\end{itemize}
	\begin{hoivadap}
		Khi cho quỳ tím ẩm tiếp xúc với khí ammonia thì có hiện tượng gì?
	\end{hoivadap}
\Noibat[\maunhan][][\faBook][]{Tính khử}
\[
4\overset{-3}{\text{N}}\text{H}_3(\text{g}) + 3\text{O}_2(\text{g}) \xrightarrow[$t^\circ$] 2\overset{0}{\text{N}_2}(\text{g}) + 6\text{H}_2\text{O}(\text{g})
\]
\[
4\overset{-3}{\text{N}}\text{H}_3(\text{g}) + 5\text{O}_2(\text{g}) \xrightarrow[$800-900^\circ\text{C}, \text{Pt}$][][2] 4\overset{+2}{\text{N}}\text{O}(\text{g}) + 6\text{H}_2\text{O}(\text{g})
\]

\Noibat[\maunhan][][][]{Tổng hợp Ammonia ($NH_3$) theo quá trình haber}
	\begin{center}
		\includegraphics[width=9cm]{Images/anhhoa11/C02_B03_AMMONIA/chu_trinh_haber.png}
		\captionof{figure}{Sơ đồ nguyên tắc quá trình haber tổng hợp ammonia }
	\end{center}
\Noibat[\maunhan][][\faApple][]{Phương trình hóa học và đặc điểm chung}
Quá trình tổng hợp ammonia từ nitrogen ($N_2$) và hydrogen ($H_2$) là một phản ứng thuận nghịch.
\[
\text{N}_2\text{(g)} + 3\text{H}_2\text{(g)} \xharpoonarrow[$400-600^\circ\text{C}$][$200\, \text{bar}, \text{Fe}$][2] 2\text{NH}_3\text{(g)} \quad \Delta_rH^\circ_{298} = -92 \text{ kJ}
\]
Đây là một phản ứng tỏa nhiệt, có nghĩa là quá trình thuận lợi về mặt năng lượng khi giải phóng nhiệt ra môi trường. Quá trình tổng hợp ammonia là một trong những quy trình hóa học quy mô lớn nhất và dẫn đầu về mức tiêu thụ năng lượng trên thế giới.

\Noibat[\maunhan][][\faApple][]{Các giai đoạn và điều kiện thực hiện}
Quá trình Haber-Bosch để tổng hợp ammonia thường được thực hiện theo các bước chính sau trong công nghiệp:
\begin{itemize}
	\item \textbf{Hỗn hợp khí đầu vào }
	Hỗn hợp khí nitrogen và hydrogen được chuẩn bị và đưa vào với tỉ lệ mol $1:3$. Nitrogen thường được lấy từ không khí, còn hydrogen thường được sản xuất từ khí thiên nhiên hoặc nước.
	\item \textbf{Tháp tổng hợp ammonia (Lò phản ứng)}
	Hỗn hợp khí đầu vào được nén và đưa vào tháp tổng hợp ammonia dưới các điều kiện tối ưu nhằm đạt được hiệu suất và tốc độ phản ứng cao nhất có thể:
	\begin{itemize}
		\item \textbf{Nhiệt độ:} Khoảng $380 - 450^\circ\text{C}$ (hoặc có thể $400 - 600 ^\circ\text{C}$ tùy công nghệ). Nhiệt độ này được lựa chọn để cân bằng giữa việc tăng tốc độ phản ứng (nhiệt độ cao) và việc dịch chuyển cân bằng theo chiều thuận (do phản ứng tỏa nhiệt, nhiệt độ thấp sẽ làm tăng hiệu suất cân bằng).
		\item \textbf{Áp suất:} Khoảng $200$ bar (hoặc $150 - 200$ bar). Áp suất cao được sử dụng vì theo nguyên lí Le Chatelier, phản ứng tổng hợp ammonia có số mol khí giảm (từ $4$ mol khí chất phản ứng tạo thành $2$ mol khí sản phẩm). Tăng áp suất sẽ làm cân bằng dịch chuyển theo chiều giảm số mol khí, tức là chiều thuận, do đó làm tăng hiệu suất tạo ra $NH_3$.
		\item \textbf{Chất xúc tác:} Sử dụng bột sắt ($Fe$) được hoạt hóa (thường có pha thêm các oxit như $Al_2O_3$, $K_2O$). Chất xúc tác giúp tăng tốc độ cả phản ứng thuận và nghịch, giúp hệ đạt trạng thái cân bằng nhanh hơn, từ đó tăng năng suất sản xuất $NH_3$ mà không làm thay đổi vị trí cân bằng.
	\end{itemize}
	
	\item \textbf{Tháp làm lạnh và tách sản phẩm }
	Hỗn hợp khí sau phản ứng (gồm $N_2$, $H_2$ chưa phản ứng và $NH_3$ đã tạo thành) được dẫn đến tháp làm lạnh. Tại đây, do nhiệt độ sôi của ammonia ($-33$ $^\circ\text{C} $) cao hơn nhiều so với $N_2$ ($-196$ $^\circ\text{C} $) và $H_2$ ($-253$ $^\circ\text{C} $), ammonia sẽ hóa lỏng và được tách riêng ra dưới dạng lỏng.
	
	\item \textbf{Tái chế khí chưa phản ứng}
	Hỗn hợp khí $N_2$ và $H_2$ chưa phản ứng sau khi tách $NH_3$ sẽ được đưa trở lại tháp tổng hợp để tiếp tục phản ứng. Việc tái chế này giúp tối ưu hóa hiệu suất sử dụng nguyên liệu và tăng hiệu quả kinh tế của toàn bộ quy trình.
\end{itemize}
\begin{hoivadap}
	Dựa vào nhiệt độ sôi của $N_2$, $H_2$ và $NH_3$ (lần lượt là $-196$ $^\circ\text{C} $, $-253$ $^\circ\text{C} $ và $-33$ $^\circ\text{C} $), hãy giải thích chi tiết phương pháp tách $NH_3$ ra khỏi hỗn hợp sau phản ứng trong quy trình Haber-Bosch.
\end{hoivadap}
\subsubsection{Muối ammonium}


\Noibat[\maunhan][][][]{Sự điện li và tính tan ion Ammonium ($NH_4^+$)}

Khi tan trong nước, muối amoni phân li hoàn toàn thành ion. Điều này giải thích tại sao dung dịch các muối amoni có tính dẫn điện tốt.
Ion amoni ($NH_4^+$) được xem là một acid yếu theo thuyết Brønsted-Lowry. Trong dung dịch, ion $NH_4^+$ có khả năng nhường proton ($H^+$) cho nước, tạo thành ammonia và ion hiđronium ($H_3O^+$). Đây là một phản ứng thuận nghịch:
\[
\text{NH}_4^+\text{ (aq)} + \text{H}_2\text{O}\text{ (l)} \xharpoonarrow \text{NH}_3\text{ (aq)} + \text{H}_3\text{O}^+\text{ (aq)}
\]
Tính acid yếu này của ion amoni ($NH_4^+$) là lý do vì sao phân bón chứa gốc amoni có thể làm tăng độ chua của đất khi được bón lâu ngày, do giải phóng $H_3O^+$.

\begin{Bancobiet}
	Phân bón chứa ion amoni ($NH_4^+$) như đạm amoni (ví dụ $NH_4Cl$, $(NH_4)_2SO_4$) có tính chất làm tăng độ chua của đất sau quá trình cây hấp thụ $NH_4^+$ và nitrat hóa. Do đó, loại phân này thường thích hợp cho đất chua nhưng cần lưu ý điều chỉnh pH đất nếu bón liên tục.
\end{Bancobiet}

\Noibat[\maunhan][][][]{Tác dụng của Muối amoni với Base}

Một trong những tính chất hóa học quan trọng của muối amoni là khả năng tác dụng với các dung dịch kiềm (base). Phản ứng này đặc biệt xảy ra dễ dàng khi đun nóng, giải phóng khí ammonia ($NH_3$) có mùi khai đặc trưng.
\[
\text{NH}_4^+\text{ (aq)} + \text{OH}^-\text{ (aq)} \xrightarrow[$t^\circ$] \text{NH}_3\uparrow + \text{H}_2\text{O}\text{ (l)}
\]
Phản ứng này được sử dụng để nhận biết ion amoni ($NH_4^+$) trong dung dịch. Khi thêm dung dịch kiềm vào dung dịch chứa ion $NH_4^+$ và đun nóng nhẹ, nếu có khí mùi khai thoát ra làm xanh giấy quỳ ẩm thì có sự hiện diện của $NH_4^+$.

\Noibat[\maunhan][][][]{Ví dụ phản ứng:}
\begin{itemize}
	\item \textbf{Amoni clorua tác dụng với natri hiđroxit:}
	\[
	\text{NH}_4\text{Cl}\text{ (aq)} + \text{NaOH}\text{ (aq)} \xrightarrow[$t^\circ$] \text{NaCl}\text{ (aq)} + \text{NH}_3\uparrow + \text{H}_2\text{O}\text{ (l)}
	\]
	\item \textbf{Amoni sunfat tác dụng với natri hiđroxit:}
	\[
	(\text{NH}_4)_2\text{SO}_4\text{ (aq)} + 2\text{NaOH}\text{ (aq)} \xrightarrow[$t^\circ$] \text{Na}_2\text{SO}_4\text{ (aq)} + 2\text{NH}_3\uparrow + 2\text{H}_2\text{O}\text{ (l)}
	\]
\end{itemize}

\begin{hoivadap}
	Một học sinh muốn kiểm tra xem một mẫu phân bón X có chứa ion amoni ($NH_4^+$) hay không. Em hãy đề xuất một thí nghiệm đơn giản để nhận biết sự có mặt của ion này.
\end{hoivadap}
\Noibat[\maunhan][][][]{Phản ứng nhiệt phân của muối ammonium}

Khi đun nóng, các muối amoni bị phân hủy. Sản phẩm nhiệt phân phụ thuộc vào bản chất của gốc axit.
\begin{itemize}
	\item \textbf{Muối của axit không có tính oxi hóa (ví dụ: $NH_4Cl, (NH_4)_2CO_3$):} Bị phân hủy tạo ra khí $NH_3$ và axit tương ứng.
	\[
	\text{NH}_4\text{Cl}\text{ (s)} \xrightarrow[$t^\circ$] \text{NH}_3\text{ (g)} + \text{HCl}\text{ (g)}
	\]
	\[
	(\text{NH}_4)_2\text{CO}_3\text{ (s)} \xrightarrow[$t^\circ$] 2\text{NH}_3\text{ (g)} + \text{CO}_2\text{ (g)} + \text{H}_2\text{O}\text{ (g)}
	\]
	\item \textbf{Muối của axit có tính oxi hóa (ví dụ: $NH_4NO_2, NH_4NO_3$):} Phản ứng oxi hóa-khử nội phân tử. Nitrogen trong ion $NH_4^+$ (số oxi hóa $-3$) bị oxi hóa bởi nitrogen trong gốc axit (số oxi hóa dương) để tạo ra các sản phẩm khí ($N_2, N_2O$).
	\[
	\text{NH}_4\text{NO}_2\text{ (s)} \xrightarrow[$t^\circ$] \text{N}_2\text{ (g)} + 2\text{H}_2\text{O}\text{ (g)}
	\]
	(Phản ứng này thường dùng để điều chế khí $N_2$ trong phòng thí nghiệm)
	\[
	\text{NH}_4\text{NO}_3\text{ (s)} \xrightarrow[$t^\circ$] \text{N}_2\text{O}\text{ (g)} + 2\text{H}_2\text{O}\text{ (g)}
	\]
	(Ở nhiệt độ cao hơn, $NH_4NO_3$ có thể phân hủy tạo $N_2$).
\end{itemize}

\Noibat[\maunhan][][][]{Ứng dụng của Muối ammonium}

Muối amoni được sử dụng chủ yếu làm phân đạm trong nông nghiệp, cung cấp nguồn nitrogen cần thiết cho cây trồng. Các loại phân đạm phổ biến bao gồm:
\begin{itemize}
	\item Amoni clorua ($NH_4Cl$)
	\item Amoni nitrat ($NH_4NO_3$)
	\item Amoni sunfat ($(NH_4)_2SO_4$)
	\item Diamoni hiđrophotphat $(NH_4)_2HPO_4$)
\end{itemize}
Ngoài ra, một số muối amoni còn có ứng dụng trong sản xuất pháo hoa (ví dụ $NH_4NO_3$), trong ngành y tế và công nghiệp khác.

\begin{tongket}{Kiến thức cần nhớ}
	\begin{itemize}
		\item \textbf{Ammonia ($NH_3$):} Tan rất tốt trong nước (do phân tử phân cực và tạo liên kết hiđro với các phân tử nước).
		\item \textbf{Muối ammonium:}
		\begin{itemize}
			\item Tính tan: Hầu hết tan tốt trong nước.
			\item Tính acid yếu của ion $NH_4^+$: $NH_4^+ + H_2O \rightleftharpoons NH_3 + H_3O^+$.
			\item Tác dụng với base: $NH_4^+ + OH^- \xrightarrow[$t^\circ$] NH_3\uparrow + H_2O$ (dùng để nhận biết $NH_4^+$).
			\item Nhiệt phân:
			\begin{itemize}
				\item Muối của axit không oxi hóa: Tạo $NH_3$ và axit.
				\item Muối của axit có tính oxi hóa: Phản ứng oxi hóa-khử nội phân tử, tạo $N_2$ hoặc $N_2O$.
			\end{itemize}
			\item Ứng dụng: Chủ yếu làm phân đạm.
		\end{itemize}
	\end{itemize}
\end{tongket}







\subsection{Các dạng bài tập}
\begin{dang}{Bài tập lý thuyết về Ammonia ($NH_3$) và Ammonium ($NH_4^+$)}
	\begin{phuongphap}
		Học sinh cần nắm vững:
		\begin{itemize}
		   \item \textbf{Tính chất vật lí đặc trưng của $NH_3$} (màu, mùi, trạng thái, độ tan, tỉ khối hơi).
		   \item \textbf{Tính chất hóa học cơ bản của $NH_3$}: tính bazơ yếu (tác dụng với nước, acid, dung dịch muối kim loại yếu) và tính khử (tác dụng với oxygen, oxit kim loại).
		   \item \textbf{Các phương trình hóa học minh họa cho từng tính chất.}
		   \item \textbf{Các ứng dụng quan trọng của $NH_3$ trong công nghiệp, nông nghiệp và đời sống.}
			\item \textbf{Tính chất vật lí của muối ammonium:} Hầu hết là chất rắn, dạng tinh thể ion, không màu và tan tốt trong nước.
			\item \textbf{Tính chất hóa học đặc trưng của $NH_4^+$:}
			\begin{itemize}
				\item \textbf{Tác dụng với dung dịch base mạnh (Phản ứng nhận biết ion $NH_4^+$):} Khi đun nóng nhẹ, muối ammonium phản ứng với dung dịch base mạnh (như $NaOH$, $KOH$, $Ca(OH)_2$) sinh ra khí ammonia ($NH_3$) có mùi khai đặc trưng.
				\[ \mathrm{NH_4}^+ + \mathrm{OH}^- \xrightarrow[$t^\circ$] \mathrm{NH_3} \uparrow + \mathrm{H_2O} \]
				\item \textbf{Phản ứng nhiệt phân:} Sản phẩm của phản ứng phụ thuộc vào bản chất của gốc acid tạo muối.
				\begin{itemize}
					\item \textit{Với gốc acid không có tính oxi hoá} ($Cl^-$, $CO_3^{2-}$,...): Phân huỷ thành $NH_3$ và acid tương ứng.
					\begin{eqnarray*}
						NH_4Cl(s) &\xharpoonarrow& NH_3(g) + HCl(g) \\
						(NH_4)_2CO_3(s) &\xrightarrow[$t^\circ$]& 2NH_3(g) + CO_2(g) + H_2O(g)
					\end{eqnarray*}
					\item \textit{Với gốc acid có tính oxi hoá} ($NO_3^-$, $NO_2^-$,...): Xảy ra phản ứng oxi hoá - khử nội phân tử, tạo ra các sản phẩm như $N_2$, $N_2O$.
					\begin{eqnarray*}
						NH_4NO_2(s) &\xrightarrow[$t^\circ$]& N_2(g) + 2H_2O(g) \\
						NH_4NO_3(s) &\xrightarrow[$t^\circ$]& N_2O(g) + 2H_2O(g)
					\end{eqnarray*}
				\end{itemize}
				\item \textbf{Tính acid trong dung dịch:} Ion $NH_4^+$ là một acid Brønsted, khi tan trong nước sẽ bị thuỷ phân tạo môi trường có tính acid (làm quỳ tím hoá hồng).
				\[ \mathrm{NH_4}^+ + \mathrm{H_2O} \xharpoonarrow \mathrm{H_3O}^+ + \mathrm{NH_3} \]
			\end{itemize}
			\item \textbf{Ứng dụng:} Chủ yếu được sử dụng làm phân đạm trong nông nghiệp (phân amoni, urê), ngoài ra còn dùng làm chất điện giải trong pin, bột nở, v.v.
		\end{itemize}
	\end{phuongphap}
	\phan{Bài tập tự luận}
	%%%=============SOẠN BT===============%%%
	\Opensolutionfile{ansbth}[Ans/LGBT-C02B04_NH3LT]
	\Opensolutionfile{ansbt}[Ans/AnsBT-C02B04_NH3LT]
	
	%%%=============BT_1=============%%%
	\begin{bt}
		Ammonia ($NH_3$) là một hợp chất quan trọng của nitrogen.
		\begin{enumerate}
			\item Nêu các tính chất vật lí đặc trưng của khí ammonia.
			\item Giải thích tại sao khí $NH_3$ tan rất nhiều trong nước.
		\end{enumerate}
		\loigiai{
			\begin{enumerate}
				\item \textbf{Các tính chất vật lí đặc trưng của khí ammonia:}
				Ammonia là một chất khí không màu, có mùi khai đặc trưng, nhẹ hơn không khí ($M_{NH_3} = 17$ g/mol).
				\item \textbf{Giải thích khả năng tan rất nhiều trong nước:}
				Ammonia tan rất nhiều trong nước là do phân tử ammonia có khả năng tạo liên kết hiđro với các phân tử nước. Nguyên tử nitrogen trong $NH_3$ có độ âm điện lớn và còn một cặp electron tự do, còn nguyên tử hiđro trong $H_2O$ liên kết với nguyên tử oxi có độ âm điện lớn. Sự hình thành các liên kết hiđro mạnh mẽ giữa $NH_3$ và $H_2O$ làm cho $NH_3$ dễ dàng hòa tan vào nước.
			\end{enumerate}
		}
	\end{bt}
	
	%%%=============BT_2=============%%%
	\begin{bt}
		Ammonia thể hiện cả tính bazơ yếu và tính khử.
		\begin{enumerate}
			\item Viết phương trình hóa học minh họa tính bazơ yếu của ammonia khi tác dụng với một acid và với dung dịch muối của kim loại yếu.
			\item Viết phương trình hóa học minh họa tính khử của ammonia khi tác dụng với oxygen (có xúc tác) và với một oxit kim loại.
		\end{enumerate}
		\loigiai{
			\begin{enumerate}
				\item \textbf{Tính bazơ yếu:}
				\begin{itemize}
					\item Tác dụng với acid (ví dụ: $HCl$):
					$\text{NH}_3\text{ (g)} + \text{HCl}\text{ (g)} \xrightarrow \text{NH}_4\text{Cl}\text{ (s)}$ (khói trắng)
					\item Tác dụng với dung dịch muối của kim loại yếu (ví dụ: $AlCl_3$):
					$3\text{NH}_3\text{ (aq)} + \text{AlCl}_3\text{ (aq)} + 3\text{H}_2\text{O}\text{ (l)} \rightarrow \text{Al(OH)}_3 \downarrow + 3\text{NH}_4\text{Cl}\text{ (aq)}$
				\end{itemize}
				\item \textbf{Tính khử:}
				\begin{itemize}
					\item Tác dụng với oxygen (có xúc tác Pt/Rh):
					$4\text{NH}_3\text{ (g)} + 5\text{O}_2\text{ (g)} \xrightarrow[$Pt/Rh,t^\circ$] 4\text{NO}\text{ (g)} + 6\text{H}_2\text{O}\text{ (g)}$
					\item Tác dụng với oxit kim loại (ví dụ: $CuO$):
					$2\text{NH}_3\text{ (g)} + 3\text{CuO}\text{ (s)} \xrightarrow[$t^\circ$] \text{N}_2\text{ (g)} + 3\text{Cu}\text{ (s)} + 3\text{H}_2\text{O}\text{ (g)}$
				\end{itemize}
			\end{enumerate}
		}
	\end{bt}
	%%%=============BT_3=============%%%
	\begin{bt}
		Ammonia là một trong những hóa chất công nghiệp quan trọng nhất. Nêu ít nhất ba ứng dụng chính của ammonia trong các lĩnh vực khác nhau.
		\loigiai{
			Ba ứng dụng chính của ammonia là:
			\begin{itemize}
				\item \textbf{Sản xuất phân đạm:} Là nguyên liệu chính để sản xuất các loại phân đạm như ure ($CO(NH_2)_2$), amoni nitrat ($NH_4NO_3$), amoni sunfat ($(NH_4)_2SO_4$), cung cấp nguồn nitrogen thiết yếu cho cây trồng.
				\item \textbf{Sản xuất acid nitric ($HNO_3$):} Ammonia được oxi hóa để sản xuất $NO$, sau đó chuyển hóa thành $HNO_3$, một acid mạnh có nhiều ứng dụng trong công nghiệp.
				\item \textbf{Chất làm lạnh:} Ammonia lỏng được sử dụng làm chất làm lạnh trong các hệ thống làm lạnh công nghiệp lớn (ví dụ: kho lạnh) do dễ hóa lỏng và có nhiệt hóa hơi lớn.
				\item (Có thể kể thêm: sản xuất chất dẻo, dược phẩm, làm chất tẩy rửa gia dụng).
			\end{itemize}
		}
	\end{bt}
	%%%=============BT_4=============%%%
	\begin{bt}
		Cho một ít chất chỉ thị phenolphtalein vào dung dịch $NH_3$ loãng thu được dung dịch (A). Màu của dung dịch (A) thay đổi như thế nào khi:
		\begin{enumerate}
			\item Đun nóng dung dịch một hồi lâu.
			\item Thêm dung dịch $HCl$ với số mol $HCl$ bằng số mol $NH_3$ có trong dung dịch (A).
			\item Thêm vài giọt dung dịch $Na_2CO_3$.
			\item Thêm từ từ dung dịch $AlCl_3$ tới dư.
		\end{enumerate}
		\loigiai{
			Dung dịch $NH_3$ loãng có tính bazơ yếu do phản ứng thuận nghịch:
			\[
			\text{NH}_3\text{ (aq)} + \text{H}_2\text{O}\text{ (l)} \xharpoonarrow \text{NH}_4^+\text{ (aq)} + \text{OH}^-\text{ (aq)}
			\]
			Do có ion $OH^-$, dung dịch amoniac làm phenolphtalein chuyển sang màu hồng.
			\begin{enumerate}
				\item \textbf{Đun nóng dung dịch một hồi lâu:}
				\begin{itemize}
					\item Khi đun nóng, khí $NH_3$ (có nhiệt độ sôi $-33,4^\circ C$) sẽ bay hơi và thoát ra khỏi dung dịch.
					\item Nồng độ $NH_3$ trong dung dịch giảm, làm cho cân bằng $NH_3 + H_2O \xharpoonarrow NH_4^+ + OH^-$ dịch chuyển theo chiều nghịch (từ phải sang trái).
					\item Nồng độ ion $OH^-$ giảm.
					\item Do nồng độ $OH^-$ giảm, tính bazơ của dung dịch yếu đi, làm màu hồng của phenolphtalein nhạt dần rồi mất màu.
				\end{itemize}
				\item \textbf{Thêm dung dịch $HCl$ với số mol $HCl$ bằng số mol $NH_3$ có trong dung dịch (A):}
				\begin{itemize}
					\item $HCl$ là một acid mạnh, sẽ phản ứng với $NH_3$ (bazơ yếu) theo phương trình:
					$\text{NH}_3\text{ (aq)} + \text{HCl}\text{ (aq)} \rightarrow \text{NH}_4\text{Cl}\text{ (aq)}$
					\item Phản ứng này tiêu thụ $NH_3$, làm cho cân bằng $NH_3 + H_2O \xharpoonarrow NH_4^+ + OH^-$ dịch chuyển theo chiều thuận (từ trái sang phải) để bù đắp lại $NH_3$ đã mất.
					\item Tuy nhiên, $HCl$ là acid, sẽ trung hòa hoàn toàn $OH^-$ có trong dung dịch amoniac. Sau khi $NH_3$ phản ứng hết với $HCl$, dung dịch thu được là $NH_4Cl$. $NH_4Cl$ là muối của bazơ yếu và acid mạnh, có tính acid yếu ($NH_4^+ \xharpoonarrow NH_3 + H^+$).
					\item Do dung dịch chuyển sang môi trường acid yếu hoặc gần trung tính (pH $< 7$), màu hồng của phenolphtalein sẽ mất.
				\end{itemize}
				\item \textbf{Thêm vài giọt dung dịch $Na_2CO_3$:}
				\begin{itemize}
					\item $Na_2CO_3$ là muối của bazơ mạnh ($NaOH$) và acid yếu ($H_2CO_3$), nên dung dịch $Na_2CO_3$ có tính bazơ do sự thủy phân của ion $CO_3^{2-}$: $CO_3^{2-} + H_2O \xharpoonarrow HCO_3^- + OH^-$.
					\item Khi thêm $Na_2CO_3$ vào dung dịch $NH_3$ loãng, nồng độ $OH^-$ trong dung dịch sẽ tăng lên.
					\item Do nồng độ $OH^-$ tăng, màu hồng của phenolphtalein sẽ đậm hơn.
				\end{itemize}
				\item \textbf{Thêm từ từ dung dịch $AlCl_3$ tới dư:}
				\begin{itemize}
					\item $AlCl_3$ là muối của kim loại yếu ($Al^{3+}$) và acid mạnh ($HCl$). Ion $Al^{3+}$ trong nước sẽ bị thủy phân tạo môi trường acid.
					\item Dung dịch $AlCl_3$ phản ứng với $NH_3$ (bazơ yếu) và $H_2O$ để tạo ra kết tủa $Al(OH)_3$:
					$3\text{NH}_3\text{ (aq)} + \text{AlCl}_3\text{ (aq)} + 3\text{H}_2\text{O}\text{ (l)} \rightarrow \text{Al(OH)}_3 \downarrow + 3\text{NH}_4\text{Cl}\text{ (aq)}$
					\item Phản ứng này tiêu thụ $NH_3$, làm cho cân bằng $NH_3 + H_2O \xharpoonarrow NH_4^+ + OH^-$ dịch chuyển theo chiều thuận (từ trái sang phải).
					\item Nồng độ $OH^-$ trong dung dịch sẽ giảm dần.
					\item Do nồng độ $OH^-$ giảm, màu hồng của phenolphtalein sẽ nhạt dần rồi mất màu.
					\item Đồng thời, sẽ xuất hiện kết tủa keo trắng $Al(OH)_3$.
				\end{itemize}
			\end{enumerate}
			}
		\end{bt}
		%%%=============BT_5=============%%%
		\begin{bt}
			Muối $NH_4NO_3$ sẽ nhiệt phân theo phản ứng nào trong $2$ phản ứng sau? Giải thích.
			\[
			\text{NH}_4\text{NO}_3\text{(s)} \xrightarrow[$t^\circ$] \text{NH}_3\text{(g)} + \text{HNO}_3\text{(g)} \quad \text{(1)}
			\]
			\[
			\text{NH}_4\text{NO}_3\text{(s)} \xrightarrow[$t^\circ$] \text{N}_2\text{O}\text{(g)} + 2\text{H}_2\text{O}\text{(g)} \quad \text{(2)}
			\]
			Biết enthalpy tạo thành chuẩn của các chất có giá trị như sau:
			\begin{center}
				\begin{tabular}{|l|c|c|c|c|c|}
					\hline
					Chất & $\text{NH}_4\text{NO}_3\text{(s)}$ & $\text{NH}_3\text{(g)}$ & $\text{N}_2\text{O}\text{(g)}$ & $\text{HNO}_3\text{(g)}$ & $\text{H}_2\text{O}\text{(g)}$ \\
					\hline
					$\Delta_fH^\circ_{298}$ (kJ/mol) & $-365{,}61$ & $-45{,}90$ & $82{,}05$ & $-134{,}31$ & $-241{,}82$ \\
					\hline
				\end{tabular}
			\end{center}
			\loigiai{
				Để xác định phản ứng nhiệt phân nào của $NH_4NO_3$ diễn ra, chúng ta cần tính biến thiên enthalpy chuẩn ($\Delta_rH^\circ$) cho mỗi phản ứng. Phản ứng có $\Delta_rH^\circ$ âm (tỏa nhiệt) sẽ thuận lợi hơn về mặt nhiệt động lực học để xảy ra.
				
				Công thức tính $\Delta_rH^\circ$:
				\[
				\Delta_rH^\circ = \sum \Delta_fH^\circ_{\text{sản phẩm}} - \sum \Delta_fH^\circ_{\text{chất phản ứng}}
				\]
				
				\textbf{1. Tính $\Delta_rH^\circ$ cho phản ứng (1):}
				$\text{NH}_4\text{NO}_3\text{(s)} \xrightarrow \text{NH}_3\text{(g)} + \text{HNO}_3\text{(g)}$
				\begin{align*}
					\Delta_rH^\circ_{(1)} &= [\Delta_fH^\circ(\text{NH}_3\text{(g)}) + \Delta_fH^\circ(\text{HNO}_3\text{(g)})] - [\Delta_fH^\circ(\text{NH}_4\text{NO}_3\text{(s)})] \\
					\Delta_rH^\circ_{(1)} &= [(-45{,}90) + (-134{,}31)] - (-365{,}61) \\
					\Delta_rH^\circ_{(1)} &= -180{,}21 + 365{,}61 \\
					\Delta_rH^\circ_{(1)} &= 185{,}4 \text{ kJ}
				\end{align*}
				Phản ứng (1) là phản ứng thu nhiệt ($\Delta_rH^\circ > 0$).
				
				\textbf{2. Tính $\Delta_rH^\circ$ cho phản ứng (2):}
				$\text{NH}_4\text{NO}_3\text{(s)} \rightarrow \text{N}_2\text{O}\text{(g)} + 2\text{H}_2\text{O}\text{(g)}$
				\begin{align*}
					\Delta_rH^\circ_{(2)} &= [\Delta_fH^\circ(\text{N}_2\text{O}\text{(g)}) + 2 \times \Delta_fH^\circ(\text{H}_2\text{O}\text{(g)})] - [\Delta_fH^\circ(\text{NH}_4\text{NO}_3\text{(s)})] \\
					\Delta_rH^\circ_{(2)} &= [(82{,}05) + 2 \times (-241{,}82)] - (-365{,}61) \\
					\Delta_rH^\circ_{(2)} &= [82{,}05 - 483{,}64] + 365{,}61 \\
					\Delta_rH^\circ_{(2)} &= -401{,}59 + 365{,}61 \\
					\Delta_rH^\circ_{(2)} &= -35{,}98 \text{ kJ}
				\end{align*}

				Phản ứng (2) là phản ứng tỏa nhiệt ($\Delta_rH^\circ < 0$).
				
				\textbf{Kết luận và giải thích:}
				Phản ứng (2) có $\Delta_rH^\circ = -35{,}98$ kJ là phản ứng tỏa nhiệt, trong khi phản ứng (1) có $\Delta_rH^\circ = 185{,}4$ kJ là phản ứng thu nhiệt.
				Về mặt nhiệt động lực học, phản ứng tỏa nhiệt (giải phóng năng lượng) thường thuận lợi hơn để xảy ra và có khả năng tự phát hơn so với phản ứng thu nhiệt.
				Do đó, \textbf{phản ứng (2)} là phản ứng nhiệt phân chính của $NH_4NO_3$ ở nhiệt độ vừa phải (khoảng $200^\circ C - 260^\circ C$). Phản ứng này là một phản ứng oxi hóa-khử nội phân tử.
				
				(Lưu ý: Nếu nhiệt độ rất cao (trên $300^\circ C$), $NH_4NO_3$ có thể phân hủy mạnh hơn tạo ra $N_2$, $O_2$ và $H_2O$, hoặc thậm chí gây nổ. Phản ứng (1) chỉ xảy ra trong điều kiện đặc biệt hoặc khi $NH_4NO_3$ ở trạng thái khí).
			}
		\end{bt}
		%%%=============BT_6=============%%%
		\begin{bt}
			Hiện nay người ta sản xuất ammonia bằng cách chuyển hóa có xúc tác một hỗn hợp gồm không khí, hơi nước và khí methane (thành phần chính của khí thiên nhiên).
			\\
			Các phản ứng hóa học chính diễn ra là:
			\begin{enumerate}
				\item Phản ứng điều chế $H_2$: $\text{CH}_4\text{ (g)} + 2\text{H}_2\text{O}\text{ (g)} \xrightarrow[$t^\circ, xt$] \text{CO}_2\text{ (g)} + 4\text{H}_2\text{ (g)}$
				\item Phản ứng loại $O_2$ để thu $N_2$: $\text{CH}_4\text{ (g)} + 2\text{O}_2\text{ (g)} \xrightarrow[$t^\circ$] \text{CO}_2\text{ (g)} + 2\text{H}_2\text{O}\text{ (g)}$
				\item Phản ứng tổng hợp $NH_3$: $\text{N}_2\text{ (g)} + 3\text{H}_2\text{ (g)} \xrightarrow[$t^\circ, xt, p$] 2\text{NH}_3\text{ (g)}$
			\end{enumerate}
			Để sản xuất khí ammonia, nếu lấy $841{,}7$ m³ không khí (chứa $21{,}03\%$ $O_2$; $78{,}02\%$ $N_2$, còn lại là khí hiếm theo thể tích), thì cần phải lấy bao nhiêu $m^3$ khí methane và bao nhiêu m³ hơi nước để có đủ lượng $N_2$ và $H_2$ theo tỉ lệ $1:3$ về thể tích dùng cho phản ứng tổng hợp ammonia. Giả thiết các phản ứng (1), (2) đều xảy ra hoàn toàn và các thể tích khí đo ở cùng điều kiện.
			\loigiai{
				Vì các thể tích khí đo ở cùng điều kiện, tỉ lệ thể tích cũng chính là tỉ lệ mol.
				\begin{enumerate}
					\item \textbf{Tính thể tích $N_2$ và $O_2$ có trong $841{,}7$ m³ không khí:}
					\begin{itemize}
						\item Thể tích $N_2$ trong không khí:
						$V_{N_2 \text{ trong KK}} = 841{,}7 \text{ m}^3 \times 78{,}02\% = 841{,}7 \times 0{,}7802 = 656{,}70 \text{ m}^3$
						\item Thể tích $O_2$ trong không khí:
						$V_{O_2 \text{ trong KK}} = 841{,}7 \text{ m}^3 \times 21{,}03\% = 841{,}7 \times 0{,}2103 = 176{,}99 \text{ m}^3$
					\end{itemize}
					
					\item \textbf{Tính thể tích $H_2$ cần thiết cho phản ứng tổng hợp $NH_3$ (Phản ứng 3):}
					\\
					Phản ứng (3): $\text{N}_2\text{ (g)} + 3\text{H}_2\text{ (g)} \rightarrow 2\text{NH}_3\text{ (g)}$
					\\
					Để có đủ $N_2$ và $H_2$ theo tỉ lệ $1:3$ cho phản ứng (3), và $N_2$ được lấy từ không khí, thì thể tích $N_2$ cần cho phản ứng (3) chính là $V_{N_2 \text{ trong KK}}$.
					\\
					$V_{N_2 \text{ cho PƯ 3}} = 656{,}70 \text{ m}^3$
					\\
					Từ phương trình phản ứng (3), cứ $1$ thể tích $N_2$ cần $3$ thể tích $H_2$.
					\\
					$V_{H_2 \text{ cần cho PƯ 3}} = 3 \times V_{N_2 \text{ cho PƯ 3}} = 3 \times 656{,}70 = 1970{,}10 \text{ m}^3$
					
					\item \textbf{Tính thể tích $CH_4$ và $H_2O$ cần cho quá trình sản xuất:}
					\begin{itemize}
						\item \textbf{Thể tích $CH_4$ cần để loại bỏ $O_2$ (Phản ứng 2):}
						\\
						Phản ứng (2): $\text{CH}_4\text{ (g)} + 2\text{O}_2\text{ (g)} \xrightarrow[$t^\circ$] \text{CO}_2\text{ (g)} + 2\text{H}_2\text{O}\text{ (g)}$
						\\
						Cứ $1$ thể tích $CH_4$ phản ứng với $2$ thể tích $O_2$. Thể tích $O_2$ cần loại bỏ là $V_{O_2 \text{ trong KK}}$.
						\\
						$V_{CH_4 \text{ để loại } O_2} = \dfrac{1}{2} \times V_{O_2 \text{ trong KK}} = \dfrac{1}{2} \times 176{,}99 = 88{,}495 \text{ m}^3$
						\\
						Lưu ý: Phản ứng này cũng tạo ra $H_2O$ với thể tích bằng $V_{O_2 \text{ trong KK}}$. $V_{H_2O \text{ sinh ra từ PƯ 2}} = 176{,}99 \text{ m}^3$.
						
						\item \textbf{Thể tích $CH_4$ và $H_2O$ cần để điều chế $H_2$ (Phản ứng 1):}
						\\
						Phản ứng (1): $\text{CH}_4\text{ (g)} + 2\text{H}_2\text{O}\text{ (g)} \xrightarrow[$t^\circ, xt$] \text{CO}_2\text{ (g)} + 4\text{H}_2\text{ (g)}$
						\\
						Để thu được $1970{,}10$ m³ $H_2$, ta cần:
						\\
						$V_{CH_4 \text{ để điều chế } H_2} = \dfrac{1}{4} \times V_{H_2 \text{ cần cho PƯ 3}} = \dfrac{1}{4} \times 1970{,}10 = 492{,}525 \text{ m}^3$
						\\
						$V_{H_2O \text{ cần cho PƯ 1}} = \dfrac{2}{1} \times V_{CH_4 \text{ để điều chế } H_2} = 2 \times 492{,}525 = 985{,}05 \text{ m}^3$
					\end{itemize}
					\item \textbf{Tính tổng thể tích $CH_4$ và $H_2O$ cần lấy:}
					\begin{itemize}
						\item \textbf{Tổng thể tích $CH_4$ cần lấy:} Là tổng $CH_4$ dùng cho phản ứng (1) và (2).
						\\
						$V_{CH_4 \text{ tổng}} = V_{CH_4 \text{ để điều chế } H_2} + V_{CH_4 \text{ để loại } O_2} \\
						V_{CH_4 \text{ tổng}} = 492{,}525 + 88{,}495 = 581{,}02 \text{ m}^3$
						\\
						Làm tròn đến hàng phần mười: $581{,}0 \text{ m}^3$.
						
						\item \textbf{Tổng thể tích $H_2O$ cần lấy:} Là $H_2O$ cần cho phản ứng (1) trừ đi $H_2O$ được sinh ra từ phản ứng (2) (nếu $H_2O$ sinh ra từ phản ứng (2) được tái sử dụng). Nếu không có thông tin tái sử dụng, ta xem đây là lượng $H_2O$ cần bổ sung vào hệ.
						\\
						$V_{H_2O \text{ tổng}} = V_{H_2O \text{ cần cho PƯ 1}} - V_{H_2O \text{ sinh ra từ PƯ 2}} \\
						V_{H_2O \text{ tổng}} = 985{,}05 - 176{,}99 = 808{,}06 \text{ m}^3$
						\\
						Làm tròn đến hàng phần mười: $808{,}1 \text{ m}^3$.
					\end{itemize}
				\end{enumerate}
				\textbf{Đáp số:}
				\begin{itemize}
					\item Thể tích khí methane cần lấy là $581{,}0$ m³.
					\item Thể tích hơi nước cần lấy là $808{,}1$ m³.
				\end{itemize}
			}
		\end{bt}
		%%%=============BT_7=============%%%
		\begin{bt}
			Hợp chất có công thức hoá học $NH_4NO_3$ được giới chức quốc gia Lebanon xác định là nguyên nhân gây ra vụ nổ thảm khốc ở thủ đô Beirut vào ngày 04/08/2020. Tia lửa hàn trong quá trình sửa chữa nhà kho có thể đã làm $2\,750$ tấn $NH_4NO_3$ cất trữ phát nổ, phá huỷ nhiều nhà cửa, dẫn đến nhiều người thiệt mạng. Hãy giải thích vì sao $NH_4NO_3$ có khả năng phát nổ.
			\loigiai{
				Ammonium nitrate ($NH_4NO_3$) là một hợp chất ion, trong đó cation $NH_4^+$ có tính khử (do $N^{-3}$) và anion $NO_3^-$ có tính oxi hoá (do $N^{+5}$). Do đó, $NH_4NO_3$ có thể tự oxi hoá - khử khi có điều kiện thích hợp (nhiệt độ cao, va đập mạnh).
				\\
				Khi bị nung nóng, $NH_4NO_3$ phân huỷ nhiệt sinh ra một lượng lớn khí và toả nhiệt mạnh:
				\[ 2\mathrm{NH}_4\mathrm{NO}_3(s) \xrightarrow[$t^\circ > 200^\circ C$] 2\mathrm{N}_2(g) + \mathrm{O}_2(g) + 4\mathrm{H}_2\mathrm{O}(g) \]
				Phản ứng này xảy ra rất nhanh, tạo ra một thể tích khí khổng lồ trong một thời gian ngắn. Sự giãn nở đột ngột của các sản phẩm khí tạo ra sóng xung kích cực mạnh, gây ra vụ nổ.
			}
		\end{bt}
		%%%=============BT_8=============%%%
		\begin{bt}[s]
			Một lượng lớn ammonium ion trong nước rác thải sinh ra khi vứt bỏ vào ao hồ được vi khuẩn oxi hoá thành nitrate và quá trình đó làm giảm oxygen hoà tan trong nước gây ngạt cho sinh vật sống dưới nước. Người ta có thể xử lí nguồn gây ô nhiễm đó bằng nước vôi trong (dung dịch $Ca(OH)_2$) và khí chlorine để chuyển ammonium ion thành ammonia rồi chuyển tiếp thành nitrogen không độc thải ra môi trường. Giải thích cách làm này bằng phương trình hoá học.
			\loigiai{
				Quá trình xử lí ammonium ion ($NH_4^+$) trong nước thải diễn ra qua các giai đoạn sau:
				\begin{enumerate}[-]
					\item \textbf{Giai đoạn 1: Chuyển hoá $NH_4^+$ thành $NH_3$.} \\
					Nước vôi trong, dung dịch $Ca(OH)_2$, là một base. Nó cung cấp ion $OH^-$ để phản ứng với ion $NH_4^+$, chuyển nó về dạng phân tử $NH_3$.
					\[ \mathrm{Ca(OH)}_2 \xrightarrow \mathrm{Ca}^{2+} + 2\mathrm{OH}^- \]
					\[ \mathrm{NH}_4^+ + \mathrm{OH}^- \xrightarrow \mathrm{NH}_3 \uparrow + \mathrm{H}_2\mathrm{O} \]
					
					\item \textbf{Giai đoạn 2: Oxi hoá $NH_3$ thành $N_2$.} \\
					Khí chlorine ($Cl_2$) được sục vào để oxi hoá $NH_3$ thành khí nitrogen ($N_2$), một chất không độc và bền vững, giải phóng ra môi trường. Phản ứng này cũng sinh ra $HCl$.
					\[ 2\mathrm{NH}_3 + 3\mathrm{Cl}_2 \xrightarrow \mathrm{N}_2 \uparrow + 6\mathrm{HCl} \]
					
					\item \textbf{Giai đoạn 3: Trung hoà axit.} \\
					Axit $HCl$ sinh ra ở giai đoạn 2 sẽ được trung hoà ngay lập tức bởi lượng $Ca(OH)_2$ dư có sẵn trong dung dịch.
					\[ 2\mathrm{HCl} + \mathrm{Ca(OH)}_2 \xrightarrow \mathrm{CaCl}_2 + 2\mathrm{H}_2\mathrm{O} \]
				\end{enumerate}
				Nhờ chuỗi phản ứng này, ion $NH_4^+$ độc hại được chuyển hoá thành khí $N_2$ an toàn.
			}
		\end{bt}
	%%%=============BT_9=============%%%
	\begin{bt}
		\immini{Sự phụ thuộc của độ tan khí ammonia ($NH_3$) trong nước vào nhiệt độ được mô tả ở đồ thị bên.
		Dựa vào đồ thị, hãy xác định:
		\begin{enumerate}[a)]
			\item Độ tan của ammonia ở $30~^\circ\text{C}$. Nhận xét về tính tan của ammonia ở nhiệt độ này.
			\item Nồng độ phần trăm của dung dịch ammonia bão hoà ở $30~^\circ\text{C}$.
			\item Độ tan của ammonia ở $60~^\circ\text{C}$. So sánh với độ tan của ammonia ở $30~^\circ\text{C}$ và giải thích.
		\end{enumerate}}{%
			\begin{tikzpicture}[declare function={d=1;}]
			\pgfmathsetmacro{\incx}{20}
			\pgfmathsetmacro{\incy}{20}
			%%Vẽ trục lưới
			\foreach \x [evaluate= \x as \xt using int(\incx*d*\x)]in {0.5,1,1.5,2,...,5}{
				\draw [\maunhan!50,ultra thin] (0,\x)--(5,\x);
				\path (\x,0) node [below, font=\scriptsize] {\xt};
			}
			\foreach \y [evaluate= \y as \yt using int(\incy*d*\y)]in {0.5,1,1.5,2,...,5}{
				\draw [\maunhan!50,ultra thin] (\y,0)--(\y,5);
				\path (0,\y) node [left, font=\scriptsize] {\yt};
			}
			%%Vẽ hai trục tọa độ
			\draw[->, >=stealth,ultra thick] (0,0) --(5.5,0);
			\draw[->, >=stealth,ultra thick] (0,0) --(0,5.5);
			\draw [\maunhan,ultra thick] (0,4.5) ..controls ++(-70:2.0) and ++(170:0) ..(2,1.5)
			.. controls ++(-10:0) and ++(170:2) .. (5,0);
			\path (0,0)--++(-135:10pt) node[font =\scriptsize] {0};
			%%%Node thông tin
			\path (0,0)--(0,5) node [sloped,above =0.5cm, font =\scriptsize\sffamily\bfseries, pos=0.5] {Độ tan g$\text{NH}_\text{3}$/100g$\text{H}_\text{2}\text{O}$};
			\path (0,0)--(5,0) node (text) [sloped,below =0.3cm, font =\scriptsize\sffamily\bfseries, pos=0.5] {Nhiệt độ ($^\circ C$)};
			\path (text.south) node [text width=7cm,font=\scriptsize\sffamily\bfseries,align =center]{Sự phụ thuộc của khí ammonia vào nhiệt độ};
		\end{tikzpicture}}
		\loigiai{
			Dựa vào đồ thị ta có:
			\begin{enumerate}[a)]
				\item Tại nhiệt độ $30~^\circ\text{C}$ (trục hoành), gióng sang trục tung ta xác định được độ tan của ammonia là $50$ g $NH_3$/$100$ g $H_2O$.
				\\
				\textbf{Nhận xét:} Ở $30~^\circ\text{C}$, ammonia tan rất tốt trong nước, vì trong $100$ g nước có thể hòa tan tới $50$ g khí ammonia.
				
				\item Ở $30~^\circ\text{C}$, dung dịch bão hòa có $50$ g $NH_3$ (chất tan) và $100$ g $H_2O$ (dung môi).
				\\
				Khối lượng dung dịch bão hòa là:
				\[ m_{\text{dd}} = m_{\text{chất tan}} + m_{\text{dung môi}} = 50 + 100 = 150 \text{ (g)} \]
				Nồng độ phần trăm của dung dịch ammonia bão hòa ở $30~^\circ\text{C}$ là:
				\[ C\% = \frac{m_{\text{chất tan}}}{m_{\text{dd}}} \times 100\% = \frac{50}{150} \times 100\% \approx 33{,}33\% \]
				
				\item Tại nhiệt độ $60~^\circ\text{C}$ (trục hoành), gióng sang trục tung ta xác định được độ tan của ammonia là $20$ g $NH_3$/$100$ g $H_2O$.
				\\
				\textbf{So sánh:} Độ tan của ammonia ở $60~^\circ\text{C}$ ($20$ g) nhỏ hơn độ tan ở $30~^\circ\text{C}$ ($50$ g).
				\\
				\textbf{Giải thích:} Quá trình hòa tan của hầu hết các chất khí trong nước là quá trình tỏa nhiệt. Theo nguyên lí chuyển dịch cân bằng Le Chatelier, khi tăng nhiệt độ, cân bằng sẽ chuyển dịch theo chiều chống lại sự tăng nhiệt độ, tức là chiều thu nhiệt (chiều nghịch, làm giảm độ tan của khí). Do đó, khi nhiệt độ tăng, độ tan của khí ammonia trong nước giảm.
			\end{enumerate}
			}
		\end{bt}
		%%%=============BT_10=============%%%
		\begin{bt}
			Hoàn thành chuỗi phản ứng sau bằng cách viết các phương trình hóa học (ghi rõ điều kiện nếu có):
			\[
			\text{N}_2 \xrightarrow[$(1)$] \text{NH}_3 \xrightarrow[$(2)$] \text{NH}_4\text{Cl} \xrightarrow[$(3)$] \text{NH}_3 \xrightarrow[$(4)$] (\text{NH}_4)_2\text{SO}_4
			\]
			\loigiai{
				\begin{enumerate}[(1)]
					\item \textbf{Tổng hợp Ammonia:}
					$\text{N}_2\text{ (g)} + 3\text{H}_2\text{ (g)} \xrightarrow[$400-600^\circ C$][$150-200 \text{ bar, Fe}$] 2\text{NH}_3\text{ (g)}$
					\item \textbf{Ammonia tác dụng với Acid clohidric:}
					$\text{NH}_3\text{ (g)} + \text{HCl}\text{ (g)} \rightarrow \text{NH}_4\text{Cl}\text{ (s)}$
					(Hoặc $NH_3\text{ (aq)} + HCl\text{ (aq)} \rightarrow NH_4Cl\text{ (aq)}$)
					\item \textbf{Nhiệt phân Amoni clorua (hoặc tác dụng với bazơ):}
					$\text{NH}_4\text{Cl}\text{ (s)} \xrightarrow[$t^\circ$] \text{NH}_3\text{ (g)} + \text{HCl}\text{ (g)}$
					(Hoặc dùng bazơ mạnh để thu $NH_3$ ở nhiệt độ thường/đun nóng):
					$\text{NH}_4\text{Cl}\text{ (aq)} + \text{NaOH}\text{ (aq)} \xrightarrow[$t^\circ$] \text{NaCl}\text{ (aq)} + \text{NH}_3\uparrow + \text{H}_2\text{O}\text{ (l)}$
					\item \textbf{Ammonia tác dụng với Acid sulfuric:}
					$2\text{NH}_3\text{ (g)} + \text{H}_2\text{SO}_4\text{ (aq)} \rightarrow (\text{NH}_4)_2\text{SO}_4\text{ (aq)}$
				\end{enumerate}
			}
		\end{bt}
		
		%%%=============BT_11=============%%%
		\begin{bt}
			Hoàn thành chuỗi phản ứng sau bằng cách viết các phương trình hóa học (ghi rõ điều kiện nếu có):
			\[
			\text{NH}_3 \xrightarrow[$(1)$] \text{NO} \xrightarrow[$(2)$] \text{NO}_2 \xrightarrow[$(3)$] \text{HNO}_3 \xrightarrow[$(4)$] \text{NH}_4\text{NO}_3 \xrightarrow[$(5)$] \text{N}_2\text{O}
			\]
			\loigiai{
				\begin{enumerate}[(1)]
					\item \textbf{Oxi hóa Ammonia có xúc tác:}
					$4\text{NH}_3\text{ (g)} + 5\text{O}_2\text{ (g)} \xrightarrow[$Pt/Rh, t^\circ$] 4\text{NO}\text{ (g)} + 6\text{H}_2\text{O}\text{ (g)}$
					\item \textbf{NO tác dụng với Oxygen:}
					$2\text{NO}\text{ (g)} + \text{O}_2\text{ (g)} \rightarrow 2\text{NO}_2\text{ (g)}$
					\item \textbf{NO$_2$ tác dụng với nước và Oxygen:}
					$4\text{NO}_2\text{ (g)} + 2\text{H}_2\text{O}\text{ (l)} + \text{O}_2\text{ (g)} \rightarrow 4\text{HNO}_3\text{ (aq)}$
					\item \textbf{Acid nitric tác dụng với Ammonia:}
					$\text{HNO}_3\text{ (aq)} + \text{NH}_3\text{ (g)} \rightarrow \text{NH}_4\text{NO}_3\text{ (aq)}$
					\item \textbf{Nhiệt phân Amoni nitrat:}
					$\text{NH}_4\text{NO}_3\text{ (s)} \xrightarrow[$t^\circ$] \text{N}_2\text{O}\text{ (g)} + 2\text{H}_2\text{O}\text{ (g)}$
				\end{enumerate}
			}
		\end{bt}
		
		%%%=============BT_12=============%%%
		\begin{bt}
			Hoàn thành chuỗi phản ứng sau bằng cách viết các phương trình hóa học (ghi rõ điều kiện nếu có):
			\[
			\text{NH}_4\text{Cl} \xrightarrow[$(1)$] \text{NH}_3 \xrightarrow[$(2)$] \text{Cu(OH)}_2 \xrightarrow[$(3)$]\left[\text{Cu(NH}_3)_4\right](\text{OH})_2 \xrightarrow[$(4)$] \text{NH}_4\text{NO}_2 \xrightarrow[$(5)$] \text{N}_2
			\]
			\loigiai{
				\begin{enumerate}[(1)]
					\item \textbf{Điều chế Ammonia từ muối amoni (trong phòng thí nghiệm):}
					$\text{NH}_4\text{Cl}\text{ (s)} + \text{NaOH}\text{ (s)} \xrightarrow[$t^\circ$] \text{NaCl}\text{ (s)} + \text{NH}_3\uparrow + \text{H}_2\text{O}\text{ (g)}$
					(Hoặc nhiệt phân $NH_4Cl$ nếu không dùng $NaOH$)
					\item \textbf{Tạo Cu(OH)$_2$ từ $NH_3$ và muối Cu(II):}
					$2\text{NH}_3\text{ (aq)} + \text{CuSO}_4\text{ (aq)} + 2\text{H}_2\text{O}\text{ (l)} \rightarrow \text{Cu(OH)}_2 \downarrow + (\text{NH}_4)_2\text{SO}_4\text{ (aq)}$
					(Hoặc dùng $CuCl_2$ tương tự)
					\item \textbf{Cu(OH)$_2$ tác dụng với $NH_3$ tạo phức tan:}
					$\text{Cu(OH)}_2\text{ (s)} + 4\text{NH}_3\text{ (aq)} \rightarrow [\text{Cu(NH}_3)_4](\text{OH})_2\text{ (aq)}$
					(Phức tetraamminedồng(II) hiđroxit, dung dịch màu xanh đậm/xanh thẫm)
					\item \textbf{Điều chế Amoni nitrit:} (Từ $NH_3$ hoặc muối amoni)
					$\text{NH}_3\text{ (g)} + \text{HNO}_2\text{ (aq)} \rightarrow \text{NH}_4\text{NO}_2\text{ (aq)}$
					(Hoặc từ muối amoni: $\text{NH}_4\text{Cl}\text{ (aq)} + \text{NaNO}_2\text{ (aq)} \rightarrow \text{NH}_4\text{NO}_2\text{ (aq)} + \text{NaCl}\text{ (aq)}$)
					\item \textbf{Nhiệt phân Amoni nitrit (điều chế $N_2$ trong phòng thí nghiệm):}
					$\text{NH}_4\text{NO}_2\text{ (s)} \xrightarrow[$t^\circ$] \text{N}_2\text{ (g)} + 2\text{H}_2\text{O}\text{ (g)}$
				\end{enumerate}
			}
		\end{bt}
		
		%%%=============BT_13=============%%%
		\begin{bt}
			Trong phòng thí nghiệm, sau khi điều chế được một khí X, người ta nghi ngờ đó là khí ammonia ($NH_3$). Trình bày cách tiến hành thí nghiệm để xác nhận khí X có phải là $NH_3$ hay không. Nêu hiện tượng quan sát được và viết các phương trình hóa học minh họa (nếu có).
			\loigiai{
				Để xác nhận khí X có phải là khí ammonia ($NH_3$), ta có thể tiến hành các thí nghiệm sau:
				\begin{enumerate}
					\item \textbf{Thử bằng mùi:}
					\begin{itemize}
						\item \textbf{Cách tiến hành:} Đưa nhẹ miệng ống nghiệm (hoặc bình chứa khí X) lại gần mũi và hít một hơi nhẹ.
						\item \textbf{Hiện tượng:} Nếu là khí $NH_3$, sẽ ngửi thấy mùi khai đặc trưng.
					\end{itemize}
					\item \textbf{Thử bằng quỳ tím ẩm:}
					\begin{itemize}
						\item \textbf{Cách tiến hành:} Dùng một mẩu giấy quỳ tím ẩm (hoặc giấy pH) đưa vào miệng ống nghiệm (hoặc bình chứa khí X).
						\item \textbf{Hiện tượng:} Nếu là khí $NH_3$, giấy quỳ tím ẩm sẽ chuyển sang màu xanh (hoặc giấy pH cho giá trị pH > 7).
						\item \textbf{Giải thích và phương trình hóa học:} Khí $NH_3$ tan trong lớp nước trên giấy quỳ tạo thành dung dịch amoniac, là một bazơ yếu, phân li ra ion $OH^-$, làm quỳ tím hóa xanh.
						$\text{NH}_3\text{ (g)} + \text{H}_2\text{O}\text{ (l)} \rightleftharpoons \text{NH}_4^+\text{ (aq)} + \text{OH}^-\text{ (aq)}$
					\end{itemize}
					\item \textbf{Thử bằng đũa thủy tinh có tẩm dung dịch acid clohidric đặc:}
					\begin{itemize}
						\item \textbf{Cách tiến hành:} Dùng một đũa thủy tinh nhúng vào dung dịch $HCl$ đặc, sau đó đưa đũa lại gần miệng ống nghiệm (hoặc bình chứa khí X).
						\item \textbf{Hiện tượng:} Nếu là khí $NH_3$, sẽ xuất hiện khói trắng dày đặc.
						\item \textbf{Giải thích và phương trình hóa học:} Khí $NH_3$ bay ra sẽ phản ứng với khí $HCl$ (từ dung dịch $HCl$ đặc bay hơi) tạo thành các hạt tinh thể amoni clorua ($NH_4Cl$) rất nhỏ, tạo thành khói trắng.
						$\text{NH}_3\text{ (g)} + \text{HCl}\text{ (g)} \rightarrow \text{NH}_4\text{Cl}\text{ (s)}$ (khói trắng)
					\end{itemize}
					Kết luận: Nếu quan sát được cả ba hiện tượng trên, có thể khẳng định khí X là ammonia ($NH_3$).
				\end{enumerate}
			}
		\end{bt}
		
		%%%=============BT_14=============%%%
		\begin{bt}
			Có bốn lọ không nhãn, mỗi lọ chứa một trong các dung dịch không màu sau: $NaCl$, $NaOH$, $NH_4NO_3$, $HCl$. Trình bày phương pháp hóa học để nhận biết từng dung dịch trong mỗi lọ. Viết các phương trình hóa học xảy ra (nếu có).
			\loigiai{
				\begin{enumerate}
					\item \textbf{Bước 1: Dùng quỳ tím hoặc giấy pH để phân loại dung dịch.}
					\begin{itemize}
						\item \textbf{Cách tiến hành:} Lấy một ít mỗi dung dịch cho vào các ống nghiệm riêng biệt đã đánh số. Nhúng giấy quỳ tím (hoặc giấy pH) vào từng ống nghiệm.
						\item \textbf{Hiện tượng và kết luận:}
						\begin{itemize}
							\item Quỳ tím hóa xanh (hoặc pH > 7): Dung dịch $NaOH$ (bazơ mạnh) và dung dịch $NH_4NO_3$ (có thể có tính acid yếu do ion $NH_4^+$ nhưng $NH_4NO_3$ tan trong nước tạo dung dịch gần trung tính hoặc hơi acid, tuy nhiên để phân biệt rõ thì cần bước tiếp theo).
							\item Quỳ tím hóa đỏ (hoặc pH < 7): Dung dịch $HCl$. $\rightarrow$ \textbf{Nhận biết được $HCl$}.
							\item Quỳ tím không đổi màu (hoặc pH $\approx 7$): Dung dịch $NaCl$. $\rightarrow$ \textbf{Nhận biết được $NaCl$}.
						\end{itemize}
					\end{itemize}
					\item \textbf{Bước 2: Nhận biết $NaOH$ và $NH_4NO_3$ bằng cách đun nóng với dung dịch kiềm (nếu chưa chắc chắn từ quỳ tím) hoặc trực tiếp thêm kiềm.}
					\begin{itemize}
						\item \textbf{Cách tiến hành:} Lấy một ít hai dung dịch còn lại (NaOH và NH4NO3) vào hai ống nghiệm riêng biệt. Thêm vào mỗi ống nghiệm vài giọt dung dịch $NaOH$ (nếu lọ không nhãn của $NaOH$ đã được nhận biết ở bước 1 thì có thể dùng $NaOH$ đó, hoặc dùng một bazơ mạnh khác như $KOH$). Đun nóng nhẹ cả hai ống nghiệm.
						\item \textbf{Hiện tượng và kết luận:}
						\begin{itemize}
							\item Ống nghiệm nào có khí mùi khai ($NH_3$) thoát ra (làm xanh giấy quỳ ẩm hoặc ngửi thấy mùi khai): Đó là dung dịch $NH_4NO_3$. $\rightarrow$ \textbf{Nhận biết được $NH_4NO_3$}.
							\item Ống nghiệm còn lại không có hiện tượng gì đặc biệt (hoặc không có khí mùi khai thoát ra): Đó là dung dịch $NaOH$. $\rightarrow$ \textbf{Nhận biết được $NaOH$}.
						\end{itemize}
						\item \textbf{Phương trình hóa học:}
						$\text{NH}_4\text{NO}_3\text{ (aq)} + \text{NaOH}\text{ (aq)} \xrightarrow[$t^\circ$] \text{NaNO}_3\text{ (aq)} + \text{NH}_3\uparrow + \text{H}_2\text{O}\text{ (l)}$
					\end{itemize}
				\end{enumerate}
				\textbf{Tóm tắt kết quả:}
				\begin{itemize}
					\item Dùng quỳ tím: $HCl$ (đỏ), $NaCl$ (không đổi màu).
					\item Dùng $NaOH$ và đun nóng: $NH_4NO_3$ (khí mùi khai), $NaOH$ (không khí mùi khai).
				\end{itemize}
			}
		\end{bt}
		
		%%%=============BT_15=============%%%
		\begin{bt}
			Trình bày cách phân biệt hai muối dạng bột màu trắng là amoni clorua ($NH_4Cl$) và amoni nitrat ($NH_4NO_3$). Viết các phương trình hóa học xảy ra (nếu có).
			\loigiai{
				Để phân biệt hai muối $NH_4Cl$ và $NH_4NO_3$, ta có thể dựa vào sản phẩm khí khác nhau khi chúng bị nhiệt phân.
				\begin{enumerate}
					\item \textbf{Bước 1: Nhiệt phân từng muối.}
					\begin{itemize}
						\item \textbf{Cách tiến hành:} Lấy một lượng nhỏ mỗi muối (dạng bột) cho vào hai ống nghiệm khô, riêng biệt và đánh số. Đun nóng nhẹ từng ống nghiệm trên ngọn lửa đèn cồn.
						\item \textbf{Hiện tượng và phương trình hóa học:}
						\begin{itemize}
							\item \textbf{Ống nghiệm 1 (chứa $NH_4Cl$):}
							\[
							\text{NH}_4\text{Cl}\text{ (s)} \xrightarrow[$t^\circ$] \text{NH}_3\text{ (g)} + \text{HCl}\text{ (g)}
							\]
							$NH_4Cl$ bị phân hủy thành khí $NH_3$ (mùi khai) và khí $HCl$ (mùi xốc). Khi khí bay ra khỏi vùng nóng và gặp lạnh, chúng sẽ kết hợp lại tạo thành khói trắng $NH_4Cl$ (do $NH_3$ và $HCl$ phản ứng thuận nghịch). Nếu đưa giấy quỳ ẩm vào miệng ống nghiệm, quỳ tím sẽ hóa đỏ (do $HCl$) và sau đó có thể xanh trở lại (do $NH_3$ dư hoặc pha loãng).
							\item \textbf{Ống nghiệm 2 (chứa $NH_4NO_3$):}
							\[
							\text{NH}_4\text{NO}_3\text{ (s)} \xrightarrow[$t^\circ$] \text{N}_2\text{O}\text{ (g)} + 2\text{H}_2\text{O}\text{ (g)}
							\]
							$NH_4NO_3$ bị phân hủy tạo ra khí dinitrogen oxit ($N_2O$, khí cười, không màu, không mùi khai) và hơi nước. Không tạo khói trắng như $NH_4Cl$.
						\end{itemize}
					\end{itemize}
					\item \textbf{Bước 2: Phân biệt khí thoát ra (nếu cần thiết để khẳng định rõ hơn).}
					\begin{itemize}
						\item \textbf{Cách tiến hành:} Có thể dẫn khí thoát ra từ mỗi ống nghiệm vào nước vôi trong ($Ca(OH)_2$) hoặc dùng tàn đóm đỏ.
						\begin{itemize}
							\item Nếu là $NH_4Cl$: Hỗn hợp khí $NH_3$ và $HCl$ sẽ không làm đục nước vôi trong (trừ khi có sự ngưng tụ hơi nước và tạo acid).
							\item Nếu là $NH_4NO_3$: Khí $N_2O$ không màu, không mùi, không cháy, không duy trì sự cháy nên không làm tàn đóm bùng cháy.
						\end{itemize}
					\end{itemize}
					\textbf{Kết luận:} Dựa vào hiện tượng khói trắng và mùi (khai, xốc) đặc trưng khi nhiệt phân $NH_4Cl$ khác với hiện tượng nhiệt phân $NH_4NO_3$ chỉ tạo khí không màu, có thể phân biệt được hai muối này.
				\end{enumerate}
			}
		\end{bt}

%		%%%=============BT_16=============%%%
%		\begin{bt}
%			Trong cơ thể người, độ pH của máu được duy trì ổn định trong một khoảng rất hẹp (từ $7{,}35$ đến $7{,}45$) nhờ vào hoạt động của các hệ đệm. Hệ đệm quan trọng nhất là hệ đệm bicarbonat, gồm cặp acid-bazơ liên hợp là acid carbonic ($H_2CO_3$) và ion bicarbonat ($HCO_3^-$).
%			\\
%			\textbf{Tình huống thực tế:} Một bệnh nhân được đưa vào viện với các biểu hiện mệt mỏi, thở nhanh và sâu. Kết quả xét nghiệm khí máu động mạch cho các thông số sau:
%			\begin{itemize}
%				\item Nồng độ ion bicarbonat: $[HCO_3^-] = 11$ mmol/L
%				\item Áp suất riêng phần của khí $CO_2$: $P_{CO_2} = 35$ mmHg
%			\end{itemize}
%			Dựa vào các kiến thức đã học và dữ kiện cho sẵn, em hãy:
%			\begin{enumerate}
%				\item Tính nồng độ acid carbonic ($[H_2CO_3]$) trong máu của bệnh nhân.
%				\item Áp dụng phương trình Henderson-Hasselbalch để xác định độ pH máu của bệnh nhân.
%				\item So sánh kết quả tính được với khoảng pH sinh lý bình thường ($7{,}35 - 7{,}45$) và đưa ra nhận xét về tình trạng sức khỏe của bệnh nhân.
%			\end{enumerate}
%			\textbf{Thông tin bổ sung:}
%			\begin{itemize}
%				\item Hằng số acid của hệ đệm bicarbonat có $\text{p}K_a = 6{,}1$.
%				\item Mối liên hệ giữa nồng độ acid carbonic và áp suất riêng phần $CO_2$ được biểu diễn qua công thức: $[H_2CO_3] = 0{,}03 \times P_{CO_2}$.
%			\end{itemize}
%			\loigiai{
%				\begin{enumerate}
%					\item \textbf{Tính nồng độ acid carbonic ($[H_2CO_3]$):}
%					Trong máu, khí $CO_2$ hòa tan sẽ phản ứng với nước để tạo thành acid carbonic. Nồng độ của nó phụ thuộc trực tiếp vào áp suất riêng phần $P_{CO_2}$.
%					Áp dụng công thức đã cho:
%					\[
%					[H_2CO_3] = 0{,}03 \times P_{CO_2} = 0{,}03 \times 35 = 1{,}05 \text{ (mmol/L)}
%					\]
%					Như vậy, nồng độ thành phần acid trong hệ đệm của bệnh nhân là $1{,}05$ mmol/L.
%					
%					\item \textbf{Áp dụng phương trình Henderson-Hasselbalch để xác định độ pH máu:}
%					Phương trình Henderson-Hasselbalch cho hệ đệm bicarbonat là:
%					\[
%					\text{pH} = \text{p}K_a + \log \left( \dfrac{[\text{Bazơ liên hợp}]}{[\text{Axit yếu}]} \right) = \text{p}K_a + \log \left( \dfrac{[HCO_3^-]}{[H_2CO_3]} \right)
%					\]
%					Ta thay các giá trị đã biết và vừa tính được vào phương trình:
%					\begin{itemize}
%						\item $\text{p}K_a = 6{,}1$
%						\item $[HCO_3^-] = 11$ mmol/L
%						\item $[H_2CO_3] = 1{,}05$ mmol/L
%					\end{itemize}
%                    \begin{align*}
%					  \text{pH}  &= 6{,}1 + \log \left( \dfrac{11}{1{,}05} \right) \\
%					             &= 6{,}1 + \log(10{,}476...) \\
%					             &\approx 6{,}1 + 1{,}02  \approx 7{,}12
%                    \end{align*}
%					\item \textbf{So sánh kết quả và đưa ra nhận xét về tình trạng sức khỏe của bệnh nhân:}
%					\\
%					Độ pH máu của bệnh nhân tính được là $7{,}12$.
%					Khoảng pH sinh lý của người khỏe mạnh là từ $7{,}35$ đến $7{,}45$.
%					\\
%					So sánh hai giá trị, ta thấy pH $7{,}12$ thấp hơn đáng kể so với giới hạn dưới của khoảng bình thường ($7{,}35$).
%					\\
%					\textbf{Nhận xét:} Kết quả tính toán cho thấy máu của bệnh nhân đang trong tình trạng nhiễm toan (acidosis). Điều này có nghĩa là nồng độ acid trong máu cao hơn mức cho phép, gây ra sự mất cân bằng nghiêm trọng trong cơ thể, giải thích cho các triệu chứng mệt mỏi và thở nhanh của bệnh nhân (cơ thể đang cố gắng thải bớt $CO_2$ để làm giảm độ acid). Đây là một tình trạng bệnh lý nguy hiểm.
%				\end{enumerate}
%			}
%		\end{bt}
	\Closesolutionfile{ansbt}
	\Closesolutionfile{ansbth}
	\phan{Bài tập trả lời ngắn}
	%%%=============SOẠN BT===============%%%
	\Opensolutionfile{ansbth}[Ans/LGSA-C02B04_NH3LT]
	\Opensolutionfile{ansbt}[Ans/AnsSA-C02B04_NH3LT]
	%%%=============SA_1=============%%%
	\begin{bt}
		Trong phân tử amoniac ($NH_3$), nitơ có số oxi hóa là bao nhiêu?
		\shortans{-3}
		\loigiai{Trong các hợp chất, hiđro thường có số oxi hóa là $+1$. Gọi số oxi hóa của nitơ là $x$, ta có: $x + 3 \times (+1) = 0 \Rightarrow x = -3$.}
	\end{bt}
	%%%=============SA_2=============%%%
	\begin{bt}
		Phân tử khối của ammonia ($NH_3$) là bao nhiêu g/mol?
		\shortans{17}
		\loigiai{
			Phân tử khối của $NH_3 = 1 \times 14 + 3 \times 1 = 17$ g/mol.
		}
	\end{bt}
	%%%=============SA_3=============%%%
	\begin{bt}
		Số oxi hóa của nitrogen trong phân tử ammonia ($NH_3$) là bao nhiêu?
		\shortans{-3}
		\loigiai{
			Trong $NH_3$, nguyên tử hiđro có số oxi hóa $+1$. Gọi số oxi hóa của nitrogen là $x$. Ta có: $x + 3 \times (+1) = 0 \Rightarrow x = -3$.
		}
	\end{bt}
	%%%=============SA_4=============%%%
	\begin{bt}
		Nguyên tử N trong phân tử $NH_3$ có bao nhiêu cặp electron chưa tham gia liên kết
		\shortans{1}
		\loigiai{Trong phân tử $NH_3$ nguyên tứ nitrogen còn 1 cặp electron chưa tham gia liên kết}
	\end{bt}
    %%%=============SA_5=============%%%
	\begin{bt}
		Trong ion $NH_4^+$ có bao nhiêu liên kết cho nhận (phối trí)
		\shortans{1}
		\loigiai{Trong phân tử $NH_4^+$ có 3 liên kết cộng hóa trị thông thường, 1 liên kết cộng hóa trị dạng phối trí}
	\end{bt}
	%%=============SA_6==============%%%
	\begin{bt}
		Xét cân bằng của dung dịch $NH_3$ $0{,}1$ M ở $25^\circ\text{C}$:
		\[NH_3 + H_2O \rightleftharpoons NH_4^+ + OH^-\quad K_C = 1{,}74 \cdot 10^{-5}\]
		Bỏ qua sự phân li của nước. Xác định giá trị pH của dung dịch trên (làm tròn đến hai chữ số thập phân).
		\shortans{$11,12$}
		\loigiai{
			Ta có phương trình cân bằng:
			\begin{center}
				\begin{tabular}{lcccccc}
					& $NH_3$ & $+$ & $H_2O$ & $\rightleftharpoons$ & $NH_4^+$ & $+$  $OH^-$ \\
					Ban đầu (M): & $0{,}1$ & & & & $0$ &  $0$ \\
					Phản ứng (M): & $x$ & & & & $x$ &  $x$ \\
					Cân bằng (M): & $0{,}1-x$ & & & & $x$ &  $x$ \\
				\end{tabular}
			\end{center}
			Hằng số cân bằng:
			\[K_C = \dfrac{[NH_4^+][OH^-]}{[NH_3]} = \dfrac{x^2}{0{,}1-x} = 1{,}74 \cdot 10^{-5}\]
			Do $K_C$ rất nhỏ, ta có thể coi $0{,}1 - x \approx 0{,}1$.
			\begin{eqnarray*}
				\dfrac{x^2}{0{,}1} & \approx & 1{,}74 \cdot 10^{-5} \\
				\Rightarrow x^2 & \approx & 1{,}74 \cdot 10^{-6} \\
				\Rightarrow x & \approx & \sqrt{1{,}74 \cdot 10^{-6}} \approx 1{,}32 \cdot 10^{-3}
			\end{eqnarray*}
			Vậy $[OH^-] = x \approx 1{,}32 \cdot 10^{-3}$ M.\\
			$pOH = -\log[OH^-] = -\log(1{,}32 \cdot 10^{-3}) \approx 2{,}88$.\\
			$pH = 14 - pOH = 14 - 2{,}88 = 11{,}12$.
		}
	\end{bt}
	
	%%%=============SA_7==============%%%
	\begin{bt}
		Xét cân bằng trong dung dịch gồm $NH_4Cl$ $0{,}10$ M và $NH_3$ $0{,}05$ M ở $25^\circ\text{C}$:
		\[NH_3 + H_2O \rightleftharpoons NH_4^+ + OH^-\quad K_C = 1{,}74 \cdot 10^{-5}\]
		Bỏ qua sự phân li của nước. Xác định giá trị pH của dung dịch trên (làm tròn đến hai chữ số thập phân).
		\shortans{$8,94$}
		\loigiai{
			Dung dịch $NH_4Cl$ điện li hoàn toàn: $NH_4Cl \rightarrow NH_4^+ + Cl^-$.
			Vậy nồng độ ban đầu của các chất là: $[NH_3] = 0{,}05$ M, $[NH_4^+] = 0{,}10$ M.
			Ta có phương trình cân bằng:
			\begin{center}
				\begin{tabular}{l c c c c c c}
					& $NH_3$ & $+$ & $H_2O$ & $\rightleftharpoons$ & $NH_4^+$ & $+$  $OH^-$ \\
					Ban đầu (M): & $0{,}05$ & & & & $0{,}10$ &  $0$ \\
					Phản ứng (M): & $x$ & & & & $x$ &  $x$ \\
					Cân bằng (M): & $0{,}05-x$ & & & & $0{,}10+x$ &  $x$ \\
				\end{tabular}
			\end{center}
			Hằng số cân bằng:
			\[K_C = \dfrac{[NH_4^+][OH^-]}{[NH_3]} = \dfrac{(0{,}10+x)x}{0{,}05-x} = 1{,}74 \cdot 10^{-5}\]
			Do $K_C$ rất nhỏ, ta có thể coi $0{,}05 - x \approx 0{,}05$ và $0{,}10 + x \approx 0{,}10$.
			\begin{eqnarray*}
				\dfrac{0{,}10 \cdot x}{0{,}05} & \approx & 1{,}74 \cdot 10^{-5} \\
				\Rightarrow 2x & \approx & 1{,}74 \cdot 10^{-5} \\
				\Rightarrow x & \approx & 8{,}7 \cdot 10^{-6}
			\end{eqnarray*}
			Vậy $[OH^-] = x \approx 8{,}7 \cdot 10^{-6}$ M.\\
			$pOH = -\log[OH^-] = -\log(8{,}7 \cdot 10^{-6}) \approx 5{,}06$.\\
			$pH = 14 - pOH = 14 - 5{,}06 = 8{,}94$.
		}
	\end{bt}
	%%%=============SA_8=============%%%
	\begin{bt}
		Một dung dịch đệm được tạo ra bằng cách hòa tan $0{,}15$ mol amoniac ($NH_3$) và $0{,}25$ mol amoni clorua ($NH_4Cl$) trong đủ nước để tạo thành $500$ mL dung dịch. Tính pH của dung dịch thu được ở $25^\circ C$. Cho biết hằng số bazơ $K_b$ của $NH_3$ là $1{,}8 \times 10^{-5}$.
		\shortans{9.03}
		\loigiai{
			\begin{enumerate}
				\item \textbf{Tính nồng độ ban đầu của bazơ yếu và acid liên hợp:}
				\\
				Thể tích dung dịch: $V = 500 \text{ mL} = 0{,}5 \text{ L}$.
				\\
				Nồng độ $NH_3$: $[NH_3] = \dfrac{0{,}15 \text{ mol}}{0{,}5 \text{ L}} = 0{,}3$ M.
				\\
				Amoni clorua ($NH_4Cl$) là muối của acid mạnh và bazơ yếu, phân li hoàn toàn trong nước tạo ra ion amoni ($NH_4^+$).
				Nồng độ $NH_4^+$: $[NH_4^+] = \dfrac{0{,}25 \text{ mol}}{0{,}5 \text{ L}} = 0{,}5$ M.
				
				\item \textbf{Tính $\text{p}K_b$ của $NH_3$:}
				\\
				$\text{p}K_b = -\log(K_b) = -\log(1{,}8 \times 10^{-5}) \approx 4{,}74$.
				
				\item \textbf{Áp dụng phương trình Henderson-Hasselbalch để tính pOH:}
				\\
				Đối với dung dịch đệm bazơ, công thức Henderson-Hasselbalch là:
            \begin{align*}
				\text{pOH} &= \text{p}K_b + \log\left(\dfrac{\text{[acid liên hợp]}}{\text{[bazơ yếu]}}\right) \\
				 &= \text{p}K_b + \log\left(\dfrac{[NH_4^+]}{[NH_3]}\right) \\
				 &= 4{,}74 + \log\left(\dfrac{0{,}5}{0{,}3}\right) \\
			&= 4{,}74 + \log(1{,}6667) \approx 4{,}74 + 0{,}22 = 4{,}96
            \end{align*}
				
				\item \textbf{Tính pH của dung dịch:}
				\\
				Ở $25^\circ C$, ta có mối quan hệ: $\text{pH} + \text{pOH} = 14$.
				\\
				$\text{pH} = 14 - \text{pOH} = 14 - 4{,}96 = 9{,}04$.
				\\
				(Nếu làm tròn chính xác hơn: $\text{pOH} \approx 4{,}7447 + \log(0{,}5/0{,}3) \approx 4{,}7447 + 0{,}2218 = 4{,}9665 \Rightarrow \text{pH} = 14 - 4{,}9665 = 9{,}0335$).
			\end{enumerate}
			\textbf{Đáp số:} pH của dung dịch thu được là $9{,}03$.
		}
	\end{bt}
	\Closesolutionfile{ansbt}
	\Closesolutionfile{ansbth}

	\phan{Trắc nghiệm nhiều lựa chọn}
	%%%=============SOẠN EX===============%%%
	\Opensolutionfile{ansex}[Ans/LGEX-C02B04_NH3LT]
	\Opensolutionfile{ans}[Ans/Ans-C02B04_NH3LT]
	%%%=============EX_1=============%%%
	\begin{ex}
		Tính chất vật lí nào sau đây không đúng với ammonia ($NH_3$)?
		\choice
		{Là chất khí không màu.}
		{Có mùi khai đặc trưng.}
		{\True Nặng hơn không khí.}
		{Tan rất nhiều trong nước.}
		\loigiai{
			Phân tử khối của $NH_3$ là $17$ g/mol, trong khi khối lượng mol trung bình của không khí xấp xỉ $29$ g/mol. Vì $17 < 29$, $NH_3$ nhẹ hơn không khí, không phải nặng hơn. Các tính chất còn lại đều đúng.
		}
	\end{ex}
	
	%%%=============EX_2=============%%%
	\begin{ex}
		Khi tác dụng với nước, ammonia ($NH_3$) thể hiện tính chất nào?
		\choice
		{Tính acid.}
		{\True Tính bazơ yếu.}
		{Tính khử.}
		{Tính oxi hóa.}
		\loigiai{
			Ammonia phản ứng với nước tạo ra ion $OH^-$ theo phương trình $NH_3 + H_2O \xharpoonarrow NH_4^+ + OH^-$, làm cho dung dịch có tính bazơ yếu.
		}
	\end{ex}
	
	%%%=============EX_3=============%%%
	\begin{ex}
		Phản ứng nào sau đây chứng minh tính khử của ammonia ($NH_3$)?
		\choice
		{$NH_3 + HCl \xrightarrow NH_4Cl$}
		{$NH_3 + H_2O \xharpoonarrow NH_4^+ + OH^-$}
		{$3NH_3 + AlCl_3 + 3H_2O \rightarrow Al(OH)_3 \downarrow + 3NH_4Cl$}
		{\True $4NH_3 + 3O_2 \xrightarrow[$t^\circ$] 2N_2 + 6H_2O$}
		\loigiai{
			Tính khử của ammonia thể hiện khi số oxi hóa của nitrogen trong $NH_3$ (là $-3$) tăng lên sau phản ứng.
			\begin{itemize}
				\item A, B, C là các phản ứng chứng minh tính bazơ của $NH_3$.
				\item D. Trong phản ứng $4NH_3 + 3O_2 \xrightarrow[$t^\circ$] 2N_2 + 6H_2O$, số oxi hóa của nitrogen tăng từ $-3$ trong $NH_3$ lên $0$ trong $N_2$. Do đó, $NH_3$ thể hiện tính khử.
			\end{itemize}
		}
	\end{ex}
	%%%=============EX_4=============%%%
	\begin{ex}
		Trong công nghiệp, ammonia được sử dụng để sản xuất hóa chất nào sau đây?
		\choice
		{Nitrogen ($N_2$).}
		{Hiđro ($H_2$).}
		{\True Acid nitric ($HNO_3$).}
		{Oxygen ($O_2$).}
		\loigiai{
			Ammonia là nguyên liệu chính để sản xuất acid nitric ($HNO_3$) thông qua quá trình oxi hóa $NH_3$ thành $NO$, sau đó là $NO_2$ và cuối cùng là $HNO_3$.
		}
	\end{ex}
	
	%%%=============EX_5=============%%%
	\begin{ex}
		Khi khí ammonia ($NH_3$) được oxi hóa bởi oxygen có mặt xúc tác Pt/Rh ở nhiệt độ cao, sản phẩm chính của phản ứng là:
		\choice
		{$N_2$ và $H_2O$}
		{$N_2O$ và $H_2O$}
		{\True $NO$ và $H_2O$}
		{$NO_2$ và $H_2O$}
		\loigiai{
			Phản ứng oxi hóa ammonia bằng oxygen có xúc tác Pt/Rh là giai đoạn đầu của quá trình sản xuất acid nitric, tạo ra nitrogen monoxide ($NO$).
			\[
			4\text{NH}_3\text{ (g)} + 5\text{O}_2\text{ (g)} \xrightarrow[$Pt/Rh, t^\circ$] 4\text{NO}\text{ (g)} + 6\text{H}_2\text{O}\text{ (g)}
			\]
		}
	\end{ex}

	%%%=============EX_6=============%%%
	\begin{ex}
		Liên kết trong phân tử $NH_3$ là
		\choice
		{\True Liên kết cộng hóa trị phân cực.}
		{Liên kết ion.}
		{Liên kết cộng hóa trị không phân cực.}
		{Liên kết hydrogen.}
		\loigiai{
			Trong phân tử $NH_3$, nguyên tử nitrogen (độ âm điện $3{,}04$) liên kết với nguyên tử hiđro (độ âm điện $2{,}20$) bằng liên kết cộng hóa trị. Do sự chênh lệch độ âm điện giữa N và H, cặp electron dùng chung bị lệch về phía nguyên tử N, tạo ra các liên kết cộng hóa trị phân cực. Liên kết hiđro chỉ hình thành giữa các phân tử $NH_3$ với nhau hoặc với các phân tử có chứa H liên kết với N, O, F, chứ không phải là liên kết trong nội bộ phân tử $NH_3$.
		}
	\end{ex}
	
	%%%=============EX_7=============%%%
	\begin{ex}
		Trong dung dịch, ammonia thể hiện tính base yếu do
		\choice
		{Phân tử ammonia chứa liên kết cộng hóa trị phân cực và liên kết hydrogen.}
		{Phân tử ammonia chứa liên kết cộng hóa trị phân cực và liên kết ion.}
		{Phần lớn các phân tử ammonia kết hợp với nước tạo ra các ion $NH_4^+$ và $OH^-$.}
		{\True Một phần nhỏ các phân tử $NH_3$ kết hợp với ion $H^+$ của nước tạo $NH_4^+$ và $OH^-$.}
		\loigiai{
			Tính bazơ của ammonia là do nguyên tử nitrogen còn một cặp electron chưa liên kết, có khả năng nhận proton ($H^+$). Ammonia là một bazơ yếu vì khi tan trong nước, chỉ một phần nhỏ các phân tử $NH_3$ phản ứng với nước để tạo ra ion $NH_4^+$ và $OH^-$ theo cân bằng: $NH_3 + H_2O \xharpoonarrow NH_4^+ + OH^-$.
		}
	\end{ex}
	
	%%%=============EX_8=============%%%
	\begin{ex}
		Để tạo độ xốp cho một số loại bánh, có thể dùng chất nào sau đây?
		\choice
		{$(NH_4)_3PO_4$}
		{\True $NH_4HCO_3$}
		{$CaCO_3$}
		{$NaCl$}
		\loigiai{
			$NH_4HCO_3$ (amoni hiđrocacbonat), còn gọi là bột khai, khi nung nóng sẽ phân hủy tạo ra các khí ($NH_3, CO_2, H_2O$):
			\[
			\text{NH}_4\text{HCO}_3 \xrightarrow[$t^\circ$] \text{NH}_3\uparrow + \text{CO}_2\uparrow + \text{H}_2\text{O}\uparrow
			\]
			Các khí này thoát ra làm cho bánh nở phồng và có độ xốp. Các chất khác như $(NH_4)_3PO_4$ (phân bón), $CaCO_3$ (đá vôi, tạo $CO_2$ nhưng cần nhiệt độ rất cao), $NaCl$ (muối ăn) không được dùng với mục đích này trong làm bánh.
		}
	\end{ex}
	%%%=============EX_9================%%%%
	\begin{ex}
		Phát biểu nào sau đây là \textbf{không} đúng khi nói về ammonia?
		\choice
		{\True Trong công nghiệp, ammonia thường được sử dụng với vai trò chất làm lạnh (chất sinh hàn).}
		{Do có hàm lượng nitrogen cao ($82{,}35\%$ theo khối lượng) nên ammonia được sử dụng làm phân đạm rất hiệu quả.}
		{Phần lớn ammonia được dùng phản ứng với acid để sản xuất các loại phân đạm.}
		{Quá trình tổng hợp ammonia từ nitrogen và hydrogen là quá trình thuận nghịch nên không thể đạt hiệu suất $100\%$.}
		\loigiai{
			Phát biểu A là không đúng trong bối cảnh so sánh với các ứng dụng khác. Mặc dù ammonia có được sử dụng làm chất làm lạnh trong các hệ thống công nghiệp lớn, nhưng ứng dụng chính và chiếm tỉ trọng lớn nhất (trên $80\%$) của ammonia là để sản xuất phân đạm. Do đó, các phát biểu còn lại mô tả đúng hơn về vai trò và tính chất của ammonia:
			\begin{itemize}
				\item \textbf{B đúng:} Ammonia có hàm lượng Nitrogen rất cao ($\%N = \dfrac{14}{17} \times 100\% \approx 82{,}35\%$), là nguyên liệu chính để sản xuất và là thành phần của các loại phân đạm.
				\item \textbf{C đúng:} Phần lớn sản lượng ammonia được dùng để sản xuất phân đạm như urea, ammonium nitrate, ammonium sulfate, thông qua phản ứng với $CO_2$ hoặc các acid tương ứng.
				\item \textbf{D đúng:} Phản ứng tổng hợp ammonia trong công nghiệp (quy trình Haber-Bosch) là một cân bằng hoá học: $N_2(g) + 3H_2(g) \xharpoonarrow 2NH_3(g)$. Vì là phản ứng thuận nghịch nên không thể đạt hiệu suất $100\%$.
			\end{itemize}
		}
	\end{ex}
	%%%=============EX_10================%%%%
	\begin{ex}
		Ở trạng thái lỏng nguyên chất, phân tử chất nào sau đây tạo được liên kết hydrogen với nhau?
		\choice
		{Nitrogen.}
		{\True Ammonia.}
		{Oxygen.}
		{Hydrogen.}
		\loigiai{
			Liên kết hydrogen được hình thành giữa các phân tử khi có nguyên tử hydrogen liên kết với một nguyên tử có độ âm điện lớn (như F, O, N). Trong các chất trên, chỉ có ammonia ($NH_3$) có nguyên tử H liên kết trực tiếp với nguyên tử N, do đó các phân tử ammonia có thể tạo liên kết hydrogen với nhau.
		}
	\end{ex}
	
	%%%=============EX_11================%%%%
	\begin{ex}
		Khí nào sau đây dễ tan trong nước do tạo được liên kết hydrogen với nước?
		\choice
		{Nitrogen.}
		{Hydrogen.}
		{\True Ammonia.}
		{Oxygen.}
		\loigiai{
			Ammonia ($NH_3$) tan rất tốt trong nước vì các phân tử $NH_3$ có thể tạo liên kết hydrogen với các phân tử nước ($H_2O$). Các khí còn lại ($N_2$, $H_2$, $O_2$) là các phân tử không phân cực, không tạo được liên kết hydrogen nên tan rất ít trong nước.
		}
	\end{ex}
	
	%%%=============EX_12================%%%%
	\begin{ex}
		Nhận định nào sau đây về phân tử ammonia \textbf{không} đúng?
		\choice
		{Phân cực mạnh.}
		{Có một cặp electron không liên kết.}
		{\True Có độ bền nhiệt cao.}
		{Có khả năng nhận proton.}
		\loigiai{
			Nhận định C là không đúng. Ammonia tương đối kém bền với nhiệt và dễ dàng bị phân hủy thành nitrogen và hydrogen ở nhiệt độ cao: $2NH_3(g) \xharpoonarrow N_2(g) + 3H_2(g)$.
			\begin{itemize}
				\item \textbf{A đúng:} Do có cấu trúc chóp tam giác và các liên kết N-H phân cực nên phân tử $NH_3$ là phân tử phân cực.
				\item \textbf{B đúng:} Nguyên tử Nitrogen trong $NH_3$ còn một cặp electron hóa trị không tham gia liên kết.
				\item \textbf{D đúng:} Do có cặp electron không liên kết, $NH_3$ thể hiện tính base, có khả năng nhận proton ($H^+$).
			\end{itemize}
		}
	\end{ex}
	
	%%%=============EX_13================%%%%
	\begin{ex}
		Khi tác dụng với nước và hydrochloric acid, ammonia đóng vai trò là
		\choice
		{acid.}
		{\True base.}
		{chất oxi hoá.}
		{chất khử.}
		\loigiai{
			Trong cả hai phản ứng, ammonia đều nhận proton ($H^+$), thể hiện tính base theo thuyết Brønsted-Lowry:
			\begin{itemize}
				\item Với nước: $NH_3 + H_2O \xharpoonarrow NH_4^+ + OH^-$.
				\item Với hydrochloric acid: $NH_3 + HCl \rightarrow NH_4Cl$.
			\end{itemize}
			Do đó, ammonia đóng vai trò là một base.
		}
	\end{ex}
	%%%=============EX_14================%%%%
	\begin{ex}
		Trong phương pháp Ostwald, ammonia bị oxi hoá bởi oxygen không khí tạo thành sản phẩm chính là
		\choice
		{\True NO.}
		{$N_2$.}
		{$N_2O$.}
		{$NO_2$.}
		\loigiai{
			Trong giai đoạn đầu của phương pháp Ostwald để sản xuất nitric acid, ammonia được oxi hoá bởi oxygen ở nhiệt độ cao ($850-950^\circ C$) với xúc tác platinum (Pt), tạo ra sản phẩm chính là nitric oxide (NO).
			\[ 4\mathrm{NH}_3 + 5\mathrm{O}_2 \xrightarrow[$t^\circ, xt$] 4\mathrm{NO} + 6\mathrm{H}_2\mathrm{O} \]
		}
	\end{ex}
	
	%%%=============EX_15================%%%%
	\begin{ex}
		Cho dung dịch $NH_3$ vào dung dịch chất nào sau đây thu được kết tủa trắng?
		\choice
		{HCl.}
		{$H_2SO_4$.}
		{$H_3PO_4$.}
		{\True $AlCl_3$.}
		\loigiai{
			Dung dịch $NH_3$ là một base yếu, tạo ra ion $OH^-$: $NH_3 + H_2O \xharpoonarrow NH_4^+ + OH^-$.
			Khi cho vào dung dịch $AlCl_3$, ion $OH^-$ sẽ phản ứng với ion $Al^{3+}$ tạo ra kết tủa keo trắng là aluminium hydroxide, $Al(OH)_3$.
			\[ \mathrm{Al}^{3+} + 3\mathrm{OH}^{-} \xrightarrow{} \mathrm{Al(OH)_3} \downarrow \]
			Các dung dịch axit còn lại chỉ phản ứng tạo muối tan.
		}
	\end{ex}
	
	%%%=============EX_16================%%%%
	\begin{ex}
		Cho vài giọt dung dịch phenolphthalein vào dung dịch $NH_3$, phenolphthalein chuyển sang màu nào sau đây?
		\choice
		{\True Hồng.}
		{Xanh.}
		{Không màu.}
		{Vàng.}
		\loigiai{
			Dung dịch ammonia ($NH_3$) có môi trường base yếu do phản ứng: $NH_3 + H_2O \xharpoonarrow NH_4^+ + OH^-$.
			Phenolphthalein là chất chỉ thị chuyển sang màu hồng trong môi trường base có $pH \approx 8{,}2 - 10$.
		}
	\end{ex}
	
	%%%=============EX_17================%%%%
	\begin{ex}
		Nhiệt phân hoàn toàn muối nào sau đây thu được sản phẩm chỉ gồm khí và hơi?
		\choice
		{NaCl.}
		{$CaCO_3$.}
		{$KClO_3$.}
		{\True $(NH_4)_2CO_3$.}
		\loigiai{
			Ta xét phương trình nhiệt phân của các muối:
			\begin{itemize}
				\item NaCl: Rất bền với nhiệt.
				\item $CaCO_3(s) \xrightarrow[$t^\circ$] CaO(s) + CO_2(g)$. (Sản phẩm có chất rắn CaO).
				\item $2KClO_3(s) \xrightarrow[$t^\circ$] 2KCl(s) + 3O_2(g)$. (Sản phẩm có chất rắn KCl).
				\item $(NH_4)_2CO_3(s) \xrightarrow[$t^\circ$] 2NH_3(g) + CO_2(g) + H_2O(g)$. (Tất cả sản phẩm đều là khí và hơi).
			\end{itemize}
		}
	\end{ex}
	
	%%%=============EX_18================%%%%
	\begin{ex}
		Phân biệt được dung dịch $NH_4Cl$ và $NaCl$ bằng thuốc thử là dung dịch
		\choice
		{KCl.}
		{$KNO_3$.}
		{\True KOH.}
		{$K_2SO_4$.}
		\loigiai{
			Dùng dung dịch base mạnh như $KOH$. Khi cho $KOH$ vào hai mẫu thử và đun nhẹ:
			\begin{itemize}
				\item Mẫu chứa $NH_4Cl$: Có phản ứng tạo ra khí ammonia ($NH_3$) mùi khai, làm xanh giấy quỳ tím ẩm.
               \[ \mathrm{NH_4Cl} + \mathrm{KOH} \xrightarrow[$t^\circ $] \mathrm{KCl} + \mathrm{NH_3} \uparrow + \mathrm{H_2O} 
               \]
				\item Mẫu chứa $NaCl$: Không có hiện tượng gì.
			\end{itemize}
		}
	\end{ex}
	
	%%%=============EX_19================%%%%
	\begin{ex}
		Trong nước, phân tử/ion nào sau đây thể hiện vai trò là acid Brønsted?
		\choice
		{$NH_3$.}
		{\True $NH_4^+$.}
		{$NO_3^-$.}
		{$N_2$.}
		\loigiai{
			Theo thuyết Brønsted-Lowry, acid là chất cho proton ($H^+$).
			\begin{itemize}
				\item $NH_3$ là một base vì nhận $H^+$: $NH_3 + H_2O \xharpoonarrow NH_4^+ + OH^-$.
				\item $NH_4^+$ là một acid vì có khả năng cho $H^+$: $NH_4^+ + H_2O \xharpoonarrow NH_3 + H_3O^+$.
				\item $NO_3^-$ là base liên hợp của acid mạnh $HNO_3$ nên tính base rất yếu, không thể hiện tính acid.
				\item $N_2$ không thể hiện tính acid hay base trong nước.
			\end{itemize}
		}
	\end{ex}
	%%%=============EX_20=============%%%
	\begin{ex}
		Phát biểu nào sau đây không đúng?
		\choice
		{Ammonia là base Brønsted khi tác dụng với nước}
		{Ammonia được sử dụng là chất làm lạnh}
		{Muối ammonium là tinh thể ion, dễ tan trong nước}
		{\True Các muối ammonium đều rất bền với nhiệt}
		\loigiai{Các muối ammonium thường kém bền với nhiệt, dễ bị phân hủy khi đun nóng. Ví dụ: $NH_4Cl(s) \xrightarrow[$t^\circ$] NH_3(g) + HCl(g)$. Do đó, phát biểu "Các muối ammonium đều rất bền với nhiệt" là không đúng.}
	\end{ex}
	%%%=============EX_21=============%%%
	\begin{ex}
		Tiến hành thí nghiệm trộn từng cặp dung dịch sau: 
		\begin{enumerate}[(a)]
			\item $NH_3$ và $AlCl_3$; 
			\item $(NH_4)_2SO_4$ và $Ba(OH)_2$; 
			\item $NH_4Cl$ và $AgNO_3$; 
			\item $NH_3$ và $HCl$. 
			Sau khi phản ứng kết thúc, số thí nghiệm thu được kết tủa là
		\end{enumerate}
		\choice
		{1}
		{\True 3}
		{2}
		{4}
		\loigiai{
			Các phương trình phản ứng xảy ra:
			\begin{itemize}
				\item[(a)] $3NH_3 + 3H_2O + AlCl_3 \xrightarrow Al(OH)_3 \downarrow + 3NH_4Cl$. Có kết tủa $Al(OH)_3$.
				\item[(b)] $(NH_4)_2SO_4 + Ba(OH)_2 \rightarrow BaSO_4 \downarrow + 2NH_3 \uparrow + 2H_2O$. Có kết tủa $BaSO_4$.
				\item[(c)] $NH_4Cl + AgNO_3 \rightarrow AgCl \downarrow + NH_4NO_3$. Có kết tủa $AgCl$.
				\item[(d)] $NH_3 + HCl \rightarrow NH_4Cl$. Không tạo thành kết tủa.
			\end{itemize}
			Vậy có 3 thí nghiệm thu được kết tủa là (a), (b), và (c).
		}
	\end{ex}
	%%%=============EX_22=============%%%
	\begin{ex}
		Xét cân bằng hoá học: $NH_3 + H_2O \rightleftharpoons NH_4^+ + OH^-$. Cân bằng sẽ chuyển dịch theo chiều thuận khi cho thêm vài giọt dung dịch nào sau đây?
		\choice
		{$NH_4Cl$}
		{$NaOH$}
		{\True $HCl$}
		{$NaCl$}
		\loigiai{
			Theo nguyên lí Le Chatelier, để cân bằng chuyển dịch theo chiều thuận (tạo ra nhiều $NH_4^+$ và $OH^-$ hơn), ta cần làm giảm nồng độ sản phẩm hoặc tăng nồng độ chất phản ứng.
			\begin{itemize}
				\item Thêm $NH_4Cl$ (cung cấp ion $NH_4^+$) làm tăng nồng độ sản phẩm, cân bằng chuyển dịch theo chiều nghịch.
				\item Thêm $NaOH$ (cung cấp ion $OH^-$) làm tăng nồng độ sản phẩm, cân bằng chuyển dịch theo chiều nghịch.
				\item Thêm $HCl$ (cung cấp ion $H^+$), ion $H^+$ sẽ phản ứng với $OH^-$: $H^+ + OH^- \rightarrow H_2O$. Điều này làm giảm nồng độ $OH^-$, khiến cân bằng chuyển dịch theo chiều thuận để bù lại lượng $OH^-$ đã mất.
				\item Thêm $NaCl$ không ảnh hưởng đến nồng độ các chất trong cân bằng.
			\end{itemize}
			Vậy, thêm $HCl$ sẽ làm cân bằng chuyển dịch theo chiều thuận.
		}
	\end{ex}
	%%%=============EX_23=============%%%
	\begin{ex}
		Xét cân bằng hoá học: $NH_3 + H_2O \rightleftharpoons NH_4^+ + OH^-$. Hằng số cân bằng ($K_c$) của phản ứng được biểu diễn bằng biểu thức nào sau đây?
		\choice
		{\True $K_c = \frac{[NH_4^+][OH^-]}{[NH_3]}$}
		{$K_c = \frac{[NH_4^+][OH^-]}{[NH_3][H_2O]}$}
		{$K_c = \frac{[NH_3][H_2O]}{[NH_4^+][OH^-]}$}
		{$K_c = \frac{[NH_3]}{[NH_4^+][OH^-]}$}
		\loigiai{
			Hằng số cân bằng $K_c$ được tính bằng tích nồng độ mol của các sản phẩm chia cho tích nồng độ mol của các chất phản ứng, mỗi nồng độ được lũy thừa với hệ số tỉ lượng tương ứng trong phương trình hoá học.
			Đối với các phản ứng trong dung dịch loãng, nồng độ của dung môi (ở đây là $H_2O$) được coi là không đổi và được gộp vào hằng số cân bằng.
			Vì vậy, biểu thức hằng số cân bằng cho phản ứng này là: $K_c = \frac{[NH_4^+][OH^-]}{[NH_3]}$.
		}
	\end{ex}
	%%%=============EX_24=============%%%
	\begin{ex}
		Xét cân bằng hoá học: $N_2(g) + 3H_2(g) \rightleftharpoons 2NH_3(g) \quad \Delta H < 0$. Hiệu suất phản ứng khi hệ đạt cân bằng ở nhiệt độ $400\,^\circ\text{C}$ và $500\,^\circ\text{C}$ lần lượt bằng x\% và y\%. Mối quan hệ giữa x và y là
		\choice
		{$x < y$}
		{\True $x > y$}
		{$x = y$}
		{$5x = 4y$}
		\loigiai{
			Phản ứng tổng hợp amoniac là một phản ứng tỏa nhiệt ($\Delta H < 0$). Theo nguyên lí chuyển dịch cân bằng Le Chatelier, khi tăng nhiệt độ, cân bằng sẽ chuyển dịch theo chiều chống lại sự tăng nhiệt độ đó, tức là chiều thu nhiệt (chiều nghịch). Do đó, hiệu suất của phản ứng tạo ra $NH_3$ sẽ giảm khi nhiệt độ tăng.
			Vì $400\,^\circ\text{C} < 500\,^\circ\text{C}$, nên hiệu suất ở $400\,^\circ\text{C}$ (x) sẽ lớn hơn hiệu suất ở $500\,^\circ\text{C}$ (y). Vậy $x > y$.
		}
	\end{ex}
	%%%=============EX_25=============%%%
	\begin{ex}
		Xét cân bằng hoá học: $N_2(g) + 3H_2(g) \rightleftharpoons 2NH_3(g) \quad \Delta H < 0$. Hiệu suất phản ứng khi hệ đạt cân bằng ở áp suất $200$ bar và $300$ bar lần lượt bằng x\% và y\%. Mối quan hệ giữa x và y là
		\choice
		{$5x = 4y$}
		{$x = y$}
		{$x > y$}
		{\True $x < y$}
		\loigiai{
			Xét phản ứng: $N_2(g) + 3H_2(g) \rightleftharpoons 2NH_3(g)$.
			Tổng số mol khí của chất phản ứng là $1 + 3 = 4$ mol.
			Tổng số mol khí của sản phẩm là $2$ mol.
			Theo nguyên lí chuyển dịch cân bằng Le Chatelier, khi tăng áp suất của hệ, cân bằng sẽ chuyển dịch theo chiều làm giảm áp suất, tức là chiều làm giảm tổng số mol khí (chiều thuận).
			Do đó, khi tăng áp suất từ $200$ bar lên $300$ bar, cân bằng chuyển dịch theo chiều thuận, làm tăng hiệu suất phản ứng.
			Vậy, hiệu suất ở áp suất $200$ bar (x) nhỏ hơn hiệu suất ở áp suất $300$ bar (y), tức là $x < y$.
		}
	\end{ex}
	%%%=============EX_26=============%%%
	\begin{ex}
		Hỗn hợp X gồm $N_2$ và $H_2$ có tỉ lệ mol tương ứng là $1:4$. Nung nóng X trong bình kín ở nhiệt độ khoảng $450^\circ C$ có bột Fe xúc tác, thu được hỗn hợp khí Y có tỉ khối so với $H_2$ bằng $4$. Hiệu suất của phản ứng tổng hợp $NH_3$ là
		\choice
		{$20\%$}
		{\True $25\%$}
		{$30\%$}
		{$10\%$}
		\loigiai{
			Giả sử ban đầu có $1$ mol $N_2$ và $4$ mol $H_2$.
			\begin{itemize}
				\item Khối lượng hỗn hợp X ban đầu: $m_X = m_{N_2} + m_{H_2} = 1 \times 28 + 4 \times 2 = 36$ (g).
				\item Theo định luật bảo toàn khối lượng: $m_Y = m_X = 36$ (g).
				\item Khối lượng mol trung bình của hỗn hợp Y: $M_Y = d_{Y/H_2} \times M_{H_2} = 4 \times 2 = 8$ (g/mol).
				\item Số mol hỗn hợp Y sau phản ứng: $n_Y = \frac{m_Y}{M_Y} = \frac{36}{8} = 4{,}5$ (mol).
			\end{itemize}
			Phương trình phản ứng:
			\[ \mathrm{N}_2(\mathrm{g}) + 3\mathrm{H}_2(\mathrm{g}) \rightleftharpoons 2\mathrm{NH}_3(\mathrm{g}) \]
			Gọi số mol $N_2$ đã phản ứng là a (mol).
			\begin{itemize}
				\item $n_{N_2}$ pư = a mol
				\item $n_{H_2}$ pư = 3a mol
				\item $n_{NH_3}$ tạo thành = 2a mol
			\end{itemize}
			Số mol khí giảm sau phản ứng = (số mol khí trước pư) - (số mol khí sau pư)
			\[ \Delta n_{khí} = (a + 3a) - 2a = 2a \text{ mol} \]
			Số mol khí ban đầu: $n_X = 1 + 4 = 5$ (mol).
			Số mol khí giảm thực tế: $n_X - n_Y = 5 - 4{,}5 = 0{,}5$ (mol).
			Vậy, $2a = 0{,}5 \Rightarrow a = 0{,}25$ mol.
			\\
			Ta có tỉ lệ ban đầu: $\frac{n_{N_2}}{1} = \frac{1}{1} < \frac{n_{H_2}}{3} = \frac{4}{3}$. Do đó, hiệu suất được tính theo $N_2$ (chất hết).
			Hiệu suất phản ứng:
			\[ 
			\mathrm{H} = \dfrac{n_{\mathrm{N}_2 \text{phản ứng}}}{n_{\mathrm{N}_2 \text{ ban đầu}}} \times 100\% = \frac{0{,}25}{1} \times 100\% = 25\% 
			\]
		}
	\end{ex}
	\Closesolutionfile{ans}
	\Closesolutionfile{ansex}
	
	\phan{Bài tập trắc nghiệm Đúng Sai}
	
	%%%=============SOẠN EXTF===============%%%
	\Opensolutionfile{ansex}[Ans/LGTF-C02B04_NH3LT]
	\Opensolutionfile{ansbook}[Ansbook/AnsTF-C02B04_NH3LT]
	\Opensolutionfile{ans}[Ans/Tempt-C02B04_NH3LT]
	%%%=============TF_1=============%%%
	\begin{ex}
		Đánh giá tính đúng/sai của các phát biểu sau về ammonia ($NH_3$):
		\choiceTF
		{\True Ammonia là chất khí không màu, có mùi khai.}
		{Ammonia là một bazơ mạnh.}
		{\True Dung dịch amoniac có khả năng hòa tan kết tủa một số hiđroxit kim loại như $Al(OH)_3$.}
		{Trong phân tử $NH_3$, nguyên tử nitrogen có cặp electron chưa liên kết.}
		\loigiai{
			\begin{itemchoice}[T1,F2,F3,T4]
				\itemch Ammonia có tính chất vật lí là khí không màu, mùi khai. Phát biểu đúng.
				\itemch Ammonia là một \textbf{bazơ yếu}, không phải bazơ mạnh. Phát biểu sai.
				\itemch Dung dịch amoniac \textbf{không} hòa tan được $Al(OH)_3$ (do $Al(OH)_3$ là hidroxit lưỡng tính nhưng $NH_3$ là bazơ yếu không đủ mạnh để hòa tan). $NH_3$ chỉ hòa tan được một số hiđroxit của kim loại tạo phức như $Cu(OH)_2$, $Ag_2O$. Phát biểu sai.
				\itemch Nguyên tử nitrogen trong $NH_3$ có một cặp electron chưa liên kết, đó là nguyên nhân gây ra tính bazơ của $NH_3$. Phát biểu đúng.
			\end{itemchoice}
		}
	\end{ex}
	
	%%%=============TF_2=============%%%
	\begin{ex}
		Đánh giá tính đúng/sai của các phát biểu sau về tính chất hóa học của ammonia:
		\choiceTF
		{\True Ammonia có thể cháy trong khí oxygen, tạo thành nitrogen và nước.}
		{Ammonia chỉ thể hiện tính khử khi tác dụng với các phi kim.}
		{\True Ammonia có thể khử một số oxit kim loại thành kim loại.}
		{Khi tác dụng với acid, ammonia tạo ra muối amoni và giải phóng khí hiđro.}
		\loigiai{
			\begin{itemchoice}[T1,F2,T3,F4]
				\itemch $4NH_3 + 3O_2 \xrightarrow[$t^\circ$] 2N_2 + 6H_2O$. Phát biểu đúng.
				\itemch Ammonia thể hiện tính khử không chỉ với phi kim (như oxygen) mà còn với một số oxit kim loại. Phát biểu sai.
				\itemch $2NH_3 + 3CuO \xrightarrow[$t^\circ$] N_2 + 3Cu + 3H_2O$. Phát biểu đúng.
				\itemch Khi tác dụng với acid, ammonia tạo ra muối amoni (ví dụ $NH_4Cl$) nhưng \textbf{không giải phóng khí hiđro}. Phát biểu sai.
			\end{itemchoice}
		}
	\end{ex}
	%%%=============TF_3=============%%%
	\begin{ex}
		Phản ứng tổng hợp ammonia ($NH_3$) trong công nghiệp được biểu diễn bằng phương trình hóa học:
		\[
		\text{N}_2\text{(g)} + 3\text{H}_2\text{(g)} \xharpoonarrow 2\text{NH}_3\text{(g)} \quad \Delta_rH^\circ = -92 \text{ kJ}
		\]
		Đánh giá tính đúng/sai của các phát biểu sau về các yếu tố ảnh hưởng đến cân bằng của phản ứng này:
		\choiceTF
		{\True Để tăng hiệu suất cân bằng của phản ứng, nên thực hiện ở nhiệt độ thấp.}
		{\True Tăng áp suất sẽ làm cân bằng dịch chuyển theo chiều tạo thành $NH_3$, giúp tăng hiệu suất.}
		{Việc sử dụng chất xúc tác sắt (Fe) trong quá trình Haber giúp tăng hiệu suất cân bằng của phản ứng.}
		{\True Việc liên tục tách $NH_3$ ra khỏi hỗn hợp khí sau phản ứng giúp cân bằng dịch chuyển theo chiều thuận, làm tăng lượng $NH_3$ tổng cộng thu được.}
		\loigiai{
			\begin{itemchoice}[T1,T2,F3,T4]
				\itemch Phản ứng tổng hợp $NH_3$ là phản ứng tỏa nhiệt ($\Delta_rH^\circ = -92$ kJ). Theo nguyên lí Le Chatelier, khi giảm nhiệt độ, cân bằng sẽ dịch chuyển theo chiều tỏa nhiệt (chiều thuận), làm tăng hiệu suất tạo thành $NH_3$. (Trong công nghiệp, nhiệt độ cần đủ cao để đảm bảo tốc độ phản ứng hợp lý, là sự đánh đổi giữa tốc độ và hiệu suất). Phát biểu đúng.
				\itemch Phản ứng có số mol khí giảm (từ $1 \text{ mol } N_2 + 3 \text{ mol } H_2 = 4 \text{ mol}$ khí phản ứng chuyển thành $2 \text{ mol } NH_3$ khí sản phẩm). Theo nguyên lí Le Chatelier, khi tăng áp suất, cân bằng sẽ dịch chuyển theo chiều giảm số mol khí (chiều thuận), làm tăng hiệu suất tạo thành $NH_3$. Phát biểu đúng.
				\itemch Chất xúc tác (Fe) làm tăng tốc độ của cả phản ứng thuận và nghịch với mức độ như nhau, giúp hệ đạt đến trạng thái cân bằng nhanh hơn. Tuy nhiên, xúc tác \textbf{không làm thay đổi vị trí cân bằng} hay hiệu suất cân bằng của phản ứng. Phát biểu sai.
				\itemch Khi $NH_3$ được tạo thành và tách ra khỏi hệ, nồng độ của $NH_3$ giảm xuống. Theo nguyên lí Le Chatelier, cân bằng sẽ dịch chuyển theo chiều chống lại sự thay đổi đó, tức là chiều thuận, để tạo thêm $NH_3$. Việc này giúp tăng lượng $NH_3$ tổng cộng thu được theo thời gian. Phát biểu đúng.
			\end{itemchoice}
		}
	\end{ex}
	\Closesolutionfile{ans}
	\Closesolutionfile{ansbook}
	\Closesolutionfile{ansex}
\end{dang}
%%%Bài tập dạng 2
\begin{dang}{Bài tập về cân bằng hóa học và hiệu suất trong quá trình tổng hợp Ammonia}
	\begin{phuongphap}
		Để giải quyết các bài tập về cân bằng hóa học và hiệu suất của phản ứng tổng hợp ammonia, học sinh cần:
		\begin{cacbuoc}
			\item Nắm vững phương trình phản ứng: $N_2\text{(g)} + 3H_2\text{(g)} \rightleftharpoons 2NH_3\text{(g)}$ và các đặc điểm của nó (tỏa nhiệt, số mol khí giảm).
			\item Vận dụng thành thạo nguyên lí Le Chatelier để dự đoán chiều dịch chuyển cân bằng khi thay đổi nhiệt độ, áp suất, hoặc nồng độ các chất.
			\item Hiểu vai trò của chất xúc tác trong việc đạt cân bằng.
			\item Áp dụng công thức tính hiệu suất phản ứng ($H = \dfrac{\text{lượng chất đã phản ứng}}{\text{lượng chất ban đầu}} \times 100\%$) dựa trên chất phản ứng hết hoặc chất thiếu.
			\item Sử dụng biểu thức hằng số cân bằng $K_c = \dfrac{[NH_3]^2}{[N_2][H_2]^3}$ và lập bảng nồng độ/số mol ban đầu - phản ứng - cân bằng để giải các bài toán định lượng.
		\end{cacbuoc}
	\end{phuongphap}
%	%%%
	\phan{Bài tập tự luận}
	%%%=============SOẠN BT===============%%%
	\Opensolutionfile{ansbth}[Ans/LGBT-C02B05_CanBang]
	\Opensolutionfile{ansbt}[Ans/AnsBT-C02B05_CanBang]
	%%%=============BT_1=============%%%
	\begin{bt}
		Hỗn hợp X gồm $N_2$ và $H_2$ có tỉ lệ mol tương ứng là $1:3$. Nung nóng X trong bình kín ($450$ $ ^\circ C$, xúc tác Fe) một thời gian, thu được hỗn hợp khí có số mol giảm $5\%$ so với ban đầu. Tính hiệu suất của phản ứng tổng hợp $NH_3$.
		\loigiai{
			\begin{enumerate}
				\item \textbf{Thiết lập tỉ lệ mol ban đầu và sự thay đổi số mol:}
				\\
				Giả sử tổng số mol hỗn hợp X ban đầu là $4$ mol.
				\\
				Do tỉ lệ mol $N_2 : H_2$ là $1:3$, ta có: $n_{N_2 \text{ ban đầu}} = 1$ mol và $n_{H_2 \text{ ban đầu}} = 3$ mol.
				\\
				Phản ứng tổng hợp $NH_3$:
				\[
				\text{N}_2\text{ (g)} + 3\text{H}_2\text{ (g)} \rightleftharpoons 2\text{NH}_3\text{ (g)}
				\]
				Từ phương trình, cứ $1$ mol $N_2$ và $3$ mol $H_2$ phản ứng (tổng $4$ mol chất phản ứng) sẽ tạo ra $2$ mol $NH_3$. Điều này có nghĩa là, cứ mỗi $2$ mol $NH_3$ được tạo thành, tổng số mol khí sẽ giảm đi $4 - 2 = 2$ mol.
				\\
				Gọi $n_{N_2 \text{ phản ứng}} = a$ mol.
				\\
				Vậy số mol khí giảm đi ($\Delta n$) là $2a$ mol.
				\item \textbf{Tính số mol khí giảm dựa trên dữ liệu đề bài:}
				\\
				Số mol hỗn hợp khí giảm $5\%$ so với ban đầu.
				\\
				$\Delta n = 5\% \times n_{\text{ban đầu}} = 0{,}05 \times 4 = 0{,}2$ mol.
				\item \textbf{Xác định số mol $N_2$ đã phản ứng:}
				\\
				Ta có $2a = 0{,}2 \text{ mol} \Rightarrow a = 0{,}1$ mol.
				\item \textbf{Tính hiệu suất của phản ứng ($H$):}
				\\
				Hiệu suất của phản ứng được tính dựa trên chất hết trước. Trong trường hợp này, tỉ lệ $N_2:H_2$ ban đầu là $1:3$ đúng bằng tỉ lệ phản ứng, nên hiệu suất có thể tính theo $N_2$ hoặc $H_2$.
				\\
				$H = \dfrac{n_{N_2 \text{ phản ứng}}}{n_{N_2 \text{ ban đầu}}} \times 100\% = \dfrac{0{,}1}{1} \times 100\% = 10\%$
			\end{enumerate}
			\textbf{Đáp số:} Hiệu suất của phản ứng tổng hợp $NH_3$ là $10\%$.
		}
	\end{bt}
	%%%=============BT_2=============%%%
	\begin{bt}
		Một bình kín có dung tích là $0{,}5$ L chứa $1{,}5$ mol $H_2$ và $0{,}5$ mol $N_2$ ở nhiệt độ xác định. Ở trạng thái cân bằng có $0{,}2$ mol $NH_3$ tạo thành. Tính hằng số cân bằng $K_c$ của phản ứng xảy ra trong bình.
		\loigiai{
			Phản ứng xảy ra trong bình là phản ứng tổng hợp ammonia:
			\[
			\text{N}_2\text{(g)} + 3\text{H}_2\text{(g)} \xharpoonarrow 2\text{NH}_3\text{(g)}
			\]
			\textbf{Lập bảng biến thiên nồng độ (hoặc số mol rồi tính nồng độ):}
			\begin{itemize}
				\item Thể tích bình $V = 0{,}5L$ .
				\item Số mol $NH_3$ tạo thành ở trạng thái cân bằng là $0{,}2$ mol.
			\end{itemize}%
			\begin{tabular}{|l|c|c|c|}
				\hline
				& $\text{N}_2$ & $3\text{H}_2$ & $2\text{NH}_3$ \\
				\hline
				\textbf{Số mol ban đầu} & $0{,}5$ & $1{,}5$ & $0$ \\
				\hline
				\textbf{Số mol phản ứng} & & & \\
				(Dựa vào $NH_3$ tạo thành) & $\dfrac{0{,}2}{2} = 0{,}1$ & $3 \times \dfrac{0{,}2}{2} = 0{,}3$ & $0{,}2$ \\
				\hline
				\textbf{Số mol cân bằng} & $0{,}5 - 0{,}1 = 0{,}4$ & $1{,}5 - 0{,}3 = 1{,}2$ & $0{,}2$ \\
				\hline
			\end{tabular}
			\\
			\textbf{Nồng độ các chất ở trạng thái cân bằng:}
			\begin{itemize}
				\item $[N_2] = \dfrac{0{,}4 \text{ mol}}{0{,}5 \text{ L}} = 0{,}8$ M
				\item $[H_2] = \dfrac{1{,}2 \text{ mol}}{0{,}5 \text{ L}} = 2{,}4$ M
				\item $[NH_3] = \dfrac{0{,}2 \text{ mol}}{0{,}5 \text{ L}} = 0{,}4$ M
			\end{itemize}
			\textbf{Tính hằng số cân bằng $K_c$:}
		\begin{align*}
			K_c &= \dfrac{[NH_3]^2}{[N_2][H_2]^3} \\
				&= \dfrac{(0{,}4)^2}{(0{,}8) \times (2{,}4)^3} \\
				&= \dfrac{0{,}16}{0{,}8 \times 13{,}824} \\
				&= \dfrac{0{,}16}{11{,}0592} \approx 0{,}014467 \approx 0{,}0145
		\end{align*}
			\textbf{Đáp số:} Hằng số cân bằng $K_c$ của phản ứng là xấp xỉ $0{,}0145$.
		}
	\end{bt}
	%%%=============BT_3=============%%%
	\begin{bt}
		Phản ứng tổng hợp ammonia ($NH_3$) từ nitrogen và hydrogen ($N_2\text{(g)} + 3H_2\text{(g)} \xharpoonarrow 2NH_3\text{(g)}$) là một trong những phản ứng công nghiệp quan trọng nhất. Phản ứng này tỏa nhiệt ($\Delta_rH^\circ = -92 \text{ kJ}$). Tuy nhiên, trong công nghiệp, người ta lại thực hiện ở nhiệt độ khá cao (khoảng $400-600^\circ C$) và áp suất cao ($150-200$ bar), có mặt xúc tác Fe.
		\begin{enumerate}
			\item Giải thích tại sao phản ứng này phải thực hiện ở nhiệt độ cao dù là phản ứng tỏa nhiệt.
			\item Giải thích tại sao phải thực hiện ở áp suất cao.
			\item Vai trò của xúc tác Fe trong phản ứng này là gì?
		\end{enumerate}
		\loigiai{
			\begin{enumerate}
				\item \textbf{Giải thích về nhiệt độ cao:}
				Mặc dù phản ứng tổng hợp ammonia là phản ứng tỏa nhiệt (theo nguyên lí Le Chatelier, nhiệt độ thấp sẽ làm cân bằng dịch chuyển theo chiều thuận, tăng hiệu suất), nhưng ở nhiệt độ thường, nitrogen rất trơ về mặt hóa học do liên kết ba $N \equiv N$ rất bền, phản ứng diễn ra cực kỳ chậm (tốc độ phản ứng gần như bằng không). Do đó, cần phải nâng nhiệt độ lên khá cao (khoảng $400-600^\circ C$) để cung cấp đủ năng lượng hoạt hóa, phá vỡ liên kết ba của $N_2$ và tăng tốc độ phản ứng đến mức có thể chấp nhận được trong công nghiệp, dù hiệu suất cân bằng có thể giảm đi một phần.
				\item \textbf{Giải thích về áp suất cao:}
				Phản ứng $N_2\text{(g)} + 3H_2\text{g)} \rightleftharpoons 2NH_3\text{(g)}$ là phản ứng có số mol khí giảm (từ $1+3=4$ mol khí phản ứng chuyển thành $2$ mol khí sản phẩm). Theo nguyên lí Le Chatelier, việc tăng áp suất sẽ làm cân bằng dịch chuyển theo chiều giảm số mol khí, tức là chiều thuận, giúp tăng hiệu suất tạo ra $NH_3$. Áp suất cao cũng góp phần làm tăng tốc độ phản ứng.
				\item \textbf{Vai trò của xúc tác Fe:}
				Xúc tác Fe (có pha thêm Al$_2$O$_3$, K$_2$O...) có vai trò làm tăng tốc độ phản ứng tổng hợp ammonia bằng cách giảm năng lượng hoạt hóa của phản ứng. Xúc tác giúp phản ứng đạt đến trạng thái cân bằng nhanh hơn, từ đó tăng năng suất sản xuất $NH_3$ trong thời gian ngắn mà không làm thay đổi vị trí cân bằng hay hiệu suất cân bằng.
			\end{enumerate}
		}
	\end{bt}
	%%%=============BT_4=============%%%
	\begin{bt}
		Trong một bình kín dung tích $1$ L, người ta cho vào $0{,}5$ mol $N_2$ và $1{,}5$ mol $H_2$. Sau một thời gian nung nóng ở $450$ $ ^\circ C$ với xúc tác Fe, hỗn hợp khí đạt trạng thái cân bằng. Biết hiệu suất của phản ứng tổng hợp $NH_3$ tính theo $N_2$ là $20\%$.
		\begin{enumerate}
			\item Tính số mol mỗi khí ở trạng thái cân bằng.
			\item Tính nồng độ mỗi khí ở trạng thái cân bằng.
			\item Tính hằng số cân bằng $K_c$ của phản ứng ở $450$ $ ^\circ C$.
		\end{enumerate}
		\loigiai{
			Phản ứng tổng hợp ammonia:
			\[
			\text{N}_2\text{(g)} + 3\text{H}_2\text{(g)} \rightleftharpoons 2\text{NH}_3\text{(g)}
			\]
			\begin{enumerate}
				\item \textbf{Tính số mol mỗi khí ở trạng thái cân bằng:}
				\\
				Hiệu suất tính theo $N_2$ là $20\%$.
				\\
				Số mol $N_2$ ban đầu: $0{,}5$ mol.
				\\
				Số mol $N_2$ đã phản ứng: $n_{N_2 \text{ pư}} = 0{,}5 \times 20\% = 0{,}5 \times 0{,}2 = 0{,}1$ mol.
				\\
				Từ phương trình phản ứng:
				\begin{itemize}
					\item Số mol $H_2$ đã phản ứng: $n_{H_2 \text{ pư}} = 3 \times n_{N_2 \text{ pư}} = 3 \times 0{,}1 = 0{,}3$ mol.
					\item Số mol $NH_3$ tạo thành: $n_{NH_3 \text{ tạo thành}} = 2 \times n_{N_2 \text{ pư}} = 2 \times 0{,}1 = 0{,}2$ mol.
				\end{itemize}
				Số mol các chất ở trạng thái cân bằng:
				\begin{itemize}
					\item $n_{N_2 \text{ cb}} = n_{N_2 \text{ ban đầu}} - n_{N_2 \text{ pư}} = 0{,}5 - 0{,}1 = 0{,}4$ mol.
					\item $n_{H_2 \text{ cb}} = n_{H_2 \text{ ban đầu}} - n_{H_2 \text{ pư}} = 1{,}5 - 0{,}3 = 1{,}2$ mol.
					\item $n_{NH_3 \text{ cb}} = 0{,}2$ mol.
				\end{itemize}
				
				\item \textbf{Tính nồng độ mỗi khí ở trạng thái cân bằng:}
				\\
				Vì dung tích bình là $1$ L, nên nồng độ mol bằng số mol.
				\begin{itemize}
					\item $[N_2] = 0{,}4$ M.
					\item $[H_2] = 1{,}2$ M.
					\item $[NH_3] = 0{,}2$ M.
				\end{itemize}
				
				\item \textbf{Tính hằng số cân bằng $K_c$:}
				\[
				K_c = \dfrac{[NH_3]^2}{[N_2][H_2]^3} \\
				K_c = \dfrac{(0{,}2)^2}{(0{,}4) \times (1{,}2)^3} \\
				K_c = \dfrac{0{,}04}{0{,}4 \times 1{,}728} \\
				K_c = \dfrac{0{,}04}{0{,}6912} \approx 0{,}05786
				\]
				(Làm tròn đến 4 chữ số thập phân: $0{,}0579$).
			\end{enumerate}
		}
	\end{bt}
	%%%
	%%%=============BT_5=============%%%
	\begin{bt}
		Phản ứng tổng hợp ammonia ($N_2\text{(g)} + 3H_2\text{(g)} \rightleftharpoons 2NH_3\text{(g)}$, $\Delta_rH^\circ < 0$) là phản ứng tỏa nhiệt và có số mol khí giảm.
		\begin{enumerate}
			\item Nêu các yếu tố (ngoài chất xúc tác) có thể tác động để làm dịch chuyển cân bằng theo chiều thuận.
			\item Phân tích sự ảnh hưởng của việc tăng nhiệt độ đến cả tốc độ phản ứng và hiệu suất cân bằng của quá trình tổng hợp $NH_3$. Tại sao trong công nghiệp vẫn dùng nhiệt độ cao?
		\end{enumerate}
		\loigiai{
			\begin{enumerate}
				\item \textbf{Các yếu tố làm dịch chuyển cân bằng theo chiều thuận:}
				\begin{itemize}
					\item \textbf{Giảm nhiệt độ:} Vì phản ứng là tỏa nhiệt, giảm nhiệt độ sẽ làm cân bằng dịch chuyển theo chiều tỏa nhiệt (chiều thuận).
					\item \textbf{Tăng áp suất:} Phản ứng có số mol khí giảm (từ 4 mol chất phản ứng sang 2 mol sản phẩm), nên tăng áp suất sẽ làm cân bằng dịch chuyển theo chiều giảm số mol khí (chiều thuận).
					\item \textbf{Tăng nồng độ chất phản ứng ($N_2$ hoặc $H_2$):} Tăng nồng độ chất phản ứng làm cân bằng dịch chuyển theo chiều thuận để giảm bớt nồng độ chất phản ứng đã tăng.
					\item \textbf{Giảm nồng độ sản phẩm ($NH_3$):} Liên tục tách $NH_3$ ra khỏi hỗn hợp phản ứng làm giảm nồng độ $NH_3$, khiến cân bằng dịch chuyển theo chiều thuận để tạo thêm $NH_3$.
				\end{itemize}
				\item \textbf{Ảnh hưởng của tăng nhiệt độ đến tốc độ và hiệu suất, và lý do dùng nhiệt độ cao trong công nghiệp:}
				\begin{itemize}
					\item \textbf{Tốc độ phản ứng:} Tăng nhiệt độ luôn làm tăng tốc độ phản ứng, vì nó làm tăng động năng của các phân tử, tăng số va chạm hiệu quả giữa các chất.
					\item \textbf{Hiệu suất cân bằng:} Vì phản ứng tổng hợp $NH_3$ là phản ứng tỏa nhiệt, theo nguyên lí Le Chatelier, tăng nhiệt độ sẽ làm cân bằng dịch chuyển theo chiều nghịch (chiều thu nhiệt), do đó làm giảm hiệu suất cân bằng của phản ứng.
					\item \textbf{Lý do dùng nhiệt độ cao trong công nghiệp:} Trong công nghiệp, người ta vẫn phải thực hiện phản ứng ở nhiệt độ khá cao (khoảng $400-600^\circ C$) dù hiệu suất cân bằng giảm. Lý do là vì ở nhiệt độ thấp, mặc dù hiệu suất cân bằng cao, nhưng tốc độ phản ứng diễn ra cực kỳ chậm (do phân tử $N_2$ rất bền). Để đạt được tốc độ sản xuất $NH_3$ đủ nhanh, mang lại hiệu quả kinh tế, cần phải chấp nhận một sự đánh đổi giữa hiệu suất cân bằng và tốc độ phản ứng. Nhiệt độ cao giúp phản ứng đạt đến trạng thái cân bằng nhanh chóng trong thời gian công nghiệp chấp nhận được.
				\end{itemize}
			\end{enumerate}
		}
	\end{bt}
	%%%%
	%%%=============BT_6=============%%%
	\begin{bt}
		Người ta cho $2$ mol $N_2$ và $6$ mol $H_2$ vào một bình kín dung tích $2$ L ở $450$ $ ^\circ C$. Sau một thời gian, hỗn hợp đạt trạng thái cân bằng và có $0{,}8$ mol $NH_3$ tạo thành.
		\begin{enumerate}
			\item Tính nồng độ mol của các chất $N_2, H_2, NH_3$ ở trạng thái cân bằng.
			\item Tính hiệu suất của phản ứng tổng hợp $NH_3$.
			\item Nếu giữ nguyên nhiệt độ và thêm $0{,}5$ mol $NH_3$ vào bình ở trạng thái cân bằng, cân bằng sẽ dịch chuyển theo chiều nào?
		\end{enumerate}
		\loigiai{
			Phản ứng: $N_2\text{(g)} + 3H_2\text{(g)} \xharpoonarrow 2NH_3\text{(g)}$
			\begin{enumerate}
				\item \textbf{Tính nồng độ mol của các chất ở trạng thái cân bằng:}
				\begin{itemize}
					\item Số mol $NH_3$ tạo thành $= 0{,}8$ mol.
					\item Từ phương trình, số mol $N_2$ đã phản ứng $= \dfrac{0{,}8}{2} = 0{,}4$ mol.
					\item Số mol $H_2$ đã phản ứng $= 3 \times 0{,}4 = 1{,}2$ mol.
				\end{itemize}
				Nồng độ ban đầu (M): $[N_2]_{\text{bđ}} = \dfrac{2}{2} = 1$ M; $[H_2]_{\text{bđ}} = \dfrac{6}{2} = 3$ M.
				Nồng độ ở trạng thái cân bằng (M):
				\begin{itemize}
					\item $[N_2]_{\text{cb}} = [N_2]_{\text{bđ}} - n_{N_2 \text{ pư}}/V = 1 - 0{,}4/2 = 1 - 0{,}2 = 0{,}8$ M. (hoặc $n_{N_2 \text{ cb}} = 2-0.4=1.6 \text{ mol} \Rightarrow [N_2]=1.6/2 = 0.8 \text{ M}$).
					\item $[H_2]_{\text{cb}} = [H_2]_{\text{bđ}} - n_{H_2 \text{ pư}}/V = 3 - 1{,}2/2 = 3 - 0{,}6 = 2{,}4$ M. (hoặc $n_{H_2 \text{ cb}} = 6-1.2=4.8 \text{ mol} \Rightarrow [H_2]=4.8/2 = 2.4 \text{ M}$).
					\item $[NH_3]_{\text{cb}} = n_{NH_3 \text{ tạo thành}}/V = 0{,}8/2 = 0{,}4$ M.
				\end{itemize}
				\item \textbf{Tính hiệu suất của phản ứng tổng hợp $NH_3$:}
				\\
				Tỉ lệ mol $N_2:H_2$ ban đầu là $2:6 = 1:3$, đúng bằng tỉ lệ phản ứng. Do đó hiệu suất có thể tính theo $N_2$ hoặc $H_2$.
				\\
				$H = \dfrac{n_{N_2 \text{ pư}}}{n_{N_2 \text{ bđ}}} \times 100\% = \dfrac{0{,}4}{2} \times 100\% = 20\%$
				\item \textbf{Ảnh hưởng của việc thêm $NH_3$:}
				\\
				Nếu giữ nguyên nhiệt độ và thêm $0{,}5$ mol $NH_3$ vào bình ở trạng thái cân bằng, nồng độ $NH_3$ trong hệ sẽ tăng lên. Theo nguyên lí Le Chatelier, cân bằng sẽ dịch chuyển theo chiều chống lại sự tăng nồng độ đó, tức là theo \textbf{chiều nghịch} (chiều phân hủy $NH_3$ thành $N_2$ và $H_2$).
			\end{enumerate}
		}
	\end{bt}
	%%%=============BT_7=============%%%
	\begin{bt}
		Người ta cho $1$ mol $N_2$ và $4$ mol $H_2$ vào bình kín có xúc tác, nung nóng ở một nhiệt độ thích hợp. Khi phản ứng đạt trạng thái cân bằng, hỗn hợp khí thu được có $0{,}8$ mol $NH_3$. Tính hiệu suất của phản ứng tổng hợp $NH_3$ theo chất phản ứng hết.
		\loigiai{
			Phản ứng tổng hợp ammonia:
			\[
			\text{N}_2\text{(g)} + 3\text{H}_2\text{(g)} \xharpoonarrow 2\text{NH}_3\text{(g)}
			\]
			\textbf{Lập bảng số mol:}
			\begin{itemize}
				\item Số mol $NH_3$ tạo thành ở trạng thái cân bằng là $0{,}8$ mol.
				\item Từ phương trình, số mol $N_2$ đã phản ứng $= \dfrac{0{,}8}{2} = 0{,}4$ mol.
				\item Số mol $H_2$ đã phản ứng $= 3 \times 0{,}4 = 1{,}2$ mol.
			\end{itemize}
			\begin{tabular}{|l|c|c|c|}
				\hline
				& $\text{N}_2$ & $3\text{H}_2$ & $2\text{NH}_3$ \\
				\hline
				\textbf{Số mol ban đầu} & $1$ & $4$ & $0$ \\
				\hline
				\textbf{Số mol phản ứng} & $0{,}4$ & $1{,}2$ & $0{,}8$ \\
				\hline
				\textbf{Số mol cân bằng} & $1 - 0{,}4 = 0{,}6$ & $4 - 1{,}2 = 2{,}8$ & $0{,}8$ \\
				\hline
			\end{tabular}
			\\
			\textbf{Xác định chất phản ứng hết (chất thiếu) để tính hiệu suất:}
			\begin{itemize}
				\item Tỉ lệ mol ban đầu: $N_2 : H_2 = 1 : 4$.
				\item Tỉ lệ mol phản ứng: $N_2 : H_2 = 1 : 3$.
				\item So sánh tỉ lệ: $\dfrac{1}{1} < \dfrac{4}{3}$ ($1 < 1.33$). Vậy $N_2$ là chất hết trước (chất thiếu). Hiệu suất sẽ tính theo $N_2$.
			\end{itemize}
			\textbf{Tính hiệu suất của phản ứng ($H$):}
			\[
			H = \dfrac{n_{N_2 \text{ phản ứng}}}{n_{N_2 \text{ ban đầu}}} \times 100\% = \dfrac{0{,}4}{1} \times 100\% = 40\%
			\]
			\textbf{Đáp số:} Hiệu suất của phản ứng tổng hợp $NH_3$ là $40\%$.
			}%
	\end{bt}

%%%=============BT_8=============%%%
\begin{bt}
	Cho cân bằng ở $1650$ $ ^\circ C$: $\text{N}_2\text{(g)} + \text{O}_2\text{(g)} \rightleftharpoons 2\text{NO}\text{(g)}$. Hằng số cân bằng $K_c = 4 \times 10^{-4}$.
	Thực hiện phản ứng trên với một hỗn hợp nitrogen và oxygen có tỉ lệ mol tương ứng là $4:1$. Tính hiệu suất của phản ứng khi hệ cân bằng ở $1650$ $ ^\circ C$.
	\loigiai{
			\begin{enumerate}
					\item \textbf{Thiết lập phương trình phản ứng và nồng độ ban đầu/cân bằng:}
					\\
					Giả sử ban đầu có $4$ mol $N_2$ và $1$ mol $O_2$ trong một bình có thể tích $V$ lít.
					\\
					Phản ứng:
					\[
					\text{N}_2\text{(g)} + \text{O}_2\text{(g)} \rightleftharpoons 2\text{NO}\text{(g)}
					\]
					\begin{tabular}{|l|c|c|c|}
							\hline
							Số mol ban đầu & $4$ & $1$ & $0$ \\
							\hline
							Số mol phản ứng & $x$ & $x$ & $2x$ \\
							\hline
							Số mol cân bằng & $4-x$ & $1-x$ & $2x$ \\
							\hline
						\end{tabular}
					\\
					Vì số mol khí không đổi trước và sau phản ứng ($1+1=2$ và $2$), nên tổng số mol khí ở trạng thái cân bằng vẫn là $4+1 = 5$ mol. Do đó, thể tích $V$ sẽ triệt tiêu trong biểu thức $K_c$.
					\\
					Nồng độ cân bằng (mol/L): $[N_2] = \dfrac{4-x}{V}$; $[O_2] = \dfrac{1-x}{V}$; $[NO] = \dfrac{2x}{V}$.
					\item \textbf{Viết biểu thức hằng số cân bằng $K_c$:}
					\[
					K_c = \dfrac{[NO]^2}{[N_2][O_2]} = \dfrac{\left(\dfrac{2x}{V}\right)^2}{\left(\dfrac{4-x}{V}\right)\left(\dfrac{1-x}{V}\right)} = \dfrac{(2x)^2}{(4-x)(1-x)}
					\]
					\item \textbf{Giải phương trình để tìm $x$:}
					\\
					$4 \times 10^{-4} = \dfrac{4x^2}{4 - 5x + x^2}$
					\\
					Vì $K_c$ rất nhỏ ($4 \times 10^{-4}$), điều này cho thấy hiệu suất phản ứng rất thấp, tức là $x$ rất nhỏ so với $4$ và $1$. Do đó, ta có thể xấp xỉ $(4-x) \approx 4$ và $(1-x) \approx 1$.
					\begin{align*}
						4 \times 10^{-4} &\approx \dfrac{4x^2}{4 \times 1} \\
						4 \times 10^{-4} &\approx x^2 \\
						x &\approx \sqrt{4 \times 10^{-4}} = 2 \times 10^{-2} = 0{,}02 \text{ mol}
					\end{align*}
					Kiểm tra xấp xỉ: $x=0{,}02$ mol thực sự rất nhỏ so với $1$ và $4$, nên xấp xỉ là hợp lệ.
					\item \textbf{Tính hiệu suất của phản ứng:}

					Hiệu suất của phản ứng được tính dựa trên chất phản ứng hết trước hoặc chất thiếu.
					\\
					Tỉ lệ mol ban đầu $N_2:O_2 = 4:1$. Tỉ lệ mol phản ứng $N_2:O_2 = 1:1$.
					\\
					$O_2$ sẽ là chất hết trước vì $1/1 < 4/1$. Do đó, hiệu suất tính theo $O_2$.
				\begin{align*}
					\text{Hiệu suất } (H) &= \dfrac{\text{Số mol } O_2 \text{ đã phản ứng}}{\text{Số mol } O_2 \text{ ban đầu}} \times 100\% \\
					&= \dfrac{x}{1} \times 100\% \\
					&= \dfrac{0{,}02}{1} \times 100\% = 2\%
				\end{align*}
				\textbf{Đáp số:} Hiệu suất của phản ứng khi hệ cân bằng ở $1650$ $ ^\circ C$ là $2\%$.
				\end{enumerate}
		}
\end{bt}
	\Closesolutionfile{ansbt}
	\Closesolutionfile{ansbth}
%%	
\phan{Bài tập trả lời ngắn}
	%%%=============SOẠN BT===============%%%
	\Opensolutionfile{ansbth}[Ans/LGSA-C02B05_CanBang]
	\Opensolutionfile{ansbt}[Ans/AnsSA-C02B05_CanBang]
	%%%=============SA_1=============%%%
	\begin{bt}
		Biến thiên enthalpy của phản ứng tổng hợp ammonia ($N_2\text{(g)} + 3H_2\text{(g)} \xharpoonarrow 2NH_3\text{(g)}$) là bao nhiêu kJ/mol (theo $NH_3$)? (Lấy giá trị $\Delta_rH^\circ = -92 \text{ kJ}$ cho phản ứng).
		\shortans{$-46$}
		\loigiai{
			Phản ứng $N_2\text{(g)} + 3H_2\text{(g)} \rightleftharpoons 2NH_3\text{(g)}$ có $\Delta_rH^\circ = -92 \text{ kJ}$ nghĩa là khi $2$ mol $NH_3$ tạo thành thì tỏa ra $92$ kJ.
			Vậy, nhiệt tỏa ra cho $1$ mol $NH_3$ là $\dfrac{-92 \text{ kJ}}{2 \text{ mol}} = -46 \text{ kJ/mol}$.
		}
	\end{bt}
	%%%=============SA_2=============%%%
	\begin{bt}
		Ở trạng thái cân bằng, nếu nồng độ $N_2$ tăng lên, cân bằng của phản ứng $N_2\text{(g)} + 3H_2\text{(g)} \xharpoonarrow 2NH_3\text{(g)}$ sẽ dịch chuyển theo chiều nào? (Trả lời: Chiều thuận / Chiều nghịch / Không dịch chuyển)
		\shortans{Chiều thuận}
		\loigiai{
			Theo nguyên lí Le Chatelier, khi tăng nồng độ chất phản ứng ($N_2$), cân bằng sẽ dịch chuyển theo chiều tạo sản phẩm (chiều thuận) để giảm bớt nồng độ $N_2$ đã tăng.
		}
	\end{bt}
	%%%=============SA_3=============%%%
	\begin{bt}
		Nếu hằng số cân bằng $K_c$ của phản ứng $N_2\text{(g)} + 3H_2\text{(g)} \rightleftharpoons 2NH_3\text{(g)}$ ở một nhiệt độ nhất định là $0{,}01$, thì nồng độ cân bằng của $NH_3$ bằng bao nhiêu khi $[N_2]=0{,}1 \text{ M}$ và $[H_2]=0{,}3 \text{ M}$?
		\shortans{$0.00016$}
		\loigiai{
			Công thức hằng số cân bằng: $K_c = \dfrac{[NH_3]^2}{[N_2][H_2]^3}$
			Ta có: $0{,}01 = \dfrac{[NH_3]^2}{(0{,}1) \times (0{,}3)^3}$
			$0{,}01 = \dfrac{[NH_3]^2}{0{,}1 \times 0{,}027}$
			$0{,}01 = \dfrac{[NH_3]^2}{0{,}0027}$
			$[NH_3]^2 = 0{,}01 \times 0{,}0027 = 0{,}000027$
			$[NH_3] = \sqrt{0{,}000027} \approx 0{,}005196 \text{ M}$
		}
	\end{bt}
	%%%=============SA_4=============%%%
	\begin{bt}
		Phản ứng tổng hợp $NH_3$ ($N_2 + 3H_2 \rightleftharpoons 2NH_3$) có $\Delta_rH^\circ = -92 \text{ kJ}$. Nếu nhiệt độ tăng lên, giá trị hằng số cân bằng $K_c$ sẽ thay đổi như thế nào? (Trả lời: Tăng / Giảm / Không đổi)
		\shortans{Giảm}
		\loigiai{
			Phản ứng tổng hợp $NH_3$ là phản ứng tỏa nhiệt ($\Delta_rH^\circ < 0$). Khi nhiệt độ tăng, cân bằng sẽ dịch chuyển theo chiều nghịch (chiều thu nhiệt) để chống lại sự tăng nhiệt độ. Sự dịch chuyển cân bằng theo chiều nghịch có nghĩa là nồng độ sản phẩm giảm và nồng độ chất phản ứng tăng, do đó giá trị hằng số cân bằng $K_c$ sẽ giảm.
		}
	\end{bt}
	
	%%%=============SA_5=============%%%
	\begin{bt}
		Nếu hiệu suất của phản ứng tổng hợp $NH_3$ là $25\%$, và ban đầu có $2$ mol $N_2$ và $6$ mol $H_2$. Tính số mol $NH_3$ tạo thành ở trạng thái cân bằng.
		\shortans{1}
		\loigiai{
			Phản ứng: $N_2\text{(g)} + 3H_2\text{(g)} \rightleftharpoons 2NH_3\text{(g)}$
			\\
			Tỉ lệ mol $N_2:H_2$ ban đầu là $2:6 = 1:3$, đúng bằng tỉ lệ phản ứng. Hiệu suất có thể tính theo $N_2$ hoặc $H_2$.
			\\
			Số mol $N_2$ đã phản ứng: $n_{N_2 \text{ pư}} = n_{N_2 \text{ ban đầu}} \times H = 2 \text{ mol} \times 25\% = 2 \times 0{,}25 = 0{,}5$ mol.
			\\
			Số mol $NH_3$ tạo thành: $n_{NH_3 \text{ tạo thành}} = 2 \times n_{N_2 \text{ pư}} = 2 \times 0{,}5 = 1{,}0$ mol.
		}
	\end{bt}
	
	%%%=============SA_6=============%%%
	\begin{bt}
		Phản ứng $N_2\text{(g)} + 3H_2\text{(g)} \rightleftharpoons 2NH_3\text{(g)}$. Nếu áp suất tổng cộng của hệ tăng lên, tốc độ phản ứng thuận sẽ thay đổi như thế nào? (Trả lời: Tăng / Giảm / Không đổi)
		\shortans{Tăng}
		\loigiai{
			Tăng áp suất tổng cộng của hệ khí (bằng cách giảm thể tích hoặc tăng số mol khí) sẽ làm tăng nồng độ của tất cả các chất khí. Việc tăng nồng độ chất phản ứng (cả $N_2$ và $H_2$) sẽ làm tăng số va chạm hiệu quả giữa các phân tử, do đó làm tăng tốc độ của cả phản ứng thuận và phản ứng nghịch.
		}
	\end{bt}
	
	%%%=============SA_7=============%%%
	\begin{bt}
		Trong phản ứng $N_2\text{(g)} + 3H_2\text{(g)} \rightleftharpoons 2NH_3\text{(g)}$, tổng số mol khí ở vế chất phản ứng là $A$ và ở vế sản phẩm là $B$. Tính giá trị của $A-B$.
		\shortans{2}
		\loigiai{
			Tổng số mol khí ở vế chất phản ứng ($A$) là $1 \text{ mol } N_2 + 3 \text{ mol } H_2 = 4 \text{ mol}$.
			Tổng số mol khí ở vế sản phẩm ($B$) là $2 \text{ mol } NH_3 = 2 \text{ mol}$.
			Vậy, $A-B = 4 - 2 = 2$.
		}
	\end{bt}
	
	%%%=============SA_8=============%%%
	\begin{bt}
		Hỗn hợp $N_2$ và $H_2$ có tỉ lệ mol $1:3$ được đưa vào bình phản ứng. Nếu ở trạng thái cân bằng, $0{,}4$ mol $N_2$ đã phản ứng, thì số mol $H_2$ đã phản ứng là bao nhiêu?
		\shortans{1.2}
		\loigiai{
			Phản ứng: $N_2\text{(g)} + 3H_2\text{(g)} \rightleftharpoons 2NH_3\text{(g)}$
			\\
			Theo tỉ lệ mol phản ứng, cứ $1$ mol $N_2$ phản ứng thì $3$ mol $H_2$ phản ứng.
			\\
			Nếu $0{,}4$ mol $N_2$ đã phản ứng, thì số mol $H_2$ đã phản ứng là $3 \times 0{,}4 = 1{,}2$ mol.
		}
	\end{bt}
	\Closesolutionfile{ansbt}
	\Closesolutionfile{ansbth}
	\phan{Trắc nghiệm nhiều lựa chọn}
	%%%=============SOẠN EX===============%%%
	\Opensolutionfile{ansex}[Ans/LGEX-C02B05_CanBang]
	\Opensolutionfile{ans}[Ans/Ans-C02B05_CanBang]
	%%%=============EX_1=============%%%
	\begin{ex}
		Để tăng hiệu suất cân bằng của phản ứng tổng hợp ammonia ($N_2 + 3H_2 \rightleftharpoons 2NH_3$, $\Delta_rH^\circ = -92 \text{ kJ}$), cần áp dụng yếu tố nào sau đây?
		\choice
		{Tăng nhiệt độ.}
		{Giảm áp suất.}
		{Sử dụng chất xúc tác Fe.}
		{\True Tăng áp suất và giảm nhiệt độ (tối ưu).}
		\loigiai{
			\begin{itemize}
				\item Phản ứng tỏa nhiệt, nên giảm nhiệt độ sẽ làm cân bằng dịch chuyển chiều thuận, tăng hiệu suất.
				\item Phản ứng có số mol khí giảm (4 mol $\rightarrow$ 2 mol), nên tăng áp suất sẽ làm cân bằng dịch chuyển chiều thuận, tăng hiệu suất.
				\item Xúc tác Fe chỉ tăng tốc độ đạt cân bằng, không làm thay đổi hiệu suất cân bằng.
			\end{itemize}
			Do đó, tăng áp suất và giảm nhiệt độ là các yếu tố làm tăng hiệu suất cân bằng.
		}
	\end{ex}
	%%%=============EX_2=============%%%
	\begin{ex}
		Trong quá trình tổng hợp ammonia, việc liên tục tách ammonia ra khỏi hỗn hợp khí sau phản ứng có tác dụng gì?
		\choice
		{Làm giảm tốc độ phản ứng.}
		{Làm tăng nhiệt độ của hệ.}
		{Làm giảm áp suất tổng cộng của hệ.}
		{\True Làm cân bằng dịch chuyển theo chiều thuận, tăng lượng $NH_3$ tạo thành.}
		\loigiai{
			Khi $NH_3$ được tách ra, nồng độ $NH_3$ trong hệ giảm. Theo nguyên lí Le Chatelier, cân bằng sẽ dịch chuyển theo chiều tạo thêm $NH_3$ (chiều thuận) để bù đắp lại lượng đã mất, từ đó tăng lượng $NH_3$ tổng cộng thu được.
		}
	\end{ex}
	%%%=============EX_3=============%%%
	\begin{ex}
		Nếu tỉ lệ mol $N_2 : H_2$ ban đầu không phải là $1:3$ mà là $1:4$, thì hiệu suất phản ứng tổng hợp $NH_3$ sẽ được tính theo chất nào?
		\choice
		{$N_2$}
		{$H_2$}
		{Cả $N_2$ và $H_2$ đều như nhau.}
		{\True $N_2$ (chất thiếu).}
		\loigiai{
			Tỉ lệ phản ứng $N_2 : H_2 = 1:3$.
			Nếu tỉ lệ ban đầu là $1:4$, ta có:
			$\dfrac{n_{N_2}}{1} = \dfrac{1}{1} = 1$
			$\dfrac{n_{H_2}}{3} = \dfrac{4}{3} \approx 1{,}33$
			Vì $1 < 1{,}33$, $N_2$ sẽ là chất hết trước (chất thiếu), do đó hiệu suất phản ứng phải tính theo $N_2$.
		}
	\end{ex}
	%%%=============EX_4=============%%%
	\begin{ex}
		Nhận định nào sau đây là đúng về vai trò của xúc tác trong phản ứng tổng hợp $NH_3$?
		\choice
		{Làm tăng hiệu suất cân bằng của phản ứng.}
		{Làm dịch chuyển cân bằng theo chiều thuận.}
		{Làm tăng nồng độ sản phẩm ở trạng thái cân bằng.}
		{\True Làm tăng tốc độ đạt đến trạng thái cân bằng.}
		\loigiai{
			Chất xúc tác làm tăng tốc độ của cả phản ứng thuận và nghịch, giúp hệ đạt đến trạng thái cân bằng nhanh hơn. Xúc tác không làm thay đổi vị trí cân bằng, do đó không ảnh hưởng đến hiệu suất cân bằng hay nồng độ các chất ở trạng thái cân bằng cuối cùng.
		}
	\end{ex}
	%%=============EX_5=============%%%
	\begin{ex}
		Yếu tố nào sau đây có tác dụng làm tăng tốc độ phản ứng tổng hợp ammonia ($N_2 + 3H_2 \rightleftharpoons 2NH_3$) nhưng không làm thay đổi vị trí cân bằng?
		\choice
		{Giảm nhiệt độ.}
		{Tăng nồng độ $NH_3$.}
		{Giảm áp suất.}
		{\True Sử dụng chất xúc tác Fe.}
		\loigiai{
				Chất xúc tác (Fe) làm tăng tốc độ của cả phản ứng thuận và nghịch lên mức độ như nhau, do đó giúp phản ứng đạt trạng thái cân bằng nhanh hơn nhưng không làm thay đổi vị trí cân bằng hay hiệu suất cân bằng. Các yếu tố còn lại đều làm dịch chuyển cân bằng.
			}
	\end{ex}
	
	%%%=============EX_6=============%%%
	\begin{ex}
		Phản ứng tổng hợp ammonia là phản ứng tỏa nhiệt và có số mol khí giảm. Điều kiện nào sau đây có thể giúp đạt được hiệu suất $NH_3$ cao nhất về mặt cân bằng?
		\choice
		{Nhiệt độ cao, áp suất thấp.}
		{Nhiệt độ thấp, áp suất thấp.}
		{Nhiệt độ cao, áp suất cao.}
		{\True Nhiệt độ thấp, áp suất cao.}
		\loigiai{
				\begin{itemize}
						\item Vì phản ứng tỏa nhiệt ($\Delta_rH^\circ < 0$), nhiệt độ thấp sẽ làm cân bằng dịch chuyển chiều thuận, tăng hiệu suất.
						\item Vì số mol khí giảm (4 mol $\rightarrow$ 2 mol), áp suất cao sẽ làm cân bằng dịch chuyển chiều thuận, tăng hiệu suất.
					\end{itemize}
				Do đó, nhiệt độ thấp và áp suất cao sẽ cho hiệu suất cân bằng cao nhất.
			}
	\end{ex}
	
	%%%=============EX_7=============%%%
	\begin{ex}
		Khi tăng áp suất của hệ phản ứng $N_2\text{(g)} + 3H_2\text{(g)} \rightleftharpoons 2NH_3\text{(g)}$, cân bằng sẽ dịch chuyển theo chiều nào?
		\choice
		{Chiều nghịch.}
		{Không dịch chuyển.}
		{Chiều làm tăng số mol khí.}
		{\True Chiều làm giảm số mol khí.}
		\loigiai{
				Theo nguyên lí Le Chatelier, khi tăng áp suất, cân bằng sẽ dịch chuyển theo chiều có tổng số mol khí giảm. Trong phản ứng này, vế thuận có $1+3=4$ mol khí, vế nghịch có $2$ mol khí. Do đó, cân bằng dịch chuyển theo chiều thuận (chiều làm giảm số mol khí).
			}
	\end{ex}
	
	%%%=============EX_8=============%%%
	\begin{ex}
		Nếu tỉ lệ mol $N_2 : H_2$ ban đầu là $1:2$, thì hiệu suất phản ứng tổng hợp $NH_3$ sẽ được tính theo chất nào?
		\choice
		{$N_2$}
		{\True $H_2$}
		{Cả $N_2$ và $H_2$ đều như nhau.}
		{Không thể xác định.}
		\loigiai{
				Tỉ lệ phản ứng $N_2 : H_2 = 1:3$.
				Tỉ lệ mol ban đầu là $1:2$.
				So sánh tỉ lệ mol ban đầu với tỉ lệ mol phản ứng:
				Đối với $N_2$: $\dfrac{1}{1} = 1$
				Đối với $H_2$: $\dfrac{2}{3} \approx 0{,}67$
				Vì $0{,}67 < 1$, $H_2$ sẽ là chất hết trước (chất thiếu). Do đó, hiệu suất phản ứng phải tính theo $H_2$.
			}
	\end{ex}
	
	%%%=============EX_9=============%%%
		\begin{ex}
				Lý do chính khiến người ta phải sử dụng áp suất cao (ví dụ $200$ bar) trong quá trình tổng hợp ammonia là gì?
				\choice
				{Để tăng khả năng chống nổ của thiết bị.}
				{Để giảm nhiệt độ cần thiết cho phản ứng.}
				{Để tăng độ bền của chất xúc tác Fe.}
				{\True Để dịch chuyển cân bằng theo chiều tạo sản phẩm và tăng hiệu suất phản ứng.}
				\loigiai{
						Áp suất cao giúp dịch chuyển cân bằng theo chiều tạo sản phẩm ($NH_3$) do số mol khí giảm từ 4 xuống 2, từ đó tăng hiệu suất phản ứng. Đây là ứng dụng trực tiếp của nguyên lí Le Chatelier.
					}
			\end{ex}
	%%%=============EX_10=============%%%
		\begin{ex}
			Cho các nhận định sau về phân tử ammonia ($NH_3$) và ion ammonium ($NH_4^+$):
			\begin{enumerate}
				\item Chứa liên kết cộng hóa trị.
				\item Là base Brønsted trong nước.
				\item Là acid Brønsted trong nước.
				\item Chứa nguyên tử N có số oxi hóa là $-3$.
			\end{enumerate}
			Số nhận định đúng là:
			\choice
			{\True 2}
			{1}
			{4}
			{3}
			\loigiai{
				Ta xét từng nhận định xem nó có đúng cho \textbf{cả} phân tử ammonia ($NH_3$) và ion ammonium ($NH_4^+$) hay không:
				\begin{itemize}
					\item \textbf{(1) Chứa liên kết cộng hóa trị:}
					\begin{itemize}
						\item Trong phân tử $NH_3$: Có $3$ liên kết cộng hóa trị $N-H$.
						\item Trong ion $NH_4^+$: Có $4$ liên kết cộng hóa trị $N-H$.
					\end{itemize}
					$\Rightarrow$ Nhận định (1) là \textbf{đúng} cho cả hai.
					\item \textbf{(2) Là base Brønsted trong nước:}
					\begin{itemize}
						\item Trong nước, $NH_3$ là base Brønsted (nhận $H^+$): $\text{NH}_3 + \text{H}_2\text{O} \rightleftharpoons \text{NH}_4^+ + \text{OH}^-$.
						\item Trong nước, $NH_4^+$ là acid Brønsted (nhường $H^+$): $\text{NH}_4^+ + \text{H}_2\text{O} \rightleftharpoons \text{NH}_3 + \text{H}_3\text{O}^+$.
					\end{itemize}
					$\Rightarrow$ Nhận định (2) là \textbf{sai} vì $NH_4^+$ không phải là base Brønsted.
					\item \textbf{(3) Là acid Brønsted trong nước:}
					\begin{itemize}
						\item Trong nước, $NH_3$ là base Brønsted.
						\item Trong nước, $NH_4^+$ là acid Brønsted.
					\end{itemize}
					$\Rightarrow$ Nhận định (3) là \textbf{sai} vì $NH_3$ không phải là acid Brønsted.
					\item \textbf{(4) Chứa nguyên tử N có số oxi hóa là $-3$:}
					\begin{itemize}
						\item Trong $NH_3$: $x + 3 \times (+1) = 0 \Rightarrow x = -3$.
						\item Trong $NH_4^+$: $x + 4 \times (+1) = +1 \Rightarrow x = -3$.
					\end{itemize}
					$\Rightarrow$ Nhận định (4) là \textbf{đúng} cho cả hai.
				\end{itemize}
				Vậy có $2$ nhận định đúng cho cả hai là (1) và (4).
			}
		\end{ex}
		%%%=============EX_11=============%%%
		\begin{ex}
			Các chất khí được thu vào bình theo đúng nguyên tắc bằng cách đẩy không khí (X, Y, Z) và đẩy nước (T) như sau:
		Nhận xét nào sau đây không đúng?
		\choice
		{X là chlorine ($Cl_2$).}
		{Y là hydrogen ($H_2$).}
		{Z là nitrogen dioxide ($NO_2$).}
		{\True T là ammonia ($NH_3$).}
		\loigiai{
			Để đánh giá nhận xét nào không đúng, ta xét tính chất của từng khí và phương pháp thu:
			\begin{itemize}
				\item \textbf{Khí X, Z (thu bằng đẩy không khí lên):} Phương pháp này dùng cho khí nặng hơn không khí.
				\begin{itemize}
					\item A. Chlorine ($Cl_2$, $M=71$) nặng hơn không khí ($M_{\text{kk}}=29$). Nhận xét \textbf{đúng}.
					\item C. Nitrogen dioxide ($NO_2$, $M=46$) nặng hơn không khí. Nhận xét \textbf{đúng}.
				\end{itemize}
				\item \textbf{Khí Y (thu bằng đẩy không khí xuống):} Phương pháp này dùng cho khí nhẹ hơn không khí.
				\begin{itemize}
					\item B. Hydrogen ($H_2$, $M=2$) nhẹ hơn không khí. Nhận xét \textbf{đúng}.
				\end{itemize}
				\item \textbf{Khí T (thu bằng đẩy nước):} Phương pháp này dùng cho khí ít tan hoặc không tan trong nước.
				\begin{itemize}
					\item D. Ammonia ($NH_3$, $M=17$) rất nhẹ hơn không khí, nhưng điều quan trọng là $NH_3$ \textbf{tan rất nhiều trong nước} (khoảng $700$ lít $NH_3$ tan trong $1$ lít nước ở điều kiện thường). Do đó, $NH_3$ không thể thu được bằng phương pháp đẩy nước. Nhận xét \textbf{không đúng}.
				\end{itemize}
			\end{itemize}
		}
		\end{ex}
	\Closesolutionfile{ans}
	\Closesolutionfile{ansex}
	
\phan{Bài tập đúng sai}
\Opensolutionfile{ansex}[Ans/LGTF-C02B04_CBHH]
\Opensolutionfile{ansbook}[Ansbook/AnsTF-C02B04_CBHH]
\Opensolutionfile{ans}[Ans/Tempt-C02B04_CBHH]
	%%%=============TF_3=============%%%
	\begin{ex}
		Đánh giá tính đúng/sai của các phát biểu sau về cân bằng hóa học:
		\choiceTF
		{\True Phản ứng $N_2\text{(g)} + 3H_2\text{(g)} \rightleftharpoons 2NH_3\text{(g)}$ là phản ứng tỏa nhiệt.}
		{Ở trạng thái cân bằng, nồng độ các chất phản ứng và sản phẩm luôn bằng nhau.}
		{\True Hằng số cân bằng $K_c$ chỉ phụ thuộc vào nhiệt độ.}
		{Nếu hiệu suất phản ứng là $100\%$, thì không có chất nào còn lại trong bình phản ứng.}
		\loigiai{
			\begin{itemchoice}[T1,F2,T3,F4]
				\itemch Phản ứng có $\Delta_rH^\circ = -92 \text{ kJ}$, là phản ứng tỏa nhiệt. Phát biểu đúng.
				\itemch Ở trạng thái cân bằng, nồng độ các chất phản ứng và sản phẩm đạt giá trị không đổi theo thời gian, nhưng \textbf{không nhất thiết phải bằng nhau}. Phát biểu sai.
				\itemch Hằng số cân bằng $K_c$ là một hằng số chỉ phụ thuộc vào bản chất của phản ứng và nhiệt độ. Phát biểu đúng.
				\itemch Nếu hiệu suất phản ứng là $100\%$, thì \textbf{ít nhất một trong các chất phản ứng ban đầu sẽ hết} (hoàn toàn chuyển hóa thành sản phẩm). Tuy nhiên, nếu có chất phản ứng dư hoặc có sản phẩm không phải là chất khí, hoặc nếu sản phẩm được tách ra, thì vẫn có thể có chất còn lại trong bình. Nếu chỉ có chất phản ứng và sản phẩm khí, thì nếu phản ứng $100\%$, chất phản ứng chính sẽ hết, nhưng sản phẩm vẫn còn. Câu này diễn giải hơi mơ hồ, nhưng ý chính là không phải tất cả các chất trong bình đều biến mất. Phát biểu sai.
			\end{itemchoice}
		}
	\end{ex}
	
	%%%=============TF_4=============%%%
	\begin{ex}
		Đánh giá tính đúng/sai của các phát biểu sau về các yếu tố ảnh hưởng đến phản ứng tổng hợp $NH_3$:
		\choiceTF
		{Tăng nồng độ $NH_3$ sẽ làm cân bằng dịch chuyển theo chiều thuận.}
		{\True Giảm thể tích bình phản ứng sẽ làm tăng hiệu suất tổng hợp $NH_3$.}
		{Nếu phản ứng $N_2 + 3H_2 \rightleftharpoons 2NH_3$ có hiệu suất $0\%$, điều đó có nghĩa là không có $NH_3$ nào được tạo thành.}
		{\True Xúc tác Fe làm tăng tốc độ phản ứng thuận và nghịch với mức độ như nhau.}
		\loigiai{
			\begin{itemchoice}[F1,T2,F3,T4]
				\itemch Tăng nồng độ sản phẩm ($NH_3$) làm cân bằng dịch chuyển theo chiều nghịch. Phát biểu sai.
				\itemch Giảm thể tích bình phản ứng làm tăng áp suất. Tăng áp suất làm cân bằng dịch chuyển theo chiều giảm số mol khí (chiều thuận), tăng hiệu suất. Phát biểu đúng.
				\itemch Nếu hiệu suất là $0\%$, điều đó có nghĩa là \textbf{không có chất phản ứng nào chuyển hóa thành sản phẩm}, tức là không có $NH_3$ được tạo thành. Phát biểu đúng. (Lưu ý: phát biểu này ban đầu tôi đánh giá là sai do nhìn nhầm ý của câu, nhưng 0% hiệu suất là không có sản phẩm).
				\itemch Xúc tác làm tăng tốc độ của cả phản ứng thuận và nghịch lên cùng một mức độ. Phát biểu đúng.
			\end{itemchoice}
		}
	\end{ex}
	
	%%%=============TF_5=============%%%
	\begin{ex}
		Đánh giá tính đúng/sai của các phát biểu sau về điều kiện của phản ứng Haber:
		\choiceTF
		{\True Phản ứng tổng hợp ammonia được thực hiện ở nhiệt độ cao để đảm bảo tốc độ phản ứng đủ nhanh.}
		{Việc liên tục tách ammonia ra khỏi hỗn hợp sản phẩm làm giảm hiệu suất của phản ứng.}
		{\True Tăng áp suất là biện pháp hiệu quả để tăng hiệu suất của phản ứng tổng hợp ammonia.}
		{Nếu nhiệt độ rất thấp, hiệu suất cân bằng của phản ứng tổng hợp ammonia sẽ rất cao, nhưng không kinh tế vì tốc độ phản ứng quá chậm.}
		\loigiai{
			\begin{itemchoice}[T1,F2,T3,T4]
				\itemch Mặc dù phản ứng tỏa nhiệt, nhiệt độ cao giúp tăng tốc độ phản ứng, điều này quan trọng cho sản xuất công nghiệp. Phát biểu đúng.
				\itemch Tách $NH_3$ ra khỏi hệ làm giảm nồng độ sản phẩm, đẩy cân bằng theo chiều thuận, do đó \textbf{tăng hiệu suất} tổng cộng thu được $NH_3$. Phát biểu sai.
				\itemch Phản ứng có số mol khí giảm, tăng áp suất đẩy cân bằng chiều thuận, tăng hiệu suất. Phát biểu đúng.
				\itemch Nhiệt độ rất thấp có thể làm hiệu suất cân bằng cao (vì là phản ứng tỏa nhiệt), nhưng tốc độ phản ứng sẽ cực kỳ chậm, không đáp ứng yêu cầu sản xuất công nghiệp, nên không kinh tế. Phát biểu đúng.
			\end{itemchoice}
		}
	\end{ex}
	%%%=============TF_6=============%%%
	\begin{ex}
		Đánh giá tính đúng/sai của các phát biểu sau:
		\choiceTF
		{\True Quá trình Haber để tổng hợp ammonia là một trong những quy trình hóa học quy mô lớn nhất thế giới.}
		{Trong công nghiệp, tỉ lệ mol $N_2 : H_2$ luôn được giữ chính xác $1:3$ để tối ưu hóa hiệu suất.}
		{\True Nhiệt độ sôi của $NH_3$ cao hơn nhiều so với $N_2$ và $H_2$, cho phép tách $NH_3$ bằng hóa lỏng.}
		{Nếu hiệu suất phản ứng là $50\%$, điều đó có nghĩa là một nửa lượng $N_2$ và một nửa lượng $H_2$ ban đầu đã phản ứng.}
		\loigiai{
			\begin{itemchoice}[T1,F2,T3,F4]
				\itemch Quá trình Haber là một trong những quy trình hóa học quan trọng và quy mô lớn nhất. Phát biểu đúng.
				\itemch Trong công nghiệp, tỉ lệ $N_2:H_2$ thường được giữ xấp xỉ $1:3$ (có thể hơi lệch một chút để tối ưu hóa việc tái chế và kiểm soát dư thừa), nhưng không phải "luôn chính xác $1:3$" và "để tối ưu hóa hiệu suất" có nhiều yếu tố, không chỉ tỉ lệ mol. Việc giữ tỉ lệ $1:3$ giúp tận dụng tối đa cả hai nguyên liệu nếu phản ứng hoàn toàn, nhưng thực tế hiệu suất không bao giờ $100\%$. Phát biểu sai.
				\itemch Nhiệt độ sôi của $NH_3$ là $-33$ $ ^\circ C$, trong khi $N_2$ là $-196$ $ ^\circ C$ và $H_2$ là $-253$ $ ^\circ C$. Sự chênh lệch này cho phép làm lạnh để hóa lỏng $NH_3$ mà $N_2$ và $H_2$ vẫn ở thể khí. Phát biểu đúng.
				\itemch Nếu hiệu suất là $50\%$ và tỉ lệ ban đầu $N_2:H_2$ là $1:3$, điều đó có nghĩa là $50\%$ lượng \textbf{chất thiếu} đã phản ứng. Không phải lúc nào cũng là một nửa của cả $N_2$ và $H_2$. Ví dụ, nếu $N_2$ dư, thì $50\%$ của $H_2$ sẽ phản ứng. Nếu tỉ lệ ban đầu là $1:3$, thì $50\%$ của cả $N_2$ và $H_2$ sẽ phản ứng. Nhưng phát biểu không nêu rõ tỉ lệ ban đầu. Nếu phát biểu tổng quát là "một nửa lượng N2 và H2 ban đầu" thì sai vì chỉ tính theo chất thiếu. Phát biểu sai.
			\end{itemchoice}
		}
	\end{ex}
	
	%%%=============TF_7=============%%%
	\begin{ex}
		Đánh giá tính đúng/sai của các phát biểu sau về các yếu tố ảnh hưởng đến cân bằng hóa học:
		\choiceTF
		{Nếu hằng số cân bằng $K_c$ của phản ứng tổng hợp ammonia rất lớn, điều đó có nghĩa là phản ứng thuận là phản ứng chính và sản phẩm $NH_3$ tạo thành rất nhiều.}
		{\True Việc tăng nhiệt độ sẽ làm tăng động năng của các phân tử, từ đó làm tăng số va chạm hiệu quả và tăng tốc độ phản ứng.}
		{Chất xúc tác chỉ làm tăng tốc độ phản ứng thuận, không ảnh hưởng đến tốc độ phản ứng nghịch.}
		{\True Đối với phản ứng có số mol khí giảm, việc tăng áp suất sẽ làm tăng nồng độ của tất cả các chất khí trong hệ, đẩy nhanh tốc độ phản ứng.}
		\loigiai{
			\begin{itemchoice}[T1,T2,F3,T4]
				\itemch Nếu $K_c$ rất lớn, điều đó có nghĩa là ở trạng thái cân bằng, nồng độ sản phẩm rất lớn so với chất phản ứng, tức là phản ứng thuận là phản ứng chính và hiệu suất rất cao. Phát biểu đúng.
				\itemch Tăng nhiệt độ làm tăng động năng phân tử, tăng tần số va chạm và tỉ lệ va chạm hiệu quả, do đó tăng tốc độ phản ứng. Phát biểu đúng.
				\itemch Xúc tác làm tăng tốc độ của \textbf{cả} phản ứng thuận và nghịch. Phát biểu sai.
				\itemch Tăng áp suất (bằng cách giảm thể tích) sẽ làm tăng nồng độ (áp suất riêng phần) của tất cả các chất khí, điều này làm tăng tốc độ phản ứng (cả thuận và nghịch). Phát biểu đúng.
			\end{itemchoice}
		}
	\end{ex}
	%%%=============TF_8=============%%%
	\begin{ex}
		Đánh giá tính đúng/sai của các phát biểu sau về cân bằng trong tổng hợp $NH_3$:
		\choiceTF
		{\True Giảm nhiệt độ sẽ làm tăng hiệu suất cân bằng của phản ứng tổng hợp $NH_3$.}
		{Tăng áp suất sẽ làm cân bằng dịch chuyển theo chiều giảm số mol khí, nhưng không làm thay đổi hiệu suất.}
		{\True Tăng nồng độ $H_2$ sẽ làm cân bằng dịch chuyển theo chiều thuận, tăng lượng $NH_3$ tạo thành.}
		{Nếu phản ứng đạt trạng thái cân bằng, số mol $N_2$ và $H_2$ không còn thay đổi.}
		\loigiai{
			\begin{itemchoice}[T1,F2,T3,F4]
				\itemch Phản ứng tỏa nhiệt, giảm nhiệt độ làm cân bằng dịch chuyển chiều thuận, tăng hiệu suất. Phát biểu đúng.
				\itemch Tăng áp suất làm cân bằng dịch chuyển chiều giảm số mol khí (chiều thuận), do đó làm \textbf{tăng hiệu suất} tạo $NH_3$. Phát biểu sai.
				\itemch Tăng nồng độ chất phản ứng ($H_2$) làm cân bằng dịch chuyển chiều thuận, tăng sản phẩm $NH_3$. Phát biểu đúng.
				\itemch Ở trạng thái cân bằng, nồng độ (hoặc số mol) của các chất phản ứng và sản phẩm \textbf{không thay đổi theo thời gian}, nhưng phản ứng thuận và nghịch vẫn diễn ra với tốc độ bằng nhau. Phát biểu sai.
			\end{itemchoice}
		}
	\end{ex}
	%%%=============TF_9=============%%%
	\begin{ex}
		Đánh giá tính đúng/sai của các phát biểu sau về các yếu tố ảnh hưởng đến phản ứng tổng hợp $NH_3$:
		\choiceTF
		{\True Phản ứng $N_2\text{(g)} + 3H_2\text{(g)} \rightleftharpoons 2NH_3\text{(g)}$ là phản ứng thuận nghịch.}
		{Xúc tác Fe làm tăng năng lượng hoạt hóa của phản ứng.}
		{\True Để sản xuất $NH_3$ trong công nghiệp, cần thực hiện ở nhiệt độ cao để tăng tốc độ phản ứng.}
		{Nếu hằng số cân bằng $K_c$ của phản ứng rất nhỏ, điều đó có nghĩa là hiệu suất phản ứng rất cao.}
		\loigiai{
			\begin{itemchoice}[T1,F2,T3,F4]
				\itemch Mũi tên hai chiều ($\rightleftharpoons$) biểu thị phản ứng thuận nghịch. Phát biểu đúng.
				\itemch Xúc tác làm \textbf{giảm} năng lượng hoạt hóa của phản ứng, giúp phản ứng diễn ra nhanh hơn. Phát biểu sai.
				\itemch Mặc dù phản ứng tỏa nhiệt (nên nhiệt độ thấp lợi hơn về hiệu suất), nhưng ở nhiệt độ thường, tốc độ phản ứng quá chậm do liên kết ba $N \equiv N$ bền. Do đó, cần nhiệt độ cao để tăng tốc độ phản ứng. Phát biểu đúng.
				\itemch Nếu hằng số cân bằng $K_c$ rất nhỏ, điều đó có nghĩa là ở trạng thái cân bằng, nồng độ sản phẩm rất thấp so với nồng độ chất phản ứng, tức là hiệu suất phản ứng \textbf{rất thấp}. Phát biểu sai.
			\end{itemchoice}
		}
	\end{ex}
	\Closesolutionfile{ans}
	\Closesolutionfile{ansbook}
	\Closesolutionfile{ansex}
\end{dang}
%%%Dạng 3
\begin{dang}{Bài tập ứng dụng thực tiễn của Ammonia và Muối amoni}
	\begin{phuongphap}
		Để giải quyết các bài tập ứng dụng thực tiễn, học sinh cần:
		\begin{cacbuoc}
			\item Nắm vững công thức hóa học và khối lượng mol của các loại phân đạm thường gặp.
			\item Áp dụng công thức tính phần trăm khối lượng nguyên tố trong hợp chất.
			\item Hiểu các tính chất hóa học đặc trưng của ion amoni ($NH_4^+$) và $NH_3$ liên quan đến tương tác với các chất khác trong môi trường (đất, nước, không khí).
			\item Liên hệ tính chất vật lí của $NH_3$ (dễ hóa lỏng, nhiệt hóa hơi lớn) với ứng dụng làm lạnh.
			\item Phân tích và giải thích các hiện tượng thực tế dựa trên kiến thức hóa học đã học.
		\end{cacbuoc}
	\end{phuongphap}
	
	\phan{Bài tập tự luận}
	
	%%%=============SOẠN BT===============%%%
	\Opensolutionfile{ansbth}[Ans/LGBT-C02B06_UngDung]
	\Opensolutionfile{ansbt}[Ans/AnsBT-C02B06_UngDung]
	
	%%%=============BT_1=============%%%
	\begin{bt}
		Trong nông nghiệp, việc lựa chọn loại phân đạm phù hợp rất quan trọng để cung cấp đủ nitrogen cho cây trồng. Nitrogen là nguyên tố thiết yếu cho sự phát triển của cây, đặc biệt là thân, lá.
		Hãy tính hàm lượng phần trăm nitrogen (theo khối lượng) trong các loại phân đạm sau:
		\begin{enumerate}
			\item Amoni nitrat ($NH_4NO_3$)
			\item Amoni sunfat ($(NH_4)_2SO_4$)
			\item Urê ($(NH_2)_2CO$)
		\end{enumerate}
		Sau đó, so sánh và cho biết loại phân nào "giàu đạm" nhất. (Cho biết nguyên tử khối: N=14, H=1, O=16, C=12, S=32).
		\loigiai{
			Để tính hàm lượng phần trăm nitrogen trong mỗi loại phân đạm, ta sử dụng công thức:
			\[
			\%N = \dfrac{\text{Tổng khối lượng N trong 1 mol hợp chất}}{\text{Khối lượng mol của hợp chất}} \times 100\%
			\]
			\begin{enumerate}
				\item \textbf{Amoni nitrat ($NH_4NO_3$):}
				\\
				Khối lượng mol của $NH_4NO_3$: $M_{NH_4NO_3} = 14 + 4 \times 1 + 14 + 3 \times 16 = 80$ g/mol.
				\\
				Tổng khối lượng N trong 1 mol $NH_4NO_3$: $2 \times 14 = 28$ g.
				\[
				\%N_{NH_4NO_3} = \dfrac{28}{80} \times 100\% = 35\%
				\]
				\item \textbf{Amoni sunfat ($(NH_4)_2SO_4$):}
				\\
				Khối lượng mol của $(NH_4)_2SO_4$: $M_{(NH_4)_2SO_4} = 2 \times (14 + 4 \times 1) + 32 + 4 \times 16 = 132$ g/mol.
				\\
				Tổng khối lượng N trong 1 mol $(NH_4)_2SO_4$: $2 \times 14 = 28$ g.
				\[
				\%N_{(NH_4)_2SO_4} = \dfrac{28}{132} \times 100\% \approx 21{,}21\%
				\]
				\item \textbf{Urê ($(NH_2)_2CO$):}
				\\
				Khối lượng mol của $(NH_2)_2CO$: $M_{(NH_2)_2CO} = (14 + 2 \times 1) \times 2 + 12 + 16 = 60$ g/mol.
				\\
				Tổng khối lượng N trong 1 mol $(NH_2)_2CO$: $2 \times 14 = 28$ g.
				\[
				\%N_{(NH_2)_2CO} = \dfrac{28}{60} \times 100\% \approx 46{,}67\%
				\]
			\end{enumerate}
			\textbf{So sánh:}
			\begin{itemize}
				\item Amoni nitrat ($NH_4NO_3$): $35\%$ N
				\item Amoni sunfat ($(NH_4)_2SO_4$): $21{,}21\%$ N
				\item Urê ($(NH_2)_2CO$): $46{,}67\%$ N
			\end{itemize}
			Urê ($(NH_2)_2CO$) có hàm lượng nitrogen cao nhất ($46{,}67\%$). Do đó, urê là loại phân \lq\lq giàu đạm \rq\rq nhất trong ba loại phân này.
		}
	\end{bt}
	%%%=============BT_2=============%%%
	\begin{bt}
		Để cung cấp $120$ kg nitrogen ($N$) cho một hecta đất, người nông dân cần bón bao nhiêu kg amoni sunfat ($(NH_4)_2SO_4$)? (Cho biết nguyên tử khối: N=14, H=1, S=32, O=16).
		\loigiai{
			\begin{enumerate}
				\item \textbf{Tính hàm lượng phần trăm nitrogen trong amoni sunfat ($(NH_4)_2SO_4$):}
				\\
				Khối lượng mol của $(NH_4)_2SO_4$: $M_{(NH_4)_2SO_4} = 2 \times (14 + 4 \times 1) + 32 + 4 \times 16 = 132$ g/mol.
				\\
				Tổng khối lượng N trong 1 mol $(NH_4)_2SO_4$: $2 \times 14 = 28$ g.
				\[
				\%N_{(NH_4)_2SO_4} = \dfrac{28}{132} \times 100\% \approx 21{,}21\%
				\]
				\item \textbf{Tính khối lượng amoni sunfat cần bón:}
				\\
				Để cung cấp $120$ kg nitrogen, với hàm lượng $N$ là $21{,}21\%$ trong $(NH_4)_2SO_4$, khối lượng phân bón cần thiết là:
				\begin{align*}
						m_{\text{phân bón}} &= \dfrac{\text{Khối lượng N cần cung cấp}}{\%N \text{ trong phân bón}} \times 100\% \\
						m_{\text{phân bón}} &= \dfrac{120 \text{ kg}}{21{,}21\%} \times 100\% \\
						m_{\text{phân bón}} &= \dfrac{120}{0{,}2121} \approx 565{,}77 \text{ kg}
				\end{align*}
				\textbf{Đáp số:} Người nông dân cần bón khoảng $565{,}77$ kg amoni sunfat ($(NH_4)_2SO_4$) để cung cấp $120$ kg nitrogen cho một hecta đất.
			\end{enumerate}
		}
	\end{bt}
	%%%=============BT_4=============%%%
	\begin{bt}
		Ammonia là một hợp chất có nhiều ứng dụng trong công nghiệp. Trong các nhà máy sản xuất nước đá hoặc hệ thống làm lạnh công nghiệp, người ta thường dùng ammonia.
		\begin{enumerate}
			\item Giải thích vai trò của ammonia trong các thiết bị làm lạnh.
			\item Khi nói về chất dùng để làm nước đá, người ta thường nhắc đến "amoniac lỏng". Đó có phải là dung dịch $NH_3$ trong nước không? Giải thích.
		\end{enumerate}
		\loigiai{
			\begin{enumerate}
				\item \textbf{Giải thích vai trò của ammonia trong các thiết bị làm lạnh:}
				\\
				Ammonia ($NH_3$) là một chất làm lạnh hiệu quả nhờ vào các tính chất vật lí đặc trưng của nó:
				\begin{itemize}
					\item \textbf{Dễ hóa lỏng:} $NH_3$ dễ dàng hóa lỏng ở nhiệt độ tương đối thấp (điểm sôi $-33,4^\circ C$) và áp suất vừa phải.
					\item \textbf{Nhiệt hóa hơi lớn:} Khi $NH_3$ lỏng bay hơi (chuyển từ trạng thái lỏng sang khí), nó thu một lượng nhiệt rất lớn từ môi trường xung quanh (gọi là nhiệt hóa hơi). Điều này làm cho nhiệt độ của môi trường giảm mạnh, tạo ra hiệu ứng làm lạnh.
				\end{itemize}
				Trong thiết bị làm lạnh, $NH_3$ được nén và làm lạnh để hóa lỏng, sau đó cho bay hơi trong dàn bay hơi để hấp thụ nhiệt từ không gian cần làm lạnh, rồi lại được nén và hóa lỏng để tuần hoàn.
				
				\item \textbf{Giải thích về \lq\lq amoniac lỏng \rq\rq trong sản xuất nước đá:}
				\\
				\lq\lq Amoniac lỏng\rq\rq dùng trong các nhà máy sản xuất nước đá \textbf{không phải là dung dịch $NH_3$ trong nước} (dung dịch amoniac). Thay vào đó, nó là \textbf{ammonia tinh khiết ở trạng thái lỏng} ($NH_3$ lỏng).
				\begin{itemize}
					\item \textbf{Ammonia lỏng ($NH_3$ lỏng):} Là khí ammonia đã được nén và làm lạnh để chuyển sang trạng thái lỏng. Đây là chất được sử dụng trực tiếp trong chu trình làm lạnh để thu nhiệt và bay hơi.
					\item \textbf{Dung dịch amoniac ($NH_3$ trong nước):} Là khí $NH_3$ hòa tan trong nước. Dung dịch này có tính bazơ yếu và được dùng làm chất tẩy rửa hoặc trong phòng thí nghiệm, nhưng không có khả năng làm lạnh hiệu quả như $NH_3$ lỏng tinh khiết do nhiệt hóa hơi của dung dịch thấp hơn và điểm sôi/đông đặc khác biệt.
				\end{itemize}
			\end{enumerate}
		}
	\end{bt}
    %%%=============BT_5=============%%%
	\begin{bt}
		Một trang trại nuôi cá cảnh lớn sử dụng hệ đệm amoniac/amoni clorua ($NH_3/NH_4Cl$) để duy trì độ pH ổn định cho môi trường nước trong các bể nuôi, nhằm đảm bảo sức khỏe tối ưu cho loài cá cần môi trường kiềm nhẹ. Ban đầu, bể chứa $500$ lít dung dịch đệm có nồng độ $NH_3$ là $0{,}15$ M và $NH_4Cl$ là $0{,}20$ M. Trong quá trình nuôi, do tích tụ một số chất thải từ cá, $0{,}5$ mol ion $H^+$ được tạo ra và hòa tan vào $500$ lít dung dịch đệm này.
		\begin{enumerate}
			\item Tính độ pH của dung dịch đệm ban đầu trong bể nuôi.
			\item Tính độ pH của dung dịch đệm sau khi $0{,}5$ mol ion $H^+$ được thêm vào.
			\item So sánh sự thay đổi pH và nhận xét về khả năng duy trì pH của hệ đệm, cũng như ảnh hưởng tiềm ẩn đến sức khỏe của cá (biết rằng loài cá này phát triển tốt nhất trong khoảng pH $8{,}8$ đến $9{,}2$).
		\end{enumerate}
		Biết rằng:
		\begin{itemize}
			\item Hằng số bazơ $K_b$ của $NH_3$ ở $25^\circ C$ là $1{,}8 \times 10^{-5}$.
			\item Thể tích dung dịch được coi là không thay đổi đáng kể sau khi thêm ion $H^+$.
		\end{itemize}
		\loigiai{
			\begin{enumerate}
				\item \textbf{Tính độ pH của dung dịch đệm ban đầu:}
				\begin{enumerate}[a)]
					\item Tính nồng độ ban đầu của bazơ yếu và acid liên hợp:\\
					Số mol $NH_3$ ban đầu: $n_{NH_3} = 0{,}15 \text{ M} \times 500 \text{ L} = 75$ mol.\\
					Số mol $NH_4Cl$ ban đầu: $n_{NH_4Cl} = 0{,}20 \text{ M} \times 500 \text{ L} = 100$ mol.
					(Do $NH_4Cl$ phân li hoàn toàn, số mol ion amoni $NH_4^+$ là $100$ mol).
					\\
					Nồng độ $NH_3$ ban đầu: $[NH_3] = 0{,}15$ M.
					Nồng độ $NH_4^+$ ban đầu: $[NH_4^+] = 0{,}20$ M.
					\item Tính $\text{p}K_b$ của $NH_3$ và $\text{p}K_a$ của $NH_4^+$:
					$K_b = 1{,}8 \times 10^{-5} \Rightarrow \text{p}K_b = -\log(1{,}8 \times 10^{-5}) \approx 4{,}74$.
					$\text{p}K_a = 14 - \text{p}K_b = 14 - 4{,}74 = 9{,}26$.
					\item Áp dụng phương trình Henderson-Hasselbalch để tính pH ban đầu:
					\begin{align*}
						\text{pH}_{\text{ban đầu}} &= \text{p}K_a + \log\left(\dfrac{\text{[bazơ yếu]}}{\text{[acid liên hợp]}}\right) \\
						&= \text{p}K_a + \log\left(\dfrac{[NH_3]}{[NH_4^+]}\right) \\
						& = 9{,}26 + \log\left(\dfrac{0{,}15}{0{,}20}\right) \\
						& \approx 9{,}26 - 0{,}12 = 9{,}14
					\end{align*}
					Độ pH của dung dịch đệm ban đầu là khoảng $9{,}14$.
				\end{enumerate}
				\item \textbf{Tính độ pH của dung dịch đệm sau khi $0{,}5$ mol ion $H^+$ được thêm vào:}				
               \begin{enumerate}[a)]
				 \item Phản ứng của ion $H^+$ với hệ đệm:
				Khi thêm acid mạnh ($H^+$) vào hệ đệm $NH_3/NH_4^+$, $H^+$ sẽ phản ứng với bazơ yếu $NH_3$.
				\[
				\text{H}^+\text{ (aq)} + \text{NH}_3\text{ (aq)} \rightarrow \text{NH}_4^+\text{ (aq)}
				\]
				Số mol $H^+$ thêm vào là $0{,}5$ mol.
				\\
				\item Tính số mol các chất sau phản ứng:
				\[
				\begin{array}{lccc}
					& \text{H}^+ & + \quad \text{NH}_3 & \rightarrow \quad \text{NH}_4^+ \\
					\text{Ban đầu (mol):} & 0{,}5 & 75 & 100 \\
					\text{Phản ứng (mol):} & 0{,}5 & 0{,}5 & 0{,}5 \\
					\text{Sau PƯ (mol):} & 0 & 74{,}5 & 100{,}5
				\end{array}
				\]
				\item Tính nồng độ các chất sau phản ứng:\\
				Thể tích dung dịch vẫn là $500$ L.
				Nồng độ $NH_3$ sau phản ứng: $[NH_3]_{\text{sau}} = \dfrac{74{,}5 \text{ mol}}{500 \text{ L}} = 0{,}149$ M.\\
				Nồng độ $NH_4^+$ sau phản ứng: $[NH_4^+]_{\text{sau}} = \dfrac{100{,}5 \text{ mol}}{500 \text{ L}} = 0{,}201$ M.
				\item Áp dụng phương trình Henderson-Hasselbalch để tính pH sau phản ứng:
				\begin{align*}
				  \text{pH}_{\text{sau}} &= \text{p}K_a +     \log\left(\dfrac{[NH_3]_{\text{sau}}}{[NH_4^+]_{\text{sau}}}\right) \\
					&= 9{,}26 + \log\left(\dfrac{0{,}149}{0{,}201}\right) \\
					&= 9{,}26 + \log(0{,}7413) \\
					&\approx 9{,}26 - 0{,}13 = 9{,}13
				\end{align*}
				Độ pH của dung dịch đệm sau khi thêm $0{,}5$ mol ion $H^+$ là khoảng $9{,}13$.
                 \end{enumerate}
				\item \textbf{So sánh sự thay đổi pH và nhận xét:}
				\\
				\textbf{So sánh sự thay đổi pH:}
				\begin{itemize}
					\item pH ban đầu: $9{,}14$.
					\item pH sau khi thêm acid: $9{,}13$.
					\item Độ thay đổi pH: $9{,}14 - 9{,}13 = 0{,}01$ đơn vị pH.
				\end{itemize}
				\textbf{Nhận xét:}
				\begin{itemize}
					\item Sự thay đổi pH chỉ là $0{,}01$ đơn vị, rất nhỏ, điều này chứng tỏ hệ đệm amoniac/amoni clorua đã hoạt động rất hiệu quả trong việc duy trì pH của môi trường nước, dù có một lượng acid đáng kể được thêm vào.
					\item Khoảng pH tối ưu cho loài cá này là $8{,}8$ đến $9{,}2$. pH sau khi thêm acid là $9{,}13$ vẫn nằm trong khoảng pH tối ưu này. Điều này cho thấy hệ đệm đã thành công trong việc bảo vệ môi trường sống của cá khỏi sự thay đổi pH đột ngột, góp phần duy trì sức khỏe cho cá.
					\item Nếu không có hệ đệm, việc thêm $0{,}5$ mol $H^+$ vào $500$ lít nước sẽ làm nồng độ $H^+$ lên $0{,}001$ M, khiến pH giảm xuống $3$, gây hại nghiêm trọng cho cá.
				\end{itemize}
			\end{enumerate}
		}
	\end{bt}
	\Closesolutionfile{ansbt}
	\Closesolutionfile{ansbth}
\end{dang}






































