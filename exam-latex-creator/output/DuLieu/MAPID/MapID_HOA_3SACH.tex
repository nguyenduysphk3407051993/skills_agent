%
%Map ID v2.0.0.2 by MyLT.
%ID 3 bộ sách mới lớp 11 do thầy Dương Phước Sang gửi.
%
%Nếu là ID5 sẽ theo định đạng: %[Tham số 1 Tham số 2 Tham số 3 Tham số 4 Tham số 5]. Ví dụ: %[1D2B3]
%Nếu là ID6 sẽ theo định đạng: %[Tham số 1 Tham số 2 Tham số 3 Tham số 4 Tham số 5 - Tham số 6]. Ví dụ: %[1D2B3-1]
%Có thể thay đổi nội dung của mức độ nhưng vị trí của mức độ trong ID không đổi (thông số thứ 3 từ trái qua).
%Cú pháp của thông số: [Giá trị] Mô tả thông số. Ví dụ: [0] Lớp 10 thì 0 là giá trị để lưu vào ID.
%
%Chú ý: Các dấu gạch ngang không được thay đổi.
%
%Cấu hình tên các tham số
%
Tên tham số 1: Lớp
Tên tham số 2: Môn
Tên tham số 3: Chương
Tên tham số 4: Mức độ
Tên tham số 5: Bài
Tên tham số 6: Dạng
%
%Cấu hình chi tiết ID
%
%%Cấu hình mức độ dùng chung.
[Y] Yếu
[B] Trung bình
[K] Khá
[G] Giỏi
[T] Thực tế
%
%Cấu hình nội dung
%
-[0] Lớp 10
----[C] Cánh Diều
-------[1] Cấu tạo nguyên tử
----------[1] Các thành phần nguyên tử
----------[2] Nguyên tố hóa học
----------[3] Mô hình nguyên tử và orbital nguyên tử
----------[4] Lớp, phân lớp và cấu hình electron
-------[2] Bảng tuần hoàn các nguyên tố
----------[1] Cấu tạo bảng tuần hoàn các nguyên tố
----------[2] Xu hướng biến đổi tính chất của đơn chất biến đổi thành phần và tính chất của hợp chất trong một chu kì và trong một nhóm
----------[3] Định luật tuần hoàn và ý nghĩa của bảng hệ thống tuần hoàn các nguyên tố hóa học
-------[3] Liên kết hóa học
----------[1] Quy tắc octet
----------[2] Liên kết ion
----------[3] Liên kết cộng hóa trị
----------[4] Liên kết hidrogen và tương tác van der waals 
-------[4] Phản ứng oxi hóa – khử
----------[1] Phản ứng oxi hóa – khử
-------[5] Năng lượng hóa học
----------[1] Phản ứng hóa học và enthalpy
----------[2] Ý nghĩa và cách tính biến thiên enthalpy của phản ứng hóa học
-------[6] Tốc độ phản ứng hóa học
----------[1] Tốc độ phản ứng hóa học
-------[7] Nguyên tố nhóm VII (nhóm halogen)
----------[1] Nguyên tố và đơn chất halogen
----------[2] Hydrogen halide và hydrohalic acid
----[T] Chân Trời Sáng Tạo
-------[1] Cấu tạo nguyên tử
----------[1] Thành phần của nguyên tử
----------[2] Nguyên tố hóa học
----------[3] Cấu trúc lớp vỏ electron của nguyên tử
-------[2] Bảng tuần hoàn các nguyên tố
----------[1] Cấu tạo bảng tuần hoàn các nguyên tố hóa học
----------[2] Xu hướng biến đổi một số tính chất của nguyên tử các nguyên tố, thành phần và một số tính chất của hợp chất trong một chu kì và nhóm
----------[3] Định luật tuần hoàn - Ý nghĩa của bảng hệ thống tuần hoàn các nguyên tố hóa học
-------[3] Liên kết hóa học
----------[1] Quy tắc octet
----------[2] Liên kết ion
----------[3] Liên kết cộng hóa trị
----------[4] Liên kết hidrogen và tương tác van der Waals
-------[4] Phản ứng oxi hóa – khử
----------[1] Phản ứng oxi hóa – khử và ứng dụng trong cuộc sống
-------[5] Năng lượng hóa học
----------[1] Enthalpy tạo thành và biến thiên enthalpy của phản ứng hóa học
----------[2] Tính biến thiên enthalpy của phản ứng hóa học
-------[6] Tốc độ phản ứng hóa học
----------[1] Phương trình tốc độ và hằng số tốc độ phản ứng
----------[2] Các yếu tố ảnh hưởng đến tốc độ phản ứng hóa học
-------[7] Nguyên tố nhóm VII - Halogen
----------[1] Tính chất vật lý và hóa học các đơn chất nhóm VII A.
----------[2] Hydrogen halide và một số phản ứng của ion halide
----[K] Kết Nối Tri Thức Với Cuộc Sống
-------[1] Cấu tạo nguyên tử
----------[1] Thành phần của nguyên tử
----------[2] Nguyên tố hóa học
----------[3] Cấu trúc lớp vỏ electron nguyên tử
----------[4] Ôn tập chương 1
-------[2] Bảng tuần hoàn các nguyên tố hóa học và định luật tuần hoàn
----------[1] Cấu tạo bảng tuần hoàn các nguyên tố hóa học
----------[2] Xu hướng biến đổi một số tính chất của nguyên tử các nguyên tố trong một chu kì và trong một nhóm
----------[3] Xu hướng biến đổi thành phần và một số tính chất của hợp chất trong một chu kì và trong một nhóm
----------[4] Định luật tuần hoàn - Ý nghĩa của bảng hệ thống tuần hoàn các nguyên tố hóa học
----------[5] Ôn tập chương 2
-------[3] Liên kết hóa học
----------[1] Quy tắc octet
----------[2] Liên kết ion
----------[3] Liên kết cộng hóa trị
----------[4] Liên kết hidrogen và tương tác van der Waals
----------[5] Ôn tập chương 3
-------[4] Phản ứng oxi hóa – khử
----------[1] Phản ứng oxi hóa – khử 
----------[2] Ôn tập chương 4
-------[5] Năng lượng hóa học
----------[1] Biến thiên enthalpy trong các phản ứng hóa học
----------[2] Ôn tập chương 5
-------[6] Tốc độ phản ứng hóa học
----------[1] Tốc độ phản ứng
----------[2] Ôn tập chương 6
-------[7] Nguyên tố nhóm Halogen
----------[1] Nhóm halogen.
----------[2] Hydrogen halide. Muối halide
----------[2] Ôn tập chương 7
%%%%%%%%%%%%%%%%%%%%%%%%%%%%%%%%%%%%%%%%%%
----[H] Dùng Chung Ba Sách
-------[1] Cấu tạo nguyên tử
----------[1] Phần Lý thuyết trọng tâm
-------------[1] Lý thuyết trọng tâm về thành phần nguyên tử
-------------[2] Lý thuyết trọng tâm về nguyên tố hóa học
-------------[3] Lý thuyết trọng tâm về cấu trúc lớp vỏ
-------------[4] Lý thuyết tổng ôn chương 1
----------[2] Phần Bài tập vận dụng
-------------[1] Bài tập xác định thành phần nguyên tử
-------------[2] Bài tập xác định khối lượng riêng, bán kính nguyên tử
-------------[3] Bài tập về đồng vị
-------[2] Bảng tuần hoàn các nguyên tố
----------[1] Phần Lý thuyết trọng tâm
-------------[1] Lý thuyết trọng tâm về cấu tạo bảng tuần hoàn các nguyên tố
-------------[2] Lý thuyết trọng tâm về xu hướng biến đổi một số tính chất trong một chu kì và trong một nhóm
-------------[3] Lý thuyết trọng tâm về định luật tuần hoàn và ý nghĩa của bảng HTTT các NTHH
-------------[4] Lý thuyết tổng ôn chương 2
----------[2] Phần Bài tập vận dụng
-------------[1] Xác định vị trí, tính chất của các nguyên tố trong bảng tuần hoàn.
-------------[2] Xác định nguyên tố hóa học dựa vào công thức oxide, công thức hợp chất khi vớl hydrogen và ngược lại.
-------------[3] Bài tập xác định nguyên tố dựa vào cấu hình e và ngược lại
-------------[4] Bài tập viết cấu hình electron của nguyên tử
-------------[5] Bài tập viết cấu hình electron của ion
-------------[6] Bài tập tổng hợp chương 2
-------[3] Liên kết hóa học
----------[1] Phần Lý thuyết trọng tâm
-------------[1] Lý thuyết trọng tâm về Quy tắc octet
-------------[2] Lý thuyết trọng tâm về Liên kết ion
-------------[3] Lý thuyết trọng tâm về Liên kết cộng hóa trị
-------------[4] Lý thuyết trọng tâm về Liên kết hidrogen và tương tác van der waals 
-------------[5] Tổng ôn lý thuyết chương 3
----------[2] Bài tập về liên kết hóa học
-------------[1] Bài tập về liên kết ion
-------------[2] Bài tập về liên kết cộng hóa trị
-------------[3] Bài tập về liên kết hydrogen_tuong tác vanderwalls
-------------[4] Bài tập tổng hợp về liên kết hóa học
-------[4] Phản ứng oxi hóa – khử
----------[1] Phần Lý thuyết trọng tâm
-------------[1] Lý thuyết trọng tâm về phản ứng oxi hóa – khử
-------------[2] Lý thuyết tổng ôn chương 4
----------[2] Phần Bài tập vận dụng
-------------[1] Bài tập xác định số oxi hóa
-------------[2] Bài tập lập phương trình phản ứng oxi hóa khử
-------------[3] Bài tập về Phản ứng oxi hóa và ứng dụng
-------------[4] Bài tập về định luật bảo toàn số mol electron
-------[5] Năng lượng hóa học
----------[1] Phần Lý thuyết trọng tâm
-------------[1] Lý thuyết trọng tâm về phản ứng hóa học và enthalpy
-------------[2] Lý thuyết trọng tâm về ý nghĩa và cách tính biến thiên enthalpy của phản ứng hóa học
----------[2] Phần Bài tập vận dụng
-------------[1] Xác định biến thiên enthalpy của phản ứng dựa vào enthalpy tạo thành.
-------------[2] Xác định biến thiên enthalpy của phản úng dựa vào năng lự̛̛ng liên kết
-------------[3] Bài tập tổng hợp xác đ̇ịnh biến thiên enthalpy của phản ứng
-------[6] Tốc độ phản ứng hóa học
----------[1] Phần Lý thuyết trọng tâm
-------------[1] Lý thuyết về tốc độ phản ứng
----------[2] Phần Bài tập vận dụng
-------------[1] Tính tốc độ trung bình của phản ứng
-------------[2] Tính tốc độ tức thời của phản ứng
-------------[3] Tính hệ số nhiệt độ van - hoff
-------[7] Nguyên tố nhóm VII (nhóm halogen)
----------[1] Phần Lý thuyết trọng tâm
-------------[1] Lí thuyết trọng tâm về tính chất vật lí và hóa học các đơn chất nhóm VIIA.
-------------[2] Lí thuyết trọng tâm về hydrogen halide, hydrohalic acid và muốl halide. 
-------------[3] Lý thuyết tổng ôn chương 7
----------[2] Phần Bài tập vận dụng
-------------[1] Bài tập liên hệ thực tế.
-------------[2] Bài tập kim loại tác dụng với hidrochloride acid.
-------------[3] Bài tập basic oxide tác dụng với hydrochloride acid.
-------------[4] Bài tập về muối carbonate tác duung vớl hydrochloride acid.
-------------[5] Bài tập về muối halide
-[1] Lớp 11
----[C] CÁNH DIỀU
----[K] KẾT NỐI TRI THỨC VỚI CUỘC SỐNG
----[T] CHÂN TRỜI SÁNG TẠO
-------[1] Cân bằng hóa học
----------[1] Khái niệm về cân bằng hóa học
----------[2] Cân bằng trong dung dịch nước
-------[2] NITROGEN VÀ SULFUR
----------[1] Đơn chất nitrogen
----------[2] Ammonia và một số hợp chất ammonia
----------[3] Một số hợp chất với oxygen của nitrogen
----------[4] Sulfur và sulfur dioxide
----------[5] Sulfur acid và muối sulfate
-------[3] Đại cương hóa hữu cơ
----------[1] Hợp chất hữu cơ và hóa học hữu cơ
----------[2] Phương pháp tách và tinh chế hợp chất hữu cơ
----------[3] Công thức phân tử hợp chất hữu cơ
----------[4] Cấu tạo hóa học hợp chất hữu cơ
-------[4] Hydrocarbon
----------[1] Alkane
----------[2] Hydrocarbon không no
----------[3] Arene (hydrocarbon thơm)
-------[5] Dẫn xuất halogen – Alcohol - Phenol
----------[1] Dẫn xuất halogen
----------[2] Alcohol 
----------[3] Phenol
-------[6] Hợp chất carbonyl (Aldehyde – Ketone – Carboxylic acid
----------[1] Hợp chất carbonyl
----------[2] Carboxylic acid
----[H] DÙNG CHUNG BA SÁCH
-------[1] CÂN BẰNG HÓA HỌC
----------[1] Lý thuyết trọng tâm
-------------[1] Lý thuyết về cân bằng hóa học
-------------[2] Lý thuyết về cân bằng hóa học trong dung dịch nước
----------[2] Bài tập vận dụng
-------------[1] Bài tập về PHƯƠNG TRÌNH HÓA HỌC CỦA PHẢN ỨNG THUẬN NGHỊCH
-------------[2] Bài tập về CÁC YÊU TỐ ẢNH HƯỞNG ĐẾN CÂN BÅ̀NG HÓA HỌC
-------------[3] Bài tập về BÀI TẬP LIÊN QUAN ĐẾN HẦNG SỐ CÂN BẢ̀NG KC
-------------[4] Bài tập về SỰ ĐIỆN LI - PHƯƠNG TRÌNH ĐIỆN LI
-------------[5] Bài tập về ACID VÀ BASE THEO THUYÊT BRONSTED - LOWRY
-------------[6] Bài tập về SỰ THỦY PHÂN CỦA CÁC ION
-------------[7] BÀI TOÁN VỀ pH VÀ Ý NGHĨA pH TRONG THỰC TIỄN
-------------[8] Bài tập về CHUẤN ĐỘ DUNG DỊCH

%%%======================================%%%%
-------[2] NITROGEN VÀ SULFUR
----------[1] NITROGEN VÀ HỢP CHẤT
-------------[1] Lý thuyết trọng tâm về nitrogen và hợp chất
-------------[2] Bài tập về NITROGEN : $N_2$
-------------[3] Bài tập về AMMONIA: $\mathbf{N H}_3$ - AMMONIUM: $\mathbf{N H}_4^{+}$
-------------[4] Bài tập về CÁC OXIDE CỦA NITROGEN VÀ MÜA ACID
-------------[5] Bài tập về NITRIC ACID \& HIỆN TƯỢNG PHÚ DƯỠNG
----------[2] Sulfur VÀ HỢP CHẤT
-------------[1] Lý thuyết trọng tâm về Sulfur và hợp chất
-------------[2] Bài tập SULFUR (S) VÀ SULFUR DIOXIDE $\left(\mathrm{SO}_2\right)$
-------------[3] Bài tập SULFURIC ACID VÀ MUỐI SULFATE
-------------[4] Bài tập VIẾT PHƯƠNG TRÌNH HÓA HỌC
-------------[5] Bài tập NHẬN BIẾT \& ĐIỀU CHÊ
-------------[6] BÀI TOÁN HỖN HỢP KIM LOẠI $+\mathrm{H}_2 \mathrm{SO}_4$
-------[3] Đại cương hóa hữu cơ
----------[1] Lý thuyết trọng tâm.
-------------[1] Lý thuyết trọng tâm về Khái niệm, phân loại, đặc điểm chung hợp chất hữu cơ
-------------[2] Lý thuyết trọng tâm về nhóm chức và phương pháp phổ IR xác định nhóm chức trong hợp chất hữu cơ
-------------[3] Lý thuyết trọng tâm về PHƯƠNG PHÁP TÁCH BIỆT VÀ TINH CHẾ HỢP CHÂT HỮU CƠ
-------------[4] Lý thuyết trọng tâm về các loại công thức
----------[2] Bài tập vận dụng
-------------[1] Bài tập Các loại công thức hợp chất hữu cơ và mối quan hệ giữa chúng.
-------------[2] Bài tập Xác định khối lượng phân tử bằng phương pháp phổ khối lượng MS và lập công thức phân tử hợp chất hữu cơ
-------------[3] Bài tập Viết công thức cấu tạo
-------------[4] Bài tập Đồng đẳng, đồng phân.
-------[4] Hydrocarbon
----------[1] Lý thuyết trọng tâm 
-------------[1] Lý thuyết trọng tâm về Alkane
-------------[2] Lý thuyết trọng tâm về Hydrocarbon không no
-------------[3] Lý thuyết trọng tâm về Arene
----------[2] Bài tập vận dụng
-------------[1] VIẾT PHƯƠNG TRÌNH HÓA HỌC CỦA CÁC PHẢN ỨNG
-------------[2] BÀI TOÁN LIÊN QUAN THỰC TÊ VÀ PHÁT TRIÊN NĂNG LỰC
-------------[3] BÀI TOÁN ĐIỀU CHẾ, ỨNG DỤNG, ẢNH HƯỞNG ĐẾN MÔI TRƯỜNG 
-------------[4] NHẬN BIẾT ALKANE, ALKENE VÀ ALKYNE
-------------[5] BÀI TOÁN PHẢN ỨNG CỘNG $\mathrm{X}_2, \mathrm{HX}, \mathrm{H}_2 \mathrm{O}, \mathrm{H}_2$ ALKENE VÀ ALKYNE
-------------[6] PHẢN ỨNG OXI HÓA ALKENE, ALKYNE
-------------[7] PHẢN ÚNG VỚI $\mathrm{AgNO}_3 / \mathrm{NH}_3$ CỦA ALK-1-YNE
-------[5] Dẫn xuất halogen – Alcohol - Phenol
----------[1] Lý thuyết trọng tâm
-------------[1] Lý thuyết trọng tâm về Dẫn xuất halogen
-------------[2] Lý thuyết trọng tâm về Alcohol 
-------------[3] Lý thuyết trọng tâm về Phenol
----------[2] Bài tập vận dụng
-------------[1] VIẾT PHƯƠNG TRÌNH HÓA HỌC CỦA CÁC PHẢN ỨNG
-------------[2] BÀI TOÁN LIÊN QUAN THỰC TÊ VÀ PHÁT TRIÊN NĂNG LỰC
-------------[3] BÀI TOÁN ĐIỀU CHẾ, ỨNG DỤNG, ẢNH HƯỞNG ĐẾN MÔI TRƯỜNG 
-------------[4] NHẬN BIẾT , HIỆN TƯỢNG PHẢN ỨNG
-------------[5] PHẢN ỨNG CHÁY
-------------[6] PHẢN ỨNG TÁCH HX
-------------[7] PHẢN ỨNG OXI HÓA Rượu bới CuO
-------------[8] Ancol tác dụng với Na
-------------[9] Phenol tác dụng với Brom
-------------[10] Bài toán về phản ứng lên men
-------[6] Hợp chất carbonyl (Aldehyde – Ketone – Carboxylic acid
----------[1] Lý thuyết trọng tâm
-------------[1] Lý thuyết trọng tâm về Hợp chất carbonyl
-------------[2] Lý thuyết trọng tâm về Carboxylic acid 
----------[2] Bài tập vận dụng
-------------[1] VIẾT PHƯƠNG TRÌNH HÓA HỌC CỦA CÁC PHẢN ỨNG
-------------[2] BÀI TOÁN LIÊN QUAN THỰC TÊ VÀ PHÁT TRIÊN NĂNG LỰC
-------------[3] BÀI TOÁN ĐIỀU CHẾ, ỨNG DỤNG, ẢNH HƯỞNG ĐẾN MÔI TRƯỜNG 
-------------[4] NHẬN BIẾT , HIỆN TƯỢNG PHẢN ỨNG
-------------[5] PHẢN ỨNG CHÁY
-------------[6] Phản ứng tráng bạc của Aldehyde
-------------[7] Phản ứng este hóa của carboxylic acid



-[2] Lớp 12
----[C] CÁNH DIỀU
----[K] KẾT NỐI TRI THỨC VỚI CUỘC SỐNG
----[T] CHÂN TRỜI SÁNG TẠO
-------[1] ESTER – LIPID. XÀ PHÒNG VÀ CHẤT GIẶT RỬA
----------[1] Ester - Lipid.
----------[2] Xà phòng và chất giặt rửa
-------[2] CARBOHYDRATE
----------[1] Glucose và fructose.
----------[2] Saccharose và maltose.
----------[3] Tinh bột và cellulose.
-------[3] HỢP CHẤT CHỨA NITROGEN
----------[1] Amine.
----------[2] Amino acid và peptide.
----------[3] Protein và enzyme.
-------[4] POLYMER
----------[1] Đại cương về polymer.
----------[2] Chất dẻo và vật liệu composite.
----------[3] Tơ - Cao su - Keo dán tổng hợp
-------[5] PIN ĐIỆN VÀ ĐIỆN PHÂN
----------[1] Thế điện cực và nguồn điện hoá học. 
----------[2] Điện phân
-------[6] ĐẠI CƯƠNG VỀ KIM LOẠI
----------[1] Đặc điểm cấu tạo và liên kết kim loại. Tính chất kim loại
----------[2] Các phương pháp tách kim loại.
----------[3] Hợp kim - Sự ăn mòn kim loại.
-------[7] NGUYÊN TỐ NHÓM IA VÀ IIA.
----------[1] Nguyên tố nhóm IA.
----------[2] Nguyên tố nhóm IIA.
-------[8] SƠ LƯỢC VỀ DÃY KIM LOẠI CHUYỂN TIẾP THỨ NHẤT VÀ PHỨC CHẤT.
----------[1] Đại cương vê̂ kim loại chuyển tiếp dãy thứ nhất. 
----------[2] Sơ lược về phức chất và sự hình thành phức chất của ion kim loại chuyển tiếp trong dung dịch.
----[H] Dùng Chung Ba Sách
-------[1] ESTER – LIPID.
----------[1] Câu hỏi lý thuyết về ester-lipid
-------------[1] Dạng 1: Khái niệm - cấu tạo - danh pháp.
-------------[2] Dạng 2: Đồng phân este.
-------------[3] Dạng 3: Tính chất hóa học
-------------[4] Dạng 4: Tính chất vật lý - Ứng dụng - Điều chế
-------------[5] Dạng 5: Lí thuyết tổng hợp
----------[2] Bài tập về este-lipid
-------------[1] Dạng 1: Bài toán đốt cháy este
-------------[2] Dạng 2: Bài toán thủy phân este đơn chức
-------------[3] Dạng 3: Bài toán thủy phân este đa chức
-------------[4] Dạng 4: Bài toán về phản ứng este hóa
-------------[5] Dạng 5: Bài toán về chất béo
-------------[6] Dạng 6: Bài toán tổng hơp
-------[2] CARBOHYDRATE
----------[1] Câu hỏi lý thuyết 
-------------[1] Dạng 1: Câu hỏi lý thuyết về glucose và fructose.
-------------[2] Dạng 2: Câu hỏi lý thuyết về Saccharose và maltose.
-------------[3] Dạng 3: Câu hỏi lý thuyết về Tinh bột và cellulose.
-------------[4] Dạng 4: Câu hỏi lí thuyết tổng hợp.
----------[2] Bài tập 
-------------[1] Dạng 1: Bài tập về phản ứng tráng gương của glucose.
-------------[2] Dạng 2: Bài tập về phản ứng lên men của glucose.
-------------[3] Dạng 3: Bài tập về phản ứng thủy phân của Saccharose, tinh bột và cellulose
-------------[4] Dạng 4: Bài toán điều chế xenlulozơ trinitrat
-------[3] HỢP CHẤT CHỨA NITROGEN
----------[1] Amine.
----------[2] Amino acid và peptide.
----------[3] Protein và enzyme.
-------[4] POLYMER
----------[1] Đại cương về polymer.
----------[2] Chất dẻo và vật liệu composite.
----------[3] Tơ - Cao su - Keo dán tổng hợp
-------[5] PIN ĐIỆN VÀ ĐIỆN PHÂN
----------[1] Thế điện cực và nguồn điện hoá học. 
----------[2] Điện phân
-------[6] ĐẠI CƯƠNG VỀ KIM LOẠI
----------[1] Đặc điểm cấu tạo và liên kết kim loại. Tính chất kim loại
----------[2] Các phương pháp tách kim loại.
----------[3] Hợp kim - Sự ăn mòn kim loại.
-------[7] NGUYÊN TỐ NHÓM IA VÀ IIA.
----------[1] Nguyên tố nhóm IA.
----------[2] Nguyên tố nhóm IIA.
-------[8] SƠ LƯỢC VỀ DÃY KIM LOẠI CHUYỂN TIẾP THỨ NHẤT VÀ PHỨC CHẤT.
----------[1] Đại cương vê̂ kim loại chuyển tiếp dãy thứ nhất. 
----------[2] Sơ lược về phức chất và sự hình thành phức chất của ion kim loại chuyển tiếp trong dung dịch.