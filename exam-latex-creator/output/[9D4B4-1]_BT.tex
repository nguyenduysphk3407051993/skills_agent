%%%=============BT_1=============%%%
\begin{bt}%[9D4B4]
	Giải phương trình bậc hai sau bằng cách dùng công thức nghiệm: $x^2 - 7x + 10 = 0$.
	\loigiai{
		Phương trình $x^2 - 7x + 10 = 0$ có $a=1$; $b=-7$; $c=10$.
		\\
		Ta có biệt thức $\Delta = b^2 - 4ac = (-7)^2 - 4 \cdot 1 \cdot 10 = 49 - 40 = 9 > 0$.
		\\
		Vì $\Delta > 0$ nên phương trình có hai nghiệm phân biệt:
		\[ x_1 = \dfrac{-b + \sqrt{\Delta}}{2a} = \dfrac{-(-7) + \sqrt{9}}{2 \cdot 1} = \dfrac{7+3}{2} = 5 \]
		\[ x_2 = \dfrac{-b - \sqrt{\Delta}}{2a} = \dfrac{-(-7) - \sqrt{9}}{2 \cdot 1} = \dfrac{7-3}{2} = 2 \]
		Vậy tập nghiệm của phương trình là $S = \{2; 5\}$.
	}
\end{bt}

%%%=============BT_2=============%%%
\begin{bt}%[9D4B4]
	Giải phương trình bậc hai: $3x^2 + 8x + 4 = 0$.
	\loigiai{
		Phương trình $3x^2 + 8x + 4 = 0$ có $a=3$; $b=8$; $c=4$.
		\\
		Ta có biệt thức $\Delta = b^2 - 4ac = 8^2 - 4 \cdot 3 \cdot 4 = 64 - 48 = 16 > 0$.
		\\
		Vì $\Delta > 0$ nên phương trình có hai nghiệm phân biệt:
		\[ x_1 = \dfrac{-b + \sqrt{\Delta}}{2a} = \dfrac{-8 + \sqrt{16}}{2 \cdot 3} = \dfrac{-8+4}{6} = -\dfrac{2}{3} \]
		\[ x_2 = \dfrac{-b - \sqrt{\Delta}}{2a} = \dfrac{-8 - \sqrt{16}}{2 \cdot 3} = \dfrac{-8-4}{6} = -2 \]
		Vậy tập nghiệm của phương trình là $S = \{-2; -\dfrac{2}{3}\}$.
	}
\end{bt}

%%%=============BT_3=============%%%
\begin{bt}%[9D4B4]
	Giải phương trình bậc hai: $4x^2 - 12x + 9 = 0$.
	\loigiai{
		Phương trình $4x^2 - 12x + 9 = 0$ có $a=4$; $b=-12$; $c=9$.
		\\
		Ta có biệt thức $\Delta = b^2 - 4ac = (-12)^2 - 4 \cdot 4 \cdot 9 = 144 - 144 = 0$.
		\\
		Vì $\Delta = 0$ nên phương trình có nghiệm kép:
		\[ x_1 = x_2 = \dfrac{-b}{2a} = \dfrac{-(-12)}{2 \cdot 4} = \dfrac{12}{8} = \dfrac{3}{2} \]
		Vậy tập nghiệm của phương trình là $S = \{\dfrac{3}{2}\}$.
	}
\end{bt}

%%%=============BT_4=============%%%
\begin{bt}%[9D4B4]
	Giải phương trình bậc hai: $2x^2 - 3x + 5 = 0$.
	\loigiai{
		Phương trình $2x^2 - 3x + 5 = 0$ có $a=2$; $b=-3$; $c=5$.
		\\
		Ta có biệt thức $\Delta = b^2 - 4ac = (-3)^2 - 4 \cdot 2 \cdot 5 = 9 - 40 = -31 < 0$.
		\\
		Vì $\Delta < 0$ nên phương trình vô nghiệm.
		\\
		Vậy tập nghiệm của phương trình là $S = \varnothing$.
	}
\end{bt}

%%%=============BT_5=============%%%
\begin{bt}%[9D4B4]
	Giải phương trình bậc hai: $3x^2 - 2\sqrt{3}x - 3 = 0$.
	\loigiai{
		Phương trình $3x^2 - 2\sqrt{3}x - 3 = 0$ có $a=3$; $b=-2\sqrt{3}$; $c=-3$.
		\\
		Ta có biệt thức $\Delta = b^2 - 4ac = (-2\sqrt{3})^2 - 4 \cdot 3 \cdot (-3) = 12 + 36 = 48 > 0$.
		\\
		Vì $\Delta > 0$ nên phương trình có hai nghiệm phân biệt:
		\[ x_1 = \dfrac{-b + \sqrt{\Delta}}{2a} = \dfrac{-(-2\sqrt{3}) + \sqrt{48}}{2 \cdot 3} = \dfrac{2\sqrt{3} + 4\sqrt{3}}{6} = \sqrt{3} \]
		\[ x_2 = \dfrac{-b - \sqrt{\Delta}}{2a} = \dfrac{-(-2\sqrt{3}) - \sqrt{48}}{2 \cdot 3} = \dfrac{2\sqrt{3} - 4\sqrt{3}}{6} = -\dfrac{\sqrt{3}}{3} \]
		Vậy tập nghiệm của phương trình là $S = \{-\dfrac{\sqrt{3}}{3}; \sqrt{3}\}$.
	}
\end{bt}
