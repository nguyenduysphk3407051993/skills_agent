\documentclass[FileMain.tex]{subfiles}
\gdef\sophong{Sở GD \& ĐT Gia Lai} 
\gdef\truong{Trường THPT Chi Lăng} 
\gdef\truongh{Trường Mầm non, THCS, THPT Sao Việt} 
\gdef\monhoc{Khoa học tự nhiên 6} 
\gdef\ngaykt{04/02/2026} 
\gdef\nh{2025 - 2026} 
\gdef\thoigian{45 phút}
\gdef\made{219} 
\setcounter{section}{0}
%\tatloigiai
%\hienthiloigiai
%\dongkeloigiai
\begin{document}
\section[Truy bài định kì - Mã đề \made]{Truy bài định kì} 
%\Tieudegiua{Kiểm tra chủ đề Lương thực - Thực phẩm - Mã đề \made}

%%%==============Phần trắc nghiệm nhiều lựa chọn==============%%% 
\subsection{Bài tập trắc nghiệm nhiều lựa chọn}\textit{\large Thí sinh trả lời từ câu 1 đến 12. Mỗi câu thí sinh chỉ chọn một phương án}
\Opensolutionfile{ansex}[Ans/LGEX-LTTP_KHTN6_219_MADE219]
\Opensolutionfile{ans}[Ans/Ans-LTTP_KHTN6_219_MADE219]

%%%============EX_1=============%%%
\begin{ex}%[6K3N4-1]
	Bệnh còi xương ở trẻ em thường do thiếu hụt vitamin và khoáng chất nào?
	\choice
	{Vitamin A}
	{{\True Vitamin D và Canxi}}
	{Vitamin C}
	{Sắt và Kẽm}
	\loigiai{
		Vitamin D giúp cơ thể hấp thụ Canxi. Thiếu hai chất này xương sẽ mềm, yếu, dẫn đến còi xương.
	}
\end{ex}

%%%============EX_2=============%%%
\begin{ex}%[6K3N4-2]
	Thực phẩm nào sau đây là nguồn cung cấp Protein có nguồn gốc thực vật (đạm thực vật)?
	\choice
	{Thịt lợn}
	{Trứng vịt}
	{{\True Các loại đậu (đậu nành, đậu xanh)}}
	{Sữa bò}
	\loigiai{
		Các loại đậu, đặc biệt là đậu nành, chứa hàm lượng protein rất cao, là nguồn đạm thực vật quý giá.
	}
\end{ex}

%%%============EX_3=============%%%
\begin{ex}%[6K3H4-3]
	Để hạn chế sự phát triển của vi khuẩn trong thực phẩm, người ta thường dùng phương pháp làm lạnh. Nhiệt độ của ngắn đá tủ lạnh gia đình thường là bao nhiêu?
	\choice
	{Khoảng $0^\circ C$}
	{Khoảng $4^\circ C$}
	{{\True Khoảng $-18^\circ C$}}
	{Khoảng $10^\circ C$}
	\loigiai{
		Ngăn đá tủ lạnh (ngăn đông) thường duy trì nhiệt độ khoảng $-18^\circ C$ hoặc thấp hơn để đóng băng thực phẩm hoàn toàn.
	}
\end{ex}

%%%============EX_4=============%%%
\begin{ex}%[6K3V4-4]
	Loại lương thực nào là nguồn cung cấp tinh bột chủ yếu cho người dân ở Việt Nam?
	\choice
	{Ngô}
	{{\True Lúa gạo}}
	{Khoai tây}
	{Lúa mạch}
	\loigiai{
		Lúa gạo là cây lương thực truyền thống và chủ yếu nhất của Việt Nam.
	}
\end{ex}

%%%============EX_5=============%%%
\begin{ex}%[6K3N4-1]
	Trong quả gấc chín có chứa nhiều chất gì tốt cho mắt?
	\choice
	{Tinh bột}
	{Chất béo}
	{{\True Beta-carotene (Tiền Vitamin A)}}
	{Canxi}
	\loigiai{
		Màu đỏ của gấc là do Beta-carotene, khi vào cơ thể sẽ chuyển hóa thành Vitamin A giúp sáng mắt.
	}
\end{ex}

%%%============EX_6=============%%%
\begin{ex}%[6K3H4-2]
	Khi chọn mua thực phẩm đóng gói, ngoài hạn sử dụng, thông tin nào sau đây giúp ta biết được thành phần dinh dưỡng của sản phẩm?
	\choice
	{Tên thương hiệu}
	{Mã vạch sản phẩm}
	{{\True Bảng giá trị dinh dưỡng (Nutrition Facts)}}
	{Khối lượng tịnh}
	\loigiai{
		Bảng giá trị dinh dưỡng cung cấp thông tin về lượng calo, chất béo, đường, protein\dots có trong sản phẩm.
	}
\end{ex}

%%%============EX_7=============%%%
\begin{ex}%[6K3V4-3]
	Vì sao người ta thường phơi khô các loại hạt như lúa, ngô, đậu sau khi thu hoạch?
	\choice
	{Để hạt đẹp hơn}
	{{\True Để giảm lượng nước, ngăn nấm mốc phát triển gây hư hỏng}}
	{Để tăng vị ngọt cho hạt}
	{Để hạt to hơn}
	\loigiai{
		Nước là môi trường cho vi sinh vật phát triển. Loại bỏ nước (làm khô) là cách bảo quản truyền thống hiệu quả nhất cho các loại hạt.
	}
\end{ex}

%%%============EX_8=============%%%
\begin{ex}%[6K3N4-4]
	Trong các loại nước uống sau, loại nào tốt nhất cho sức khỏe để uống hàng ngày?
	\choice
	{Nước ngọt có ga}
	{Trà sữa trân châu}
	{{\True Nước lọc (nước đun sôi để nguội)}}
	{Nước tăng lực}
	\loigiai{
		Nước lọc sạch là tốt nhất, giúp thanh lọc cơ thể mà không nạp thêm đường hay hóa chất độc hại.
	}
\end{ex}

%%%============EX_9=============%%%
\begin{ex}%[6K3H4-1]
	Dấu hiệu nào cảnh báo cá biển đã bị ươn và có thể gây ngộ độc Histamine?
	\choice
	{Thịt cá săn chắc, đàn hồi}
	{Mắt cá trong veo}
	{{\True Mắt cá đỏ hoặc đục, mang cá thâm, thịt mềm nhũn}}
	{Vảy cá bám chặt vào da}
	\loigiai{
		Cá ươn, đặc biệt là cá ngừ, cá thu, vi khuẩn biến đổi axit amin thành Histamine gây dị ứng, ngộ độc. Dấu hiệu là mắt đục, thịt nát.
	}
\end{ex}

%%%============EX_10=============%%%
\begin{ex}%[6K3V4-2]
	Cây Sắn (củ mì) ngoài việc dùng để ăn củ, còn được dùng để sản xuất ra sản phẩm công nghiệp nào sau đây?
	\choice
	{{\True Bột ngọt (mì chính), cồn sinh học (E5)}}
	{Nước mắm}
	{Dầu ăn}
	{Sữa tươi}
	\loigiai{
		Tinh bột sắn là nguyên liệu chính để lên men sản xuất bột ngọt và cồn sinh học (Ethanol).
	}
\end{ex}

%%%============EX_11=============%%%
\begin{ex}%[6K3N4-3]
	Thiếu Vitamin C thường dẫn đến bệnh lý nào?
	\choice
	{Mù lòa}
	{{\True Bệnh scobat (chảy máu chân răng/lợi)}}
	{Béo phì}
	{Đau dạ dày}
	\loigiai{
		Vitamin C giúp bền vững thành mạch máu. Thiếu nó gây xuất huyết dưới da, chảy máu lợi (bệnh Scobat).
	}
\end{ex}

%%%============EX_12=============%%%
\begin{ex}%[6K3H4-4]
	Tại sao mật ong có thể để được rất lâu mà không bị hỏng, dù không cần để tủ lạnh?
	\choice
	{Vì mật ong có chất bảo quản hóa học}
	{{\True Vì mật ong có hàm lượng đường rất cao và rất ít nước}}
	{Vì vi khuẩn sợ đồ ngọt}
	{Vì mật ong được nấu chín}
	\loigiai{
		Nồng độ đường trong mật ong quá cao làm vi khuẩn bị rút nước (hiện tượng co nguyên sinh) và chết, nên mật ong tự nhiên là chất bảo quản.
	}
\end{ex}

\Closesolutionfile{ans}
\Closesolutionfile{ansex}
%\bangdapan{Ans-LTTP_KHTN6_219_MADE219}

%%%==============Phần trắc nghiệm đúng sai==============%%% 
\subsection{Trắc nghiệm đúng sai}\textit{\large Thí sinh trả lời từ câu 1 đến câu 4. Trong mỗi ý a), b), c), d) ở mỗi câu thí sinh chọn đúng hoặc sai}
\Opensolutionfile{ansex}[Ans/LGTF-LTTP_KHTN6_219_MADE219]
\Opensolutionfile{ansbook}[Ansbook/AnsTF-LTTP_KHTN6_219_MADE219]
\Opensolutionfile{ans}[Ans/Tempt-LTTP_KHTN6_219_MADE219]
\setcounter{ex}{0}

%%%=============TF_1=============%%%
\begin{ex}%[6K3H4-1]
	Các phát biểu về chất dinh dưỡng:
	\choiceTF
	{\True Cơ thể cần bổ sung nước thường xuyên, ngay cả khi chưa thấy khát.}
	{Vitamin chỉ có trong rau quả, không có trong thịt cá.}
	{\True Chất béo là dung môi hòa tan các vitamin A, D, E, K giúp cơ thể hấp thụ chúng.}
	{Ăn càng nhiều Protein thì cơ thể càng khỏe mạnh, không có tác hại gì.}
	\loigiai{
		\begin{itemchoice}[T1,F2,T3,F4]
			\itemch Khi thấy khát là cơ thể đã mất nước.
			\itemch Thịt cá chứa nhiều Vitamin nhóm B (B1, B12\dots).
			\itemch Thiếu chất béo sẽ thiếu hụt các vitamin này.
			\itemch Ăn quá nhiều Protein gây gánh nặng cho gan, thận và có thể gây bệnh Gout.
		\end{itemchoice}
	}
\end{ex}

%%%=============TF_2=============%%%
\begin{ex}%[6K3V4-2]
	Về vệ sinh an toàn thực phẩm:
	\choiceTF
	{\True Dùng chung thớt thái thịt sống và cắt trái cây ăn liền có nguy cơ nhiễm khuẩn chéo.}
	{Rửa rau sống dưới vòi nước chảy sạch hơn là chỉ ngâm trong chậu nước.}
	{\True Thực phẩm sau khi rã đông nếu không nấu ngay thì có thể cấp đông lại để dùng lần sau.}
	{Cần rửa tay bằng xà phòng trước khi chế biến thức ăn và sau khi đi vệ sinh.}
	\loigiai{
		\begin{itemchoice}[T1,T2,F3,T4]
			\itemch Vi khuẩn từ thịt sống (E.coli, Salmonella) sẽ dính vào trái cây.
			\itemch Dòng nước chảy giúp trôi sạch đất cát và vi khuẩn, hóa chất.
			\itemch Thực phẩm rã đông vi khuẩn đã bắt đầu phát triển nhanh, cấp đông lại sẽ không đảm bảo chất lượng và an toàn. Nên nấu chín rồi mới cấp đông.
			\itemch Quy tắc vệ sinh cơ bản.
		\end{itemchoice}
	}
\end{ex}

%%%=============TF_3=============%%%
\begin{ex}%[6K3N4-3]
	Các dấu hiệu nhận biết thực phẩm hư hỏng:
	\choiceTF
	{\True Sữa tươi bị vón cục, có mùi chua gắt là sữa đã hỏng.}
	{Trái cây chín mềm, có mùi thơm đặc trưng là dấu hiệu hư hỏng.}
	{\True Khoai tây mọc mầm hoặc có vỏ chuyển màu xanh lá cây là loại củ có độc.}
	{Bánh chưng bị nhớt ở vỏ, có mùi thiu là dấu hiệu vi khuẩn phát triển.}
	\loigiai{
		\begin{itemchoice}[T1,F2,T3,T4]
			\itemch Protein trong sữa bị vi khuẩn làm kết tủa.
			\itemch Đó là trạng thái chín bình thường. Khi ủng, thối mới là hỏng.
			\itemch Chứa chất độc Solanine.
			\itemch Dấu hiệu điển hình của bánh chưng hỏng.
		\end{itemchoice}
	}
\end{ex}

%%%=============TF_4=============%%%
\begin{ex}%[6K3H4-4]
	Về vai trò của lương thực:
	\choiceTF
	{Ở các nước phương Tây, cơm (gạo) là lương thực chính trong bữa ăn hàng ngày.}
	{\True Lương thực cung cấp nguồn năng lượng chủ yếu cho con người hoạt động.}
	{\True Ngoài ăn trực tiếp, lương thực còn dùng làm nguyên liệu chế biến mì, bún, bánh phở, bánh đa.}
	{Ngô (bắp) chỉ dùng làm thức ăn chăn nuôi, con người không ăn được.}
	\loigiai{
		\begin{itemchoice}[F1,T2,T3,F4]
			\itemch Họ dùng lúa mì (bánh mì, mì ống) hoặc khoai tây.
			\itemch Tinh bột chiếm phần lớn khẩu phần.
			\itemch Các món sợi của VN chủ yếu làm từ gạo.
			\itemch Ngô là lương thực quan trọng của nhiều dân tộc (Châu Mỹ, vùng núi cao).
		\end{itemchoice}
	}
\end{ex}

\Closesolutionfile{ans}
\Closesolutionfile{ansbook}
\Closesolutionfile{ansex}
%\bangdapanTF{AnsTF-LTTP_KHTN6_219_MADE219}

%%==============Phần bài tập trả lời ngắn==============%%% 
\subsection{Bài tập trả lời ngắn}\textit{\large Thí sinh trả lời từ câu 1 đến câu 4}
\Opensolutionfile{ansex}[Ans/LGSA-LTTP_KHTN6_219_MADE219]
\Opensolutionfile{ansexh}[Ans/AnsSA-LTTP_KHTN6_219_MADE219]
\setcounter{ex}{0}

%%%=============SA_1=============%%%
\begin{ex}%[6K3V4-1]
	Trên nhãn một hộp sữa có ghi dòng chữ \lq\lq HSD: 20/05/2026 \rq\rq. Nếu hôm nay là ngày 25/05/2026 thì hộp sữa đó đã quá hạn sử dụng bao nhiêu ngày?
	\shortans{$5$}
	\loigiai{
		$25 - 20 = 5$ ngày.
	}
\end{ex}

%%%=============SA_2=============%%%
\begin{ex}%[6K3V4-2]
	Nhiệt độ sôi của nước tinh khiết ở điều kiện thường (áp suất 1 atm) là bao nhiêu độ C? (Nhập số nguyên).
	\shortans{$100$}
	\loigiai{
		Nước sôi ở $100^\circ C$. Đây là nhiệt độ cần thiết để nấu chín nhiều loại thực phẩm và tiêu diệt vi khuẩn.
	}
\end{ex}

%%%=============SA_3=============%%%
\begin{ex}%[6K3H4-3]
	Để điều trị cho người bị tiêu chảy mất nước do ngộ độc thực phẩm nhẹ, người ta thường cho uống dung dịch gì để bù nước và điện giải? (Viết tên viết tắt thông dụng gồm 6 chữ cái bắt đầu bằng chữ O).
	\shortans{Oresol}
	\loigiai{
		Dung dịch Oresol (Oral Rehydration Salts).
	}
\end{ex}

%%%=============SA_4=============%%%
\begin{ex}%[6K3N4-4]
	Có bao nhiêu loại Vitamin tan trong dầu phổ biến? (Gợi ý: bao gồm A, D, E, K).
	\shortans{$4$}
	\loigiai{
		Có 4 loại vitamin tan trong dầu là A, D, E, K.
	}
\end{ex}

\Closesolutionfile{ansexh}
\Closesolutionfile{ansex}
%\bangdapanSA{AnsSA-LTTP_KHTN6_219_MADE219}

%%%==============Phần bài tập tự luận==============%%% 
\subsection{Bài tập tự luận}\textit{\large Thí sinh trả lời từ bài 1 đến bài 3}
\Opensolutionfile{ansbth}[Ans/LGBT-LTTP_KHTN6_219_MADE467]
\Opensolutionfile{ansbt}[Ans/AnsBT-LTTP_KHTN6_219_MADE467]

%%%=============BT_1=============%%%
\begin{bt}%[6K3H4-1]
	Em hãy giải thích tại sao không được dùng nước ngọt có ga để thay thế nước lọc uống hàng ngày?
	\loigiai{
		Không được dùng nước ngọt có ga thay nước lọc vì:
		\begin{enumerate}
			\item **Nhiều đường:** Gây béo phì, tiểu đường và sâu răng.
			\item **Chất tạo màu, hương liệu:** Là hóa chất công nghiệp, dùng nhiều không tốt cho gan, thận.
			\item **Gây mất canxi:** Nước ngọt có ga chứa axit photphoric làm cơ thể tăng đào thải canxi, gây hại xương.
			\item **Không giải khát thực sự:** Càng uống càng khát do lượng đường cao làm cơ thể mất nước.
		\end{enumerate}
	}
\end{bt}

%%%=============BT_2=============%%%
\begin{bt}%[6K3V4-2]
	Nêu quy tắc \lq\lq Ăn chín, uống sôi \rq\rq và giải thích tại sao quy tắc này lại quan trọng trong việc phòng tránh bệnh giun sán và ngộ độc thực phẩm.
	\loigiai{
		\begin{enumerate}
			\item **Quy tắc:** Chỉ ăn thức ăn đã được nấu chín kỹ và uống nước đã được đun sôi (diệt khuẩn).
			\item **Giải thích:**
			   - Trứng giun, sán và các vi khuẩn gây bệnh (Thương hàn, Tả\dots) thường tồn tại trong thực phẩm sống (rau sống, thịt tái, cá gỏi) và nước chưa đun sôi.
			   - Nhiệt độ cao (khi nước sôi $100^\circ C$) sẽ tiêu diệt hầu hết các ấu trùng giun sán và vi khuẩn này.
			   - Nếu ăn sống hoặc tái, trứng giun sán sẽ vào ruột, nở ra, ký sinh và gây bệnh cho người.
		\end{enumerate}
	}
\end{bt}

%%%=============BT_3=============%%%
\begin{bt}%[6K3C4-3]
	Có ý kiến cho rằng: \lq\lq Tủ lạnh là nơi an toàn nhất nên cứ cho thực phẩm vào đó là để bao lâu cũng được, không bao giờ hỏng \rq\rq. Em hãy dùng kiến thức đã học để phản biện ý kiến trên là sai.
	\loigiai{
		Ý kiến trên là **Sai**. Vì:
		\begin{enumerate}
			\item Tủ lạnh chỉ làm **chậm** sự phát triển của vi khuẩn chứ không tiêu diệt được chúng hoàn toàn. Vi khuẩn vẫn sống và phát triển chậm ở nhiệt độ thấp.
			\item Mỗi loại thực phẩm có thời hạn bảo quản nhất định. Để quá lâu thực phẩm vẫn bị mất chất dinh dưỡng, biến chất và hư hỏng (ví dụ thịt để đông quá lâu bị oxy hóa, rau trong ngăn mát bị nẫu).
			\item Nếu tủ lạnh chứa quá nhiều đồ, khí lạnh không lưu thông được hoặc tủ bị mở nhiều lần, nhiệt độ không đảm bảo thì thực phẩm hỏng rất nhanh.
			\item Có những vi khuẩn ưa lạnh (như Listeria) vẫn phát triển tốt trong tủ lạnh gây ngộ độc.
		\end{enumerate}
	}
\end{bt}

\Closesolutionfile{ansbt}
\Closesolutionfile{ansbth}

\begin{center}
 \rule[4pt]{2cm}{1pt}\,\large\bfseries Hết\,\rule[4pt]{2cm}{1pt}
\end{center}
\label{x}
\end{document}
