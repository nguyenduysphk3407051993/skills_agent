%%%=============EX_1=============%%%
\begin{ex}%[9K8N9-1]
	Công thức phân tử của ethylic alcohol là
	\choice
	{$CH_3OH$}
	{\True $C_2H_6O$}
	{$C_2H_4O$}
	{$C_3H_8O$}
	\loigiai{
		Công thức phân tử của ethylic alcohol là $C_2H_6O$ (hay $C_2H_5OH$).
	}
\end{ex}

%%%=============EX_2=============%%%
\begin{ex}%[9K8N9-1]
	Nhóm chức đặc trưng cho phản ứng hóa học của ethylic alcohol (alcohol etylic) là
	\choice
	{nhóm $-COOH$}
	{nhóm $-CHO$}
	{\True nhóm $-OH$}
	{nhóm $-CO-$}
	\loigiai{
		Phân tử ethylic alcohol có nhóm hydroxyl ($-OH$) liên kết với nguyên tử carbon no. Đây là nhóm chức quyết định tính chất hóa học đặc trưng của alcohol.
	}
\end{ex}

%%%=============EX_3=============%%%
\begin{ex}%[9K8N9-1]
	Ở điều kiện thường, ethylic alcohol là chất lỏng
	\choice
	{màu xanh, nhẹ hơn nước, tan vô hạn trong nước}
	{không màu, nặng hơn nước, ít tan trong nước}
	{\True không màu, nhẹ hơn nước, tan vô hạn trong nước}
	{màu vàng, nhẹ hơn nước, không tan trong nước}
	\loigiai{
		Ở điều kiện thường, ethylic alcohol là chất lỏng không màu, nhẹ hơn nước và tan vô hạn trong nước.
	}
\end{ex}

%%%=============EX_4=============%%%
\begin{ex}%[9K8N9-1]
	Trong công nghiệp, ethylic alcohol được điều chế bằng phương pháp nào sau đây?
	\choice
	{Cho ethylene tác dụng với dung dịch $KMnO_4$}
	{\True Hydrat hóa ethylene (cộng nước) xúc tác acid}
	{Oxi hóa methane}
	{Thủy phân dẫn xuất halogen}
	\loigiai{
		Trong công nghiệp, ethylic alcohol thường được điều chế bằng phản ứng hợp nước của ethylene ($C_2H_4$) với xúc tác acid (như $H_2SO_4$ hoặc $H_3PO_4$):
		$C_2H_4 + H_2O \xrightarrow{H^+} C_2H_5OH$.
	}
\end{ex}

%%%=============EX_5=============%%%
\begin{ex}%[9K8N9-1]
	Chất nào sau đây tác dụng với $Na$ giải phóng khí $H_2$?
	\choice
	{$CH_3-O-CH_3$}
	{$CH_4$}
	{$C_6H_6$}
	{\True $C_2H_5OH$}
	\loigiai{
		Ethylic alcohol ($C_2H_5OH$) có nguyên tử $H$ linh động trong nhóm $-OH$ nên phản ứng được với $Na$ giải phóng khí hydrogen ($H_2$).
		$2C_2H_5OH + 2Na \xrightarrow{} 2C_2H_5ONa + H_2$.
	}
\end{ex}
