%%%=============BT_1=============%%%
\begin{bt}%[9K8C9-6]
	Từ $10$ kg gạo nếp (chứa $80\%$ tinh bột), người ta lên men để sản xuất ethylic alcohol. Biết hiệu suất của toàn bộ quá trình lên men đạt $50\%$.
	a) Viết các phương trình hóa học của quá trình chuyển hóa từ tinh bột thành glucose và từ glucose thành ethylic alcohol.
	b) Tính thể tích ethylic alcohol $40^\circ$ thu được. Biết khối lượng riêng của ethylic alcohol nguyên chất là $D = 0{,}8$ g/ml.
	\loigiai{
		a) Các phương trình hóa học:
		\\
		1. Thủy phân tinh bột:
		$(C_6H_{10}O_5)_n + nH_2O \xrightarrow{enzyme} nC_6H_{12}O_6$ (glucose)
		\\
		2. Lên men glucose:
		$C_6H_{12}O_6 \xrightarrow{enzyme, 30-35^\circ C} 2C_2H_5OH + 2CO_2$
		
		b) Tính toán:
		\\
		Khối lượng tinh bột trong gạo: $m_{TB} = 10 \times \dfrac{80}{100} = 8$ (kg) $= 8000$ gam.
		\\
		Số mol của mắt xích tinh bột ($C_6H_{10}O_5, M=162$):
		$n_{TB} = \dfrac{8000}{162} \approx 49{,}38$ (mol).
		\\
		Theo sơ đồ phản ứng: $1$ mắt xích tinh bột $\to$ $1$ glucose $\to$ $2$ $C_2H_5OH$.
		\\
		Số mol $C_2H_5OH$ lý thuyết: $n_{C_2H_5OH(LT)} = 2 \times n_{TB} = 2 \times 49{,}38 = 98{,}76$ (mol).
		\\
		Do hiệu suất $H = 50\%$, số mol thực tế thu được:
		$n_{C_2H_5OH(TT)} = 98{,}76 \times \dfrac{50}{100} = 49{,}38$ (mol).
		\\
		Khối lượng $C_2H_5OH$ thu được:
		$m_{C_2H_5OH} = 49{,}38 \times 46 \approx 2271{,}48$ (gam).
		\\
		Thể tích $C_2H_5OH$ nguyên chất:
		$V_{nc} = \dfrac{m}{D} = \dfrac{2271{,}48}{0{,}8} \approx 2839{,}35$ (ml).
		\\
		Thể tích rượu $40^\circ$ thu được:
		$V_{ruou} = \dfrac{V_{nc} \times 100}{40} = \dfrac{2839{,}35 \times 100}{40} \approx 7098$ (ml) $\approx 7{,}1$ lít.
	}
\end{bt}

%%%=============BT_2=============%%%
\begin{bt}%[9K8C9-6]
	Xăng E5 là một loại nhiên liệu sinh học được tạo thành bằng cách pha trộn xăng khoáng (RON 92) với ethylic alcohol (bio-ethanol) theo tỉ lệ thể tích $95:5$.
	a) Tại sao việc sử dụng xăng E5 lại được khuyến khích để bảo vệ môi trường, mặc dù khi đốt cháy ethanol vẫn sinh ra khí $CO_2$?
	b) Một chiếc xe máy tiêu thụ hết $4$ lít xăng E5. Tính lượng $CO_2$ (tính theo gam) thải ra môi trường từ lượng ethanol có trong xăng đó. Giả sử khối lượng riêng của ethanol là $0{,}8$ g/ml.
	\loigiai{
		a) Sử dụng xăng E5 được khuyến khích vì:
		- Ethanol được sản xuất từ thực vật (lên men tinh bột/đường), nguồn nguyên liệu tái tạo. Cây xanh hấp thụ $CO_2$ trong quá trình quang hợp để tạo ra sinh khối, nên lượng $CO_2$ thải ra khi đốt cháy ethanol coi như được cân bằng (chu trình carbon khép kín), giảm hiệu ứng nhà kính so với nhiên liệu hóa thạch hoàn toàn.
		- Ethanol chứa oxygen giúp quá trình cháy nhiên liệu hoàn toàn hơn, giảm khí thải độc hại ($CO$, $HC$).
		
		b) Tính lượng $CO_2$:
		\\
		Trong $4$ lít xăng E5 có $5\%$ thể tích là ethanol:
		$V_{C_2H_5OH} = 4 \times \dfrac{5}{100} = 0{,}2$ (lít) $= 200$ ml.
		\\
		Khối lượng ethanol: $m = V \times D = 200 \times 0{,}8 = 160$ (gam).
		\\
		Số mol ethanol: $n = \dfrac{160}{46} \approx 3{,}48$ (mol).
		\\
		Phương trình cháy: $C_2H_5OH + 3O_2 \xrightarrow{t^\circ} 2CO_2 + 3H_2O$.
		\\
		Số mol $CO_2$ sinh ra: $n_{CO_2} = 2 \times n_{C_2H_5OH} = 2 \times 3{,}48 = 6{,}96$ (mol).
		\\
		Khối lượng $CO_2$ thải ra: $m_{CO_2} = 6{,}96 \times 44 \approx 306{,}2$ (gam).
	}
\end{bt}
