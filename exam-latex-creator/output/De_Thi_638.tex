\documentclass[FileMain.tex]{subfiles}
\gdef\sophong{Sở GD \& ĐT Gia Lai} 
\gdef\truong{Trường THPT Chi Lăng} 
\gdef\truongh{Trường Mầm non, THCS, THPT Sao Việt} 
\gdef\monhoc{Khoa học tự nhiên 6} 
\gdef\ngaykt{04/02/2026} 
\gdef\nh{2025 - 2026} 
\gdef\thoigian{45 phút}
\gdef\made{638} 
\setcounter{section}{0}
%\tatloigiai
%\hienthiloigiai
%\dongkeloigiai
\begin{document}
\section[Truy bài định kì - Mã đề \made]{Truy bài định kì} 
%\Tieudegiua{Kiểm tra chủ đề Lương thực - Thực phẩm - Mã đề \made}

%%%==============Phần trắc nghiệm nhiều lựa chọn==============%%% 
\subsection{Bài tập trắc nghiệm nhiều lựa chọn}\textit{\large Thí sinh trả lời từ câu 1 đến 12. Mỗi câu thí sinh chỉ chọn một phương án}
\Opensolutionfile{ansex}[Ans/LGEX-LTTP_KHTN6_638_MADE638]
\Opensolutionfile{ans}[Ans/Ans-LTTP_KHTN6_638_MADE638]

%%%============EX_1=============%%%
\begin{ex}%[6K3N4-1]
	Chất xơ có nhiều trong loại thực phẩm nào sau đây?
	\choice
	{Dầu ăn}
	{Thịt nạc}
	{{\True Rau xanh và trái cây}}
	{Trứng gà}
	\loigiai{
		Rau xanh và trái cây là nguồn cung cấp chất xơ chủ yếu, giúp hệ tiêu hóa hoạt động tốt.
	}
\end{ex}

%%%============EX_2=============%%%
\begin{ex}%[6K3N4-2]
	Thực phẩm nào sau đây cung cấp nhiều chất béo (lipid) nhất?
	\choice
	{Gạo}
	{{\True Bơ thực vật, phô mai}}
	{Khoai tây}
	{Cà chua}
	\loigiai{
		Bơ, phô mai, dầu, mỡ là những thực phẩm giàu chất béo.
	}
\end{ex}

%%%============EX_3=============%%%
\begin{ex}%[6K3H4-3]
	Vì sao não bộ cần được cung cấp đủ chất bột đường (Glucose)?
	\choice
	{{\True Vì đường Glucose là nguồn năng lượng chính duy nhất cho não hoạt động}}
	{Vì não cần đường để phát triển kích thước}
	{Vì đường giúp não cứng hơn}
	{Vì đường giúp giảm đau đầu}
	\loigiai{
		Tế bào não hầu như chỉ sử dụng Glucose làm nhiên liệu để hoạt động. Thiếu đường sẽ gây chóng mặt, mất tập trung.
	}
\end{ex}

%%%============EX_4=============%%%
\begin{ex}%[6K3V4-4]
	Trong các phương pháp chế biến món ăn sau, phương pháp nào giữ được nhiều chất dinh dưỡng nhất?
	\choice
	{Chiên (rán) ngập dầu}
	{Nướng trên than}
	{{\True Hấp cách thủy}}
	{Hầm thật kỹ trong nhiều giờ}
	\loigiai{
		Hấp giúp giữ lại tối đa vitamin và khoáng chất vì thực phẩm không tiếp xúc trực tiếp với nước sôi và không bị mất chất vào nước luộc.
	}
\end{ex}

%%%============EX_5=============%%%
\begin{ex}%[6K3N4-1]
	Bệnh loãng xương ở người già thường do thiếu hụt chất nào?
	\choice
	{Sắt}
	{Iốt}
	{{\True Canxi}}
	{Vitamin C}
	\loigiai{
		Thiếu Canxi làm mật độ xương giảm, dẫn đến xương xốp, giòn và dễ gãy (loãng xương).
	}
\end{ex}

%%%============EX_6=============%%%
\begin{ex}%[6K3H4-2]
	Tại sao ta nên ăn chín uống sôi và hạn chế ăn rau sống?
	\choice
	{Vì thức ăn chín ngon hơn}
	{{\True Để tiêu diệt vi khuẩn và ký sinh trùng gây bệnh}}
	{Để thức ăn mềm dễ nhai}
	{Vì rau sống có vị đắng}
	\loigiai{
		Nấu chín giúp tiêu diệt trứng giun sán và vi khuẩn có trong thực phẩm, đảm bảo an toàn.
	}
\end{ex}

%%%============EX_7=============%%%
\begin{ex}%[6K3V4-3]
	Khi bị ngộ độc thực phẩm gây mất nước nghiêm trọng, nếu không có dung dịch Oresol thì có thể thay thế bằng nước gì?
	\choice
	{Nước ngọt có ga}
	{{\True Nước cháo loãng pha xíu muối hoặc nước dừa tươi}}
	{Bia, rượu}
	{Cà phê}
	\loigiai{
		Nước cháo muối hoặc nước dừa tươi có thành phần điện giải gần giống với dịch cơ thể, giúp bù nước hiệu quả.
	}
\end{ex}

%%%============EX_8=============%%%
\begin{ex}%[6K3N4-4]
	Loại vitamin nào được tạo ra trên da khi tiếp xúc với ánh nắng mặt trời buổi sớm?
	\choice
	{Vitamin A}
	{Vitamin B}
	{Vitamin C}
	{{\True Vitamin D}}
	\loigiai{
		Dưới tác dụng của tia cực tím trong ánh nắng sớm, tiền vitamin D dưới da chuyển hóa thành Vitamin D.
	}
\end{ex}

%%%============EX_9=============%%%
\begin{ex}%[6K3H4-1]
	Tại sao cá khô, tôm khô đễ được rất lâu mà không bị hỏng?
	\choice
	{{\True Vì đã được loại bỏ nước nên vi khuẩn không phát triển được}}
	{Vì cá khô cứng nên vi khuẩn không cắn được}
	{Vì cá khô không có chất dinh dưỡng}
	{Vì người ta bôi thuốc súng vào}
	\loigiai{
		Vi sinh vật cần nước để sống. Làm khô (giảm hoạt độ nước) ức chế vi sinh vật.
	}
\end{ex}

%%%============EX_10=============%%%
\begin{ex}%[6K3V4-2]
	Khoai tây mọc mầm rất độc, chất độc trong mầm khoai tây có tên là gì?
	\choice
	{Histamine}
	{{\True Solanine}}
	{Formaldehyde}
	{Cholesterol}
	\loigiai{
		Solanine là chất độc tự nhiên có trong mầm khoai tây và vỏ khoai tây khi chuyển xanh, gây ngộ độc thần kinh.
	}
\end{ex}

%%%============EX_11=============%%%
\begin{ex}%[6K3N4-3]
	Nguồn lương thực nào được chế biến thành bánh mì, mì ống (pasta)?
	\choice
	{Lúa gạo}
	{{\True Lúa mì}}
	{Ngô}
	{Sắn}
	\loigiai{
		Bột mì (từ lúa mì) có Gluten dẻo dai nên thích hợp làm bánh mì và mì ống.
	}
\end{ex}

%%%============EX_12=============%%%
\begin{ex}%[6K3H4-4]
	Lớp vỏ cám của gạo lứt chứa nhiều vitamin nhóm nào nhất?
	\choice
	{Vitamin A}
	{{\True Vitamin B (đặc biệt là B1)}}
	{Vitamin C}
	{Vitamin D}
	\loigiai{
		Vitamin B1 (Thiamine) tập trung chủ yếu ở lớp vỏ cám. Ăn gạo xát quá trắng dễ thiếu B1 gây bệnh tê phù (Beri-beri).
	}
\end{ex}

\Closesolutionfile{ans}
\Closesolutionfile{ansex}
%\bangdapan{Ans-LTTP_KHTN6_638_MADE638}

%%%==============Phần trắc nghiệm đúng sai==============%%% 
\subsection{Trắc nghiệm đúng sai}\textit{\large Thí sinh trả lời từ câu 1 đến câu 4. Trong mỗi ý a), b), c), d) ở mỗi câu thí sinh chọn đúng hoặc sai}
\Opensolutionfile{ansex}[Ans/LGTF-LTTP_KHTN6_638_MADE638]
\Opensolutionfile{ansbook}[Ansbook/AnsTF-LTTP_KHTN6_638_MADE638]
\Opensolutionfile{ans}[Ans/Tempt-LTTP_KHTN6_638_MADE638]
\setcounter{ex}{0}

%%%=============TF_1=============%%%
\begin{ex}%[6K3H4-1]
	Về việc sử dụng phụ gia thực phẩm:
	\choiceTF
	{\True Chỉ được sử dụng các loại phụ gia trong danh mục cho phép của Bộ Y tế.}
	{Có thể dùng hàn the (borax) để làm giò chả dai và giòn hơn.}
	{\True Phẩm màu thực phẩm tự nhiên (gấc, lá dứa, nghệ) an toàn hơn phẩm màu hóa học.}
	{Phụ gia thực phẩm càng sặc sỡ thì càng tốt cho sức khỏe.}
	\loigiai{
		\begin{itemchoice}[T1,F2,T3,F4]
			\itemch Để đảm bảo an toàn.
			\itemch Hàn the là chất độc tích lũy, bị cấm sử dụng trong thực phẩm.
			\itemch An toàn và không gây hại.
			\itemch Phẩm màu hóa học sặc sỡ thường gây hại nếu dùng quá liều hoặc dùng loại công nghiệp.
		\end{itemchoice}
	}
\end{ex}

%%%=============TF_2=============%%%
\begin{ex}%[6K3V4-2]
	Về thói quen ăn uống hàng quán (thức ăn đường phố):
	\choiceTF
	{\True Nên chọn quán ăn uy tín, có tủ kính che đậy thức ăn tránh bụi và ruồi nhặng.}
	{Ăn thức ăn chiên đi chiên lại nhiều lần bằng dầu đen không ảnh hưởng gì.}
	{\True Không nên dùng hộp xốp để đựng thức ăn đang nóng sôi.}
	{Dùng đũa dùng một lần tẩm hóa chất tẩy trắng là rất vệ sinh.}
	\loigiai{
		\begin{itemchoice}[T1,F2,T3,F4]
			\itemch Đảm bảo vệ sinh an toàn.
			\itemch Dầu chiên lại sinh ra nhiều chất độc (Trans fat, Acrylamide) gây ung thư.
			\itemch Vì hộp xốp gặp nhiệt sẽ thôi nhiễm chất độc.
			\itemch Đũa tre tẩy trắng bằng lưu huỳnh hay chất tẩy mạnh rất độc hại.
		\end{itemchoice}
	}
\end{ex}

%%%=============TF_3=============%%%
\begin{ex}%[6K3N4-3]
	Các bệnh liên quan đến dinh dưỡng:
	\choiceTF
	{\True Béo phì là do ăn quá nhiều năng lượng (đường, béo) nhưng lười vận động.}
	{Thiếu máu do thiếu Sắt chỉ xảy ra ở người già, trẻ em không bị.}
	{\True Suy dinh dưỡng thể thấp còi là do thiếu Canxi, Vitamin D và Protein kéo dài.}
	{Bướu cổ là do thừa I-ốt.}
	\loigiai{
		\begin{itemchoice}[T1,F2,T3,F4]
			\itemch Năng lượng thừa tích tụ thành mỡ.
			\itemch Trẻ em, phụ nữ mang thai rất dễ bị thiếu máu thiếu sắt.
			\itemch Ảnh hưởng đến sự phát triển khung xương.
			\itemch Bướu cổ đơn thuần là do Thiếu I-ốt. (Thừa I-ốt gây bệnh khác như cường giáp).
		\end{itemchoice}
	}
\end{ex}

%%%=============TF_4=============%%%
\begin{ex}%[6K3H4-4]
	Về tính chất của lương thực:
	\choiceTF
	{Ngô, Khoai, Sắn đều chứa hàm lượng Protein cao hơn thịt cá.}
	{\True Gạo nếp dẻo và dính hơn gạo tẻ do thành phần tinh bột khác nhau.}
	{\True Khoai tây để lâu trong không khí bị xanh vỏ là do tiếp xúc với ánh sáng tạo chất độc.}
	{Lương thực khô như gạo không bao giờ bị mọt ăn.}
	\loigiai{
		\begin{itemchoice}[F1,T2,T3,F4]
			\itemch Lương thực chủ yếu là tinh bột (Carbohydrate), rất ít Protein.
			\itemch Gạo nếp nhiều Amylopectin nên dẻo dính.
			\itemch Ánh sáng thúc đẩy quá trình tổng hợp Solanine và Chlorophyll.
			\itemch Gạo, ngô khô vẫn bị mọt ăn nếu bảo quản không kín.
		\end{itemchoice}
	}
\end{ex}

\Closesolutionfile{ans}
\Closesolutionfile{ansbook}
\Closesolutionfile{ansex}
%\bangdapanTF{AnsTF-LTTP_KHTN6_638_MADE638}

%%==============Phần bài tập trả lời ngắn==============%%% 
\subsection{Bài tập trả lời ngắn}\textit{\large Thí sinh trả lời từ câu 1 đến câu 4}
\Opensolutionfile{ansex}[Ans/LGSA-LTTP_KHTN6_638_MADE638]
\Opensolutionfile{ansexh}[Ans/AnsSA-LTTP_KHTN6_638_MADE638]
\setcounter{ex}{0}

%%%=============SA_1=============%%%
\begin{ex}%[6K3V4-1]
	Trung bình mỗi ngày một người trưởng thành cần cung cấp khoảng bao nhiêu lít nước cho cơ thể? (Ghi khoảng giá trị từ số nhỏ đến số lớn, ví dụ 1,5 - 2).
	\shortans{$1{,}5 - 2$}
	\loigiai{
		Lượng nước trung bình cần thiết là khoảng 1,5 đến 2 lít/ngày tùy hoạt động và cân nặng.
	}
\end{ex}

%%%=============SA_2=============%%%
\begin{ex}%[6K3V4-2]
	Để tiêu diệt hầu hết các vi khuẩn gây bệnh trong nước uống, chúng ta cần đun sôi nước và duy trì sôi trong ít nhất bao nhiêu phút? (Nhập số nguyên, ví dụ 1, 2, 3..).
	\shortans{$1$}
	\loigiai{
		Nước sôi $100^\circ C$ tiêu diệt hầu hết vi khuẩn ngay lập tức, nhưng nên duy trì sôi khoảng 1-3 phút để đảm bảo an toàn tuyệt đối.
	}
\end{ex}

%%%=============SA_3=============%%%
\begin{ex}%[6K3H4-3]
	Tên loại khí gas thường dùng làm nhiên liệu đun nấu trong gia đình (bếp ga) là gì? (Viết tắt 3 chữ cái).
	\shortans{LPG}
	\loigiai{
		LPG (Liquefied Petroleum Gas) - Khí dầu mỏ hóa lỏng.
	}
\end{ex}

%%%=============SA_4=============%%%
\begin{ex}%[6K3N4-4]
	Trong 4 nhóm chất dinh dưỡng, nhóm chất nào cung cấp năng lượng nhiều nhất cho cơ thể (tính trên cùng 1 gam khối lượng)?
	\shortans{Lipid}
	\loigiai{
		Lipid (Chất béo) cung cấp khoảng 9 kcal/g, gấp hơn 2 lần so với Protein và Carbohydrate (4 kcal/g).
	}
\end{ex}

\Closesolutionfile{ansexh}
\Closesolutionfile{ansex}
%\bangdapanSA{AnsSA-LTTP_KHTN6_638_MADE638}

%%%==============Phần bài tập tự luận==============%%% 
\subsection{Bài tập tự luận}\textit{\large Thí sinh trả lời từ bài 1 đến bài 3}
\Opensolutionfile{ansbth}[Ans/LGBT-LTTP_KHTN6_638_MADE638]
\Opensolutionfile{ansbt}[Ans/AnsBT-LTTP_KHTN6_638_MADE638]

%%%=============BT_1=============%%%
\begin{bt}%[6K3H4-1]
	Em hãy kể tên 3 dấu hiệu nhận biết một người đang bị ngộ độc thực phẩm và nêu 2 việc cần làm ngay để sơ cứu.
	\loigiai{
		**3 dấu hiệu:**
		1. Đau bụng dữ dội, quặn thắt.
		2. Buồn nôn và nôn mửa liên tục.
		3. Tiêu chảy (đi ngoài phân lỏng nhiều lần). Có thể kèm sốt, đau đầu.
		**2 việc cần làm ngay:**
		1. **Gây nôn:** Nếu người bệnh tỉnh táo, kích thích họng để nôn thức ăn độc ra ngoài.
		2. **Bù nước:** Cho uống dung dịch Oresol hoặc nước cháo muối, nước đun sôi để nguội liên tục để bù nước mất do nôn và tiêu chảy.
	}
\end{bt}

%%%=============BT_2=============%%%
\begin{bt}%[6K3V4-2]
	Tại sao chúng ta không nên ăn quá nhiều muối (đồ mặn)? Hậu quả của việc ăn mặn lâu dài là gì?
	\loigiai{
		**Không nên ăn quá mặn vì:**
		- Muối làm cơ thể giữ nước, tăng khối lượng máu lưu thông, gây áp lực lên tim và thành mạch máu.
		- Thận phải làm việc quá sức để lọc thải lượng muối thừa.
		**Hậu quả lâu dài:**
		- Tăng nguy cơ mắc bệnh **Cao huyết áp**.
		- Gây các bệnh về **Tim mạch** (đột quỵ, nhồi máu cơ tim).
		- Gây hại cho **Thận** (suy thận, sỏi thận).
		- Làm tăng nguy cơ ung thư dạ dày và loãng xương (do mất canxi).
	}
\end{bt}

%%%=============BT_3=============%%%
\begin{bt}%[6K3C4-3]
	Ở cổng trường thường bán các loại xiên que, xúc xích giá rẻ không rõ nguồn gốc. Theo em, học sinh có nên ăn các loại thực phẩm này không? Tại sao?
	\loigiai{
		**Không nên ăn.**
		**Lý do:**
		1. **Nguồn gốc không rõ ràng:** Có thể làm từ thịt ôi thiu, thịt bẩn, không qua kiểm dịch.
		2. **Phụ gia độc hại:** Chứa hàn the, phẩm màu công nghiệp, hương liệu hóa học để đánh lừa vị giác, che giấu mùi ôi thiu.
		3. **Chế biến mất vệ sinh:** Dầu chiên đen ngòm (chiên đi chiên lại nhiều lần) chứa chất gây ung thư. Bụi bặm, ruồi nhặng bám vào.
		4. **Nguy cơ sức khỏe:** Dễ gây ngộ độc cấp tính (đau bụng, tiêu chảy) hoặc tích tụ chất độc gây bệnh mãn tính (ung thư, suy gan thận) về sau.
	}
\end{bt}

\Closesolutionfile{ansbt}
\Closesolutionfile{ansbth}

\begin{center}
 \rule[4pt]{2cm}{1pt}\,\large\bfseries Hết\,\rule[4pt]{2cm}{1pt}
\end{center}
\label{x}
\end{document}
