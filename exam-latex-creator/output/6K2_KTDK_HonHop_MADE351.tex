\documentclass[FileMain.tex]{subfiles}
\gdef\sophong{Sở GD \& ĐT Gia Lai}
\gdef\truong{Trường THCS \& THPT}
\gdef\monhoc{Khoa học tự nhiên 6}
\gdef\ngaykt{04/02/2026}
\gdef\nh{2025 - 2026}
\gdef\thoigian{45}
\gdef\made{351}
\setcounter{section}{0}
\begin{document}
\section[Kiểm tra định kì - KHTN 6 - Mã đề \made]{Kiểm tra định kì}

%%%==============Phần trắc nghiệm nhiều lựa chọn==============%%%
\subsection{Bài tập trắc nghiệm nhiều lựa chọn}\textit{\large Thí sinh trả lời từ câu 1 đến câu 12. Mỗi câu thí sinh chỉ chọn một phương án}
\Opensolutionfile{ansex}[Ans/LGEX-6K2_HonHop_MADE351]
\Opensolutionfile{ans}[Ans/Ans-6K2_HonHop_MADE351]

%%%%%============EX_1================%%%%%%
\begin{ex}%[6K2NB1-1]
	Hỗn hợp là gì?
	\choice
	{Chất được tạo nên từ một nguyên tố hóa học}
	{\True Hai hay nhiều chất trộn lẫn với nhau}
	{Chất có thành phần và tính chất xác định}
	{Chất được tạo nên từ một loại phân tử}
	\loigiai{
		Hỗn hợp là hai hay nhiều chất trộn lẫn với nhau. Trong hỗn hợp, mỗi chất vẫn giữ nguyên tính chất của nó.
	}
\end{ex}

%%%%%============EX_2================%%%%%%
\begin{ex}%[6K2NB1-1]
	Chất tinh khiết là chất
	\choice
	{được tạo nên từ nhiều loại phân tử}
	{có nhiều tạp chất lẫn vào}
	{\True không lẫn chất khác, có tính chất nhất định}
	{được trộn lẫn từ nhiều chất}
	\loigiai{
		Chất tinh khiết là chất không lẫn chất khác, có thành phần và tính chất xác định.
	}
\end{ex}

%%%%%============EX_3================%%%%%%
\begin{ex}%[6K2NB1-2]
	Trong các trường hợp sau, trường hợp nào là hỗn hợp?
	\choice
	{Nước cất}
	{Khí oxygen nguyên chất}
	{\True Không khí}
	{Vàng nguyên chất $24$K}
	\loigiai{
		Không khí là hỗn hợp gồm nhiều chất khí như nitrogen (khoảng $78\%$), oxygen (khoảng $21\%$), carbon dioxide, hơi nước và các khí khác.
	}
\end{ex}

%%%%%============EX_4================%%%%%%
\begin{ex}%[6K2TH1-2]
	Hỗn hợp đồng nhất là hỗn hợp
	\choice
	{nhìn thấy rõ ranh giới giữa các thành phần}
	{\True không nhìn thấy ranh giới giữa các thành phần}
	{các chất không tan vào nhau}
	{các hạt chất lơ lửng trong hỗn hợp}
	\loigiai{
		Hỗn hợp đồng nhất là hỗn hợp không nhìn thấy ranh giới giữa các thành phần, các chất phân bố đều trong toàn bộ hỗn hợp.
	}
\end{ex}

%%%%%============EX_5================%%%%%%
\begin{ex}%[6K2TH1-2]
	Hỗn hợp nào sau đây là hỗn hợp không đồng nhất?
	\choice
	{Nước muối}
	{Nước đường}
	{Giấm ăn}
	{\True Nước cam có tép}
	\loigiai{
		Nước cam có tép là hỗn hợp không đồng nhất vì có thể nhìn thấy rõ các tép cam trong phần nước cam.
	}
\end{ex}

%%%%%============EX_6================%%%%%%
\begin{ex}%[6K2NB2-1]
	Huyền phù là hỗn hợp gồm
	\choice
	{chất lỏng tan trong chất lỏng}
	{chất khí tan trong chất lỏng}
	{\True chất rắn lơ lửng trong chất lỏng}
	{chất lỏng tan trong chất rắn}
	\loigiai{
		Huyền phù là hỗn hợp không đồng nhất gồm các hạt chất rắn lơ lửng trong chất lỏng.
	}
\end{ex}

%%%%%============EX_7================%%%%%%
\begin{ex}%[6K2NB2-1]
	Nhũ tương là hỗn hợp gồm
	\choice
	{chất rắn phân tán trong chất lỏng}
	{\True chất lỏng phân tán trong chất lỏng khác}
	{chất khí phân tán trong chất lỏng}
	{chất rắn phân tán trong chất rắn}
	\loigiai{
		Nhũ tương là hỗn hợp không đồng nhất gồm các giọt chất lỏng phân tán trong chất lỏng khác (hai chất lỏng không tan vào nhau).
	}
\end{ex}

%%%%%============EX_8================%%%%%%
\begin{ex}%[6K2TH2-2]
	Sữa tươi thuộc loại hỗn hợp nào?
	\choice
	{Dung dịch}
	{Huyền phù}
	{\True Nhũ tương}
	{Chất tinh khiết}
	\loigiai{
		Sữa tươi là nhũ tương vì gồm các giọt chất béo (chất lỏng) phân tán trong nước.
	}
\end{ex}

%%%%%============EX_9================%%%%%%
\begin{ex}%[6K2NB3-1]
	Phương pháp tách chất nào sau đây dựa trên sự khác nhau về kích thước hạt?
	\choice
	{Cô cạn}
	{Chiết}
	{\True Lọc}
	{Chưng cất}
	\loigiai{
		Phương pháp lọc dựa trên sự khác nhau về kích thước hạt. Chất có kích thước hạt lớn hơn lỗ lọc sẽ bị giữ lại.
	}
\end{ex}

%%%%%============EX_10================%%%%%%
\begin{ex}%[6K2TH3-2]
	Để tách muối ăn ra khỏi nước biển, người ta sử dụng phương pháp
	\choice
	{lọc}
	{chiết}
	{\True cô cạn}
	{chưng cất}
	\loigiai{
		Để tách muối ăn ra khỏi nước biển, người ta sử dụng phương pháp cô cạn. Khi đun nóng (hoặc phơi nắng), nước bay hơi và muối ăn kết tinh lại.
	}
\end{ex}

%%%%%============EX_11================%%%%%%
\begin{ex}%[6K2TH3-2]
	Phương pháp chưng cất dựa trên sự khác nhau về tính chất nào của các chất?
	\choice
	{Kích thước hạt}
	{Khối lượng riêng}
	{\True Nhiệt độ sôi}
	{Độ tan trong nước}
	\loigiai{
		Phương pháp chưng cất dựa trên sự khác nhau về nhiệt độ sôi của các chất.
	}
\end{ex}

%%%%%============EX_12================%%%%%%
\begin{ex}%[6K2VD3-3]
	Để tách riêng hỗn hợp cát và muối ăn, ta cần thực hiện các bước theo thứ tự
	\choice
	{Lọc $\rightarrow$ Hòa tan $\rightarrow$ Cô cạn}
	{Cô cạn $\rightarrow$ Hòa tan $\rightarrow$ Lọc}
	{\True Hòa tan $\rightarrow$ Lọc $\rightarrow$ Cô cạn}
	{Lọc $\rightarrow$ Cô cạn $\rightarrow$ Hòa tan}
	\loigiai{
		Để tách riêng hỗn hợp cát và muối ăn: Hòa tan hỗn hợp vào nước (muối tan, cát không tan) $\rightarrow$ Lọc (tách cát) $\rightarrow$ Cô cạn (thu muối).
	}
\end{ex}

\Closesolutionfile{ans}
\Closesolutionfile{ansex}

%%%==============Phần trắc nghiệm đúng sai==============%%%
\subsection{Trắc nghiệm đúng sai}\textit{\large Thí sinh trả lời từ câu 1 đến câu 4. Trong mỗi ý a), b), c), d) ở mỗi câu thí sinh chọn đúng hoặc sai}
\Opensolutionfile{ansex}[Ans/LGTF-6K2_HonHop_MADE351]
\Opensolutionfile{ansbook}[Ansbook/AnsTF-6K2_HonHop_MADE351]
\Opensolutionfile{ans}[Ans/Tempt-6K2_HonHop_MADE351]
\setcounter{ex}{0}

%%%%%============TF_1================%%%%%%
\begin{ex}%[6K2TH1-2]
	Cho các phát biểu sau về hỗn hợp và chất tinh khiết:
	\choiceTF
	{\True Nước khoáng là hỗn hợp vì chứa nhiều chất khoáng hòa tan}
	{Nước cất là hỗn hợp vì được làm từ nước máy}
	{\True Không khí là hỗn hợp đồng nhất của nhiều chất khí}
	{Đường kính trắng là hỗn hợp vì có màu trắng}
	\loigiai{
		\begin{itemchoice}[T1,F2,T3,F4]
			\itemch Nước khoáng chứa các muối khoáng hòa tan nên là hỗn hợp
			\itemch Nước cất là chất tinh khiết, được điều chế bằng phương pháp chưng cất để loại bỏ tạp chất
			\itemch Không khí gồm nhiều chất khí (nitrogen, oxygen, carbon dioxide\dots) phân bố đều nên là hỗn hợp đồng nhất
			\itemch Đường kính trắng là chất tinh khiết (saccharose), màu trắng là tính chất vật lý không liên quan đến việc phân loại hỗn hợp hay chất tinh khiết
		\end{itemchoice}
	}
\end{ex}

%%%%%============TF_2================%%%%%%
\begin{ex}%[6K2TH2-2]
	Cho các phát biểu sau về dung dịch, huyền phù và nhũ tương:
	\choiceTF
	{\True Nước muối là dung dịch vì muối tan hoàn toàn trong nước}
	{\True Nước bùn là huyền phù vì các hạt bùn lơ lửng trong nước}
	{Sữa tươi là dung dịch vì có màu trắng đồng đều}
	{\True Dầu giấm là nhũ tương vì dầu và giấm không tan vào nhau}
	\loigiai{
		\begin{itemchoice}[T1,T2,F3,T4]
			\itemch Muối ăn tan hoàn toàn trong nước tạo thành dung dịch đồng nhất
			\itemch Các hạt bùn (chất rắn) lơ lửng trong nước là đặc điểm của huyền phù
			\itemch Sữa tươi là nhũ tương vì chứa các giọt chất béo phân tán trong nước, không phải dung dịch
			\itemch Dầu ăn và giấm là hai chất lỏng không tan vào nhau, khi trộn tạo thành nhũ tương
		\end{itemchoice}
	}
\end{ex}

%%%%%============TF_3================%%%%%%
\begin{ex}%[6K2TH3-3]
	Cho các phát biểu sau về phương pháp tách chất:
	\choiceTF
	{Phương pháp lọc dùng để tách muối ăn ra khỏi nước biển}
	{\True Phương pháp chiết dùng để tách dầu ăn ra khỏi nước}
	{\True Phương pháp chưng cất dùng để thu nước cất từ nước máy}
	{Phương pháp cô cạn dùng để tách cát ra khỏi nước}
	\loigiai{
		\begin{itemchoice}[F1,T2,T3,F4]
			\itemch Phương pháp cô cạn (không phải lọc) dùng để tách muối ăn ra khỏi nước biển vì muối tan trong nước
			\itemch Dầu ăn không tan và nhẹ hơn nước nên nổi lên trên, có thể dùng phương pháp chiết để tách
			\itemch Chưng cất dựa vào sự khác nhau về nhiệt độ sôi, nước bay hơi rồi ngưng tụ thành nước cất tinh khiết
			\itemch Cát không tan trong nước nên dùng phương pháp lọc (không phải cô cạn) để tách
		\end{itemchoice}
	}
\end{ex}

%%%%%============TF_4================%%%%%%
\begin{ex}%[6K2VD3-3]
	Cho các phát biểu sau về ứng dụng của phương pháp tách chất trong đời sống:
	\choiceTF
	{\True Lọc nước bằng bình lọc gia đình là ứng dụng của phương pháp lọc}
	{\True Làm muối từ nước biển bằng cách phơi nắng là ứng dụng của phương pháp cô cạn}
	{Nấu rượu từ cơm rượu là ứng dụng của phương pháp lọc}
	{\True Vớt váng mỡ khi nấu nước dùng là ứng dụng của phương pháp chiết}
	\loigiai{
		\begin{itemchoice}[T1,T2,F3,T4]
			\itemch Bình lọc nước giữ lại các hạt cặn bẩn có kích thước lớn, đây là ứng dụng của phương pháp lọc
			\itemch Phơi nắng làm nước bay hơi, muối kết tinh lại, đây là ứng dụng của phương pháp cô cạn
			\itemch Nấu rượu là ứng dụng của phương pháp chưng cất (không phải lọc) vì rượu có nhiệt độ sôi thấp hơn nước
			\itemch Mỡ (chất lỏng) nổi lên trên nước dùng, vớt váng mỡ là ứng dụng của phương pháp chiết
		\end{itemchoice}
	}
\end{ex}

\Closesolutionfile{ans}
\Closesolutionfile{ansbook}
\Closesolutionfile{ansex}

%%%==============Phần bài tập trả lời ngắn==============%%%
\subsection{Bài tập trả lời ngắn}\textit{\large Thí sinh trả lời từ câu 1 đến câu 4}
\Opensolutionfile{ansex}[Ans/LGSA-6K2_HonHop_MADE351]
\Opensolutionfile{ansexh}[Ans/AnsSA-6K2_HonHop_MADE351]
\setcounter{ex}{0}

%%%%%============SA_1================%%%%%%
\begin{ex}%[6K2TH2-2]
	Một học sinh pha $20$ g đường vào $180$ g nước và khuấy đều cho đường tan hết. Tính khối lượng (theo đơn vị gam) của dung dịch nước đường thu được.
	\shortans{$200$}
	\loigiai{
		Khối lượng dung dịch nước đường thu được:
		\[ m_{\text{dung dịch}} = m_{\text{đường}} + m_{\text{nước}} = 20 + 180 = 200 \text{ g} \]
	}
\end{ex}

%%%%%============SA_2================%%%%%%
\begin{ex}%[6K2VD2-3]
	Hòa tan $15$ g muối ăn vào nước được $150$ g dung dịch nước muối. Tính nồng độ phần trăm (theo đơn vị $\%$) của dung dịch nước muối.
	\shortans{$10$}
	\loigiai{
		Nồng độ phần trăm của dung dịch:
		\[ C\% = \dfrac{m_{\text{chất tan}}}{m_{\text{dung dịch}}} \times 100\% = \dfrac{15}{150} \times 100\% = 10\% \]
	}
\end{ex}

%%%%%============SA_3================%%%%%%
\begin{ex}%[6K2VD3-3]
	Cô cạn $250$ g dung dịch nước muối có nồng độ $8\%$. Tính khối lượng muối (theo đơn vị gam) thu được sau khi cô cạn hoàn toàn.
	\shortans{$20$}
	\loigiai{
		Khối lượng muối thu được:
		\[ m_{\text{muối}} = \dfrac{C\% \times m_{\text{dung dịch}}}{100\%} = \dfrac{8 \times 250}{100} = 20 \text{ g} \]
	}
\end{ex}

%%%%%============SA_4================%%%%%%
\begin{ex}%[6K2VD3-4]
	Trong không khí, nitrogen chiếm khoảng $78\%$ về thể tích. Tính thể tích khí nitrogen (theo đơn vị lít) có trong $500$ lít không khí.
	\shortans{$390$}
	\loigiai{
		Thể tích khí nitrogen trong $500$ lít không khí:
		\[ V_{N_2} = \dfrac{78\% \times 500}{100\%} = \dfrac{78 \times 500}{100} = 390 \text{ lít} \]
	}
\end{ex}

\Closesolutionfile{ansexh}
\Closesolutionfile{ansex}

%%%==============Phần bài tập tự luận==============%%%
\subsection{Bài tập tự luận}\textit{\large Thí sinh trả lời từ bài 1 đến bài 3}
\Opensolutionfile{ansbth}[Ans/LGBT-6K2_HonHop_MADE351]
\Opensolutionfile{ansbt}[Ans/AnsBT-6K2_HonHop_MADE351]

%%%%%============BT_1================%%%%%%
\begin{bt}%[6K2TH1-2]
	Phân biệt hỗn hợp đồng nhất và hỗn hợp không đồng nhất. Cho $2$ ví dụ minh họa mỗi loại.
	\loigiai{
		\textbf{Hỗn hợp đồng nhất:}
		\begin{itemize}
		\item Là hỗn hợp không nhìn thấy ranh giới giữa các thành phần
		\item Các chất phân bố đều trong toàn bộ hỗn hợp
		\item Ví dụ: nước muối, nước đường, không khí, giấm ăn
		\end{itemize}

		\textbf{Hỗn hợp không đồng nhất:}
		\begin{itemize}
		\item Là hỗn hợp nhìn thấy rõ ranh giới giữa các thành phần
		\item Các chất không phân bố đều trong hỗn hợp
		\item Ví dụ: nước cam có tép, nước bùn (huyền phù), sữa tươi (nhũ tương), cát trộn sỏi
		\end{itemize}
	}
\end{bt}

%%%%%============BT_2================%%%%%%
\begin{bt}%[6K2VD2-3]
	Hãy phân loại các hỗn hợp sau thành dung dịch, huyền phù hoặc nhũ tương. Giải thích ngắn gọn.
	\begin{enumerate}
	\item Nước mắm
	\item Nước bột sắn khuấy đều
	\item Sữa tươi
	\item Nước chanh pha đường
	\end{enumerate}
	\loigiai{
		\begin{enumerate}
		\item \textbf{Nước mắm:} Là \textbf{dung dịch} vì muối và các chất hữu cơ tan hoàn toàn trong nước, tạo thành hỗn hợp đồng nhất.

		\item \textbf{Nước bột sắn khuấy đều:} Là \textbf{huyền phù} vì các hạt bột sắn (chất rắn) lơ lửng trong nước. Để lâu, bột sắn sẽ lắng xuống đáy.

		\item \textbf{Sữa tươi:} Là \textbf{nhũ tương} vì gồm các giọt chất béo (chất lỏng) phân tán trong nước. Hai chất lỏng này không tan vào nhau.

		\item \textbf{Nước chanh pha đường:} Là \textbf{dung dịch} vì đường và axit citric trong chanh tan hoàn toàn trong nước.
		\end{enumerate}
	}
\end{bt}

%%%%%============BT_3================%%%%%%
\begin{bt}%[6K2VD3-4]
	Một học sinh có $300$ ml nước ao hồ bị vẩn đục (có lẫn bùn đất) và có lẫn một ít dầu ăn nổi trên mặt. Hãy trình bày các bước để thu được nước sạch nhất có thể từ mẫu nước này. Giải thích ngắn gọn tại sao lại chọn các phương pháp đó.
	\loigiai{
		\textbf{Các bước thực hiện:}

		\textbf{Bước 1: Dùng phương pháp chiết để tách dầu ăn}
		\begin{itemize}
		\item Đổ hỗn hợp vào phễu chiết, để yên cho dầu nổi lên trên
		\item Mở khóa phễu chiết, cho nước chảy ra trước, đóng khóa khi dầu sắp chảy xuống
		\item \textit{Giải thích:} Dầu ăn không tan và nhẹ hơn nước nên nổi lên trên
		\end{itemize}

		\textbf{Bước 2: Dùng phương pháp lọc để tách bùn đất}
		\begin{itemize}
		\item Đặt giấy lọc vào phễu lọc, đổ nước qua phễu lọc
		\item Bùn đất bị giữ lại trên giấy lọc, nước trong chảy qua
		\item \textit{Giải thích:} Bùn đất là chất rắn không tan, có kích thước hạt lớn hơn lỗ giấy lọc
		\end{itemize}

		\textbf{Bước 3: Dùng phương pháp chưng cất để thu nước tinh khiết}
		\begin{itemize}
		\item Đun sôi nước trong bình chưng cất
		\item Hơi nước bay lên, được làm lạnh và ngưng tụ thành nước cất
		\item \textit{Giải thích:} Nước có nhiệt độ sôi $100^\circ$C, khi ngưng tụ thu được nước tinh khiết, các chất tan còn lại trong bình
		\end{itemize}
	}
\end{bt}

\Closesolutionfile{ansbt}
\Closesolutionfile{ansbth}

\begin{center}
 \rule[4pt]{2cm}{1pt}\,\large\bfseries Hết\,\rule[4pt]{2cm}{1pt}
\end{center}
\label{x}
\end{document}
