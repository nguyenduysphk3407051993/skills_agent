%Cấu hình tên các tham số
%
Tên tham số 1: Lớp
Tên tham số 2: Môn
Tên tham số 3: Chương
Tên tham số 4: Mức độ
Tên tham số 5: Bài
Tên tham số 6: Dạng
%
%Cấu hình chi tiết ID
%
%%Cấu hình mức độ dùng chung.
[Y] Yếu
[B] Trung bình
[K] Khá
[G] Giỏi
[T] Thực tế
[N] Nhận biết
[H] Thông hiểu
[V] Vận dụng
[C] Vận dụng cao

-[7] Lớp 7

----[K] Khoa học tự nhiên

-------[1] Nguyên tử - Sơ lược về bảng tuần hoàn các nguyên tố hóa học

----------[1] Nguyên tử

-------------[1] Nêu mô hình nguyên tử (vỏ, hạt nhân)

-------------[2] Xác định các loại hạt (p, n, e) trong nguyên tử

-------------[3] So sánh khối lượng của hạt nhân với khối lượng của electron

-------------[4] Vận dụng kiến thức về nguyên tử giải thích tại sao nguyên tử trung hoà về điện

----------[2] Nguyên tố hoá học

-------------[1] Phân biệt nguyên tử và nguyên tố hoá học

-------------[2] Viết kí hiệu hoá học của nguyên tố

-------------[3] Tên gọi, kí hiệu của 20 nguyên tố đầu tiên

-------------[4] Bài tập về số hiệu nguyên tử và số khối

----------[3] Sơ lược về bảng tuần hoàn các nguyên tố hoá học

-------------[1] Mô tả cấu tạo của bảng tuần hoàn (ô, chu kì, nhóm)

-------------[2] Xác định vị trí (ô, chu kì, nhóm) của một nguyên tố trong bảng tuần hoàn (với 20 nguyên tố đầu)

-------------[3] Dựa vào vị trí suy ra một số thông tin cơ bản của nguyên tố

-------------[4] Phân biệt kim loại, phi kim, khí hiếm dựa vào vị trí và tính chất

-------------[5] Bìa toán tìm 2 nguyên tố trong cùng chu kì liên tiếp, hai chu kì liên tiếp nhau trong cùng nhóm A

-------------[6] Hoàn thành bảng thông tin về tên nguyên tố, kí hiệu hóa học, số e, p,n, khối lượng nguyên tử


-------[2] Phân tử - Liên kết hóa học

----------[1] Phân tử - Đơn chất - Hợp chất

-------------[1] Lý thuyết - Định nghĩa, khái niệm

-------------[2] Phân loại - Nhận dạng đơn chất/hợp chất

-------------[3] Công thức - Viết và đọc CTHH

-------------[4] Định lượng cơ bản - Tính khối lượng phân tử

-------------[5] Phân tích - Thành phần, tỉ lệ nguyên tố

-------------[6] Tổng hợp - Vận dụng, liên hệ thực tế

----------[2] Giới thiệu về liên kết hoá học

-------------[1] Lý thuyết - Định nghĩa khái niệm liên kết

-------------[2] Cấu hình electron bền vững - Quy tắc octet

-------------[3] Nhận dạng - Phân loại loại liên kết

-------------[4] Sơ đồ hình thành - Mô tả quá trình (định tính) sự hình thành phân tử (sự kết hợp của các nguyên tử)

-------------[5] Sơ đồ hình thành - Mô tả quá trình (định tính) hình thành liên kết ion (cho và nhận electron),cộng hoá trị (dùng chung electron) 

-------------[6] Tính toán electron - Số electron nhường/nhận/dùng chung (định lượng)

-------------[7] Tính chất - So sánh chất ion và cộng hóa trị

-------------[8] Tổng hợp - Vận dụng, liên hệ thực tế

----------[3] Hoá trị và công thức hoá học

-------------[1] Lý thuyết - Định nghĩa hóa trị, quy ước

-------------[2] Đọc công thức - Phân tích thông tin từ CTHH

-------------[3] Tìm hóa trị - Từ công thức xác định hóa trị (định tính)

-------------[4] Lập công thức - Từ hóa trị lập CTHH (định lượng cơ bản)

-------------[5] Tổng hợp - Kết hợp tìm hóa trị và lập công thức

-------------[6] Nâng cao - Tính % khối lượng nguyên tố (định lượng)

-------------[7] Xác định công thức hoá học dựa vào phần trăm khối lượng nguyên tố

-------------[8] Bài tập vận dụng quy tắc hoá trị trong thực tế