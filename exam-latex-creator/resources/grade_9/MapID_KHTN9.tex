
%Cấu hình tên các tham số
%
Tên tham số 1: Lớp
Tên tham số 2: Môn
Tên tham số 3: Chương
Tên tham số 4: Mức độ
Tên tham số 5: Bài
Tên tham số 6: Dạng
%
%Cấu hình chi tiết ID
%
%%Cấu hình mức độ dùng chung.
[Y] Yếu
[B] Trung bình
[K] Khá
[G] Giỏi
[T] Thực tế
[N] Nhận biết
[H] Thông hiểu
[V] Vận dụng
[C] Vận dụng cao

-[9] Lớp 9

----[K] Khoa học tự nhiên

-------[6] KIM LOẠI. SỰ KHÁC NHAU CƠ BẢN GIỮA PHI KIM VÀ KIM LOẠI

----------[1] Tính chất chung của kim loại

-------------[1] Lý thuyết về tính chất vật lí của kim loại (dẻo, dẫn điện, dẫn nhiệt, ánh kim)

-------------[2] Lý thuyết về tính chất hóa học chung của kim loại (tác dụng với phi kim, acid, dung dịch muối)

-------------[3] Viết phương trình hóa học minh họa tính chất của kim loại

-------------[4] Bài tập nhận biết kim loại và hợp chất dựa vào tính chất đặc trưng

-------------[5] Bài tập định lượng về kim loại tác dụng với acid (HCl, H₂SO₄ loãng)

-------------[6] Bài tập định lượng về kim loại tác dụng với phi kim (O₂, Cl₂)

-------------[7] Bài tập liên quan đến ứng dụng thực tiễn của kim loại trong đời sống

----------[2] Dãy hoạt động hoá học

-------------[1] Lý thuyết về cấu tạo và ý nghĩa của dãy hoạt động hóa học

-------------[2] So sánh mức độ hoạt động của các kim loại

-------------[3] Xét điều kiện xảy ra phản ứng của kim loại với nước, acid, dung dịch muối

-------------[4] Viết phương trình hóa học của kim loại tác dụng với dung dịch muối

-------------[5] Bài tập nhận biết, tách chất, tinh chế dung dịch muối

-------------[6] Bài tập định lượng kim loại tác dụng với dung dịch muối (tăng giảm khối lượng)

-------------[7] Bài toán về kim loại kiềm, kiềm thổ (Na, K, Ca, Ba) tác dụng với dung dịch muối

-------------[8] Bài toán hỗn hợp kim loại tác dụng với dung dịch acid/muối

----------[3] Tách kim loại và việc sử dụng hợp kim

-------------[1] Lý thuyết về các phương pháp điều chế kim loại (thủy luyện, nhiệt luyện, điện phân)

-------------[2] Lý thuyết về hợp kim, gang, thép và ứng dụng

-------------[3] Viết phương trình hóa học điều chế kim loại từ hợp chất

-------------[4] Bài tập tách kim loại ra khỏi hỗn hợp

-------------[5] Bài toán định lượng liên quan đến quá trình nhiệt luyện (dùng CO, H₂ khử oxide kim loại)

-------------[6] Bài toán về thành phần phần trăm của hợp kim

----------[4] Sự khác nhau cơ bản giữa phi kim và kim loại

-------------[1] So sánh tính chất vật lí và hóa học cơ bản của kim loại và phi kim

-------------[2] Phân loại oxide (oxide acid, oxide base, oxide lưỡng tính, oxide trung tính)

-------------[3] Viết phương trình hóa học minh họa sự khác biệt giữa kim loại và phi kim

-------------[4] Bài tập nhận biết kim loại, phi kim và các hợp chất của chúng

-------[7] GIỚI THIỆU VỀ CHẤT HỮU CƠ. HYDROCARBON VÀ NGUỒN NHIÊN LIỆU

----------[5] Giới thiệu về hợp chất hữu cơ

-------------[1] Lý thuyết về hợp chất hữu cơ, hóa học hữu cơ, đặc điểm cấu tạo

-------------[2] Phân loại hợp chất hữu cơ (hydrocarbon, dẫn xuất hydrocarbon)

-------------[3] Viết công thức cấu tạo, công thức cấu tạo thu gọn của các hợp chất hữu cơ đơn giản

-------------[4] Tính phần trăm khối lượng các nguyên tố trong họp chất hữu cơ

-------------[5] Lý thuyết về đồng đẳng, đồng phân

-------------[6] Lập công thức phân tử khi biết phần trăm khối lượng các nguyên tố và khối lượng phân tử.

-------------[7] Lập công thức đơn giản của hợp chất hữu cơ

-------------[87] Lập công thức phan tử thông qua công thức đơn giản nhất.

----------[6] Alkane

-------------[1] Lý thuyết về dãy đồng đẳng, công thức chung, danh pháp, tính chất vật lí của alkane

-------------[2] Lý thuyết về tính chất hóa học đặc trưng (phản ứng thế, phản ứng cháy)

-------------[3] Viết phương trình  phản ứng cháy của alkane

-------------[4] Viết các công thức cấu tạo và gọi tên một số alkane từ C1 đến C4

-------------[5] Bài toán tính phần trăm khối lượng mối nguyên tố trong hợp chất hữu cơ

-------------[6] Bài toán xác định công thức phân tử hợp chất hữu cơ khi biết phần trắm khối lượng và khối lượng phân tử

-------------[7] Lập công thức đơn giản của hợp chất hữu cơ

-------------[8] Lập công thức phân tử khi biết công thức đơn giản và các thông tin khác như khối lượng phân tử, số lượng nguyên tố khác,...

----------[7] Alkene

-------------[1] Lý thuyết về dãy đồng đẳng, công thức chung, danh pháp, tính chất vật lí của alkene

-------------[2] Lý thuyết về tính chất hóa học (phản ứng cộng, phản ứng trùng hợp, phản ứng oxi hóa)

-------------[3] Viết phương trình phản ứng cộng (H₂, Br₂, HX), trùng hợp, oxi hóa của alkene

-------------[4] Bài tập nhận biết alkene và alkane

-------------[5] Bài toán đốt cháy alkene

-------------[6] Bài toán alkene tác dụng với dung dịch Bromine

-------------[7] Bài toán tìm công thức phân tử alkene

----------[8] Nguồn nhiên liệu

-------------[1] Lý thuyết về dầu mỏ, khí thiên nhiên, than đá và ứng dụng

-------------[2] Lý thuyết về chưng cất dầu mỏ, cracking và nhiên liệu sinh học

-------------[3] Bài toán tính nhiệt lượng tỏa ra khi đốt cháy nhiên liệu

-------------[4] Bài tập về các vấn đề môi trường liên quan đến việc khai thác và sử dụng nhiên liệu hóa thạch

-------[8] ETHYLIC ALCOHOL VÀ ACETIC ACID

----------[9] Ethylic alcohol

-------------[1] Lý thuyết về cấu tạo, tính chất vật lí của ethylic alcohol

-------------[2] Lý thuyết về tính chất hóa học (phản ứng với Na, phản ứng cháy, phản ứng lên men)

-------------[3] Lý thuyết về độ cồn và cách pha chế

-------------[4] Bài toán định lượng liên quan đến phản ứng của ethylic alcohol

-------------[5] Bài toán về độ cồn và ứng dụng thực tiễn

-------------[6] Bài toán điều chế ethylic alcohol từ ethylene hoặc tinh bột

----------[0] Acetic acid

-------------[1] Lý thuyết về cấu tạo, tính chất vật lí của acetic acid

-------------[2] Lý thuyết về tính chất hóa học (tính acid, phản ứng este hóa)

-------------[3] Viết các phương trình hóa học minh họa tính chất

-------------[4] So sánh tính acid với các acid vô cơ và hữu cơ khác

-------------[5] Bài toán định lượng về tính acid (tác dụng với kim loại, base, oxide base, muối)

-------------[6] Bài toán về phản ứng este hóa (tính hiệu suất)

-------------[7] Bài toán về nồng độ dung dịch giấm ăn trong thực tế

-------[9] LIPID. CARBOHYDRATE. PROTEIN, POLYMER

----------[28] Lipid

-------------[1] Lý thuyết về khái niệm, trạng thái tự nhiên và tính chất vật lí của lipid (chất béo)

-------------[2] Lý thuyết về tính chất hóa học (phản ứng thủy phân, phản ứng xà phòng hóa)

-------------[3] Viết phương trình thủy phân và xà phòng hóa chất béo

-------------[4] Bài tập về vai trò của chất béo và các vấn đề sức khỏe liên quan

-------------[5] Bài toán tính khối lượng xà phòng, glycerol từ chất béo (chỉ số acid, chỉ số xà phòng hóa)

----------[A] Carbohydrate (Glucose và Saccharose)

-------------[1] Lý thuyết về trạng thái tự nhiên, tính chất vật lí của glucose và saccharose

-------------[2] Lý thuyết về tính chất hóa học của glucose (phản ứng tráng gương, lên men) và saccharose (phản ứng thủy phân)

-------------[3] Viết các phương trình hóa học minh họa

-------------[4] Bài tập nhận biết glucose, saccharose và các carbohydrate khác

-------------[5] Bài toán về phản ứng tráng gương của glucose

-------------[6] Bài toán về phản ứng lên men rượu từ glucose

-------------[7] Bài toán về phản ứng thủy phân saccharose

----------[B] Tinh bột và Cellulose

-------------[1] Lý thuyết về cấu tạo, tính chất vật lí của tinh bột và cellulose

-------------[2] Lý thuyết về tính chất hóa học (phản ứng thủy phân, phản ứng màu với iodine của tinh bột)

-------------[3] Viết các phương trình hóa học minh họa

-------------[4] Bài tập về ứng dụng của tinh bột và cellulose trong thực phẩm, công nghiệp

-------------[5] Bài toán thủy phân tinh bột/cellulose (tính hiệu suất)

-------------[6] Bài toán liên quan đến chuỗi phản ứng: Tinh bột → glucose → alcohol → acetic acid

----------[C] Protein

-------------[1] Lý thuyết về khái niệm, cấu tạo và vai trò của protein

-------------[2] Lý thuyết về tính chất của protein (đông tụ, thủy phân, phản ứng màu biuret)

-------------[3] Bài tập nhận biết protein và phân biệt với các chất hữu cơ khác

-------------[4] Bài tập liên quan đến vai trò của protein trong dinh dưỡng và đời sống

----------[D] Polymer

-------------[1] Lý thuyết về khái niệm, phân loại và tính chất chung của polymer

-------------[2] Lý thuyết về tính chất và ứng dụng của một số polymer thông dụng (PE, PVC, cao su, tơ)

-------------[3] Viết phương trình phản ứng trùng hợp điều chế polymer

-------------[4] Bài tập nhận biết và phân loại các loại vật liệu polymer (chất dẻo, tơ, cao su)

-------------[5] Bài toán tính khối lượng monomer/polymer (có hiệu suất phản ứng)

-------------[6] Bài tập về vấn đề ô nhiễm môi trường do rác thải nhựa và các giải pháp