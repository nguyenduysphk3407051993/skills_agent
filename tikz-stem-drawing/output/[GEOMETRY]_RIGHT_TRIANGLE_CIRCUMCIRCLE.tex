\documentclass[tikz,border=5mm]{standalone}
\usepackage[utf8]{vietnam}
\usepackage{amsmath}
\usetikzlibrary{calc,angles,quotes,backgrounds}

\begin{document}
\begin{tikzpicture}[scale=0.5]
    % === ĐỊNH NGHĨA ĐIỂM ===
    \coordinate (A) at (0,0);
    \coordinate (B) at (9,0);
    \coordinate (C) at (0,12);
    
    % === TÍNH TOÁN ===
    % Trung điểm BC là tâm đường tròn ngoại tiếp M
    \coordinate (M) at ($(B)!0.5!(C)$);
    
    % === VẼ HÌNH ===
    % Vẽ tam giác
    \draw[thick] (A) -- (B) -- (C) -- cycle;
    
    % Vẽ đường tròn ngoại tiếp
    \draw (M) circle (7.5);
    
    % === ĐÁNH DẤU ===
    % Góc vuông
    \draw (A) rectangle ++(0.5,0.5);
    
    % Điểm
    \foreach \p/\pos in {A/below left, B/below right, C/above left, M/above right} {
        \fill (\p) circle (3pt);
        \node[\pos] at (\p) {$\p$};
    }
    
    % Kích thước
    % Vẽ đường kích thước AB (dời xuống 1cm)
    \draw[dashed, gray] (A) -- ++(0,-1);
    \draw[dashed, gray] (B) -- ++(0,-1);
    \draw[|<->|] ($(A)+(0,-1)$) -- ($(B)+(0,-1)$) node[midway, fill=white] {9 cm};
    
    % Vẽ đường kích thước AC (dời sang trái 1cm)
    \draw[dashed, gray] (A) -- ++(-1,0);
    \draw[dashed, gray] (C) -- ++(-1,0);
    \draw[|<->|] ($(A)+(-1,0)$) -- ($(C)+(-1,0)$) node[midway, fill=white] {12 cm};
    
    % Bán kính
    \draw[red, dashed] (M) -- (B);
    \node[red, above] at ($(M)!0.5!(B)$) {$R$};
    
    % === LỜI GIẢI ===
    \node[right=2.5cm, align=justify, text width=10cm] at (B) {
        \textbf{Lời giải:}\\
        Tam giác $ABC$ vuông tại $A$ có $AB = 9$ cm, $AC = 12$ cm.\\
        Áp dụng định lý Pytago trong $\Delta ABC$:
        \[ BC = \sqrt{AB^2 + AC^2} = \sqrt{9^2 + 12^2} = \sqrt{81 + 144} = \sqrt{225} = 15 \text{ cm.} \]
        Tâm đường tròn ngoại tiếp tam giác vuông là trung điểm của cạnh huyền $BC$.\\
        Do đó, bán kính đường tròn ngoại tiếp tam giác là:
        \[ R = \frac{BC}{2} = \frac{15}{2} = 7,5 \text{ cm.} \]
    };
\end{tikzpicture}
\end{document}
