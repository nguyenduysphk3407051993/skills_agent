
\documentclass[FileMain.tex]{subfiles}
\gdef\sophong{Sở GD \& ĐT Gia Lai} 
\gdef\truong{Trường THPT Chi Lăng} 
\gdef\truongh{Trường Mầm non, THCS, THPT Sao Việt} 
\gdef\monhoc{Hóa học 7} 
\gdef\ngaykt{26/01/2026} 
\gdef\nh{2025 - 2026} 
\gdef\thoigian{45}
\gdef\made{533} 
\setcounter{section}{0}
\tatloigiai
%\hienthiloigiai
%\dongkeloigiai
\begin{document}
%\section[Truy bài định kì hóa học 7 - Mã đề \made]{Truy bài định kì}
\Tieudegiua{Truy bài định kì hóa học 7 - Mã đề \made}

%%%==============Phần trắc nghiệm nhiều lựa chọn==============%%% 
\subsection{Bài tập trắc nghiệm nhiều lựa chọn}\textit{\large Thí sinh trả lời từ câu 1 đến câu 18. Mỗi câu thí sinh chỉ chọn một phương án}
\Opensolutionfile{ansex}[Ans/LGEX-TBDK_HOA_HOC_7_MADE533]
\Opensolutionfile{ans}[Ans/Ans-TBDK_HOA_HOC_7_MADE533]
%%%========EX_1=========%%%
\begin{ex}
	Công thức hóa học của nước là:
	\choice
		{\True $H_2O$}
		{$HO$}
		{$H_2O_2$}
		{$HO_2$}
	\loigiai{
		Nước được tạo bởi 2 H và 1 O. Công thức là $H_2O$.
	}
\end{ex}

%%%========EX_2=========%%%
\begin{ex}
	Chất nào sau đây là đơn chất?
	\choice
		{Muối ăn ($NaCl$)}
		{Khí Carbon dioxide ($CO_2$)}
		{\True Khí Hydrogen ($H_2$)}
		{Nước ($H_2O$)}
	\loigiai{
		Hydrogen ($H_2$) chỉ gồm 1 nguyên tố H nên là đơn chất. Các chất còn lại đều gồm 2 nguyên tố trở lên.
	}
\end{ex}

%%%========EX_3=========%%%
\begin{ex}
	Đơn chất kim loại Iron (Sắt) có hạt hợp thành là nguyên tử. Công thức hóa học của nó là:
	\choice
		{$Fe_3$}
		{$2Fe$}
		{$Fe_2$}
		{\True $Fe$}
	\loigiai{
		Với đơn chất kim loại, hạt đại diện là nguyên tử nên CTHH trùng với kí hiệu nguyên tố: $Fe$.
	}
\end{ex}

%%%========EX_4=========%%%
\begin{ex}
	Trong các chất sau, chất nào là đơn chất kim loại?
	\choice
		{Phosphorus ($P$)}
		{\True Magnesium ($Mg$)}
		{Khí Chlorine ($Cl_2$)}
		{Sulfur ($S$)}
	\loigiai{
		Magnesium (Mg) là kim loại. Các chất còn lại là phi kim.
	}
\end{ex}

%%%========EX_5=========%%%
\begin{ex}
	Đơn chất là những chất được tạo nên từ:
	\choice
		{Hai nguyên tố hóa học trở lên}
		{Nhiều loại nguyên tử khác nhau}
		{\True Một nguyên tố hóa học}
		{Các hợp chất trộn lẫn với nhau}
	\loigiai{
		Đơn chất là những chất được tạo nên từ một nguyên tố hóa học.
	}
\end{ex}

%%%========EX_6=========%%%
\begin{ex}
	Một hợp chất có phân tử gồm 1 nguyên tử Magnesium (Mg) và 2 nguyên tử Chlorine (Cl). Công thức hóa học là:
	\choice
		{\True $MgCl_2$}
		{$MgCl$}
		{$Cl_2Mg$}
		{$Mg_2Cl$}
	\loigiai{
		Viết kim loại trước, phi kim sau kèm chỉ số. $MgCl_2$.
	}
\end{ex}

%%%========EX_7=========%%%
\begin{ex}
	Khí Methane có phân tử gồm 1 nguyên tử Carbon và 4 nguyên tử Hydrogen. CTHH là:
	\choice
		{\True $CH_4$}
		{$CH$}
		{$C_1H_4$}
		{$C_4H$}
	\loigiai{
		Chỉ số 1 không ghi. $CH_4$.
	}
\end{ex}

%%%========EX_8=========%%%
\begin{ex}
	Chất nào sau đây là hợp chất?
	\choice
		{Khí Nitrogen ($N_2$)}
		{\True Acetic acid ($C_2H_4O_2$)}
		{Khí Oxygen ($O_2$)}
		{Kim cương ($C$)}
	\loigiai{
		Acetic acid cấu tạo từ B, H, O (3 nguyên tố) nên là hợp chất.
	}
\end{ex}

%%%========EX_9=========%%%
\begin{ex}
	Khối lượng phân tử của khí Ammonia ($NH_3$) bằng: (Biết $N=14, H=1$).
	\choice
		{\True 17 amu}
		{14 amu}
		{15 amu}
		{16 amu}
	\loigiai{
		$M_{NH_3} = 14 + 1 \times 3 = 17$ (amu).
	}
\end{ex}

%%%========EX_10=========%%%
\begin{ex}
	Chất X có công thức $MgO$. Phân tử khối của X là: (Biết $Mg=24, O=16$).
	\choice
		{\True 40 amu}
		{50 amu}
		{30 amu}
		{20 amu}
	\loigiai{
		$M_{MgO} = 24 + 16 = 40$ (amu).
	}
\end{ex}

%%%========EX_11=========%%%
\begin{ex}
	Phân tử Sulfuric acid ($H_2SO_4$) có khối lượng là: (Biết $H=1, S=32, O=16$).
	\choice
		{49 amu}
		{97 amu}
		{96 amu}
		{\True 98 amu}
	\loigiai{
		$M_{H_2SO_4} = 1 \times 2 + 32 + 16 \times 4 = 2 + 32 + 64 = 98$ (amu).
	}
\end{ex}

%%%========EX_12=========%%%
\begin{ex}
	Tính phân tử khối của nước ($H_2O$). (Biết $H=1, O=16$).
	\choice
		{\True 18 amu}
		{17 amu}
		{16 amu}
		{34 amu}
	\loigiai{
		Khối lượng phân tử của $H_2O$ = $1 \times 2 + 16 = 18$ (amu).
	}
\end{ex}
%%%=============EX_13=============%%%
\begin{ex}
	Phân tử là
	\choice
	{hạt đại diện cho chất, được tạo bởi một nguyên tố hoá học}
	{hạt đại diện cho hợp chất, được tạo bởi nhiều nguyên tố hoá học}
	{\True hạt đại diện cho chất do một hoặc nhiều nguyên tử kết hợp với nhau và mang đầy đủ tính chất của chất}
	{hạt nhỏ nhất do các nguyên tố hoá học kết hợp với nhau tạo thành chất}
	\loigiai{
		Phân tử là hạt đại diện cho chất, gồm một số nguyên tử liên kết với nhau và thể hiện đầy đủ tính chất hóa học của chất.
	}
\end{ex}

%%%=============EX_14=============%%%
\begin{ex}
	Khối lượng phân tử là
	\choice
	{\True tổng khối lượng các nguyên tử có trong phân tử}
	{tổng khối lượng các hạt hợp thành của chất có trong phân tử}
	{tổng khối lượng các nguyên tử có trong hạt hợp thành chất}
	{khối lượng của nhiều nguyên tử}
	\loigiai{
		Khối lượng phân tử bằng tổng khối lượng của các nguyên tử hợp thành phân tử. Đơn vị thường dùng là amu (atomic mass unit).
	}
\end{ex}

%%%=============EX_15=============%%%
\begin{ex}
	Khối lượng phân tử của hợp chất iron (III) hydroxide tạo bởi 1 nguyên tử Fe, 3 nguyên tử O và 3 nguyên tử H là
	\choice
	{$48$ amu}
	{$72$ amu}
	{$80$ amu}
	{\True $107$ amu}
	\loigiai{
		Công thức hóa học của hợp chất là $Fe(OH)_3$.
		\\
		Khối lượng nguyên tử: $Fe = 56$, $O = 16$, $H = 1$.
		\\
		Khối lượng phân tử = $56 + (16 + 1) \times 3 = 56 + 17 \times 3 = 56 + 51 = 107$ amu.
	}
\end{ex}

%%%=============EX_16=============%%%
\begin{ex}
	Khối lượng phân tử của phân tử giấm ăn tạo bởi 2 nguyên tử C, 4 nguyên tử H và 2 nguyên tử O là
	\choice
	{\True $60$ amu}
	{$61$ amu}
	{$59$ amu}
	{$70$ amu}
	\loigiai{
		Công thức phân tử: $C_2H_4O_2$.
		\\
		Khối lượng nguyên tử: $C = 12$, $H = 1$, $O = 16$.
		\\
		Khối lượng phân tử = $12 \times 2 + 1 \times 4 + 16 \times 2 = 24 + 4 + 32 = 60$ amu.
	}
\end{ex}

%%%=============EX_17=============%%%
\begin{ex}
	Biết phosphoric acid gồm 3H, 1P và 4O. Khối lượng phân tử của phosphoric acid là
	\choice
	{$48$ amu}
	{$86$ amu}
	{\True $98$ amu}
	{$96$ amu}
	\loigiai{
		Công thức hóa học: $H_3PO_4$.
		\\
		Khối lượng nguyên tử: $H = 1$, $P = 31$, $O = 16$.
		\\
		Khối lượng phân tử = $1 \times 3 + 31 + 16 \times 4 = 3 + 31 + 64 = 98$ amu.
	}
\end{ex}

%%%=============EX_18=============%%%
\begin{ex}
	Phân tử glycerol chứa ba nguyên tử carbon, tám nguyên tử hydrogen và ba nguyên tử oxygen. Khối lượng phân tử của glycerol là
	\choice
	{$14$ amu}
	{$29$ amu}
	{\True $92$ amu}
	{$42$ amu}
	\loigiai{
		Công thức hóa học: $C_3H_8O_3$.
		\\
		Khối lượng nguyên tử: $C = 12$, $H = 1$, $O = 16$.
		\\
		Khối lượng phân tử = $12 \times 3 + 1 \times 8 + 16 \times 3 = 36 + 8 + 48 = 92$ amu.
	}
\end{ex}
\Closesolutionfile{ans}
\Closesolutionfile{ansex}
%\bangdapan{Ans-TBDK_HOA_HOC_7_MADE533}



%%%==============Phần trắc nghiệm đúng sai==============%%% 
\subsection{Trắc nghiệm đúng sai}\textit{\large Thí sinh trả lời từ câu 1 đến câu 5. Trong mỗi ý a), b), c), d) ở mỗi câu thí sinh chọn đúng hoặc sai}
\Opensolutionfile{ansex}[Ans/LGTF-TBDK_HOA_HOC_7_MADE533]
\Opensolutionfile{ansbook}[Ansbook/AnsTF-TBDK_HOA_HOC_7_MADE533]
\Opensolutionfile{ans}[Ans/Tempt-TBDK_HOA_HOC_7_MADE533]
\setcounter{ex}{0}
%%%========TF_1=========%%%
\begin{ex}
	So sánh giữa đơn chất và hợp chất giúp hiểu rõ về sự đa dạng của chất. Xét các khẳng định:
	\choiceTF
		{\True Hợp chất có thể bị phân hủy thành các đơn chất thông qua các phản ứng hóa học}
		{Đơn chất luôn có khối lượng phân tử nhỏ hơn bất kỳ hợp chất nào}
		{\True Một nguyên tố hóa học chỉ có thể tạo ra duy nhất một loại đơn chất}
		{\True Cả đơn chất và hợp chất đều có thể tồn tại ở các trạng thái rắn, lỏng, khí}
	\loigiai{
		\begin{itemchoice}[T1,F2,T3,T4]
			\itemch Ví dụ điện phân nước tạo ra khí $H_2$ và $O_2$.
			\itemch Có những đơn chất rất nặng (như Vàng Au = 197 amu) nặng hơn nhiều hợp chất nhẹ (như $H_2O$ = 18 amu).
			\itemch Một nguyên tố có thể tạo nhiều đơn chất (thù hình), ví dụ O tạo $O_2$ và $O_3$.
			\itemch Đây là tính chất chung của mọi chất tùy thuộc nhiệt độ và áp suất.
		\end{itemchoice}
	}
\end{ex}

%%%========TF_2=========%%%
\begin{ex}
	Về đặc điểm cấu tạo của đơn chất và hợp chất:
	\choiceTF
		{\True Đơn chất kim loại thường có hạt đại diện là nguyên tử}
		{Đơn chất phi kim luôn tồn tại ở dạng phân tử gồm 2 nguyên tử}
		{\True Hợp chất luôn có hạt đại diện là phân tử}
		{Nước ($H_2O$) là đơn chất vì rất thông dụng}
	\loigiai{
		\begin{itemchoice}[T1,F2,T3,F4]
			\itemch Ví dụ Cu, Fe, Al.
			\itemch Ví dụ Ozone ($O_3$) hay Carbon ($C$).
			\itemch Với hợp chất cộng hóa trị, ion cũng có thể coi đơn vị công thức. Ở lớp 7 coi là phân tử.
			\itemch Nước là hợp chất.
		\end{itemchoice}
	}
\end{ex}

%%%========TF_3=========%%%
\begin{ex}
	Công thức hóa học là cách biểu diễn thành phần của phân tử. Xét các ký hiệu:
	\choiceTF
		{Ký hiệu $2H$ biểu diễn một phân tử hydrogen bền vững trong tự nhiên}
		{\True Công thức $CaCO_3$ cho biết đây là một hợp chất của Calcium, Carbon và Oxygen}
		{\True Ký hiệu $P_4$ biểu diễn phân tử đơn chất trắng của nguyên tố phosphorus}
		{\True Ký hiệu $H_2$ chỉ rõ một phân tử đơn chất hydrogen gồm 2 nguyên tử}
	\loigiai{
		\begin{itemchoice}[F1,T2,T3,T4]
			\itemch $2H$ nghĩa là 2 nguyên tử hydrogen rời rạc, không phải phân tử $H_2$.
			\itemch $CaCO_3$ là đá vôi, một hợp chất vô cơ.
			\itemch Phosphorus trắng tồn tại dạng phân tử 4 nguyên tử.
			\itemch Số chỉ ở chân (chỉ số) xác định số nguyên tử trong một phân tử.
		\end{itemchoice}
	}
\end{ex}

%%%========TF_4=========%%%
\begin{ex}
	Cho các chất có công thức hóa học: $I_2, KOH, H_2SO_4, Al$.
	\choiceTF
		{$Al$ là hợp chất kim loại}
		{\True Có 2 đơn chất trong dãy trên}
		{$I_2$ là hợp chất vì có chỉ số 2}
		{\True $KOH$ và $H_2SO_4$ là hợp chất}
	\loigiai{
		\begin{itemchoice}[F1,T2,F3,T4]
			\itemch Al là đơn chất kim loại.
			\itemch $I_2$ và $Al$ là đơn chất $\Rightarrow$ Có 2 đơn chất.
			\itemch $I_2$ chỉ có 1 nguyên tố I.
			\itemch Vì tạo từ nhiều nguyên tố.
		\end{itemchoice}
	}
\end{ex}
%%%=============TF_5=============%%%
\begin{ex}
	Các phát biểu sau là đúng hay sai?
	\choiceTF
	{Đơn chất là chất mà phân tử gồm các nguyên tử có khối lượng bằng nhau}
	{\True Trong đơn chất, các nguyên tử đều thuộc cùng nguyên tố}
	{Phân tử hợp chất tạo nên từ 2 nguyên tử giống nhau hoặc khác nhau}
	{Một bình khí chứa hỗn hợp chất gồm $20\%$ Ne, $30\%$ $N_2$ và $50\%$ $O_2$ sẽ có khối lượng trung bình của phân tử khí trong bình là $32$ amu}
	\loigiai{
		\begin{itemchoice}[F1,T2,F3,F4]
			\itemch Đơn chất là những chất tạo nên từ một nguyên tố hóa học. Định nghĩa dựa trên khối lượng là không chính xác (ví dụ các nguyên tố khác nhau vẫn có thể có nguyên tử khối xấp xỉ bằng nhau).
			\itemch Đơn chất là những chất được tạo nên từ một nguyên tố hóa học.
			\itemch Phân tử hợp chất được tạo nên từ 2 hay nhiều nguyên tố hóa học khác nhau. Nếu phân tử tạo từ các nguyên tử giống nhau thì đó là đơn chất.
			\itemch Khối lượng phân tử trung bình của hỗn hợp khí là:
			\[ \bar{M} = 20 \cdot 20\% + 28 \cdot 30\% + 32 \cdot 50\% = 4 + 8{,}4 + 16 = 28{,}4 \text{ (amu)} \]
			Kết quả $28{,}4 \text{ amu} \neq 32 \text{ amu}$.
		\end{itemchoice}
	}
\end{ex}
\Closesolutionfile{ans}
\Closesolutionfile{ansbook}
\Closesolutionfile{ansex}
%\bangdapanTF{AnsTF-TBDK_HOA_HOC_7_MADE533}



%%==============Phần bài tập trả lời ngắn==============%%% 
\subsection{Bài tập trả lời ngắn}\textit{\large Thí sinh trả lời từ câu 1 đến câu 5}
\Opensolutionfile{ansex}[Ans/LGSA-TBDK_HOA_HOC_7_MADE533]
\Opensolutionfile{ansexh}[Ans/AnsSA-TBDK_HOA_HOC_7_MADE533]
\setcounter{ex}{0}
%%%========SA_1=========%%%
\begin{ex}
	Phân tử oxygen thường ($O_2$) gồm bao nhiêu nguyên tử oxygen?
	\shortans{2}
	\loigiai{
		Công thức $O_2$ cho biết phân tử gồm 2 nguyên tử oxygen liên kết với nhau. So với ozone ($O_3$ gồm 3 nguyên tử O), sự khác nhau về số lượng nguyên tử trong phân tử (thù hình) dẫn đến tính chất hóa học khác nhau.
	}
\end{ex}

%%%========SA_2=========%%%
\begin{ex}
	Tính khối lượng phân tử của khí Carbon dioxide ($CO_2$). Biết $C=12, O=16$.
	\shortans{44}
	\loigiai{
		Khối lượng phân tử của $CO_2$ là:
		\[ 12 + 16 \times 2 = 44 \text{ (amu)} \]
	}
\end{ex}

%%%========SA_3=========%%%
\begin{ex}
	Hợp chất $Fe_2O_3$ có tổng bao nhiêu nguyên tử trong một phân tử?
	\shortans{5}
	\loigiai{
		Trong $Fe_2O_3$ có 2 nguyên tử Fe và 3 nguyên tử O. Tổng: $2 + 3 = 5$.
	}
\end{ex}

%%%========SA_4=========%%%
\begin{ex}%[7K2H1-5]
	Phân tử Calcium carbonate ($CaCO_3$) được tạo thành từ bao nhiêu nguyên tố hóa học?
	\shortans{3}
	\loigiai{
		$CaCO_3$ được tạo từ 3 nguyên tố: Calcium (Ca), Carbon (C), và Oxygen (O).
	}
\end{ex}
%%%==============SA_5==============%%%
\begin{ex}
	Cho các chất sau:
	\begin{enumerate}
		\item[(a)] Nước được tạo nên từ H và O.
		\item[(b)] Sodium chloride được tạo nên từ Na và Cl.
		\item[(c)] Bột sulfur được tạo nên từ S.
		\item[(d)] Kim loại copper được tạo nên từ Cu.
		\item[(e)] Đường mía được tạo nên từ C, H và O.
	\end{enumerate}
	Số hợp chất trong các chất trên là bao nhiêu?
	\shortans{$3$}
	\loigiai{
		Hợp chất là những chất tạo nên từ hai nguyên tố hóa học trở lên.
		\begin{itemize}
			\item (a) Nước được tạo từ $2$ nguyên tố là H và O $\Rightarrow$ Là hợp chất.
			\item (b) Sodium chloride được tạo từ $2$ nguyên tố là Na và Cl $\Rightarrow$ Là hợp chất.
			\item (c) Bột sulfur chỉ được tạo từ $1$ nguyên tố là S $\Rightarrow$ Là đơn chất.
			\item (d) Kim loại copper chỉ được tạo từ $1$ nguyên tố là Cu $\Rightarrow$ Là đơn chất.
			\item (e) Đường mía được tạo từ $3$ nguyên tố là C, H và O $\Rightarrow$ Là hợp chất.
		\end{itemize}
		Vậy có $3$ hợp chất là: nước, sodium chloride và đường mía.
	}
\end{ex}
\Closesolutionfile{ansexh}
\Closesolutionfile{ansex}
%\bangdapanSA{AnsSA-TBDK_HOA_HOC_7_MADE533}



%%%==============Phần bài tập tự luận==============%%% 
\subsection{Bài tập tự luận}\textit{\large Thí sinh trả lời từ bài 1 đến bài 4}
\Opensolutionfile{ansbth}[Ans/LGBT-TBDK_HOA_HOC_7_MADE533]
\Opensolutionfile{ansbt}[Ans/AnsBT-TBDK_HOA_HOC_7_MADE533]
%%%========BT_1=========%%%
\begin{bt}
	Phân tử của hợp chất A gồm 1 nguyên tử X liên kết với 4 nguyên tử Hydrogen, có khối lượng phân tử là 16 amu.
	\begin{enumerate}
		\item Tính khối lượng nguyên tử của X.
		\item Xác định tên và kí hiệu hóa học của nguyên tố X.
		\item Viết công thức hóa học của hợp chất A.
	\end{enumerate}
	(Cho biết $H=1$)
	\loigiai{
		\begin{enumerate}
			\item Gọi khối lượng nguyên tử của X là $M_X$.
			Ta có phương trình khối lượng phân tử của A:
			\[ M_X + 4 \times 1 = 16 \]
			\[ M_X = 16 - 4 = 12 \text{ (amu)} \]
			\item Nguyên tố có khối lượng nguyên tử 12 amu là Carbon, kí hiệu: C.
			\item Công thức hóa học của A là: $CH_4$ (Methane).
		\end{enumerate}
	}
\end{bt}

%%%%%============BT_2================%%%%%%
\begin{bt}%[7K2C3-5]
	Hợp chất Oxide cao nhất của nguyễn tố R có dạng $RO_3$. Biết trong hợp chất này, Oxygen chiếm 60\% về khối lượng.
	\begin{enumerate}
		\item Xác định nguyên tử khối và tên của nguyên tố R.
		\item Viết công thức hóa học của oxide đó.
	\end{enumerate}
	\loigiai{
		\begin{enumerate}
			\item Công thức là $RO_3$.
				$\%O = \frac{16 \times 3}{R + 16 \times 3} \times 100\% = 60\%$
				$\Rightarrow \frac{48}{R + 48} = 0,6 \Rightarrow 48 = 0,6R + 28,8$
				$\Rightarrow 0,6R = 19,2 \Rightarrow R = 32$.
				Vậy R là Sulfur (Lưu huỳnh).
			\item Công thức hóa học: $SO_3$.
		\end{enumerate}
	}
\end{bt}

%%%========BT_3=========%%%
\begin{bt}
	Sulfuric acid ($H_2SO_4$) được gọi là \lq\lq máu của ngành hóa chất \rq\rq.
	\begin{enumerate}
		\item Tính tổng số nguyên tử có trong một phân tử Sulfuric acid.
		\item Tính tỉ lệ khối lượng của Sulfur so với Oxygen trong phân tử này.
	\end{enumerate}
	(Cho $S=32, O=16$)
	\loigiai{
		\begin{enumerate}
			\item Tổng số nguyên tử = $2 + 1 + 4 = 7$ nguyên tử.
			\item Tỉ lệ khối lượng:
			\[ \dfrac{m_S}{m_O} = \dfrac{1 \times 32}{4 \times 16} = \dfrac{32}{64} = \dfrac{1}{2} \]
		\end{enumerate}
	}
\end{bt}
%%%=============BT_4=============%%%
\begin{bt}
	Hãy đánh dấu X vào ô thích hợp để hoàn thiện bảng sau về sự phân loại một số chất
	\begin{center}
		\renewcommand{\arraystretch}{0.9} 
		\begin{tabular}{|l|c|c|c|}
			\hline
			\multicolumn{1}{|c|}{\multirow{2}{*}{\textbf{Chất}}} & \multicolumn{2}{c|}{\textbf{Chất nguyên chất}} & \multirow{2}{*}{\textbf{Hỗn hợp}} \\ \cline{2-3}
			\multicolumn{1}{|c|}{} & \textbf{Đơn chất} & \textbf{Hợp chất} & \\ \hline
			Sắt &  & & \\ \hline
			Đường ăn + Nước cất & & &  \\ \hline
			Nước cam & & &  \\ \hline
			Nước biển & & &  \\ \hline
			Không khí trong quả bóng bay & & &  \\ \hline
			Nhôm &  & & \\ \hline
			Nước cất & &  & \\ \hline
		\end{tabular}
	\end{center}
	\loigiai{
		\begin{center}
			\renewcommand{\arraystretch}{0.9} 
			\begin{tabular}{|l|c|c|c|}
				\hline
				\multicolumn{1}{|c|}{\multirow{2}{*}{\textbf{Chất}}} & \multicolumn{2}{c|}{\textbf{Chất nguyên chất}} & \multirow{2}{*}{\textbf{Hỗn hợp}} \\ \cline{2-3}
				\multicolumn{1}{|c|}{} & \textbf{Đơn chất} & \textbf{Hợp chất} & \\ \hline
				Sắt & X & & \\ \hline
				Đường ăn + Nước cất & & & X \\ \hline
				Nước cam & & & X \\ \hline
				Nước biển & & & X \\ \hline
				Không khí trong quả bóng bay & & & X \\ \hline
				Nhôm & X & & \\ \hline
				Nước cất & & X & \\ \hline
			\end{tabular}
		\end{center}
	}
\end{bt}
\Closesolutionfile{ansbt}
\Closesolutionfile{ansbth}

\begin{center}
 \rule[4pt]{2cm}{1pt}\,\large\bfseries Hết\,\rule[4pt]{2cm}{1pt}
\end{center}
\label{x}
\end{document}
 