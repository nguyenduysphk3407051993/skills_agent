-[6] Lớp 6

----[K] Khoa học tự nhiên

-------[1] Các thể của chất

----------[1] Sự đa dạng và các thể cơ bản của chất
-------------[1] Nhận biết vật thể tự nhiên, vật thể nhân tạo, vật vô sinh, vật hữu sinh
-------------[2] Đặc điểm các thể cơ bản của chất (Rắn, Lỏng, Khí)
-------------[3] Phân biệt các thể của chất dựa trên tính chất vật lí

----------[2] Sự chuyển thể của chất
-------------[1] Sự nóng chảy và sự đông đặc
-------------[2] Sự bay hơi và sự ngưng tụ
-------------[3] Sự sôi
-------------[4] Vẽ và phân tích sơ đồ vòng tuần hoàn của nước trong tự nhiên

-------[2] Oxygen và không khí

----------[1] Oxygen
-------------[1] Tính chất vật lí và tầm quan trọng của Oxygen
-------------[2] Sự cháy và các yếu tố cần thiết cho sự cháy
-------------[3] Biện pháp dập tắt đám cháy

----------[2] Không khí và bảo vệ môi trường không khí
-------------[1] Thành phần của không khí
-------------[2] Thí nghiệm xác định thành phần phần trăm thể tích Oxygen trong không khí
-------------[3] Vai trò của không khí đối với tự nhiên và sự sống
-------------[4] Nguyên nhân và hậu quả của ô nhiễm không khí
-------------[5] Biện pháp bảo vệ môi trường không khí

-------[3] Một số vật liệu, nhiên liệu, nguyên liệu, lương thực - thực phẩm

----------[1] Một số vật liệu thông dụng
-------------[1] Tính chất và ứng dụng của vật liệu (Kim loại, Nhựa, Gỗ, Thủy tinh, Gốm sứ, Cao su...)
-------------[2] Dấu hiệu nhận biết vật liệu bị hư hỏng
-------------[3] Quy tắc an toàn và hiệu quả khi sử dụng vật liệu (3R)

----------[2] Một số nhiên liệu thông dụng
-------------[1] Khái niệm và phân loại nhiên liệu (Rắn, Lỏng, Khí)
-------------[2] Tính chất và ứng dụng của một số nhiên liệu phổ biến (Than, Xăng, Dầu, Gas...)
-------------[3] An ninh năng lượng và sử dụng nhiên liệu an toàn, tiết kiệm

----------[3] Một số nguyên liệu thông dụng
-------------[1] Khái niệm và phân loại nguyên liệu (Quặng, Đá vôi...)
-------------[2] Tính chất và ứng dụng của nguyên liệu trong sản xuất
-------------[3] Khai thác nhiên liệu và phát triển bền vững

----------[4] Một số lương thực - thực phẩm thông dụng
-------------[1] Vai trò của lương thực, thực phẩm và các nhóm chất dinh dưỡng
-------------[2] Tính chất của lương thực, thực phẩm (trạng thái, màu sắc, mùi vị...)
-------------[3] Dấu hiệu ngộ độc thực phẩm và cách phòng tránh
-------------[4] Bảo quản lương thực, thực phẩm đúng cách

-------[4] Chất tinh khiết - Hỗn hợp - Phương pháp tách chất

----------[1] Chất tinh khiết - Hỗn hợp
-------------[1] Khái niệm và cách nhận biết chất tinh khiết, hỗn hợp
-------------[2] Phân biệt hỗn hợp đồng nhất và hỗn hợp không đồng nhất

----------[2] Dung dịch - Huyền phù - Nhũ tương
-------------[1] Khái niệm và phân biệt dung dịch, huyền phù, nhũ tương
-------------[2] Thành phần của dung dịch (Dung môi, Chất tan)
-------------[3] Khả năng hòa tan của chất rắn trong nước (Ảnh hưởng của nhiệt độ, khuấy trộn...)
-------------[4] Tính toán nồng độ phần trăm, tính lượng chất tan/dung môi (Mức độ nâng cao)

----------[3] Tách chất ra khỏi hỗn hợp
-------------[1] Phương pháp Lắng, Gạn, Lọc (Nguyên tắc và ứng dụng)
-------------[2] Phương pháp Cô cạn (Nguyên tắc và ứng dụng)
-------------[3] Phương pháp Chiết (Nguyên tắc và ứng dụng)
-------------[4] Lựa chọn phương pháp tách biệt phù hợp cho hỗn hợp cụ thể
-------------[5] Thực hành tách chất (Tách muối ăn, tách dầu ăn khỏi nước...)
